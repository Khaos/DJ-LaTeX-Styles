%%%%%%%%%%%%%%%%%%%%%%%%%%%%%%%%%%%%%%%%%%%%%%%%%%%%%%%%%%%%%%%%%%%%%%%%%%%%%
%
% Package pgfplots.sty documentation. 
%
% Copyright 2007/2008 by Christian Feuersaenger.
%
% This program is free software: you can redistribute it and/or modify
% it under the terms of the GNU General Public License as published by
% the Free Software Foundation, either version 3 of the License, or
% (at your option) any later version.
% 
% This program is distributed in the hope that it will be useful,
% but WITHOUT ANY WARRANTY; without even the implied warranty of
% MERCHANTABILITY or FITNESS FOR A PARTICULAR PURPOSE.  See the
% GNU General Public License for more details.
% 
% You should have received a copy of the GNU General Public License
% along with this program.  If not, see <http://www.gnu.org/licenses/>.
%
%
%%%%%%%%%%%%%%%%%%%%%%%%%%%%%%%%%%%%%%%%%%%%%%%%%%%%%%%%%%%%%%%%%%%%%%%%%%%%%
%%%%%%%%%%%%%%%%%%%%%%%%%%%%%%%%%%%%%%%%%%%%%%%%%%%%%%%%%%%%%%%%%%%%%%%%%%%%%
%
% Package pgfplots.sty documentation. 
%
% Copyright 2007/2008 by Christian Feuersaenger.
%
% This program is free software: you can redistribute it and/or modify
% it under the terms of the GNU General Public License as published by
% the Free Software Foundation, either version 3 of the License, or
% (at your option) any later version.
% 
% This program is distributed in the hope that it will be useful,
% but WITHOUT ANY WARRANTY; without even the implied warranty of
% MERCHANTABILITY or FITNESS FOR A PARTICULAR PURPOSE.  See the
% GNU General Public License for more details.
% 
% You should have received a copy of the GNU General Public License
% along with this program.  If not, see <http://www.gnu.org/licenses/>.
%
%
%%%%%%%%%%%%%%%%%%%%%%%%%%%%%%%%%%%%%%%%%%%%%%%%%%%%%%%%%%%%%%%%%%%%%%%%%%%%%
\documentclass[a4paper]{ltxdoc}

\usepackage{makeidx}

% DON't let hyperref overload the format of index and glossary. 
% I want to do that on my own in the stylefiles for makeindex...
\makeatletter
\let\@old@wrindex=\@wrindex
\makeatother

\usepackage{ifpdf}
\ifpdf
	\usepackage[pdftex]{hyperref}
\else
	\usepackage[dvipdfm]{hyperref}
\fi
\hypersetup{pdfborder={0 0 0}}

\makeatletter
\let\@wrindex=\@old@wrindex
\makeatother

% Formatiere Seitennummern im Index:
\newcommand{\indexpageno}[1]{%
	{\bfseries\hyperpage{#1}}%
}


\newcommand{\R}{\mathbb{R}}
\newcommand{\N}{\mathbb{N}}
\newcommand{\Z}{\mathbb{Z}}

\long\def\COMMENTLOWLEVEL#1\ENDCOMMENT{}
\def\ENDCOMMENT{}

\usepackage{textcomp}
\usepackage{booktabs}

\usepackage{calc}
\usepackage[formats]{listings}
%\usepackage{courier} % don't use it - the '^' character can't be copy-pasted in courier

\usepackage{array}
\lstset{%
	basicstyle=\ttfamily,
	language=[LaTeX]tex, % Seems as if \lstset{language=tex} must be invoked BEFORE loading tikz!?
	tabsize=4,
	breaklines=true,
	breakindent=0pt
}

\ifpdf
	\pdfinfo {
		/Author	(Christian Feuersaenger)
	}

\else
	\def\pgfsysdriver{pgfsys-dvipdfm.def}
\fi
%\def\pgfsysdriver{pgfsys-pdftex.def}
\usepackage{pgfplots}

\ifpdf
	\usetikzlibrary{pgfplotsclickable}
\fi

%\usepackage{fp}
% ATTENTION:
% this requires pgf version NEWER than 2.00 :
%\usetikzlibrary{fixedpointarithmetic}

\usetikzlibrary{dateplot}

\usepackage[a4paper,left=2.25cm,right=2.25cm,top=2.5cm,bottom=2.5cm,nohead]{geometry}
\usepackage{amsmath,amssymb}
\usepackage{xxcolor}
\usepackage{pifont}
\usepackage[latin1]{inputenc}
\usepackage{amsmath}
\usepackage{eurosym}

\def\eps{\textsc{eps}}

\input pgfmanual-en-macros.tex

\def\pgfmanualbar{\char`\|}
\makeatletter
\newenvironment{addplotoperation}[3][]{
  \begin{pgfmanualentry}
  	{%
	\let\ltxdoc@marg=\marg
	\let\ltxdoc@oarg=\oarg
	\let\ltxdoc@parg=\parg
	\let\ltxdoc@meta=\meta
	\def\marg##1{{\normalfont\ltxdoc@marg{##1}}}%
	\def\oarg##1{{\normalfont\ltxdoc@oarg{##1}}}%
	\def\parg##1{{\normalfont\ltxdoc@parg{##1}}}%
	\def\meta##1{{\normalfont\ltxdoc@meta{##1}}}%
    \pgfmanualentryheadline{\textcolor{gray}{{\ttfamily\char`\\addplot\ }}%
      \declare{\texttt{#2}} \texttt{#3;}}%
	  \unskip
	 \nobreak
    \pgfmanualentryheadline{\textcolor{gray}{{\texttt{\char`\\addplot}\oarg{style options} \texttt{plot}\oarg{behavior options}}\ }%
      \declare{\texttt{#2}} \texttt{#3} \textcolor{gray}{\meta{trailing path commands}}\texttt{;}}%
    \def\pgfmanualtest{#1}%
    \ifx\pgfmanualtest\@empty%
      \index{#2@\protect\textcolor{gray}{\protect\texttt{plot}}\protect\texttt{ #2}}%
      \index{Plot operations!plot #2@\protect\texttt{plot #2}}%
    \fi%
	}%
    \pgfmanualbody
}
{
  \end{pgfmanualentry}
}

\newenvironment{codekey}[1]{%
  \begin{pgfmanualentry}
 	\pgfmanualentryheadline{{\ttfamily\declare{#1}\textcolor{gray}{/.code}=\marg{...}}\hfill}%
	\def\mykey{#1}%
	\def\mypath{}%
	\def\myname{}%
	\firsttimetrue%
	\decompose#1/\nil%
    \pgfmanualbody
}
{
  \end{pgfmanualentry}
}

\newenvironment{pgfplotscodekey}[1]{%
	\begin{codekey}{/pgfplots/#1}%
}
{
  \end{codekey}
}
\newenvironment{pgfplotscodetwokey}[1]{%
  \begin{pgfmanualentry}
 	\pgfmanualentryheadline{{\ttfamily\declare{/pgfplots/#1}\textcolor{gray}{/.code 2 args}=\marg{...}}\hfill}%
	\def\mykey{/pgfplots/#1}%
	\def\mypath{}%
	\def\myname{}%
	\firsttimetrue%
	\decompose/pgfplots/#1/\nil%
    \pgfmanualbody
}
{
  \end{pgfmanualentry}
}

\newenvironment{pgfplotsxycodekeylist}[1]{%
	\begingroup
	\let\oldpgfmanualentryheadline=\pgfmanualentryheadline
	\def\pgfmanualentryheadline##1{%
		\pgfmanualentryheadline@##1\pgfplots@EOI
	}%
	\def\pgfmanualentryheadline@##1\hfill##2\pgfplots@EOI{%
		\oldpgfmanualentryheadline{{\ttfamily\declare{##1}\textcolor{gray}{/.code}=\marg{...}}\hfill}%
	}
	\begin{pgfplotsxykeylist}{#1}%
}
{
	\end{pgfplotsxykeylist}
	\endgroup
}

\newenvironment{pgfplotskey}[1]{%
  \begin{key}{/pgfplots/#1}%
}
{
  \end{key}
}

\def\choicesep{$\vert$}%
\def\choicearg#1{\texttt{#1}}

\newif\iffirstchoice
\newcommand\mchoice[1]{%
	\begingroup
	\firstchoicetrue
	\foreach \mchoice@ in {#1} {%
		\iffirstchoice
			\global\firstchoicefalse
		\else
			\choicesep
		\fi
		\choicearg{\mchoice@}%
	}%
	\endgroup
}%

% \begin{xykey}{/path/\x label=value}
% \end{xykey}
%
% has same features with 'default', 'initially' etc as key environment
\newenvironment{xykey}[2][]{%
	\begin{pgfmanualentry}
    \def\extrakeytext{}
	\insertpathifneeded{#2}{#1}%
	\expandafter\pgfutil@in@\expandafter=\expandafter{\mykey}%
	\ifpgfutil@in@%
		\expandafter\xykey@eq\mykey\@nil
	\else
		\expandafter\xykey@noeq\mykey\@nil
	\fi
	\pgfmanualbody
}{%
	\end{pgfmanualentry}
}%

% \begin{xystylekey}{/path/\x label=value}
% \end{xystylekey}
%
% has same features with 'default', 'initially' etc as key environment
\newenvironment{xystylekey}[2][]{%
	\begin{pgfmanualentry}
    \def\extrakeytext{style, }
	\insertpathifneeded{#2}{#1}%
	\expandafter\pgfutil@in@\expandafter=\expandafter{\mykey}%
	\ifpgfutil@in@%
		\expandafter\xykey@eq\mykey\@nil
	\else
		\expandafter\xykey@noeq\mykey\@nil
	\fi
	\pgfmanualbody
}{%
	\end{pgfmanualentry}
}%

% \insertpathifneeded{a key}{/pgfplots} -> assign mykey={/pgfplots/a key}
% \insertpathifneeded{/tikz/a key}{/pgfplots} -> assign mykey={/tikz/a key}
%
% #1: the key
% #2: a default path (or empty)
\def\insertpathifneeded#1#2{%
	\def\insertpathifneeded@@{#2}%
	\ifx\insertpathifneeded@@\empty
		\def\mykey{#1}%
	\else
		\insertpathifneeded@#2\@nil
		\ifpgfutil@in@
			\def\mykey{#2/#1}%
		\else
			\def\mykey{#1}%
		\fi
	\fi
}%
\def\insertpathifneeded@#1#2\@nil{%
	\def\insertpathifneeded@@{#1}%
	\def\insertpathifneeded@@@{/}%
	\ifx\insertpathifneeded@@\insertpathifneeded@@@
		\pgfutil@in@true
	\else
		\pgfutil@in@false
	\fi
}%

% \begin{keylist}[default path]
% 	{/path/option 1=value,/path/option 2=value2}
% \end{keylist}
\newenvironment{keylist}[2][]{%
	\begin{pgfmanualentry}
    \def\extrakeytext{}%
	\foreach \xx in {#2} {%
		\expandafter\insertpathifneeded\expandafter{\xx}{#1}%
		\expandafter\extractkey\mykey\@nil%
	}%
	\pgfmanualbody
}{%
  \end{pgfmanualentry}
}%

\newenvironment{pgfplotskeylist}[1]{%
	\begin{keylist}[/pgfplots]{#1}%
}{%
	\end{keylist}%
}

% \begin{xykeylist}[default path]
% 	{/path/option \x1=value,/path/option \x2=value2,/path/option \x3=value}
% \end{xykeylist}
\newenvironment{xykeylist}[2][]{%
	\begin{pgfmanualentry}
    \def\extrakeytext{}
	\foreach \xx in {#2} {%
		\expandafter\insertpathifneeded\expandafter{\xx}{#1}%
		\expandafter\pgfutil@in@\expandafter=\expandafter{\mykey}%
		\ifpgfutil@in@%
			\expandafter\xykey@eq\mykey\@nil
		\else
			\expandafter\xykey@noeq\mykey\@nil
		\fi
	}%
	\pgfmanualbody
}{%
  \end{pgfmanualentry}
}%

\newif\ifxykeyfound

\def\xykey@eq#1=#2\@nil{%
	\def\x{x}%
	\xdef\mykey{#1}%
	\def\xykey@@{#1}%
	\ifx\xykey@@\mykey
		\xykeyfoundfalse
	\else
		\xykeyfoundtrue
	\fi
    \expandafter\extractkey\mykey=#2\@nil%
	\ifxykeyfound
		\def\x{y}%
		\xdef\mykey{#1}%
		\expandafter\extractkey\mykey=#2\@nil%
	\fi
}
\def\xykey@noeq#1\@nil{%
	\def\x{x}%
	\xdef\mykey{#1}%
	\def\xykey@@{#1}%
	\ifx\xykey@@\mykey
		\xykeyfoundfalse
	\else
		\xykeyfoundtrue
	\fi
    \expandafter\extractkey\mykey\@nil%
	\ifxykeyfound
		\def\x{y}%
		\xdef\mykey{#1}%
		\expandafter\extractkey\mykey\@nil%
	\fi
}

% \begin{pgfplotsxykey}{\x label=value}
% \end{pgfplotsxykey}
%
% It introduces the path /pgfplots/ automatically.
%
% has same features with 'default', 'initially' etc as key environment
\newenvironment{pgfplotsxykey}[1]{%
	\begin{xykey}[/pgfplots]{#1}%
}{%
	\end{xykey}%
}


\newenvironment{pgfplotsxykeylist}[1]{%
	\begin{xykeylist}[/pgfplots]{#1}%
}{%
	\end{xykeylist}%
}


\newenvironment{plottype}[1]{%
	\begin{key}{/tikz/#1}%
	\end{key}
  \begin{pgfmanualentry}
    \pgfmanualentryheadline{\textcolor{gray}{{\ttfamily\char`\\addplot+[\declare{#1}]}}}%
    \pgfmanualbody
}
{
  \end{pgfmanualentry}
}

\def\index@prologue{\section*{Index}\addcontentsline{toc}{section}{Index}
}

\newenvironment{pgfplotstablecolumnkey}{%
  \begin{pgfmanualentry}
 	\pgfmanualentryheadline{{\ttfamily\textcolor{gray}{/pgfplots/table/}\declare{columns/\meta{column name}}\textcolor{gray}{/.style}=\marg{key-value-list}}\hfill}%
	\pgfplotsmanualkeyindex{/pgfplots/table/columns}%
    \pgfmanualbody
}
{
  \end{pgfmanualentry}
}
\newenvironment{pgfplotstabledisplaycolumnkey}{%
  \begin{pgfmanualentry}
 	\pgfmanualentryheadline{{\ttfamily\textcolor{gray}{/pgfplots/table/}\declare{display columns/\meta{index}}\textcolor{gray}{/.style}=\marg{key-value-list}}\hfill}%
	\pgfplotsmanualkeyindex{/pgfplots/table/display columns}%
    \pgfmanualbody
}
{
  \end{pgfmanualentry}
}
\newenvironment{pgfplotstablealiaskey}{%
  \begin{pgfmanualentry}
 	\pgfmanualentryheadline{{\ttfamily\textcolor{gray}{/pgfplots/table/}\declare{alias/\meta{col name}}\textcolor{gray}{/.initial}=\marg{real col name}}\hfill}%
	\pgfplotsmanualkeyindex{/pgfplots/table/alias}%
    \pgfmanualbody
}
{
  \end{pgfmanualentry}
}


\def\pgfplotsmanualkeyindex#1{%
	\def\mypath{#1}%
	\def\myname{}%
	\firsttimetrue%
	\decompose#1/\nil%
}
\newenvironment{pgfplotstablecreateonusekey}{%
  \begin{pgfmanualentry}
 	\pgfmanualentryheadline{{\ttfamily\textcolor{gray}{/pgfplots/table/}\declare{create on use/\meta{col name}}\textcolor{gray}{/.style}=\marg{create options}}\hfill}%
	\def\mykey{/pgfplots/table/create on use}%
    \pgfmanualbody
	\pgfplotsmanualkeyindex{/pgfplots/table/create on use}%
}
{
  \end{pgfmanualentry}
}

\def\pgfplotsassertcmdkeyexists#1{%
	\pgfkeysifdefined{/pgfplots/#1/.@cmd}\relax{%
		\pgfplots@error{DOCUMENTATION ERROR: command key /pgfplots/#1 does not exist!}%
	}%
}%

{
\catcode`\ =12%
\gdef\makespaceexpandable{\def\ { }}}%

\def\pgfplotsassertXYcmdkeyexists#1{%
	{\makespaceexpandable\def\x{x}\edef\pgfplotsassertXYcmdkeyexists@tmp{#1}%
	\pgfkeysifdefined{/pgfplots/\pgfplotsassertXYcmdkeyexists@tmp/.@cmd}\relax{%
		\pgfplots@error{DOCUMENTATION ERROR: command key /pgfplots/#1 does not exist!}%
	}}%
	{\makespaceexpandable\def\x{y}\edef\pgfplotsassertXYcmdkeyexists@tmp{#1}%
	\pgfkeysifdefined{/pgfplots/\pgfplotsassertXYcmdkeyexists@tmp/.@cmd}\relax{%
		\pgfplots@error{DOCUMENTATION ERROR: command key /pgfplots/#1 does not exist!}%
	}}%
}%

\def\pgfplotsshortstylekey #1=#2\pgfeov{%
	\pgfplotsassertcmdkeyexists{#1}%
	\pgfplotsassertcmdkeyexists{#2}%
	\begin{pgfplotskey}{#1=\marg{key-value-list}}
		An abbreviation for \texttt{#2/.append style=}\marg{key-value-list}.
	\end{pgfplotskey}
}
\def\pgfplotsshortxystylekey #1=#2\pgfeov{%
	\pgfplotsassertXYcmdkeyexists{#1}%
	\pgfplotsassertXYcmdkeyexists{#2}%
	\begin{pgfplotsxykey}{#1=\marg{key-value-list}}
		An abbreviation for {\def\x{x}\texttt{#2/.append style=}}\marg{key-value-list} (or the respective style for $y$, {\def\x{y}\texttt{#2}}).
	\end{pgfplotsxykey}
}
\def\pgfplotsshortstylekeys #1,#2=#3\pgfeov{%
	\pgfplotsassertcmdkeyexists{#1}%
	\pgfplotsassertcmdkeyexists{#2}%
	\pgfplotsassertcmdkeyexists{#3}%
	\begin{pgfplotskeylist}{%
		#1=\marg{key-value-list},
		#2=\marg{key-value-list}}
		Different abbreviations for \texttt{#3/.append style=}\marg{key-value-list}.
	\end{pgfplotskeylist}
}
\def\pgfplotsshortxystylekeys #1,#2=#3\pgfeov{%
	\pgfplotsassertXYcmdkeyexists{#1}%
	\pgfplotsassertXYcmdkeyexists{#2}%
	\pgfplotsassertXYcmdkeyexists{#3}%
	\begin{pgfplotsxykeylist}{%
		#1=\marg{key-value-list},
		#2=\marg{key-value-list}}
		Different abbreviations for {\def\x{x}\texttt{#3/.append style=}}\marg{key-value-list} (or the respective style for $y$, {\def\x{y}\texttt{#3}}).
	\end{pgfplotsxykeylist}
}

%
% Creates and shows a colormap with specification '#1'.
\def\pgfplotsshowcolormapexample#1{%
	\pgfplotscreatecolormap{tempcolormap}{#1}%
	\pgfplotsshowcolormap{tempcolormap}%
}

% Shows the colormap named '#1'.
\def\pgfplotsshowcolormap#1{%
	\pgfplotscolormaptoshadingspec{#1}{8cm}\result
	\def\tempb{\pgfdeclarehorizontalshading{tempshading}{1cm}}%
	\expandafter\tempb\expandafter{\result}%
	\pgfuseshading{tempshading}%
}

\makeatother


\pgfqkeys{/codeexample}{%
	every codeexample/.style={
		width=8cm,
		/pgfplots/every axis/.append style={legend style={fill=graphicbackground}}
	},
	tabsize=4,
}
\pgfplotsset{
	every axis/.append style={width=7cm},
	filter discard warning=false,
}

\usetikzlibrary{backgrounds,patterns}
% Global styles:
\tikzset{
  shape example/.style={
    color=black!30,
    draw,
    fill=yellow!30,
    line width=.5cm,
    inner xsep=2.5cm,
    inner ysep=0.5cm}
}

\newcommand{\FIXME}[1]{\textcolor{red}{(FIXME: #1)}}

% fuer endvironment 'sidewaysfigure' bspw
% \usepackage{rotating}

\newcommand\Tikz{Ti\textit kZ}
\newcommand\PGF{\textsc{pgf}}
\newcommand\PGFPlots{\textsc{pgfplots}}
\def\PGFPlotstable{\textsc{PgfplotsTable}}


\makeindex

% Fix overful hboxes automatically:
\tolerance=2000
\emergencystretch=10pt

\tikzset{prefix=gnuplot/pgfplots_} % prefix for 'plot function'

\author{%
	Christian Feuers\"anger\footnote{\url{http://wissrech.ins.uni-bonn.de/people/feuersaenger}}\\%
	Institut f\"ur Numerische Simulation\\
	Universit\"at Bonn, Germany}


%\RequirePackage[german,english,francais]{babel}

\pgfplotsmanualexternalexpensivetrue

%\usetikzlibrary{external}
%\tikzexternalize[prefix=figures/]{pgfplots}

\title{%
	Manual for Package \PGFPlots\\
	{\small Version \pgfplotsversion}\\
	{\small\href{http://sourceforge.net/projects/pgfplots}{http://sourceforge.net/projects/pgfplots}}}

%\includeonly{pgfplots.libs}


\begin{document}

\def\plotcoords{%
\addplot coordinates {
(5,8.312e-02)    (17,2.547e-02)   (49,7.407e-03)
(129,2.102e-03)  (321,5.874e-04)  (769,1.623e-04)
(1793,4.442e-05) (4097,1.207e-05) (9217,3.261e-06)
};

\addplot coordinates{
(7,8.472e-02)    (31,3.044e-02)    (111,1.022e-02)
(351,3.303e-03)  (1023,1.039e-03)  (2815,3.196e-04)
(7423,9.658e-05) (18943,2.873e-05) (47103,8.437e-06)
};

\addplot coordinates{
(9,7.881e-02)     (49,3.243e-02)    (209,1.232e-02)
(769,4.454e-03)   (2561,1.551e-03)  (7937,5.236e-04)
(23297,1.723e-04) (65537,5.545e-05) (178177,1.751e-05)
};

\addplot coordinates{
(11,6.887e-02)    (71,3.177e-02)     (351,1.341e-02)
(1471,5.334e-03)  (5503,2.027e-03)   (18943,7.415e-04)
(61183,2.628e-04) (187903,9.063e-05) (553983,3.053e-05)
};

\addplot coordinates{
(13,5.755e-02)     (97,2.925e-02)     (545,1.351e-02)
(2561,5.842e-03)   (10625,2.397e-03)  (40193,9.414e-04)
(141569,3.564e-04) (471041,1.308e-04) 
(1496065,4.670e-05)
};
}%
\maketitle
\begin{abstract}%
\PGFPlots\ draws high--quality function plots in normal or logarithmic scaling with a user-friendly interface directly in \TeX. The user supplies axis labels, legend entries and the plot coordinates for one or more plots and \PGFPlots\ applies axis scaling, computes any logarithms and axis ticks and draws the plots, supporting line plots, scatter plots, piecewise constant plots, bar plots, area plots, mesh-- and surface plots and some more. It is based on Till Tantau's package \PGF/\Tikz.
\end{abstract}
\tableofcontents
\section{Introduction}
This package provides tools to generate plots and labeled axes easily. It draws normal plots, logplots and semi-logplots, in two and three dimensions. Axis ticks, labels, legends (in case of multiple plots) can be added with key-value options. It can cycle through a set of predefined line/marker/color specifications. In summary, its purpose is to simplify the generation of high-quality function and/or data plots, and solving the problems of
\begin{itemize}
	\item consistency of document and font type and font size,
	\item direct use of \TeX\ math mode in axis descriptions,
	\item consistency of data and figures (no third party tool necessary),
	\item inter document consistency using preamble configurations and styles.
\end{itemize}
Although not necessary, separate |.pdf| or |.eps| graphics can be generated using the |external| library developed as part of \Tikz.

\PGFPlots\ is build completely on \Tikz/\PGF. Knowledge of \Tikz\ will simplify the work with \PGFPlots, although it is not required.

\section[About PGFPlots: Preliminaries]{About {\normalfont\PGFPlots}: Preliminaries}
This section contains information about upgrades, the team, the installation (in case you need to do it manually) and troubleshooting. You may skip it completely except for the upgrade remarks.

However, note that this library requires at least \PGF\ version $2.00$. At the time of this writing, many \TeX-distributions still contain the older \PGF\ version $1.18$, so it may be necessary to install a recent \PGF\ prior to using \PGFPlots.

\subsection{Components}
\PGFPlots\ comes with two components:
\begin{enumerate}
	\item the plotting component (which you are currently reading) and
	\item the \PGFPlotstable\ component which simplifies number formatting and postprocessing of numerical tables. It comes as a separate package and has its own manual \href{file:pgfplotstable.pdf}{pgfplotstable.pdf}.
\end{enumerate}

\subsection{Upgrade remarks}
This release provides a lot of improvements which can be found in all detail in \texttt{ChangeLog} for interested readers. However, some attention is useful with respect to the following changes.

\subsubsection{New Optional Features}
\begin{enumerate}
	\item \PGFPlots\ 1.3 comes with user interface improvements. The technical distinction between ``behavior options'' and ``style options'' of older versions is no longer necessary (although still fully supported).

	\item \PGFPlots\ 1.3 has a new feature which allows to \emph{move axis labels tight to tick labels} automatically.

	Since this affects the spacing, it is not enabled be default.

	Use
\begin{codeexample}[code only]
\usepackage{pgfplots}
\pgfplotsset{compat=1.3}
\end{codeexample}
	in your preamble to benefit from the improved spacing\footnote{The |compat=1.3| setting is necessary for new features which move axis labels around like reversed axis. However, \PGFPlots\ will still work as it does in earlier versions, your documents will remain the same if you don't set it explicitly.}.
	Take a look at the next page for the precise definition of |compat|.

	\item \PGFPlots\ 1.3 now supports reversed axes. It is no longer necessary to use work--arounds with negative units.
\pgfkeys{/pdflinks/search key prefixes in/.add={/pgfplots/,}{}}

	Take a look at the |x dir=reverse| key.

	Existing work--arounds will still function properly. Use |\pgfplotsset{compat=1.3}| together with |x dir=reverse| to switch to the new version.
\end{enumerate}

\subsubsection{Old Features Which May Need Attention}
\begin{enumerate}
	\item The |scatter/classes| feature produces proper legends as of version 1.3. This may change the appearance of existing legends of plots with |scatter/classes|.

	\item Due to a small math bug in \PGF\ $2.00$, you \emph{can't} use the math expression `|-x^2|'. It is necessary to use `|0-x^2|' instead. The same holds for
		`|exp(-x^2)|'; use `|exp(0-x^2)|' instead.

		This will be fixed with \PGF\ $>2.00$.

	\item Starting with \PGFPlots\ $1.1$, |\tikzstyle| should \emph{no longer be used} to set \PGFPlots\ options.
	
	Although |\tikzstyle| is still supported for some older \PGFPlots\ options, you should replace any occurance of |\tikzstyle| with |\pgfplotsset{|\meta{style name}|/.style={|\meta{key-value-list}|}}| or the associated |/.append style| variant. See section~\ref{sec:styles} for more detail.
\end{enumerate}
I apologize for any inconvenience caused by these changes.

\begin{pgfplotskey}{compat=\mchoice{1.3,pre 1.3,default,newest} (initially default)}
	The preamble configuration 
\begin{codeexample}[code only]
\usepackage{pgfplots}
\pgfplotsset{compat=1.3}
\end{codeexample}
	allows to choose between backwards compatibility and most recent features.

	\PGFPlots\ version 1.3 comes with new features which may lead to different spacings compared to earlier versions:
	\begin{enumerate}
		\item Version 1.3 comes with the |xlabel near ticks| feature which places (in this case $x$) axis labels ``near'' the tick labels, taking the size of tick labels into account.
		
		Older versions used |xlabel absolute|, i.e.\ an absolute distance between the axis and axis labels (now matter how large or small tick labels are).

		The initial setting \emph{keeps backwards compatibility}.

		You are encouraged to use
\begin{codeexample}[code only]
\usepackage{pgfplots}
\pgfplotsset{compat=1.3}
\end{codeexample}
		in your preamble.
		
		\item Version 1.3 fixes a bug with |anchor=south west| which occurred mainly for reversed axes: the anchors where upside--down. This is fixed now.

		
		If you have written some work--around and want to keep it going, use

		|\pgfplotsset{compat/anchors=pre 1.3}|\pgfmanualpdflabel{/pgfplots/compat/anchors}{} 

		to disable the bug fix.
	\end{enumerate}

	Use |\pgfplotsset{compat=default}| to restore the factory settings.

	The setting |\pgfplotsset{compat=newest}| will always use features of the most recent version. This might result in small changes in the document's appearance.
\end{pgfplotskey}

\subsection{The Team}
\PGFPlots\ has been written mainly by Christian Feuers�nger with many improvements of Pascal Wolkotte and Nick Papior Andersen as a spare time project. We hope it is useful and provides valuable plots.

If you are interested in writing something but don't know how, consider reading the auxiliary manual \href{file:TeX-programming-notes.pdf}{TeX-programming-notes.pdf} which comes with \PGFPlots. It is far from complete, but maybe it is a good starting point (at least for more literature).

\subsection{Acknowledgements}
I thank God for all hours of enjoyed programming. I thank Pascal Wolkotte and Nick Papior Andersen for their programming efforts and contributions as part of the development team. I thank Stefan Tibus, who contributed the |plot shell| feature. I thank Tom Cashman for the contribution of the |reverse legend| feature. Special thanks go to Stefan Pinnow whose tests of \PGFPlots\ lead to numerous quality improvements. Furthermore, I thank Dr.~Schweitzer for many fruitful discussions and Dr.~Meine for his ideas and suggestions. Thanks as well to the many international contributors who provided feature requests or identified bugs or simply manual improvements!

Last but not least, I thank Till Tantau and Mark Wibrow for their excellent graphics (and more) package \PGF\ and \Tikz\ which is the base of \PGFPlots.

%%%%%%%%%%%%%%%%%%%%%%%%%%%%%%%%%%%%%%%%%%%%%%%%%%%%%%%%%%%%%%%%%%%%%%%%%%%%%
%
% Package pgfplots.sty documentation. 
%
% Copyright 2007/2008 by Christian Feuersaenger.
%
% This program is free software: you can redistribute it and/or modify
% it under the terms of the GNU General Public License as published by
% the Free Software Foundation, either version 3 of the License, or
% (at your option) any later version.
% 
% This program is distributed in the hope that it will be useful,
% but WITHOUT ANY WARRANTY; without even the implied warranty of
% MERCHANTABILITY or FITNESS FOR A PARTICULAR PURPOSE.  See the
% GNU General Public License for more details.
% 
% You should have received a copy of the GNU General Public License
% along with this program.  If not, see <http://www.gnu.org/licenses/>.
%
%
%%%%%%%%%%%%%%%%%%%%%%%%%%%%%%%%%%%%%%%%%%%%%%%%%%%%%%%%%%%%%%%%%%%%%%%%%%%%%
%%%%%%%%%%%%%%%%%%%%%%%%%%%%%%%%%%%%%%%%%%%%%%%%%%%%%%%%%%%%%%%%%%%%%%%%%%%%%
%
% Package pgfplots.sty documentation. 
%
% Copyright 2007/2008 by Christian Feuersaenger.
%
% This program is free software: you can redistribute it and/or modify
% it under the terms of the GNU General Public License as published by
% the Free Software Foundation, either version 3 of the License, or
% (at your option) any later version.
% 
% This program is distributed in the hope that it will be useful,
% but WITHOUT ANY WARRANTY; without even the implied warranty of
% MERCHANTABILITY or FITNESS FOR A PARTICULAR PURPOSE.  See the
% GNU General Public License for more details.
% 
% You should have received a copy of the GNU General Public License
% along with this program.  If not, see <http://www.gnu.org/licenses/>.
%
%
%%%%%%%%%%%%%%%%%%%%%%%%%%%%%%%%%%%%%%%%%%%%%%%%%%%%%%%%%%%%%%%%%%%%%%%%%%%%%
\documentclass[a4paper]{ltxdoc}

\usepackage{makeidx}

% DON't let hyperref overload the format of index and glossary. 
% I want to do that on my own in the stylefiles for makeindex...
\makeatletter
\let\@old@wrindex=\@wrindex
\makeatother

\usepackage{ifpdf}
\ifpdf
	\usepackage[pdftex]{hyperref}
\else
	\usepackage[dvipdfm]{hyperref}
\fi
\hypersetup{pdfborder={0 0 0}}

\makeatletter
\let\@wrindex=\@old@wrindex
\makeatother

% Formatiere Seitennummern im Index:
\newcommand{\indexpageno}[1]{%
	{\bfseries\hyperpage{#1}}%
}


\newcommand{\R}{\mathbb{R}}
\newcommand{\N}{\mathbb{N}}
\newcommand{\Z}{\mathbb{Z}}

\long\def\COMMENTLOWLEVEL#1\ENDCOMMENT{}
\def\ENDCOMMENT{}

\usepackage{textcomp}
\usepackage{booktabs}

\usepackage{calc}
\usepackage[formats]{listings}
%\usepackage{courier} % don't use it - the '^' character can't be copy-pasted in courier

\usepackage{array}
\lstset{%
	basicstyle=\ttfamily,
	language=[LaTeX]tex, % Seems as if \lstset{language=tex} must be invoked BEFORE loading tikz!?
	tabsize=4,
	breaklines=true,
	breakindent=0pt
}

\ifpdf
	\pdfinfo {
		/Author	(Christian Feuersaenger)
	}

\else
	\def\pgfsysdriver{pgfsys-dvipdfm.def}
\fi
%\def\pgfsysdriver{pgfsys-pdftex.def}
\usepackage{pgfplots}

\ifpdf
	\usetikzlibrary{pgfplotsclickable}
\fi

%\usepackage{fp}
% ATTENTION:
% this requires pgf version NEWER than 2.00 :
%\usetikzlibrary{fixedpointarithmetic}

\usetikzlibrary{dateplot}

\usepackage[a4paper,left=2.25cm,right=2.25cm,top=2.5cm,bottom=2.5cm,nohead]{geometry}
\usepackage{amsmath,amssymb}
\usepackage{xxcolor}
\usepackage{pifont}
\usepackage[latin1]{inputenc}
\usepackage{amsmath}
\usepackage{eurosym}

\def\eps{\textsc{eps}}

\input pgfmanual-en-macros.tex

\def\pgfmanualbar{\char`\|}
\makeatletter
\newenvironment{addplotoperation}[3][]{
  \begin{pgfmanualentry}
  	{%
	\let\ltxdoc@marg=\marg
	\let\ltxdoc@oarg=\oarg
	\let\ltxdoc@parg=\parg
	\let\ltxdoc@meta=\meta
	\def\marg##1{{\normalfont\ltxdoc@marg{##1}}}%
	\def\oarg##1{{\normalfont\ltxdoc@oarg{##1}}}%
	\def\parg##1{{\normalfont\ltxdoc@parg{##1}}}%
	\def\meta##1{{\normalfont\ltxdoc@meta{##1}}}%
    \pgfmanualentryheadline{\textcolor{gray}{{\ttfamily\char`\\addplot\ }}%
      \declare{\texttt{#2}} \texttt{#3;}}%
	  \unskip
	 \nobreak
    \pgfmanualentryheadline{\textcolor{gray}{{\texttt{\char`\\addplot}\oarg{style options} \texttt{plot}\oarg{behavior options}}\ }%
      \declare{\texttt{#2}} \texttt{#3} \textcolor{gray}{\meta{trailing path commands}}\texttt{;}}%
    \def\pgfmanualtest{#1}%
    \ifx\pgfmanualtest\@empty%
      \index{#2@\protect\textcolor{gray}{\protect\texttt{plot}}\protect\texttt{ #2}}%
      \index{Plot operations!plot #2@\protect\texttt{plot #2}}%
    \fi%
	}%
    \pgfmanualbody
}
{
  \end{pgfmanualentry}
}

\newenvironment{codekey}[1]{%
  \begin{pgfmanualentry}
 	\pgfmanualentryheadline{{\ttfamily\declare{#1}\textcolor{gray}{/.code}=\marg{...}}\hfill}%
	\def\mykey{#1}%
	\def\mypath{}%
	\def\myname{}%
	\firsttimetrue%
	\decompose#1/\nil%
    \pgfmanualbody
}
{
  \end{pgfmanualentry}
}

\newenvironment{pgfplotscodekey}[1]{%
	\begin{codekey}{/pgfplots/#1}%
}
{
  \end{codekey}
}
\newenvironment{pgfplotscodetwokey}[1]{%
  \begin{pgfmanualentry}
 	\pgfmanualentryheadline{{\ttfamily\declare{/pgfplots/#1}\textcolor{gray}{/.code 2 args}=\marg{...}}\hfill}%
	\def\mykey{/pgfplots/#1}%
	\def\mypath{}%
	\def\myname{}%
	\firsttimetrue%
	\decompose/pgfplots/#1/\nil%
    \pgfmanualbody
}
{
  \end{pgfmanualentry}
}

\newenvironment{pgfplotsxycodekeylist}[1]{%
	\begingroup
	\let\oldpgfmanualentryheadline=\pgfmanualentryheadline
	\def\pgfmanualentryheadline##1{%
		\pgfmanualentryheadline@##1\pgfplots@EOI
	}%
	\def\pgfmanualentryheadline@##1\hfill##2\pgfplots@EOI{%
		\oldpgfmanualentryheadline{{\ttfamily\declare{##1}\textcolor{gray}{/.code}=\marg{...}}\hfill}%
	}
	\begin{pgfplotsxykeylist}{#1}%
}
{
	\end{pgfplotsxykeylist}
	\endgroup
}

\newenvironment{pgfplotskey}[1]{%
  \begin{key}{/pgfplots/#1}%
}
{
  \end{key}
}

\def\choicesep{$\vert$}%
\def\choicearg#1{\texttt{#1}}

\newif\iffirstchoice
\newcommand\mchoice[1]{%
	\begingroup
	\firstchoicetrue
	\foreach \mchoice@ in {#1} {%
		\iffirstchoice
			\global\firstchoicefalse
		\else
			\choicesep
		\fi
		\choicearg{\mchoice@}%
	}%
	\endgroup
}%

% \begin{xykey}{/path/\x label=value}
% \end{xykey}
%
% has same features with 'default', 'initially' etc as key environment
\newenvironment{xykey}[2][]{%
	\begin{pgfmanualentry}
    \def\extrakeytext{}
	\insertpathifneeded{#2}{#1}%
	\expandafter\pgfutil@in@\expandafter=\expandafter{\mykey}%
	\ifpgfutil@in@%
		\expandafter\xykey@eq\mykey\@nil
	\else
		\expandafter\xykey@noeq\mykey\@nil
	\fi
	\pgfmanualbody
}{%
	\end{pgfmanualentry}
}%

% \begin{xystylekey}{/path/\x label=value}
% \end{xystylekey}
%
% has same features with 'default', 'initially' etc as key environment
\newenvironment{xystylekey}[2][]{%
	\begin{pgfmanualentry}
    \def\extrakeytext{style, }
	\insertpathifneeded{#2}{#1}%
	\expandafter\pgfutil@in@\expandafter=\expandafter{\mykey}%
	\ifpgfutil@in@%
		\expandafter\xykey@eq\mykey\@nil
	\else
		\expandafter\xykey@noeq\mykey\@nil
	\fi
	\pgfmanualbody
}{%
	\end{pgfmanualentry}
}%

% \insertpathifneeded{a key}{/pgfplots} -> assign mykey={/pgfplots/a key}
% \insertpathifneeded{/tikz/a key}{/pgfplots} -> assign mykey={/tikz/a key}
%
% #1: the key
% #2: a default path (or empty)
\def\insertpathifneeded#1#2{%
	\def\insertpathifneeded@@{#2}%
	\ifx\insertpathifneeded@@\empty
		\def\mykey{#1}%
	\else
		\insertpathifneeded@#2\@nil
		\ifpgfutil@in@
			\def\mykey{#2/#1}%
		\else
			\def\mykey{#1}%
		\fi
	\fi
}%
\def\insertpathifneeded@#1#2\@nil{%
	\def\insertpathifneeded@@{#1}%
	\def\insertpathifneeded@@@{/}%
	\ifx\insertpathifneeded@@\insertpathifneeded@@@
		\pgfutil@in@true
	\else
		\pgfutil@in@false
	\fi
}%

% \begin{keylist}[default path]
% 	{/path/option 1=value,/path/option 2=value2}
% \end{keylist}
\newenvironment{keylist}[2][]{%
	\begin{pgfmanualentry}
    \def\extrakeytext{}%
	\foreach \xx in {#2} {%
		\expandafter\insertpathifneeded\expandafter{\xx}{#1}%
		\expandafter\extractkey\mykey\@nil%
	}%
	\pgfmanualbody
}{%
  \end{pgfmanualentry}
}%

\newenvironment{pgfplotskeylist}[1]{%
	\begin{keylist}[/pgfplots]{#1}%
}{%
	\end{keylist}%
}

% \begin{xykeylist}[default path]
% 	{/path/option \x1=value,/path/option \x2=value2,/path/option \x3=value}
% \end{xykeylist}
\newenvironment{xykeylist}[2][]{%
	\begin{pgfmanualentry}
    \def\extrakeytext{}
	\foreach \xx in {#2} {%
		\expandafter\insertpathifneeded\expandafter{\xx}{#1}%
		\expandafter\pgfutil@in@\expandafter=\expandafter{\mykey}%
		\ifpgfutil@in@%
			\expandafter\xykey@eq\mykey\@nil
		\else
			\expandafter\xykey@noeq\mykey\@nil
		\fi
	}%
	\pgfmanualbody
}{%
  \end{pgfmanualentry}
}%

\newif\ifxykeyfound

\def\xykey@eq#1=#2\@nil{%
	\def\x{x}%
	\xdef\mykey{#1}%
	\def\xykey@@{#1}%
	\ifx\xykey@@\mykey
		\xykeyfoundfalse
	\else
		\xykeyfoundtrue
	\fi
    \expandafter\extractkey\mykey=#2\@nil%
	\ifxykeyfound
		\def\x{y}%
		\xdef\mykey{#1}%
		\expandafter\extractkey\mykey=#2\@nil%
	\fi
}
\def\xykey@noeq#1\@nil{%
	\def\x{x}%
	\xdef\mykey{#1}%
	\def\xykey@@{#1}%
	\ifx\xykey@@\mykey
		\xykeyfoundfalse
	\else
		\xykeyfoundtrue
	\fi
    \expandafter\extractkey\mykey\@nil%
	\ifxykeyfound
		\def\x{y}%
		\xdef\mykey{#1}%
		\expandafter\extractkey\mykey\@nil%
	\fi
}

% \begin{pgfplotsxykey}{\x label=value}
% \end{pgfplotsxykey}
%
% It introduces the path /pgfplots/ automatically.
%
% has same features with 'default', 'initially' etc as key environment
\newenvironment{pgfplotsxykey}[1]{%
	\begin{xykey}[/pgfplots]{#1}%
}{%
	\end{xykey}%
}


\newenvironment{pgfplotsxykeylist}[1]{%
	\begin{xykeylist}[/pgfplots]{#1}%
}{%
	\end{xykeylist}%
}


\newenvironment{plottype}[1]{%
	\begin{key}{/tikz/#1}%
	\end{key}
  \begin{pgfmanualentry}
    \pgfmanualentryheadline{\textcolor{gray}{{\ttfamily\char`\\addplot+[\declare{#1}]}}}%
    \pgfmanualbody
}
{
  \end{pgfmanualentry}
}

\def\index@prologue{\section*{Index}\addcontentsline{toc}{section}{Index}
}

\newenvironment{pgfplotstablecolumnkey}{%
  \begin{pgfmanualentry}
 	\pgfmanualentryheadline{{\ttfamily\textcolor{gray}{/pgfplots/table/}\declare{columns/\meta{column name}}\textcolor{gray}{/.style}=\marg{key-value-list}}\hfill}%
	\pgfplotsmanualkeyindex{/pgfplots/table/columns}%
    \pgfmanualbody
}
{
  \end{pgfmanualentry}
}
\newenvironment{pgfplotstabledisplaycolumnkey}{%
  \begin{pgfmanualentry}
 	\pgfmanualentryheadline{{\ttfamily\textcolor{gray}{/pgfplots/table/}\declare{display columns/\meta{index}}\textcolor{gray}{/.style}=\marg{key-value-list}}\hfill}%
	\pgfplotsmanualkeyindex{/pgfplots/table/display columns}%
    \pgfmanualbody
}
{
  \end{pgfmanualentry}
}
\newenvironment{pgfplotstablealiaskey}{%
  \begin{pgfmanualentry}
 	\pgfmanualentryheadline{{\ttfamily\textcolor{gray}{/pgfplots/table/}\declare{alias/\meta{col name}}\textcolor{gray}{/.initial}=\marg{real col name}}\hfill}%
	\pgfplotsmanualkeyindex{/pgfplots/table/alias}%
    \pgfmanualbody
}
{
  \end{pgfmanualentry}
}


\def\pgfplotsmanualkeyindex#1{%
	\def\mypath{#1}%
	\def\myname{}%
	\firsttimetrue%
	\decompose#1/\nil%
}
\newenvironment{pgfplotstablecreateonusekey}{%
  \begin{pgfmanualentry}
 	\pgfmanualentryheadline{{\ttfamily\textcolor{gray}{/pgfplots/table/}\declare{create on use/\meta{col name}}\textcolor{gray}{/.style}=\marg{create options}}\hfill}%
	\def\mykey{/pgfplots/table/create on use}%
    \pgfmanualbody
	\pgfplotsmanualkeyindex{/pgfplots/table/create on use}%
}
{
  \end{pgfmanualentry}
}

\def\pgfplotsassertcmdkeyexists#1{%
	\pgfkeysifdefined{/pgfplots/#1/.@cmd}\relax{%
		\pgfplots@error{DOCUMENTATION ERROR: command key /pgfplots/#1 does not exist!}%
	}%
}%

{
\catcode`\ =12%
\gdef\makespaceexpandable{\def\ { }}}%

\def\pgfplotsassertXYcmdkeyexists#1{%
	{\makespaceexpandable\def\x{x}\edef\pgfplotsassertXYcmdkeyexists@tmp{#1}%
	\pgfkeysifdefined{/pgfplots/\pgfplotsassertXYcmdkeyexists@tmp/.@cmd}\relax{%
		\pgfplots@error{DOCUMENTATION ERROR: command key /pgfplots/#1 does not exist!}%
	}}%
	{\makespaceexpandable\def\x{y}\edef\pgfplotsassertXYcmdkeyexists@tmp{#1}%
	\pgfkeysifdefined{/pgfplots/\pgfplotsassertXYcmdkeyexists@tmp/.@cmd}\relax{%
		\pgfplots@error{DOCUMENTATION ERROR: command key /pgfplots/#1 does not exist!}%
	}}%
}%

\def\pgfplotsshortstylekey #1=#2\pgfeov{%
	\pgfplotsassertcmdkeyexists{#1}%
	\pgfplotsassertcmdkeyexists{#2}%
	\begin{pgfplotskey}{#1=\marg{key-value-list}}
		An abbreviation for \texttt{#2/.append style=}\marg{key-value-list}.
	\end{pgfplotskey}
}
\def\pgfplotsshortxystylekey #1=#2\pgfeov{%
	\pgfplotsassertXYcmdkeyexists{#1}%
	\pgfplotsassertXYcmdkeyexists{#2}%
	\begin{pgfplotsxykey}{#1=\marg{key-value-list}}
		An abbreviation for {\def\x{x}\texttt{#2/.append style=}}\marg{key-value-list} (or the respective style for $y$, {\def\x{y}\texttt{#2}}).
	\end{pgfplotsxykey}
}
\def\pgfplotsshortstylekeys #1,#2=#3\pgfeov{%
	\pgfplotsassertcmdkeyexists{#1}%
	\pgfplotsassertcmdkeyexists{#2}%
	\pgfplotsassertcmdkeyexists{#3}%
	\begin{pgfplotskeylist}{%
		#1=\marg{key-value-list},
		#2=\marg{key-value-list}}
		Different abbreviations for \texttt{#3/.append style=}\marg{key-value-list}.
	\end{pgfplotskeylist}
}
\def\pgfplotsshortxystylekeys #1,#2=#3\pgfeov{%
	\pgfplotsassertXYcmdkeyexists{#1}%
	\pgfplotsassertXYcmdkeyexists{#2}%
	\pgfplotsassertXYcmdkeyexists{#3}%
	\begin{pgfplotsxykeylist}{%
		#1=\marg{key-value-list},
		#2=\marg{key-value-list}}
		Different abbreviations for {\def\x{x}\texttt{#3/.append style=}}\marg{key-value-list} (or the respective style for $y$, {\def\x{y}\texttt{#3}}).
	\end{pgfplotsxykeylist}
}

%
% Creates and shows a colormap with specification '#1'.
\def\pgfplotsshowcolormapexample#1{%
	\pgfplotscreatecolormap{tempcolormap}{#1}%
	\pgfplotsshowcolormap{tempcolormap}%
}

% Shows the colormap named '#1'.
\def\pgfplotsshowcolormap#1{%
	\pgfplotscolormaptoshadingspec{#1}{8cm}\result
	\def\tempb{\pgfdeclarehorizontalshading{tempshading}{1cm}}%
	\expandafter\tempb\expandafter{\result}%
	\pgfuseshading{tempshading}%
}

\makeatother


\pgfqkeys{/codeexample}{%
	every codeexample/.style={
		width=8cm,
		/pgfplots/every axis/.append style={legend style={fill=graphicbackground}}
	},
	tabsize=4,
}
\pgfplotsset{
	every axis/.append style={width=7cm},
	filter discard warning=false,
}

\usetikzlibrary{backgrounds,patterns}
% Global styles:
\tikzset{
  shape example/.style={
    color=black!30,
    draw,
    fill=yellow!30,
    line width=.5cm,
    inner xsep=2.5cm,
    inner ysep=0.5cm}
}

\newcommand{\FIXME}[1]{\textcolor{red}{(FIXME: #1)}}

% fuer endvironment 'sidewaysfigure' bspw
% \usepackage{rotating}

\newcommand\Tikz{Ti\textit kZ}
\newcommand\PGF{\textsc{pgf}}
\newcommand\PGFPlots{\textsc{pgfplots}}
\def\PGFPlotstable{\textsc{PgfplotsTable}}


\makeindex

% Fix overful hboxes automatically:
\tolerance=2000
\emergencystretch=10pt

\tikzset{prefix=gnuplot/pgfplots_} % prefix for 'plot function'

\author{%
	Christian Feuers\"anger\footnote{\url{http://wissrech.ins.uni-bonn.de/people/feuersaenger}}\\%
	Institut f\"ur Numerische Simulation\\
	Universit\"at Bonn, Germany}


%\RequirePackage[german,english,francais]{babel}

\pgfplotsmanualexternalexpensivetrue

%\usetikzlibrary{external}
%\tikzexternalize[prefix=figures/]{pgfplots}

\title{%
	Manual for Package \PGFPlots\\
	{\small Version \pgfplotsversion}\\
	{\small\href{http://sourceforge.net/projects/pgfplots}{http://sourceforge.net/projects/pgfplots}}}

%\includeonly{pgfplots.libs}


\begin{document}

\def\plotcoords{%
\addplot coordinates {
(5,8.312e-02)    (17,2.547e-02)   (49,7.407e-03)
(129,2.102e-03)  (321,5.874e-04)  (769,1.623e-04)
(1793,4.442e-05) (4097,1.207e-05) (9217,3.261e-06)
};

\addplot coordinates{
(7,8.472e-02)    (31,3.044e-02)    (111,1.022e-02)
(351,3.303e-03)  (1023,1.039e-03)  (2815,3.196e-04)
(7423,9.658e-05) (18943,2.873e-05) (47103,8.437e-06)
};

\addplot coordinates{
(9,7.881e-02)     (49,3.243e-02)    (209,1.232e-02)
(769,4.454e-03)   (2561,1.551e-03)  (7937,5.236e-04)
(23297,1.723e-04) (65537,5.545e-05) (178177,1.751e-05)
};

\addplot coordinates{
(11,6.887e-02)    (71,3.177e-02)     (351,1.341e-02)
(1471,5.334e-03)  (5503,2.027e-03)   (18943,7.415e-04)
(61183,2.628e-04) (187903,9.063e-05) (553983,3.053e-05)
};

\addplot coordinates{
(13,5.755e-02)     (97,2.925e-02)     (545,1.351e-02)
(2561,5.842e-03)   (10625,2.397e-03)  (40193,9.414e-04)
(141569,3.564e-04) (471041,1.308e-04) 
(1496065,4.670e-05)
};
}%
\maketitle
\begin{abstract}%
\PGFPlots\ draws high--quality function plots in normal or logarithmic scaling with a user-friendly interface directly in \TeX. The user supplies axis labels, legend entries and the plot coordinates for one or more plots and \PGFPlots\ applies axis scaling, computes any logarithms and axis ticks and draws the plots, supporting line plots, scatter plots, piecewise constant plots, bar plots, area plots, mesh-- and surface plots and some more. It is based on Till Tantau's package \PGF/\Tikz.
\end{abstract}
\tableofcontents
\section{Introduction}
This package provides tools to generate plots and labeled axes easily. It draws normal plots, logplots and semi-logplots, in two and three dimensions. Axis ticks, labels, legends (in case of multiple plots) can be added with key-value options. It can cycle through a set of predefined line/marker/color specifications. In summary, its purpose is to simplify the generation of high-quality function and/or data plots, and solving the problems of
\begin{itemize}
	\item consistency of document and font type and font size,
	\item direct use of \TeX\ math mode in axis descriptions,
	\item consistency of data and figures (no third party tool necessary),
	\item inter document consistency using preamble configurations and styles.
\end{itemize}
Although not necessary, separate |.pdf| or |.eps| graphics can be generated using the |external| library developed as part of \Tikz.

\PGFPlots\ is build completely on \Tikz/\PGF. Knowledge of \Tikz\ will simplify the work with \PGFPlots, although it is not required.

\section[About PGFPlots: Preliminaries]{About {\normalfont\PGFPlots}: Preliminaries}
This section contains information about upgrades, the team, the installation (in case you need to do it manually) and troubleshooting. You may skip it completely except for the upgrade remarks.

However, note that this library requires at least \PGF\ version $2.00$. At the time of this writing, many \TeX-distributions still contain the older \PGF\ version $1.18$, so it may be necessary to install a recent \PGF\ prior to using \PGFPlots.

\subsection{Components}
\PGFPlots\ comes with two components:
\begin{enumerate}
	\item the plotting component (which you are currently reading) and
	\item the \PGFPlotstable\ component which simplifies number formatting and postprocessing of numerical tables. It comes as a separate package and has its own manual \href{file:pgfplotstable.pdf}{pgfplotstable.pdf}.
\end{enumerate}

\subsection{Upgrade remarks}
This release provides a lot of improvements which can be found in all detail in \texttt{ChangeLog} for interested readers. However, some attention is useful with respect to the following changes.

\subsubsection{New Optional Features}
\begin{enumerate}
	\item \PGFPlots\ 1.3 comes with user interface improvements. The technical distinction between ``behavior options'' and ``style options'' of older versions is no longer necessary (although still fully supported).

	\item \PGFPlots\ 1.3 has a new feature which allows to \emph{move axis labels tight to tick labels} automatically.

	Since this affects the spacing, it is not enabled be default.

	Use
\begin{codeexample}[code only]
\usepackage{pgfplots}
\pgfplotsset{compat=1.3}
\end{codeexample}
	in your preamble to benefit from the improved spacing\footnote{The |compat=1.3| setting is necessary for new features which move axis labels around like reversed axis. However, \PGFPlots\ will still work as it does in earlier versions, your documents will remain the same if you don't set it explicitly.}.
	Take a look at the next page for the precise definition of |compat|.

	\item \PGFPlots\ 1.3 now supports reversed axes. It is no longer necessary to use work--arounds with negative units.
\pgfkeys{/pdflinks/search key prefixes in/.add={/pgfplots/,}{}}

	Take a look at the |x dir=reverse| key.

	Existing work--arounds will still function properly. Use |\pgfplotsset{compat=1.3}| together with |x dir=reverse| to switch to the new version.
\end{enumerate}

\subsubsection{Old Features Which May Need Attention}
\begin{enumerate}
	\item The |scatter/classes| feature produces proper legends as of version 1.3. This may change the appearance of existing legends of plots with |scatter/classes|.

	\item Due to a small math bug in \PGF\ $2.00$, you \emph{can't} use the math expression `|-x^2|'. It is necessary to use `|0-x^2|' instead. The same holds for
		`|exp(-x^2)|'; use `|exp(0-x^2)|' instead.

		This will be fixed with \PGF\ $>2.00$.

	\item Starting with \PGFPlots\ $1.1$, |\tikzstyle| should \emph{no longer be used} to set \PGFPlots\ options.
	
	Although |\tikzstyle| is still supported for some older \PGFPlots\ options, you should replace any occurance of |\tikzstyle| with |\pgfplotsset{|\meta{style name}|/.style={|\meta{key-value-list}|}}| or the associated |/.append style| variant. See section~\ref{sec:styles} for more detail.
\end{enumerate}
I apologize for any inconvenience caused by these changes.

\begin{pgfplotskey}{compat=\mchoice{1.3,pre 1.3,default,newest} (initially default)}
	The preamble configuration 
\begin{codeexample}[code only]
\usepackage{pgfplots}
\pgfplotsset{compat=1.3}
\end{codeexample}
	allows to choose between backwards compatibility and most recent features.

	\PGFPlots\ version 1.3 comes with new features which may lead to different spacings compared to earlier versions:
	\begin{enumerate}
		\item Version 1.3 comes with the |xlabel near ticks| feature which places (in this case $x$) axis labels ``near'' the tick labels, taking the size of tick labels into account.
		
		Older versions used |xlabel absolute|, i.e.\ an absolute distance between the axis and axis labels (now matter how large or small tick labels are).

		The initial setting \emph{keeps backwards compatibility}.

		You are encouraged to use
\begin{codeexample}[code only]
\usepackage{pgfplots}
\pgfplotsset{compat=1.3}
\end{codeexample}
		in your preamble.
		
		\item Version 1.3 fixes a bug with |anchor=south west| which occurred mainly for reversed axes: the anchors where upside--down. This is fixed now.

		
		If you have written some work--around and want to keep it going, use

		|\pgfplotsset{compat/anchors=pre 1.3}|\pgfmanualpdflabel{/pgfplots/compat/anchors}{} 

		to disable the bug fix.
	\end{enumerate}

	Use |\pgfplotsset{compat=default}| to restore the factory settings.

	The setting |\pgfplotsset{compat=newest}| will always use features of the most recent version. This might result in small changes in the document's appearance.
\end{pgfplotskey}

\subsection{The Team}
\PGFPlots\ has been written mainly by Christian Feuers�nger with many improvements of Pascal Wolkotte and Nick Papior Andersen as a spare time project. We hope it is useful and provides valuable plots.

If you are interested in writing something but don't know how, consider reading the auxiliary manual \href{file:TeX-programming-notes.pdf}{TeX-programming-notes.pdf} which comes with \PGFPlots. It is far from complete, but maybe it is a good starting point (at least for more literature).

\subsection{Acknowledgements}
I thank God for all hours of enjoyed programming. I thank Pascal Wolkotte and Nick Papior Andersen for their programming efforts and contributions as part of the development team. I thank Stefan Tibus, who contributed the |plot shell| feature. I thank Tom Cashman for the contribution of the |reverse legend| feature. Special thanks go to Stefan Pinnow whose tests of \PGFPlots\ lead to numerous quality improvements. Furthermore, I thank Dr.~Schweitzer for many fruitful discussions and Dr.~Meine for his ideas and suggestions. Thanks as well to the many international contributors who provided feature requests or identified bugs or simply manual improvements!

Last but not least, I thank Till Tantau and Mark Wibrow for their excellent graphics (and more) package \PGF\ and \Tikz\ which is the base of \PGFPlots.

%%%%%%%%%%%%%%%%%%%%%%%%%%%%%%%%%%%%%%%%%%%%%%%%%%%%%%%%%%%%%%%%%%%%%%%%%%%%%
%
% Package pgfplots.sty documentation. 
%
% Copyright 2007/2008 by Christian Feuersaenger.
%
% This program is free software: you can redistribute it and/or modify
% it under the terms of the GNU General Public License as published by
% the Free Software Foundation, either version 3 of the License, or
% (at your option) any later version.
% 
% This program is distributed in the hope that it will be useful,
% but WITHOUT ANY WARRANTY; without even the implied warranty of
% MERCHANTABILITY or FITNESS FOR A PARTICULAR PURPOSE.  See the
% GNU General Public License for more details.
% 
% You should have received a copy of the GNU General Public License
% along with this program.  If not, see <http://www.gnu.org/licenses/>.
%
%
%%%%%%%%%%%%%%%%%%%%%%%%%%%%%%%%%%%%%%%%%%%%%%%%%%%%%%%%%%%%%%%%%%%%%%%%%%%%%
%%%%%%%%%%%%%%%%%%%%%%%%%%%%%%%%%%%%%%%%%%%%%%%%%%%%%%%%%%%%%%%%%%%%%%%%%%%%%
%
% Package pgfplots.sty documentation. 
%
% Copyright 2007/2008 by Christian Feuersaenger.
%
% This program is free software: you can redistribute it and/or modify
% it under the terms of the GNU General Public License as published by
% the Free Software Foundation, either version 3 of the License, or
% (at your option) any later version.
% 
% This program is distributed in the hope that it will be useful,
% but WITHOUT ANY WARRANTY; without even the implied warranty of
% MERCHANTABILITY or FITNESS FOR A PARTICULAR PURPOSE.  See the
% GNU General Public License for more details.
% 
% You should have received a copy of the GNU General Public License
% along with this program.  If not, see <http://www.gnu.org/licenses/>.
%
%
%%%%%%%%%%%%%%%%%%%%%%%%%%%%%%%%%%%%%%%%%%%%%%%%%%%%%%%%%%%%%%%%%%%%%%%%%%%%%
\documentclass[a4paper]{ltxdoc}

\usepackage{makeidx}

% DON't let hyperref overload the format of index and glossary. 
% I want to do that on my own in the stylefiles for makeindex...
\makeatletter
\let\@old@wrindex=\@wrindex
\makeatother

\usepackage{ifpdf}
\ifpdf
	\usepackage[pdftex]{hyperref}
\else
	\usepackage[dvipdfm]{hyperref}
\fi
\hypersetup{pdfborder={0 0 0}}

\makeatletter
\let\@wrindex=\@old@wrindex
\makeatother

% Formatiere Seitennummern im Index:
\newcommand{\indexpageno}[1]{%
	{\bfseries\hyperpage{#1}}%
}


\newcommand{\R}{\mathbb{R}}
\newcommand{\N}{\mathbb{N}}
\newcommand{\Z}{\mathbb{Z}}

\long\def\COMMENTLOWLEVEL#1\ENDCOMMENT{}
\def\ENDCOMMENT{}

\usepackage{textcomp}
\usepackage{booktabs}

\usepackage{calc}
\usepackage[formats]{listings}
%\usepackage{courier} % don't use it - the '^' character can't be copy-pasted in courier

\usepackage{array}
\lstset{%
	basicstyle=\ttfamily,
	language=[LaTeX]tex, % Seems as if \lstset{language=tex} must be invoked BEFORE loading tikz!?
	tabsize=4,
	breaklines=true,
	breakindent=0pt
}

\ifpdf
	\pdfinfo {
		/Author	(Christian Feuersaenger)
	}

\else
	\def\pgfsysdriver{pgfsys-dvipdfm.def}
\fi
%\def\pgfsysdriver{pgfsys-pdftex.def}
\usepackage{pgfplots}

\ifpdf
	\usetikzlibrary{pgfplotsclickable}
\fi

%\usepackage{fp}
% ATTENTION:
% this requires pgf version NEWER than 2.00 :
%\usetikzlibrary{fixedpointarithmetic}

\usetikzlibrary{dateplot}

\usepackage[a4paper,left=2.25cm,right=2.25cm,top=2.5cm,bottom=2.5cm,nohead]{geometry}
\usepackage{amsmath,amssymb}
\usepackage{xxcolor}
\usepackage{pifont}
\usepackage[latin1]{inputenc}
\usepackage{amsmath}
\usepackage{eurosym}
\input{pgfplots-macros}

\pgfqkeys{/codeexample}{%
	every codeexample/.style={
		width=8cm,
		/pgfplots/every axis/.append style={legend style={fill=graphicbackground}}
	},
	tabsize=4,
}
\pgfplotsset{
	every axis/.append style={width=7cm},
	filter discard warning=false,
}

\usetikzlibrary{backgrounds,patterns}
% Global styles:
\tikzset{
  shape example/.style={
    color=black!30,
    draw,
    fill=yellow!30,
    line width=.5cm,
    inner xsep=2.5cm,
    inner ysep=0.5cm}
}

\newcommand{\FIXME}[1]{\textcolor{red}{(FIXME: #1)}}

% fuer endvironment 'sidewaysfigure' bspw
% \usepackage{rotating}

\newcommand\Tikz{Ti\textit kZ}
\newcommand\PGF{\textsc{pgf}}
\newcommand\PGFPlots{\textsc{pgfplots}}
\def\PGFPlotstable{\textsc{PgfplotsTable}}


\makeindex

% Fix overful hboxes automatically:
\tolerance=2000
\emergencystretch=10pt

\tikzset{prefix=gnuplot/pgfplots_} % prefix for 'plot function'

\author{%
	Christian Feuers\"anger\footnote{\url{http://wissrech.ins.uni-bonn.de/people/feuersaenger}}\\%
	Institut f\"ur Numerische Simulation\\
	Universit\"at Bonn, Germany}


%\RequirePackage[german,english,francais]{babel}

\pgfplotsmanualexternalexpensivetrue

%\usetikzlibrary{external}
%\tikzexternalize[prefix=figures/]{pgfplots}

\title{%
	Manual for Package \PGFPlots\\
	{\small Version \pgfplotsversion}\\
	{\small\href{http://sourceforge.net/projects/pgfplots}{http://sourceforge.net/projects/pgfplots}}}

%\includeonly{pgfplots.libs}


\begin{document}

\def\plotcoords{%
\addplot coordinates {
(5,8.312e-02)    (17,2.547e-02)   (49,7.407e-03)
(129,2.102e-03)  (321,5.874e-04)  (769,1.623e-04)
(1793,4.442e-05) (4097,1.207e-05) (9217,3.261e-06)
};

\addplot coordinates{
(7,8.472e-02)    (31,3.044e-02)    (111,1.022e-02)
(351,3.303e-03)  (1023,1.039e-03)  (2815,3.196e-04)
(7423,9.658e-05) (18943,2.873e-05) (47103,8.437e-06)
};

\addplot coordinates{
(9,7.881e-02)     (49,3.243e-02)    (209,1.232e-02)
(769,4.454e-03)   (2561,1.551e-03)  (7937,5.236e-04)
(23297,1.723e-04) (65537,5.545e-05) (178177,1.751e-05)
};

\addplot coordinates{
(11,6.887e-02)    (71,3.177e-02)     (351,1.341e-02)
(1471,5.334e-03)  (5503,2.027e-03)   (18943,7.415e-04)
(61183,2.628e-04) (187903,9.063e-05) (553983,3.053e-05)
};

\addplot coordinates{
(13,5.755e-02)     (97,2.925e-02)     (545,1.351e-02)
(2561,5.842e-03)   (10625,2.397e-03)  (40193,9.414e-04)
(141569,3.564e-04) (471041,1.308e-04) 
(1496065,4.670e-05)
};
}%
\maketitle
\begin{abstract}%
\PGFPlots\ draws high--quality function plots in normal or logarithmic scaling with a user-friendly interface directly in \TeX. The user supplies axis labels, legend entries and the plot coordinates for one or more plots and \PGFPlots\ applies axis scaling, computes any logarithms and axis ticks and draws the plots, supporting line plots, scatter plots, piecewise constant plots, bar plots, area plots, mesh-- and surface plots and some more. It is based on Till Tantau's package \PGF/\Tikz.
\end{abstract}
\tableofcontents
\section{Introduction}
This package provides tools to generate plots and labeled axes easily. It draws normal plots, logplots and semi-logplots, in two and three dimensions. Axis ticks, labels, legends (in case of multiple plots) can be added with key-value options. It can cycle through a set of predefined line/marker/color specifications. In summary, its purpose is to simplify the generation of high-quality function and/or data plots, and solving the problems of
\begin{itemize}
	\item consistency of document and font type and font size,
	\item direct use of \TeX\ math mode in axis descriptions,
	\item consistency of data and figures (no third party tool necessary),
	\item inter document consistency using preamble configurations and styles.
\end{itemize}
Although not necessary, separate |.pdf| or |.eps| graphics can be generated using the |external| library developed as part of \Tikz.

\PGFPlots\ is build completely on \Tikz/\PGF. Knowledge of \Tikz\ will simplify the work with \PGFPlots, although it is not required.

\section[About PGFPlots: Preliminaries]{About {\normalfont\PGFPlots}: Preliminaries}
This section contains information about upgrades, the team, the installation (in case you need to do it manually) and troubleshooting. You may skip it completely except for the upgrade remarks.

However, note that this library requires at least \PGF\ version $2.00$. At the time of this writing, many \TeX-distributions still contain the older \PGF\ version $1.18$, so it may be necessary to install a recent \PGF\ prior to using \PGFPlots.

\subsection{Components}
\PGFPlots\ comes with two components:
\begin{enumerate}
	\item the plotting component (which you are currently reading) and
	\item the \PGFPlotstable\ component which simplifies number formatting and postprocessing of numerical tables. It comes as a separate package and has its own manual \href{file:pgfplotstable.pdf}{pgfplotstable.pdf}.
\end{enumerate}

\subsection{Upgrade remarks}
This release provides a lot of improvements which can be found in all detail in \texttt{ChangeLog} for interested readers. However, some attention is useful with respect to the following changes.

\subsubsection{New Optional Features}
\begin{enumerate}
	\item \PGFPlots\ 1.3 comes with user interface improvements. The technical distinction between ``behavior options'' and ``style options'' of older versions is no longer necessary (although still fully supported).

	\item \PGFPlots\ 1.3 has a new feature which allows to \emph{move axis labels tight to tick labels} automatically.

	Since this affects the spacing, it is not enabled be default.

	Use
\begin{codeexample}[code only]
\usepackage{pgfplots}
\pgfplotsset{compat=1.3}
\end{codeexample}
	in your preamble to benefit from the improved spacing\footnote{The |compat=1.3| setting is necessary for new features which move axis labels around like reversed axis. However, \PGFPlots\ will still work as it does in earlier versions, your documents will remain the same if you don't set it explicitly.}.
	Take a look at the next page for the precise definition of |compat|.

	\item \PGFPlots\ 1.3 now supports reversed axes. It is no longer necessary to use work--arounds with negative units.
\pgfkeys{/pdflinks/search key prefixes in/.add={/pgfplots/,}{}}

	Take a look at the |x dir=reverse| key.

	Existing work--arounds will still function properly. Use |\pgfplotsset{compat=1.3}| together with |x dir=reverse| to switch to the new version.
\end{enumerate}

\subsubsection{Old Features Which May Need Attention}
\begin{enumerate}
	\item The |scatter/classes| feature produces proper legends as of version 1.3. This may change the appearance of existing legends of plots with |scatter/classes|.

	\item Due to a small math bug in \PGF\ $2.00$, you \emph{can't} use the math expression `|-x^2|'. It is necessary to use `|0-x^2|' instead. The same holds for
		`|exp(-x^2)|'; use `|exp(0-x^2)|' instead.

		This will be fixed with \PGF\ $>2.00$.

	\item Starting with \PGFPlots\ $1.1$, |\tikzstyle| should \emph{no longer be used} to set \PGFPlots\ options.
	
	Although |\tikzstyle| is still supported for some older \PGFPlots\ options, you should replace any occurance of |\tikzstyle| with |\pgfplotsset{|\meta{style name}|/.style={|\meta{key-value-list}|}}| or the associated |/.append style| variant. See section~\ref{sec:styles} for more detail.
\end{enumerate}
I apologize for any inconvenience caused by these changes.

\begin{pgfplotskey}{compat=\mchoice{1.3,pre 1.3,default,newest} (initially default)}
	The preamble configuration 
\begin{codeexample}[code only]
\usepackage{pgfplots}
\pgfplotsset{compat=1.3}
\end{codeexample}
	allows to choose between backwards compatibility and most recent features.

	\PGFPlots\ version 1.3 comes with new features which may lead to different spacings compared to earlier versions:
	\begin{enumerate}
		\item Version 1.3 comes with the |xlabel near ticks| feature which places (in this case $x$) axis labels ``near'' the tick labels, taking the size of tick labels into account.
		
		Older versions used |xlabel absolute|, i.e.\ an absolute distance between the axis and axis labels (now matter how large or small tick labels are).

		The initial setting \emph{keeps backwards compatibility}.

		You are encouraged to use
\begin{codeexample}[code only]
\usepackage{pgfplots}
\pgfplotsset{compat=1.3}
\end{codeexample}
		in your preamble.
		
		\item Version 1.3 fixes a bug with |anchor=south west| which occurred mainly for reversed axes: the anchors where upside--down. This is fixed now.

		
		If you have written some work--around and want to keep it going, use

		|\pgfplotsset{compat/anchors=pre 1.3}|\pgfmanualpdflabel{/pgfplots/compat/anchors}{} 

		to disable the bug fix.
	\end{enumerate}

	Use |\pgfplotsset{compat=default}| to restore the factory settings.

	The setting |\pgfplotsset{compat=newest}| will always use features of the most recent version. This might result in small changes in the document's appearance.
\end{pgfplotskey}

\subsection{The Team}
\PGFPlots\ has been written mainly by Christian Feuers�nger with many improvements of Pascal Wolkotte and Nick Papior Andersen as a spare time project. We hope it is useful and provides valuable plots.

If you are interested in writing something but don't know how, consider reading the auxiliary manual \href{file:TeX-programming-notes.pdf}{TeX-programming-notes.pdf} which comes with \PGFPlots. It is far from complete, but maybe it is a good starting point (at least for more literature).

\subsection{Acknowledgements}
I thank God for all hours of enjoyed programming. I thank Pascal Wolkotte and Nick Papior Andersen for their programming efforts and contributions as part of the development team. I thank Stefan Tibus, who contributed the |plot shell| feature. I thank Tom Cashman for the contribution of the |reverse legend| feature. Special thanks go to Stefan Pinnow whose tests of \PGFPlots\ lead to numerous quality improvements. Furthermore, I thank Dr.~Schweitzer for many fruitful discussions and Dr.~Meine for his ideas and suggestions. Thanks as well to the many international contributors who provided feature requests or identified bugs or simply manual improvements!

Last but not least, I thank Till Tantau and Mark Wibrow for their excellent graphics (and more) package \PGF\ and \Tikz\ which is the base of \PGFPlots.

%%%%%%%%%%%%%%%%%%%%%%%%%%%%%%%%%%%%%%%%%%%%%%%%%%%%%%%%%%%%%%%%%%%%%%%%%%%%%
%
% Package pgfplots.sty documentation. 
%
% Copyright 2007/2008 by Christian Feuersaenger.
%
% This program is free software: you can redistribute it and/or modify
% it under the terms of the GNU General Public License as published by
% the Free Software Foundation, either version 3 of the License, or
% (at your option) any later version.
% 
% This program is distributed in the hope that it will be useful,
% but WITHOUT ANY WARRANTY; without even the implied warranty of
% MERCHANTABILITY or FITNESS FOR A PARTICULAR PURPOSE.  See the
% GNU General Public License for more details.
% 
% You should have received a copy of the GNU General Public License
% along with this program.  If not, see <http://www.gnu.org/licenses/>.
%
%
%%%%%%%%%%%%%%%%%%%%%%%%%%%%%%%%%%%%%%%%%%%%%%%%%%%%%%%%%%%%%%%%%%%%%%%%%%%%%
\input{pgfplots.preamble.tex}
%\RequirePackage[german,english,francais]{babel}

\pgfplotsmanualexternalexpensivetrue

%\usetikzlibrary{external}
%\tikzexternalize[prefix=figures/]{pgfplots}

\title{%
	Manual for Package \PGFPlots\\
	{\small Version \pgfplotsversion}\\
	{\small\href{http://sourceforge.net/projects/pgfplots}{http://sourceforge.net/projects/pgfplots}}}

%\includeonly{pgfplots.libs}


\begin{document}

\def\plotcoords{%
\addplot coordinates {
(5,8.312e-02)    (17,2.547e-02)   (49,7.407e-03)
(129,2.102e-03)  (321,5.874e-04)  (769,1.623e-04)
(1793,4.442e-05) (4097,1.207e-05) (9217,3.261e-06)
};

\addplot coordinates{
(7,8.472e-02)    (31,3.044e-02)    (111,1.022e-02)
(351,3.303e-03)  (1023,1.039e-03)  (2815,3.196e-04)
(7423,9.658e-05) (18943,2.873e-05) (47103,8.437e-06)
};

\addplot coordinates{
(9,7.881e-02)     (49,3.243e-02)    (209,1.232e-02)
(769,4.454e-03)   (2561,1.551e-03)  (7937,5.236e-04)
(23297,1.723e-04) (65537,5.545e-05) (178177,1.751e-05)
};

\addplot coordinates{
(11,6.887e-02)    (71,3.177e-02)     (351,1.341e-02)
(1471,5.334e-03)  (5503,2.027e-03)   (18943,7.415e-04)
(61183,2.628e-04) (187903,9.063e-05) (553983,3.053e-05)
};

\addplot coordinates{
(13,5.755e-02)     (97,2.925e-02)     (545,1.351e-02)
(2561,5.842e-03)   (10625,2.397e-03)  (40193,9.414e-04)
(141569,3.564e-04) (471041,1.308e-04) 
(1496065,4.670e-05)
};
}%
\maketitle
\begin{abstract}%
\PGFPlots\ draws high--quality function plots in normal or logarithmic scaling with a user-friendly interface directly in \TeX. The user supplies axis labels, legend entries and the plot coordinates for one or more plots and \PGFPlots\ applies axis scaling, computes any logarithms and axis ticks and draws the plots, supporting line plots, scatter plots, piecewise constant plots, bar plots, area plots, mesh-- and surface plots and some more. It is based on Till Tantau's package \PGF/\Tikz.
\end{abstract}
\tableofcontents
\section{Introduction}
This package provides tools to generate plots and labeled axes easily. It draws normal plots, logplots and semi-logplots, in two and three dimensions. Axis ticks, labels, legends (in case of multiple plots) can be added with key-value options. It can cycle through a set of predefined line/marker/color specifications. In summary, its purpose is to simplify the generation of high-quality function and/or data plots, and solving the problems of
\begin{itemize}
	\item consistency of document and font type and font size,
	\item direct use of \TeX\ math mode in axis descriptions,
	\item consistency of data and figures (no third party tool necessary),
	\item inter document consistency using preamble configurations and styles.
\end{itemize}
Although not necessary, separate |.pdf| or |.eps| graphics can be generated using the |external| library developed as part of \Tikz.

\PGFPlots\ is build completely on \Tikz/\PGF. Knowledge of \Tikz\ will simplify the work with \PGFPlots, although it is not required.

\section[About PGFPlots: Preliminaries]{About {\normalfont\PGFPlots}: Preliminaries}
This section contains information about upgrades, the team, the installation (in case you need to do it manually) and troubleshooting. You may skip it completely except for the upgrade remarks.

However, note that this library requires at least \PGF\ version $2.00$. At the time of this writing, many \TeX-distributions still contain the older \PGF\ version $1.18$, so it may be necessary to install a recent \PGF\ prior to using \PGFPlots.

\subsection{Components}
\PGFPlots\ comes with two components:
\begin{enumerate}
	\item the plotting component (which you are currently reading) and
	\item the \PGFPlotstable\ component which simplifies number formatting and postprocessing of numerical tables. It comes as a separate package and has its own manual \href{file:pgfplotstable.pdf}{pgfplotstable.pdf}.
\end{enumerate}

\subsection{Upgrade remarks}
This release provides a lot of improvements which can be found in all detail in \texttt{ChangeLog} for interested readers. However, some attention is useful with respect to the following changes.

\subsubsection{New Optional Features}
\begin{enumerate}
	\item \PGFPlots\ 1.3 comes with user interface improvements. The technical distinction between ``behavior options'' and ``style options'' of older versions is no longer necessary (although still fully supported).

	\item \PGFPlots\ 1.3 has a new feature which allows to \emph{move axis labels tight to tick labels} automatically.

	Since this affects the spacing, it is not enabled be default.

	Use
\begin{codeexample}[code only]
\usepackage{pgfplots}
\pgfplotsset{compat=1.3}
\end{codeexample}
	in your preamble to benefit from the improved spacing\footnote{The |compat=1.3| setting is necessary for new features which move axis labels around like reversed axis. However, \PGFPlots\ will still work as it does in earlier versions, your documents will remain the same if you don't set it explicitly.}.
	Take a look at the next page for the precise definition of |compat|.

	\item \PGFPlots\ 1.3 now supports reversed axes. It is no longer necessary to use work--arounds with negative units.
\pgfkeys{/pdflinks/search key prefixes in/.add={/pgfplots/,}{}}

	Take a look at the |x dir=reverse| key.

	Existing work--arounds will still function properly. Use |\pgfplotsset{compat=1.3}| together with |x dir=reverse| to switch to the new version.
\end{enumerate}

\subsubsection{Old Features Which May Need Attention}
\begin{enumerate}
	\item The |scatter/classes| feature produces proper legends as of version 1.3. This may change the appearance of existing legends of plots with |scatter/classes|.

	\item Due to a small math bug in \PGF\ $2.00$, you \emph{can't} use the math expression `|-x^2|'. It is necessary to use `|0-x^2|' instead. The same holds for
		`|exp(-x^2)|'; use `|exp(0-x^2)|' instead.

		This will be fixed with \PGF\ $>2.00$.

	\item Starting with \PGFPlots\ $1.1$, |\tikzstyle| should \emph{no longer be used} to set \PGFPlots\ options.
	
	Although |\tikzstyle| is still supported for some older \PGFPlots\ options, you should replace any occurance of |\tikzstyle| with |\pgfplotsset{|\meta{style name}|/.style={|\meta{key-value-list}|}}| or the associated |/.append style| variant. See section~\ref{sec:styles} for more detail.
\end{enumerate}
I apologize for any inconvenience caused by these changes.

\begin{pgfplotskey}{compat=\mchoice{1.3,pre 1.3,default,newest} (initially default)}
	The preamble configuration 
\begin{codeexample}[code only]
\usepackage{pgfplots}
\pgfplotsset{compat=1.3}
\end{codeexample}
	allows to choose between backwards compatibility and most recent features.

	\PGFPlots\ version 1.3 comes with new features which may lead to different spacings compared to earlier versions:
	\begin{enumerate}
		\item Version 1.3 comes with the |xlabel near ticks| feature which places (in this case $x$) axis labels ``near'' the tick labels, taking the size of tick labels into account.
		
		Older versions used |xlabel absolute|, i.e.\ an absolute distance between the axis and axis labels (now matter how large or small tick labels are).

		The initial setting \emph{keeps backwards compatibility}.

		You are encouraged to use
\begin{codeexample}[code only]
\usepackage{pgfplots}
\pgfplotsset{compat=1.3}
\end{codeexample}
		in your preamble.
		
		\item Version 1.3 fixes a bug with |anchor=south west| which occurred mainly for reversed axes: the anchors where upside--down. This is fixed now.

		
		If you have written some work--around and want to keep it going, use

		|\pgfplotsset{compat/anchors=pre 1.3}|\pgfmanualpdflabel{/pgfplots/compat/anchors}{} 

		to disable the bug fix.
	\end{enumerate}

	Use |\pgfplotsset{compat=default}| to restore the factory settings.

	The setting |\pgfplotsset{compat=newest}| will always use features of the most recent version. This might result in small changes in the document's appearance.
\end{pgfplotskey}

\subsection{The Team}
\PGFPlots\ has been written mainly by Christian Feuers�nger with many improvements of Pascal Wolkotte and Nick Papior Andersen as a spare time project. We hope it is useful and provides valuable plots.

If you are interested in writing something but don't know how, consider reading the auxiliary manual \href{file:TeX-programming-notes.pdf}{TeX-programming-notes.pdf} which comes with \PGFPlots. It is far from complete, but maybe it is a good starting point (at least for more literature).

\subsection{Acknowledgements}
I thank God for all hours of enjoyed programming. I thank Pascal Wolkotte and Nick Papior Andersen for their programming efforts and contributions as part of the development team. I thank Stefan Tibus, who contributed the |plot shell| feature. I thank Tom Cashman for the contribution of the |reverse legend| feature. Special thanks go to Stefan Pinnow whose tests of \PGFPlots\ lead to numerous quality improvements. Furthermore, I thank Dr.~Schweitzer for many fruitful discussions and Dr.~Meine for his ideas and suggestions. Thanks as well to the many international contributors who provided feature requests or identified bugs or simply manual improvements!

Last but not least, I thank Till Tantau and Mark Wibrow for their excellent graphics (and more) package \PGF\ and \Tikz\ which is the base of \PGFPlots.

\include{pgfplots.install}%
\include{pgfplots.intro}%
\include{pgfplots.reference}%
\include{pgfplots.libs}%
\include{pgfplots.resources}%
\include{pgfplots.importexport}%
\include{pgfplots.basic.reference}%

\printindex

\bibliographystyle{abbrv} %gerapali} %gerabbrv} %gerunsrt.bst} %gerabbrv}% gerplain}
\nocite{pgfplotstable}
\nocite{programmingnotes}
\bibliography{pgfplots}
\end{document}
%
%%%%%%%%%%%%%%%%%%%%%%%%%%%%%%%%%%%%%%%%%%%%%%%%%%%%%%%%%%%%%%%%%%%%%%%%%%%%%
%
% Package pgfplots.sty documentation. 
%
% Copyright 2007/2008 by Christian Feuersaenger.
%
% This program is free software: you can redistribute it and/or modify
% it under the terms of the GNU General Public License as published by
% the Free Software Foundation, either version 3 of the License, or
% (at your option) any later version.
% 
% This program is distributed in the hope that it will be useful,
% but WITHOUT ANY WARRANTY; without even the implied warranty of
% MERCHANTABILITY or FITNESS FOR A PARTICULAR PURPOSE.  See the
% GNU General Public License for more details.
% 
% You should have received a copy of the GNU General Public License
% along with this program.  If not, see <http://www.gnu.org/licenses/>.
%
%
%%%%%%%%%%%%%%%%%%%%%%%%%%%%%%%%%%%%%%%%%%%%%%%%%%%%%%%%%%%%%%%%%%%%%%%%%%%%%
\input{pgfplots.preamble.tex}
%\RequirePackage[german,english,francais]{babel}

\pgfplotsmanualexternalexpensivetrue

%\usetikzlibrary{external}
%\tikzexternalize[prefix=figures/]{pgfplots}

\title{%
	Manual for Package \PGFPlots\\
	{\small Version \pgfplotsversion}\\
	{\small\href{http://sourceforge.net/projects/pgfplots}{http://sourceforge.net/projects/pgfplots}}}

%\includeonly{pgfplots.libs}


\begin{document}

\def\plotcoords{%
\addplot coordinates {
(5,8.312e-02)    (17,2.547e-02)   (49,7.407e-03)
(129,2.102e-03)  (321,5.874e-04)  (769,1.623e-04)
(1793,4.442e-05) (4097,1.207e-05) (9217,3.261e-06)
};

\addplot coordinates{
(7,8.472e-02)    (31,3.044e-02)    (111,1.022e-02)
(351,3.303e-03)  (1023,1.039e-03)  (2815,3.196e-04)
(7423,9.658e-05) (18943,2.873e-05) (47103,8.437e-06)
};

\addplot coordinates{
(9,7.881e-02)     (49,3.243e-02)    (209,1.232e-02)
(769,4.454e-03)   (2561,1.551e-03)  (7937,5.236e-04)
(23297,1.723e-04) (65537,5.545e-05) (178177,1.751e-05)
};

\addplot coordinates{
(11,6.887e-02)    (71,3.177e-02)     (351,1.341e-02)
(1471,5.334e-03)  (5503,2.027e-03)   (18943,7.415e-04)
(61183,2.628e-04) (187903,9.063e-05) (553983,3.053e-05)
};

\addplot coordinates{
(13,5.755e-02)     (97,2.925e-02)     (545,1.351e-02)
(2561,5.842e-03)   (10625,2.397e-03)  (40193,9.414e-04)
(141569,3.564e-04) (471041,1.308e-04) 
(1496065,4.670e-05)
};
}%
\maketitle
\begin{abstract}%
\PGFPlots\ draws high--quality function plots in normal or logarithmic scaling with a user-friendly interface directly in \TeX. The user supplies axis labels, legend entries and the plot coordinates for one or more plots and \PGFPlots\ applies axis scaling, computes any logarithms and axis ticks and draws the plots, supporting line plots, scatter plots, piecewise constant plots, bar plots, area plots, mesh-- and surface plots and some more. It is based on Till Tantau's package \PGF/\Tikz.
\end{abstract}
\tableofcontents
\section{Introduction}
This package provides tools to generate plots and labeled axes easily. It draws normal plots, logplots and semi-logplots, in two and three dimensions. Axis ticks, labels, legends (in case of multiple plots) can be added with key-value options. It can cycle through a set of predefined line/marker/color specifications. In summary, its purpose is to simplify the generation of high-quality function and/or data plots, and solving the problems of
\begin{itemize}
	\item consistency of document and font type and font size,
	\item direct use of \TeX\ math mode in axis descriptions,
	\item consistency of data and figures (no third party tool necessary),
	\item inter document consistency using preamble configurations and styles.
\end{itemize}
Although not necessary, separate |.pdf| or |.eps| graphics can be generated using the |external| library developed as part of \Tikz.

\PGFPlots\ is build completely on \Tikz/\PGF. Knowledge of \Tikz\ will simplify the work with \PGFPlots, although it is not required.

\section[About PGFPlots: Preliminaries]{About {\normalfont\PGFPlots}: Preliminaries}
This section contains information about upgrades, the team, the installation (in case you need to do it manually) and troubleshooting. You may skip it completely except for the upgrade remarks.

However, note that this library requires at least \PGF\ version $2.00$. At the time of this writing, many \TeX-distributions still contain the older \PGF\ version $1.18$, so it may be necessary to install a recent \PGF\ prior to using \PGFPlots.

\subsection{Components}
\PGFPlots\ comes with two components:
\begin{enumerate}
	\item the plotting component (which you are currently reading) and
	\item the \PGFPlotstable\ component which simplifies number formatting and postprocessing of numerical tables. It comes as a separate package and has its own manual \href{file:pgfplotstable.pdf}{pgfplotstable.pdf}.
\end{enumerate}

\subsection{Upgrade remarks}
This release provides a lot of improvements which can be found in all detail in \texttt{ChangeLog} for interested readers. However, some attention is useful with respect to the following changes.

\subsubsection{New Optional Features}
\begin{enumerate}
	\item \PGFPlots\ 1.3 comes with user interface improvements. The technical distinction between ``behavior options'' and ``style options'' of older versions is no longer necessary (although still fully supported).

	\item \PGFPlots\ 1.3 has a new feature which allows to \emph{move axis labels tight to tick labels} automatically.

	Since this affects the spacing, it is not enabled be default.

	Use
\begin{codeexample}[code only]
\usepackage{pgfplots}
\pgfplotsset{compat=1.3}
\end{codeexample}
	in your preamble to benefit from the improved spacing\footnote{The |compat=1.3| setting is necessary for new features which move axis labels around like reversed axis. However, \PGFPlots\ will still work as it does in earlier versions, your documents will remain the same if you don't set it explicitly.}.
	Take a look at the next page for the precise definition of |compat|.

	\item \PGFPlots\ 1.3 now supports reversed axes. It is no longer necessary to use work--arounds with negative units.
\pgfkeys{/pdflinks/search key prefixes in/.add={/pgfplots/,}{}}

	Take a look at the |x dir=reverse| key.

	Existing work--arounds will still function properly. Use |\pgfplotsset{compat=1.3}| together with |x dir=reverse| to switch to the new version.
\end{enumerate}

\subsubsection{Old Features Which May Need Attention}
\begin{enumerate}
	\item The |scatter/classes| feature produces proper legends as of version 1.3. This may change the appearance of existing legends of plots with |scatter/classes|.

	\item Due to a small math bug in \PGF\ $2.00$, you \emph{can't} use the math expression `|-x^2|'. It is necessary to use `|0-x^2|' instead. The same holds for
		`|exp(-x^2)|'; use `|exp(0-x^2)|' instead.

		This will be fixed with \PGF\ $>2.00$.

	\item Starting with \PGFPlots\ $1.1$, |\tikzstyle| should \emph{no longer be used} to set \PGFPlots\ options.
	
	Although |\tikzstyle| is still supported for some older \PGFPlots\ options, you should replace any occurance of |\tikzstyle| with |\pgfplotsset{|\meta{style name}|/.style={|\meta{key-value-list}|}}| or the associated |/.append style| variant. See section~\ref{sec:styles} for more detail.
\end{enumerate}
I apologize for any inconvenience caused by these changes.

\begin{pgfplotskey}{compat=\mchoice{1.3,pre 1.3,default,newest} (initially default)}
	The preamble configuration 
\begin{codeexample}[code only]
\usepackage{pgfplots}
\pgfplotsset{compat=1.3}
\end{codeexample}
	allows to choose between backwards compatibility and most recent features.

	\PGFPlots\ version 1.3 comes with new features which may lead to different spacings compared to earlier versions:
	\begin{enumerate}
		\item Version 1.3 comes with the |xlabel near ticks| feature which places (in this case $x$) axis labels ``near'' the tick labels, taking the size of tick labels into account.
		
		Older versions used |xlabel absolute|, i.e.\ an absolute distance between the axis and axis labels (now matter how large or small tick labels are).

		The initial setting \emph{keeps backwards compatibility}.

		You are encouraged to use
\begin{codeexample}[code only]
\usepackage{pgfplots}
\pgfplotsset{compat=1.3}
\end{codeexample}
		in your preamble.
		
		\item Version 1.3 fixes a bug with |anchor=south west| which occurred mainly for reversed axes: the anchors where upside--down. This is fixed now.

		
		If you have written some work--around and want to keep it going, use

		|\pgfplotsset{compat/anchors=pre 1.3}|\pgfmanualpdflabel{/pgfplots/compat/anchors}{} 

		to disable the bug fix.
	\end{enumerate}

	Use |\pgfplotsset{compat=default}| to restore the factory settings.

	The setting |\pgfplotsset{compat=newest}| will always use features of the most recent version. This might result in small changes in the document's appearance.
\end{pgfplotskey}

\subsection{The Team}
\PGFPlots\ has been written mainly by Christian Feuers�nger with many improvements of Pascal Wolkotte and Nick Papior Andersen as a spare time project. We hope it is useful and provides valuable plots.

If you are interested in writing something but don't know how, consider reading the auxiliary manual \href{file:TeX-programming-notes.pdf}{TeX-programming-notes.pdf} which comes with \PGFPlots. It is far from complete, but maybe it is a good starting point (at least for more literature).

\subsection{Acknowledgements}
I thank God for all hours of enjoyed programming. I thank Pascal Wolkotte and Nick Papior Andersen for their programming efforts and contributions as part of the development team. I thank Stefan Tibus, who contributed the |plot shell| feature. I thank Tom Cashman for the contribution of the |reverse legend| feature. Special thanks go to Stefan Pinnow whose tests of \PGFPlots\ lead to numerous quality improvements. Furthermore, I thank Dr.~Schweitzer for many fruitful discussions and Dr.~Meine for his ideas and suggestions. Thanks as well to the many international contributors who provided feature requests or identified bugs or simply manual improvements!

Last but not least, I thank Till Tantau and Mark Wibrow for their excellent graphics (and more) package \PGF\ and \Tikz\ which is the base of \PGFPlots.

\include{pgfplots.install}%
\include{pgfplots.intro}%
\include{pgfplots.reference}%
\include{pgfplots.libs}%
\include{pgfplots.resources}%
\include{pgfplots.importexport}%
\include{pgfplots.basic.reference}%

\printindex

\bibliographystyle{abbrv} %gerapali} %gerabbrv} %gerunsrt.bst} %gerabbrv}% gerplain}
\nocite{pgfplotstable}
\nocite{programmingnotes}
\bibliography{pgfplots}
\end{document}
%
%%%%%%%%%%%%%%%%%%%%%%%%%%%%%%%%%%%%%%%%%%%%%%%%%%%%%%%%%%%%%%%%%%%%%%%%%%%%%
%
% Package pgfplots.sty documentation. 
%
% Copyright 2007/2008 by Christian Feuersaenger.
%
% This program is free software: you can redistribute it and/or modify
% it under the terms of the GNU General Public License as published by
% the Free Software Foundation, either version 3 of the License, or
% (at your option) any later version.
% 
% This program is distributed in the hope that it will be useful,
% but WITHOUT ANY WARRANTY; without even the implied warranty of
% MERCHANTABILITY or FITNESS FOR A PARTICULAR PURPOSE.  See the
% GNU General Public License for more details.
% 
% You should have received a copy of the GNU General Public License
% along with this program.  If not, see <http://www.gnu.org/licenses/>.
%
%
%%%%%%%%%%%%%%%%%%%%%%%%%%%%%%%%%%%%%%%%%%%%%%%%%%%%%%%%%%%%%%%%%%%%%%%%%%%%%
\input{pgfplots.preamble.tex}
%\RequirePackage[german,english,francais]{babel}

\pgfplotsmanualexternalexpensivetrue

%\usetikzlibrary{external}
%\tikzexternalize[prefix=figures/]{pgfplots}

\title{%
	Manual for Package \PGFPlots\\
	{\small Version \pgfplotsversion}\\
	{\small\href{http://sourceforge.net/projects/pgfplots}{http://sourceforge.net/projects/pgfplots}}}

%\includeonly{pgfplots.libs}


\begin{document}

\def\plotcoords{%
\addplot coordinates {
(5,8.312e-02)    (17,2.547e-02)   (49,7.407e-03)
(129,2.102e-03)  (321,5.874e-04)  (769,1.623e-04)
(1793,4.442e-05) (4097,1.207e-05) (9217,3.261e-06)
};

\addplot coordinates{
(7,8.472e-02)    (31,3.044e-02)    (111,1.022e-02)
(351,3.303e-03)  (1023,1.039e-03)  (2815,3.196e-04)
(7423,9.658e-05) (18943,2.873e-05) (47103,8.437e-06)
};

\addplot coordinates{
(9,7.881e-02)     (49,3.243e-02)    (209,1.232e-02)
(769,4.454e-03)   (2561,1.551e-03)  (7937,5.236e-04)
(23297,1.723e-04) (65537,5.545e-05) (178177,1.751e-05)
};

\addplot coordinates{
(11,6.887e-02)    (71,3.177e-02)     (351,1.341e-02)
(1471,5.334e-03)  (5503,2.027e-03)   (18943,7.415e-04)
(61183,2.628e-04) (187903,9.063e-05) (553983,3.053e-05)
};

\addplot coordinates{
(13,5.755e-02)     (97,2.925e-02)     (545,1.351e-02)
(2561,5.842e-03)   (10625,2.397e-03)  (40193,9.414e-04)
(141569,3.564e-04) (471041,1.308e-04) 
(1496065,4.670e-05)
};
}%
\maketitle
\begin{abstract}%
\PGFPlots\ draws high--quality function plots in normal or logarithmic scaling with a user-friendly interface directly in \TeX. The user supplies axis labels, legend entries and the plot coordinates for one or more plots and \PGFPlots\ applies axis scaling, computes any logarithms and axis ticks and draws the plots, supporting line plots, scatter plots, piecewise constant plots, bar plots, area plots, mesh-- and surface plots and some more. It is based on Till Tantau's package \PGF/\Tikz.
\end{abstract}
\tableofcontents
\section{Introduction}
This package provides tools to generate plots and labeled axes easily. It draws normal plots, logplots and semi-logplots, in two and three dimensions. Axis ticks, labels, legends (in case of multiple plots) can be added with key-value options. It can cycle through a set of predefined line/marker/color specifications. In summary, its purpose is to simplify the generation of high-quality function and/or data plots, and solving the problems of
\begin{itemize}
	\item consistency of document and font type and font size,
	\item direct use of \TeX\ math mode in axis descriptions,
	\item consistency of data and figures (no third party tool necessary),
	\item inter document consistency using preamble configurations and styles.
\end{itemize}
Although not necessary, separate |.pdf| or |.eps| graphics can be generated using the |external| library developed as part of \Tikz.

\PGFPlots\ is build completely on \Tikz/\PGF. Knowledge of \Tikz\ will simplify the work with \PGFPlots, although it is not required.

\section[About PGFPlots: Preliminaries]{About {\normalfont\PGFPlots}: Preliminaries}
This section contains information about upgrades, the team, the installation (in case you need to do it manually) and troubleshooting. You may skip it completely except for the upgrade remarks.

However, note that this library requires at least \PGF\ version $2.00$. At the time of this writing, many \TeX-distributions still contain the older \PGF\ version $1.18$, so it may be necessary to install a recent \PGF\ prior to using \PGFPlots.

\subsection{Components}
\PGFPlots\ comes with two components:
\begin{enumerate}
	\item the plotting component (which you are currently reading) and
	\item the \PGFPlotstable\ component which simplifies number formatting and postprocessing of numerical tables. It comes as a separate package and has its own manual \href{file:pgfplotstable.pdf}{pgfplotstable.pdf}.
\end{enumerate}

\subsection{Upgrade remarks}
This release provides a lot of improvements which can be found in all detail in \texttt{ChangeLog} for interested readers. However, some attention is useful with respect to the following changes.

\subsubsection{New Optional Features}
\begin{enumerate}
	\item \PGFPlots\ 1.3 comes with user interface improvements. The technical distinction between ``behavior options'' and ``style options'' of older versions is no longer necessary (although still fully supported).

	\item \PGFPlots\ 1.3 has a new feature which allows to \emph{move axis labels tight to tick labels} automatically.

	Since this affects the spacing, it is not enabled be default.

	Use
\begin{codeexample}[code only]
\usepackage{pgfplots}
\pgfplotsset{compat=1.3}
\end{codeexample}
	in your preamble to benefit from the improved spacing\footnote{The |compat=1.3| setting is necessary for new features which move axis labels around like reversed axis. However, \PGFPlots\ will still work as it does in earlier versions, your documents will remain the same if you don't set it explicitly.}.
	Take a look at the next page for the precise definition of |compat|.

	\item \PGFPlots\ 1.3 now supports reversed axes. It is no longer necessary to use work--arounds with negative units.
\pgfkeys{/pdflinks/search key prefixes in/.add={/pgfplots/,}{}}

	Take a look at the |x dir=reverse| key.

	Existing work--arounds will still function properly. Use |\pgfplotsset{compat=1.3}| together with |x dir=reverse| to switch to the new version.
\end{enumerate}

\subsubsection{Old Features Which May Need Attention}
\begin{enumerate}
	\item The |scatter/classes| feature produces proper legends as of version 1.3. This may change the appearance of existing legends of plots with |scatter/classes|.

	\item Due to a small math bug in \PGF\ $2.00$, you \emph{can't} use the math expression `|-x^2|'. It is necessary to use `|0-x^2|' instead. The same holds for
		`|exp(-x^2)|'; use `|exp(0-x^2)|' instead.

		This will be fixed with \PGF\ $>2.00$.

	\item Starting with \PGFPlots\ $1.1$, |\tikzstyle| should \emph{no longer be used} to set \PGFPlots\ options.
	
	Although |\tikzstyle| is still supported for some older \PGFPlots\ options, you should replace any occurance of |\tikzstyle| with |\pgfplotsset{|\meta{style name}|/.style={|\meta{key-value-list}|}}| or the associated |/.append style| variant. See section~\ref{sec:styles} for more detail.
\end{enumerate}
I apologize for any inconvenience caused by these changes.

\begin{pgfplotskey}{compat=\mchoice{1.3,pre 1.3,default,newest} (initially default)}
	The preamble configuration 
\begin{codeexample}[code only]
\usepackage{pgfplots}
\pgfplotsset{compat=1.3}
\end{codeexample}
	allows to choose between backwards compatibility and most recent features.

	\PGFPlots\ version 1.3 comes with new features which may lead to different spacings compared to earlier versions:
	\begin{enumerate}
		\item Version 1.3 comes with the |xlabel near ticks| feature which places (in this case $x$) axis labels ``near'' the tick labels, taking the size of tick labels into account.
		
		Older versions used |xlabel absolute|, i.e.\ an absolute distance between the axis and axis labels (now matter how large or small tick labels are).

		The initial setting \emph{keeps backwards compatibility}.

		You are encouraged to use
\begin{codeexample}[code only]
\usepackage{pgfplots}
\pgfplotsset{compat=1.3}
\end{codeexample}
		in your preamble.
		
		\item Version 1.3 fixes a bug with |anchor=south west| which occurred mainly for reversed axes: the anchors where upside--down. This is fixed now.

		
		If you have written some work--around and want to keep it going, use

		|\pgfplotsset{compat/anchors=pre 1.3}|\pgfmanualpdflabel{/pgfplots/compat/anchors}{} 

		to disable the bug fix.
	\end{enumerate}

	Use |\pgfplotsset{compat=default}| to restore the factory settings.

	The setting |\pgfplotsset{compat=newest}| will always use features of the most recent version. This might result in small changes in the document's appearance.
\end{pgfplotskey}

\subsection{The Team}
\PGFPlots\ has been written mainly by Christian Feuers�nger with many improvements of Pascal Wolkotte and Nick Papior Andersen as a spare time project. We hope it is useful and provides valuable plots.

If you are interested in writing something but don't know how, consider reading the auxiliary manual \href{file:TeX-programming-notes.pdf}{TeX-programming-notes.pdf} which comes with \PGFPlots. It is far from complete, but maybe it is a good starting point (at least for more literature).

\subsection{Acknowledgements}
I thank God for all hours of enjoyed programming. I thank Pascal Wolkotte and Nick Papior Andersen for their programming efforts and contributions as part of the development team. I thank Stefan Tibus, who contributed the |plot shell| feature. I thank Tom Cashman for the contribution of the |reverse legend| feature. Special thanks go to Stefan Pinnow whose tests of \PGFPlots\ lead to numerous quality improvements. Furthermore, I thank Dr.~Schweitzer for many fruitful discussions and Dr.~Meine for his ideas and suggestions. Thanks as well to the many international contributors who provided feature requests or identified bugs or simply manual improvements!

Last but not least, I thank Till Tantau and Mark Wibrow for their excellent graphics (and more) package \PGF\ and \Tikz\ which is the base of \PGFPlots.

\include{pgfplots.install}%
\include{pgfplots.intro}%
\include{pgfplots.reference}%
\include{pgfplots.libs}%
\include{pgfplots.resources}%
\include{pgfplots.importexport}%
\include{pgfplots.basic.reference}%

\printindex

\bibliographystyle{abbrv} %gerapali} %gerabbrv} %gerunsrt.bst} %gerabbrv}% gerplain}
\nocite{pgfplotstable}
\nocite{programmingnotes}
\bibliography{pgfplots}
\end{document}
%
%%%%%%%%%%%%%%%%%%%%%%%%%%%%%%%%%%%%%%%%%%%%%%%%%%%%%%%%%%%%%%%%%%%%%%%%%%%%%
%
% Package pgfplots.sty documentation. 
%
% Copyright 2007/2008 by Christian Feuersaenger.
%
% This program is free software: you can redistribute it and/or modify
% it under the terms of the GNU General Public License as published by
% the Free Software Foundation, either version 3 of the License, or
% (at your option) any later version.
% 
% This program is distributed in the hope that it will be useful,
% but WITHOUT ANY WARRANTY; without even the implied warranty of
% MERCHANTABILITY or FITNESS FOR A PARTICULAR PURPOSE.  See the
% GNU General Public License for more details.
% 
% You should have received a copy of the GNU General Public License
% along with this program.  If not, see <http://www.gnu.org/licenses/>.
%
%
%%%%%%%%%%%%%%%%%%%%%%%%%%%%%%%%%%%%%%%%%%%%%%%%%%%%%%%%%%%%%%%%%%%%%%%%%%%%%
\input{pgfplots.preamble.tex}
%\RequirePackage[german,english,francais]{babel}

\pgfplotsmanualexternalexpensivetrue

%\usetikzlibrary{external}
%\tikzexternalize[prefix=figures/]{pgfplots}

\title{%
	Manual for Package \PGFPlots\\
	{\small Version \pgfplotsversion}\\
	{\small\href{http://sourceforge.net/projects/pgfplots}{http://sourceforge.net/projects/pgfplots}}}

%\includeonly{pgfplots.libs}


\begin{document}

\def\plotcoords{%
\addplot coordinates {
(5,8.312e-02)    (17,2.547e-02)   (49,7.407e-03)
(129,2.102e-03)  (321,5.874e-04)  (769,1.623e-04)
(1793,4.442e-05) (4097,1.207e-05) (9217,3.261e-06)
};

\addplot coordinates{
(7,8.472e-02)    (31,3.044e-02)    (111,1.022e-02)
(351,3.303e-03)  (1023,1.039e-03)  (2815,3.196e-04)
(7423,9.658e-05) (18943,2.873e-05) (47103,8.437e-06)
};

\addplot coordinates{
(9,7.881e-02)     (49,3.243e-02)    (209,1.232e-02)
(769,4.454e-03)   (2561,1.551e-03)  (7937,5.236e-04)
(23297,1.723e-04) (65537,5.545e-05) (178177,1.751e-05)
};

\addplot coordinates{
(11,6.887e-02)    (71,3.177e-02)     (351,1.341e-02)
(1471,5.334e-03)  (5503,2.027e-03)   (18943,7.415e-04)
(61183,2.628e-04) (187903,9.063e-05) (553983,3.053e-05)
};

\addplot coordinates{
(13,5.755e-02)     (97,2.925e-02)     (545,1.351e-02)
(2561,5.842e-03)   (10625,2.397e-03)  (40193,9.414e-04)
(141569,3.564e-04) (471041,1.308e-04) 
(1496065,4.670e-05)
};
}%
\maketitle
\begin{abstract}%
\PGFPlots\ draws high--quality function plots in normal or logarithmic scaling with a user-friendly interface directly in \TeX. The user supplies axis labels, legend entries and the plot coordinates for one or more plots and \PGFPlots\ applies axis scaling, computes any logarithms and axis ticks and draws the plots, supporting line plots, scatter plots, piecewise constant plots, bar plots, area plots, mesh-- and surface plots and some more. It is based on Till Tantau's package \PGF/\Tikz.
\end{abstract}
\tableofcontents
\section{Introduction}
This package provides tools to generate plots and labeled axes easily. It draws normal plots, logplots and semi-logplots, in two and three dimensions. Axis ticks, labels, legends (in case of multiple plots) can be added with key-value options. It can cycle through a set of predefined line/marker/color specifications. In summary, its purpose is to simplify the generation of high-quality function and/or data plots, and solving the problems of
\begin{itemize}
	\item consistency of document and font type and font size,
	\item direct use of \TeX\ math mode in axis descriptions,
	\item consistency of data and figures (no third party tool necessary),
	\item inter document consistency using preamble configurations and styles.
\end{itemize}
Although not necessary, separate |.pdf| or |.eps| graphics can be generated using the |external| library developed as part of \Tikz.

\PGFPlots\ is build completely on \Tikz/\PGF. Knowledge of \Tikz\ will simplify the work with \PGFPlots, although it is not required.

\section[About PGFPlots: Preliminaries]{About {\normalfont\PGFPlots}: Preliminaries}
This section contains information about upgrades, the team, the installation (in case you need to do it manually) and troubleshooting. You may skip it completely except for the upgrade remarks.

However, note that this library requires at least \PGF\ version $2.00$. At the time of this writing, many \TeX-distributions still contain the older \PGF\ version $1.18$, so it may be necessary to install a recent \PGF\ prior to using \PGFPlots.

\subsection{Components}
\PGFPlots\ comes with two components:
\begin{enumerate}
	\item the plotting component (which you are currently reading) and
	\item the \PGFPlotstable\ component which simplifies number formatting and postprocessing of numerical tables. It comes as a separate package and has its own manual \href{file:pgfplotstable.pdf}{pgfplotstable.pdf}.
\end{enumerate}

\subsection{Upgrade remarks}
This release provides a lot of improvements which can be found in all detail in \texttt{ChangeLog} for interested readers. However, some attention is useful with respect to the following changes.

\subsubsection{New Optional Features}
\begin{enumerate}
	\item \PGFPlots\ 1.3 comes with user interface improvements. The technical distinction between ``behavior options'' and ``style options'' of older versions is no longer necessary (although still fully supported).

	\item \PGFPlots\ 1.3 has a new feature which allows to \emph{move axis labels tight to tick labels} automatically.

	Since this affects the spacing, it is not enabled be default.

	Use
\begin{codeexample}[code only]
\usepackage{pgfplots}
\pgfplotsset{compat=1.3}
\end{codeexample}
	in your preamble to benefit from the improved spacing\footnote{The |compat=1.3| setting is necessary for new features which move axis labels around like reversed axis. However, \PGFPlots\ will still work as it does in earlier versions, your documents will remain the same if you don't set it explicitly.}.
	Take a look at the next page for the precise definition of |compat|.

	\item \PGFPlots\ 1.3 now supports reversed axes. It is no longer necessary to use work--arounds with negative units.
\pgfkeys{/pdflinks/search key prefixes in/.add={/pgfplots/,}{}}

	Take a look at the |x dir=reverse| key.

	Existing work--arounds will still function properly. Use |\pgfplotsset{compat=1.3}| together with |x dir=reverse| to switch to the new version.
\end{enumerate}

\subsubsection{Old Features Which May Need Attention}
\begin{enumerate}
	\item The |scatter/classes| feature produces proper legends as of version 1.3. This may change the appearance of existing legends of plots with |scatter/classes|.

	\item Due to a small math bug in \PGF\ $2.00$, you \emph{can't} use the math expression `|-x^2|'. It is necessary to use `|0-x^2|' instead. The same holds for
		`|exp(-x^2)|'; use `|exp(0-x^2)|' instead.

		This will be fixed with \PGF\ $>2.00$.

	\item Starting with \PGFPlots\ $1.1$, |\tikzstyle| should \emph{no longer be used} to set \PGFPlots\ options.
	
	Although |\tikzstyle| is still supported for some older \PGFPlots\ options, you should replace any occurance of |\tikzstyle| with |\pgfplotsset{|\meta{style name}|/.style={|\meta{key-value-list}|}}| or the associated |/.append style| variant. See section~\ref{sec:styles} for more detail.
\end{enumerate}
I apologize for any inconvenience caused by these changes.

\begin{pgfplotskey}{compat=\mchoice{1.3,pre 1.3,default,newest} (initially default)}
	The preamble configuration 
\begin{codeexample}[code only]
\usepackage{pgfplots}
\pgfplotsset{compat=1.3}
\end{codeexample}
	allows to choose between backwards compatibility and most recent features.

	\PGFPlots\ version 1.3 comes with new features which may lead to different spacings compared to earlier versions:
	\begin{enumerate}
		\item Version 1.3 comes with the |xlabel near ticks| feature which places (in this case $x$) axis labels ``near'' the tick labels, taking the size of tick labels into account.
		
		Older versions used |xlabel absolute|, i.e.\ an absolute distance between the axis and axis labels (now matter how large or small tick labels are).

		The initial setting \emph{keeps backwards compatibility}.

		You are encouraged to use
\begin{codeexample}[code only]
\usepackage{pgfplots}
\pgfplotsset{compat=1.3}
\end{codeexample}
		in your preamble.
		
		\item Version 1.3 fixes a bug with |anchor=south west| which occurred mainly for reversed axes: the anchors where upside--down. This is fixed now.

		
		If you have written some work--around and want to keep it going, use

		|\pgfplotsset{compat/anchors=pre 1.3}|\pgfmanualpdflabel{/pgfplots/compat/anchors}{} 

		to disable the bug fix.
	\end{enumerate}

	Use |\pgfplotsset{compat=default}| to restore the factory settings.

	The setting |\pgfplotsset{compat=newest}| will always use features of the most recent version. This might result in small changes in the document's appearance.
\end{pgfplotskey}

\subsection{The Team}
\PGFPlots\ has been written mainly by Christian Feuers�nger with many improvements of Pascal Wolkotte and Nick Papior Andersen as a spare time project. We hope it is useful and provides valuable plots.

If you are interested in writing something but don't know how, consider reading the auxiliary manual \href{file:TeX-programming-notes.pdf}{TeX-programming-notes.pdf} which comes with \PGFPlots. It is far from complete, but maybe it is a good starting point (at least for more literature).

\subsection{Acknowledgements}
I thank God for all hours of enjoyed programming. I thank Pascal Wolkotte and Nick Papior Andersen for their programming efforts and contributions as part of the development team. I thank Stefan Tibus, who contributed the |plot shell| feature. I thank Tom Cashman for the contribution of the |reverse legend| feature. Special thanks go to Stefan Pinnow whose tests of \PGFPlots\ lead to numerous quality improvements. Furthermore, I thank Dr.~Schweitzer for many fruitful discussions and Dr.~Meine for his ideas and suggestions. Thanks as well to the many international contributors who provided feature requests or identified bugs or simply manual improvements!

Last but not least, I thank Till Tantau and Mark Wibrow for their excellent graphics (and more) package \PGF\ and \Tikz\ which is the base of \PGFPlots.

\include{pgfplots.install}%
\include{pgfplots.intro}%
\include{pgfplots.reference}%
\include{pgfplots.libs}%
\include{pgfplots.resources}%
\include{pgfplots.importexport}%
\include{pgfplots.basic.reference}%

\printindex

\bibliographystyle{abbrv} %gerapali} %gerabbrv} %gerunsrt.bst} %gerabbrv}% gerplain}
\nocite{pgfplotstable}
\nocite{programmingnotes}
\bibliography{pgfplots}
\end{document}
%
%%%%%%%%%%%%%%%%%%%%%%%%%%%%%%%%%%%%%%%%%%%%%%%%%%%%%%%%%%%%%%%%%%%%%%%%%%%%%
%
% Package pgfplots.sty documentation. 
%
% Copyright 2007/2008 by Christian Feuersaenger.
%
% This program is free software: you can redistribute it and/or modify
% it under the terms of the GNU General Public License as published by
% the Free Software Foundation, either version 3 of the License, or
% (at your option) any later version.
% 
% This program is distributed in the hope that it will be useful,
% but WITHOUT ANY WARRANTY; without even the implied warranty of
% MERCHANTABILITY or FITNESS FOR A PARTICULAR PURPOSE.  See the
% GNU General Public License for more details.
% 
% You should have received a copy of the GNU General Public License
% along with this program.  If not, see <http://www.gnu.org/licenses/>.
%
%
%%%%%%%%%%%%%%%%%%%%%%%%%%%%%%%%%%%%%%%%%%%%%%%%%%%%%%%%%%%%%%%%%%%%%%%%%%%%%
\input{pgfplots.preamble.tex}
%\RequirePackage[german,english,francais]{babel}

\pgfplotsmanualexternalexpensivetrue

%\usetikzlibrary{external}
%\tikzexternalize[prefix=figures/]{pgfplots}

\title{%
	Manual for Package \PGFPlots\\
	{\small Version \pgfplotsversion}\\
	{\small\href{http://sourceforge.net/projects/pgfplots}{http://sourceforge.net/projects/pgfplots}}}

%\includeonly{pgfplots.libs}


\begin{document}

\def\plotcoords{%
\addplot coordinates {
(5,8.312e-02)    (17,2.547e-02)   (49,7.407e-03)
(129,2.102e-03)  (321,5.874e-04)  (769,1.623e-04)
(1793,4.442e-05) (4097,1.207e-05) (9217,3.261e-06)
};

\addplot coordinates{
(7,8.472e-02)    (31,3.044e-02)    (111,1.022e-02)
(351,3.303e-03)  (1023,1.039e-03)  (2815,3.196e-04)
(7423,9.658e-05) (18943,2.873e-05) (47103,8.437e-06)
};

\addplot coordinates{
(9,7.881e-02)     (49,3.243e-02)    (209,1.232e-02)
(769,4.454e-03)   (2561,1.551e-03)  (7937,5.236e-04)
(23297,1.723e-04) (65537,5.545e-05) (178177,1.751e-05)
};

\addplot coordinates{
(11,6.887e-02)    (71,3.177e-02)     (351,1.341e-02)
(1471,5.334e-03)  (5503,2.027e-03)   (18943,7.415e-04)
(61183,2.628e-04) (187903,9.063e-05) (553983,3.053e-05)
};

\addplot coordinates{
(13,5.755e-02)     (97,2.925e-02)     (545,1.351e-02)
(2561,5.842e-03)   (10625,2.397e-03)  (40193,9.414e-04)
(141569,3.564e-04) (471041,1.308e-04) 
(1496065,4.670e-05)
};
}%
\maketitle
\begin{abstract}%
\PGFPlots\ draws high--quality function plots in normal or logarithmic scaling with a user-friendly interface directly in \TeX. The user supplies axis labels, legend entries and the plot coordinates for one or more plots and \PGFPlots\ applies axis scaling, computes any logarithms and axis ticks and draws the plots, supporting line plots, scatter plots, piecewise constant plots, bar plots, area plots, mesh-- and surface plots and some more. It is based on Till Tantau's package \PGF/\Tikz.
\end{abstract}
\tableofcontents
\section{Introduction}
This package provides tools to generate plots and labeled axes easily. It draws normal plots, logplots and semi-logplots, in two and three dimensions. Axis ticks, labels, legends (in case of multiple plots) can be added with key-value options. It can cycle through a set of predefined line/marker/color specifications. In summary, its purpose is to simplify the generation of high-quality function and/or data plots, and solving the problems of
\begin{itemize}
	\item consistency of document and font type and font size,
	\item direct use of \TeX\ math mode in axis descriptions,
	\item consistency of data and figures (no third party tool necessary),
	\item inter document consistency using preamble configurations and styles.
\end{itemize}
Although not necessary, separate |.pdf| or |.eps| graphics can be generated using the |external| library developed as part of \Tikz.

\PGFPlots\ is build completely on \Tikz/\PGF. Knowledge of \Tikz\ will simplify the work with \PGFPlots, although it is not required.

\section[About PGFPlots: Preliminaries]{About {\normalfont\PGFPlots}: Preliminaries}
This section contains information about upgrades, the team, the installation (in case you need to do it manually) and troubleshooting. You may skip it completely except for the upgrade remarks.

However, note that this library requires at least \PGF\ version $2.00$. At the time of this writing, many \TeX-distributions still contain the older \PGF\ version $1.18$, so it may be necessary to install a recent \PGF\ prior to using \PGFPlots.

\subsection{Components}
\PGFPlots\ comes with two components:
\begin{enumerate}
	\item the plotting component (which you are currently reading) and
	\item the \PGFPlotstable\ component which simplifies number formatting and postprocessing of numerical tables. It comes as a separate package and has its own manual \href{file:pgfplotstable.pdf}{pgfplotstable.pdf}.
\end{enumerate}

\subsection{Upgrade remarks}
This release provides a lot of improvements which can be found in all detail in \texttt{ChangeLog} for interested readers. However, some attention is useful with respect to the following changes.

\subsubsection{New Optional Features}
\begin{enumerate}
	\item \PGFPlots\ 1.3 comes with user interface improvements. The technical distinction between ``behavior options'' and ``style options'' of older versions is no longer necessary (although still fully supported).

	\item \PGFPlots\ 1.3 has a new feature which allows to \emph{move axis labels tight to tick labels} automatically.

	Since this affects the spacing, it is not enabled be default.

	Use
\begin{codeexample}[code only]
\usepackage{pgfplots}
\pgfplotsset{compat=1.3}
\end{codeexample}
	in your preamble to benefit from the improved spacing\footnote{The |compat=1.3| setting is necessary for new features which move axis labels around like reversed axis. However, \PGFPlots\ will still work as it does in earlier versions, your documents will remain the same if you don't set it explicitly.}.
	Take a look at the next page for the precise definition of |compat|.

	\item \PGFPlots\ 1.3 now supports reversed axes. It is no longer necessary to use work--arounds with negative units.
\pgfkeys{/pdflinks/search key prefixes in/.add={/pgfplots/,}{}}

	Take a look at the |x dir=reverse| key.

	Existing work--arounds will still function properly. Use |\pgfplotsset{compat=1.3}| together with |x dir=reverse| to switch to the new version.
\end{enumerate}

\subsubsection{Old Features Which May Need Attention}
\begin{enumerate}
	\item The |scatter/classes| feature produces proper legends as of version 1.3. This may change the appearance of existing legends of plots with |scatter/classes|.

	\item Due to a small math bug in \PGF\ $2.00$, you \emph{can't} use the math expression `|-x^2|'. It is necessary to use `|0-x^2|' instead. The same holds for
		`|exp(-x^2)|'; use `|exp(0-x^2)|' instead.

		This will be fixed with \PGF\ $>2.00$.

	\item Starting with \PGFPlots\ $1.1$, |\tikzstyle| should \emph{no longer be used} to set \PGFPlots\ options.
	
	Although |\tikzstyle| is still supported for some older \PGFPlots\ options, you should replace any occurance of |\tikzstyle| with |\pgfplotsset{|\meta{style name}|/.style={|\meta{key-value-list}|}}| or the associated |/.append style| variant. See section~\ref{sec:styles} for more detail.
\end{enumerate}
I apologize for any inconvenience caused by these changes.

\begin{pgfplotskey}{compat=\mchoice{1.3,pre 1.3,default,newest} (initially default)}
	The preamble configuration 
\begin{codeexample}[code only]
\usepackage{pgfplots}
\pgfplotsset{compat=1.3}
\end{codeexample}
	allows to choose between backwards compatibility and most recent features.

	\PGFPlots\ version 1.3 comes with new features which may lead to different spacings compared to earlier versions:
	\begin{enumerate}
		\item Version 1.3 comes with the |xlabel near ticks| feature which places (in this case $x$) axis labels ``near'' the tick labels, taking the size of tick labels into account.
		
		Older versions used |xlabel absolute|, i.e.\ an absolute distance between the axis and axis labels (now matter how large or small tick labels are).

		The initial setting \emph{keeps backwards compatibility}.

		You are encouraged to use
\begin{codeexample}[code only]
\usepackage{pgfplots}
\pgfplotsset{compat=1.3}
\end{codeexample}
		in your preamble.
		
		\item Version 1.3 fixes a bug with |anchor=south west| which occurred mainly for reversed axes: the anchors where upside--down. This is fixed now.

		
		If you have written some work--around and want to keep it going, use

		|\pgfplotsset{compat/anchors=pre 1.3}|\pgfmanualpdflabel{/pgfplots/compat/anchors}{} 

		to disable the bug fix.
	\end{enumerate}

	Use |\pgfplotsset{compat=default}| to restore the factory settings.

	The setting |\pgfplotsset{compat=newest}| will always use features of the most recent version. This might result in small changes in the document's appearance.
\end{pgfplotskey}

\subsection{The Team}
\PGFPlots\ has been written mainly by Christian Feuers�nger with many improvements of Pascal Wolkotte and Nick Papior Andersen as a spare time project. We hope it is useful and provides valuable plots.

If you are interested in writing something but don't know how, consider reading the auxiliary manual \href{file:TeX-programming-notes.pdf}{TeX-programming-notes.pdf} which comes with \PGFPlots. It is far from complete, but maybe it is a good starting point (at least for more literature).

\subsection{Acknowledgements}
I thank God for all hours of enjoyed programming. I thank Pascal Wolkotte and Nick Papior Andersen for their programming efforts and contributions as part of the development team. I thank Stefan Tibus, who contributed the |plot shell| feature. I thank Tom Cashman for the contribution of the |reverse legend| feature. Special thanks go to Stefan Pinnow whose tests of \PGFPlots\ lead to numerous quality improvements. Furthermore, I thank Dr.~Schweitzer for many fruitful discussions and Dr.~Meine for his ideas and suggestions. Thanks as well to the many international contributors who provided feature requests or identified bugs or simply manual improvements!

Last but not least, I thank Till Tantau and Mark Wibrow for their excellent graphics (and more) package \PGF\ and \Tikz\ which is the base of \PGFPlots.

\include{pgfplots.install}%
\include{pgfplots.intro}%
\include{pgfplots.reference}%
\include{pgfplots.libs}%
\include{pgfplots.resources}%
\include{pgfplots.importexport}%
\include{pgfplots.basic.reference}%

\printindex

\bibliographystyle{abbrv} %gerapali} %gerabbrv} %gerunsrt.bst} %gerabbrv}% gerplain}
\nocite{pgfplotstable}
\nocite{programmingnotes}
\bibliography{pgfplots}
\end{document}
%
%%%%%%%%%%%%%%%%%%%%%%%%%%%%%%%%%%%%%%%%%%%%%%%%%%%%%%%%%%%%%%%%%%%%%%%%%%%%%
%
% Package pgfplots.sty documentation. 
%
% Copyright 2007/2008 by Christian Feuersaenger.
%
% This program is free software: you can redistribute it and/or modify
% it under the terms of the GNU General Public License as published by
% the Free Software Foundation, either version 3 of the License, or
% (at your option) any later version.
% 
% This program is distributed in the hope that it will be useful,
% but WITHOUT ANY WARRANTY; without even the implied warranty of
% MERCHANTABILITY or FITNESS FOR A PARTICULAR PURPOSE.  See the
% GNU General Public License for more details.
% 
% You should have received a copy of the GNU General Public License
% along with this program.  If not, see <http://www.gnu.org/licenses/>.
%
%
%%%%%%%%%%%%%%%%%%%%%%%%%%%%%%%%%%%%%%%%%%%%%%%%%%%%%%%%%%%%%%%%%%%%%%%%%%%%%
\input{pgfplots.preamble.tex}
%\RequirePackage[german,english,francais]{babel}

\pgfplotsmanualexternalexpensivetrue

%\usetikzlibrary{external}
%\tikzexternalize[prefix=figures/]{pgfplots}

\title{%
	Manual for Package \PGFPlots\\
	{\small Version \pgfplotsversion}\\
	{\small\href{http://sourceforge.net/projects/pgfplots}{http://sourceforge.net/projects/pgfplots}}}

%\includeonly{pgfplots.libs}


\begin{document}

\def\plotcoords{%
\addplot coordinates {
(5,8.312e-02)    (17,2.547e-02)   (49,7.407e-03)
(129,2.102e-03)  (321,5.874e-04)  (769,1.623e-04)
(1793,4.442e-05) (4097,1.207e-05) (9217,3.261e-06)
};

\addplot coordinates{
(7,8.472e-02)    (31,3.044e-02)    (111,1.022e-02)
(351,3.303e-03)  (1023,1.039e-03)  (2815,3.196e-04)
(7423,9.658e-05) (18943,2.873e-05) (47103,8.437e-06)
};

\addplot coordinates{
(9,7.881e-02)     (49,3.243e-02)    (209,1.232e-02)
(769,4.454e-03)   (2561,1.551e-03)  (7937,5.236e-04)
(23297,1.723e-04) (65537,5.545e-05) (178177,1.751e-05)
};

\addplot coordinates{
(11,6.887e-02)    (71,3.177e-02)     (351,1.341e-02)
(1471,5.334e-03)  (5503,2.027e-03)   (18943,7.415e-04)
(61183,2.628e-04) (187903,9.063e-05) (553983,3.053e-05)
};

\addplot coordinates{
(13,5.755e-02)     (97,2.925e-02)     (545,1.351e-02)
(2561,5.842e-03)   (10625,2.397e-03)  (40193,9.414e-04)
(141569,3.564e-04) (471041,1.308e-04) 
(1496065,4.670e-05)
};
}%
\maketitle
\begin{abstract}%
\PGFPlots\ draws high--quality function plots in normal or logarithmic scaling with a user-friendly interface directly in \TeX. The user supplies axis labels, legend entries and the plot coordinates for one or more plots and \PGFPlots\ applies axis scaling, computes any logarithms and axis ticks and draws the plots, supporting line plots, scatter plots, piecewise constant plots, bar plots, area plots, mesh-- and surface plots and some more. It is based on Till Tantau's package \PGF/\Tikz.
\end{abstract}
\tableofcontents
\section{Introduction}
This package provides tools to generate plots and labeled axes easily. It draws normal plots, logplots and semi-logplots, in two and three dimensions. Axis ticks, labels, legends (in case of multiple plots) can be added with key-value options. It can cycle through a set of predefined line/marker/color specifications. In summary, its purpose is to simplify the generation of high-quality function and/or data plots, and solving the problems of
\begin{itemize}
	\item consistency of document and font type and font size,
	\item direct use of \TeX\ math mode in axis descriptions,
	\item consistency of data and figures (no third party tool necessary),
	\item inter document consistency using preamble configurations and styles.
\end{itemize}
Although not necessary, separate |.pdf| or |.eps| graphics can be generated using the |external| library developed as part of \Tikz.

\PGFPlots\ is build completely on \Tikz/\PGF. Knowledge of \Tikz\ will simplify the work with \PGFPlots, although it is not required.

\section[About PGFPlots: Preliminaries]{About {\normalfont\PGFPlots}: Preliminaries}
This section contains information about upgrades, the team, the installation (in case you need to do it manually) and troubleshooting. You may skip it completely except for the upgrade remarks.

However, note that this library requires at least \PGF\ version $2.00$. At the time of this writing, many \TeX-distributions still contain the older \PGF\ version $1.18$, so it may be necessary to install a recent \PGF\ prior to using \PGFPlots.

\subsection{Components}
\PGFPlots\ comes with two components:
\begin{enumerate}
	\item the plotting component (which you are currently reading) and
	\item the \PGFPlotstable\ component which simplifies number formatting and postprocessing of numerical tables. It comes as a separate package and has its own manual \href{file:pgfplotstable.pdf}{pgfplotstable.pdf}.
\end{enumerate}

\subsection{Upgrade remarks}
This release provides a lot of improvements which can be found in all detail in \texttt{ChangeLog} for interested readers. However, some attention is useful with respect to the following changes.

\subsubsection{New Optional Features}
\begin{enumerate}
	\item \PGFPlots\ 1.3 comes with user interface improvements. The technical distinction between ``behavior options'' and ``style options'' of older versions is no longer necessary (although still fully supported).

	\item \PGFPlots\ 1.3 has a new feature which allows to \emph{move axis labels tight to tick labels} automatically.

	Since this affects the spacing, it is not enabled be default.

	Use
\begin{codeexample}[code only]
\usepackage{pgfplots}
\pgfplotsset{compat=1.3}
\end{codeexample}
	in your preamble to benefit from the improved spacing\footnote{The |compat=1.3| setting is necessary for new features which move axis labels around like reversed axis. However, \PGFPlots\ will still work as it does in earlier versions, your documents will remain the same if you don't set it explicitly.}.
	Take a look at the next page for the precise definition of |compat|.

	\item \PGFPlots\ 1.3 now supports reversed axes. It is no longer necessary to use work--arounds with negative units.
\pgfkeys{/pdflinks/search key prefixes in/.add={/pgfplots/,}{}}

	Take a look at the |x dir=reverse| key.

	Existing work--arounds will still function properly. Use |\pgfplotsset{compat=1.3}| together with |x dir=reverse| to switch to the new version.
\end{enumerate}

\subsubsection{Old Features Which May Need Attention}
\begin{enumerate}
	\item The |scatter/classes| feature produces proper legends as of version 1.3. This may change the appearance of existing legends of plots with |scatter/classes|.

	\item Due to a small math bug in \PGF\ $2.00$, you \emph{can't} use the math expression `|-x^2|'. It is necessary to use `|0-x^2|' instead. The same holds for
		`|exp(-x^2)|'; use `|exp(0-x^2)|' instead.

		This will be fixed with \PGF\ $>2.00$.

	\item Starting with \PGFPlots\ $1.1$, |\tikzstyle| should \emph{no longer be used} to set \PGFPlots\ options.
	
	Although |\tikzstyle| is still supported for some older \PGFPlots\ options, you should replace any occurance of |\tikzstyle| with |\pgfplotsset{|\meta{style name}|/.style={|\meta{key-value-list}|}}| or the associated |/.append style| variant. See section~\ref{sec:styles} for more detail.
\end{enumerate}
I apologize for any inconvenience caused by these changes.

\begin{pgfplotskey}{compat=\mchoice{1.3,pre 1.3,default,newest} (initially default)}
	The preamble configuration 
\begin{codeexample}[code only]
\usepackage{pgfplots}
\pgfplotsset{compat=1.3}
\end{codeexample}
	allows to choose between backwards compatibility and most recent features.

	\PGFPlots\ version 1.3 comes with new features which may lead to different spacings compared to earlier versions:
	\begin{enumerate}
		\item Version 1.3 comes with the |xlabel near ticks| feature which places (in this case $x$) axis labels ``near'' the tick labels, taking the size of tick labels into account.
		
		Older versions used |xlabel absolute|, i.e.\ an absolute distance between the axis and axis labels (now matter how large or small tick labels are).

		The initial setting \emph{keeps backwards compatibility}.

		You are encouraged to use
\begin{codeexample}[code only]
\usepackage{pgfplots}
\pgfplotsset{compat=1.3}
\end{codeexample}
		in your preamble.
		
		\item Version 1.3 fixes a bug with |anchor=south west| which occurred mainly for reversed axes: the anchors where upside--down. This is fixed now.

		
		If you have written some work--around and want to keep it going, use

		|\pgfplotsset{compat/anchors=pre 1.3}|\pgfmanualpdflabel{/pgfplots/compat/anchors}{} 

		to disable the bug fix.
	\end{enumerate}

	Use |\pgfplotsset{compat=default}| to restore the factory settings.

	The setting |\pgfplotsset{compat=newest}| will always use features of the most recent version. This might result in small changes in the document's appearance.
\end{pgfplotskey}

\subsection{The Team}
\PGFPlots\ has been written mainly by Christian Feuers�nger with many improvements of Pascal Wolkotte and Nick Papior Andersen as a spare time project. We hope it is useful and provides valuable plots.

If you are interested in writing something but don't know how, consider reading the auxiliary manual \href{file:TeX-programming-notes.pdf}{TeX-programming-notes.pdf} which comes with \PGFPlots. It is far from complete, but maybe it is a good starting point (at least for more literature).

\subsection{Acknowledgements}
I thank God for all hours of enjoyed programming. I thank Pascal Wolkotte and Nick Papior Andersen for their programming efforts and contributions as part of the development team. I thank Stefan Tibus, who contributed the |plot shell| feature. I thank Tom Cashman for the contribution of the |reverse legend| feature. Special thanks go to Stefan Pinnow whose tests of \PGFPlots\ lead to numerous quality improvements. Furthermore, I thank Dr.~Schweitzer for many fruitful discussions and Dr.~Meine for his ideas and suggestions. Thanks as well to the many international contributors who provided feature requests or identified bugs or simply manual improvements!

Last but not least, I thank Till Tantau and Mark Wibrow for their excellent graphics (and more) package \PGF\ and \Tikz\ which is the base of \PGFPlots.

\include{pgfplots.install}%
\include{pgfplots.intro}%
\include{pgfplots.reference}%
\include{pgfplots.libs}%
\include{pgfplots.resources}%
\include{pgfplots.importexport}%
\include{pgfplots.basic.reference}%

\printindex

\bibliographystyle{abbrv} %gerapali} %gerabbrv} %gerunsrt.bst} %gerabbrv}% gerplain}
\nocite{pgfplotstable}
\nocite{programmingnotes}
\bibliography{pgfplots}
\end{document}
%
\include{pgfplots.basic.reference}%

\printindex

\bibliographystyle{abbrv} %gerapali} %gerabbrv} %gerunsrt.bst} %gerabbrv}% gerplain}
\nocite{pgfplotstable}
\nocite{programmingnotes}
\bibliography{pgfplots}
\end{document}
%
%%%%%%%%%%%%%%%%%%%%%%%%%%%%%%%%%%%%%%%%%%%%%%%%%%%%%%%%%%%%%%%%%%%%%%%%%%%%%
%
% Package pgfplots.sty documentation. 
%
% Copyright 2007/2008 by Christian Feuersaenger.
%
% This program is free software: you can redistribute it and/or modify
% it under the terms of the GNU General Public License as published by
% the Free Software Foundation, either version 3 of the License, or
% (at your option) any later version.
% 
% This program is distributed in the hope that it will be useful,
% but WITHOUT ANY WARRANTY; without even the implied warranty of
% MERCHANTABILITY or FITNESS FOR A PARTICULAR PURPOSE.  See the
% GNU General Public License for more details.
% 
% You should have received a copy of the GNU General Public License
% along with this program.  If not, see <http://www.gnu.org/licenses/>.
%
%
%%%%%%%%%%%%%%%%%%%%%%%%%%%%%%%%%%%%%%%%%%%%%%%%%%%%%%%%%%%%%%%%%%%%%%%%%%%%%
%%%%%%%%%%%%%%%%%%%%%%%%%%%%%%%%%%%%%%%%%%%%%%%%%%%%%%%%%%%%%%%%%%%%%%%%%%%%%
%
% Package pgfplots.sty documentation. 
%
% Copyright 2007/2008 by Christian Feuersaenger.
%
% This program is free software: you can redistribute it and/or modify
% it under the terms of the GNU General Public License as published by
% the Free Software Foundation, either version 3 of the License, or
% (at your option) any later version.
% 
% This program is distributed in the hope that it will be useful,
% but WITHOUT ANY WARRANTY; without even the implied warranty of
% MERCHANTABILITY or FITNESS FOR A PARTICULAR PURPOSE.  See the
% GNU General Public License for more details.
% 
% You should have received a copy of the GNU General Public License
% along with this program.  If not, see <http://www.gnu.org/licenses/>.
%
%
%%%%%%%%%%%%%%%%%%%%%%%%%%%%%%%%%%%%%%%%%%%%%%%%%%%%%%%%%%%%%%%%%%%%%%%%%%%%%
\documentclass[a4paper]{ltxdoc}

\usepackage{makeidx}

% DON't let hyperref overload the format of index and glossary. 
% I want to do that on my own in the stylefiles for makeindex...
\makeatletter
\let\@old@wrindex=\@wrindex
\makeatother

\usepackage{ifpdf}
\ifpdf
	\usepackage[pdftex]{hyperref}
\else
	\usepackage[dvipdfm]{hyperref}
\fi
\hypersetup{pdfborder={0 0 0}}

\makeatletter
\let\@wrindex=\@old@wrindex
\makeatother

% Formatiere Seitennummern im Index:
\newcommand{\indexpageno}[1]{%
	{\bfseries\hyperpage{#1}}%
}


\newcommand{\R}{\mathbb{R}}
\newcommand{\N}{\mathbb{N}}
\newcommand{\Z}{\mathbb{Z}}

\long\def\COMMENTLOWLEVEL#1\ENDCOMMENT{}
\def\ENDCOMMENT{}

\usepackage{textcomp}
\usepackage{booktabs}

\usepackage{calc}
\usepackage[formats]{listings}
%\usepackage{courier} % don't use it - the '^' character can't be copy-pasted in courier

\usepackage{array}
\lstset{%
	basicstyle=\ttfamily,
	language=[LaTeX]tex, % Seems as if \lstset{language=tex} must be invoked BEFORE loading tikz!?
	tabsize=4,
	breaklines=true,
	breakindent=0pt
}

\ifpdf
	\pdfinfo {
		/Author	(Christian Feuersaenger)
	}

\else
	\def\pgfsysdriver{pgfsys-dvipdfm.def}
\fi
%\def\pgfsysdriver{pgfsys-pdftex.def}
\usepackage{pgfplots}

\ifpdf
	\usetikzlibrary{pgfplotsclickable}
\fi

%\usepackage{fp}
% ATTENTION:
% this requires pgf version NEWER than 2.00 :
%\usetikzlibrary{fixedpointarithmetic}

\usetikzlibrary{dateplot}

\usepackage[a4paper,left=2.25cm,right=2.25cm,top=2.5cm,bottom=2.5cm,nohead]{geometry}
\usepackage{amsmath,amssymb}
\usepackage{xxcolor}
\usepackage{pifont}
\usepackage[latin1]{inputenc}
\usepackage{amsmath}
\usepackage{eurosym}
\input{pgfplots-macros}

\pgfqkeys{/codeexample}{%
	every codeexample/.style={
		width=8cm,
		/pgfplots/every axis/.append style={legend style={fill=graphicbackground}}
	},
	tabsize=4,
}
\pgfplotsset{
	every axis/.append style={width=7cm},
	filter discard warning=false,
}

\usetikzlibrary{backgrounds,patterns}
% Global styles:
\tikzset{
  shape example/.style={
    color=black!30,
    draw,
    fill=yellow!30,
    line width=.5cm,
    inner xsep=2.5cm,
    inner ysep=0.5cm}
}

\newcommand{\FIXME}[1]{\textcolor{red}{(FIXME: #1)}}

% fuer endvironment 'sidewaysfigure' bspw
% \usepackage{rotating}

\newcommand\Tikz{Ti\textit kZ}
\newcommand\PGF{\textsc{pgf}}
\newcommand\PGFPlots{\textsc{pgfplots}}
\def\PGFPlotstable{\textsc{PgfplotsTable}}


\makeindex

% Fix overful hboxes automatically:
\tolerance=2000
\emergencystretch=10pt

\tikzset{prefix=gnuplot/pgfplots_} % prefix for 'plot function'

\author{%
	Christian Feuers\"anger\footnote{\url{http://wissrech.ins.uni-bonn.de/people/feuersaenger}}\\%
	Institut f\"ur Numerische Simulation\\
	Universit\"at Bonn, Germany}


%\RequirePackage[german,english,francais]{babel}

\pgfplotsmanualexternalexpensivetrue

%\usetikzlibrary{external}
%\tikzexternalize[prefix=figures/]{pgfplots}

\title{%
	Manual for Package \PGFPlots\\
	{\small Version \pgfplotsversion}\\
	{\small\href{http://sourceforge.net/projects/pgfplots}{http://sourceforge.net/projects/pgfplots}}}

%\includeonly{pgfplots.libs}


\begin{document}

\def\plotcoords{%
\addplot coordinates {
(5,8.312e-02)    (17,2.547e-02)   (49,7.407e-03)
(129,2.102e-03)  (321,5.874e-04)  (769,1.623e-04)
(1793,4.442e-05) (4097,1.207e-05) (9217,3.261e-06)
};

\addplot coordinates{
(7,8.472e-02)    (31,3.044e-02)    (111,1.022e-02)
(351,3.303e-03)  (1023,1.039e-03)  (2815,3.196e-04)
(7423,9.658e-05) (18943,2.873e-05) (47103,8.437e-06)
};

\addplot coordinates{
(9,7.881e-02)     (49,3.243e-02)    (209,1.232e-02)
(769,4.454e-03)   (2561,1.551e-03)  (7937,5.236e-04)
(23297,1.723e-04) (65537,5.545e-05) (178177,1.751e-05)
};

\addplot coordinates{
(11,6.887e-02)    (71,3.177e-02)     (351,1.341e-02)
(1471,5.334e-03)  (5503,2.027e-03)   (18943,7.415e-04)
(61183,2.628e-04) (187903,9.063e-05) (553983,3.053e-05)
};

\addplot coordinates{
(13,5.755e-02)     (97,2.925e-02)     (545,1.351e-02)
(2561,5.842e-03)   (10625,2.397e-03)  (40193,9.414e-04)
(141569,3.564e-04) (471041,1.308e-04) 
(1496065,4.670e-05)
};
}%
\maketitle
\begin{abstract}%
\PGFPlots\ draws high--quality function plots in normal or logarithmic scaling with a user-friendly interface directly in \TeX. The user supplies axis labels, legend entries and the plot coordinates for one or more plots and \PGFPlots\ applies axis scaling, computes any logarithms and axis ticks and draws the plots, supporting line plots, scatter plots, piecewise constant plots, bar plots, area plots, mesh-- and surface plots and some more. It is based on Till Tantau's package \PGF/\Tikz.
\end{abstract}
\tableofcontents
\section{Introduction}
This package provides tools to generate plots and labeled axes easily. It draws normal plots, logplots and semi-logplots, in two and three dimensions. Axis ticks, labels, legends (in case of multiple plots) can be added with key-value options. It can cycle through a set of predefined line/marker/color specifications. In summary, its purpose is to simplify the generation of high-quality function and/or data plots, and solving the problems of
\begin{itemize}
	\item consistency of document and font type and font size,
	\item direct use of \TeX\ math mode in axis descriptions,
	\item consistency of data and figures (no third party tool necessary),
	\item inter document consistency using preamble configurations and styles.
\end{itemize}
Although not necessary, separate |.pdf| or |.eps| graphics can be generated using the |external| library developed as part of \Tikz.

\PGFPlots\ is build completely on \Tikz/\PGF. Knowledge of \Tikz\ will simplify the work with \PGFPlots, although it is not required.

\section[About PGFPlots: Preliminaries]{About {\normalfont\PGFPlots}: Preliminaries}
This section contains information about upgrades, the team, the installation (in case you need to do it manually) and troubleshooting. You may skip it completely except for the upgrade remarks.

However, note that this library requires at least \PGF\ version $2.00$. At the time of this writing, many \TeX-distributions still contain the older \PGF\ version $1.18$, so it may be necessary to install a recent \PGF\ prior to using \PGFPlots.

\subsection{Components}
\PGFPlots\ comes with two components:
\begin{enumerate}
	\item the plotting component (which you are currently reading) and
	\item the \PGFPlotstable\ component which simplifies number formatting and postprocessing of numerical tables. It comes as a separate package and has its own manual \href{file:pgfplotstable.pdf}{pgfplotstable.pdf}.
\end{enumerate}

\subsection{Upgrade remarks}
This release provides a lot of improvements which can be found in all detail in \texttt{ChangeLog} for interested readers. However, some attention is useful with respect to the following changes.

\subsubsection{New Optional Features}
\begin{enumerate}
	\item \PGFPlots\ 1.3 comes with user interface improvements. The technical distinction between ``behavior options'' and ``style options'' of older versions is no longer necessary (although still fully supported).

	\item \PGFPlots\ 1.3 has a new feature which allows to \emph{move axis labels tight to tick labels} automatically.

	Since this affects the spacing, it is not enabled be default.

	Use
\begin{codeexample}[code only]
\usepackage{pgfplots}
\pgfplotsset{compat=1.3}
\end{codeexample}
	in your preamble to benefit from the improved spacing\footnote{The |compat=1.3| setting is necessary for new features which move axis labels around like reversed axis. However, \PGFPlots\ will still work as it does in earlier versions, your documents will remain the same if you don't set it explicitly.}.
	Take a look at the next page for the precise definition of |compat|.

	\item \PGFPlots\ 1.3 now supports reversed axes. It is no longer necessary to use work--arounds with negative units.
\pgfkeys{/pdflinks/search key prefixes in/.add={/pgfplots/,}{}}

	Take a look at the |x dir=reverse| key.

	Existing work--arounds will still function properly. Use |\pgfplotsset{compat=1.3}| together with |x dir=reverse| to switch to the new version.
\end{enumerate}

\subsubsection{Old Features Which May Need Attention}
\begin{enumerate}
	\item The |scatter/classes| feature produces proper legends as of version 1.3. This may change the appearance of existing legends of plots with |scatter/classes|.

	\item Due to a small math bug in \PGF\ $2.00$, you \emph{can't} use the math expression `|-x^2|'. It is necessary to use `|0-x^2|' instead. The same holds for
		`|exp(-x^2)|'; use `|exp(0-x^2)|' instead.

		This will be fixed with \PGF\ $>2.00$.

	\item Starting with \PGFPlots\ $1.1$, |\tikzstyle| should \emph{no longer be used} to set \PGFPlots\ options.
	
	Although |\tikzstyle| is still supported for some older \PGFPlots\ options, you should replace any occurance of |\tikzstyle| with |\pgfplotsset{|\meta{style name}|/.style={|\meta{key-value-list}|}}| or the associated |/.append style| variant. See section~\ref{sec:styles} for more detail.
\end{enumerate}
I apologize for any inconvenience caused by these changes.

\begin{pgfplotskey}{compat=\mchoice{1.3,pre 1.3,default,newest} (initially default)}
	The preamble configuration 
\begin{codeexample}[code only]
\usepackage{pgfplots}
\pgfplotsset{compat=1.3}
\end{codeexample}
	allows to choose between backwards compatibility and most recent features.

	\PGFPlots\ version 1.3 comes with new features which may lead to different spacings compared to earlier versions:
	\begin{enumerate}
		\item Version 1.3 comes with the |xlabel near ticks| feature which places (in this case $x$) axis labels ``near'' the tick labels, taking the size of tick labels into account.
		
		Older versions used |xlabel absolute|, i.e.\ an absolute distance between the axis and axis labels (now matter how large or small tick labels are).

		The initial setting \emph{keeps backwards compatibility}.

		You are encouraged to use
\begin{codeexample}[code only]
\usepackage{pgfplots}
\pgfplotsset{compat=1.3}
\end{codeexample}
		in your preamble.
		
		\item Version 1.3 fixes a bug with |anchor=south west| which occurred mainly for reversed axes: the anchors where upside--down. This is fixed now.

		
		If you have written some work--around and want to keep it going, use

		|\pgfplotsset{compat/anchors=pre 1.3}|\pgfmanualpdflabel{/pgfplots/compat/anchors}{} 

		to disable the bug fix.
	\end{enumerate}

	Use |\pgfplotsset{compat=default}| to restore the factory settings.

	The setting |\pgfplotsset{compat=newest}| will always use features of the most recent version. This might result in small changes in the document's appearance.
\end{pgfplotskey}

\subsection{The Team}
\PGFPlots\ has been written mainly by Christian Feuers�nger with many improvements of Pascal Wolkotte and Nick Papior Andersen as a spare time project. We hope it is useful and provides valuable plots.

If you are interested in writing something but don't know how, consider reading the auxiliary manual \href{file:TeX-programming-notes.pdf}{TeX-programming-notes.pdf} which comes with \PGFPlots. It is far from complete, but maybe it is a good starting point (at least for more literature).

\subsection{Acknowledgements}
I thank God for all hours of enjoyed programming. I thank Pascal Wolkotte and Nick Papior Andersen for their programming efforts and contributions as part of the development team. I thank Stefan Tibus, who contributed the |plot shell| feature. I thank Tom Cashman for the contribution of the |reverse legend| feature. Special thanks go to Stefan Pinnow whose tests of \PGFPlots\ lead to numerous quality improvements. Furthermore, I thank Dr.~Schweitzer for many fruitful discussions and Dr.~Meine for his ideas and suggestions. Thanks as well to the many international contributors who provided feature requests or identified bugs or simply manual improvements!

Last but not least, I thank Till Tantau and Mark Wibrow for their excellent graphics (and more) package \PGF\ and \Tikz\ which is the base of \PGFPlots.

%%%%%%%%%%%%%%%%%%%%%%%%%%%%%%%%%%%%%%%%%%%%%%%%%%%%%%%%%%%%%%%%%%%%%%%%%%%%%
%
% Package pgfplots.sty documentation. 
%
% Copyright 2007/2008 by Christian Feuersaenger.
%
% This program is free software: you can redistribute it and/or modify
% it under the terms of the GNU General Public License as published by
% the Free Software Foundation, either version 3 of the License, or
% (at your option) any later version.
% 
% This program is distributed in the hope that it will be useful,
% but WITHOUT ANY WARRANTY; without even the implied warranty of
% MERCHANTABILITY or FITNESS FOR A PARTICULAR PURPOSE.  See the
% GNU General Public License for more details.
% 
% You should have received a copy of the GNU General Public License
% along with this program.  If not, see <http://www.gnu.org/licenses/>.
%
%
%%%%%%%%%%%%%%%%%%%%%%%%%%%%%%%%%%%%%%%%%%%%%%%%%%%%%%%%%%%%%%%%%%%%%%%%%%%%%
\input{pgfplots.preamble.tex}
%\RequirePackage[german,english,francais]{babel}

\pgfplotsmanualexternalexpensivetrue

%\usetikzlibrary{external}
%\tikzexternalize[prefix=figures/]{pgfplots}

\title{%
	Manual for Package \PGFPlots\\
	{\small Version \pgfplotsversion}\\
	{\small\href{http://sourceforge.net/projects/pgfplots}{http://sourceforge.net/projects/pgfplots}}}

%\includeonly{pgfplots.libs}


\begin{document}

\def\plotcoords{%
\addplot coordinates {
(5,8.312e-02)    (17,2.547e-02)   (49,7.407e-03)
(129,2.102e-03)  (321,5.874e-04)  (769,1.623e-04)
(1793,4.442e-05) (4097,1.207e-05) (9217,3.261e-06)
};

\addplot coordinates{
(7,8.472e-02)    (31,3.044e-02)    (111,1.022e-02)
(351,3.303e-03)  (1023,1.039e-03)  (2815,3.196e-04)
(7423,9.658e-05) (18943,2.873e-05) (47103,8.437e-06)
};

\addplot coordinates{
(9,7.881e-02)     (49,3.243e-02)    (209,1.232e-02)
(769,4.454e-03)   (2561,1.551e-03)  (7937,5.236e-04)
(23297,1.723e-04) (65537,5.545e-05) (178177,1.751e-05)
};

\addplot coordinates{
(11,6.887e-02)    (71,3.177e-02)     (351,1.341e-02)
(1471,5.334e-03)  (5503,2.027e-03)   (18943,7.415e-04)
(61183,2.628e-04) (187903,9.063e-05) (553983,3.053e-05)
};

\addplot coordinates{
(13,5.755e-02)     (97,2.925e-02)     (545,1.351e-02)
(2561,5.842e-03)   (10625,2.397e-03)  (40193,9.414e-04)
(141569,3.564e-04) (471041,1.308e-04) 
(1496065,4.670e-05)
};
}%
\maketitle
\begin{abstract}%
\PGFPlots\ draws high--quality function plots in normal or logarithmic scaling with a user-friendly interface directly in \TeX. The user supplies axis labels, legend entries and the plot coordinates for one or more plots and \PGFPlots\ applies axis scaling, computes any logarithms and axis ticks and draws the plots, supporting line plots, scatter plots, piecewise constant plots, bar plots, area plots, mesh-- and surface plots and some more. It is based on Till Tantau's package \PGF/\Tikz.
\end{abstract}
\tableofcontents
\section{Introduction}
This package provides tools to generate plots and labeled axes easily. It draws normal plots, logplots and semi-logplots, in two and three dimensions. Axis ticks, labels, legends (in case of multiple plots) can be added with key-value options. It can cycle through a set of predefined line/marker/color specifications. In summary, its purpose is to simplify the generation of high-quality function and/or data plots, and solving the problems of
\begin{itemize}
	\item consistency of document and font type and font size,
	\item direct use of \TeX\ math mode in axis descriptions,
	\item consistency of data and figures (no third party tool necessary),
	\item inter document consistency using preamble configurations and styles.
\end{itemize}
Although not necessary, separate |.pdf| or |.eps| graphics can be generated using the |external| library developed as part of \Tikz.

\PGFPlots\ is build completely on \Tikz/\PGF. Knowledge of \Tikz\ will simplify the work with \PGFPlots, although it is not required.

\section[About PGFPlots: Preliminaries]{About {\normalfont\PGFPlots}: Preliminaries}
This section contains information about upgrades, the team, the installation (in case you need to do it manually) and troubleshooting. You may skip it completely except for the upgrade remarks.

However, note that this library requires at least \PGF\ version $2.00$. At the time of this writing, many \TeX-distributions still contain the older \PGF\ version $1.18$, so it may be necessary to install a recent \PGF\ prior to using \PGFPlots.

\subsection{Components}
\PGFPlots\ comes with two components:
\begin{enumerate}
	\item the plotting component (which you are currently reading) and
	\item the \PGFPlotstable\ component which simplifies number formatting and postprocessing of numerical tables. It comes as a separate package and has its own manual \href{file:pgfplotstable.pdf}{pgfplotstable.pdf}.
\end{enumerate}

\subsection{Upgrade remarks}
This release provides a lot of improvements which can be found in all detail in \texttt{ChangeLog} for interested readers. However, some attention is useful with respect to the following changes.

\subsubsection{New Optional Features}
\begin{enumerate}
	\item \PGFPlots\ 1.3 comes with user interface improvements. The technical distinction between ``behavior options'' and ``style options'' of older versions is no longer necessary (although still fully supported).

	\item \PGFPlots\ 1.3 has a new feature which allows to \emph{move axis labels tight to tick labels} automatically.

	Since this affects the spacing, it is not enabled be default.

	Use
\begin{codeexample}[code only]
\usepackage{pgfplots}
\pgfplotsset{compat=1.3}
\end{codeexample}
	in your preamble to benefit from the improved spacing\footnote{The |compat=1.3| setting is necessary for new features which move axis labels around like reversed axis. However, \PGFPlots\ will still work as it does in earlier versions, your documents will remain the same if you don't set it explicitly.}.
	Take a look at the next page for the precise definition of |compat|.

	\item \PGFPlots\ 1.3 now supports reversed axes. It is no longer necessary to use work--arounds with negative units.
\pgfkeys{/pdflinks/search key prefixes in/.add={/pgfplots/,}{}}

	Take a look at the |x dir=reverse| key.

	Existing work--arounds will still function properly. Use |\pgfplotsset{compat=1.3}| together with |x dir=reverse| to switch to the new version.
\end{enumerate}

\subsubsection{Old Features Which May Need Attention}
\begin{enumerate}
	\item The |scatter/classes| feature produces proper legends as of version 1.3. This may change the appearance of existing legends of plots with |scatter/classes|.

	\item Due to a small math bug in \PGF\ $2.00$, you \emph{can't} use the math expression `|-x^2|'. It is necessary to use `|0-x^2|' instead. The same holds for
		`|exp(-x^2)|'; use `|exp(0-x^2)|' instead.

		This will be fixed with \PGF\ $>2.00$.

	\item Starting with \PGFPlots\ $1.1$, |\tikzstyle| should \emph{no longer be used} to set \PGFPlots\ options.
	
	Although |\tikzstyle| is still supported for some older \PGFPlots\ options, you should replace any occurance of |\tikzstyle| with |\pgfplotsset{|\meta{style name}|/.style={|\meta{key-value-list}|}}| or the associated |/.append style| variant. See section~\ref{sec:styles} for more detail.
\end{enumerate}
I apologize for any inconvenience caused by these changes.

\begin{pgfplotskey}{compat=\mchoice{1.3,pre 1.3,default,newest} (initially default)}
	The preamble configuration 
\begin{codeexample}[code only]
\usepackage{pgfplots}
\pgfplotsset{compat=1.3}
\end{codeexample}
	allows to choose between backwards compatibility and most recent features.

	\PGFPlots\ version 1.3 comes with new features which may lead to different spacings compared to earlier versions:
	\begin{enumerate}
		\item Version 1.3 comes with the |xlabel near ticks| feature which places (in this case $x$) axis labels ``near'' the tick labels, taking the size of tick labels into account.
		
		Older versions used |xlabel absolute|, i.e.\ an absolute distance between the axis and axis labels (now matter how large or small tick labels are).

		The initial setting \emph{keeps backwards compatibility}.

		You are encouraged to use
\begin{codeexample}[code only]
\usepackage{pgfplots}
\pgfplotsset{compat=1.3}
\end{codeexample}
		in your preamble.
		
		\item Version 1.3 fixes a bug with |anchor=south west| which occurred mainly for reversed axes: the anchors where upside--down. This is fixed now.

		
		If you have written some work--around and want to keep it going, use

		|\pgfplotsset{compat/anchors=pre 1.3}|\pgfmanualpdflabel{/pgfplots/compat/anchors}{} 

		to disable the bug fix.
	\end{enumerate}

	Use |\pgfplotsset{compat=default}| to restore the factory settings.

	The setting |\pgfplotsset{compat=newest}| will always use features of the most recent version. This might result in small changes in the document's appearance.
\end{pgfplotskey}

\subsection{The Team}
\PGFPlots\ has been written mainly by Christian Feuers�nger with many improvements of Pascal Wolkotte and Nick Papior Andersen as a spare time project. We hope it is useful and provides valuable plots.

If you are interested in writing something but don't know how, consider reading the auxiliary manual \href{file:TeX-programming-notes.pdf}{TeX-programming-notes.pdf} which comes with \PGFPlots. It is far from complete, but maybe it is a good starting point (at least for more literature).

\subsection{Acknowledgements}
I thank God for all hours of enjoyed programming. I thank Pascal Wolkotte and Nick Papior Andersen for their programming efforts and contributions as part of the development team. I thank Stefan Tibus, who contributed the |plot shell| feature. I thank Tom Cashman for the contribution of the |reverse legend| feature. Special thanks go to Stefan Pinnow whose tests of \PGFPlots\ lead to numerous quality improvements. Furthermore, I thank Dr.~Schweitzer for many fruitful discussions and Dr.~Meine for his ideas and suggestions. Thanks as well to the many international contributors who provided feature requests or identified bugs or simply manual improvements!

Last but not least, I thank Till Tantau and Mark Wibrow for their excellent graphics (and more) package \PGF\ and \Tikz\ which is the base of \PGFPlots.

\include{pgfplots.install}%
\include{pgfplots.intro}%
\include{pgfplots.reference}%
\include{pgfplots.libs}%
\include{pgfplots.resources}%
\include{pgfplots.importexport}%
\include{pgfplots.basic.reference}%

\printindex

\bibliographystyle{abbrv} %gerapali} %gerabbrv} %gerunsrt.bst} %gerabbrv}% gerplain}
\nocite{pgfplotstable}
\nocite{programmingnotes}
\bibliography{pgfplots}
\end{document}
%
%%%%%%%%%%%%%%%%%%%%%%%%%%%%%%%%%%%%%%%%%%%%%%%%%%%%%%%%%%%%%%%%%%%%%%%%%%%%%
%
% Package pgfplots.sty documentation. 
%
% Copyright 2007/2008 by Christian Feuersaenger.
%
% This program is free software: you can redistribute it and/or modify
% it under the terms of the GNU General Public License as published by
% the Free Software Foundation, either version 3 of the License, or
% (at your option) any later version.
% 
% This program is distributed in the hope that it will be useful,
% but WITHOUT ANY WARRANTY; without even the implied warranty of
% MERCHANTABILITY or FITNESS FOR A PARTICULAR PURPOSE.  See the
% GNU General Public License for more details.
% 
% You should have received a copy of the GNU General Public License
% along with this program.  If not, see <http://www.gnu.org/licenses/>.
%
%
%%%%%%%%%%%%%%%%%%%%%%%%%%%%%%%%%%%%%%%%%%%%%%%%%%%%%%%%%%%%%%%%%%%%%%%%%%%%%
\input{pgfplots.preamble.tex}
%\RequirePackage[german,english,francais]{babel}

\pgfplotsmanualexternalexpensivetrue

%\usetikzlibrary{external}
%\tikzexternalize[prefix=figures/]{pgfplots}

\title{%
	Manual for Package \PGFPlots\\
	{\small Version \pgfplotsversion}\\
	{\small\href{http://sourceforge.net/projects/pgfplots}{http://sourceforge.net/projects/pgfplots}}}

%\includeonly{pgfplots.libs}


\begin{document}

\def\plotcoords{%
\addplot coordinates {
(5,8.312e-02)    (17,2.547e-02)   (49,7.407e-03)
(129,2.102e-03)  (321,5.874e-04)  (769,1.623e-04)
(1793,4.442e-05) (4097,1.207e-05) (9217,3.261e-06)
};

\addplot coordinates{
(7,8.472e-02)    (31,3.044e-02)    (111,1.022e-02)
(351,3.303e-03)  (1023,1.039e-03)  (2815,3.196e-04)
(7423,9.658e-05) (18943,2.873e-05) (47103,8.437e-06)
};

\addplot coordinates{
(9,7.881e-02)     (49,3.243e-02)    (209,1.232e-02)
(769,4.454e-03)   (2561,1.551e-03)  (7937,5.236e-04)
(23297,1.723e-04) (65537,5.545e-05) (178177,1.751e-05)
};

\addplot coordinates{
(11,6.887e-02)    (71,3.177e-02)     (351,1.341e-02)
(1471,5.334e-03)  (5503,2.027e-03)   (18943,7.415e-04)
(61183,2.628e-04) (187903,9.063e-05) (553983,3.053e-05)
};

\addplot coordinates{
(13,5.755e-02)     (97,2.925e-02)     (545,1.351e-02)
(2561,5.842e-03)   (10625,2.397e-03)  (40193,9.414e-04)
(141569,3.564e-04) (471041,1.308e-04) 
(1496065,4.670e-05)
};
}%
\maketitle
\begin{abstract}%
\PGFPlots\ draws high--quality function plots in normal or logarithmic scaling with a user-friendly interface directly in \TeX. The user supplies axis labels, legend entries and the plot coordinates for one or more plots and \PGFPlots\ applies axis scaling, computes any logarithms and axis ticks and draws the plots, supporting line plots, scatter plots, piecewise constant plots, bar plots, area plots, mesh-- and surface plots and some more. It is based on Till Tantau's package \PGF/\Tikz.
\end{abstract}
\tableofcontents
\section{Introduction}
This package provides tools to generate plots and labeled axes easily. It draws normal plots, logplots and semi-logplots, in two and three dimensions. Axis ticks, labels, legends (in case of multiple plots) can be added with key-value options. It can cycle through a set of predefined line/marker/color specifications. In summary, its purpose is to simplify the generation of high-quality function and/or data plots, and solving the problems of
\begin{itemize}
	\item consistency of document and font type and font size,
	\item direct use of \TeX\ math mode in axis descriptions,
	\item consistency of data and figures (no third party tool necessary),
	\item inter document consistency using preamble configurations and styles.
\end{itemize}
Although not necessary, separate |.pdf| or |.eps| graphics can be generated using the |external| library developed as part of \Tikz.

\PGFPlots\ is build completely on \Tikz/\PGF. Knowledge of \Tikz\ will simplify the work with \PGFPlots, although it is not required.

\section[About PGFPlots: Preliminaries]{About {\normalfont\PGFPlots}: Preliminaries}
This section contains information about upgrades, the team, the installation (in case you need to do it manually) and troubleshooting. You may skip it completely except for the upgrade remarks.

However, note that this library requires at least \PGF\ version $2.00$. At the time of this writing, many \TeX-distributions still contain the older \PGF\ version $1.18$, so it may be necessary to install a recent \PGF\ prior to using \PGFPlots.

\subsection{Components}
\PGFPlots\ comes with two components:
\begin{enumerate}
	\item the plotting component (which you are currently reading) and
	\item the \PGFPlotstable\ component which simplifies number formatting and postprocessing of numerical tables. It comes as a separate package and has its own manual \href{file:pgfplotstable.pdf}{pgfplotstable.pdf}.
\end{enumerate}

\subsection{Upgrade remarks}
This release provides a lot of improvements which can be found in all detail in \texttt{ChangeLog} for interested readers. However, some attention is useful with respect to the following changes.

\subsubsection{New Optional Features}
\begin{enumerate}
	\item \PGFPlots\ 1.3 comes with user interface improvements. The technical distinction between ``behavior options'' and ``style options'' of older versions is no longer necessary (although still fully supported).

	\item \PGFPlots\ 1.3 has a new feature which allows to \emph{move axis labels tight to tick labels} automatically.

	Since this affects the spacing, it is not enabled be default.

	Use
\begin{codeexample}[code only]
\usepackage{pgfplots}
\pgfplotsset{compat=1.3}
\end{codeexample}
	in your preamble to benefit from the improved spacing\footnote{The |compat=1.3| setting is necessary for new features which move axis labels around like reversed axis. However, \PGFPlots\ will still work as it does in earlier versions, your documents will remain the same if you don't set it explicitly.}.
	Take a look at the next page for the precise definition of |compat|.

	\item \PGFPlots\ 1.3 now supports reversed axes. It is no longer necessary to use work--arounds with negative units.
\pgfkeys{/pdflinks/search key prefixes in/.add={/pgfplots/,}{}}

	Take a look at the |x dir=reverse| key.

	Existing work--arounds will still function properly. Use |\pgfplotsset{compat=1.3}| together with |x dir=reverse| to switch to the new version.
\end{enumerate}

\subsubsection{Old Features Which May Need Attention}
\begin{enumerate}
	\item The |scatter/classes| feature produces proper legends as of version 1.3. This may change the appearance of existing legends of plots with |scatter/classes|.

	\item Due to a small math bug in \PGF\ $2.00$, you \emph{can't} use the math expression `|-x^2|'. It is necessary to use `|0-x^2|' instead. The same holds for
		`|exp(-x^2)|'; use `|exp(0-x^2)|' instead.

		This will be fixed with \PGF\ $>2.00$.

	\item Starting with \PGFPlots\ $1.1$, |\tikzstyle| should \emph{no longer be used} to set \PGFPlots\ options.
	
	Although |\tikzstyle| is still supported for some older \PGFPlots\ options, you should replace any occurance of |\tikzstyle| with |\pgfplotsset{|\meta{style name}|/.style={|\meta{key-value-list}|}}| or the associated |/.append style| variant. See section~\ref{sec:styles} for more detail.
\end{enumerate}
I apologize for any inconvenience caused by these changes.

\begin{pgfplotskey}{compat=\mchoice{1.3,pre 1.3,default,newest} (initially default)}
	The preamble configuration 
\begin{codeexample}[code only]
\usepackage{pgfplots}
\pgfplotsset{compat=1.3}
\end{codeexample}
	allows to choose between backwards compatibility and most recent features.

	\PGFPlots\ version 1.3 comes with new features which may lead to different spacings compared to earlier versions:
	\begin{enumerate}
		\item Version 1.3 comes with the |xlabel near ticks| feature which places (in this case $x$) axis labels ``near'' the tick labels, taking the size of tick labels into account.
		
		Older versions used |xlabel absolute|, i.e.\ an absolute distance between the axis and axis labels (now matter how large or small tick labels are).

		The initial setting \emph{keeps backwards compatibility}.

		You are encouraged to use
\begin{codeexample}[code only]
\usepackage{pgfplots}
\pgfplotsset{compat=1.3}
\end{codeexample}
		in your preamble.
		
		\item Version 1.3 fixes a bug with |anchor=south west| which occurred mainly for reversed axes: the anchors where upside--down. This is fixed now.

		
		If you have written some work--around and want to keep it going, use

		|\pgfplotsset{compat/anchors=pre 1.3}|\pgfmanualpdflabel{/pgfplots/compat/anchors}{} 

		to disable the bug fix.
	\end{enumerate}

	Use |\pgfplotsset{compat=default}| to restore the factory settings.

	The setting |\pgfplotsset{compat=newest}| will always use features of the most recent version. This might result in small changes in the document's appearance.
\end{pgfplotskey}

\subsection{The Team}
\PGFPlots\ has been written mainly by Christian Feuers�nger with many improvements of Pascal Wolkotte and Nick Papior Andersen as a spare time project. We hope it is useful and provides valuable plots.

If you are interested in writing something but don't know how, consider reading the auxiliary manual \href{file:TeX-programming-notes.pdf}{TeX-programming-notes.pdf} which comes with \PGFPlots. It is far from complete, but maybe it is a good starting point (at least for more literature).

\subsection{Acknowledgements}
I thank God for all hours of enjoyed programming. I thank Pascal Wolkotte and Nick Papior Andersen for their programming efforts and contributions as part of the development team. I thank Stefan Tibus, who contributed the |plot shell| feature. I thank Tom Cashman for the contribution of the |reverse legend| feature. Special thanks go to Stefan Pinnow whose tests of \PGFPlots\ lead to numerous quality improvements. Furthermore, I thank Dr.~Schweitzer for many fruitful discussions and Dr.~Meine for his ideas and suggestions. Thanks as well to the many international contributors who provided feature requests or identified bugs or simply manual improvements!

Last but not least, I thank Till Tantau and Mark Wibrow for their excellent graphics (and more) package \PGF\ and \Tikz\ which is the base of \PGFPlots.

\include{pgfplots.install}%
\include{pgfplots.intro}%
\include{pgfplots.reference}%
\include{pgfplots.libs}%
\include{pgfplots.resources}%
\include{pgfplots.importexport}%
\include{pgfplots.basic.reference}%

\printindex

\bibliographystyle{abbrv} %gerapali} %gerabbrv} %gerunsrt.bst} %gerabbrv}% gerplain}
\nocite{pgfplotstable}
\nocite{programmingnotes}
\bibliography{pgfplots}
\end{document}
%
%%%%%%%%%%%%%%%%%%%%%%%%%%%%%%%%%%%%%%%%%%%%%%%%%%%%%%%%%%%%%%%%%%%%%%%%%%%%%
%
% Package pgfplots.sty documentation. 
%
% Copyright 2007/2008 by Christian Feuersaenger.
%
% This program is free software: you can redistribute it and/or modify
% it under the terms of the GNU General Public License as published by
% the Free Software Foundation, either version 3 of the License, or
% (at your option) any later version.
% 
% This program is distributed in the hope that it will be useful,
% but WITHOUT ANY WARRANTY; without even the implied warranty of
% MERCHANTABILITY or FITNESS FOR A PARTICULAR PURPOSE.  See the
% GNU General Public License for more details.
% 
% You should have received a copy of the GNU General Public License
% along with this program.  If not, see <http://www.gnu.org/licenses/>.
%
%
%%%%%%%%%%%%%%%%%%%%%%%%%%%%%%%%%%%%%%%%%%%%%%%%%%%%%%%%%%%%%%%%%%%%%%%%%%%%%
\input{pgfplots.preamble.tex}
%\RequirePackage[german,english,francais]{babel}

\pgfplotsmanualexternalexpensivetrue

%\usetikzlibrary{external}
%\tikzexternalize[prefix=figures/]{pgfplots}

\title{%
	Manual for Package \PGFPlots\\
	{\small Version \pgfplotsversion}\\
	{\small\href{http://sourceforge.net/projects/pgfplots}{http://sourceforge.net/projects/pgfplots}}}

%\includeonly{pgfplots.libs}


\begin{document}

\def\plotcoords{%
\addplot coordinates {
(5,8.312e-02)    (17,2.547e-02)   (49,7.407e-03)
(129,2.102e-03)  (321,5.874e-04)  (769,1.623e-04)
(1793,4.442e-05) (4097,1.207e-05) (9217,3.261e-06)
};

\addplot coordinates{
(7,8.472e-02)    (31,3.044e-02)    (111,1.022e-02)
(351,3.303e-03)  (1023,1.039e-03)  (2815,3.196e-04)
(7423,9.658e-05) (18943,2.873e-05) (47103,8.437e-06)
};

\addplot coordinates{
(9,7.881e-02)     (49,3.243e-02)    (209,1.232e-02)
(769,4.454e-03)   (2561,1.551e-03)  (7937,5.236e-04)
(23297,1.723e-04) (65537,5.545e-05) (178177,1.751e-05)
};

\addplot coordinates{
(11,6.887e-02)    (71,3.177e-02)     (351,1.341e-02)
(1471,5.334e-03)  (5503,2.027e-03)   (18943,7.415e-04)
(61183,2.628e-04) (187903,9.063e-05) (553983,3.053e-05)
};

\addplot coordinates{
(13,5.755e-02)     (97,2.925e-02)     (545,1.351e-02)
(2561,5.842e-03)   (10625,2.397e-03)  (40193,9.414e-04)
(141569,3.564e-04) (471041,1.308e-04) 
(1496065,4.670e-05)
};
}%
\maketitle
\begin{abstract}%
\PGFPlots\ draws high--quality function plots in normal or logarithmic scaling with a user-friendly interface directly in \TeX. The user supplies axis labels, legend entries and the plot coordinates for one or more plots and \PGFPlots\ applies axis scaling, computes any logarithms and axis ticks and draws the plots, supporting line plots, scatter plots, piecewise constant plots, bar plots, area plots, mesh-- and surface plots and some more. It is based on Till Tantau's package \PGF/\Tikz.
\end{abstract}
\tableofcontents
\section{Introduction}
This package provides tools to generate plots and labeled axes easily. It draws normal plots, logplots and semi-logplots, in two and three dimensions. Axis ticks, labels, legends (in case of multiple plots) can be added with key-value options. It can cycle through a set of predefined line/marker/color specifications. In summary, its purpose is to simplify the generation of high-quality function and/or data plots, and solving the problems of
\begin{itemize}
	\item consistency of document and font type and font size,
	\item direct use of \TeX\ math mode in axis descriptions,
	\item consistency of data and figures (no third party tool necessary),
	\item inter document consistency using preamble configurations and styles.
\end{itemize}
Although not necessary, separate |.pdf| or |.eps| graphics can be generated using the |external| library developed as part of \Tikz.

\PGFPlots\ is build completely on \Tikz/\PGF. Knowledge of \Tikz\ will simplify the work with \PGFPlots, although it is not required.

\section[About PGFPlots: Preliminaries]{About {\normalfont\PGFPlots}: Preliminaries}
This section contains information about upgrades, the team, the installation (in case you need to do it manually) and troubleshooting. You may skip it completely except for the upgrade remarks.

However, note that this library requires at least \PGF\ version $2.00$. At the time of this writing, many \TeX-distributions still contain the older \PGF\ version $1.18$, so it may be necessary to install a recent \PGF\ prior to using \PGFPlots.

\subsection{Components}
\PGFPlots\ comes with two components:
\begin{enumerate}
	\item the plotting component (which you are currently reading) and
	\item the \PGFPlotstable\ component which simplifies number formatting and postprocessing of numerical tables. It comes as a separate package and has its own manual \href{file:pgfplotstable.pdf}{pgfplotstable.pdf}.
\end{enumerate}

\subsection{Upgrade remarks}
This release provides a lot of improvements which can be found in all detail in \texttt{ChangeLog} for interested readers. However, some attention is useful with respect to the following changes.

\subsubsection{New Optional Features}
\begin{enumerate}
	\item \PGFPlots\ 1.3 comes with user interface improvements. The technical distinction between ``behavior options'' and ``style options'' of older versions is no longer necessary (although still fully supported).

	\item \PGFPlots\ 1.3 has a new feature which allows to \emph{move axis labels tight to tick labels} automatically.

	Since this affects the spacing, it is not enabled be default.

	Use
\begin{codeexample}[code only]
\usepackage{pgfplots}
\pgfplotsset{compat=1.3}
\end{codeexample}
	in your preamble to benefit from the improved spacing\footnote{The |compat=1.3| setting is necessary for new features which move axis labels around like reversed axis. However, \PGFPlots\ will still work as it does in earlier versions, your documents will remain the same if you don't set it explicitly.}.
	Take a look at the next page for the precise definition of |compat|.

	\item \PGFPlots\ 1.3 now supports reversed axes. It is no longer necessary to use work--arounds with negative units.
\pgfkeys{/pdflinks/search key prefixes in/.add={/pgfplots/,}{}}

	Take a look at the |x dir=reverse| key.

	Existing work--arounds will still function properly. Use |\pgfplotsset{compat=1.3}| together with |x dir=reverse| to switch to the new version.
\end{enumerate}

\subsubsection{Old Features Which May Need Attention}
\begin{enumerate}
	\item The |scatter/classes| feature produces proper legends as of version 1.3. This may change the appearance of existing legends of plots with |scatter/classes|.

	\item Due to a small math bug in \PGF\ $2.00$, you \emph{can't} use the math expression `|-x^2|'. It is necessary to use `|0-x^2|' instead. The same holds for
		`|exp(-x^2)|'; use `|exp(0-x^2)|' instead.

		This will be fixed with \PGF\ $>2.00$.

	\item Starting with \PGFPlots\ $1.1$, |\tikzstyle| should \emph{no longer be used} to set \PGFPlots\ options.
	
	Although |\tikzstyle| is still supported for some older \PGFPlots\ options, you should replace any occurance of |\tikzstyle| with |\pgfplotsset{|\meta{style name}|/.style={|\meta{key-value-list}|}}| or the associated |/.append style| variant. See section~\ref{sec:styles} for more detail.
\end{enumerate}
I apologize for any inconvenience caused by these changes.

\begin{pgfplotskey}{compat=\mchoice{1.3,pre 1.3,default,newest} (initially default)}
	The preamble configuration 
\begin{codeexample}[code only]
\usepackage{pgfplots}
\pgfplotsset{compat=1.3}
\end{codeexample}
	allows to choose between backwards compatibility and most recent features.

	\PGFPlots\ version 1.3 comes with new features which may lead to different spacings compared to earlier versions:
	\begin{enumerate}
		\item Version 1.3 comes with the |xlabel near ticks| feature which places (in this case $x$) axis labels ``near'' the tick labels, taking the size of tick labels into account.
		
		Older versions used |xlabel absolute|, i.e.\ an absolute distance between the axis and axis labels (now matter how large or small tick labels are).

		The initial setting \emph{keeps backwards compatibility}.

		You are encouraged to use
\begin{codeexample}[code only]
\usepackage{pgfplots}
\pgfplotsset{compat=1.3}
\end{codeexample}
		in your preamble.
		
		\item Version 1.3 fixes a bug with |anchor=south west| which occurred mainly for reversed axes: the anchors where upside--down. This is fixed now.

		
		If you have written some work--around and want to keep it going, use

		|\pgfplotsset{compat/anchors=pre 1.3}|\pgfmanualpdflabel{/pgfplots/compat/anchors}{} 

		to disable the bug fix.
	\end{enumerate}

	Use |\pgfplotsset{compat=default}| to restore the factory settings.

	The setting |\pgfplotsset{compat=newest}| will always use features of the most recent version. This might result in small changes in the document's appearance.
\end{pgfplotskey}

\subsection{The Team}
\PGFPlots\ has been written mainly by Christian Feuers�nger with many improvements of Pascal Wolkotte and Nick Papior Andersen as a spare time project. We hope it is useful and provides valuable plots.

If you are interested in writing something but don't know how, consider reading the auxiliary manual \href{file:TeX-programming-notes.pdf}{TeX-programming-notes.pdf} which comes with \PGFPlots. It is far from complete, but maybe it is a good starting point (at least for more literature).

\subsection{Acknowledgements}
I thank God for all hours of enjoyed programming. I thank Pascal Wolkotte and Nick Papior Andersen for their programming efforts and contributions as part of the development team. I thank Stefan Tibus, who contributed the |plot shell| feature. I thank Tom Cashman for the contribution of the |reverse legend| feature. Special thanks go to Stefan Pinnow whose tests of \PGFPlots\ lead to numerous quality improvements. Furthermore, I thank Dr.~Schweitzer for many fruitful discussions and Dr.~Meine for his ideas and suggestions. Thanks as well to the many international contributors who provided feature requests or identified bugs or simply manual improvements!

Last but not least, I thank Till Tantau and Mark Wibrow for their excellent graphics (and more) package \PGF\ and \Tikz\ which is the base of \PGFPlots.

\include{pgfplots.install}%
\include{pgfplots.intro}%
\include{pgfplots.reference}%
\include{pgfplots.libs}%
\include{pgfplots.resources}%
\include{pgfplots.importexport}%
\include{pgfplots.basic.reference}%

\printindex

\bibliographystyle{abbrv} %gerapali} %gerabbrv} %gerunsrt.bst} %gerabbrv}% gerplain}
\nocite{pgfplotstable}
\nocite{programmingnotes}
\bibliography{pgfplots}
\end{document}
%
%%%%%%%%%%%%%%%%%%%%%%%%%%%%%%%%%%%%%%%%%%%%%%%%%%%%%%%%%%%%%%%%%%%%%%%%%%%%%
%
% Package pgfplots.sty documentation. 
%
% Copyright 2007/2008 by Christian Feuersaenger.
%
% This program is free software: you can redistribute it and/or modify
% it under the terms of the GNU General Public License as published by
% the Free Software Foundation, either version 3 of the License, or
% (at your option) any later version.
% 
% This program is distributed in the hope that it will be useful,
% but WITHOUT ANY WARRANTY; without even the implied warranty of
% MERCHANTABILITY or FITNESS FOR A PARTICULAR PURPOSE.  See the
% GNU General Public License for more details.
% 
% You should have received a copy of the GNU General Public License
% along with this program.  If not, see <http://www.gnu.org/licenses/>.
%
%
%%%%%%%%%%%%%%%%%%%%%%%%%%%%%%%%%%%%%%%%%%%%%%%%%%%%%%%%%%%%%%%%%%%%%%%%%%%%%
\input{pgfplots.preamble.tex}
%\RequirePackage[german,english,francais]{babel}

\pgfplotsmanualexternalexpensivetrue

%\usetikzlibrary{external}
%\tikzexternalize[prefix=figures/]{pgfplots}

\title{%
	Manual for Package \PGFPlots\\
	{\small Version \pgfplotsversion}\\
	{\small\href{http://sourceforge.net/projects/pgfplots}{http://sourceforge.net/projects/pgfplots}}}

%\includeonly{pgfplots.libs}


\begin{document}

\def\plotcoords{%
\addplot coordinates {
(5,8.312e-02)    (17,2.547e-02)   (49,7.407e-03)
(129,2.102e-03)  (321,5.874e-04)  (769,1.623e-04)
(1793,4.442e-05) (4097,1.207e-05) (9217,3.261e-06)
};

\addplot coordinates{
(7,8.472e-02)    (31,3.044e-02)    (111,1.022e-02)
(351,3.303e-03)  (1023,1.039e-03)  (2815,3.196e-04)
(7423,9.658e-05) (18943,2.873e-05) (47103,8.437e-06)
};

\addplot coordinates{
(9,7.881e-02)     (49,3.243e-02)    (209,1.232e-02)
(769,4.454e-03)   (2561,1.551e-03)  (7937,5.236e-04)
(23297,1.723e-04) (65537,5.545e-05) (178177,1.751e-05)
};

\addplot coordinates{
(11,6.887e-02)    (71,3.177e-02)     (351,1.341e-02)
(1471,5.334e-03)  (5503,2.027e-03)   (18943,7.415e-04)
(61183,2.628e-04) (187903,9.063e-05) (553983,3.053e-05)
};

\addplot coordinates{
(13,5.755e-02)     (97,2.925e-02)     (545,1.351e-02)
(2561,5.842e-03)   (10625,2.397e-03)  (40193,9.414e-04)
(141569,3.564e-04) (471041,1.308e-04) 
(1496065,4.670e-05)
};
}%
\maketitle
\begin{abstract}%
\PGFPlots\ draws high--quality function plots in normal or logarithmic scaling with a user-friendly interface directly in \TeX. The user supplies axis labels, legend entries and the plot coordinates for one or more plots and \PGFPlots\ applies axis scaling, computes any logarithms and axis ticks and draws the plots, supporting line plots, scatter plots, piecewise constant plots, bar plots, area plots, mesh-- and surface plots and some more. It is based on Till Tantau's package \PGF/\Tikz.
\end{abstract}
\tableofcontents
\section{Introduction}
This package provides tools to generate plots and labeled axes easily. It draws normal plots, logplots and semi-logplots, in two and three dimensions. Axis ticks, labels, legends (in case of multiple plots) can be added with key-value options. It can cycle through a set of predefined line/marker/color specifications. In summary, its purpose is to simplify the generation of high-quality function and/or data plots, and solving the problems of
\begin{itemize}
	\item consistency of document and font type and font size,
	\item direct use of \TeX\ math mode in axis descriptions,
	\item consistency of data and figures (no third party tool necessary),
	\item inter document consistency using preamble configurations and styles.
\end{itemize}
Although not necessary, separate |.pdf| or |.eps| graphics can be generated using the |external| library developed as part of \Tikz.

\PGFPlots\ is build completely on \Tikz/\PGF. Knowledge of \Tikz\ will simplify the work with \PGFPlots, although it is not required.

\section[About PGFPlots: Preliminaries]{About {\normalfont\PGFPlots}: Preliminaries}
This section contains information about upgrades, the team, the installation (in case you need to do it manually) and troubleshooting. You may skip it completely except for the upgrade remarks.

However, note that this library requires at least \PGF\ version $2.00$. At the time of this writing, many \TeX-distributions still contain the older \PGF\ version $1.18$, so it may be necessary to install a recent \PGF\ prior to using \PGFPlots.

\subsection{Components}
\PGFPlots\ comes with two components:
\begin{enumerate}
	\item the plotting component (which you are currently reading) and
	\item the \PGFPlotstable\ component which simplifies number formatting and postprocessing of numerical tables. It comes as a separate package and has its own manual \href{file:pgfplotstable.pdf}{pgfplotstable.pdf}.
\end{enumerate}

\subsection{Upgrade remarks}
This release provides a lot of improvements which can be found in all detail in \texttt{ChangeLog} for interested readers. However, some attention is useful with respect to the following changes.

\subsubsection{New Optional Features}
\begin{enumerate}
	\item \PGFPlots\ 1.3 comes with user interface improvements. The technical distinction between ``behavior options'' and ``style options'' of older versions is no longer necessary (although still fully supported).

	\item \PGFPlots\ 1.3 has a new feature which allows to \emph{move axis labels tight to tick labels} automatically.

	Since this affects the spacing, it is not enabled be default.

	Use
\begin{codeexample}[code only]
\usepackage{pgfplots}
\pgfplotsset{compat=1.3}
\end{codeexample}
	in your preamble to benefit from the improved spacing\footnote{The |compat=1.3| setting is necessary for new features which move axis labels around like reversed axis. However, \PGFPlots\ will still work as it does in earlier versions, your documents will remain the same if you don't set it explicitly.}.
	Take a look at the next page for the precise definition of |compat|.

	\item \PGFPlots\ 1.3 now supports reversed axes. It is no longer necessary to use work--arounds with negative units.
\pgfkeys{/pdflinks/search key prefixes in/.add={/pgfplots/,}{}}

	Take a look at the |x dir=reverse| key.

	Existing work--arounds will still function properly. Use |\pgfplotsset{compat=1.3}| together with |x dir=reverse| to switch to the new version.
\end{enumerate}

\subsubsection{Old Features Which May Need Attention}
\begin{enumerate}
	\item The |scatter/classes| feature produces proper legends as of version 1.3. This may change the appearance of existing legends of plots with |scatter/classes|.

	\item Due to a small math bug in \PGF\ $2.00$, you \emph{can't} use the math expression `|-x^2|'. It is necessary to use `|0-x^2|' instead. The same holds for
		`|exp(-x^2)|'; use `|exp(0-x^2)|' instead.

		This will be fixed with \PGF\ $>2.00$.

	\item Starting with \PGFPlots\ $1.1$, |\tikzstyle| should \emph{no longer be used} to set \PGFPlots\ options.
	
	Although |\tikzstyle| is still supported for some older \PGFPlots\ options, you should replace any occurance of |\tikzstyle| with |\pgfplotsset{|\meta{style name}|/.style={|\meta{key-value-list}|}}| or the associated |/.append style| variant. See section~\ref{sec:styles} for more detail.
\end{enumerate}
I apologize for any inconvenience caused by these changes.

\begin{pgfplotskey}{compat=\mchoice{1.3,pre 1.3,default,newest} (initially default)}
	The preamble configuration 
\begin{codeexample}[code only]
\usepackage{pgfplots}
\pgfplotsset{compat=1.3}
\end{codeexample}
	allows to choose between backwards compatibility and most recent features.

	\PGFPlots\ version 1.3 comes with new features which may lead to different spacings compared to earlier versions:
	\begin{enumerate}
		\item Version 1.3 comes with the |xlabel near ticks| feature which places (in this case $x$) axis labels ``near'' the tick labels, taking the size of tick labels into account.
		
		Older versions used |xlabel absolute|, i.e.\ an absolute distance between the axis and axis labels (now matter how large or small tick labels are).

		The initial setting \emph{keeps backwards compatibility}.

		You are encouraged to use
\begin{codeexample}[code only]
\usepackage{pgfplots}
\pgfplotsset{compat=1.3}
\end{codeexample}
		in your preamble.
		
		\item Version 1.3 fixes a bug with |anchor=south west| which occurred mainly for reversed axes: the anchors where upside--down. This is fixed now.

		
		If you have written some work--around and want to keep it going, use

		|\pgfplotsset{compat/anchors=pre 1.3}|\pgfmanualpdflabel{/pgfplots/compat/anchors}{} 

		to disable the bug fix.
	\end{enumerate}

	Use |\pgfplotsset{compat=default}| to restore the factory settings.

	The setting |\pgfplotsset{compat=newest}| will always use features of the most recent version. This might result in small changes in the document's appearance.
\end{pgfplotskey}

\subsection{The Team}
\PGFPlots\ has been written mainly by Christian Feuers�nger with many improvements of Pascal Wolkotte and Nick Papior Andersen as a spare time project. We hope it is useful and provides valuable plots.

If you are interested in writing something but don't know how, consider reading the auxiliary manual \href{file:TeX-programming-notes.pdf}{TeX-programming-notes.pdf} which comes with \PGFPlots. It is far from complete, but maybe it is a good starting point (at least for more literature).

\subsection{Acknowledgements}
I thank God for all hours of enjoyed programming. I thank Pascal Wolkotte and Nick Papior Andersen for their programming efforts and contributions as part of the development team. I thank Stefan Tibus, who contributed the |plot shell| feature. I thank Tom Cashman for the contribution of the |reverse legend| feature. Special thanks go to Stefan Pinnow whose tests of \PGFPlots\ lead to numerous quality improvements. Furthermore, I thank Dr.~Schweitzer for many fruitful discussions and Dr.~Meine for his ideas and suggestions. Thanks as well to the many international contributors who provided feature requests or identified bugs or simply manual improvements!

Last but not least, I thank Till Tantau and Mark Wibrow for their excellent graphics (and more) package \PGF\ and \Tikz\ which is the base of \PGFPlots.

\include{pgfplots.install}%
\include{pgfplots.intro}%
\include{pgfplots.reference}%
\include{pgfplots.libs}%
\include{pgfplots.resources}%
\include{pgfplots.importexport}%
\include{pgfplots.basic.reference}%

\printindex

\bibliographystyle{abbrv} %gerapali} %gerabbrv} %gerunsrt.bst} %gerabbrv}% gerplain}
\nocite{pgfplotstable}
\nocite{programmingnotes}
\bibliography{pgfplots}
\end{document}
%
%%%%%%%%%%%%%%%%%%%%%%%%%%%%%%%%%%%%%%%%%%%%%%%%%%%%%%%%%%%%%%%%%%%%%%%%%%%%%
%
% Package pgfplots.sty documentation. 
%
% Copyright 2007/2008 by Christian Feuersaenger.
%
% This program is free software: you can redistribute it and/or modify
% it under the terms of the GNU General Public License as published by
% the Free Software Foundation, either version 3 of the License, or
% (at your option) any later version.
% 
% This program is distributed in the hope that it will be useful,
% but WITHOUT ANY WARRANTY; without even the implied warranty of
% MERCHANTABILITY or FITNESS FOR A PARTICULAR PURPOSE.  See the
% GNU General Public License for more details.
% 
% You should have received a copy of the GNU General Public License
% along with this program.  If not, see <http://www.gnu.org/licenses/>.
%
%
%%%%%%%%%%%%%%%%%%%%%%%%%%%%%%%%%%%%%%%%%%%%%%%%%%%%%%%%%%%%%%%%%%%%%%%%%%%%%
\input{pgfplots.preamble.tex}
%\RequirePackage[german,english,francais]{babel}

\pgfplotsmanualexternalexpensivetrue

%\usetikzlibrary{external}
%\tikzexternalize[prefix=figures/]{pgfplots}

\title{%
	Manual for Package \PGFPlots\\
	{\small Version \pgfplotsversion}\\
	{\small\href{http://sourceforge.net/projects/pgfplots}{http://sourceforge.net/projects/pgfplots}}}

%\includeonly{pgfplots.libs}


\begin{document}

\def\plotcoords{%
\addplot coordinates {
(5,8.312e-02)    (17,2.547e-02)   (49,7.407e-03)
(129,2.102e-03)  (321,5.874e-04)  (769,1.623e-04)
(1793,4.442e-05) (4097,1.207e-05) (9217,3.261e-06)
};

\addplot coordinates{
(7,8.472e-02)    (31,3.044e-02)    (111,1.022e-02)
(351,3.303e-03)  (1023,1.039e-03)  (2815,3.196e-04)
(7423,9.658e-05) (18943,2.873e-05) (47103,8.437e-06)
};

\addplot coordinates{
(9,7.881e-02)     (49,3.243e-02)    (209,1.232e-02)
(769,4.454e-03)   (2561,1.551e-03)  (7937,5.236e-04)
(23297,1.723e-04) (65537,5.545e-05) (178177,1.751e-05)
};

\addplot coordinates{
(11,6.887e-02)    (71,3.177e-02)     (351,1.341e-02)
(1471,5.334e-03)  (5503,2.027e-03)   (18943,7.415e-04)
(61183,2.628e-04) (187903,9.063e-05) (553983,3.053e-05)
};

\addplot coordinates{
(13,5.755e-02)     (97,2.925e-02)     (545,1.351e-02)
(2561,5.842e-03)   (10625,2.397e-03)  (40193,9.414e-04)
(141569,3.564e-04) (471041,1.308e-04) 
(1496065,4.670e-05)
};
}%
\maketitle
\begin{abstract}%
\PGFPlots\ draws high--quality function plots in normal or logarithmic scaling with a user-friendly interface directly in \TeX. The user supplies axis labels, legend entries and the plot coordinates for one or more plots and \PGFPlots\ applies axis scaling, computes any logarithms and axis ticks and draws the plots, supporting line plots, scatter plots, piecewise constant plots, bar plots, area plots, mesh-- and surface plots and some more. It is based on Till Tantau's package \PGF/\Tikz.
\end{abstract}
\tableofcontents
\section{Introduction}
This package provides tools to generate plots and labeled axes easily. It draws normal plots, logplots and semi-logplots, in two and three dimensions. Axis ticks, labels, legends (in case of multiple plots) can be added with key-value options. It can cycle through a set of predefined line/marker/color specifications. In summary, its purpose is to simplify the generation of high-quality function and/or data plots, and solving the problems of
\begin{itemize}
	\item consistency of document and font type and font size,
	\item direct use of \TeX\ math mode in axis descriptions,
	\item consistency of data and figures (no third party tool necessary),
	\item inter document consistency using preamble configurations and styles.
\end{itemize}
Although not necessary, separate |.pdf| or |.eps| graphics can be generated using the |external| library developed as part of \Tikz.

\PGFPlots\ is build completely on \Tikz/\PGF. Knowledge of \Tikz\ will simplify the work with \PGFPlots, although it is not required.

\section[About PGFPlots: Preliminaries]{About {\normalfont\PGFPlots}: Preliminaries}
This section contains information about upgrades, the team, the installation (in case you need to do it manually) and troubleshooting. You may skip it completely except for the upgrade remarks.

However, note that this library requires at least \PGF\ version $2.00$. At the time of this writing, many \TeX-distributions still contain the older \PGF\ version $1.18$, so it may be necessary to install a recent \PGF\ prior to using \PGFPlots.

\subsection{Components}
\PGFPlots\ comes with two components:
\begin{enumerate}
	\item the plotting component (which you are currently reading) and
	\item the \PGFPlotstable\ component which simplifies number formatting and postprocessing of numerical tables. It comes as a separate package and has its own manual \href{file:pgfplotstable.pdf}{pgfplotstable.pdf}.
\end{enumerate}

\subsection{Upgrade remarks}
This release provides a lot of improvements which can be found in all detail in \texttt{ChangeLog} for interested readers. However, some attention is useful with respect to the following changes.

\subsubsection{New Optional Features}
\begin{enumerate}
	\item \PGFPlots\ 1.3 comes with user interface improvements. The technical distinction between ``behavior options'' and ``style options'' of older versions is no longer necessary (although still fully supported).

	\item \PGFPlots\ 1.3 has a new feature which allows to \emph{move axis labels tight to tick labels} automatically.

	Since this affects the spacing, it is not enabled be default.

	Use
\begin{codeexample}[code only]
\usepackage{pgfplots}
\pgfplotsset{compat=1.3}
\end{codeexample}
	in your preamble to benefit from the improved spacing\footnote{The |compat=1.3| setting is necessary for new features which move axis labels around like reversed axis. However, \PGFPlots\ will still work as it does in earlier versions, your documents will remain the same if you don't set it explicitly.}.
	Take a look at the next page for the precise definition of |compat|.

	\item \PGFPlots\ 1.3 now supports reversed axes. It is no longer necessary to use work--arounds with negative units.
\pgfkeys{/pdflinks/search key prefixes in/.add={/pgfplots/,}{}}

	Take a look at the |x dir=reverse| key.

	Existing work--arounds will still function properly. Use |\pgfplotsset{compat=1.3}| together with |x dir=reverse| to switch to the new version.
\end{enumerate}

\subsubsection{Old Features Which May Need Attention}
\begin{enumerate}
	\item The |scatter/classes| feature produces proper legends as of version 1.3. This may change the appearance of existing legends of plots with |scatter/classes|.

	\item Due to a small math bug in \PGF\ $2.00$, you \emph{can't} use the math expression `|-x^2|'. It is necessary to use `|0-x^2|' instead. The same holds for
		`|exp(-x^2)|'; use `|exp(0-x^2)|' instead.

		This will be fixed with \PGF\ $>2.00$.

	\item Starting with \PGFPlots\ $1.1$, |\tikzstyle| should \emph{no longer be used} to set \PGFPlots\ options.
	
	Although |\tikzstyle| is still supported for some older \PGFPlots\ options, you should replace any occurance of |\tikzstyle| with |\pgfplotsset{|\meta{style name}|/.style={|\meta{key-value-list}|}}| or the associated |/.append style| variant. See section~\ref{sec:styles} for more detail.
\end{enumerate}
I apologize for any inconvenience caused by these changes.

\begin{pgfplotskey}{compat=\mchoice{1.3,pre 1.3,default,newest} (initially default)}
	The preamble configuration 
\begin{codeexample}[code only]
\usepackage{pgfplots}
\pgfplotsset{compat=1.3}
\end{codeexample}
	allows to choose between backwards compatibility and most recent features.

	\PGFPlots\ version 1.3 comes with new features which may lead to different spacings compared to earlier versions:
	\begin{enumerate}
		\item Version 1.3 comes with the |xlabel near ticks| feature which places (in this case $x$) axis labels ``near'' the tick labels, taking the size of tick labels into account.
		
		Older versions used |xlabel absolute|, i.e.\ an absolute distance between the axis and axis labels (now matter how large or small tick labels are).

		The initial setting \emph{keeps backwards compatibility}.

		You are encouraged to use
\begin{codeexample}[code only]
\usepackage{pgfplots}
\pgfplotsset{compat=1.3}
\end{codeexample}
		in your preamble.
		
		\item Version 1.3 fixes a bug with |anchor=south west| which occurred mainly for reversed axes: the anchors where upside--down. This is fixed now.

		
		If you have written some work--around and want to keep it going, use

		|\pgfplotsset{compat/anchors=pre 1.3}|\pgfmanualpdflabel{/pgfplots/compat/anchors}{} 

		to disable the bug fix.
	\end{enumerate}

	Use |\pgfplotsset{compat=default}| to restore the factory settings.

	The setting |\pgfplotsset{compat=newest}| will always use features of the most recent version. This might result in small changes in the document's appearance.
\end{pgfplotskey}

\subsection{The Team}
\PGFPlots\ has been written mainly by Christian Feuers�nger with many improvements of Pascal Wolkotte and Nick Papior Andersen as a spare time project. We hope it is useful and provides valuable plots.

If you are interested in writing something but don't know how, consider reading the auxiliary manual \href{file:TeX-programming-notes.pdf}{TeX-programming-notes.pdf} which comes with \PGFPlots. It is far from complete, but maybe it is a good starting point (at least for more literature).

\subsection{Acknowledgements}
I thank God for all hours of enjoyed programming. I thank Pascal Wolkotte and Nick Papior Andersen for their programming efforts and contributions as part of the development team. I thank Stefan Tibus, who contributed the |plot shell| feature. I thank Tom Cashman for the contribution of the |reverse legend| feature. Special thanks go to Stefan Pinnow whose tests of \PGFPlots\ lead to numerous quality improvements. Furthermore, I thank Dr.~Schweitzer for many fruitful discussions and Dr.~Meine for his ideas and suggestions. Thanks as well to the many international contributors who provided feature requests or identified bugs or simply manual improvements!

Last but not least, I thank Till Tantau and Mark Wibrow for their excellent graphics (and more) package \PGF\ and \Tikz\ which is the base of \PGFPlots.

\include{pgfplots.install}%
\include{pgfplots.intro}%
\include{pgfplots.reference}%
\include{pgfplots.libs}%
\include{pgfplots.resources}%
\include{pgfplots.importexport}%
\include{pgfplots.basic.reference}%

\printindex

\bibliographystyle{abbrv} %gerapali} %gerabbrv} %gerunsrt.bst} %gerabbrv}% gerplain}
\nocite{pgfplotstable}
\nocite{programmingnotes}
\bibliography{pgfplots}
\end{document}
%
%%%%%%%%%%%%%%%%%%%%%%%%%%%%%%%%%%%%%%%%%%%%%%%%%%%%%%%%%%%%%%%%%%%%%%%%%%%%%
%
% Package pgfplots.sty documentation. 
%
% Copyright 2007/2008 by Christian Feuersaenger.
%
% This program is free software: you can redistribute it and/or modify
% it under the terms of the GNU General Public License as published by
% the Free Software Foundation, either version 3 of the License, or
% (at your option) any later version.
% 
% This program is distributed in the hope that it will be useful,
% but WITHOUT ANY WARRANTY; without even the implied warranty of
% MERCHANTABILITY or FITNESS FOR A PARTICULAR PURPOSE.  See the
% GNU General Public License for more details.
% 
% You should have received a copy of the GNU General Public License
% along with this program.  If not, see <http://www.gnu.org/licenses/>.
%
%
%%%%%%%%%%%%%%%%%%%%%%%%%%%%%%%%%%%%%%%%%%%%%%%%%%%%%%%%%%%%%%%%%%%%%%%%%%%%%
\input{pgfplots.preamble.tex}
%\RequirePackage[german,english,francais]{babel}

\pgfplotsmanualexternalexpensivetrue

%\usetikzlibrary{external}
%\tikzexternalize[prefix=figures/]{pgfplots}

\title{%
	Manual for Package \PGFPlots\\
	{\small Version \pgfplotsversion}\\
	{\small\href{http://sourceforge.net/projects/pgfplots}{http://sourceforge.net/projects/pgfplots}}}

%\includeonly{pgfplots.libs}


\begin{document}

\def\plotcoords{%
\addplot coordinates {
(5,8.312e-02)    (17,2.547e-02)   (49,7.407e-03)
(129,2.102e-03)  (321,5.874e-04)  (769,1.623e-04)
(1793,4.442e-05) (4097,1.207e-05) (9217,3.261e-06)
};

\addplot coordinates{
(7,8.472e-02)    (31,3.044e-02)    (111,1.022e-02)
(351,3.303e-03)  (1023,1.039e-03)  (2815,3.196e-04)
(7423,9.658e-05) (18943,2.873e-05) (47103,8.437e-06)
};

\addplot coordinates{
(9,7.881e-02)     (49,3.243e-02)    (209,1.232e-02)
(769,4.454e-03)   (2561,1.551e-03)  (7937,5.236e-04)
(23297,1.723e-04) (65537,5.545e-05) (178177,1.751e-05)
};

\addplot coordinates{
(11,6.887e-02)    (71,3.177e-02)     (351,1.341e-02)
(1471,5.334e-03)  (5503,2.027e-03)   (18943,7.415e-04)
(61183,2.628e-04) (187903,9.063e-05) (553983,3.053e-05)
};

\addplot coordinates{
(13,5.755e-02)     (97,2.925e-02)     (545,1.351e-02)
(2561,5.842e-03)   (10625,2.397e-03)  (40193,9.414e-04)
(141569,3.564e-04) (471041,1.308e-04) 
(1496065,4.670e-05)
};
}%
\maketitle
\begin{abstract}%
\PGFPlots\ draws high--quality function plots in normal or logarithmic scaling with a user-friendly interface directly in \TeX. The user supplies axis labels, legend entries and the plot coordinates for one or more plots and \PGFPlots\ applies axis scaling, computes any logarithms and axis ticks and draws the plots, supporting line plots, scatter plots, piecewise constant plots, bar plots, area plots, mesh-- and surface plots and some more. It is based on Till Tantau's package \PGF/\Tikz.
\end{abstract}
\tableofcontents
\section{Introduction}
This package provides tools to generate plots and labeled axes easily. It draws normal plots, logplots and semi-logplots, in two and three dimensions. Axis ticks, labels, legends (in case of multiple plots) can be added with key-value options. It can cycle through a set of predefined line/marker/color specifications. In summary, its purpose is to simplify the generation of high-quality function and/or data plots, and solving the problems of
\begin{itemize}
	\item consistency of document and font type and font size,
	\item direct use of \TeX\ math mode in axis descriptions,
	\item consistency of data and figures (no third party tool necessary),
	\item inter document consistency using preamble configurations and styles.
\end{itemize}
Although not necessary, separate |.pdf| or |.eps| graphics can be generated using the |external| library developed as part of \Tikz.

\PGFPlots\ is build completely on \Tikz/\PGF. Knowledge of \Tikz\ will simplify the work with \PGFPlots, although it is not required.

\section[About PGFPlots: Preliminaries]{About {\normalfont\PGFPlots}: Preliminaries}
This section contains information about upgrades, the team, the installation (in case you need to do it manually) and troubleshooting. You may skip it completely except for the upgrade remarks.

However, note that this library requires at least \PGF\ version $2.00$. At the time of this writing, many \TeX-distributions still contain the older \PGF\ version $1.18$, so it may be necessary to install a recent \PGF\ prior to using \PGFPlots.

\subsection{Components}
\PGFPlots\ comes with two components:
\begin{enumerate}
	\item the plotting component (which you are currently reading) and
	\item the \PGFPlotstable\ component which simplifies number formatting and postprocessing of numerical tables. It comes as a separate package and has its own manual \href{file:pgfplotstable.pdf}{pgfplotstable.pdf}.
\end{enumerate}

\subsection{Upgrade remarks}
This release provides a lot of improvements which can be found in all detail in \texttt{ChangeLog} for interested readers. However, some attention is useful with respect to the following changes.

\subsubsection{New Optional Features}
\begin{enumerate}
	\item \PGFPlots\ 1.3 comes with user interface improvements. The technical distinction between ``behavior options'' and ``style options'' of older versions is no longer necessary (although still fully supported).

	\item \PGFPlots\ 1.3 has a new feature which allows to \emph{move axis labels tight to tick labels} automatically.

	Since this affects the spacing, it is not enabled be default.

	Use
\begin{codeexample}[code only]
\usepackage{pgfplots}
\pgfplotsset{compat=1.3}
\end{codeexample}
	in your preamble to benefit from the improved spacing\footnote{The |compat=1.3| setting is necessary for new features which move axis labels around like reversed axis. However, \PGFPlots\ will still work as it does in earlier versions, your documents will remain the same if you don't set it explicitly.}.
	Take a look at the next page for the precise definition of |compat|.

	\item \PGFPlots\ 1.3 now supports reversed axes. It is no longer necessary to use work--arounds with negative units.
\pgfkeys{/pdflinks/search key prefixes in/.add={/pgfplots/,}{}}

	Take a look at the |x dir=reverse| key.

	Existing work--arounds will still function properly. Use |\pgfplotsset{compat=1.3}| together with |x dir=reverse| to switch to the new version.
\end{enumerate}

\subsubsection{Old Features Which May Need Attention}
\begin{enumerate}
	\item The |scatter/classes| feature produces proper legends as of version 1.3. This may change the appearance of existing legends of plots with |scatter/classes|.

	\item Due to a small math bug in \PGF\ $2.00$, you \emph{can't} use the math expression `|-x^2|'. It is necessary to use `|0-x^2|' instead. The same holds for
		`|exp(-x^2)|'; use `|exp(0-x^2)|' instead.

		This will be fixed with \PGF\ $>2.00$.

	\item Starting with \PGFPlots\ $1.1$, |\tikzstyle| should \emph{no longer be used} to set \PGFPlots\ options.
	
	Although |\tikzstyle| is still supported for some older \PGFPlots\ options, you should replace any occurance of |\tikzstyle| with |\pgfplotsset{|\meta{style name}|/.style={|\meta{key-value-list}|}}| or the associated |/.append style| variant. See section~\ref{sec:styles} for more detail.
\end{enumerate}
I apologize for any inconvenience caused by these changes.

\begin{pgfplotskey}{compat=\mchoice{1.3,pre 1.3,default,newest} (initially default)}
	The preamble configuration 
\begin{codeexample}[code only]
\usepackage{pgfplots}
\pgfplotsset{compat=1.3}
\end{codeexample}
	allows to choose between backwards compatibility and most recent features.

	\PGFPlots\ version 1.3 comes with new features which may lead to different spacings compared to earlier versions:
	\begin{enumerate}
		\item Version 1.3 comes with the |xlabel near ticks| feature which places (in this case $x$) axis labels ``near'' the tick labels, taking the size of tick labels into account.
		
		Older versions used |xlabel absolute|, i.e.\ an absolute distance between the axis and axis labels (now matter how large or small tick labels are).

		The initial setting \emph{keeps backwards compatibility}.

		You are encouraged to use
\begin{codeexample}[code only]
\usepackage{pgfplots}
\pgfplotsset{compat=1.3}
\end{codeexample}
		in your preamble.
		
		\item Version 1.3 fixes a bug with |anchor=south west| which occurred mainly for reversed axes: the anchors where upside--down. This is fixed now.

		
		If you have written some work--around and want to keep it going, use

		|\pgfplotsset{compat/anchors=pre 1.3}|\pgfmanualpdflabel{/pgfplots/compat/anchors}{} 

		to disable the bug fix.
	\end{enumerate}

	Use |\pgfplotsset{compat=default}| to restore the factory settings.

	The setting |\pgfplotsset{compat=newest}| will always use features of the most recent version. This might result in small changes in the document's appearance.
\end{pgfplotskey}

\subsection{The Team}
\PGFPlots\ has been written mainly by Christian Feuers�nger with many improvements of Pascal Wolkotte and Nick Papior Andersen as a spare time project. We hope it is useful and provides valuable plots.

If you are interested in writing something but don't know how, consider reading the auxiliary manual \href{file:TeX-programming-notes.pdf}{TeX-programming-notes.pdf} which comes with \PGFPlots. It is far from complete, but maybe it is a good starting point (at least for more literature).

\subsection{Acknowledgements}
I thank God for all hours of enjoyed programming. I thank Pascal Wolkotte and Nick Papior Andersen for their programming efforts and contributions as part of the development team. I thank Stefan Tibus, who contributed the |plot shell| feature. I thank Tom Cashman for the contribution of the |reverse legend| feature. Special thanks go to Stefan Pinnow whose tests of \PGFPlots\ lead to numerous quality improvements. Furthermore, I thank Dr.~Schweitzer for many fruitful discussions and Dr.~Meine for his ideas and suggestions. Thanks as well to the many international contributors who provided feature requests or identified bugs or simply manual improvements!

Last but not least, I thank Till Tantau and Mark Wibrow for their excellent graphics (and more) package \PGF\ and \Tikz\ which is the base of \PGFPlots.

\include{pgfplots.install}%
\include{pgfplots.intro}%
\include{pgfplots.reference}%
\include{pgfplots.libs}%
\include{pgfplots.resources}%
\include{pgfplots.importexport}%
\include{pgfplots.basic.reference}%

\printindex

\bibliographystyle{abbrv} %gerapali} %gerabbrv} %gerunsrt.bst} %gerabbrv}% gerplain}
\nocite{pgfplotstable}
\nocite{programmingnotes}
\bibliography{pgfplots}
\end{document}
%
\include{pgfplots.basic.reference}%

\printindex

\bibliographystyle{abbrv} %gerapali} %gerabbrv} %gerunsrt.bst} %gerabbrv}% gerplain}
\nocite{pgfplotstable}
\nocite{programmingnotes}
\bibliography{pgfplots}
\end{document}
%
%%%%%%%%%%%%%%%%%%%%%%%%%%%%%%%%%%%%%%%%%%%%%%%%%%%%%%%%%%%%%%%%%%%%%%%%%%%%%
%
% Package pgfplots.sty documentation. 
%
% Copyright 2007/2008 by Christian Feuersaenger.
%
% This program is free software: you can redistribute it and/or modify
% it under the terms of the GNU General Public License as published by
% the Free Software Foundation, either version 3 of the License, or
% (at your option) any later version.
% 
% This program is distributed in the hope that it will be useful,
% but WITHOUT ANY WARRANTY; without even the implied warranty of
% MERCHANTABILITY or FITNESS FOR A PARTICULAR PURPOSE.  See the
% GNU General Public License for more details.
% 
% You should have received a copy of the GNU General Public License
% along with this program.  If not, see <http://www.gnu.org/licenses/>.
%
%
%%%%%%%%%%%%%%%%%%%%%%%%%%%%%%%%%%%%%%%%%%%%%%%%%%%%%%%%%%%%%%%%%%%%%%%%%%%%%
%%%%%%%%%%%%%%%%%%%%%%%%%%%%%%%%%%%%%%%%%%%%%%%%%%%%%%%%%%%%%%%%%%%%%%%%%%%%%
%
% Package pgfplots.sty documentation. 
%
% Copyright 2007/2008 by Christian Feuersaenger.
%
% This program is free software: you can redistribute it and/or modify
% it under the terms of the GNU General Public License as published by
% the Free Software Foundation, either version 3 of the License, or
% (at your option) any later version.
% 
% This program is distributed in the hope that it will be useful,
% but WITHOUT ANY WARRANTY; without even the implied warranty of
% MERCHANTABILITY or FITNESS FOR A PARTICULAR PURPOSE.  See the
% GNU General Public License for more details.
% 
% You should have received a copy of the GNU General Public License
% along with this program.  If not, see <http://www.gnu.org/licenses/>.
%
%
%%%%%%%%%%%%%%%%%%%%%%%%%%%%%%%%%%%%%%%%%%%%%%%%%%%%%%%%%%%%%%%%%%%%%%%%%%%%%
\documentclass[a4paper]{ltxdoc}

\usepackage{makeidx}

% DON't let hyperref overload the format of index and glossary. 
% I want to do that on my own in the stylefiles for makeindex...
\makeatletter
\let\@old@wrindex=\@wrindex
\makeatother

\usepackage{ifpdf}
\ifpdf
	\usepackage[pdftex]{hyperref}
\else
	\usepackage[dvipdfm]{hyperref}
\fi
\hypersetup{pdfborder={0 0 0}}

\makeatletter
\let\@wrindex=\@old@wrindex
\makeatother

% Formatiere Seitennummern im Index:
\newcommand{\indexpageno}[1]{%
	{\bfseries\hyperpage{#1}}%
}


\newcommand{\R}{\mathbb{R}}
\newcommand{\N}{\mathbb{N}}
\newcommand{\Z}{\mathbb{Z}}

\long\def\COMMENTLOWLEVEL#1\ENDCOMMENT{}
\def\ENDCOMMENT{}

\usepackage{textcomp}
\usepackage{booktabs}

\usepackage{calc}
\usepackage[formats]{listings}
%\usepackage{courier} % don't use it - the '^' character can't be copy-pasted in courier

\usepackage{array}
\lstset{%
	basicstyle=\ttfamily,
	language=[LaTeX]tex, % Seems as if \lstset{language=tex} must be invoked BEFORE loading tikz!?
	tabsize=4,
	breaklines=true,
	breakindent=0pt
}

\ifpdf
	\pdfinfo {
		/Author	(Christian Feuersaenger)
	}

\else
	\def\pgfsysdriver{pgfsys-dvipdfm.def}
\fi
%\def\pgfsysdriver{pgfsys-pdftex.def}
\usepackage{pgfplots}

\ifpdf
	\usetikzlibrary{pgfplotsclickable}
\fi

%\usepackage{fp}
% ATTENTION:
% this requires pgf version NEWER than 2.00 :
%\usetikzlibrary{fixedpointarithmetic}

\usetikzlibrary{dateplot}

\usepackage[a4paper,left=2.25cm,right=2.25cm,top=2.5cm,bottom=2.5cm,nohead]{geometry}
\usepackage{amsmath,amssymb}
\usepackage{xxcolor}
\usepackage{pifont}
\usepackage[latin1]{inputenc}
\usepackage{amsmath}
\usepackage{eurosym}
\input{pgfplots-macros}

\pgfqkeys{/codeexample}{%
	every codeexample/.style={
		width=8cm,
		/pgfplots/every axis/.append style={legend style={fill=graphicbackground}}
	},
	tabsize=4,
}
\pgfplotsset{
	every axis/.append style={width=7cm},
	filter discard warning=false,
}

\usetikzlibrary{backgrounds,patterns}
% Global styles:
\tikzset{
  shape example/.style={
    color=black!30,
    draw,
    fill=yellow!30,
    line width=.5cm,
    inner xsep=2.5cm,
    inner ysep=0.5cm}
}

\newcommand{\FIXME}[1]{\textcolor{red}{(FIXME: #1)}}

% fuer endvironment 'sidewaysfigure' bspw
% \usepackage{rotating}

\newcommand\Tikz{Ti\textit kZ}
\newcommand\PGF{\textsc{pgf}}
\newcommand\PGFPlots{\textsc{pgfplots}}
\def\PGFPlotstable{\textsc{PgfplotsTable}}


\makeindex

% Fix overful hboxes automatically:
\tolerance=2000
\emergencystretch=10pt

\tikzset{prefix=gnuplot/pgfplots_} % prefix for 'plot function'

\author{%
	Christian Feuers\"anger\footnote{\url{http://wissrech.ins.uni-bonn.de/people/feuersaenger}}\\%
	Institut f\"ur Numerische Simulation\\
	Universit\"at Bonn, Germany}


%\RequirePackage[german,english,francais]{babel}

\pgfplotsmanualexternalexpensivetrue

%\usetikzlibrary{external}
%\tikzexternalize[prefix=figures/]{pgfplots}

\title{%
	Manual for Package \PGFPlots\\
	{\small Version \pgfplotsversion}\\
	{\small\href{http://sourceforge.net/projects/pgfplots}{http://sourceforge.net/projects/pgfplots}}}

%\includeonly{pgfplots.libs}


\begin{document}

\def\plotcoords{%
\addplot coordinates {
(5,8.312e-02)    (17,2.547e-02)   (49,7.407e-03)
(129,2.102e-03)  (321,5.874e-04)  (769,1.623e-04)
(1793,4.442e-05) (4097,1.207e-05) (9217,3.261e-06)
};

\addplot coordinates{
(7,8.472e-02)    (31,3.044e-02)    (111,1.022e-02)
(351,3.303e-03)  (1023,1.039e-03)  (2815,3.196e-04)
(7423,9.658e-05) (18943,2.873e-05) (47103,8.437e-06)
};

\addplot coordinates{
(9,7.881e-02)     (49,3.243e-02)    (209,1.232e-02)
(769,4.454e-03)   (2561,1.551e-03)  (7937,5.236e-04)
(23297,1.723e-04) (65537,5.545e-05) (178177,1.751e-05)
};

\addplot coordinates{
(11,6.887e-02)    (71,3.177e-02)     (351,1.341e-02)
(1471,5.334e-03)  (5503,2.027e-03)   (18943,7.415e-04)
(61183,2.628e-04) (187903,9.063e-05) (553983,3.053e-05)
};

\addplot coordinates{
(13,5.755e-02)     (97,2.925e-02)     (545,1.351e-02)
(2561,5.842e-03)   (10625,2.397e-03)  (40193,9.414e-04)
(141569,3.564e-04) (471041,1.308e-04) 
(1496065,4.670e-05)
};
}%
\maketitle
\begin{abstract}%
\PGFPlots\ draws high--quality function plots in normal or logarithmic scaling with a user-friendly interface directly in \TeX. The user supplies axis labels, legend entries and the plot coordinates for one or more plots and \PGFPlots\ applies axis scaling, computes any logarithms and axis ticks and draws the plots, supporting line plots, scatter plots, piecewise constant plots, bar plots, area plots, mesh-- and surface plots and some more. It is based on Till Tantau's package \PGF/\Tikz.
\end{abstract}
\tableofcontents
\section{Introduction}
This package provides tools to generate plots and labeled axes easily. It draws normal plots, logplots and semi-logplots, in two and three dimensions. Axis ticks, labels, legends (in case of multiple plots) can be added with key-value options. It can cycle through a set of predefined line/marker/color specifications. In summary, its purpose is to simplify the generation of high-quality function and/or data plots, and solving the problems of
\begin{itemize}
	\item consistency of document and font type and font size,
	\item direct use of \TeX\ math mode in axis descriptions,
	\item consistency of data and figures (no third party tool necessary),
	\item inter document consistency using preamble configurations and styles.
\end{itemize}
Although not necessary, separate |.pdf| or |.eps| graphics can be generated using the |external| library developed as part of \Tikz.

\PGFPlots\ is build completely on \Tikz/\PGF. Knowledge of \Tikz\ will simplify the work with \PGFPlots, although it is not required.

\section[About PGFPlots: Preliminaries]{About {\normalfont\PGFPlots}: Preliminaries}
This section contains information about upgrades, the team, the installation (in case you need to do it manually) and troubleshooting. You may skip it completely except for the upgrade remarks.

However, note that this library requires at least \PGF\ version $2.00$. At the time of this writing, many \TeX-distributions still contain the older \PGF\ version $1.18$, so it may be necessary to install a recent \PGF\ prior to using \PGFPlots.

\subsection{Components}
\PGFPlots\ comes with two components:
\begin{enumerate}
	\item the plotting component (which you are currently reading) and
	\item the \PGFPlotstable\ component which simplifies number formatting and postprocessing of numerical tables. It comes as a separate package and has its own manual \href{file:pgfplotstable.pdf}{pgfplotstable.pdf}.
\end{enumerate}

\subsection{Upgrade remarks}
This release provides a lot of improvements which can be found in all detail in \texttt{ChangeLog} for interested readers. However, some attention is useful with respect to the following changes.

\subsubsection{New Optional Features}
\begin{enumerate}
	\item \PGFPlots\ 1.3 comes with user interface improvements. The technical distinction between ``behavior options'' and ``style options'' of older versions is no longer necessary (although still fully supported).

	\item \PGFPlots\ 1.3 has a new feature which allows to \emph{move axis labels tight to tick labels} automatically.

	Since this affects the spacing, it is not enabled be default.

	Use
\begin{codeexample}[code only]
\usepackage{pgfplots}
\pgfplotsset{compat=1.3}
\end{codeexample}
	in your preamble to benefit from the improved spacing\footnote{The |compat=1.3| setting is necessary for new features which move axis labels around like reversed axis. However, \PGFPlots\ will still work as it does in earlier versions, your documents will remain the same if you don't set it explicitly.}.
	Take a look at the next page for the precise definition of |compat|.

	\item \PGFPlots\ 1.3 now supports reversed axes. It is no longer necessary to use work--arounds with negative units.
\pgfkeys{/pdflinks/search key prefixes in/.add={/pgfplots/,}{}}

	Take a look at the |x dir=reverse| key.

	Existing work--arounds will still function properly. Use |\pgfplotsset{compat=1.3}| together with |x dir=reverse| to switch to the new version.
\end{enumerate}

\subsubsection{Old Features Which May Need Attention}
\begin{enumerate}
	\item The |scatter/classes| feature produces proper legends as of version 1.3. This may change the appearance of existing legends of plots with |scatter/classes|.

	\item Due to a small math bug in \PGF\ $2.00$, you \emph{can't} use the math expression `|-x^2|'. It is necessary to use `|0-x^2|' instead. The same holds for
		`|exp(-x^2)|'; use `|exp(0-x^2)|' instead.

		This will be fixed with \PGF\ $>2.00$.

	\item Starting with \PGFPlots\ $1.1$, |\tikzstyle| should \emph{no longer be used} to set \PGFPlots\ options.
	
	Although |\tikzstyle| is still supported for some older \PGFPlots\ options, you should replace any occurance of |\tikzstyle| with |\pgfplotsset{|\meta{style name}|/.style={|\meta{key-value-list}|}}| or the associated |/.append style| variant. See section~\ref{sec:styles} for more detail.
\end{enumerate}
I apologize for any inconvenience caused by these changes.

\begin{pgfplotskey}{compat=\mchoice{1.3,pre 1.3,default,newest} (initially default)}
	The preamble configuration 
\begin{codeexample}[code only]
\usepackage{pgfplots}
\pgfplotsset{compat=1.3}
\end{codeexample}
	allows to choose between backwards compatibility and most recent features.

	\PGFPlots\ version 1.3 comes with new features which may lead to different spacings compared to earlier versions:
	\begin{enumerate}
		\item Version 1.3 comes with the |xlabel near ticks| feature which places (in this case $x$) axis labels ``near'' the tick labels, taking the size of tick labels into account.
		
		Older versions used |xlabel absolute|, i.e.\ an absolute distance between the axis and axis labels (now matter how large or small tick labels are).

		The initial setting \emph{keeps backwards compatibility}.

		You are encouraged to use
\begin{codeexample}[code only]
\usepackage{pgfplots}
\pgfplotsset{compat=1.3}
\end{codeexample}
		in your preamble.
		
		\item Version 1.3 fixes a bug with |anchor=south west| which occurred mainly for reversed axes: the anchors where upside--down. This is fixed now.

		
		If you have written some work--around and want to keep it going, use

		|\pgfplotsset{compat/anchors=pre 1.3}|\pgfmanualpdflabel{/pgfplots/compat/anchors}{} 

		to disable the bug fix.
	\end{enumerate}

	Use |\pgfplotsset{compat=default}| to restore the factory settings.

	The setting |\pgfplotsset{compat=newest}| will always use features of the most recent version. This might result in small changes in the document's appearance.
\end{pgfplotskey}

\subsection{The Team}
\PGFPlots\ has been written mainly by Christian Feuers�nger with many improvements of Pascal Wolkotte and Nick Papior Andersen as a spare time project. We hope it is useful and provides valuable plots.

If you are interested in writing something but don't know how, consider reading the auxiliary manual \href{file:TeX-programming-notes.pdf}{TeX-programming-notes.pdf} which comes with \PGFPlots. It is far from complete, but maybe it is a good starting point (at least for more literature).

\subsection{Acknowledgements}
I thank God for all hours of enjoyed programming. I thank Pascal Wolkotte and Nick Papior Andersen for their programming efforts and contributions as part of the development team. I thank Stefan Tibus, who contributed the |plot shell| feature. I thank Tom Cashman for the contribution of the |reverse legend| feature. Special thanks go to Stefan Pinnow whose tests of \PGFPlots\ lead to numerous quality improvements. Furthermore, I thank Dr.~Schweitzer for many fruitful discussions and Dr.~Meine for his ideas and suggestions. Thanks as well to the many international contributors who provided feature requests or identified bugs or simply manual improvements!

Last but not least, I thank Till Tantau and Mark Wibrow for their excellent graphics (and more) package \PGF\ and \Tikz\ which is the base of \PGFPlots.

%%%%%%%%%%%%%%%%%%%%%%%%%%%%%%%%%%%%%%%%%%%%%%%%%%%%%%%%%%%%%%%%%%%%%%%%%%%%%
%
% Package pgfplots.sty documentation. 
%
% Copyright 2007/2008 by Christian Feuersaenger.
%
% This program is free software: you can redistribute it and/or modify
% it under the terms of the GNU General Public License as published by
% the Free Software Foundation, either version 3 of the License, or
% (at your option) any later version.
% 
% This program is distributed in the hope that it will be useful,
% but WITHOUT ANY WARRANTY; without even the implied warranty of
% MERCHANTABILITY or FITNESS FOR A PARTICULAR PURPOSE.  See the
% GNU General Public License for more details.
% 
% You should have received a copy of the GNU General Public License
% along with this program.  If not, see <http://www.gnu.org/licenses/>.
%
%
%%%%%%%%%%%%%%%%%%%%%%%%%%%%%%%%%%%%%%%%%%%%%%%%%%%%%%%%%%%%%%%%%%%%%%%%%%%%%
\input{pgfplots.preamble.tex}
%\RequirePackage[german,english,francais]{babel}

\pgfplotsmanualexternalexpensivetrue

%\usetikzlibrary{external}
%\tikzexternalize[prefix=figures/]{pgfplots}

\title{%
	Manual for Package \PGFPlots\\
	{\small Version \pgfplotsversion}\\
	{\small\href{http://sourceforge.net/projects/pgfplots}{http://sourceforge.net/projects/pgfplots}}}

%\includeonly{pgfplots.libs}


\begin{document}

\def\plotcoords{%
\addplot coordinates {
(5,8.312e-02)    (17,2.547e-02)   (49,7.407e-03)
(129,2.102e-03)  (321,5.874e-04)  (769,1.623e-04)
(1793,4.442e-05) (4097,1.207e-05) (9217,3.261e-06)
};

\addplot coordinates{
(7,8.472e-02)    (31,3.044e-02)    (111,1.022e-02)
(351,3.303e-03)  (1023,1.039e-03)  (2815,3.196e-04)
(7423,9.658e-05) (18943,2.873e-05) (47103,8.437e-06)
};

\addplot coordinates{
(9,7.881e-02)     (49,3.243e-02)    (209,1.232e-02)
(769,4.454e-03)   (2561,1.551e-03)  (7937,5.236e-04)
(23297,1.723e-04) (65537,5.545e-05) (178177,1.751e-05)
};

\addplot coordinates{
(11,6.887e-02)    (71,3.177e-02)     (351,1.341e-02)
(1471,5.334e-03)  (5503,2.027e-03)   (18943,7.415e-04)
(61183,2.628e-04) (187903,9.063e-05) (553983,3.053e-05)
};

\addplot coordinates{
(13,5.755e-02)     (97,2.925e-02)     (545,1.351e-02)
(2561,5.842e-03)   (10625,2.397e-03)  (40193,9.414e-04)
(141569,3.564e-04) (471041,1.308e-04) 
(1496065,4.670e-05)
};
}%
\maketitle
\begin{abstract}%
\PGFPlots\ draws high--quality function plots in normal or logarithmic scaling with a user-friendly interface directly in \TeX. The user supplies axis labels, legend entries and the plot coordinates for one or more plots and \PGFPlots\ applies axis scaling, computes any logarithms and axis ticks and draws the plots, supporting line plots, scatter plots, piecewise constant plots, bar plots, area plots, mesh-- and surface plots and some more. It is based on Till Tantau's package \PGF/\Tikz.
\end{abstract}
\tableofcontents
\section{Introduction}
This package provides tools to generate plots and labeled axes easily. It draws normal plots, logplots and semi-logplots, in two and three dimensions. Axis ticks, labels, legends (in case of multiple plots) can be added with key-value options. It can cycle through a set of predefined line/marker/color specifications. In summary, its purpose is to simplify the generation of high-quality function and/or data plots, and solving the problems of
\begin{itemize}
	\item consistency of document and font type and font size,
	\item direct use of \TeX\ math mode in axis descriptions,
	\item consistency of data and figures (no third party tool necessary),
	\item inter document consistency using preamble configurations and styles.
\end{itemize}
Although not necessary, separate |.pdf| or |.eps| graphics can be generated using the |external| library developed as part of \Tikz.

\PGFPlots\ is build completely on \Tikz/\PGF. Knowledge of \Tikz\ will simplify the work with \PGFPlots, although it is not required.

\section[About PGFPlots: Preliminaries]{About {\normalfont\PGFPlots}: Preliminaries}
This section contains information about upgrades, the team, the installation (in case you need to do it manually) and troubleshooting. You may skip it completely except for the upgrade remarks.

However, note that this library requires at least \PGF\ version $2.00$. At the time of this writing, many \TeX-distributions still contain the older \PGF\ version $1.18$, so it may be necessary to install a recent \PGF\ prior to using \PGFPlots.

\subsection{Components}
\PGFPlots\ comes with two components:
\begin{enumerate}
	\item the plotting component (which you are currently reading) and
	\item the \PGFPlotstable\ component which simplifies number formatting and postprocessing of numerical tables. It comes as a separate package and has its own manual \href{file:pgfplotstable.pdf}{pgfplotstable.pdf}.
\end{enumerate}

\subsection{Upgrade remarks}
This release provides a lot of improvements which can be found in all detail in \texttt{ChangeLog} for interested readers. However, some attention is useful with respect to the following changes.

\subsubsection{New Optional Features}
\begin{enumerate}
	\item \PGFPlots\ 1.3 comes with user interface improvements. The technical distinction between ``behavior options'' and ``style options'' of older versions is no longer necessary (although still fully supported).

	\item \PGFPlots\ 1.3 has a new feature which allows to \emph{move axis labels tight to tick labels} automatically.

	Since this affects the spacing, it is not enabled be default.

	Use
\begin{codeexample}[code only]
\usepackage{pgfplots}
\pgfplotsset{compat=1.3}
\end{codeexample}
	in your preamble to benefit from the improved spacing\footnote{The |compat=1.3| setting is necessary for new features which move axis labels around like reversed axis. However, \PGFPlots\ will still work as it does in earlier versions, your documents will remain the same if you don't set it explicitly.}.
	Take a look at the next page for the precise definition of |compat|.

	\item \PGFPlots\ 1.3 now supports reversed axes. It is no longer necessary to use work--arounds with negative units.
\pgfkeys{/pdflinks/search key prefixes in/.add={/pgfplots/,}{}}

	Take a look at the |x dir=reverse| key.

	Existing work--arounds will still function properly. Use |\pgfplotsset{compat=1.3}| together with |x dir=reverse| to switch to the new version.
\end{enumerate}

\subsubsection{Old Features Which May Need Attention}
\begin{enumerate}
	\item The |scatter/classes| feature produces proper legends as of version 1.3. This may change the appearance of existing legends of plots with |scatter/classes|.

	\item Due to a small math bug in \PGF\ $2.00$, you \emph{can't} use the math expression `|-x^2|'. It is necessary to use `|0-x^2|' instead. The same holds for
		`|exp(-x^2)|'; use `|exp(0-x^2)|' instead.

		This will be fixed with \PGF\ $>2.00$.

	\item Starting with \PGFPlots\ $1.1$, |\tikzstyle| should \emph{no longer be used} to set \PGFPlots\ options.
	
	Although |\tikzstyle| is still supported for some older \PGFPlots\ options, you should replace any occurance of |\tikzstyle| with |\pgfplotsset{|\meta{style name}|/.style={|\meta{key-value-list}|}}| or the associated |/.append style| variant. See section~\ref{sec:styles} for more detail.
\end{enumerate}
I apologize for any inconvenience caused by these changes.

\begin{pgfplotskey}{compat=\mchoice{1.3,pre 1.3,default,newest} (initially default)}
	The preamble configuration 
\begin{codeexample}[code only]
\usepackage{pgfplots}
\pgfplotsset{compat=1.3}
\end{codeexample}
	allows to choose between backwards compatibility and most recent features.

	\PGFPlots\ version 1.3 comes with new features which may lead to different spacings compared to earlier versions:
	\begin{enumerate}
		\item Version 1.3 comes with the |xlabel near ticks| feature which places (in this case $x$) axis labels ``near'' the tick labels, taking the size of tick labels into account.
		
		Older versions used |xlabel absolute|, i.e.\ an absolute distance between the axis and axis labels (now matter how large or small tick labels are).

		The initial setting \emph{keeps backwards compatibility}.

		You are encouraged to use
\begin{codeexample}[code only]
\usepackage{pgfplots}
\pgfplotsset{compat=1.3}
\end{codeexample}
		in your preamble.
		
		\item Version 1.3 fixes a bug with |anchor=south west| which occurred mainly for reversed axes: the anchors where upside--down. This is fixed now.

		
		If you have written some work--around and want to keep it going, use

		|\pgfplotsset{compat/anchors=pre 1.3}|\pgfmanualpdflabel{/pgfplots/compat/anchors}{} 

		to disable the bug fix.
	\end{enumerate}

	Use |\pgfplotsset{compat=default}| to restore the factory settings.

	The setting |\pgfplotsset{compat=newest}| will always use features of the most recent version. This might result in small changes in the document's appearance.
\end{pgfplotskey}

\subsection{The Team}
\PGFPlots\ has been written mainly by Christian Feuers�nger with many improvements of Pascal Wolkotte and Nick Papior Andersen as a spare time project. We hope it is useful and provides valuable plots.

If you are interested in writing something but don't know how, consider reading the auxiliary manual \href{file:TeX-programming-notes.pdf}{TeX-programming-notes.pdf} which comes with \PGFPlots. It is far from complete, but maybe it is a good starting point (at least for more literature).

\subsection{Acknowledgements}
I thank God for all hours of enjoyed programming. I thank Pascal Wolkotte and Nick Papior Andersen for their programming efforts and contributions as part of the development team. I thank Stefan Tibus, who contributed the |plot shell| feature. I thank Tom Cashman for the contribution of the |reverse legend| feature. Special thanks go to Stefan Pinnow whose tests of \PGFPlots\ lead to numerous quality improvements. Furthermore, I thank Dr.~Schweitzer for many fruitful discussions and Dr.~Meine for his ideas and suggestions. Thanks as well to the many international contributors who provided feature requests or identified bugs or simply manual improvements!

Last but not least, I thank Till Tantau and Mark Wibrow for their excellent graphics (and more) package \PGF\ and \Tikz\ which is the base of \PGFPlots.

\include{pgfplots.install}%
\include{pgfplots.intro}%
\include{pgfplots.reference}%
\include{pgfplots.libs}%
\include{pgfplots.resources}%
\include{pgfplots.importexport}%
\include{pgfplots.basic.reference}%

\printindex

\bibliographystyle{abbrv} %gerapali} %gerabbrv} %gerunsrt.bst} %gerabbrv}% gerplain}
\nocite{pgfplotstable}
\nocite{programmingnotes}
\bibliography{pgfplots}
\end{document}
%
%%%%%%%%%%%%%%%%%%%%%%%%%%%%%%%%%%%%%%%%%%%%%%%%%%%%%%%%%%%%%%%%%%%%%%%%%%%%%
%
% Package pgfplots.sty documentation. 
%
% Copyright 2007/2008 by Christian Feuersaenger.
%
% This program is free software: you can redistribute it and/or modify
% it under the terms of the GNU General Public License as published by
% the Free Software Foundation, either version 3 of the License, or
% (at your option) any later version.
% 
% This program is distributed in the hope that it will be useful,
% but WITHOUT ANY WARRANTY; without even the implied warranty of
% MERCHANTABILITY or FITNESS FOR A PARTICULAR PURPOSE.  See the
% GNU General Public License for more details.
% 
% You should have received a copy of the GNU General Public License
% along with this program.  If not, see <http://www.gnu.org/licenses/>.
%
%
%%%%%%%%%%%%%%%%%%%%%%%%%%%%%%%%%%%%%%%%%%%%%%%%%%%%%%%%%%%%%%%%%%%%%%%%%%%%%
\input{pgfplots.preamble.tex}
%\RequirePackage[german,english,francais]{babel}

\pgfplotsmanualexternalexpensivetrue

%\usetikzlibrary{external}
%\tikzexternalize[prefix=figures/]{pgfplots}

\title{%
	Manual for Package \PGFPlots\\
	{\small Version \pgfplotsversion}\\
	{\small\href{http://sourceforge.net/projects/pgfplots}{http://sourceforge.net/projects/pgfplots}}}

%\includeonly{pgfplots.libs}


\begin{document}

\def\plotcoords{%
\addplot coordinates {
(5,8.312e-02)    (17,2.547e-02)   (49,7.407e-03)
(129,2.102e-03)  (321,5.874e-04)  (769,1.623e-04)
(1793,4.442e-05) (4097,1.207e-05) (9217,3.261e-06)
};

\addplot coordinates{
(7,8.472e-02)    (31,3.044e-02)    (111,1.022e-02)
(351,3.303e-03)  (1023,1.039e-03)  (2815,3.196e-04)
(7423,9.658e-05) (18943,2.873e-05) (47103,8.437e-06)
};

\addplot coordinates{
(9,7.881e-02)     (49,3.243e-02)    (209,1.232e-02)
(769,4.454e-03)   (2561,1.551e-03)  (7937,5.236e-04)
(23297,1.723e-04) (65537,5.545e-05) (178177,1.751e-05)
};

\addplot coordinates{
(11,6.887e-02)    (71,3.177e-02)     (351,1.341e-02)
(1471,5.334e-03)  (5503,2.027e-03)   (18943,7.415e-04)
(61183,2.628e-04) (187903,9.063e-05) (553983,3.053e-05)
};

\addplot coordinates{
(13,5.755e-02)     (97,2.925e-02)     (545,1.351e-02)
(2561,5.842e-03)   (10625,2.397e-03)  (40193,9.414e-04)
(141569,3.564e-04) (471041,1.308e-04) 
(1496065,4.670e-05)
};
}%
\maketitle
\begin{abstract}%
\PGFPlots\ draws high--quality function plots in normal or logarithmic scaling with a user-friendly interface directly in \TeX. The user supplies axis labels, legend entries and the plot coordinates for one or more plots and \PGFPlots\ applies axis scaling, computes any logarithms and axis ticks and draws the plots, supporting line plots, scatter plots, piecewise constant plots, bar plots, area plots, mesh-- and surface plots and some more. It is based on Till Tantau's package \PGF/\Tikz.
\end{abstract}
\tableofcontents
\section{Introduction}
This package provides tools to generate plots and labeled axes easily. It draws normal plots, logplots and semi-logplots, in two and three dimensions. Axis ticks, labels, legends (in case of multiple plots) can be added with key-value options. It can cycle through a set of predefined line/marker/color specifications. In summary, its purpose is to simplify the generation of high-quality function and/or data plots, and solving the problems of
\begin{itemize}
	\item consistency of document and font type and font size,
	\item direct use of \TeX\ math mode in axis descriptions,
	\item consistency of data and figures (no third party tool necessary),
	\item inter document consistency using preamble configurations and styles.
\end{itemize}
Although not necessary, separate |.pdf| or |.eps| graphics can be generated using the |external| library developed as part of \Tikz.

\PGFPlots\ is build completely on \Tikz/\PGF. Knowledge of \Tikz\ will simplify the work with \PGFPlots, although it is not required.

\section[About PGFPlots: Preliminaries]{About {\normalfont\PGFPlots}: Preliminaries}
This section contains information about upgrades, the team, the installation (in case you need to do it manually) and troubleshooting. You may skip it completely except for the upgrade remarks.

However, note that this library requires at least \PGF\ version $2.00$. At the time of this writing, many \TeX-distributions still contain the older \PGF\ version $1.18$, so it may be necessary to install a recent \PGF\ prior to using \PGFPlots.

\subsection{Components}
\PGFPlots\ comes with two components:
\begin{enumerate}
	\item the plotting component (which you are currently reading) and
	\item the \PGFPlotstable\ component which simplifies number formatting and postprocessing of numerical tables. It comes as a separate package and has its own manual \href{file:pgfplotstable.pdf}{pgfplotstable.pdf}.
\end{enumerate}

\subsection{Upgrade remarks}
This release provides a lot of improvements which can be found in all detail in \texttt{ChangeLog} for interested readers. However, some attention is useful with respect to the following changes.

\subsubsection{New Optional Features}
\begin{enumerate}
	\item \PGFPlots\ 1.3 comes with user interface improvements. The technical distinction between ``behavior options'' and ``style options'' of older versions is no longer necessary (although still fully supported).

	\item \PGFPlots\ 1.3 has a new feature which allows to \emph{move axis labels tight to tick labels} automatically.

	Since this affects the spacing, it is not enabled be default.

	Use
\begin{codeexample}[code only]
\usepackage{pgfplots}
\pgfplotsset{compat=1.3}
\end{codeexample}
	in your preamble to benefit from the improved spacing\footnote{The |compat=1.3| setting is necessary for new features which move axis labels around like reversed axis. However, \PGFPlots\ will still work as it does in earlier versions, your documents will remain the same if you don't set it explicitly.}.
	Take a look at the next page for the precise definition of |compat|.

	\item \PGFPlots\ 1.3 now supports reversed axes. It is no longer necessary to use work--arounds with negative units.
\pgfkeys{/pdflinks/search key prefixes in/.add={/pgfplots/,}{}}

	Take a look at the |x dir=reverse| key.

	Existing work--arounds will still function properly. Use |\pgfplotsset{compat=1.3}| together with |x dir=reverse| to switch to the new version.
\end{enumerate}

\subsubsection{Old Features Which May Need Attention}
\begin{enumerate}
	\item The |scatter/classes| feature produces proper legends as of version 1.3. This may change the appearance of existing legends of plots with |scatter/classes|.

	\item Due to a small math bug in \PGF\ $2.00$, you \emph{can't} use the math expression `|-x^2|'. It is necessary to use `|0-x^2|' instead. The same holds for
		`|exp(-x^2)|'; use `|exp(0-x^2)|' instead.

		This will be fixed with \PGF\ $>2.00$.

	\item Starting with \PGFPlots\ $1.1$, |\tikzstyle| should \emph{no longer be used} to set \PGFPlots\ options.
	
	Although |\tikzstyle| is still supported for some older \PGFPlots\ options, you should replace any occurance of |\tikzstyle| with |\pgfplotsset{|\meta{style name}|/.style={|\meta{key-value-list}|}}| or the associated |/.append style| variant. See section~\ref{sec:styles} for more detail.
\end{enumerate}
I apologize for any inconvenience caused by these changes.

\begin{pgfplotskey}{compat=\mchoice{1.3,pre 1.3,default,newest} (initially default)}
	The preamble configuration 
\begin{codeexample}[code only]
\usepackage{pgfplots}
\pgfplotsset{compat=1.3}
\end{codeexample}
	allows to choose between backwards compatibility and most recent features.

	\PGFPlots\ version 1.3 comes with new features which may lead to different spacings compared to earlier versions:
	\begin{enumerate}
		\item Version 1.3 comes with the |xlabel near ticks| feature which places (in this case $x$) axis labels ``near'' the tick labels, taking the size of tick labels into account.
		
		Older versions used |xlabel absolute|, i.e.\ an absolute distance between the axis and axis labels (now matter how large or small tick labels are).

		The initial setting \emph{keeps backwards compatibility}.

		You are encouraged to use
\begin{codeexample}[code only]
\usepackage{pgfplots}
\pgfplotsset{compat=1.3}
\end{codeexample}
		in your preamble.
		
		\item Version 1.3 fixes a bug with |anchor=south west| which occurred mainly for reversed axes: the anchors where upside--down. This is fixed now.

		
		If you have written some work--around and want to keep it going, use

		|\pgfplotsset{compat/anchors=pre 1.3}|\pgfmanualpdflabel{/pgfplots/compat/anchors}{} 

		to disable the bug fix.
	\end{enumerate}

	Use |\pgfplotsset{compat=default}| to restore the factory settings.

	The setting |\pgfplotsset{compat=newest}| will always use features of the most recent version. This might result in small changes in the document's appearance.
\end{pgfplotskey}

\subsection{The Team}
\PGFPlots\ has been written mainly by Christian Feuers�nger with many improvements of Pascal Wolkotte and Nick Papior Andersen as a spare time project. We hope it is useful and provides valuable plots.

If you are interested in writing something but don't know how, consider reading the auxiliary manual \href{file:TeX-programming-notes.pdf}{TeX-programming-notes.pdf} which comes with \PGFPlots. It is far from complete, but maybe it is a good starting point (at least for more literature).

\subsection{Acknowledgements}
I thank God for all hours of enjoyed programming. I thank Pascal Wolkotte and Nick Papior Andersen for their programming efforts and contributions as part of the development team. I thank Stefan Tibus, who contributed the |plot shell| feature. I thank Tom Cashman for the contribution of the |reverse legend| feature. Special thanks go to Stefan Pinnow whose tests of \PGFPlots\ lead to numerous quality improvements. Furthermore, I thank Dr.~Schweitzer for many fruitful discussions and Dr.~Meine for his ideas and suggestions. Thanks as well to the many international contributors who provided feature requests or identified bugs or simply manual improvements!

Last but not least, I thank Till Tantau and Mark Wibrow for their excellent graphics (and more) package \PGF\ and \Tikz\ which is the base of \PGFPlots.

\include{pgfplots.install}%
\include{pgfplots.intro}%
\include{pgfplots.reference}%
\include{pgfplots.libs}%
\include{pgfplots.resources}%
\include{pgfplots.importexport}%
\include{pgfplots.basic.reference}%

\printindex

\bibliographystyle{abbrv} %gerapali} %gerabbrv} %gerunsrt.bst} %gerabbrv}% gerplain}
\nocite{pgfplotstable}
\nocite{programmingnotes}
\bibliography{pgfplots}
\end{document}
%
%%%%%%%%%%%%%%%%%%%%%%%%%%%%%%%%%%%%%%%%%%%%%%%%%%%%%%%%%%%%%%%%%%%%%%%%%%%%%
%
% Package pgfplots.sty documentation. 
%
% Copyright 2007/2008 by Christian Feuersaenger.
%
% This program is free software: you can redistribute it and/or modify
% it under the terms of the GNU General Public License as published by
% the Free Software Foundation, either version 3 of the License, or
% (at your option) any later version.
% 
% This program is distributed in the hope that it will be useful,
% but WITHOUT ANY WARRANTY; without even the implied warranty of
% MERCHANTABILITY or FITNESS FOR A PARTICULAR PURPOSE.  See the
% GNU General Public License for more details.
% 
% You should have received a copy of the GNU General Public License
% along with this program.  If not, see <http://www.gnu.org/licenses/>.
%
%
%%%%%%%%%%%%%%%%%%%%%%%%%%%%%%%%%%%%%%%%%%%%%%%%%%%%%%%%%%%%%%%%%%%%%%%%%%%%%
\input{pgfplots.preamble.tex}
%\RequirePackage[german,english,francais]{babel}

\pgfplotsmanualexternalexpensivetrue

%\usetikzlibrary{external}
%\tikzexternalize[prefix=figures/]{pgfplots}

\title{%
	Manual for Package \PGFPlots\\
	{\small Version \pgfplotsversion}\\
	{\small\href{http://sourceforge.net/projects/pgfplots}{http://sourceforge.net/projects/pgfplots}}}

%\includeonly{pgfplots.libs}


\begin{document}

\def\plotcoords{%
\addplot coordinates {
(5,8.312e-02)    (17,2.547e-02)   (49,7.407e-03)
(129,2.102e-03)  (321,5.874e-04)  (769,1.623e-04)
(1793,4.442e-05) (4097,1.207e-05) (9217,3.261e-06)
};

\addplot coordinates{
(7,8.472e-02)    (31,3.044e-02)    (111,1.022e-02)
(351,3.303e-03)  (1023,1.039e-03)  (2815,3.196e-04)
(7423,9.658e-05) (18943,2.873e-05) (47103,8.437e-06)
};

\addplot coordinates{
(9,7.881e-02)     (49,3.243e-02)    (209,1.232e-02)
(769,4.454e-03)   (2561,1.551e-03)  (7937,5.236e-04)
(23297,1.723e-04) (65537,5.545e-05) (178177,1.751e-05)
};

\addplot coordinates{
(11,6.887e-02)    (71,3.177e-02)     (351,1.341e-02)
(1471,5.334e-03)  (5503,2.027e-03)   (18943,7.415e-04)
(61183,2.628e-04) (187903,9.063e-05) (553983,3.053e-05)
};

\addplot coordinates{
(13,5.755e-02)     (97,2.925e-02)     (545,1.351e-02)
(2561,5.842e-03)   (10625,2.397e-03)  (40193,9.414e-04)
(141569,3.564e-04) (471041,1.308e-04) 
(1496065,4.670e-05)
};
}%
\maketitle
\begin{abstract}%
\PGFPlots\ draws high--quality function plots in normal or logarithmic scaling with a user-friendly interface directly in \TeX. The user supplies axis labels, legend entries and the plot coordinates for one or more plots and \PGFPlots\ applies axis scaling, computes any logarithms and axis ticks and draws the plots, supporting line plots, scatter plots, piecewise constant plots, bar plots, area plots, mesh-- and surface plots and some more. It is based on Till Tantau's package \PGF/\Tikz.
\end{abstract}
\tableofcontents
\section{Introduction}
This package provides tools to generate plots and labeled axes easily. It draws normal plots, logplots and semi-logplots, in two and three dimensions. Axis ticks, labels, legends (in case of multiple plots) can be added with key-value options. It can cycle through a set of predefined line/marker/color specifications. In summary, its purpose is to simplify the generation of high-quality function and/or data plots, and solving the problems of
\begin{itemize}
	\item consistency of document and font type and font size,
	\item direct use of \TeX\ math mode in axis descriptions,
	\item consistency of data and figures (no third party tool necessary),
	\item inter document consistency using preamble configurations and styles.
\end{itemize}
Although not necessary, separate |.pdf| or |.eps| graphics can be generated using the |external| library developed as part of \Tikz.

\PGFPlots\ is build completely on \Tikz/\PGF. Knowledge of \Tikz\ will simplify the work with \PGFPlots, although it is not required.

\section[About PGFPlots: Preliminaries]{About {\normalfont\PGFPlots}: Preliminaries}
This section contains information about upgrades, the team, the installation (in case you need to do it manually) and troubleshooting. You may skip it completely except for the upgrade remarks.

However, note that this library requires at least \PGF\ version $2.00$. At the time of this writing, many \TeX-distributions still contain the older \PGF\ version $1.18$, so it may be necessary to install a recent \PGF\ prior to using \PGFPlots.

\subsection{Components}
\PGFPlots\ comes with two components:
\begin{enumerate}
	\item the plotting component (which you are currently reading) and
	\item the \PGFPlotstable\ component which simplifies number formatting and postprocessing of numerical tables. It comes as a separate package and has its own manual \href{file:pgfplotstable.pdf}{pgfplotstable.pdf}.
\end{enumerate}

\subsection{Upgrade remarks}
This release provides a lot of improvements which can be found in all detail in \texttt{ChangeLog} for interested readers. However, some attention is useful with respect to the following changes.

\subsubsection{New Optional Features}
\begin{enumerate}
	\item \PGFPlots\ 1.3 comes with user interface improvements. The technical distinction between ``behavior options'' and ``style options'' of older versions is no longer necessary (although still fully supported).

	\item \PGFPlots\ 1.3 has a new feature which allows to \emph{move axis labels tight to tick labels} automatically.

	Since this affects the spacing, it is not enabled be default.

	Use
\begin{codeexample}[code only]
\usepackage{pgfplots}
\pgfplotsset{compat=1.3}
\end{codeexample}
	in your preamble to benefit from the improved spacing\footnote{The |compat=1.3| setting is necessary for new features which move axis labels around like reversed axis. However, \PGFPlots\ will still work as it does in earlier versions, your documents will remain the same if you don't set it explicitly.}.
	Take a look at the next page for the precise definition of |compat|.

	\item \PGFPlots\ 1.3 now supports reversed axes. It is no longer necessary to use work--arounds with negative units.
\pgfkeys{/pdflinks/search key prefixes in/.add={/pgfplots/,}{}}

	Take a look at the |x dir=reverse| key.

	Existing work--arounds will still function properly. Use |\pgfplotsset{compat=1.3}| together with |x dir=reverse| to switch to the new version.
\end{enumerate}

\subsubsection{Old Features Which May Need Attention}
\begin{enumerate}
	\item The |scatter/classes| feature produces proper legends as of version 1.3. This may change the appearance of existing legends of plots with |scatter/classes|.

	\item Due to a small math bug in \PGF\ $2.00$, you \emph{can't} use the math expression `|-x^2|'. It is necessary to use `|0-x^2|' instead. The same holds for
		`|exp(-x^2)|'; use `|exp(0-x^2)|' instead.

		This will be fixed with \PGF\ $>2.00$.

	\item Starting with \PGFPlots\ $1.1$, |\tikzstyle| should \emph{no longer be used} to set \PGFPlots\ options.
	
	Although |\tikzstyle| is still supported for some older \PGFPlots\ options, you should replace any occurance of |\tikzstyle| with |\pgfplotsset{|\meta{style name}|/.style={|\meta{key-value-list}|}}| or the associated |/.append style| variant. See section~\ref{sec:styles} for more detail.
\end{enumerate}
I apologize for any inconvenience caused by these changes.

\begin{pgfplotskey}{compat=\mchoice{1.3,pre 1.3,default,newest} (initially default)}
	The preamble configuration 
\begin{codeexample}[code only]
\usepackage{pgfplots}
\pgfplotsset{compat=1.3}
\end{codeexample}
	allows to choose between backwards compatibility and most recent features.

	\PGFPlots\ version 1.3 comes with new features which may lead to different spacings compared to earlier versions:
	\begin{enumerate}
		\item Version 1.3 comes with the |xlabel near ticks| feature which places (in this case $x$) axis labels ``near'' the tick labels, taking the size of tick labels into account.
		
		Older versions used |xlabel absolute|, i.e.\ an absolute distance between the axis and axis labels (now matter how large or small tick labels are).

		The initial setting \emph{keeps backwards compatibility}.

		You are encouraged to use
\begin{codeexample}[code only]
\usepackage{pgfplots}
\pgfplotsset{compat=1.3}
\end{codeexample}
		in your preamble.
		
		\item Version 1.3 fixes a bug with |anchor=south west| which occurred mainly for reversed axes: the anchors where upside--down. This is fixed now.

		
		If you have written some work--around and want to keep it going, use

		|\pgfplotsset{compat/anchors=pre 1.3}|\pgfmanualpdflabel{/pgfplots/compat/anchors}{} 

		to disable the bug fix.
	\end{enumerate}

	Use |\pgfplotsset{compat=default}| to restore the factory settings.

	The setting |\pgfplotsset{compat=newest}| will always use features of the most recent version. This might result in small changes in the document's appearance.
\end{pgfplotskey}

\subsection{The Team}
\PGFPlots\ has been written mainly by Christian Feuers�nger with many improvements of Pascal Wolkotte and Nick Papior Andersen as a spare time project. We hope it is useful and provides valuable plots.

If you are interested in writing something but don't know how, consider reading the auxiliary manual \href{file:TeX-programming-notes.pdf}{TeX-programming-notes.pdf} which comes with \PGFPlots. It is far from complete, but maybe it is a good starting point (at least for more literature).

\subsection{Acknowledgements}
I thank God for all hours of enjoyed programming. I thank Pascal Wolkotte and Nick Papior Andersen for their programming efforts and contributions as part of the development team. I thank Stefan Tibus, who contributed the |plot shell| feature. I thank Tom Cashman for the contribution of the |reverse legend| feature. Special thanks go to Stefan Pinnow whose tests of \PGFPlots\ lead to numerous quality improvements. Furthermore, I thank Dr.~Schweitzer for many fruitful discussions and Dr.~Meine for his ideas and suggestions. Thanks as well to the many international contributors who provided feature requests or identified bugs or simply manual improvements!

Last but not least, I thank Till Tantau and Mark Wibrow for their excellent graphics (and more) package \PGF\ and \Tikz\ which is the base of \PGFPlots.

\include{pgfplots.install}%
\include{pgfplots.intro}%
\include{pgfplots.reference}%
\include{pgfplots.libs}%
\include{pgfplots.resources}%
\include{pgfplots.importexport}%
\include{pgfplots.basic.reference}%

\printindex

\bibliographystyle{abbrv} %gerapali} %gerabbrv} %gerunsrt.bst} %gerabbrv}% gerplain}
\nocite{pgfplotstable}
\nocite{programmingnotes}
\bibliography{pgfplots}
\end{document}
%
%%%%%%%%%%%%%%%%%%%%%%%%%%%%%%%%%%%%%%%%%%%%%%%%%%%%%%%%%%%%%%%%%%%%%%%%%%%%%
%
% Package pgfplots.sty documentation. 
%
% Copyright 2007/2008 by Christian Feuersaenger.
%
% This program is free software: you can redistribute it and/or modify
% it under the terms of the GNU General Public License as published by
% the Free Software Foundation, either version 3 of the License, or
% (at your option) any later version.
% 
% This program is distributed in the hope that it will be useful,
% but WITHOUT ANY WARRANTY; without even the implied warranty of
% MERCHANTABILITY or FITNESS FOR A PARTICULAR PURPOSE.  See the
% GNU General Public License for more details.
% 
% You should have received a copy of the GNU General Public License
% along with this program.  If not, see <http://www.gnu.org/licenses/>.
%
%
%%%%%%%%%%%%%%%%%%%%%%%%%%%%%%%%%%%%%%%%%%%%%%%%%%%%%%%%%%%%%%%%%%%%%%%%%%%%%
\input{pgfplots.preamble.tex}
%\RequirePackage[german,english,francais]{babel}

\pgfplotsmanualexternalexpensivetrue

%\usetikzlibrary{external}
%\tikzexternalize[prefix=figures/]{pgfplots}

\title{%
	Manual for Package \PGFPlots\\
	{\small Version \pgfplotsversion}\\
	{\small\href{http://sourceforge.net/projects/pgfplots}{http://sourceforge.net/projects/pgfplots}}}

%\includeonly{pgfplots.libs}


\begin{document}

\def\plotcoords{%
\addplot coordinates {
(5,8.312e-02)    (17,2.547e-02)   (49,7.407e-03)
(129,2.102e-03)  (321,5.874e-04)  (769,1.623e-04)
(1793,4.442e-05) (4097,1.207e-05) (9217,3.261e-06)
};

\addplot coordinates{
(7,8.472e-02)    (31,3.044e-02)    (111,1.022e-02)
(351,3.303e-03)  (1023,1.039e-03)  (2815,3.196e-04)
(7423,9.658e-05) (18943,2.873e-05) (47103,8.437e-06)
};

\addplot coordinates{
(9,7.881e-02)     (49,3.243e-02)    (209,1.232e-02)
(769,4.454e-03)   (2561,1.551e-03)  (7937,5.236e-04)
(23297,1.723e-04) (65537,5.545e-05) (178177,1.751e-05)
};

\addplot coordinates{
(11,6.887e-02)    (71,3.177e-02)     (351,1.341e-02)
(1471,5.334e-03)  (5503,2.027e-03)   (18943,7.415e-04)
(61183,2.628e-04) (187903,9.063e-05) (553983,3.053e-05)
};

\addplot coordinates{
(13,5.755e-02)     (97,2.925e-02)     (545,1.351e-02)
(2561,5.842e-03)   (10625,2.397e-03)  (40193,9.414e-04)
(141569,3.564e-04) (471041,1.308e-04) 
(1496065,4.670e-05)
};
}%
\maketitle
\begin{abstract}%
\PGFPlots\ draws high--quality function plots in normal or logarithmic scaling with a user-friendly interface directly in \TeX. The user supplies axis labels, legend entries and the plot coordinates for one or more plots and \PGFPlots\ applies axis scaling, computes any logarithms and axis ticks and draws the plots, supporting line plots, scatter plots, piecewise constant plots, bar plots, area plots, mesh-- and surface plots and some more. It is based on Till Tantau's package \PGF/\Tikz.
\end{abstract}
\tableofcontents
\section{Introduction}
This package provides tools to generate plots and labeled axes easily. It draws normal plots, logplots and semi-logplots, in two and three dimensions. Axis ticks, labels, legends (in case of multiple plots) can be added with key-value options. It can cycle through a set of predefined line/marker/color specifications. In summary, its purpose is to simplify the generation of high-quality function and/or data plots, and solving the problems of
\begin{itemize}
	\item consistency of document and font type and font size,
	\item direct use of \TeX\ math mode in axis descriptions,
	\item consistency of data and figures (no third party tool necessary),
	\item inter document consistency using preamble configurations and styles.
\end{itemize}
Although not necessary, separate |.pdf| or |.eps| graphics can be generated using the |external| library developed as part of \Tikz.

\PGFPlots\ is build completely on \Tikz/\PGF. Knowledge of \Tikz\ will simplify the work with \PGFPlots, although it is not required.

\section[About PGFPlots: Preliminaries]{About {\normalfont\PGFPlots}: Preliminaries}
This section contains information about upgrades, the team, the installation (in case you need to do it manually) and troubleshooting. You may skip it completely except for the upgrade remarks.

However, note that this library requires at least \PGF\ version $2.00$. At the time of this writing, many \TeX-distributions still contain the older \PGF\ version $1.18$, so it may be necessary to install a recent \PGF\ prior to using \PGFPlots.

\subsection{Components}
\PGFPlots\ comes with two components:
\begin{enumerate}
	\item the plotting component (which you are currently reading) and
	\item the \PGFPlotstable\ component which simplifies number formatting and postprocessing of numerical tables. It comes as a separate package and has its own manual \href{file:pgfplotstable.pdf}{pgfplotstable.pdf}.
\end{enumerate}

\subsection{Upgrade remarks}
This release provides a lot of improvements which can be found in all detail in \texttt{ChangeLog} for interested readers. However, some attention is useful with respect to the following changes.

\subsubsection{New Optional Features}
\begin{enumerate}
	\item \PGFPlots\ 1.3 comes with user interface improvements. The technical distinction between ``behavior options'' and ``style options'' of older versions is no longer necessary (although still fully supported).

	\item \PGFPlots\ 1.3 has a new feature which allows to \emph{move axis labels tight to tick labels} automatically.

	Since this affects the spacing, it is not enabled be default.

	Use
\begin{codeexample}[code only]
\usepackage{pgfplots}
\pgfplotsset{compat=1.3}
\end{codeexample}
	in your preamble to benefit from the improved spacing\footnote{The |compat=1.3| setting is necessary for new features which move axis labels around like reversed axis. However, \PGFPlots\ will still work as it does in earlier versions, your documents will remain the same if you don't set it explicitly.}.
	Take a look at the next page for the precise definition of |compat|.

	\item \PGFPlots\ 1.3 now supports reversed axes. It is no longer necessary to use work--arounds with negative units.
\pgfkeys{/pdflinks/search key prefixes in/.add={/pgfplots/,}{}}

	Take a look at the |x dir=reverse| key.

	Existing work--arounds will still function properly. Use |\pgfplotsset{compat=1.3}| together with |x dir=reverse| to switch to the new version.
\end{enumerate}

\subsubsection{Old Features Which May Need Attention}
\begin{enumerate}
	\item The |scatter/classes| feature produces proper legends as of version 1.3. This may change the appearance of existing legends of plots with |scatter/classes|.

	\item Due to a small math bug in \PGF\ $2.00$, you \emph{can't} use the math expression `|-x^2|'. It is necessary to use `|0-x^2|' instead. The same holds for
		`|exp(-x^2)|'; use `|exp(0-x^2)|' instead.

		This will be fixed with \PGF\ $>2.00$.

	\item Starting with \PGFPlots\ $1.1$, |\tikzstyle| should \emph{no longer be used} to set \PGFPlots\ options.
	
	Although |\tikzstyle| is still supported for some older \PGFPlots\ options, you should replace any occurance of |\tikzstyle| with |\pgfplotsset{|\meta{style name}|/.style={|\meta{key-value-list}|}}| or the associated |/.append style| variant. See section~\ref{sec:styles} for more detail.
\end{enumerate}
I apologize for any inconvenience caused by these changes.

\begin{pgfplotskey}{compat=\mchoice{1.3,pre 1.3,default,newest} (initially default)}
	The preamble configuration 
\begin{codeexample}[code only]
\usepackage{pgfplots}
\pgfplotsset{compat=1.3}
\end{codeexample}
	allows to choose between backwards compatibility and most recent features.

	\PGFPlots\ version 1.3 comes with new features which may lead to different spacings compared to earlier versions:
	\begin{enumerate}
		\item Version 1.3 comes with the |xlabel near ticks| feature which places (in this case $x$) axis labels ``near'' the tick labels, taking the size of tick labels into account.
		
		Older versions used |xlabel absolute|, i.e.\ an absolute distance between the axis and axis labels (now matter how large or small tick labels are).

		The initial setting \emph{keeps backwards compatibility}.

		You are encouraged to use
\begin{codeexample}[code only]
\usepackage{pgfplots}
\pgfplotsset{compat=1.3}
\end{codeexample}
		in your preamble.
		
		\item Version 1.3 fixes a bug with |anchor=south west| which occurred mainly for reversed axes: the anchors where upside--down. This is fixed now.

		
		If you have written some work--around and want to keep it going, use

		|\pgfplotsset{compat/anchors=pre 1.3}|\pgfmanualpdflabel{/pgfplots/compat/anchors}{} 

		to disable the bug fix.
	\end{enumerate}

	Use |\pgfplotsset{compat=default}| to restore the factory settings.

	The setting |\pgfplotsset{compat=newest}| will always use features of the most recent version. This might result in small changes in the document's appearance.
\end{pgfplotskey}

\subsection{The Team}
\PGFPlots\ has been written mainly by Christian Feuers�nger with many improvements of Pascal Wolkotte and Nick Papior Andersen as a spare time project. We hope it is useful and provides valuable plots.

If you are interested in writing something but don't know how, consider reading the auxiliary manual \href{file:TeX-programming-notes.pdf}{TeX-programming-notes.pdf} which comes with \PGFPlots. It is far from complete, but maybe it is a good starting point (at least for more literature).

\subsection{Acknowledgements}
I thank God for all hours of enjoyed programming. I thank Pascal Wolkotte and Nick Papior Andersen for their programming efforts and contributions as part of the development team. I thank Stefan Tibus, who contributed the |plot shell| feature. I thank Tom Cashman for the contribution of the |reverse legend| feature. Special thanks go to Stefan Pinnow whose tests of \PGFPlots\ lead to numerous quality improvements. Furthermore, I thank Dr.~Schweitzer for many fruitful discussions and Dr.~Meine for his ideas and suggestions. Thanks as well to the many international contributors who provided feature requests or identified bugs or simply manual improvements!

Last but not least, I thank Till Tantau and Mark Wibrow for their excellent graphics (and more) package \PGF\ and \Tikz\ which is the base of \PGFPlots.

\include{pgfplots.install}%
\include{pgfplots.intro}%
\include{pgfplots.reference}%
\include{pgfplots.libs}%
\include{pgfplots.resources}%
\include{pgfplots.importexport}%
\include{pgfplots.basic.reference}%

\printindex

\bibliographystyle{abbrv} %gerapali} %gerabbrv} %gerunsrt.bst} %gerabbrv}% gerplain}
\nocite{pgfplotstable}
\nocite{programmingnotes}
\bibliography{pgfplots}
\end{document}
%
%%%%%%%%%%%%%%%%%%%%%%%%%%%%%%%%%%%%%%%%%%%%%%%%%%%%%%%%%%%%%%%%%%%%%%%%%%%%%
%
% Package pgfplots.sty documentation. 
%
% Copyright 2007/2008 by Christian Feuersaenger.
%
% This program is free software: you can redistribute it and/or modify
% it under the terms of the GNU General Public License as published by
% the Free Software Foundation, either version 3 of the License, or
% (at your option) any later version.
% 
% This program is distributed in the hope that it will be useful,
% but WITHOUT ANY WARRANTY; without even the implied warranty of
% MERCHANTABILITY or FITNESS FOR A PARTICULAR PURPOSE.  See the
% GNU General Public License for more details.
% 
% You should have received a copy of the GNU General Public License
% along with this program.  If not, see <http://www.gnu.org/licenses/>.
%
%
%%%%%%%%%%%%%%%%%%%%%%%%%%%%%%%%%%%%%%%%%%%%%%%%%%%%%%%%%%%%%%%%%%%%%%%%%%%%%
\input{pgfplots.preamble.tex}
%\RequirePackage[german,english,francais]{babel}

\pgfplotsmanualexternalexpensivetrue

%\usetikzlibrary{external}
%\tikzexternalize[prefix=figures/]{pgfplots}

\title{%
	Manual for Package \PGFPlots\\
	{\small Version \pgfplotsversion}\\
	{\small\href{http://sourceforge.net/projects/pgfplots}{http://sourceforge.net/projects/pgfplots}}}

%\includeonly{pgfplots.libs}


\begin{document}

\def\plotcoords{%
\addplot coordinates {
(5,8.312e-02)    (17,2.547e-02)   (49,7.407e-03)
(129,2.102e-03)  (321,5.874e-04)  (769,1.623e-04)
(1793,4.442e-05) (4097,1.207e-05) (9217,3.261e-06)
};

\addplot coordinates{
(7,8.472e-02)    (31,3.044e-02)    (111,1.022e-02)
(351,3.303e-03)  (1023,1.039e-03)  (2815,3.196e-04)
(7423,9.658e-05) (18943,2.873e-05) (47103,8.437e-06)
};

\addplot coordinates{
(9,7.881e-02)     (49,3.243e-02)    (209,1.232e-02)
(769,4.454e-03)   (2561,1.551e-03)  (7937,5.236e-04)
(23297,1.723e-04) (65537,5.545e-05) (178177,1.751e-05)
};

\addplot coordinates{
(11,6.887e-02)    (71,3.177e-02)     (351,1.341e-02)
(1471,5.334e-03)  (5503,2.027e-03)   (18943,7.415e-04)
(61183,2.628e-04) (187903,9.063e-05) (553983,3.053e-05)
};

\addplot coordinates{
(13,5.755e-02)     (97,2.925e-02)     (545,1.351e-02)
(2561,5.842e-03)   (10625,2.397e-03)  (40193,9.414e-04)
(141569,3.564e-04) (471041,1.308e-04) 
(1496065,4.670e-05)
};
}%
\maketitle
\begin{abstract}%
\PGFPlots\ draws high--quality function plots in normal or logarithmic scaling with a user-friendly interface directly in \TeX. The user supplies axis labels, legend entries and the plot coordinates for one or more plots and \PGFPlots\ applies axis scaling, computes any logarithms and axis ticks and draws the plots, supporting line plots, scatter plots, piecewise constant plots, bar plots, area plots, mesh-- and surface plots and some more. It is based on Till Tantau's package \PGF/\Tikz.
\end{abstract}
\tableofcontents
\section{Introduction}
This package provides tools to generate plots and labeled axes easily. It draws normal plots, logplots and semi-logplots, in two and three dimensions. Axis ticks, labels, legends (in case of multiple plots) can be added with key-value options. It can cycle through a set of predefined line/marker/color specifications. In summary, its purpose is to simplify the generation of high-quality function and/or data plots, and solving the problems of
\begin{itemize}
	\item consistency of document and font type and font size,
	\item direct use of \TeX\ math mode in axis descriptions,
	\item consistency of data and figures (no third party tool necessary),
	\item inter document consistency using preamble configurations and styles.
\end{itemize}
Although not necessary, separate |.pdf| or |.eps| graphics can be generated using the |external| library developed as part of \Tikz.

\PGFPlots\ is build completely on \Tikz/\PGF. Knowledge of \Tikz\ will simplify the work with \PGFPlots, although it is not required.

\section[About PGFPlots: Preliminaries]{About {\normalfont\PGFPlots}: Preliminaries}
This section contains information about upgrades, the team, the installation (in case you need to do it manually) and troubleshooting. You may skip it completely except for the upgrade remarks.

However, note that this library requires at least \PGF\ version $2.00$. At the time of this writing, many \TeX-distributions still contain the older \PGF\ version $1.18$, so it may be necessary to install a recent \PGF\ prior to using \PGFPlots.

\subsection{Components}
\PGFPlots\ comes with two components:
\begin{enumerate}
	\item the plotting component (which you are currently reading) and
	\item the \PGFPlotstable\ component which simplifies number formatting and postprocessing of numerical tables. It comes as a separate package and has its own manual \href{file:pgfplotstable.pdf}{pgfplotstable.pdf}.
\end{enumerate}

\subsection{Upgrade remarks}
This release provides a lot of improvements which can be found in all detail in \texttt{ChangeLog} for interested readers. However, some attention is useful with respect to the following changes.

\subsubsection{New Optional Features}
\begin{enumerate}
	\item \PGFPlots\ 1.3 comes with user interface improvements. The technical distinction between ``behavior options'' and ``style options'' of older versions is no longer necessary (although still fully supported).

	\item \PGFPlots\ 1.3 has a new feature which allows to \emph{move axis labels tight to tick labels} automatically.

	Since this affects the spacing, it is not enabled be default.

	Use
\begin{codeexample}[code only]
\usepackage{pgfplots}
\pgfplotsset{compat=1.3}
\end{codeexample}
	in your preamble to benefit from the improved spacing\footnote{The |compat=1.3| setting is necessary for new features which move axis labels around like reversed axis. However, \PGFPlots\ will still work as it does in earlier versions, your documents will remain the same if you don't set it explicitly.}.
	Take a look at the next page for the precise definition of |compat|.

	\item \PGFPlots\ 1.3 now supports reversed axes. It is no longer necessary to use work--arounds with negative units.
\pgfkeys{/pdflinks/search key prefixes in/.add={/pgfplots/,}{}}

	Take a look at the |x dir=reverse| key.

	Existing work--arounds will still function properly. Use |\pgfplotsset{compat=1.3}| together with |x dir=reverse| to switch to the new version.
\end{enumerate}

\subsubsection{Old Features Which May Need Attention}
\begin{enumerate}
	\item The |scatter/classes| feature produces proper legends as of version 1.3. This may change the appearance of existing legends of plots with |scatter/classes|.

	\item Due to a small math bug in \PGF\ $2.00$, you \emph{can't} use the math expression `|-x^2|'. It is necessary to use `|0-x^2|' instead. The same holds for
		`|exp(-x^2)|'; use `|exp(0-x^2)|' instead.

		This will be fixed with \PGF\ $>2.00$.

	\item Starting with \PGFPlots\ $1.1$, |\tikzstyle| should \emph{no longer be used} to set \PGFPlots\ options.
	
	Although |\tikzstyle| is still supported for some older \PGFPlots\ options, you should replace any occurance of |\tikzstyle| with |\pgfplotsset{|\meta{style name}|/.style={|\meta{key-value-list}|}}| or the associated |/.append style| variant. See section~\ref{sec:styles} for more detail.
\end{enumerate}
I apologize for any inconvenience caused by these changes.

\begin{pgfplotskey}{compat=\mchoice{1.3,pre 1.3,default,newest} (initially default)}
	The preamble configuration 
\begin{codeexample}[code only]
\usepackage{pgfplots}
\pgfplotsset{compat=1.3}
\end{codeexample}
	allows to choose between backwards compatibility and most recent features.

	\PGFPlots\ version 1.3 comes with new features which may lead to different spacings compared to earlier versions:
	\begin{enumerate}
		\item Version 1.3 comes with the |xlabel near ticks| feature which places (in this case $x$) axis labels ``near'' the tick labels, taking the size of tick labels into account.
		
		Older versions used |xlabel absolute|, i.e.\ an absolute distance between the axis and axis labels (now matter how large or small tick labels are).

		The initial setting \emph{keeps backwards compatibility}.

		You are encouraged to use
\begin{codeexample}[code only]
\usepackage{pgfplots}
\pgfplotsset{compat=1.3}
\end{codeexample}
		in your preamble.
		
		\item Version 1.3 fixes a bug with |anchor=south west| which occurred mainly for reversed axes: the anchors where upside--down. This is fixed now.

		
		If you have written some work--around and want to keep it going, use

		|\pgfplotsset{compat/anchors=pre 1.3}|\pgfmanualpdflabel{/pgfplots/compat/anchors}{} 

		to disable the bug fix.
	\end{enumerate}

	Use |\pgfplotsset{compat=default}| to restore the factory settings.

	The setting |\pgfplotsset{compat=newest}| will always use features of the most recent version. This might result in small changes in the document's appearance.
\end{pgfplotskey}

\subsection{The Team}
\PGFPlots\ has been written mainly by Christian Feuers�nger with many improvements of Pascal Wolkotte and Nick Papior Andersen as a spare time project. We hope it is useful and provides valuable plots.

If you are interested in writing something but don't know how, consider reading the auxiliary manual \href{file:TeX-programming-notes.pdf}{TeX-programming-notes.pdf} which comes with \PGFPlots. It is far from complete, but maybe it is a good starting point (at least for more literature).

\subsection{Acknowledgements}
I thank God for all hours of enjoyed programming. I thank Pascal Wolkotte and Nick Papior Andersen for their programming efforts and contributions as part of the development team. I thank Stefan Tibus, who contributed the |plot shell| feature. I thank Tom Cashman for the contribution of the |reverse legend| feature. Special thanks go to Stefan Pinnow whose tests of \PGFPlots\ lead to numerous quality improvements. Furthermore, I thank Dr.~Schweitzer for many fruitful discussions and Dr.~Meine for his ideas and suggestions. Thanks as well to the many international contributors who provided feature requests or identified bugs or simply manual improvements!

Last but not least, I thank Till Tantau and Mark Wibrow for their excellent graphics (and more) package \PGF\ and \Tikz\ which is the base of \PGFPlots.

\include{pgfplots.install}%
\include{pgfplots.intro}%
\include{pgfplots.reference}%
\include{pgfplots.libs}%
\include{pgfplots.resources}%
\include{pgfplots.importexport}%
\include{pgfplots.basic.reference}%

\printindex

\bibliographystyle{abbrv} %gerapali} %gerabbrv} %gerunsrt.bst} %gerabbrv}% gerplain}
\nocite{pgfplotstable}
\nocite{programmingnotes}
\bibliography{pgfplots}
\end{document}
%
%%%%%%%%%%%%%%%%%%%%%%%%%%%%%%%%%%%%%%%%%%%%%%%%%%%%%%%%%%%%%%%%%%%%%%%%%%%%%
%
% Package pgfplots.sty documentation. 
%
% Copyright 2007/2008 by Christian Feuersaenger.
%
% This program is free software: you can redistribute it and/or modify
% it under the terms of the GNU General Public License as published by
% the Free Software Foundation, either version 3 of the License, or
% (at your option) any later version.
% 
% This program is distributed in the hope that it will be useful,
% but WITHOUT ANY WARRANTY; without even the implied warranty of
% MERCHANTABILITY or FITNESS FOR A PARTICULAR PURPOSE.  See the
% GNU General Public License for more details.
% 
% You should have received a copy of the GNU General Public License
% along with this program.  If not, see <http://www.gnu.org/licenses/>.
%
%
%%%%%%%%%%%%%%%%%%%%%%%%%%%%%%%%%%%%%%%%%%%%%%%%%%%%%%%%%%%%%%%%%%%%%%%%%%%%%
\input{pgfplots.preamble.tex}
%\RequirePackage[german,english,francais]{babel}

\pgfplotsmanualexternalexpensivetrue

%\usetikzlibrary{external}
%\tikzexternalize[prefix=figures/]{pgfplots}

\title{%
	Manual for Package \PGFPlots\\
	{\small Version \pgfplotsversion}\\
	{\small\href{http://sourceforge.net/projects/pgfplots}{http://sourceforge.net/projects/pgfplots}}}

%\includeonly{pgfplots.libs}


\begin{document}

\def\plotcoords{%
\addplot coordinates {
(5,8.312e-02)    (17,2.547e-02)   (49,7.407e-03)
(129,2.102e-03)  (321,5.874e-04)  (769,1.623e-04)
(1793,4.442e-05) (4097,1.207e-05) (9217,3.261e-06)
};

\addplot coordinates{
(7,8.472e-02)    (31,3.044e-02)    (111,1.022e-02)
(351,3.303e-03)  (1023,1.039e-03)  (2815,3.196e-04)
(7423,9.658e-05) (18943,2.873e-05) (47103,8.437e-06)
};

\addplot coordinates{
(9,7.881e-02)     (49,3.243e-02)    (209,1.232e-02)
(769,4.454e-03)   (2561,1.551e-03)  (7937,5.236e-04)
(23297,1.723e-04) (65537,5.545e-05) (178177,1.751e-05)
};

\addplot coordinates{
(11,6.887e-02)    (71,3.177e-02)     (351,1.341e-02)
(1471,5.334e-03)  (5503,2.027e-03)   (18943,7.415e-04)
(61183,2.628e-04) (187903,9.063e-05) (553983,3.053e-05)
};

\addplot coordinates{
(13,5.755e-02)     (97,2.925e-02)     (545,1.351e-02)
(2561,5.842e-03)   (10625,2.397e-03)  (40193,9.414e-04)
(141569,3.564e-04) (471041,1.308e-04) 
(1496065,4.670e-05)
};
}%
\maketitle
\begin{abstract}%
\PGFPlots\ draws high--quality function plots in normal or logarithmic scaling with a user-friendly interface directly in \TeX. The user supplies axis labels, legend entries and the plot coordinates for one or more plots and \PGFPlots\ applies axis scaling, computes any logarithms and axis ticks and draws the plots, supporting line plots, scatter plots, piecewise constant plots, bar plots, area plots, mesh-- and surface plots and some more. It is based on Till Tantau's package \PGF/\Tikz.
\end{abstract}
\tableofcontents
\section{Introduction}
This package provides tools to generate plots and labeled axes easily. It draws normal plots, logplots and semi-logplots, in two and three dimensions. Axis ticks, labels, legends (in case of multiple plots) can be added with key-value options. It can cycle through a set of predefined line/marker/color specifications. In summary, its purpose is to simplify the generation of high-quality function and/or data plots, and solving the problems of
\begin{itemize}
	\item consistency of document and font type and font size,
	\item direct use of \TeX\ math mode in axis descriptions,
	\item consistency of data and figures (no third party tool necessary),
	\item inter document consistency using preamble configurations and styles.
\end{itemize}
Although not necessary, separate |.pdf| or |.eps| graphics can be generated using the |external| library developed as part of \Tikz.

\PGFPlots\ is build completely on \Tikz/\PGF. Knowledge of \Tikz\ will simplify the work with \PGFPlots, although it is not required.

\section[About PGFPlots: Preliminaries]{About {\normalfont\PGFPlots}: Preliminaries}
This section contains information about upgrades, the team, the installation (in case you need to do it manually) and troubleshooting. You may skip it completely except for the upgrade remarks.

However, note that this library requires at least \PGF\ version $2.00$. At the time of this writing, many \TeX-distributions still contain the older \PGF\ version $1.18$, so it may be necessary to install a recent \PGF\ prior to using \PGFPlots.

\subsection{Components}
\PGFPlots\ comes with two components:
\begin{enumerate}
	\item the plotting component (which you are currently reading) and
	\item the \PGFPlotstable\ component which simplifies number formatting and postprocessing of numerical tables. It comes as a separate package and has its own manual \href{file:pgfplotstable.pdf}{pgfplotstable.pdf}.
\end{enumerate}

\subsection{Upgrade remarks}
This release provides a lot of improvements which can be found in all detail in \texttt{ChangeLog} for interested readers. However, some attention is useful with respect to the following changes.

\subsubsection{New Optional Features}
\begin{enumerate}
	\item \PGFPlots\ 1.3 comes with user interface improvements. The technical distinction between ``behavior options'' and ``style options'' of older versions is no longer necessary (although still fully supported).

	\item \PGFPlots\ 1.3 has a new feature which allows to \emph{move axis labels tight to tick labels} automatically.

	Since this affects the spacing, it is not enabled be default.

	Use
\begin{codeexample}[code only]
\usepackage{pgfplots}
\pgfplotsset{compat=1.3}
\end{codeexample}
	in your preamble to benefit from the improved spacing\footnote{The |compat=1.3| setting is necessary for new features which move axis labels around like reversed axis. However, \PGFPlots\ will still work as it does in earlier versions, your documents will remain the same if you don't set it explicitly.}.
	Take a look at the next page for the precise definition of |compat|.

	\item \PGFPlots\ 1.3 now supports reversed axes. It is no longer necessary to use work--arounds with negative units.
\pgfkeys{/pdflinks/search key prefixes in/.add={/pgfplots/,}{}}

	Take a look at the |x dir=reverse| key.

	Existing work--arounds will still function properly. Use |\pgfplotsset{compat=1.3}| together with |x dir=reverse| to switch to the new version.
\end{enumerate}

\subsubsection{Old Features Which May Need Attention}
\begin{enumerate}
	\item The |scatter/classes| feature produces proper legends as of version 1.3. This may change the appearance of existing legends of plots with |scatter/classes|.

	\item Due to a small math bug in \PGF\ $2.00$, you \emph{can't} use the math expression `|-x^2|'. It is necessary to use `|0-x^2|' instead. The same holds for
		`|exp(-x^2)|'; use `|exp(0-x^2)|' instead.

		This will be fixed with \PGF\ $>2.00$.

	\item Starting with \PGFPlots\ $1.1$, |\tikzstyle| should \emph{no longer be used} to set \PGFPlots\ options.
	
	Although |\tikzstyle| is still supported for some older \PGFPlots\ options, you should replace any occurance of |\tikzstyle| with |\pgfplotsset{|\meta{style name}|/.style={|\meta{key-value-list}|}}| or the associated |/.append style| variant. See section~\ref{sec:styles} for more detail.
\end{enumerate}
I apologize for any inconvenience caused by these changes.

\begin{pgfplotskey}{compat=\mchoice{1.3,pre 1.3,default,newest} (initially default)}
	The preamble configuration 
\begin{codeexample}[code only]
\usepackage{pgfplots}
\pgfplotsset{compat=1.3}
\end{codeexample}
	allows to choose between backwards compatibility and most recent features.

	\PGFPlots\ version 1.3 comes with new features which may lead to different spacings compared to earlier versions:
	\begin{enumerate}
		\item Version 1.3 comes with the |xlabel near ticks| feature which places (in this case $x$) axis labels ``near'' the tick labels, taking the size of tick labels into account.
		
		Older versions used |xlabel absolute|, i.e.\ an absolute distance between the axis and axis labels (now matter how large or small tick labels are).

		The initial setting \emph{keeps backwards compatibility}.

		You are encouraged to use
\begin{codeexample}[code only]
\usepackage{pgfplots}
\pgfplotsset{compat=1.3}
\end{codeexample}
		in your preamble.
		
		\item Version 1.3 fixes a bug with |anchor=south west| which occurred mainly for reversed axes: the anchors where upside--down. This is fixed now.

		
		If you have written some work--around and want to keep it going, use

		|\pgfplotsset{compat/anchors=pre 1.3}|\pgfmanualpdflabel{/pgfplots/compat/anchors}{} 

		to disable the bug fix.
	\end{enumerate}

	Use |\pgfplotsset{compat=default}| to restore the factory settings.

	The setting |\pgfplotsset{compat=newest}| will always use features of the most recent version. This might result in small changes in the document's appearance.
\end{pgfplotskey}

\subsection{The Team}
\PGFPlots\ has been written mainly by Christian Feuers�nger with many improvements of Pascal Wolkotte and Nick Papior Andersen as a spare time project. We hope it is useful and provides valuable plots.

If you are interested in writing something but don't know how, consider reading the auxiliary manual \href{file:TeX-programming-notes.pdf}{TeX-programming-notes.pdf} which comes with \PGFPlots. It is far from complete, but maybe it is a good starting point (at least for more literature).

\subsection{Acknowledgements}
I thank God for all hours of enjoyed programming. I thank Pascal Wolkotte and Nick Papior Andersen for their programming efforts and contributions as part of the development team. I thank Stefan Tibus, who contributed the |plot shell| feature. I thank Tom Cashman for the contribution of the |reverse legend| feature. Special thanks go to Stefan Pinnow whose tests of \PGFPlots\ lead to numerous quality improvements. Furthermore, I thank Dr.~Schweitzer for many fruitful discussions and Dr.~Meine for his ideas and suggestions. Thanks as well to the many international contributors who provided feature requests or identified bugs or simply manual improvements!

Last but not least, I thank Till Tantau and Mark Wibrow for their excellent graphics (and more) package \PGF\ and \Tikz\ which is the base of \PGFPlots.

\include{pgfplots.install}%
\include{pgfplots.intro}%
\include{pgfplots.reference}%
\include{pgfplots.libs}%
\include{pgfplots.resources}%
\include{pgfplots.importexport}%
\include{pgfplots.basic.reference}%

\printindex

\bibliographystyle{abbrv} %gerapali} %gerabbrv} %gerunsrt.bst} %gerabbrv}% gerplain}
\nocite{pgfplotstable}
\nocite{programmingnotes}
\bibliography{pgfplots}
\end{document}
%
\include{pgfplots.basic.reference}%

\printindex

\bibliographystyle{abbrv} %gerapali} %gerabbrv} %gerunsrt.bst} %gerabbrv}% gerplain}
\nocite{pgfplotstable}
\nocite{programmingnotes}
\bibliography{pgfplots}
\end{document}
%
%%%%%%%%%%%%%%%%%%%%%%%%%%%%%%%%%%%%%%%%%%%%%%%%%%%%%%%%%%%%%%%%%%%%%%%%%%%%%
%
% Package pgfplots.sty documentation. 
%
% Copyright 2007/2008 by Christian Feuersaenger.
%
% This program is free software: you can redistribute it and/or modify
% it under the terms of the GNU General Public License as published by
% the Free Software Foundation, either version 3 of the License, or
% (at your option) any later version.
% 
% This program is distributed in the hope that it will be useful,
% but WITHOUT ANY WARRANTY; without even the implied warranty of
% MERCHANTABILITY or FITNESS FOR A PARTICULAR PURPOSE.  See the
% GNU General Public License for more details.
% 
% You should have received a copy of the GNU General Public License
% along with this program.  If not, see <http://www.gnu.org/licenses/>.
%
%
%%%%%%%%%%%%%%%%%%%%%%%%%%%%%%%%%%%%%%%%%%%%%%%%%%%%%%%%%%%%%%%%%%%%%%%%%%%%%
%%%%%%%%%%%%%%%%%%%%%%%%%%%%%%%%%%%%%%%%%%%%%%%%%%%%%%%%%%%%%%%%%%%%%%%%%%%%%
%
% Package pgfplots.sty documentation. 
%
% Copyright 2007/2008 by Christian Feuersaenger.
%
% This program is free software: you can redistribute it and/or modify
% it under the terms of the GNU General Public License as published by
% the Free Software Foundation, either version 3 of the License, or
% (at your option) any later version.
% 
% This program is distributed in the hope that it will be useful,
% but WITHOUT ANY WARRANTY; without even the implied warranty of
% MERCHANTABILITY or FITNESS FOR A PARTICULAR PURPOSE.  See the
% GNU General Public License for more details.
% 
% You should have received a copy of the GNU General Public License
% along with this program.  If not, see <http://www.gnu.org/licenses/>.
%
%
%%%%%%%%%%%%%%%%%%%%%%%%%%%%%%%%%%%%%%%%%%%%%%%%%%%%%%%%%%%%%%%%%%%%%%%%%%%%%
\documentclass[a4paper]{ltxdoc}

\usepackage{makeidx}

% DON't let hyperref overload the format of index and glossary. 
% I want to do that on my own in the stylefiles for makeindex...
\makeatletter
\let\@old@wrindex=\@wrindex
\makeatother

\usepackage{ifpdf}
\ifpdf
	\usepackage[pdftex]{hyperref}
\else
	\usepackage[dvipdfm]{hyperref}
\fi
\hypersetup{pdfborder={0 0 0}}

\makeatletter
\let\@wrindex=\@old@wrindex
\makeatother

% Formatiere Seitennummern im Index:
\newcommand{\indexpageno}[1]{%
	{\bfseries\hyperpage{#1}}%
}


\newcommand{\R}{\mathbb{R}}
\newcommand{\N}{\mathbb{N}}
\newcommand{\Z}{\mathbb{Z}}

\long\def\COMMENTLOWLEVEL#1\ENDCOMMENT{}
\def\ENDCOMMENT{}

\usepackage{textcomp}
\usepackage{booktabs}

\usepackage{calc}
\usepackage[formats]{listings}
%\usepackage{courier} % don't use it - the '^' character can't be copy-pasted in courier

\usepackage{array}
\lstset{%
	basicstyle=\ttfamily,
	language=[LaTeX]tex, % Seems as if \lstset{language=tex} must be invoked BEFORE loading tikz!?
	tabsize=4,
	breaklines=true,
	breakindent=0pt
}

\ifpdf
	\pdfinfo {
		/Author	(Christian Feuersaenger)
	}

\else
	\def\pgfsysdriver{pgfsys-dvipdfm.def}
\fi
%\def\pgfsysdriver{pgfsys-pdftex.def}
\usepackage{pgfplots}

\ifpdf
	\usetikzlibrary{pgfplotsclickable}
\fi

%\usepackage{fp}
% ATTENTION:
% this requires pgf version NEWER than 2.00 :
%\usetikzlibrary{fixedpointarithmetic}

\usetikzlibrary{dateplot}

\usepackage[a4paper,left=2.25cm,right=2.25cm,top=2.5cm,bottom=2.5cm,nohead]{geometry}
\usepackage{amsmath,amssymb}
\usepackage{xxcolor}
\usepackage{pifont}
\usepackage[latin1]{inputenc}
\usepackage{amsmath}
\usepackage{eurosym}
\input{pgfplots-macros}

\pgfqkeys{/codeexample}{%
	every codeexample/.style={
		width=8cm,
		/pgfplots/every axis/.append style={legend style={fill=graphicbackground}}
	},
	tabsize=4,
}
\pgfplotsset{
	every axis/.append style={width=7cm},
	filter discard warning=false,
}

\usetikzlibrary{backgrounds,patterns}
% Global styles:
\tikzset{
  shape example/.style={
    color=black!30,
    draw,
    fill=yellow!30,
    line width=.5cm,
    inner xsep=2.5cm,
    inner ysep=0.5cm}
}

\newcommand{\FIXME}[1]{\textcolor{red}{(FIXME: #1)}}

% fuer endvironment 'sidewaysfigure' bspw
% \usepackage{rotating}

\newcommand\Tikz{Ti\textit kZ}
\newcommand\PGF{\textsc{pgf}}
\newcommand\PGFPlots{\textsc{pgfplots}}
\def\PGFPlotstable{\textsc{PgfplotsTable}}


\makeindex

% Fix overful hboxes automatically:
\tolerance=2000
\emergencystretch=10pt

\tikzset{prefix=gnuplot/pgfplots_} % prefix for 'plot function'

\author{%
	Christian Feuers\"anger\footnote{\url{http://wissrech.ins.uni-bonn.de/people/feuersaenger}}\\%
	Institut f\"ur Numerische Simulation\\
	Universit\"at Bonn, Germany}


%\RequirePackage[german,english,francais]{babel}

\pgfplotsmanualexternalexpensivetrue

%\usetikzlibrary{external}
%\tikzexternalize[prefix=figures/]{pgfplots}

\title{%
	Manual for Package \PGFPlots\\
	{\small Version \pgfplotsversion}\\
	{\small\href{http://sourceforge.net/projects/pgfplots}{http://sourceforge.net/projects/pgfplots}}}

%\includeonly{pgfplots.libs}


\begin{document}

\def\plotcoords{%
\addplot coordinates {
(5,8.312e-02)    (17,2.547e-02)   (49,7.407e-03)
(129,2.102e-03)  (321,5.874e-04)  (769,1.623e-04)
(1793,4.442e-05) (4097,1.207e-05) (9217,3.261e-06)
};

\addplot coordinates{
(7,8.472e-02)    (31,3.044e-02)    (111,1.022e-02)
(351,3.303e-03)  (1023,1.039e-03)  (2815,3.196e-04)
(7423,9.658e-05) (18943,2.873e-05) (47103,8.437e-06)
};

\addplot coordinates{
(9,7.881e-02)     (49,3.243e-02)    (209,1.232e-02)
(769,4.454e-03)   (2561,1.551e-03)  (7937,5.236e-04)
(23297,1.723e-04) (65537,5.545e-05) (178177,1.751e-05)
};

\addplot coordinates{
(11,6.887e-02)    (71,3.177e-02)     (351,1.341e-02)
(1471,5.334e-03)  (5503,2.027e-03)   (18943,7.415e-04)
(61183,2.628e-04) (187903,9.063e-05) (553983,3.053e-05)
};

\addplot coordinates{
(13,5.755e-02)     (97,2.925e-02)     (545,1.351e-02)
(2561,5.842e-03)   (10625,2.397e-03)  (40193,9.414e-04)
(141569,3.564e-04) (471041,1.308e-04) 
(1496065,4.670e-05)
};
}%
\maketitle
\begin{abstract}%
\PGFPlots\ draws high--quality function plots in normal or logarithmic scaling with a user-friendly interface directly in \TeX. The user supplies axis labels, legend entries and the plot coordinates for one or more plots and \PGFPlots\ applies axis scaling, computes any logarithms and axis ticks and draws the plots, supporting line plots, scatter plots, piecewise constant plots, bar plots, area plots, mesh-- and surface plots and some more. It is based on Till Tantau's package \PGF/\Tikz.
\end{abstract}
\tableofcontents
\section{Introduction}
This package provides tools to generate plots and labeled axes easily. It draws normal plots, logplots and semi-logplots, in two and three dimensions. Axis ticks, labels, legends (in case of multiple plots) can be added with key-value options. It can cycle through a set of predefined line/marker/color specifications. In summary, its purpose is to simplify the generation of high-quality function and/or data plots, and solving the problems of
\begin{itemize}
	\item consistency of document and font type and font size,
	\item direct use of \TeX\ math mode in axis descriptions,
	\item consistency of data and figures (no third party tool necessary),
	\item inter document consistency using preamble configurations and styles.
\end{itemize}
Although not necessary, separate |.pdf| or |.eps| graphics can be generated using the |external| library developed as part of \Tikz.

\PGFPlots\ is build completely on \Tikz/\PGF. Knowledge of \Tikz\ will simplify the work with \PGFPlots, although it is not required.

\section[About PGFPlots: Preliminaries]{About {\normalfont\PGFPlots}: Preliminaries}
This section contains information about upgrades, the team, the installation (in case you need to do it manually) and troubleshooting. You may skip it completely except for the upgrade remarks.

However, note that this library requires at least \PGF\ version $2.00$. At the time of this writing, many \TeX-distributions still contain the older \PGF\ version $1.18$, so it may be necessary to install a recent \PGF\ prior to using \PGFPlots.

\subsection{Components}
\PGFPlots\ comes with two components:
\begin{enumerate}
	\item the plotting component (which you are currently reading) and
	\item the \PGFPlotstable\ component which simplifies number formatting and postprocessing of numerical tables. It comes as a separate package and has its own manual \href{file:pgfplotstable.pdf}{pgfplotstable.pdf}.
\end{enumerate}

\subsection{Upgrade remarks}
This release provides a lot of improvements which can be found in all detail in \texttt{ChangeLog} for interested readers. However, some attention is useful with respect to the following changes.

\subsubsection{New Optional Features}
\begin{enumerate}
	\item \PGFPlots\ 1.3 comes with user interface improvements. The technical distinction between ``behavior options'' and ``style options'' of older versions is no longer necessary (although still fully supported).

	\item \PGFPlots\ 1.3 has a new feature which allows to \emph{move axis labels tight to tick labels} automatically.

	Since this affects the spacing, it is not enabled be default.

	Use
\begin{codeexample}[code only]
\usepackage{pgfplots}
\pgfplotsset{compat=1.3}
\end{codeexample}
	in your preamble to benefit from the improved spacing\footnote{The |compat=1.3| setting is necessary for new features which move axis labels around like reversed axis. However, \PGFPlots\ will still work as it does in earlier versions, your documents will remain the same if you don't set it explicitly.}.
	Take a look at the next page for the precise definition of |compat|.

	\item \PGFPlots\ 1.3 now supports reversed axes. It is no longer necessary to use work--arounds with negative units.
\pgfkeys{/pdflinks/search key prefixes in/.add={/pgfplots/,}{}}

	Take a look at the |x dir=reverse| key.

	Existing work--arounds will still function properly. Use |\pgfplotsset{compat=1.3}| together with |x dir=reverse| to switch to the new version.
\end{enumerate}

\subsubsection{Old Features Which May Need Attention}
\begin{enumerate}
	\item The |scatter/classes| feature produces proper legends as of version 1.3. This may change the appearance of existing legends of plots with |scatter/classes|.

	\item Due to a small math bug in \PGF\ $2.00$, you \emph{can't} use the math expression `|-x^2|'. It is necessary to use `|0-x^2|' instead. The same holds for
		`|exp(-x^2)|'; use `|exp(0-x^2)|' instead.

		This will be fixed with \PGF\ $>2.00$.

	\item Starting with \PGFPlots\ $1.1$, |\tikzstyle| should \emph{no longer be used} to set \PGFPlots\ options.
	
	Although |\tikzstyle| is still supported for some older \PGFPlots\ options, you should replace any occurance of |\tikzstyle| with |\pgfplotsset{|\meta{style name}|/.style={|\meta{key-value-list}|}}| or the associated |/.append style| variant. See section~\ref{sec:styles} for more detail.
\end{enumerate}
I apologize for any inconvenience caused by these changes.

\begin{pgfplotskey}{compat=\mchoice{1.3,pre 1.3,default,newest} (initially default)}
	The preamble configuration 
\begin{codeexample}[code only]
\usepackage{pgfplots}
\pgfplotsset{compat=1.3}
\end{codeexample}
	allows to choose between backwards compatibility and most recent features.

	\PGFPlots\ version 1.3 comes with new features which may lead to different spacings compared to earlier versions:
	\begin{enumerate}
		\item Version 1.3 comes with the |xlabel near ticks| feature which places (in this case $x$) axis labels ``near'' the tick labels, taking the size of tick labels into account.
		
		Older versions used |xlabel absolute|, i.e.\ an absolute distance between the axis and axis labels (now matter how large or small tick labels are).

		The initial setting \emph{keeps backwards compatibility}.

		You are encouraged to use
\begin{codeexample}[code only]
\usepackage{pgfplots}
\pgfplotsset{compat=1.3}
\end{codeexample}
		in your preamble.
		
		\item Version 1.3 fixes a bug with |anchor=south west| which occurred mainly for reversed axes: the anchors where upside--down. This is fixed now.

		
		If you have written some work--around and want to keep it going, use

		|\pgfplotsset{compat/anchors=pre 1.3}|\pgfmanualpdflabel{/pgfplots/compat/anchors}{} 

		to disable the bug fix.
	\end{enumerate}

	Use |\pgfplotsset{compat=default}| to restore the factory settings.

	The setting |\pgfplotsset{compat=newest}| will always use features of the most recent version. This might result in small changes in the document's appearance.
\end{pgfplotskey}

\subsection{The Team}
\PGFPlots\ has been written mainly by Christian Feuers�nger with many improvements of Pascal Wolkotte and Nick Papior Andersen as a spare time project. We hope it is useful and provides valuable plots.

If you are interested in writing something but don't know how, consider reading the auxiliary manual \href{file:TeX-programming-notes.pdf}{TeX-programming-notes.pdf} which comes with \PGFPlots. It is far from complete, but maybe it is a good starting point (at least for more literature).

\subsection{Acknowledgements}
I thank God for all hours of enjoyed programming. I thank Pascal Wolkotte and Nick Papior Andersen for their programming efforts and contributions as part of the development team. I thank Stefan Tibus, who contributed the |plot shell| feature. I thank Tom Cashman for the contribution of the |reverse legend| feature. Special thanks go to Stefan Pinnow whose tests of \PGFPlots\ lead to numerous quality improvements. Furthermore, I thank Dr.~Schweitzer for many fruitful discussions and Dr.~Meine for his ideas and suggestions. Thanks as well to the many international contributors who provided feature requests or identified bugs or simply manual improvements!

Last but not least, I thank Till Tantau and Mark Wibrow for their excellent graphics (and more) package \PGF\ and \Tikz\ which is the base of \PGFPlots.

%%%%%%%%%%%%%%%%%%%%%%%%%%%%%%%%%%%%%%%%%%%%%%%%%%%%%%%%%%%%%%%%%%%%%%%%%%%%%
%
% Package pgfplots.sty documentation. 
%
% Copyright 2007/2008 by Christian Feuersaenger.
%
% This program is free software: you can redistribute it and/or modify
% it under the terms of the GNU General Public License as published by
% the Free Software Foundation, either version 3 of the License, or
% (at your option) any later version.
% 
% This program is distributed in the hope that it will be useful,
% but WITHOUT ANY WARRANTY; without even the implied warranty of
% MERCHANTABILITY or FITNESS FOR A PARTICULAR PURPOSE.  See the
% GNU General Public License for more details.
% 
% You should have received a copy of the GNU General Public License
% along with this program.  If not, see <http://www.gnu.org/licenses/>.
%
%
%%%%%%%%%%%%%%%%%%%%%%%%%%%%%%%%%%%%%%%%%%%%%%%%%%%%%%%%%%%%%%%%%%%%%%%%%%%%%
\input{pgfplots.preamble.tex}
%\RequirePackage[german,english,francais]{babel}

\pgfplotsmanualexternalexpensivetrue

%\usetikzlibrary{external}
%\tikzexternalize[prefix=figures/]{pgfplots}

\title{%
	Manual for Package \PGFPlots\\
	{\small Version \pgfplotsversion}\\
	{\small\href{http://sourceforge.net/projects/pgfplots}{http://sourceforge.net/projects/pgfplots}}}

%\includeonly{pgfplots.libs}


\begin{document}

\def\plotcoords{%
\addplot coordinates {
(5,8.312e-02)    (17,2.547e-02)   (49,7.407e-03)
(129,2.102e-03)  (321,5.874e-04)  (769,1.623e-04)
(1793,4.442e-05) (4097,1.207e-05) (9217,3.261e-06)
};

\addplot coordinates{
(7,8.472e-02)    (31,3.044e-02)    (111,1.022e-02)
(351,3.303e-03)  (1023,1.039e-03)  (2815,3.196e-04)
(7423,9.658e-05) (18943,2.873e-05) (47103,8.437e-06)
};

\addplot coordinates{
(9,7.881e-02)     (49,3.243e-02)    (209,1.232e-02)
(769,4.454e-03)   (2561,1.551e-03)  (7937,5.236e-04)
(23297,1.723e-04) (65537,5.545e-05) (178177,1.751e-05)
};

\addplot coordinates{
(11,6.887e-02)    (71,3.177e-02)     (351,1.341e-02)
(1471,5.334e-03)  (5503,2.027e-03)   (18943,7.415e-04)
(61183,2.628e-04) (187903,9.063e-05) (553983,3.053e-05)
};

\addplot coordinates{
(13,5.755e-02)     (97,2.925e-02)     (545,1.351e-02)
(2561,5.842e-03)   (10625,2.397e-03)  (40193,9.414e-04)
(141569,3.564e-04) (471041,1.308e-04) 
(1496065,4.670e-05)
};
}%
\maketitle
\begin{abstract}%
\PGFPlots\ draws high--quality function plots in normal or logarithmic scaling with a user-friendly interface directly in \TeX. The user supplies axis labels, legend entries and the plot coordinates for one or more plots and \PGFPlots\ applies axis scaling, computes any logarithms and axis ticks and draws the plots, supporting line plots, scatter plots, piecewise constant plots, bar plots, area plots, mesh-- and surface plots and some more. It is based on Till Tantau's package \PGF/\Tikz.
\end{abstract}
\tableofcontents
\section{Introduction}
This package provides tools to generate plots and labeled axes easily. It draws normal plots, logplots and semi-logplots, in two and three dimensions. Axis ticks, labels, legends (in case of multiple plots) can be added with key-value options. It can cycle through a set of predefined line/marker/color specifications. In summary, its purpose is to simplify the generation of high-quality function and/or data plots, and solving the problems of
\begin{itemize}
	\item consistency of document and font type and font size,
	\item direct use of \TeX\ math mode in axis descriptions,
	\item consistency of data and figures (no third party tool necessary),
	\item inter document consistency using preamble configurations and styles.
\end{itemize}
Although not necessary, separate |.pdf| or |.eps| graphics can be generated using the |external| library developed as part of \Tikz.

\PGFPlots\ is build completely on \Tikz/\PGF. Knowledge of \Tikz\ will simplify the work with \PGFPlots, although it is not required.

\section[About PGFPlots: Preliminaries]{About {\normalfont\PGFPlots}: Preliminaries}
This section contains information about upgrades, the team, the installation (in case you need to do it manually) and troubleshooting. You may skip it completely except for the upgrade remarks.

However, note that this library requires at least \PGF\ version $2.00$. At the time of this writing, many \TeX-distributions still contain the older \PGF\ version $1.18$, so it may be necessary to install a recent \PGF\ prior to using \PGFPlots.

\subsection{Components}
\PGFPlots\ comes with two components:
\begin{enumerate}
	\item the plotting component (which you are currently reading) and
	\item the \PGFPlotstable\ component which simplifies number formatting and postprocessing of numerical tables. It comes as a separate package and has its own manual \href{file:pgfplotstable.pdf}{pgfplotstable.pdf}.
\end{enumerate}

\subsection{Upgrade remarks}
This release provides a lot of improvements which can be found in all detail in \texttt{ChangeLog} for interested readers. However, some attention is useful with respect to the following changes.

\subsubsection{New Optional Features}
\begin{enumerate}
	\item \PGFPlots\ 1.3 comes with user interface improvements. The technical distinction between ``behavior options'' and ``style options'' of older versions is no longer necessary (although still fully supported).

	\item \PGFPlots\ 1.3 has a new feature which allows to \emph{move axis labels tight to tick labels} automatically.

	Since this affects the spacing, it is not enabled be default.

	Use
\begin{codeexample}[code only]
\usepackage{pgfplots}
\pgfplotsset{compat=1.3}
\end{codeexample}
	in your preamble to benefit from the improved spacing\footnote{The |compat=1.3| setting is necessary for new features which move axis labels around like reversed axis. However, \PGFPlots\ will still work as it does in earlier versions, your documents will remain the same if you don't set it explicitly.}.
	Take a look at the next page for the precise definition of |compat|.

	\item \PGFPlots\ 1.3 now supports reversed axes. It is no longer necessary to use work--arounds with negative units.
\pgfkeys{/pdflinks/search key prefixes in/.add={/pgfplots/,}{}}

	Take a look at the |x dir=reverse| key.

	Existing work--arounds will still function properly. Use |\pgfplotsset{compat=1.3}| together with |x dir=reverse| to switch to the new version.
\end{enumerate}

\subsubsection{Old Features Which May Need Attention}
\begin{enumerate}
	\item The |scatter/classes| feature produces proper legends as of version 1.3. This may change the appearance of existing legends of plots with |scatter/classes|.

	\item Due to a small math bug in \PGF\ $2.00$, you \emph{can't} use the math expression `|-x^2|'. It is necessary to use `|0-x^2|' instead. The same holds for
		`|exp(-x^2)|'; use `|exp(0-x^2)|' instead.

		This will be fixed with \PGF\ $>2.00$.

	\item Starting with \PGFPlots\ $1.1$, |\tikzstyle| should \emph{no longer be used} to set \PGFPlots\ options.
	
	Although |\tikzstyle| is still supported for some older \PGFPlots\ options, you should replace any occurance of |\tikzstyle| with |\pgfplotsset{|\meta{style name}|/.style={|\meta{key-value-list}|}}| or the associated |/.append style| variant. See section~\ref{sec:styles} for more detail.
\end{enumerate}
I apologize for any inconvenience caused by these changes.

\begin{pgfplotskey}{compat=\mchoice{1.3,pre 1.3,default,newest} (initially default)}
	The preamble configuration 
\begin{codeexample}[code only]
\usepackage{pgfplots}
\pgfplotsset{compat=1.3}
\end{codeexample}
	allows to choose between backwards compatibility and most recent features.

	\PGFPlots\ version 1.3 comes with new features which may lead to different spacings compared to earlier versions:
	\begin{enumerate}
		\item Version 1.3 comes with the |xlabel near ticks| feature which places (in this case $x$) axis labels ``near'' the tick labels, taking the size of tick labels into account.
		
		Older versions used |xlabel absolute|, i.e.\ an absolute distance between the axis and axis labels (now matter how large or small tick labels are).

		The initial setting \emph{keeps backwards compatibility}.

		You are encouraged to use
\begin{codeexample}[code only]
\usepackage{pgfplots}
\pgfplotsset{compat=1.3}
\end{codeexample}
		in your preamble.
		
		\item Version 1.3 fixes a bug with |anchor=south west| which occurred mainly for reversed axes: the anchors where upside--down. This is fixed now.

		
		If you have written some work--around and want to keep it going, use

		|\pgfplotsset{compat/anchors=pre 1.3}|\pgfmanualpdflabel{/pgfplots/compat/anchors}{} 

		to disable the bug fix.
	\end{enumerate}

	Use |\pgfplotsset{compat=default}| to restore the factory settings.

	The setting |\pgfplotsset{compat=newest}| will always use features of the most recent version. This might result in small changes in the document's appearance.
\end{pgfplotskey}

\subsection{The Team}
\PGFPlots\ has been written mainly by Christian Feuers�nger with many improvements of Pascal Wolkotte and Nick Papior Andersen as a spare time project. We hope it is useful and provides valuable plots.

If you are interested in writing something but don't know how, consider reading the auxiliary manual \href{file:TeX-programming-notes.pdf}{TeX-programming-notes.pdf} which comes with \PGFPlots. It is far from complete, but maybe it is a good starting point (at least for more literature).

\subsection{Acknowledgements}
I thank God for all hours of enjoyed programming. I thank Pascal Wolkotte and Nick Papior Andersen for their programming efforts and contributions as part of the development team. I thank Stefan Tibus, who contributed the |plot shell| feature. I thank Tom Cashman for the contribution of the |reverse legend| feature. Special thanks go to Stefan Pinnow whose tests of \PGFPlots\ lead to numerous quality improvements. Furthermore, I thank Dr.~Schweitzer for many fruitful discussions and Dr.~Meine for his ideas and suggestions. Thanks as well to the many international contributors who provided feature requests or identified bugs or simply manual improvements!

Last but not least, I thank Till Tantau and Mark Wibrow for their excellent graphics (and more) package \PGF\ and \Tikz\ which is the base of \PGFPlots.

\include{pgfplots.install}%
\include{pgfplots.intro}%
\include{pgfplots.reference}%
\include{pgfplots.libs}%
\include{pgfplots.resources}%
\include{pgfplots.importexport}%
\include{pgfplots.basic.reference}%

\printindex

\bibliographystyle{abbrv} %gerapali} %gerabbrv} %gerunsrt.bst} %gerabbrv}% gerplain}
\nocite{pgfplotstable}
\nocite{programmingnotes}
\bibliography{pgfplots}
\end{document}
%
%%%%%%%%%%%%%%%%%%%%%%%%%%%%%%%%%%%%%%%%%%%%%%%%%%%%%%%%%%%%%%%%%%%%%%%%%%%%%
%
% Package pgfplots.sty documentation. 
%
% Copyright 2007/2008 by Christian Feuersaenger.
%
% This program is free software: you can redistribute it and/or modify
% it under the terms of the GNU General Public License as published by
% the Free Software Foundation, either version 3 of the License, or
% (at your option) any later version.
% 
% This program is distributed in the hope that it will be useful,
% but WITHOUT ANY WARRANTY; without even the implied warranty of
% MERCHANTABILITY or FITNESS FOR A PARTICULAR PURPOSE.  See the
% GNU General Public License for more details.
% 
% You should have received a copy of the GNU General Public License
% along with this program.  If not, see <http://www.gnu.org/licenses/>.
%
%
%%%%%%%%%%%%%%%%%%%%%%%%%%%%%%%%%%%%%%%%%%%%%%%%%%%%%%%%%%%%%%%%%%%%%%%%%%%%%
\input{pgfplots.preamble.tex}
%\RequirePackage[german,english,francais]{babel}

\pgfplotsmanualexternalexpensivetrue

%\usetikzlibrary{external}
%\tikzexternalize[prefix=figures/]{pgfplots}

\title{%
	Manual for Package \PGFPlots\\
	{\small Version \pgfplotsversion}\\
	{\small\href{http://sourceforge.net/projects/pgfplots}{http://sourceforge.net/projects/pgfplots}}}

%\includeonly{pgfplots.libs}


\begin{document}

\def\plotcoords{%
\addplot coordinates {
(5,8.312e-02)    (17,2.547e-02)   (49,7.407e-03)
(129,2.102e-03)  (321,5.874e-04)  (769,1.623e-04)
(1793,4.442e-05) (4097,1.207e-05) (9217,3.261e-06)
};

\addplot coordinates{
(7,8.472e-02)    (31,3.044e-02)    (111,1.022e-02)
(351,3.303e-03)  (1023,1.039e-03)  (2815,3.196e-04)
(7423,9.658e-05) (18943,2.873e-05) (47103,8.437e-06)
};

\addplot coordinates{
(9,7.881e-02)     (49,3.243e-02)    (209,1.232e-02)
(769,4.454e-03)   (2561,1.551e-03)  (7937,5.236e-04)
(23297,1.723e-04) (65537,5.545e-05) (178177,1.751e-05)
};

\addplot coordinates{
(11,6.887e-02)    (71,3.177e-02)     (351,1.341e-02)
(1471,5.334e-03)  (5503,2.027e-03)   (18943,7.415e-04)
(61183,2.628e-04) (187903,9.063e-05) (553983,3.053e-05)
};

\addplot coordinates{
(13,5.755e-02)     (97,2.925e-02)     (545,1.351e-02)
(2561,5.842e-03)   (10625,2.397e-03)  (40193,9.414e-04)
(141569,3.564e-04) (471041,1.308e-04) 
(1496065,4.670e-05)
};
}%
\maketitle
\begin{abstract}%
\PGFPlots\ draws high--quality function plots in normal or logarithmic scaling with a user-friendly interface directly in \TeX. The user supplies axis labels, legend entries and the plot coordinates for one or more plots and \PGFPlots\ applies axis scaling, computes any logarithms and axis ticks and draws the plots, supporting line plots, scatter plots, piecewise constant plots, bar plots, area plots, mesh-- and surface plots and some more. It is based on Till Tantau's package \PGF/\Tikz.
\end{abstract}
\tableofcontents
\section{Introduction}
This package provides tools to generate plots and labeled axes easily. It draws normal plots, logplots and semi-logplots, in two and three dimensions. Axis ticks, labels, legends (in case of multiple plots) can be added with key-value options. It can cycle through a set of predefined line/marker/color specifications. In summary, its purpose is to simplify the generation of high-quality function and/or data plots, and solving the problems of
\begin{itemize}
	\item consistency of document and font type and font size,
	\item direct use of \TeX\ math mode in axis descriptions,
	\item consistency of data and figures (no third party tool necessary),
	\item inter document consistency using preamble configurations and styles.
\end{itemize}
Although not necessary, separate |.pdf| or |.eps| graphics can be generated using the |external| library developed as part of \Tikz.

\PGFPlots\ is build completely on \Tikz/\PGF. Knowledge of \Tikz\ will simplify the work with \PGFPlots, although it is not required.

\section[About PGFPlots: Preliminaries]{About {\normalfont\PGFPlots}: Preliminaries}
This section contains information about upgrades, the team, the installation (in case you need to do it manually) and troubleshooting. You may skip it completely except for the upgrade remarks.

However, note that this library requires at least \PGF\ version $2.00$. At the time of this writing, many \TeX-distributions still contain the older \PGF\ version $1.18$, so it may be necessary to install a recent \PGF\ prior to using \PGFPlots.

\subsection{Components}
\PGFPlots\ comes with two components:
\begin{enumerate}
	\item the plotting component (which you are currently reading) and
	\item the \PGFPlotstable\ component which simplifies number formatting and postprocessing of numerical tables. It comes as a separate package and has its own manual \href{file:pgfplotstable.pdf}{pgfplotstable.pdf}.
\end{enumerate}

\subsection{Upgrade remarks}
This release provides a lot of improvements which can be found in all detail in \texttt{ChangeLog} for interested readers. However, some attention is useful with respect to the following changes.

\subsubsection{New Optional Features}
\begin{enumerate}
	\item \PGFPlots\ 1.3 comes with user interface improvements. The technical distinction between ``behavior options'' and ``style options'' of older versions is no longer necessary (although still fully supported).

	\item \PGFPlots\ 1.3 has a new feature which allows to \emph{move axis labels tight to tick labels} automatically.

	Since this affects the spacing, it is not enabled be default.

	Use
\begin{codeexample}[code only]
\usepackage{pgfplots}
\pgfplotsset{compat=1.3}
\end{codeexample}
	in your preamble to benefit from the improved spacing\footnote{The |compat=1.3| setting is necessary for new features which move axis labels around like reversed axis. However, \PGFPlots\ will still work as it does in earlier versions, your documents will remain the same if you don't set it explicitly.}.
	Take a look at the next page for the precise definition of |compat|.

	\item \PGFPlots\ 1.3 now supports reversed axes. It is no longer necessary to use work--arounds with negative units.
\pgfkeys{/pdflinks/search key prefixes in/.add={/pgfplots/,}{}}

	Take a look at the |x dir=reverse| key.

	Existing work--arounds will still function properly. Use |\pgfplotsset{compat=1.3}| together with |x dir=reverse| to switch to the new version.
\end{enumerate}

\subsubsection{Old Features Which May Need Attention}
\begin{enumerate}
	\item The |scatter/classes| feature produces proper legends as of version 1.3. This may change the appearance of existing legends of plots with |scatter/classes|.

	\item Due to a small math bug in \PGF\ $2.00$, you \emph{can't} use the math expression `|-x^2|'. It is necessary to use `|0-x^2|' instead. The same holds for
		`|exp(-x^2)|'; use `|exp(0-x^2)|' instead.

		This will be fixed with \PGF\ $>2.00$.

	\item Starting with \PGFPlots\ $1.1$, |\tikzstyle| should \emph{no longer be used} to set \PGFPlots\ options.
	
	Although |\tikzstyle| is still supported for some older \PGFPlots\ options, you should replace any occurance of |\tikzstyle| with |\pgfplotsset{|\meta{style name}|/.style={|\meta{key-value-list}|}}| or the associated |/.append style| variant. See section~\ref{sec:styles} for more detail.
\end{enumerate}
I apologize for any inconvenience caused by these changes.

\begin{pgfplotskey}{compat=\mchoice{1.3,pre 1.3,default,newest} (initially default)}
	The preamble configuration 
\begin{codeexample}[code only]
\usepackage{pgfplots}
\pgfplotsset{compat=1.3}
\end{codeexample}
	allows to choose between backwards compatibility and most recent features.

	\PGFPlots\ version 1.3 comes with new features which may lead to different spacings compared to earlier versions:
	\begin{enumerate}
		\item Version 1.3 comes with the |xlabel near ticks| feature which places (in this case $x$) axis labels ``near'' the tick labels, taking the size of tick labels into account.
		
		Older versions used |xlabel absolute|, i.e.\ an absolute distance between the axis and axis labels (now matter how large or small tick labels are).

		The initial setting \emph{keeps backwards compatibility}.

		You are encouraged to use
\begin{codeexample}[code only]
\usepackage{pgfplots}
\pgfplotsset{compat=1.3}
\end{codeexample}
		in your preamble.
		
		\item Version 1.3 fixes a bug with |anchor=south west| which occurred mainly for reversed axes: the anchors where upside--down. This is fixed now.

		
		If you have written some work--around and want to keep it going, use

		|\pgfplotsset{compat/anchors=pre 1.3}|\pgfmanualpdflabel{/pgfplots/compat/anchors}{} 

		to disable the bug fix.
	\end{enumerate}

	Use |\pgfplotsset{compat=default}| to restore the factory settings.

	The setting |\pgfplotsset{compat=newest}| will always use features of the most recent version. This might result in small changes in the document's appearance.
\end{pgfplotskey}

\subsection{The Team}
\PGFPlots\ has been written mainly by Christian Feuers�nger with many improvements of Pascal Wolkotte and Nick Papior Andersen as a spare time project. We hope it is useful and provides valuable plots.

If you are interested in writing something but don't know how, consider reading the auxiliary manual \href{file:TeX-programming-notes.pdf}{TeX-programming-notes.pdf} which comes with \PGFPlots. It is far from complete, but maybe it is a good starting point (at least for more literature).

\subsection{Acknowledgements}
I thank God for all hours of enjoyed programming. I thank Pascal Wolkotte and Nick Papior Andersen for their programming efforts and contributions as part of the development team. I thank Stefan Tibus, who contributed the |plot shell| feature. I thank Tom Cashman for the contribution of the |reverse legend| feature. Special thanks go to Stefan Pinnow whose tests of \PGFPlots\ lead to numerous quality improvements. Furthermore, I thank Dr.~Schweitzer for many fruitful discussions and Dr.~Meine for his ideas and suggestions. Thanks as well to the many international contributors who provided feature requests or identified bugs or simply manual improvements!

Last but not least, I thank Till Tantau and Mark Wibrow for their excellent graphics (and more) package \PGF\ and \Tikz\ which is the base of \PGFPlots.

\include{pgfplots.install}%
\include{pgfplots.intro}%
\include{pgfplots.reference}%
\include{pgfplots.libs}%
\include{pgfplots.resources}%
\include{pgfplots.importexport}%
\include{pgfplots.basic.reference}%

\printindex

\bibliographystyle{abbrv} %gerapali} %gerabbrv} %gerunsrt.bst} %gerabbrv}% gerplain}
\nocite{pgfplotstable}
\nocite{programmingnotes}
\bibliography{pgfplots}
\end{document}
%
%%%%%%%%%%%%%%%%%%%%%%%%%%%%%%%%%%%%%%%%%%%%%%%%%%%%%%%%%%%%%%%%%%%%%%%%%%%%%
%
% Package pgfplots.sty documentation. 
%
% Copyright 2007/2008 by Christian Feuersaenger.
%
% This program is free software: you can redistribute it and/or modify
% it under the terms of the GNU General Public License as published by
% the Free Software Foundation, either version 3 of the License, or
% (at your option) any later version.
% 
% This program is distributed in the hope that it will be useful,
% but WITHOUT ANY WARRANTY; without even the implied warranty of
% MERCHANTABILITY or FITNESS FOR A PARTICULAR PURPOSE.  See the
% GNU General Public License for more details.
% 
% You should have received a copy of the GNU General Public License
% along with this program.  If not, see <http://www.gnu.org/licenses/>.
%
%
%%%%%%%%%%%%%%%%%%%%%%%%%%%%%%%%%%%%%%%%%%%%%%%%%%%%%%%%%%%%%%%%%%%%%%%%%%%%%
\input{pgfplots.preamble.tex}
%\RequirePackage[german,english,francais]{babel}

\pgfplotsmanualexternalexpensivetrue

%\usetikzlibrary{external}
%\tikzexternalize[prefix=figures/]{pgfplots}

\title{%
	Manual for Package \PGFPlots\\
	{\small Version \pgfplotsversion}\\
	{\small\href{http://sourceforge.net/projects/pgfplots}{http://sourceforge.net/projects/pgfplots}}}

%\includeonly{pgfplots.libs}


\begin{document}

\def\plotcoords{%
\addplot coordinates {
(5,8.312e-02)    (17,2.547e-02)   (49,7.407e-03)
(129,2.102e-03)  (321,5.874e-04)  (769,1.623e-04)
(1793,4.442e-05) (4097,1.207e-05) (9217,3.261e-06)
};

\addplot coordinates{
(7,8.472e-02)    (31,3.044e-02)    (111,1.022e-02)
(351,3.303e-03)  (1023,1.039e-03)  (2815,3.196e-04)
(7423,9.658e-05) (18943,2.873e-05) (47103,8.437e-06)
};

\addplot coordinates{
(9,7.881e-02)     (49,3.243e-02)    (209,1.232e-02)
(769,4.454e-03)   (2561,1.551e-03)  (7937,5.236e-04)
(23297,1.723e-04) (65537,5.545e-05) (178177,1.751e-05)
};

\addplot coordinates{
(11,6.887e-02)    (71,3.177e-02)     (351,1.341e-02)
(1471,5.334e-03)  (5503,2.027e-03)   (18943,7.415e-04)
(61183,2.628e-04) (187903,9.063e-05) (553983,3.053e-05)
};

\addplot coordinates{
(13,5.755e-02)     (97,2.925e-02)     (545,1.351e-02)
(2561,5.842e-03)   (10625,2.397e-03)  (40193,9.414e-04)
(141569,3.564e-04) (471041,1.308e-04) 
(1496065,4.670e-05)
};
}%
\maketitle
\begin{abstract}%
\PGFPlots\ draws high--quality function plots in normal or logarithmic scaling with a user-friendly interface directly in \TeX. The user supplies axis labels, legend entries and the plot coordinates for one or more plots and \PGFPlots\ applies axis scaling, computes any logarithms and axis ticks and draws the plots, supporting line plots, scatter plots, piecewise constant plots, bar plots, area plots, mesh-- and surface plots and some more. It is based on Till Tantau's package \PGF/\Tikz.
\end{abstract}
\tableofcontents
\section{Introduction}
This package provides tools to generate plots and labeled axes easily. It draws normal plots, logplots and semi-logplots, in two and three dimensions. Axis ticks, labels, legends (in case of multiple plots) can be added with key-value options. It can cycle through a set of predefined line/marker/color specifications. In summary, its purpose is to simplify the generation of high-quality function and/or data plots, and solving the problems of
\begin{itemize}
	\item consistency of document and font type and font size,
	\item direct use of \TeX\ math mode in axis descriptions,
	\item consistency of data and figures (no third party tool necessary),
	\item inter document consistency using preamble configurations and styles.
\end{itemize}
Although not necessary, separate |.pdf| or |.eps| graphics can be generated using the |external| library developed as part of \Tikz.

\PGFPlots\ is build completely on \Tikz/\PGF. Knowledge of \Tikz\ will simplify the work with \PGFPlots, although it is not required.

\section[About PGFPlots: Preliminaries]{About {\normalfont\PGFPlots}: Preliminaries}
This section contains information about upgrades, the team, the installation (in case you need to do it manually) and troubleshooting. You may skip it completely except for the upgrade remarks.

However, note that this library requires at least \PGF\ version $2.00$. At the time of this writing, many \TeX-distributions still contain the older \PGF\ version $1.18$, so it may be necessary to install a recent \PGF\ prior to using \PGFPlots.

\subsection{Components}
\PGFPlots\ comes with two components:
\begin{enumerate}
	\item the plotting component (which you are currently reading) and
	\item the \PGFPlotstable\ component which simplifies number formatting and postprocessing of numerical tables. It comes as a separate package and has its own manual \href{file:pgfplotstable.pdf}{pgfplotstable.pdf}.
\end{enumerate}

\subsection{Upgrade remarks}
This release provides a lot of improvements which can be found in all detail in \texttt{ChangeLog} for interested readers. However, some attention is useful with respect to the following changes.

\subsubsection{New Optional Features}
\begin{enumerate}
	\item \PGFPlots\ 1.3 comes with user interface improvements. The technical distinction between ``behavior options'' and ``style options'' of older versions is no longer necessary (although still fully supported).

	\item \PGFPlots\ 1.3 has a new feature which allows to \emph{move axis labels tight to tick labels} automatically.

	Since this affects the spacing, it is not enabled be default.

	Use
\begin{codeexample}[code only]
\usepackage{pgfplots}
\pgfplotsset{compat=1.3}
\end{codeexample}
	in your preamble to benefit from the improved spacing\footnote{The |compat=1.3| setting is necessary for new features which move axis labels around like reversed axis. However, \PGFPlots\ will still work as it does in earlier versions, your documents will remain the same if you don't set it explicitly.}.
	Take a look at the next page for the precise definition of |compat|.

	\item \PGFPlots\ 1.3 now supports reversed axes. It is no longer necessary to use work--arounds with negative units.
\pgfkeys{/pdflinks/search key prefixes in/.add={/pgfplots/,}{}}

	Take a look at the |x dir=reverse| key.

	Existing work--arounds will still function properly. Use |\pgfplotsset{compat=1.3}| together with |x dir=reverse| to switch to the new version.
\end{enumerate}

\subsubsection{Old Features Which May Need Attention}
\begin{enumerate}
	\item The |scatter/classes| feature produces proper legends as of version 1.3. This may change the appearance of existing legends of plots with |scatter/classes|.

	\item Due to a small math bug in \PGF\ $2.00$, you \emph{can't} use the math expression `|-x^2|'. It is necessary to use `|0-x^2|' instead. The same holds for
		`|exp(-x^2)|'; use `|exp(0-x^2)|' instead.

		This will be fixed with \PGF\ $>2.00$.

	\item Starting with \PGFPlots\ $1.1$, |\tikzstyle| should \emph{no longer be used} to set \PGFPlots\ options.
	
	Although |\tikzstyle| is still supported for some older \PGFPlots\ options, you should replace any occurance of |\tikzstyle| with |\pgfplotsset{|\meta{style name}|/.style={|\meta{key-value-list}|}}| or the associated |/.append style| variant. See section~\ref{sec:styles} for more detail.
\end{enumerate}
I apologize for any inconvenience caused by these changes.

\begin{pgfplotskey}{compat=\mchoice{1.3,pre 1.3,default,newest} (initially default)}
	The preamble configuration 
\begin{codeexample}[code only]
\usepackage{pgfplots}
\pgfplotsset{compat=1.3}
\end{codeexample}
	allows to choose between backwards compatibility and most recent features.

	\PGFPlots\ version 1.3 comes with new features which may lead to different spacings compared to earlier versions:
	\begin{enumerate}
		\item Version 1.3 comes with the |xlabel near ticks| feature which places (in this case $x$) axis labels ``near'' the tick labels, taking the size of tick labels into account.
		
		Older versions used |xlabel absolute|, i.e.\ an absolute distance between the axis and axis labels (now matter how large or small tick labels are).

		The initial setting \emph{keeps backwards compatibility}.

		You are encouraged to use
\begin{codeexample}[code only]
\usepackage{pgfplots}
\pgfplotsset{compat=1.3}
\end{codeexample}
		in your preamble.
		
		\item Version 1.3 fixes a bug with |anchor=south west| which occurred mainly for reversed axes: the anchors where upside--down. This is fixed now.

		
		If you have written some work--around and want to keep it going, use

		|\pgfplotsset{compat/anchors=pre 1.3}|\pgfmanualpdflabel{/pgfplots/compat/anchors}{} 

		to disable the bug fix.
	\end{enumerate}

	Use |\pgfplotsset{compat=default}| to restore the factory settings.

	The setting |\pgfplotsset{compat=newest}| will always use features of the most recent version. This might result in small changes in the document's appearance.
\end{pgfplotskey}

\subsection{The Team}
\PGFPlots\ has been written mainly by Christian Feuers�nger with many improvements of Pascal Wolkotte and Nick Papior Andersen as a spare time project. We hope it is useful and provides valuable plots.

If you are interested in writing something but don't know how, consider reading the auxiliary manual \href{file:TeX-programming-notes.pdf}{TeX-programming-notes.pdf} which comes with \PGFPlots. It is far from complete, but maybe it is a good starting point (at least for more literature).

\subsection{Acknowledgements}
I thank God for all hours of enjoyed programming. I thank Pascal Wolkotte and Nick Papior Andersen for their programming efforts and contributions as part of the development team. I thank Stefan Tibus, who contributed the |plot shell| feature. I thank Tom Cashman for the contribution of the |reverse legend| feature. Special thanks go to Stefan Pinnow whose tests of \PGFPlots\ lead to numerous quality improvements. Furthermore, I thank Dr.~Schweitzer for many fruitful discussions and Dr.~Meine for his ideas and suggestions. Thanks as well to the many international contributors who provided feature requests or identified bugs or simply manual improvements!

Last but not least, I thank Till Tantau and Mark Wibrow for their excellent graphics (and more) package \PGF\ and \Tikz\ which is the base of \PGFPlots.

\include{pgfplots.install}%
\include{pgfplots.intro}%
\include{pgfplots.reference}%
\include{pgfplots.libs}%
\include{pgfplots.resources}%
\include{pgfplots.importexport}%
\include{pgfplots.basic.reference}%

\printindex

\bibliographystyle{abbrv} %gerapali} %gerabbrv} %gerunsrt.bst} %gerabbrv}% gerplain}
\nocite{pgfplotstable}
\nocite{programmingnotes}
\bibliography{pgfplots}
\end{document}
%
%%%%%%%%%%%%%%%%%%%%%%%%%%%%%%%%%%%%%%%%%%%%%%%%%%%%%%%%%%%%%%%%%%%%%%%%%%%%%
%
% Package pgfplots.sty documentation. 
%
% Copyright 2007/2008 by Christian Feuersaenger.
%
% This program is free software: you can redistribute it and/or modify
% it under the terms of the GNU General Public License as published by
% the Free Software Foundation, either version 3 of the License, or
% (at your option) any later version.
% 
% This program is distributed in the hope that it will be useful,
% but WITHOUT ANY WARRANTY; without even the implied warranty of
% MERCHANTABILITY or FITNESS FOR A PARTICULAR PURPOSE.  See the
% GNU General Public License for more details.
% 
% You should have received a copy of the GNU General Public License
% along with this program.  If not, see <http://www.gnu.org/licenses/>.
%
%
%%%%%%%%%%%%%%%%%%%%%%%%%%%%%%%%%%%%%%%%%%%%%%%%%%%%%%%%%%%%%%%%%%%%%%%%%%%%%
\input{pgfplots.preamble.tex}
%\RequirePackage[german,english,francais]{babel}

\pgfplotsmanualexternalexpensivetrue

%\usetikzlibrary{external}
%\tikzexternalize[prefix=figures/]{pgfplots}

\title{%
	Manual for Package \PGFPlots\\
	{\small Version \pgfplotsversion}\\
	{\small\href{http://sourceforge.net/projects/pgfplots}{http://sourceforge.net/projects/pgfplots}}}

%\includeonly{pgfplots.libs}


\begin{document}

\def\plotcoords{%
\addplot coordinates {
(5,8.312e-02)    (17,2.547e-02)   (49,7.407e-03)
(129,2.102e-03)  (321,5.874e-04)  (769,1.623e-04)
(1793,4.442e-05) (4097,1.207e-05) (9217,3.261e-06)
};

\addplot coordinates{
(7,8.472e-02)    (31,3.044e-02)    (111,1.022e-02)
(351,3.303e-03)  (1023,1.039e-03)  (2815,3.196e-04)
(7423,9.658e-05) (18943,2.873e-05) (47103,8.437e-06)
};

\addplot coordinates{
(9,7.881e-02)     (49,3.243e-02)    (209,1.232e-02)
(769,4.454e-03)   (2561,1.551e-03)  (7937,5.236e-04)
(23297,1.723e-04) (65537,5.545e-05) (178177,1.751e-05)
};

\addplot coordinates{
(11,6.887e-02)    (71,3.177e-02)     (351,1.341e-02)
(1471,5.334e-03)  (5503,2.027e-03)   (18943,7.415e-04)
(61183,2.628e-04) (187903,9.063e-05) (553983,3.053e-05)
};

\addplot coordinates{
(13,5.755e-02)     (97,2.925e-02)     (545,1.351e-02)
(2561,5.842e-03)   (10625,2.397e-03)  (40193,9.414e-04)
(141569,3.564e-04) (471041,1.308e-04) 
(1496065,4.670e-05)
};
}%
\maketitle
\begin{abstract}%
\PGFPlots\ draws high--quality function plots in normal or logarithmic scaling with a user-friendly interface directly in \TeX. The user supplies axis labels, legend entries and the plot coordinates for one or more plots and \PGFPlots\ applies axis scaling, computes any logarithms and axis ticks and draws the plots, supporting line plots, scatter plots, piecewise constant plots, bar plots, area plots, mesh-- and surface plots and some more. It is based on Till Tantau's package \PGF/\Tikz.
\end{abstract}
\tableofcontents
\section{Introduction}
This package provides tools to generate plots and labeled axes easily. It draws normal plots, logplots and semi-logplots, in two and three dimensions. Axis ticks, labels, legends (in case of multiple plots) can be added with key-value options. It can cycle through a set of predefined line/marker/color specifications. In summary, its purpose is to simplify the generation of high-quality function and/or data plots, and solving the problems of
\begin{itemize}
	\item consistency of document and font type and font size,
	\item direct use of \TeX\ math mode in axis descriptions,
	\item consistency of data and figures (no third party tool necessary),
	\item inter document consistency using preamble configurations and styles.
\end{itemize}
Although not necessary, separate |.pdf| or |.eps| graphics can be generated using the |external| library developed as part of \Tikz.

\PGFPlots\ is build completely on \Tikz/\PGF. Knowledge of \Tikz\ will simplify the work with \PGFPlots, although it is not required.

\section[About PGFPlots: Preliminaries]{About {\normalfont\PGFPlots}: Preliminaries}
This section contains information about upgrades, the team, the installation (in case you need to do it manually) and troubleshooting. You may skip it completely except for the upgrade remarks.

However, note that this library requires at least \PGF\ version $2.00$. At the time of this writing, many \TeX-distributions still contain the older \PGF\ version $1.18$, so it may be necessary to install a recent \PGF\ prior to using \PGFPlots.

\subsection{Components}
\PGFPlots\ comes with two components:
\begin{enumerate}
	\item the plotting component (which you are currently reading) and
	\item the \PGFPlotstable\ component which simplifies number formatting and postprocessing of numerical tables. It comes as a separate package and has its own manual \href{file:pgfplotstable.pdf}{pgfplotstable.pdf}.
\end{enumerate}

\subsection{Upgrade remarks}
This release provides a lot of improvements which can be found in all detail in \texttt{ChangeLog} for interested readers. However, some attention is useful with respect to the following changes.

\subsubsection{New Optional Features}
\begin{enumerate}
	\item \PGFPlots\ 1.3 comes with user interface improvements. The technical distinction between ``behavior options'' and ``style options'' of older versions is no longer necessary (although still fully supported).

	\item \PGFPlots\ 1.3 has a new feature which allows to \emph{move axis labels tight to tick labels} automatically.

	Since this affects the spacing, it is not enabled be default.

	Use
\begin{codeexample}[code only]
\usepackage{pgfplots}
\pgfplotsset{compat=1.3}
\end{codeexample}
	in your preamble to benefit from the improved spacing\footnote{The |compat=1.3| setting is necessary for new features which move axis labels around like reversed axis. However, \PGFPlots\ will still work as it does in earlier versions, your documents will remain the same if you don't set it explicitly.}.
	Take a look at the next page for the precise definition of |compat|.

	\item \PGFPlots\ 1.3 now supports reversed axes. It is no longer necessary to use work--arounds with negative units.
\pgfkeys{/pdflinks/search key prefixes in/.add={/pgfplots/,}{}}

	Take a look at the |x dir=reverse| key.

	Existing work--arounds will still function properly. Use |\pgfplotsset{compat=1.3}| together with |x dir=reverse| to switch to the new version.
\end{enumerate}

\subsubsection{Old Features Which May Need Attention}
\begin{enumerate}
	\item The |scatter/classes| feature produces proper legends as of version 1.3. This may change the appearance of existing legends of plots with |scatter/classes|.

	\item Due to a small math bug in \PGF\ $2.00$, you \emph{can't} use the math expression `|-x^2|'. It is necessary to use `|0-x^2|' instead. The same holds for
		`|exp(-x^2)|'; use `|exp(0-x^2)|' instead.

		This will be fixed with \PGF\ $>2.00$.

	\item Starting with \PGFPlots\ $1.1$, |\tikzstyle| should \emph{no longer be used} to set \PGFPlots\ options.
	
	Although |\tikzstyle| is still supported for some older \PGFPlots\ options, you should replace any occurance of |\tikzstyle| with |\pgfplotsset{|\meta{style name}|/.style={|\meta{key-value-list}|}}| or the associated |/.append style| variant. See section~\ref{sec:styles} for more detail.
\end{enumerate}
I apologize for any inconvenience caused by these changes.

\begin{pgfplotskey}{compat=\mchoice{1.3,pre 1.3,default,newest} (initially default)}
	The preamble configuration 
\begin{codeexample}[code only]
\usepackage{pgfplots}
\pgfplotsset{compat=1.3}
\end{codeexample}
	allows to choose between backwards compatibility and most recent features.

	\PGFPlots\ version 1.3 comes with new features which may lead to different spacings compared to earlier versions:
	\begin{enumerate}
		\item Version 1.3 comes with the |xlabel near ticks| feature which places (in this case $x$) axis labels ``near'' the tick labels, taking the size of tick labels into account.
		
		Older versions used |xlabel absolute|, i.e.\ an absolute distance between the axis and axis labels (now matter how large or small tick labels are).

		The initial setting \emph{keeps backwards compatibility}.

		You are encouraged to use
\begin{codeexample}[code only]
\usepackage{pgfplots}
\pgfplotsset{compat=1.3}
\end{codeexample}
		in your preamble.
		
		\item Version 1.3 fixes a bug with |anchor=south west| which occurred mainly for reversed axes: the anchors where upside--down. This is fixed now.

		
		If you have written some work--around and want to keep it going, use

		|\pgfplotsset{compat/anchors=pre 1.3}|\pgfmanualpdflabel{/pgfplots/compat/anchors}{} 

		to disable the bug fix.
	\end{enumerate}

	Use |\pgfplotsset{compat=default}| to restore the factory settings.

	The setting |\pgfplotsset{compat=newest}| will always use features of the most recent version. This might result in small changes in the document's appearance.
\end{pgfplotskey}

\subsection{The Team}
\PGFPlots\ has been written mainly by Christian Feuers�nger with many improvements of Pascal Wolkotte and Nick Papior Andersen as a spare time project. We hope it is useful and provides valuable plots.

If you are interested in writing something but don't know how, consider reading the auxiliary manual \href{file:TeX-programming-notes.pdf}{TeX-programming-notes.pdf} which comes with \PGFPlots. It is far from complete, but maybe it is a good starting point (at least for more literature).

\subsection{Acknowledgements}
I thank God for all hours of enjoyed programming. I thank Pascal Wolkotte and Nick Papior Andersen for their programming efforts and contributions as part of the development team. I thank Stefan Tibus, who contributed the |plot shell| feature. I thank Tom Cashman for the contribution of the |reverse legend| feature. Special thanks go to Stefan Pinnow whose tests of \PGFPlots\ lead to numerous quality improvements. Furthermore, I thank Dr.~Schweitzer for many fruitful discussions and Dr.~Meine for his ideas and suggestions. Thanks as well to the many international contributors who provided feature requests or identified bugs or simply manual improvements!

Last but not least, I thank Till Tantau and Mark Wibrow for their excellent graphics (and more) package \PGF\ and \Tikz\ which is the base of \PGFPlots.

\include{pgfplots.install}%
\include{pgfplots.intro}%
\include{pgfplots.reference}%
\include{pgfplots.libs}%
\include{pgfplots.resources}%
\include{pgfplots.importexport}%
\include{pgfplots.basic.reference}%

\printindex

\bibliographystyle{abbrv} %gerapali} %gerabbrv} %gerunsrt.bst} %gerabbrv}% gerplain}
\nocite{pgfplotstable}
\nocite{programmingnotes}
\bibliography{pgfplots}
\end{document}
%
%%%%%%%%%%%%%%%%%%%%%%%%%%%%%%%%%%%%%%%%%%%%%%%%%%%%%%%%%%%%%%%%%%%%%%%%%%%%%
%
% Package pgfplots.sty documentation. 
%
% Copyright 2007/2008 by Christian Feuersaenger.
%
% This program is free software: you can redistribute it and/or modify
% it under the terms of the GNU General Public License as published by
% the Free Software Foundation, either version 3 of the License, or
% (at your option) any later version.
% 
% This program is distributed in the hope that it will be useful,
% but WITHOUT ANY WARRANTY; without even the implied warranty of
% MERCHANTABILITY or FITNESS FOR A PARTICULAR PURPOSE.  See the
% GNU General Public License for more details.
% 
% You should have received a copy of the GNU General Public License
% along with this program.  If not, see <http://www.gnu.org/licenses/>.
%
%
%%%%%%%%%%%%%%%%%%%%%%%%%%%%%%%%%%%%%%%%%%%%%%%%%%%%%%%%%%%%%%%%%%%%%%%%%%%%%
\input{pgfplots.preamble.tex}
%\RequirePackage[german,english,francais]{babel}

\pgfplotsmanualexternalexpensivetrue

%\usetikzlibrary{external}
%\tikzexternalize[prefix=figures/]{pgfplots}

\title{%
	Manual for Package \PGFPlots\\
	{\small Version \pgfplotsversion}\\
	{\small\href{http://sourceforge.net/projects/pgfplots}{http://sourceforge.net/projects/pgfplots}}}

%\includeonly{pgfplots.libs}


\begin{document}

\def\plotcoords{%
\addplot coordinates {
(5,8.312e-02)    (17,2.547e-02)   (49,7.407e-03)
(129,2.102e-03)  (321,5.874e-04)  (769,1.623e-04)
(1793,4.442e-05) (4097,1.207e-05) (9217,3.261e-06)
};

\addplot coordinates{
(7,8.472e-02)    (31,3.044e-02)    (111,1.022e-02)
(351,3.303e-03)  (1023,1.039e-03)  (2815,3.196e-04)
(7423,9.658e-05) (18943,2.873e-05) (47103,8.437e-06)
};

\addplot coordinates{
(9,7.881e-02)     (49,3.243e-02)    (209,1.232e-02)
(769,4.454e-03)   (2561,1.551e-03)  (7937,5.236e-04)
(23297,1.723e-04) (65537,5.545e-05) (178177,1.751e-05)
};

\addplot coordinates{
(11,6.887e-02)    (71,3.177e-02)     (351,1.341e-02)
(1471,5.334e-03)  (5503,2.027e-03)   (18943,7.415e-04)
(61183,2.628e-04) (187903,9.063e-05) (553983,3.053e-05)
};

\addplot coordinates{
(13,5.755e-02)     (97,2.925e-02)     (545,1.351e-02)
(2561,5.842e-03)   (10625,2.397e-03)  (40193,9.414e-04)
(141569,3.564e-04) (471041,1.308e-04) 
(1496065,4.670e-05)
};
}%
\maketitle
\begin{abstract}%
\PGFPlots\ draws high--quality function plots in normal or logarithmic scaling with a user-friendly interface directly in \TeX. The user supplies axis labels, legend entries and the plot coordinates for one or more plots and \PGFPlots\ applies axis scaling, computes any logarithms and axis ticks and draws the plots, supporting line plots, scatter plots, piecewise constant plots, bar plots, area plots, mesh-- and surface plots and some more. It is based on Till Tantau's package \PGF/\Tikz.
\end{abstract}
\tableofcontents
\section{Introduction}
This package provides tools to generate plots and labeled axes easily. It draws normal plots, logplots and semi-logplots, in two and three dimensions. Axis ticks, labels, legends (in case of multiple plots) can be added with key-value options. It can cycle through a set of predefined line/marker/color specifications. In summary, its purpose is to simplify the generation of high-quality function and/or data plots, and solving the problems of
\begin{itemize}
	\item consistency of document and font type and font size,
	\item direct use of \TeX\ math mode in axis descriptions,
	\item consistency of data and figures (no third party tool necessary),
	\item inter document consistency using preamble configurations and styles.
\end{itemize}
Although not necessary, separate |.pdf| or |.eps| graphics can be generated using the |external| library developed as part of \Tikz.

\PGFPlots\ is build completely on \Tikz/\PGF. Knowledge of \Tikz\ will simplify the work with \PGFPlots, although it is not required.

\section[About PGFPlots: Preliminaries]{About {\normalfont\PGFPlots}: Preliminaries}
This section contains information about upgrades, the team, the installation (in case you need to do it manually) and troubleshooting. You may skip it completely except for the upgrade remarks.

However, note that this library requires at least \PGF\ version $2.00$. At the time of this writing, many \TeX-distributions still contain the older \PGF\ version $1.18$, so it may be necessary to install a recent \PGF\ prior to using \PGFPlots.

\subsection{Components}
\PGFPlots\ comes with two components:
\begin{enumerate}
	\item the plotting component (which you are currently reading) and
	\item the \PGFPlotstable\ component which simplifies number formatting and postprocessing of numerical tables. It comes as a separate package and has its own manual \href{file:pgfplotstable.pdf}{pgfplotstable.pdf}.
\end{enumerate}

\subsection{Upgrade remarks}
This release provides a lot of improvements which can be found in all detail in \texttt{ChangeLog} for interested readers. However, some attention is useful with respect to the following changes.

\subsubsection{New Optional Features}
\begin{enumerate}
	\item \PGFPlots\ 1.3 comes with user interface improvements. The technical distinction between ``behavior options'' and ``style options'' of older versions is no longer necessary (although still fully supported).

	\item \PGFPlots\ 1.3 has a new feature which allows to \emph{move axis labels tight to tick labels} automatically.

	Since this affects the spacing, it is not enabled be default.

	Use
\begin{codeexample}[code only]
\usepackage{pgfplots}
\pgfplotsset{compat=1.3}
\end{codeexample}
	in your preamble to benefit from the improved spacing\footnote{The |compat=1.3| setting is necessary for new features which move axis labels around like reversed axis. However, \PGFPlots\ will still work as it does in earlier versions, your documents will remain the same if you don't set it explicitly.}.
	Take a look at the next page for the precise definition of |compat|.

	\item \PGFPlots\ 1.3 now supports reversed axes. It is no longer necessary to use work--arounds with negative units.
\pgfkeys{/pdflinks/search key prefixes in/.add={/pgfplots/,}{}}

	Take a look at the |x dir=reverse| key.

	Existing work--arounds will still function properly. Use |\pgfplotsset{compat=1.3}| together with |x dir=reverse| to switch to the new version.
\end{enumerate}

\subsubsection{Old Features Which May Need Attention}
\begin{enumerate}
	\item The |scatter/classes| feature produces proper legends as of version 1.3. This may change the appearance of existing legends of plots with |scatter/classes|.

	\item Due to a small math bug in \PGF\ $2.00$, you \emph{can't} use the math expression `|-x^2|'. It is necessary to use `|0-x^2|' instead. The same holds for
		`|exp(-x^2)|'; use `|exp(0-x^2)|' instead.

		This will be fixed with \PGF\ $>2.00$.

	\item Starting with \PGFPlots\ $1.1$, |\tikzstyle| should \emph{no longer be used} to set \PGFPlots\ options.
	
	Although |\tikzstyle| is still supported for some older \PGFPlots\ options, you should replace any occurance of |\tikzstyle| with |\pgfplotsset{|\meta{style name}|/.style={|\meta{key-value-list}|}}| or the associated |/.append style| variant. See section~\ref{sec:styles} for more detail.
\end{enumerate}
I apologize for any inconvenience caused by these changes.

\begin{pgfplotskey}{compat=\mchoice{1.3,pre 1.3,default,newest} (initially default)}
	The preamble configuration 
\begin{codeexample}[code only]
\usepackage{pgfplots}
\pgfplotsset{compat=1.3}
\end{codeexample}
	allows to choose between backwards compatibility and most recent features.

	\PGFPlots\ version 1.3 comes with new features which may lead to different spacings compared to earlier versions:
	\begin{enumerate}
		\item Version 1.3 comes with the |xlabel near ticks| feature which places (in this case $x$) axis labels ``near'' the tick labels, taking the size of tick labels into account.
		
		Older versions used |xlabel absolute|, i.e.\ an absolute distance between the axis and axis labels (now matter how large or small tick labels are).

		The initial setting \emph{keeps backwards compatibility}.

		You are encouraged to use
\begin{codeexample}[code only]
\usepackage{pgfplots}
\pgfplotsset{compat=1.3}
\end{codeexample}
		in your preamble.
		
		\item Version 1.3 fixes a bug with |anchor=south west| which occurred mainly for reversed axes: the anchors where upside--down. This is fixed now.

		
		If you have written some work--around and want to keep it going, use

		|\pgfplotsset{compat/anchors=pre 1.3}|\pgfmanualpdflabel{/pgfplots/compat/anchors}{} 

		to disable the bug fix.
	\end{enumerate}

	Use |\pgfplotsset{compat=default}| to restore the factory settings.

	The setting |\pgfplotsset{compat=newest}| will always use features of the most recent version. This might result in small changes in the document's appearance.
\end{pgfplotskey}

\subsection{The Team}
\PGFPlots\ has been written mainly by Christian Feuers�nger with many improvements of Pascal Wolkotte and Nick Papior Andersen as a spare time project. We hope it is useful and provides valuable plots.

If you are interested in writing something but don't know how, consider reading the auxiliary manual \href{file:TeX-programming-notes.pdf}{TeX-programming-notes.pdf} which comes with \PGFPlots. It is far from complete, but maybe it is a good starting point (at least for more literature).

\subsection{Acknowledgements}
I thank God for all hours of enjoyed programming. I thank Pascal Wolkotte and Nick Papior Andersen for their programming efforts and contributions as part of the development team. I thank Stefan Tibus, who contributed the |plot shell| feature. I thank Tom Cashman for the contribution of the |reverse legend| feature. Special thanks go to Stefan Pinnow whose tests of \PGFPlots\ lead to numerous quality improvements. Furthermore, I thank Dr.~Schweitzer for many fruitful discussions and Dr.~Meine for his ideas and suggestions. Thanks as well to the many international contributors who provided feature requests or identified bugs or simply manual improvements!

Last but not least, I thank Till Tantau and Mark Wibrow for their excellent graphics (and more) package \PGF\ and \Tikz\ which is the base of \PGFPlots.

\include{pgfplots.install}%
\include{pgfplots.intro}%
\include{pgfplots.reference}%
\include{pgfplots.libs}%
\include{pgfplots.resources}%
\include{pgfplots.importexport}%
\include{pgfplots.basic.reference}%

\printindex

\bibliographystyle{abbrv} %gerapali} %gerabbrv} %gerunsrt.bst} %gerabbrv}% gerplain}
\nocite{pgfplotstable}
\nocite{programmingnotes}
\bibliography{pgfplots}
\end{document}
%
%%%%%%%%%%%%%%%%%%%%%%%%%%%%%%%%%%%%%%%%%%%%%%%%%%%%%%%%%%%%%%%%%%%%%%%%%%%%%
%
% Package pgfplots.sty documentation. 
%
% Copyright 2007/2008 by Christian Feuersaenger.
%
% This program is free software: you can redistribute it and/or modify
% it under the terms of the GNU General Public License as published by
% the Free Software Foundation, either version 3 of the License, or
% (at your option) any later version.
% 
% This program is distributed in the hope that it will be useful,
% but WITHOUT ANY WARRANTY; without even the implied warranty of
% MERCHANTABILITY or FITNESS FOR A PARTICULAR PURPOSE.  See the
% GNU General Public License for more details.
% 
% You should have received a copy of the GNU General Public License
% along with this program.  If not, see <http://www.gnu.org/licenses/>.
%
%
%%%%%%%%%%%%%%%%%%%%%%%%%%%%%%%%%%%%%%%%%%%%%%%%%%%%%%%%%%%%%%%%%%%%%%%%%%%%%
\input{pgfplots.preamble.tex}
%\RequirePackage[german,english,francais]{babel}

\pgfplotsmanualexternalexpensivetrue

%\usetikzlibrary{external}
%\tikzexternalize[prefix=figures/]{pgfplots}

\title{%
	Manual for Package \PGFPlots\\
	{\small Version \pgfplotsversion}\\
	{\small\href{http://sourceforge.net/projects/pgfplots}{http://sourceforge.net/projects/pgfplots}}}

%\includeonly{pgfplots.libs}


\begin{document}

\def\plotcoords{%
\addplot coordinates {
(5,8.312e-02)    (17,2.547e-02)   (49,7.407e-03)
(129,2.102e-03)  (321,5.874e-04)  (769,1.623e-04)
(1793,4.442e-05) (4097,1.207e-05) (9217,3.261e-06)
};

\addplot coordinates{
(7,8.472e-02)    (31,3.044e-02)    (111,1.022e-02)
(351,3.303e-03)  (1023,1.039e-03)  (2815,3.196e-04)
(7423,9.658e-05) (18943,2.873e-05) (47103,8.437e-06)
};

\addplot coordinates{
(9,7.881e-02)     (49,3.243e-02)    (209,1.232e-02)
(769,4.454e-03)   (2561,1.551e-03)  (7937,5.236e-04)
(23297,1.723e-04) (65537,5.545e-05) (178177,1.751e-05)
};

\addplot coordinates{
(11,6.887e-02)    (71,3.177e-02)     (351,1.341e-02)
(1471,5.334e-03)  (5503,2.027e-03)   (18943,7.415e-04)
(61183,2.628e-04) (187903,9.063e-05) (553983,3.053e-05)
};

\addplot coordinates{
(13,5.755e-02)     (97,2.925e-02)     (545,1.351e-02)
(2561,5.842e-03)   (10625,2.397e-03)  (40193,9.414e-04)
(141569,3.564e-04) (471041,1.308e-04) 
(1496065,4.670e-05)
};
}%
\maketitle
\begin{abstract}%
\PGFPlots\ draws high--quality function plots in normal or logarithmic scaling with a user-friendly interface directly in \TeX. The user supplies axis labels, legend entries and the plot coordinates for one or more plots and \PGFPlots\ applies axis scaling, computes any logarithms and axis ticks and draws the plots, supporting line plots, scatter plots, piecewise constant plots, bar plots, area plots, mesh-- and surface plots and some more. It is based on Till Tantau's package \PGF/\Tikz.
\end{abstract}
\tableofcontents
\section{Introduction}
This package provides tools to generate plots and labeled axes easily. It draws normal plots, logplots and semi-logplots, in two and three dimensions. Axis ticks, labels, legends (in case of multiple plots) can be added with key-value options. It can cycle through a set of predefined line/marker/color specifications. In summary, its purpose is to simplify the generation of high-quality function and/or data plots, and solving the problems of
\begin{itemize}
	\item consistency of document and font type and font size,
	\item direct use of \TeX\ math mode in axis descriptions,
	\item consistency of data and figures (no third party tool necessary),
	\item inter document consistency using preamble configurations and styles.
\end{itemize}
Although not necessary, separate |.pdf| or |.eps| graphics can be generated using the |external| library developed as part of \Tikz.

\PGFPlots\ is build completely on \Tikz/\PGF. Knowledge of \Tikz\ will simplify the work with \PGFPlots, although it is not required.

\section[About PGFPlots: Preliminaries]{About {\normalfont\PGFPlots}: Preliminaries}
This section contains information about upgrades, the team, the installation (in case you need to do it manually) and troubleshooting. You may skip it completely except for the upgrade remarks.

However, note that this library requires at least \PGF\ version $2.00$. At the time of this writing, many \TeX-distributions still contain the older \PGF\ version $1.18$, so it may be necessary to install a recent \PGF\ prior to using \PGFPlots.

\subsection{Components}
\PGFPlots\ comes with two components:
\begin{enumerate}
	\item the plotting component (which you are currently reading) and
	\item the \PGFPlotstable\ component which simplifies number formatting and postprocessing of numerical tables. It comes as a separate package and has its own manual \href{file:pgfplotstable.pdf}{pgfplotstable.pdf}.
\end{enumerate}

\subsection{Upgrade remarks}
This release provides a lot of improvements which can be found in all detail in \texttt{ChangeLog} for interested readers. However, some attention is useful with respect to the following changes.

\subsubsection{New Optional Features}
\begin{enumerate}
	\item \PGFPlots\ 1.3 comes with user interface improvements. The technical distinction between ``behavior options'' and ``style options'' of older versions is no longer necessary (although still fully supported).

	\item \PGFPlots\ 1.3 has a new feature which allows to \emph{move axis labels tight to tick labels} automatically.

	Since this affects the spacing, it is not enabled be default.

	Use
\begin{codeexample}[code only]
\usepackage{pgfplots}
\pgfplotsset{compat=1.3}
\end{codeexample}
	in your preamble to benefit from the improved spacing\footnote{The |compat=1.3| setting is necessary for new features which move axis labels around like reversed axis. However, \PGFPlots\ will still work as it does in earlier versions, your documents will remain the same if you don't set it explicitly.}.
	Take a look at the next page for the precise definition of |compat|.

	\item \PGFPlots\ 1.3 now supports reversed axes. It is no longer necessary to use work--arounds with negative units.
\pgfkeys{/pdflinks/search key prefixes in/.add={/pgfplots/,}{}}

	Take a look at the |x dir=reverse| key.

	Existing work--arounds will still function properly. Use |\pgfplotsset{compat=1.3}| together with |x dir=reverse| to switch to the new version.
\end{enumerate}

\subsubsection{Old Features Which May Need Attention}
\begin{enumerate}
	\item The |scatter/classes| feature produces proper legends as of version 1.3. This may change the appearance of existing legends of plots with |scatter/classes|.

	\item Due to a small math bug in \PGF\ $2.00$, you \emph{can't} use the math expression `|-x^2|'. It is necessary to use `|0-x^2|' instead. The same holds for
		`|exp(-x^2)|'; use `|exp(0-x^2)|' instead.

		This will be fixed with \PGF\ $>2.00$.

	\item Starting with \PGFPlots\ $1.1$, |\tikzstyle| should \emph{no longer be used} to set \PGFPlots\ options.
	
	Although |\tikzstyle| is still supported for some older \PGFPlots\ options, you should replace any occurance of |\tikzstyle| with |\pgfplotsset{|\meta{style name}|/.style={|\meta{key-value-list}|}}| or the associated |/.append style| variant. See section~\ref{sec:styles} for more detail.
\end{enumerate}
I apologize for any inconvenience caused by these changes.

\begin{pgfplotskey}{compat=\mchoice{1.3,pre 1.3,default,newest} (initially default)}
	The preamble configuration 
\begin{codeexample}[code only]
\usepackage{pgfplots}
\pgfplotsset{compat=1.3}
\end{codeexample}
	allows to choose between backwards compatibility and most recent features.

	\PGFPlots\ version 1.3 comes with new features which may lead to different spacings compared to earlier versions:
	\begin{enumerate}
		\item Version 1.3 comes with the |xlabel near ticks| feature which places (in this case $x$) axis labels ``near'' the tick labels, taking the size of tick labels into account.
		
		Older versions used |xlabel absolute|, i.e.\ an absolute distance between the axis and axis labels (now matter how large or small tick labels are).

		The initial setting \emph{keeps backwards compatibility}.

		You are encouraged to use
\begin{codeexample}[code only]
\usepackage{pgfplots}
\pgfplotsset{compat=1.3}
\end{codeexample}
		in your preamble.
		
		\item Version 1.3 fixes a bug with |anchor=south west| which occurred mainly for reversed axes: the anchors where upside--down. This is fixed now.

		
		If you have written some work--around and want to keep it going, use

		|\pgfplotsset{compat/anchors=pre 1.3}|\pgfmanualpdflabel{/pgfplots/compat/anchors}{} 

		to disable the bug fix.
	\end{enumerate}

	Use |\pgfplotsset{compat=default}| to restore the factory settings.

	The setting |\pgfplotsset{compat=newest}| will always use features of the most recent version. This might result in small changes in the document's appearance.
\end{pgfplotskey}

\subsection{The Team}
\PGFPlots\ has been written mainly by Christian Feuers�nger with many improvements of Pascal Wolkotte and Nick Papior Andersen as a spare time project. We hope it is useful and provides valuable plots.

If you are interested in writing something but don't know how, consider reading the auxiliary manual \href{file:TeX-programming-notes.pdf}{TeX-programming-notes.pdf} which comes with \PGFPlots. It is far from complete, but maybe it is a good starting point (at least for more literature).

\subsection{Acknowledgements}
I thank God for all hours of enjoyed programming. I thank Pascal Wolkotte and Nick Papior Andersen for their programming efforts and contributions as part of the development team. I thank Stefan Tibus, who contributed the |plot shell| feature. I thank Tom Cashman for the contribution of the |reverse legend| feature. Special thanks go to Stefan Pinnow whose tests of \PGFPlots\ lead to numerous quality improvements. Furthermore, I thank Dr.~Schweitzer for many fruitful discussions and Dr.~Meine for his ideas and suggestions. Thanks as well to the many international contributors who provided feature requests or identified bugs or simply manual improvements!

Last but not least, I thank Till Tantau and Mark Wibrow for their excellent graphics (and more) package \PGF\ and \Tikz\ which is the base of \PGFPlots.

\include{pgfplots.install}%
\include{pgfplots.intro}%
\include{pgfplots.reference}%
\include{pgfplots.libs}%
\include{pgfplots.resources}%
\include{pgfplots.importexport}%
\include{pgfplots.basic.reference}%

\printindex

\bibliographystyle{abbrv} %gerapali} %gerabbrv} %gerunsrt.bst} %gerabbrv}% gerplain}
\nocite{pgfplotstable}
\nocite{programmingnotes}
\bibliography{pgfplots}
\end{document}
%
\include{pgfplots.basic.reference}%

\printindex

\bibliographystyle{abbrv} %gerapali} %gerabbrv} %gerunsrt.bst} %gerabbrv}% gerplain}
\nocite{pgfplotstable}
\nocite{programmingnotes}
\bibliography{pgfplots}
\end{document}
%
%%%%%%%%%%%%%%%%%%%%%%%%%%%%%%%%%%%%%%%%%%%%%%%%%%%%%%%%%%%%%%%%%%%%%%%%%%%%%
%
% Package pgfplots.sty documentation. 
%
% Copyright 2007/2008 by Christian Feuersaenger.
%
% This program is free software: you can redistribute it and/or modify
% it under the terms of the GNU General Public License as published by
% the Free Software Foundation, either version 3 of the License, or
% (at your option) any later version.
% 
% This program is distributed in the hope that it will be useful,
% but WITHOUT ANY WARRANTY; without even the implied warranty of
% MERCHANTABILITY or FITNESS FOR A PARTICULAR PURPOSE.  See the
% GNU General Public License for more details.
% 
% You should have received a copy of the GNU General Public License
% along with this program.  If not, see <http://www.gnu.org/licenses/>.
%
%
%%%%%%%%%%%%%%%%%%%%%%%%%%%%%%%%%%%%%%%%%%%%%%%%%%%%%%%%%%%%%%%%%%%%%%%%%%%%%
%%%%%%%%%%%%%%%%%%%%%%%%%%%%%%%%%%%%%%%%%%%%%%%%%%%%%%%%%%%%%%%%%%%%%%%%%%%%%
%
% Package pgfplots.sty documentation. 
%
% Copyright 2007/2008 by Christian Feuersaenger.
%
% This program is free software: you can redistribute it and/or modify
% it under the terms of the GNU General Public License as published by
% the Free Software Foundation, either version 3 of the License, or
% (at your option) any later version.
% 
% This program is distributed in the hope that it will be useful,
% but WITHOUT ANY WARRANTY; without even the implied warranty of
% MERCHANTABILITY or FITNESS FOR A PARTICULAR PURPOSE.  See the
% GNU General Public License for more details.
% 
% You should have received a copy of the GNU General Public License
% along with this program.  If not, see <http://www.gnu.org/licenses/>.
%
%
%%%%%%%%%%%%%%%%%%%%%%%%%%%%%%%%%%%%%%%%%%%%%%%%%%%%%%%%%%%%%%%%%%%%%%%%%%%%%
\documentclass[a4paper]{ltxdoc}

\usepackage{makeidx}

% DON't let hyperref overload the format of index and glossary. 
% I want to do that on my own in the stylefiles for makeindex...
\makeatletter
\let\@old@wrindex=\@wrindex
\makeatother

\usepackage{ifpdf}
\ifpdf
	\usepackage[pdftex]{hyperref}
\else
	\usepackage[dvipdfm]{hyperref}
\fi
\hypersetup{pdfborder={0 0 0}}

\makeatletter
\let\@wrindex=\@old@wrindex
\makeatother

% Formatiere Seitennummern im Index:
\newcommand{\indexpageno}[1]{%
	{\bfseries\hyperpage{#1}}%
}


\newcommand{\R}{\mathbb{R}}
\newcommand{\N}{\mathbb{N}}
\newcommand{\Z}{\mathbb{Z}}

\long\def\COMMENTLOWLEVEL#1\ENDCOMMENT{}
\def\ENDCOMMENT{}

\usepackage{textcomp}
\usepackage{booktabs}

\usepackage{calc}
\usepackage[formats]{listings}
%\usepackage{courier} % don't use it - the '^' character can't be copy-pasted in courier

\usepackage{array}
\lstset{%
	basicstyle=\ttfamily,
	language=[LaTeX]tex, % Seems as if \lstset{language=tex} must be invoked BEFORE loading tikz!?
	tabsize=4,
	breaklines=true,
	breakindent=0pt
}

\ifpdf
	\pdfinfo {
		/Author	(Christian Feuersaenger)
	}

\else
	\def\pgfsysdriver{pgfsys-dvipdfm.def}
\fi
%\def\pgfsysdriver{pgfsys-pdftex.def}
\usepackage{pgfplots}

\ifpdf
	\usetikzlibrary{pgfplotsclickable}
\fi

%\usepackage{fp}
% ATTENTION:
% this requires pgf version NEWER than 2.00 :
%\usetikzlibrary{fixedpointarithmetic}

\usetikzlibrary{dateplot}

\usepackage[a4paper,left=2.25cm,right=2.25cm,top=2.5cm,bottom=2.5cm,nohead]{geometry}
\usepackage{amsmath,amssymb}
\usepackage{xxcolor}
\usepackage{pifont}
\usepackage[latin1]{inputenc}
\usepackage{amsmath}
\usepackage{eurosym}
\input{pgfplots-macros}

\pgfqkeys{/codeexample}{%
	every codeexample/.style={
		width=8cm,
		/pgfplots/every axis/.append style={legend style={fill=graphicbackground}}
	},
	tabsize=4,
}
\pgfplotsset{
	every axis/.append style={width=7cm},
	filter discard warning=false,
}

\usetikzlibrary{backgrounds,patterns}
% Global styles:
\tikzset{
  shape example/.style={
    color=black!30,
    draw,
    fill=yellow!30,
    line width=.5cm,
    inner xsep=2.5cm,
    inner ysep=0.5cm}
}

\newcommand{\FIXME}[1]{\textcolor{red}{(FIXME: #1)}}

% fuer endvironment 'sidewaysfigure' bspw
% \usepackage{rotating}

\newcommand\Tikz{Ti\textit kZ}
\newcommand\PGF{\textsc{pgf}}
\newcommand\PGFPlots{\textsc{pgfplots}}
\def\PGFPlotstable{\textsc{PgfplotsTable}}


\makeindex

% Fix overful hboxes automatically:
\tolerance=2000
\emergencystretch=10pt

\tikzset{prefix=gnuplot/pgfplots_} % prefix for 'plot function'

\author{%
	Christian Feuers\"anger\footnote{\url{http://wissrech.ins.uni-bonn.de/people/feuersaenger}}\\%
	Institut f\"ur Numerische Simulation\\
	Universit\"at Bonn, Germany}


%\RequirePackage[german,english,francais]{babel}

\pgfplotsmanualexternalexpensivetrue

%\usetikzlibrary{external}
%\tikzexternalize[prefix=figures/]{pgfplots}

\title{%
	Manual for Package \PGFPlots\\
	{\small Version \pgfplotsversion}\\
	{\small\href{http://sourceforge.net/projects/pgfplots}{http://sourceforge.net/projects/pgfplots}}}

%\includeonly{pgfplots.libs}


\begin{document}

\def\plotcoords{%
\addplot coordinates {
(5,8.312e-02)    (17,2.547e-02)   (49,7.407e-03)
(129,2.102e-03)  (321,5.874e-04)  (769,1.623e-04)
(1793,4.442e-05) (4097,1.207e-05) (9217,3.261e-06)
};

\addplot coordinates{
(7,8.472e-02)    (31,3.044e-02)    (111,1.022e-02)
(351,3.303e-03)  (1023,1.039e-03)  (2815,3.196e-04)
(7423,9.658e-05) (18943,2.873e-05) (47103,8.437e-06)
};

\addplot coordinates{
(9,7.881e-02)     (49,3.243e-02)    (209,1.232e-02)
(769,4.454e-03)   (2561,1.551e-03)  (7937,5.236e-04)
(23297,1.723e-04) (65537,5.545e-05) (178177,1.751e-05)
};

\addplot coordinates{
(11,6.887e-02)    (71,3.177e-02)     (351,1.341e-02)
(1471,5.334e-03)  (5503,2.027e-03)   (18943,7.415e-04)
(61183,2.628e-04) (187903,9.063e-05) (553983,3.053e-05)
};

\addplot coordinates{
(13,5.755e-02)     (97,2.925e-02)     (545,1.351e-02)
(2561,5.842e-03)   (10625,2.397e-03)  (40193,9.414e-04)
(141569,3.564e-04) (471041,1.308e-04) 
(1496065,4.670e-05)
};
}%
\maketitle
\begin{abstract}%
\PGFPlots\ draws high--quality function plots in normal or logarithmic scaling with a user-friendly interface directly in \TeX. The user supplies axis labels, legend entries and the plot coordinates for one or more plots and \PGFPlots\ applies axis scaling, computes any logarithms and axis ticks and draws the plots, supporting line plots, scatter plots, piecewise constant plots, bar plots, area plots, mesh-- and surface plots and some more. It is based on Till Tantau's package \PGF/\Tikz.
\end{abstract}
\tableofcontents
\section{Introduction}
This package provides tools to generate plots and labeled axes easily. It draws normal plots, logplots and semi-logplots, in two and three dimensions. Axis ticks, labels, legends (in case of multiple plots) can be added with key-value options. It can cycle through a set of predefined line/marker/color specifications. In summary, its purpose is to simplify the generation of high-quality function and/or data plots, and solving the problems of
\begin{itemize}
	\item consistency of document and font type and font size,
	\item direct use of \TeX\ math mode in axis descriptions,
	\item consistency of data and figures (no third party tool necessary),
	\item inter document consistency using preamble configurations and styles.
\end{itemize}
Although not necessary, separate |.pdf| or |.eps| graphics can be generated using the |external| library developed as part of \Tikz.

\PGFPlots\ is build completely on \Tikz/\PGF. Knowledge of \Tikz\ will simplify the work with \PGFPlots, although it is not required.

\section[About PGFPlots: Preliminaries]{About {\normalfont\PGFPlots}: Preliminaries}
This section contains information about upgrades, the team, the installation (in case you need to do it manually) and troubleshooting. You may skip it completely except for the upgrade remarks.

However, note that this library requires at least \PGF\ version $2.00$. At the time of this writing, many \TeX-distributions still contain the older \PGF\ version $1.18$, so it may be necessary to install a recent \PGF\ prior to using \PGFPlots.

\subsection{Components}
\PGFPlots\ comes with two components:
\begin{enumerate}
	\item the plotting component (which you are currently reading) and
	\item the \PGFPlotstable\ component which simplifies number formatting and postprocessing of numerical tables. It comes as a separate package and has its own manual \href{file:pgfplotstable.pdf}{pgfplotstable.pdf}.
\end{enumerate}

\subsection{Upgrade remarks}
This release provides a lot of improvements which can be found in all detail in \texttt{ChangeLog} for interested readers. However, some attention is useful with respect to the following changes.

\subsubsection{New Optional Features}
\begin{enumerate}
	\item \PGFPlots\ 1.3 comes with user interface improvements. The technical distinction between ``behavior options'' and ``style options'' of older versions is no longer necessary (although still fully supported).

	\item \PGFPlots\ 1.3 has a new feature which allows to \emph{move axis labels tight to tick labels} automatically.

	Since this affects the spacing, it is not enabled be default.

	Use
\begin{codeexample}[code only]
\usepackage{pgfplots}
\pgfplotsset{compat=1.3}
\end{codeexample}
	in your preamble to benefit from the improved spacing\footnote{The |compat=1.3| setting is necessary for new features which move axis labels around like reversed axis. However, \PGFPlots\ will still work as it does in earlier versions, your documents will remain the same if you don't set it explicitly.}.
	Take a look at the next page for the precise definition of |compat|.

	\item \PGFPlots\ 1.3 now supports reversed axes. It is no longer necessary to use work--arounds with negative units.
\pgfkeys{/pdflinks/search key prefixes in/.add={/pgfplots/,}{}}

	Take a look at the |x dir=reverse| key.

	Existing work--arounds will still function properly. Use |\pgfplotsset{compat=1.3}| together with |x dir=reverse| to switch to the new version.
\end{enumerate}

\subsubsection{Old Features Which May Need Attention}
\begin{enumerate}
	\item The |scatter/classes| feature produces proper legends as of version 1.3. This may change the appearance of existing legends of plots with |scatter/classes|.

	\item Due to a small math bug in \PGF\ $2.00$, you \emph{can't} use the math expression `|-x^2|'. It is necessary to use `|0-x^2|' instead. The same holds for
		`|exp(-x^2)|'; use `|exp(0-x^2)|' instead.

		This will be fixed with \PGF\ $>2.00$.

	\item Starting with \PGFPlots\ $1.1$, |\tikzstyle| should \emph{no longer be used} to set \PGFPlots\ options.
	
	Although |\tikzstyle| is still supported for some older \PGFPlots\ options, you should replace any occurance of |\tikzstyle| with |\pgfplotsset{|\meta{style name}|/.style={|\meta{key-value-list}|}}| or the associated |/.append style| variant. See section~\ref{sec:styles} for more detail.
\end{enumerate}
I apologize for any inconvenience caused by these changes.

\begin{pgfplotskey}{compat=\mchoice{1.3,pre 1.3,default,newest} (initially default)}
	The preamble configuration 
\begin{codeexample}[code only]
\usepackage{pgfplots}
\pgfplotsset{compat=1.3}
\end{codeexample}
	allows to choose between backwards compatibility and most recent features.

	\PGFPlots\ version 1.3 comes with new features which may lead to different spacings compared to earlier versions:
	\begin{enumerate}
		\item Version 1.3 comes with the |xlabel near ticks| feature which places (in this case $x$) axis labels ``near'' the tick labels, taking the size of tick labels into account.
		
		Older versions used |xlabel absolute|, i.e.\ an absolute distance between the axis and axis labels (now matter how large or small tick labels are).

		The initial setting \emph{keeps backwards compatibility}.

		You are encouraged to use
\begin{codeexample}[code only]
\usepackage{pgfplots}
\pgfplotsset{compat=1.3}
\end{codeexample}
		in your preamble.
		
		\item Version 1.3 fixes a bug with |anchor=south west| which occurred mainly for reversed axes: the anchors where upside--down. This is fixed now.

		
		If you have written some work--around and want to keep it going, use

		|\pgfplotsset{compat/anchors=pre 1.3}|\pgfmanualpdflabel{/pgfplots/compat/anchors}{} 

		to disable the bug fix.
	\end{enumerate}

	Use |\pgfplotsset{compat=default}| to restore the factory settings.

	The setting |\pgfplotsset{compat=newest}| will always use features of the most recent version. This might result in small changes in the document's appearance.
\end{pgfplotskey}

\subsection{The Team}
\PGFPlots\ has been written mainly by Christian Feuers�nger with many improvements of Pascal Wolkotte and Nick Papior Andersen as a spare time project. We hope it is useful and provides valuable plots.

If you are interested in writing something but don't know how, consider reading the auxiliary manual \href{file:TeX-programming-notes.pdf}{TeX-programming-notes.pdf} which comes with \PGFPlots. It is far from complete, but maybe it is a good starting point (at least for more literature).

\subsection{Acknowledgements}
I thank God for all hours of enjoyed programming. I thank Pascal Wolkotte and Nick Papior Andersen for their programming efforts and contributions as part of the development team. I thank Stefan Tibus, who contributed the |plot shell| feature. I thank Tom Cashman for the contribution of the |reverse legend| feature. Special thanks go to Stefan Pinnow whose tests of \PGFPlots\ lead to numerous quality improvements. Furthermore, I thank Dr.~Schweitzer for many fruitful discussions and Dr.~Meine for his ideas and suggestions. Thanks as well to the many international contributors who provided feature requests or identified bugs or simply manual improvements!

Last but not least, I thank Till Tantau and Mark Wibrow for their excellent graphics (and more) package \PGF\ and \Tikz\ which is the base of \PGFPlots.

%%%%%%%%%%%%%%%%%%%%%%%%%%%%%%%%%%%%%%%%%%%%%%%%%%%%%%%%%%%%%%%%%%%%%%%%%%%%%
%
% Package pgfplots.sty documentation. 
%
% Copyright 2007/2008 by Christian Feuersaenger.
%
% This program is free software: you can redistribute it and/or modify
% it under the terms of the GNU General Public License as published by
% the Free Software Foundation, either version 3 of the License, or
% (at your option) any later version.
% 
% This program is distributed in the hope that it will be useful,
% but WITHOUT ANY WARRANTY; without even the implied warranty of
% MERCHANTABILITY or FITNESS FOR A PARTICULAR PURPOSE.  See the
% GNU General Public License for more details.
% 
% You should have received a copy of the GNU General Public License
% along with this program.  If not, see <http://www.gnu.org/licenses/>.
%
%
%%%%%%%%%%%%%%%%%%%%%%%%%%%%%%%%%%%%%%%%%%%%%%%%%%%%%%%%%%%%%%%%%%%%%%%%%%%%%
\input{pgfplots.preamble.tex}
%\RequirePackage[german,english,francais]{babel}

\pgfplotsmanualexternalexpensivetrue

%\usetikzlibrary{external}
%\tikzexternalize[prefix=figures/]{pgfplots}

\title{%
	Manual for Package \PGFPlots\\
	{\small Version \pgfplotsversion}\\
	{\small\href{http://sourceforge.net/projects/pgfplots}{http://sourceforge.net/projects/pgfplots}}}

%\includeonly{pgfplots.libs}


\begin{document}

\def\plotcoords{%
\addplot coordinates {
(5,8.312e-02)    (17,2.547e-02)   (49,7.407e-03)
(129,2.102e-03)  (321,5.874e-04)  (769,1.623e-04)
(1793,4.442e-05) (4097,1.207e-05) (9217,3.261e-06)
};

\addplot coordinates{
(7,8.472e-02)    (31,3.044e-02)    (111,1.022e-02)
(351,3.303e-03)  (1023,1.039e-03)  (2815,3.196e-04)
(7423,9.658e-05) (18943,2.873e-05) (47103,8.437e-06)
};

\addplot coordinates{
(9,7.881e-02)     (49,3.243e-02)    (209,1.232e-02)
(769,4.454e-03)   (2561,1.551e-03)  (7937,5.236e-04)
(23297,1.723e-04) (65537,5.545e-05) (178177,1.751e-05)
};

\addplot coordinates{
(11,6.887e-02)    (71,3.177e-02)     (351,1.341e-02)
(1471,5.334e-03)  (5503,2.027e-03)   (18943,7.415e-04)
(61183,2.628e-04) (187903,9.063e-05) (553983,3.053e-05)
};

\addplot coordinates{
(13,5.755e-02)     (97,2.925e-02)     (545,1.351e-02)
(2561,5.842e-03)   (10625,2.397e-03)  (40193,9.414e-04)
(141569,3.564e-04) (471041,1.308e-04) 
(1496065,4.670e-05)
};
}%
\maketitle
\begin{abstract}%
\PGFPlots\ draws high--quality function plots in normal or logarithmic scaling with a user-friendly interface directly in \TeX. The user supplies axis labels, legend entries and the plot coordinates for one or more plots and \PGFPlots\ applies axis scaling, computes any logarithms and axis ticks and draws the plots, supporting line plots, scatter plots, piecewise constant plots, bar plots, area plots, mesh-- and surface plots and some more. It is based on Till Tantau's package \PGF/\Tikz.
\end{abstract}
\tableofcontents
\section{Introduction}
This package provides tools to generate plots and labeled axes easily. It draws normal plots, logplots and semi-logplots, in two and three dimensions. Axis ticks, labels, legends (in case of multiple plots) can be added with key-value options. It can cycle through a set of predefined line/marker/color specifications. In summary, its purpose is to simplify the generation of high-quality function and/or data plots, and solving the problems of
\begin{itemize}
	\item consistency of document and font type and font size,
	\item direct use of \TeX\ math mode in axis descriptions,
	\item consistency of data and figures (no third party tool necessary),
	\item inter document consistency using preamble configurations and styles.
\end{itemize}
Although not necessary, separate |.pdf| or |.eps| graphics can be generated using the |external| library developed as part of \Tikz.

\PGFPlots\ is build completely on \Tikz/\PGF. Knowledge of \Tikz\ will simplify the work with \PGFPlots, although it is not required.

\section[About PGFPlots: Preliminaries]{About {\normalfont\PGFPlots}: Preliminaries}
This section contains information about upgrades, the team, the installation (in case you need to do it manually) and troubleshooting. You may skip it completely except for the upgrade remarks.

However, note that this library requires at least \PGF\ version $2.00$. At the time of this writing, many \TeX-distributions still contain the older \PGF\ version $1.18$, so it may be necessary to install a recent \PGF\ prior to using \PGFPlots.

\subsection{Components}
\PGFPlots\ comes with two components:
\begin{enumerate}
	\item the plotting component (which you are currently reading) and
	\item the \PGFPlotstable\ component which simplifies number formatting and postprocessing of numerical tables. It comes as a separate package and has its own manual \href{file:pgfplotstable.pdf}{pgfplotstable.pdf}.
\end{enumerate}

\subsection{Upgrade remarks}
This release provides a lot of improvements which can be found in all detail in \texttt{ChangeLog} for interested readers. However, some attention is useful with respect to the following changes.

\subsubsection{New Optional Features}
\begin{enumerate}
	\item \PGFPlots\ 1.3 comes with user interface improvements. The technical distinction between ``behavior options'' and ``style options'' of older versions is no longer necessary (although still fully supported).

	\item \PGFPlots\ 1.3 has a new feature which allows to \emph{move axis labels tight to tick labels} automatically.

	Since this affects the spacing, it is not enabled be default.

	Use
\begin{codeexample}[code only]
\usepackage{pgfplots}
\pgfplotsset{compat=1.3}
\end{codeexample}
	in your preamble to benefit from the improved spacing\footnote{The |compat=1.3| setting is necessary for new features which move axis labels around like reversed axis. However, \PGFPlots\ will still work as it does in earlier versions, your documents will remain the same if you don't set it explicitly.}.
	Take a look at the next page for the precise definition of |compat|.

	\item \PGFPlots\ 1.3 now supports reversed axes. It is no longer necessary to use work--arounds with negative units.
\pgfkeys{/pdflinks/search key prefixes in/.add={/pgfplots/,}{}}

	Take a look at the |x dir=reverse| key.

	Existing work--arounds will still function properly. Use |\pgfplotsset{compat=1.3}| together with |x dir=reverse| to switch to the new version.
\end{enumerate}

\subsubsection{Old Features Which May Need Attention}
\begin{enumerate}
	\item The |scatter/classes| feature produces proper legends as of version 1.3. This may change the appearance of existing legends of plots with |scatter/classes|.

	\item Due to a small math bug in \PGF\ $2.00$, you \emph{can't} use the math expression `|-x^2|'. It is necessary to use `|0-x^2|' instead. The same holds for
		`|exp(-x^2)|'; use `|exp(0-x^2)|' instead.

		This will be fixed with \PGF\ $>2.00$.

	\item Starting with \PGFPlots\ $1.1$, |\tikzstyle| should \emph{no longer be used} to set \PGFPlots\ options.
	
	Although |\tikzstyle| is still supported for some older \PGFPlots\ options, you should replace any occurance of |\tikzstyle| with |\pgfplotsset{|\meta{style name}|/.style={|\meta{key-value-list}|}}| or the associated |/.append style| variant. See section~\ref{sec:styles} for more detail.
\end{enumerate}
I apologize for any inconvenience caused by these changes.

\begin{pgfplotskey}{compat=\mchoice{1.3,pre 1.3,default,newest} (initially default)}
	The preamble configuration 
\begin{codeexample}[code only]
\usepackage{pgfplots}
\pgfplotsset{compat=1.3}
\end{codeexample}
	allows to choose between backwards compatibility and most recent features.

	\PGFPlots\ version 1.3 comes with new features which may lead to different spacings compared to earlier versions:
	\begin{enumerate}
		\item Version 1.3 comes with the |xlabel near ticks| feature which places (in this case $x$) axis labels ``near'' the tick labels, taking the size of tick labels into account.
		
		Older versions used |xlabel absolute|, i.e.\ an absolute distance between the axis and axis labels (now matter how large or small tick labels are).

		The initial setting \emph{keeps backwards compatibility}.

		You are encouraged to use
\begin{codeexample}[code only]
\usepackage{pgfplots}
\pgfplotsset{compat=1.3}
\end{codeexample}
		in your preamble.
		
		\item Version 1.3 fixes a bug with |anchor=south west| which occurred mainly for reversed axes: the anchors where upside--down. This is fixed now.

		
		If you have written some work--around and want to keep it going, use

		|\pgfplotsset{compat/anchors=pre 1.3}|\pgfmanualpdflabel{/pgfplots/compat/anchors}{} 

		to disable the bug fix.
	\end{enumerate}

	Use |\pgfplotsset{compat=default}| to restore the factory settings.

	The setting |\pgfplotsset{compat=newest}| will always use features of the most recent version. This might result in small changes in the document's appearance.
\end{pgfplotskey}

\subsection{The Team}
\PGFPlots\ has been written mainly by Christian Feuers�nger with many improvements of Pascal Wolkotte and Nick Papior Andersen as a spare time project. We hope it is useful and provides valuable plots.

If you are interested in writing something but don't know how, consider reading the auxiliary manual \href{file:TeX-programming-notes.pdf}{TeX-programming-notes.pdf} which comes with \PGFPlots. It is far from complete, but maybe it is a good starting point (at least for more literature).

\subsection{Acknowledgements}
I thank God for all hours of enjoyed programming. I thank Pascal Wolkotte and Nick Papior Andersen for their programming efforts and contributions as part of the development team. I thank Stefan Tibus, who contributed the |plot shell| feature. I thank Tom Cashman for the contribution of the |reverse legend| feature. Special thanks go to Stefan Pinnow whose tests of \PGFPlots\ lead to numerous quality improvements. Furthermore, I thank Dr.~Schweitzer for many fruitful discussions and Dr.~Meine for his ideas and suggestions. Thanks as well to the many international contributors who provided feature requests or identified bugs or simply manual improvements!

Last but not least, I thank Till Tantau and Mark Wibrow for their excellent graphics (and more) package \PGF\ and \Tikz\ which is the base of \PGFPlots.

\include{pgfplots.install}%
\include{pgfplots.intro}%
\include{pgfplots.reference}%
\include{pgfplots.libs}%
\include{pgfplots.resources}%
\include{pgfplots.importexport}%
\include{pgfplots.basic.reference}%

\printindex

\bibliographystyle{abbrv} %gerapali} %gerabbrv} %gerunsrt.bst} %gerabbrv}% gerplain}
\nocite{pgfplotstable}
\nocite{programmingnotes}
\bibliography{pgfplots}
\end{document}
%
%%%%%%%%%%%%%%%%%%%%%%%%%%%%%%%%%%%%%%%%%%%%%%%%%%%%%%%%%%%%%%%%%%%%%%%%%%%%%
%
% Package pgfplots.sty documentation. 
%
% Copyright 2007/2008 by Christian Feuersaenger.
%
% This program is free software: you can redistribute it and/or modify
% it under the terms of the GNU General Public License as published by
% the Free Software Foundation, either version 3 of the License, or
% (at your option) any later version.
% 
% This program is distributed in the hope that it will be useful,
% but WITHOUT ANY WARRANTY; without even the implied warranty of
% MERCHANTABILITY or FITNESS FOR A PARTICULAR PURPOSE.  See the
% GNU General Public License for more details.
% 
% You should have received a copy of the GNU General Public License
% along with this program.  If not, see <http://www.gnu.org/licenses/>.
%
%
%%%%%%%%%%%%%%%%%%%%%%%%%%%%%%%%%%%%%%%%%%%%%%%%%%%%%%%%%%%%%%%%%%%%%%%%%%%%%
\input{pgfplots.preamble.tex}
%\RequirePackage[german,english,francais]{babel}

\pgfplotsmanualexternalexpensivetrue

%\usetikzlibrary{external}
%\tikzexternalize[prefix=figures/]{pgfplots}

\title{%
	Manual for Package \PGFPlots\\
	{\small Version \pgfplotsversion}\\
	{\small\href{http://sourceforge.net/projects/pgfplots}{http://sourceforge.net/projects/pgfplots}}}

%\includeonly{pgfplots.libs}


\begin{document}

\def\plotcoords{%
\addplot coordinates {
(5,8.312e-02)    (17,2.547e-02)   (49,7.407e-03)
(129,2.102e-03)  (321,5.874e-04)  (769,1.623e-04)
(1793,4.442e-05) (4097,1.207e-05) (9217,3.261e-06)
};

\addplot coordinates{
(7,8.472e-02)    (31,3.044e-02)    (111,1.022e-02)
(351,3.303e-03)  (1023,1.039e-03)  (2815,3.196e-04)
(7423,9.658e-05) (18943,2.873e-05) (47103,8.437e-06)
};

\addplot coordinates{
(9,7.881e-02)     (49,3.243e-02)    (209,1.232e-02)
(769,4.454e-03)   (2561,1.551e-03)  (7937,5.236e-04)
(23297,1.723e-04) (65537,5.545e-05) (178177,1.751e-05)
};

\addplot coordinates{
(11,6.887e-02)    (71,3.177e-02)     (351,1.341e-02)
(1471,5.334e-03)  (5503,2.027e-03)   (18943,7.415e-04)
(61183,2.628e-04) (187903,9.063e-05) (553983,3.053e-05)
};

\addplot coordinates{
(13,5.755e-02)     (97,2.925e-02)     (545,1.351e-02)
(2561,5.842e-03)   (10625,2.397e-03)  (40193,9.414e-04)
(141569,3.564e-04) (471041,1.308e-04) 
(1496065,4.670e-05)
};
}%
\maketitle
\begin{abstract}%
\PGFPlots\ draws high--quality function plots in normal or logarithmic scaling with a user-friendly interface directly in \TeX. The user supplies axis labels, legend entries and the plot coordinates for one or more plots and \PGFPlots\ applies axis scaling, computes any logarithms and axis ticks and draws the plots, supporting line plots, scatter plots, piecewise constant plots, bar plots, area plots, mesh-- and surface plots and some more. It is based on Till Tantau's package \PGF/\Tikz.
\end{abstract}
\tableofcontents
\section{Introduction}
This package provides tools to generate plots and labeled axes easily. It draws normal plots, logplots and semi-logplots, in two and three dimensions. Axis ticks, labels, legends (in case of multiple plots) can be added with key-value options. It can cycle through a set of predefined line/marker/color specifications. In summary, its purpose is to simplify the generation of high-quality function and/or data plots, and solving the problems of
\begin{itemize}
	\item consistency of document and font type and font size,
	\item direct use of \TeX\ math mode in axis descriptions,
	\item consistency of data and figures (no third party tool necessary),
	\item inter document consistency using preamble configurations and styles.
\end{itemize}
Although not necessary, separate |.pdf| or |.eps| graphics can be generated using the |external| library developed as part of \Tikz.

\PGFPlots\ is build completely on \Tikz/\PGF. Knowledge of \Tikz\ will simplify the work with \PGFPlots, although it is not required.

\section[About PGFPlots: Preliminaries]{About {\normalfont\PGFPlots}: Preliminaries}
This section contains information about upgrades, the team, the installation (in case you need to do it manually) and troubleshooting. You may skip it completely except for the upgrade remarks.

However, note that this library requires at least \PGF\ version $2.00$. At the time of this writing, many \TeX-distributions still contain the older \PGF\ version $1.18$, so it may be necessary to install a recent \PGF\ prior to using \PGFPlots.

\subsection{Components}
\PGFPlots\ comes with two components:
\begin{enumerate}
	\item the plotting component (which you are currently reading) and
	\item the \PGFPlotstable\ component which simplifies number formatting and postprocessing of numerical tables. It comes as a separate package and has its own manual \href{file:pgfplotstable.pdf}{pgfplotstable.pdf}.
\end{enumerate}

\subsection{Upgrade remarks}
This release provides a lot of improvements which can be found in all detail in \texttt{ChangeLog} for interested readers. However, some attention is useful with respect to the following changes.

\subsubsection{New Optional Features}
\begin{enumerate}
	\item \PGFPlots\ 1.3 comes with user interface improvements. The technical distinction between ``behavior options'' and ``style options'' of older versions is no longer necessary (although still fully supported).

	\item \PGFPlots\ 1.3 has a new feature which allows to \emph{move axis labels tight to tick labels} automatically.

	Since this affects the spacing, it is not enabled be default.

	Use
\begin{codeexample}[code only]
\usepackage{pgfplots}
\pgfplotsset{compat=1.3}
\end{codeexample}
	in your preamble to benefit from the improved spacing\footnote{The |compat=1.3| setting is necessary for new features which move axis labels around like reversed axis. However, \PGFPlots\ will still work as it does in earlier versions, your documents will remain the same if you don't set it explicitly.}.
	Take a look at the next page for the precise definition of |compat|.

	\item \PGFPlots\ 1.3 now supports reversed axes. It is no longer necessary to use work--arounds with negative units.
\pgfkeys{/pdflinks/search key prefixes in/.add={/pgfplots/,}{}}

	Take a look at the |x dir=reverse| key.

	Existing work--arounds will still function properly. Use |\pgfplotsset{compat=1.3}| together with |x dir=reverse| to switch to the new version.
\end{enumerate}

\subsubsection{Old Features Which May Need Attention}
\begin{enumerate}
	\item The |scatter/classes| feature produces proper legends as of version 1.3. This may change the appearance of existing legends of plots with |scatter/classes|.

	\item Due to a small math bug in \PGF\ $2.00$, you \emph{can't} use the math expression `|-x^2|'. It is necessary to use `|0-x^2|' instead. The same holds for
		`|exp(-x^2)|'; use `|exp(0-x^2)|' instead.

		This will be fixed with \PGF\ $>2.00$.

	\item Starting with \PGFPlots\ $1.1$, |\tikzstyle| should \emph{no longer be used} to set \PGFPlots\ options.
	
	Although |\tikzstyle| is still supported for some older \PGFPlots\ options, you should replace any occurance of |\tikzstyle| with |\pgfplotsset{|\meta{style name}|/.style={|\meta{key-value-list}|}}| or the associated |/.append style| variant. See section~\ref{sec:styles} for more detail.
\end{enumerate}
I apologize for any inconvenience caused by these changes.

\begin{pgfplotskey}{compat=\mchoice{1.3,pre 1.3,default,newest} (initially default)}
	The preamble configuration 
\begin{codeexample}[code only]
\usepackage{pgfplots}
\pgfplotsset{compat=1.3}
\end{codeexample}
	allows to choose between backwards compatibility and most recent features.

	\PGFPlots\ version 1.3 comes with new features which may lead to different spacings compared to earlier versions:
	\begin{enumerate}
		\item Version 1.3 comes with the |xlabel near ticks| feature which places (in this case $x$) axis labels ``near'' the tick labels, taking the size of tick labels into account.
		
		Older versions used |xlabel absolute|, i.e.\ an absolute distance between the axis and axis labels (now matter how large or small tick labels are).

		The initial setting \emph{keeps backwards compatibility}.

		You are encouraged to use
\begin{codeexample}[code only]
\usepackage{pgfplots}
\pgfplotsset{compat=1.3}
\end{codeexample}
		in your preamble.
		
		\item Version 1.3 fixes a bug with |anchor=south west| which occurred mainly for reversed axes: the anchors where upside--down. This is fixed now.

		
		If you have written some work--around and want to keep it going, use

		|\pgfplotsset{compat/anchors=pre 1.3}|\pgfmanualpdflabel{/pgfplots/compat/anchors}{} 

		to disable the bug fix.
	\end{enumerate}

	Use |\pgfplotsset{compat=default}| to restore the factory settings.

	The setting |\pgfplotsset{compat=newest}| will always use features of the most recent version. This might result in small changes in the document's appearance.
\end{pgfplotskey}

\subsection{The Team}
\PGFPlots\ has been written mainly by Christian Feuers�nger with many improvements of Pascal Wolkotte and Nick Papior Andersen as a spare time project. We hope it is useful and provides valuable plots.

If you are interested in writing something but don't know how, consider reading the auxiliary manual \href{file:TeX-programming-notes.pdf}{TeX-programming-notes.pdf} which comes with \PGFPlots. It is far from complete, but maybe it is a good starting point (at least for more literature).

\subsection{Acknowledgements}
I thank God for all hours of enjoyed programming. I thank Pascal Wolkotte and Nick Papior Andersen for their programming efforts and contributions as part of the development team. I thank Stefan Tibus, who contributed the |plot shell| feature. I thank Tom Cashman for the contribution of the |reverse legend| feature. Special thanks go to Stefan Pinnow whose tests of \PGFPlots\ lead to numerous quality improvements. Furthermore, I thank Dr.~Schweitzer for many fruitful discussions and Dr.~Meine for his ideas and suggestions. Thanks as well to the many international contributors who provided feature requests or identified bugs or simply manual improvements!

Last but not least, I thank Till Tantau and Mark Wibrow for their excellent graphics (and more) package \PGF\ and \Tikz\ which is the base of \PGFPlots.

\include{pgfplots.install}%
\include{pgfplots.intro}%
\include{pgfplots.reference}%
\include{pgfplots.libs}%
\include{pgfplots.resources}%
\include{pgfplots.importexport}%
\include{pgfplots.basic.reference}%

\printindex

\bibliographystyle{abbrv} %gerapali} %gerabbrv} %gerunsrt.bst} %gerabbrv}% gerplain}
\nocite{pgfplotstable}
\nocite{programmingnotes}
\bibliography{pgfplots}
\end{document}
%
%%%%%%%%%%%%%%%%%%%%%%%%%%%%%%%%%%%%%%%%%%%%%%%%%%%%%%%%%%%%%%%%%%%%%%%%%%%%%
%
% Package pgfplots.sty documentation. 
%
% Copyright 2007/2008 by Christian Feuersaenger.
%
% This program is free software: you can redistribute it and/or modify
% it under the terms of the GNU General Public License as published by
% the Free Software Foundation, either version 3 of the License, or
% (at your option) any later version.
% 
% This program is distributed in the hope that it will be useful,
% but WITHOUT ANY WARRANTY; without even the implied warranty of
% MERCHANTABILITY or FITNESS FOR A PARTICULAR PURPOSE.  See the
% GNU General Public License for more details.
% 
% You should have received a copy of the GNU General Public License
% along with this program.  If not, see <http://www.gnu.org/licenses/>.
%
%
%%%%%%%%%%%%%%%%%%%%%%%%%%%%%%%%%%%%%%%%%%%%%%%%%%%%%%%%%%%%%%%%%%%%%%%%%%%%%
\input{pgfplots.preamble.tex}
%\RequirePackage[german,english,francais]{babel}

\pgfplotsmanualexternalexpensivetrue

%\usetikzlibrary{external}
%\tikzexternalize[prefix=figures/]{pgfplots}

\title{%
	Manual for Package \PGFPlots\\
	{\small Version \pgfplotsversion}\\
	{\small\href{http://sourceforge.net/projects/pgfplots}{http://sourceforge.net/projects/pgfplots}}}

%\includeonly{pgfplots.libs}


\begin{document}

\def\plotcoords{%
\addplot coordinates {
(5,8.312e-02)    (17,2.547e-02)   (49,7.407e-03)
(129,2.102e-03)  (321,5.874e-04)  (769,1.623e-04)
(1793,4.442e-05) (4097,1.207e-05) (9217,3.261e-06)
};

\addplot coordinates{
(7,8.472e-02)    (31,3.044e-02)    (111,1.022e-02)
(351,3.303e-03)  (1023,1.039e-03)  (2815,3.196e-04)
(7423,9.658e-05) (18943,2.873e-05) (47103,8.437e-06)
};

\addplot coordinates{
(9,7.881e-02)     (49,3.243e-02)    (209,1.232e-02)
(769,4.454e-03)   (2561,1.551e-03)  (7937,5.236e-04)
(23297,1.723e-04) (65537,5.545e-05) (178177,1.751e-05)
};

\addplot coordinates{
(11,6.887e-02)    (71,3.177e-02)     (351,1.341e-02)
(1471,5.334e-03)  (5503,2.027e-03)   (18943,7.415e-04)
(61183,2.628e-04) (187903,9.063e-05) (553983,3.053e-05)
};

\addplot coordinates{
(13,5.755e-02)     (97,2.925e-02)     (545,1.351e-02)
(2561,5.842e-03)   (10625,2.397e-03)  (40193,9.414e-04)
(141569,3.564e-04) (471041,1.308e-04) 
(1496065,4.670e-05)
};
}%
\maketitle
\begin{abstract}%
\PGFPlots\ draws high--quality function plots in normal or logarithmic scaling with a user-friendly interface directly in \TeX. The user supplies axis labels, legend entries and the plot coordinates for one or more plots and \PGFPlots\ applies axis scaling, computes any logarithms and axis ticks and draws the plots, supporting line plots, scatter plots, piecewise constant plots, bar plots, area plots, mesh-- and surface plots and some more. It is based on Till Tantau's package \PGF/\Tikz.
\end{abstract}
\tableofcontents
\section{Introduction}
This package provides tools to generate plots and labeled axes easily. It draws normal plots, logplots and semi-logplots, in two and three dimensions. Axis ticks, labels, legends (in case of multiple plots) can be added with key-value options. It can cycle through a set of predefined line/marker/color specifications. In summary, its purpose is to simplify the generation of high-quality function and/or data plots, and solving the problems of
\begin{itemize}
	\item consistency of document and font type and font size,
	\item direct use of \TeX\ math mode in axis descriptions,
	\item consistency of data and figures (no third party tool necessary),
	\item inter document consistency using preamble configurations and styles.
\end{itemize}
Although not necessary, separate |.pdf| or |.eps| graphics can be generated using the |external| library developed as part of \Tikz.

\PGFPlots\ is build completely on \Tikz/\PGF. Knowledge of \Tikz\ will simplify the work with \PGFPlots, although it is not required.

\section[About PGFPlots: Preliminaries]{About {\normalfont\PGFPlots}: Preliminaries}
This section contains information about upgrades, the team, the installation (in case you need to do it manually) and troubleshooting. You may skip it completely except for the upgrade remarks.

However, note that this library requires at least \PGF\ version $2.00$. At the time of this writing, many \TeX-distributions still contain the older \PGF\ version $1.18$, so it may be necessary to install a recent \PGF\ prior to using \PGFPlots.

\subsection{Components}
\PGFPlots\ comes with two components:
\begin{enumerate}
	\item the plotting component (which you are currently reading) and
	\item the \PGFPlotstable\ component which simplifies number formatting and postprocessing of numerical tables. It comes as a separate package and has its own manual \href{file:pgfplotstable.pdf}{pgfplotstable.pdf}.
\end{enumerate}

\subsection{Upgrade remarks}
This release provides a lot of improvements which can be found in all detail in \texttt{ChangeLog} for interested readers. However, some attention is useful with respect to the following changes.

\subsubsection{New Optional Features}
\begin{enumerate}
	\item \PGFPlots\ 1.3 comes with user interface improvements. The technical distinction between ``behavior options'' and ``style options'' of older versions is no longer necessary (although still fully supported).

	\item \PGFPlots\ 1.3 has a new feature which allows to \emph{move axis labels tight to tick labels} automatically.

	Since this affects the spacing, it is not enabled be default.

	Use
\begin{codeexample}[code only]
\usepackage{pgfplots}
\pgfplotsset{compat=1.3}
\end{codeexample}
	in your preamble to benefit from the improved spacing\footnote{The |compat=1.3| setting is necessary for new features which move axis labels around like reversed axis. However, \PGFPlots\ will still work as it does in earlier versions, your documents will remain the same if you don't set it explicitly.}.
	Take a look at the next page for the precise definition of |compat|.

	\item \PGFPlots\ 1.3 now supports reversed axes. It is no longer necessary to use work--arounds with negative units.
\pgfkeys{/pdflinks/search key prefixes in/.add={/pgfplots/,}{}}

	Take a look at the |x dir=reverse| key.

	Existing work--arounds will still function properly. Use |\pgfplotsset{compat=1.3}| together with |x dir=reverse| to switch to the new version.
\end{enumerate}

\subsubsection{Old Features Which May Need Attention}
\begin{enumerate}
	\item The |scatter/classes| feature produces proper legends as of version 1.3. This may change the appearance of existing legends of plots with |scatter/classes|.

	\item Due to a small math bug in \PGF\ $2.00$, you \emph{can't} use the math expression `|-x^2|'. It is necessary to use `|0-x^2|' instead. The same holds for
		`|exp(-x^2)|'; use `|exp(0-x^2)|' instead.

		This will be fixed with \PGF\ $>2.00$.

	\item Starting with \PGFPlots\ $1.1$, |\tikzstyle| should \emph{no longer be used} to set \PGFPlots\ options.
	
	Although |\tikzstyle| is still supported for some older \PGFPlots\ options, you should replace any occurance of |\tikzstyle| with |\pgfplotsset{|\meta{style name}|/.style={|\meta{key-value-list}|}}| or the associated |/.append style| variant. See section~\ref{sec:styles} for more detail.
\end{enumerate}
I apologize for any inconvenience caused by these changes.

\begin{pgfplotskey}{compat=\mchoice{1.3,pre 1.3,default,newest} (initially default)}
	The preamble configuration 
\begin{codeexample}[code only]
\usepackage{pgfplots}
\pgfplotsset{compat=1.3}
\end{codeexample}
	allows to choose between backwards compatibility and most recent features.

	\PGFPlots\ version 1.3 comes with new features which may lead to different spacings compared to earlier versions:
	\begin{enumerate}
		\item Version 1.3 comes with the |xlabel near ticks| feature which places (in this case $x$) axis labels ``near'' the tick labels, taking the size of tick labels into account.
		
		Older versions used |xlabel absolute|, i.e.\ an absolute distance between the axis and axis labels (now matter how large or small tick labels are).

		The initial setting \emph{keeps backwards compatibility}.

		You are encouraged to use
\begin{codeexample}[code only]
\usepackage{pgfplots}
\pgfplotsset{compat=1.3}
\end{codeexample}
		in your preamble.
		
		\item Version 1.3 fixes a bug with |anchor=south west| which occurred mainly for reversed axes: the anchors where upside--down. This is fixed now.

		
		If you have written some work--around and want to keep it going, use

		|\pgfplotsset{compat/anchors=pre 1.3}|\pgfmanualpdflabel{/pgfplots/compat/anchors}{} 

		to disable the bug fix.
	\end{enumerate}

	Use |\pgfplotsset{compat=default}| to restore the factory settings.

	The setting |\pgfplotsset{compat=newest}| will always use features of the most recent version. This might result in small changes in the document's appearance.
\end{pgfplotskey}

\subsection{The Team}
\PGFPlots\ has been written mainly by Christian Feuers�nger with many improvements of Pascal Wolkotte and Nick Papior Andersen as a spare time project. We hope it is useful and provides valuable plots.

If you are interested in writing something but don't know how, consider reading the auxiliary manual \href{file:TeX-programming-notes.pdf}{TeX-programming-notes.pdf} which comes with \PGFPlots. It is far from complete, but maybe it is a good starting point (at least for more literature).

\subsection{Acknowledgements}
I thank God for all hours of enjoyed programming. I thank Pascal Wolkotte and Nick Papior Andersen for their programming efforts and contributions as part of the development team. I thank Stefan Tibus, who contributed the |plot shell| feature. I thank Tom Cashman for the contribution of the |reverse legend| feature. Special thanks go to Stefan Pinnow whose tests of \PGFPlots\ lead to numerous quality improvements. Furthermore, I thank Dr.~Schweitzer for many fruitful discussions and Dr.~Meine for his ideas and suggestions. Thanks as well to the many international contributors who provided feature requests or identified bugs or simply manual improvements!

Last but not least, I thank Till Tantau and Mark Wibrow for their excellent graphics (and more) package \PGF\ and \Tikz\ which is the base of \PGFPlots.

\include{pgfplots.install}%
\include{pgfplots.intro}%
\include{pgfplots.reference}%
\include{pgfplots.libs}%
\include{pgfplots.resources}%
\include{pgfplots.importexport}%
\include{pgfplots.basic.reference}%

\printindex

\bibliographystyle{abbrv} %gerapali} %gerabbrv} %gerunsrt.bst} %gerabbrv}% gerplain}
\nocite{pgfplotstable}
\nocite{programmingnotes}
\bibliography{pgfplots}
\end{document}
%
%%%%%%%%%%%%%%%%%%%%%%%%%%%%%%%%%%%%%%%%%%%%%%%%%%%%%%%%%%%%%%%%%%%%%%%%%%%%%
%
% Package pgfplots.sty documentation. 
%
% Copyright 2007/2008 by Christian Feuersaenger.
%
% This program is free software: you can redistribute it and/or modify
% it under the terms of the GNU General Public License as published by
% the Free Software Foundation, either version 3 of the License, or
% (at your option) any later version.
% 
% This program is distributed in the hope that it will be useful,
% but WITHOUT ANY WARRANTY; without even the implied warranty of
% MERCHANTABILITY or FITNESS FOR A PARTICULAR PURPOSE.  See the
% GNU General Public License for more details.
% 
% You should have received a copy of the GNU General Public License
% along with this program.  If not, see <http://www.gnu.org/licenses/>.
%
%
%%%%%%%%%%%%%%%%%%%%%%%%%%%%%%%%%%%%%%%%%%%%%%%%%%%%%%%%%%%%%%%%%%%%%%%%%%%%%
\input{pgfplots.preamble.tex}
%\RequirePackage[german,english,francais]{babel}

\pgfplotsmanualexternalexpensivetrue

%\usetikzlibrary{external}
%\tikzexternalize[prefix=figures/]{pgfplots}

\title{%
	Manual for Package \PGFPlots\\
	{\small Version \pgfplotsversion}\\
	{\small\href{http://sourceforge.net/projects/pgfplots}{http://sourceforge.net/projects/pgfplots}}}

%\includeonly{pgfplots.libs}


\begin{document}

\def\plotcoords{%
\addplot coordinates {
(5,8.312e-02)    (17,2.547e-02)   (49,7.407e-03)
(129,2.102e-03)  (321,5.874e-04)  (769,1.623e-04)
(1793,4.442e-05) (4097,1.207e-05) (9217,3.261e-06)
};

\addplot coordinates{
(7,8.472e-02)    (31,3.044e-02)    (111,1.022e-02)
(351,3.303e-03)  (1023,1.039e-03)  (2815,3.196e-04)
(7423,9.658e-05) (18943,2.873e-05) (47103,8.437e-06)
};

\addplot coordinates{
(9,7.881e-02)     (49,3.243e-02)    (209,1.232e-02)
(769,4.454e-03)   (2561,1.551e-03)  (7937,5.236e-04)
(23297,1.723e-04) (65537,5.545e-05) (178177,1.751e-05)
};

\addplot coordinates{
(11,6.887e-02)    (71,3.177e-02)     (351,1.341e-02)
(1471,5.334e-03)  (5503,2.027e-03)   (18943,7.415e-04)
(61183,2.628e-04) (187903,9.063e-05) (553983,3.053e-05)
};

\addplot coordinates{
(13,5.755e-02)     (97,2.925e-02)     (545,1.351e-02)
(2561,5.842e-03)   (10625,2.397e-03)  (40193,9.414e-04)
(141569,3.564e-04) (471041,1.308e-04) 
(1496065,4.670e-05)
};
}%
\maketitle
\begin{abstract}%
\PGFPlots\ draws high--quality function plots in normal or logarithmic scaling with a user-friendly interface directly in \TeX. The user supplies axis labels, legend entries and the plot coordinates for one or more plots and \PGFPlots\ applies axis scaling, computes any logarithms and axis ticks and draws the plots, supporting line plots, scatter plots, piecewise constant plots, bar plots, area plots, mesh-- and surface plots and some more. It is based on Till Tantau's package \PGF/\Tikz.
\end{abstract}
\tableofcontents
\section{Introduction}
This package provides tools to generate plots and labeled axes easily. It draws normal plots, logplots and semi-logplots, in two and three dimensions. Axis ticks, labels, legends (in case of multiple plots) can be added with key-value options. It can cycle through a set of predefined line/marker/color specifications. In summary, its purpose is to simplify the generation of high-quality function and/or data plots, and solving the problems of
\begin{itemize}
	\item consistency of document and font type and font size,
	\item direct use of \TeX\ math mode in axis descriptions,
	\item consistency of data and figures (no third party tool necessary),
	\item inter document consistency using preamble configurations and styles.
\end{itemize}
Although not necessary, separate |.pdf| or |.eps| graphics can be generated using the |external| library developed as part of \Tikz.

\PGFPlots\ is build completely on \Tikz/\PGF. Knowledge of \Tikz\ will simplify the work with \PGFPlots, although it is not required.

\section[About PGFPlots: Preliminaries]{About {\normalfont\PGFPlots}: Preliminaries}
This section contains information about upgrades, the team, the installation (in case you need to do it manually) and troubleshooting. You may skip it completely except for the upgrade remarks.

However, note that this library requires at least \PGF\ version $2.00$. At the time of this writing, many \TeX-distributions still contain the older \PGF\ version $1.18$, so it may be necessary to install a recent \PGF\ prior to using \PGFPlots.

\subsection{Components}
\PGFPlots\ comes with two components:
\begin{enumerate}
	\item the plotting component (which you are currently reading) and
	\item the \PGFPlotstable\ component which simplifies number formatting and postprocessing of numerical tables. It comes as a separate package and has its own manual \href{file:pgfplotstable.pdf}{pgfplotstable.pdf}.
\end{enumerate}

\subsection{Upgrade remarks}
This release provides a lot of improvements which can be found in all detail in \texttt{ChangeLog} for interested readers. However, some attention is useful with respect to the following changes.

\subsubsection{New Optional Features}
\begin{enumerate}
	\item \PGFPlots\ 1.3 comes with user interface improvements. The technical distinction between ``behavior options'' and ``style options'' of older versions is no longer necessary (although still fully supported).

	\item \PGFPlots\ 1.3 has a new feature which allows to \emph{move axis labels tight to tick labels} automatically.

	Since this affects the spacing, it is not enabled be default.

	Use
\begin{codeexample}[code only]
\usepackage{pgfplots}
\pgfplotsset{compat=1.3}
\end{codeexample}
	in your preamble to benefit from the improved spacing\footnote{The |compat=1.3| setting is necessary for new features which move axis labels around like reversed axis. However, \PGFPlots\ will still work as it does in earlier versions, your documents will remain the same if you don't set it explicitly.}.
	Take a look at the next page for the precise definition of |compat|.

	\item \PGFPlots\ 1.3 now supports reversed axes. It is no longer necessary to use work--arounds with negative units.
\pgfkeys{/pdflinks/search key prefixes in/.add={/pgfplots/,}{}}

	Take a look at the |x dir=reverse| key.

	Existing work--arounds will still function properly. Use |\pgfplotsset{compat=1.3}| together with |x dir=reverse| to switch to the new version.
\end{enumerate}

\subsubsection{Old Features Which May Need Attention}
\begin{enumerate}
	\item The |scatter/classes| feature produces proper legends as of version 1.3. This may change the appearance of existing legends of plots with |scatter/classes|.

	\item Due to a small math bug in \PGF\ $2.00$, you \emph{can't} use the math expression `|-x^2|'. It is necessary to use `|0-x^2|' instead. The same holds for
		`|exp(-x^2)|'; use `|exp(0-x^2)|' instead.

		This will be fixed with \PGF\ $>2.00$.

	\item Starting with \PGFPlots\ $1.1$, |\tikzstyle| should \emph{no longer be used} to set \PGFPlots\ options.
	
	Although |\tikzstyle| is still supported for some older \PGFPlots\ options, you should replace any occurance of |\tikzstyle| with |\pgfplotsset{|\meta{style name}|/.style={|\meta{key-value-list}|}}| or the associated |/.append style| variant. See section~\ref{sec:styles} for more detail.
\end{enumerate}
I apologize for any inconvenience caused by these changes.

\begin{pgfplotskey}{compat=\mchoice{1.3,pre 1.3,default,newest} (initially default)}
	The preamble configuration 
\begin{codeexample}[code only]
\usepackage{pgfplots}
\pgfplotsset{compat=1.3}
\end{codeexample}
	allows to choose between backwards compatibility and most recent features.

	\PGFPlots\ version 1.3 comes with new features which may lead to different spacings compared to earlier versions:
	\begin{enumerate}
		\item Version 1.3 comes with the |xlabel near ticks| feature which places (in this case $x$) axis labels ``near'' the tick labels, taking the size of tick labels into account.
		
		Older versions used |xlabel absolute|, i.e.\ an absolute distance between the axis and axis labels (now matter how large or small tick labels are).

		The initial setting \emph{keeps backwards compatibility}.

		You are encouraged to use
\begin{codeexample}[code only]
\usepackage{pgfplots}
\pgfplotsset{compat=1.3}
\end{codeexample}
		in your preamble.
		
		\item Version 1.3 fixes a bug with |anchor=south west| which occurred mainly for reversed axes: the anchors where upside--down. This is fixed now.

		
		If you have written some work--around and want to keep it going, use

		|\pgfplotsset{compat/anchors=pre 1.3}|\pgfmanualpdflabel{/pgfplots/compat/anchors}{} 

		to disable the bug fix.
	\end{enumerate}

	Use |\pgfplotsset{compat=default}| to restore the factory settings.

	The setting |\pgfplotsset{compat=newest}| will always use features of the most recent version. This might result in small changes in the document's appearance.
\end{pgfplotskey}

\subsection{The Team}
\PGFPlots\ has been written mainly by Christian Feuers�nger with many improvements of Pascal Wolkotte and Nick Papior Andersen as a spare time project. We hope it is useful and provides valuable plots.

If you are interested in writing something but don't know how, consider reading the auxiliary manual \href{file:TeX-programming-notes.pdf}{TeX-programming-notes.pdf} which comes with \PGFPlots. It is far from complete, but maybe it is a good starting point (at least for more literature).

\subsection{Acknowledgements}
I thank God for all hours of enjoyed programming. I thank Pascal Wolkotte and Nick Papior Andersen for their programming efforts and contributions as part of the development team. I thank Stefan Tibus, who contributed the |plot shell| feature. I thank Tom Cashman for the contribution of the |reverse legend| feature. Special thanks go to Stefan Pinnow whose tests of \PGFPlots\ lead to numerous quality improvements. Furthermore, I thank Dr.~Schweitzer for many fruitful discussions and Dr.~Meine for his ideas and suggestions. Thanks as well to the many international contributors who provided feature requests or identified bugs or simply manual improvements!

Last but not least, I thank Till Tantau and Mark Wibrow for their excellent graphics (and more) package \PGF\ and \Tikz\ which is the base of \PGFPlots.

\include{pgfplots.install}%
\include{pgfplots.intro}%
\include{pgfplots.reference}%
\include{pgfplots.libs}%
\include{pgfplots.resources}%
\include{pgfplots.importexport}%
\include{pgfplots.basic.reference}%

\printindex

\bibliographystyle{abbrv} %gerapali} %gerabbrv} %gerunsrt.bst} %gerabbrv}% gerplain}
\nocite{pgfplotstable}
\nocite{programmingnotes}
\bibliography{pgfplots}
\end{document}
%
%%%%%%%%%%%%%%%%%%%%%%%%%%%%%%%%%%%%%%%%%%%%%%%%%%%%%%%%%%%%%%%%%%%%%%%%%%%%%
%
% Package pgfplots.sty documentation. 
%
% Copyright 2007/2008 by Christian Feuersaenger.
%
% This program is free software: you can redistribute it and/or modify
% it under the terms of the GNU General Public License as published by
% the Free Software Foundation, either version 3 of the License, or
% (at your option) any later version.
% 
% This program is distributed in the hope that it will be useful,
% but WITHOUT ANY WARRANTY; without even the implied warranty of
% MERCHANTABILITY or FITNESS FOR A PARTICULAR PURPOSE.  See the
% GNU General Public License for more details.
% 
% You should have received a copy of the GNU General Public License
% along with this program.  If not, see <http://www.gnu.org/licenses/>.
%
%
%%%%%%%%%%%%%%%%%%%%%%%%%%%%%%%%%%%%%%%%%%%%%%%%%%%%%%%%%%%%%%%%%%%%%%%%%%%%%
\input{pgfplots.preamble.tex}
%\RequirePackage[german,english,francais]{babel}

\pgfplotsmanualexternalexpensivetrue

%\usetikzlibrary{external}
%\tikzexternalize[prefix=figures/]{pgfplots}

\title{%
	Manual for Package \PGFPlots\\
	{\small Version \pgfplotsversion}\\
	{\small\href{http://sourceforge.net/projects/pgfplots}{http://sourceforge.net/projects/pgfplots}}}

%\includeonly{pgfplots.libs}


\begin{document}

\def\plotcoords{%
\addplot coordinates {
(5,8.312e-02)    (17,2.547e-02)   (49,7.407e-03)
(129,2.102e-03)  (321,5.874e-04)  (769,1.623e-04)
(1793,4.442e-05) (4097,1.207e-05) (9217,3.261e-06)
};

\addplot coordinates{
(7,8.472e-02)    (31,3.044e-02)    (111,1.022e-02)
(351,3.303e-03)  (1023,1.039e-03)  (2815,3.196e-04)
(7423,9.658e-05) (18943,2.873e-05) (47103,8.437e-06)
};

\addplot coordinates{
(9,7.881e-02)     (49,3.243e-02)    (209,1.232e-02)
(769,4.454e-03)   (2561,1.551e-03)  (7937,5.236e-04)
(23297,1.723e-04) (65537,5.545e-05) (178177,1.751e-05)
};

\addplot coordinates{
(11,6.887e-02)    (71,3.177e-02)     (351,1.341e-02)
(1471,5.334e-03)  (5503,2.027e-03)   (18943,7.415e-04)
(61183,2.628e-04) (187903,9.063e-05) (553983,3.053e-05)
};

\addplot coordinates{
(13,5.755e-02)     (97,2.925e-02)     (545,1.351e-02)
(2561,5.842e-03)   (10625,2.397e-03)  (40193,9.414e-04)
(141569,3.564e-04) (471041,1.308e-04) 
(1496065,4.670e-05)
};
}%
\maketitle
\begin{abstract}%
\PGFPlots\ draws high--quality function plots in normal or logarithmic scaling with a user-friendly interface directly in \TeX. The user supplies axis labels, legend entries and the plot coordinates for one or more plots and \PGFPlots\ applies axis scaling, computes any logarithms and axis ticks and draws the plots, supporting line plots, scatter plots, piecewise constant plots, bar plots, area plots, mesh-- and surface plots and some more. It is based on Till Tantau's package \PGF/\Tikz.
\end{abstract}
\tableofcontents
\section{Introduction}
This package provides tools to generate plots and labeled axes easily. It draws normal plots, logplots and semi-logplots, in two and three dimensions. Axis ticks, labels, legends (in case of multiple plots) can be added with key-value options. It can cycle through a set of predefined line/marker/color specifications. In summary, its purpose is to simplify the generation of high-quality function and/or data plots, and solving the problems of
\begin{itemize}
	\item consistency of document and font type and font size,
	\item direct use of \TeX\ math mode in axis descriptions,
	\item consistency of data and figures (no third party tool necessary),
	\item inter document consistency using preamble configurations and styles.
\end{itemize}
Although not necessary, separate |.pdf| or |.eps| graphics can be generated using the |external| library developed as part of \Tikz.

\PGFPlots\ is build completely on \Tikz/\PGF. Knowledge of \Tikz\ will simplify the work with \PGFPlots, although it is not required.

\section[About PGFPlots: Preliminaries]{About {\normalfont\PGFPlots}: Preliminaries}
This section contains information about upgrades, the team, the installation (in case you need to do it manually) and troubleshooting. You may skip it completely except for the upgrade remarks.

However, note that this library requires at least \PGF\ version $2.00$. At the time of this writing, many \TeX-distributions still contain the older \PGF\ version $1.18$, so it may be necessary to install a recent \PGF\ prior to using \PGFPlots.

\subsection{Components}
\PGFPlots\ comes with two components:
\begin{enumerate}
	\item the plotting component (which you are currently reading) and
	\item the \PGFPlotstable\ component which simplifies number formatting and postprocessing of numerical tables. It comes as a separate package and has its own manual \href{file:pgfplotstable.pdf}{pgfplotstable.pdf}.
\end{enumerate}

\subsection{Upgrade remarks}
This release provides a lot of improvements which can be found in all detail in \texttt{ChangeLog} for interested readers. However, some attention is useful with respect to the following changes.

\subsubsection{New Optional Features}
\begin{enumerate}
	\item \PGFPlots\ 1.3 comes with user interface improvements. The technical distinction between ``behavior options'' and ``style options'' of older versions is no longer necessary (although still fully supported).

	\item \PGFPlots\ 1.3 has a new feature which allows to \emph{move axis labels tight to tick labels} automatically.

	Since this affects the spacing, it is not enabled be default.

	Use
\begin{codeexample}[code only]
\usepackage{pgfplots}
\pgfplotsset{compat=1.3}
\end{codeexample}
	in your preamble to benefit from the improved spacing\footnote{The |compat=1.3| setting is necessary for new features which move axis labels around like reversed axis. However, \PGFPlots\ will still work as it does in earlier versions, your documents will remain the same if you don't set it explicitly.}.
	Take a look at the next page for the precise definition of |compat|.

	\item \PGFPlots\ 1.3 now supports reversed axes. It is no longer necessary to use work--arounds with negative units.
\pgfkeys{/pdflinks/search key prefixes in/.add={/pgfplots/,}{}}

	Take a look at the |x dir=reverse| key.

	Existing work--arounds will still function properly. Use |\pgfplotsset{compat=1.3}| together with |x dir=reverse| to switch to the new version.
\end{enumerate}

\subsubsection{Old Features Which May Need Attention}
\begin{enumerate}
	\item The |scatter/classes| feature produces proper legends as of version 1.3. This may change the appearance of existing legends of plots with |scatter/classes|.

	\item Due to a small math bug in \PGF\ $2.00$, you \emph{can't} use the math expression `|-x^2|'. It is necessary to use `|0-x^2|' instead. The same holds for
		`|exp(-x^2)|'; use `|exp(0-x^2)|' instead.

		This will be fixed with \PGF\ $>2.00$.

	\item Starting with \PGFPlots\ $1.1$, |\tikzstyle| should \emph{no longer be used} to set \PGFPlots\ options.
	
	Although |\tikzstyle| is still supported for some older \PGFPlots\ options, you should replace any occurance of |\tikzstyle| with |\pgfplotsset{|\meta{style name}|/.style={|\meta{key-value-list}|}}| or the associated |/.append style| variant. See section~\ref{sec:styles} for more detail.
\end{enumerate}
I apologize for any inconvenience caused by these changes.

\begin{pgfplotskey}{compat=\mchoice{1.3,pre 1.3,default,newest} (initially default)}
	The preamble configuration 
\begin{codeexample}[code only]
\usepackage{pgfplots}
\pgfplotsset{compat=1.3}
\end{codeexample}
	allows to choose between backwards compatibility and most recent features.

	\PGFPlots\ version 1.3 comes with new features which may lead to different spacings compared to earlier versions:
	\begin{enumerate}
		\item Version 1.3 comes with the |xlabel near ticks| feature which places (in this case $x$) axis labels ``near'' the tick labels, taking the size of tick labels into account.
		
		Older versions used |xlabel absolute|, i.e.\ an absolute distance between the axis and axis labels (now matter how large or small tick labels are).

		The initial setting \emph{keeps backwards compatibility}.

		You are encouraged to use
\begin{codeexample}[code only]
\usepackage{pgfplots}
\pgfplotsset{compat=1.3}
\end{codeexample}
		in your preamble.
		
		\item Version 1.3 fixes a bug with |anchor=south west| which occurred mainly for reversed axes: the anchors where upside--down. This is fixed now.

		
		If you have written some work--around and want to keep it going, use

		|\pgfplotsset{compat/anchors=pre 1.3}|\pgfmanualpdflabel{/pgfplots/compat/anchors}{} 

		to disable the bug fix.
	\end{enumerate}

	Use |\pgfplotsset{compat=default}| to restore the factory settings.

	The setting |\pgfplotsset{compat=newest}| will always use features of the most recent version. This might result in small changes in the document's appearance.
\end{pgfplotskey}

\subsection{The Team}
\PGFPlots\ has been written mainly by Christian Feuers�nger with many improvements of Pascal Wolkotte and Nick Papior Andersen as a spare time project. We hope it is useful and provides valuable plots.

If you are interested in writing something but don't know how, consider reading the auxiliary manual \href{file:TeX-programming-notes.pdf}{TeX-programming-notes.pdf} which comes with \PGFPlots. It is far from complete, but maybe it is a good starting point (at least for more literature).

\subsection{Acknowledgements}
I thank God for all hours of enjoyed programming. I thank Pascal Wolkotte and Nick Papior Andersen for their programming efforts and contributions as part of the development team. I thank Stefan Tibus, who contributed the |plot shell| feature. I thank Tom Cashman for the contribution of the |reverse legend| feature. Special thanks go to Stefan Pinnow whose tests of \PGFPlots\ lead to numerous quality improvements. Furthermore, I thank Dr.~Schweitzer for many fruitful discussions and Dr.~Meine for his ideas and suggestions. Thanks as well to the many international contributors who provided feature requests or identified bugs or simply manual improvements!

Last but not least, I thank Till Tantau and Mark Wibrow for their excellent graphics (and more) package \PGF\ and \Tikz\ which is the base of \PGFPlots.

\include{pgfplots.install}%
\include{pgfplots.intro}%
\include{pgfplots.reference}%
\include{pgfplots.libs}%
\include{pgfplots.resources}%
\include{pgfplots.importexport}%
\include{pgfplots.basic.reference}%

\printindex

\bibliographystyle{abbrv} %gerapali} %gerabbrv} %gerunsrt.bst} %gerabbrv}% gerplain}
\nocite{pgfplotstable}
\nocite{programmingnotes}
\bibliography{pgfplots}
\end{document}
%
%%%%%%%%%%%%%%%%%%%%%%%%%%%%%%%%%%%%%%%%%%%%%%%%%%%%%%%%%%%%%%%%%%%%%%%%%%%%%
%
% Package pgfplots.sty documentation. 
%
% Copyright 2007/2008 by Christian Feuersaenger.
%
% This program is free software: you can redistribute it and/or modify
% it under the terms of the GNU General Public License as published by
% the Free Software Foundation, either version 3 of the License, or
% (at your option) any later version.
% 
% This program is distributed in the hope that it will be useful,
% but WITHOUT ANY WARRANTY; without even the implied warranty of
% MERCHANTABILITY or FITNESS FOR A PARTICULAR PURPOSE.  See the
% GNU General Public License for more details.
% 
% You should have received a copy of the GNU General Public License
% along with this program.  If not, see <http://www.gnu.org/licenses/>.
%
%
%%%%%%%%%%%%%%%%%%%%%%%%%%%%%%%%%%%%%%%%%%%%%%%%%%%%%%%%%%%%%%%%%%%%%%%%%%%%%
\input{pgfplots.preamble.tex}
%\RequirePackage[german,english,francais]{babel}

\pgfplotsmanualexternalexpensivetrue

%\usetikzlibrary{external}
%\tikzexternalize[prefix=figures/]{pgfplots}

\title{%
	Manual for Package \PGFPlots\\
	{\small Version \pgfplotsversion}\\
	{\small\href{http://sourceforge.net/projects/pgfplots}{http://sourceforge.net/projects/pgfplots}}}

%\includeonly{pgfplots.libs}


\begin{document}

\def\plotcoords{%
\addplot coordinates {
(5,8.312e-02)    (17,2.547e-02)   (49,7.407e-03)
(129,2.102e-03)  (321,5.874e-04)  (769,1.623e-04)
(1793,4.442e-05) (4097,1.207e-05) (9217,3.261e-06)
};

\addplot coordinates{
(7,8.472e-02)    (31,3.044e-02)    (111,1.022e-02)
(351,3.303e-03)  (1023,1.039e-03)  (2815,3.196e-04)
(7423,9.658e-05) (18943,2.873e-05) (47103,8.437e-06)
};

\addplot coordinates{
(9,7.881e-02)     (49,3.243e-02)    (209,1.232e-02)
(769,4.454e-03)   (2561,1.551e-03)  (7937,5.236e-04)
(23297,1.723e-04) (65537,5.545e-05) (178177,1.751e-05)
};

\addplot coordinates{
(11,6.887e-02)    (71,3.177e-02)     (351,1.341e-02)
(1471,5.334e-03)  (5503,2.027e-03)   (18943,7.415e-04)
(61183,2.628e-04) (187903,9.063e-05) (553983,3.053e-05)
};

\addplot coordinates{
(13,5.755e-02)     (97,2.925e-02)     (545,1.351e-02)
(2561,5.842e-03)   (10625,2.397e-03)  (40193,9.414e-04)
(141569,3.564e-04) (471041,1.308e-04) 
(1496065,4.670e-05)
};
}%
\maketitle
\begin{abstract}%
\PGFPlots\ draws high--quality function plots in normal or logarithmic scaling with a user-friendly interface directly in \TeX. The user supplies axis labels, legend entries and the plot coordinates for one or more plots and \PGFPlots\ applies axis scaling, computes any logarithms and axis ticks and draws the plots, supporting line plots, scatter plots, piecewise constant plots, bar plots, area plots, mesh-- and surface plots and some more. It is based on Till Tantau's package \PGF/\Tikz.
\end{abstract}
\tableofcontents
\section{Introduction}
This package provides tools to generate plots and labeled axes easily. It draws normal plots, logplots and semi-logplots, in two and three dimensions. Axis ticks, labels, legends (in case of multiple plots) can be added with key-value options. It can cycle through a set of predefined line/marker/color specifications. In summary, its purpose is to simplify the generation of high-quality function and/or data plots, and solving the problems of
\begin{itemize}
	\item consistency of document and font type and font size,
	\item direct use of \TeX\ math mode in axis descriptions,
	\item consistency of data and figures (no third party tool necessary),
	\item inter document consistency using preamble configurations and styles.
\end{itemize}
Although not necessary, separate |.pdf| or |.eps| graphics can be generated using the |external| library developed as part of \Tikz.

\PGFPlots\ is build completely on \Tikz/\PGF. Knowledge of \Tikz\ will simplify the work with \PGFPlots, although it is not required.

\section[About PGFPlots: Preliminaries]{About {\normalfont\PGFPlots}: Preliminaries}
This section contains information about upgrades, the team, the installation (in case you need to do it manually) and troubleshooting. You may skip it completely except for the upgrade remarks.

However, note that this library requires at least \PGF\ version $2.00$. At the time of this writing, many \TeX-distributions still contain the older \PGF\ version $1.18$, so it may be necessary to install a recent \PGF\ prior to using \PGFPlots.

\subsection{Components}
\PGFPlots\ comes with two components:
\begin{enumerate}
	\item the plotting component (which you are currently reading) and
	\item the \PGFPlotstable\ component which simplifies number formatting and postprocessing of numerical tables. It comes as a separate package and has its own manual \href{file:pgfplotstable.pdf}{pgfplotstable.pdf}.
\end{enumerate}

\subsection{Upgrade remarks}
This release provides a lot of improvements which can be found in all detail in \texttt{ChangeLog} for interested readers. However, some attention is useful with respect to the following changes.

\subsubsection{New Optional Features}
\begin{enumerate}
	\item \PGFPlots\ 1.3 comes with user interface improvements. The technical distinction between ``behavior options'' and ``style options'' of older versions is no longer necessary (although still fully supported).

	\item \PGFPlots\ 1.3 has a new feature which allows to \emph{move axis labels tight to tick labels} automatically.

	Since this affects the spacing, it is not enabled be default.

	Use
\begin{codeexample}[code only]
\usepackage{pgfplots}
\pgfplotsset{compat=1.3}
\end{codeexample}
	in your preamble to benefit from the improved spacing\footnote{The |compat=1.3| setting is necessary for new features which move axis labels around like reversed axis. However, \PGFPlots\ will still work as it does in earlier versions, your documents will remain the same if you don't set it explicitly.}.
	Take a look at the next page for the precise definition of |compat|.

	\item \PGFPlots\ 1.3 now supports reversed axes. It is no longer necessary to use work--arounds with negative units.
\pgfkeys{/pdflinks/search key prefixes in/.add={/pgfplots/,}{}}

	Take a look at the |x dir=reverse| key.

	Existing work--arounds will still function properly. Use |\pgfplotsset{compat=1.3}| together with |x dir=reverse| to switch to the new version.
\end{enumerate}

\subsubsection{Old Features Which May Need Attention}
\begin{enumerate}
	\item The |scatter/classes| feature produces proper legends as of version 1.3. This may change the appearance of existing legends of plots with |scatter/classes|.

	\item Due to a small math bug in \PGF\ $2.00$, you \emph{can't} use the math expression `|-x^2|'. It is necessary to use `|0-x^2|' instead. The same holds for
		`|exp(-x^2)|'; use `|exp(0-x^2)|' instead.

		This will be fixed with \PGF\ $>2.00$.

	\item Starting with \PGFPlots\ $1.1$, |\tikzstyle| should \emph{no longer be used} to set \PGFPlots\ options.
	
	Although |\tikzstyle| is still supported for some older \PGFPlots\ options, you should replace any occurance of |\tikzstyle| with |\pgfplotsset{|\meta{style name}|/.style={|\meta{key-value-list}|}}| or the associated |/.append style| variant. See section~\ref{sec:styles} for more detail.
\end{enumerate}
I apologize for any inconvenience caused by these changes.

\begin{pgfplotskey}{compat=\mchoice{1.3,pre 1.3,default,newest} (initially default)}
	The preamble configuration 
\begin{codeexample}[code only]
\usepackage{pgfplots}
\pgfplotsset{compat=1.3}
\end{codeexample}
	allows to choose between backwards compatibility and most recent features.

	\PGFPlots\ version 1.3 comes with new features which may lead to different spacings compared to earlier versions:
	\begin{enumerate}
		\item Version 1.3 comes with the |xlabel near ticks| feature which places (in this case $x$) axis labels ``near'' the tick labels, taking the size of tick labels into account.
		
		Older versions used |xlabel absolute|, i.e.\ an absolute distance between the axis and axis labels (now matter how large or small tick labels are).

		The initial setting \emph{keeps backwards compatibility}.

		You are encouraged to use
\begin{codeexample}[code only]
\usepackage{pgfplots}
\pgfplotsset{compat=1.3}
\end{codeexample}
		in your preamble.
		
		\item Version 1.3 fixes a bug with |anchor=south west| which occurred mainly for reversed axes: the anchors where upside--down. This is fixed now.

		
		If you have written some work--around and want to keep it going, use

		|\pgfplotsset{compat/anchors=pre 1.3}|\pgfmanualpdflabel{/pgfplots/compat/anchors}{} 

		to disable the bug fix.
	\end{enumerate}

	Use |\pgfplotsset{compat=default}| to restore the factory settings.

	The setting |\pgfplotsset{compat=newest}| will always use features of the most recent version. This might result in small changes in the document's appearance.
\end{pgfplotskey}

\subsection{The Team}
\PGFPlots\ has been written mainly by Christian Feuers�nger with many improvements of Pascal Wolkotte and Nick Papior Andersen as a spare time project. We hope it is useful and provides valuable plots.

If you are interested in writing something but don't know how, consider reading the auxiliary manual \href{file:TeX-programming-notes.pdf}{TeX-programming-notes.pdf} which comes with \PGFPlots. It is far from complete, but maybe it is a good starting point (at least for more literature).

\subsection{Acknowledgements}
I thank God for all hours of enjoyed programming. I thank Pascal Wolkotte and Nick Papior Andersen for their programming efforts and contributions as part of the development team. I thank Stefan Tibus, who contributed the |plot shell| feature. I thank Tom Cashman for the contribution of the |reverse legend| feature. Special thanks go to Stefan Pinnow whose tests of \PGFPlots\ lead to numerous quality improvements. Furthermore, I thank Dr.~Schweitzer for many fruitful discussions and Dr.~Meine for his ideas and suggestions. Thanks as well to the many international contributors who provided feature requests or identified bugs or simply manual improvements!

Last but not least, I thank Till Tantau and Mark Wibrow for their excellent graphics (and more) package \PGF\ and \Tikz\ which is the base of \PGFPlots.

\include{pgfplots.install}%
\include{pgfplots.intro}%
\include{pgfplots.reference}%
\include{pgfplots.libs}%
\include{pgfplots.resources}%
\include{pgfplots.importexport}%
\include{pgfplots.basic.reference}%

\printindex

\bibliographystyle{abbrv} %gerapali} %gerabbrv} %gerunsrt.bst} %gerabbrv}% gerplain}
\nocite{pgfplotstable}
\nocite{programmingnotes}
\bibliography{pgfplots}
\end{document}
%
\include{pgfplots.basic.reference}%

\printindex

\bibliographystyle{abbrv} %gerapali} %gerabbrv} %gerunsrt.bst} %gerabbrv}% gerplain}
\nocite{pgfplotstable}
\nocite{programmingnotes}
\bibliography{pgfplots}
\end{document}
%
%%%%%%%%%%%%%%%%%%%%%%%%%%%%%%%%%%%%%%%%%%%%%%%%%%%%%%%%%%%%%%%%%%%%%%%%%%%%%
%
% Package pgfplots.sty documentation. 
%
% Copyright 2007/2008 by Christian Feuersaenger.
%
% This program is free software: you can redistribute it and/or modify
% it under the terms of the GNU General Public License as published by
% the Free Software Foundation, either version 3 of the License, or
% (at your option) any later version.
% 
% This program is distributed in the hope that it will be useful,
% but WITHOUT ANY WARRANTY; without even the implied warranty of
% MERCHANTABILITY or FITNESS FOR A PARTICULAR PURPOSE.  See the
% GNU General Public License for more details.
% 
% You should have received a copy of the GNU General Public License
% along with this program.  If not, see <http://www.gnu.org/licenses/>.
%
%
%%%%%%%%%%%%%%%%%%%%%%%%%%%%%%%%%%%%%%%%%%%%%%%%%%%%%%%%%%%%%%%%%%%%%%%%%%%%%
%%%%%%%%%%%%%%%%%%%%%%%%%%%%%%%%%%%%%%%%%%%%%%%%%%%%%%%%%%%%%%%%%%%%%%%%%%%%%
%
% Package pgfplots.sty documentation. 
%
% Copyright 2007/2008 by Christian Feuersaenger.
%
% This program is free software: you can redistribute it and/or modify
% it under the terms of the GNU General Public License as published by
% the Free Software Foundation, either version 3 of the License, or
% (at your option) any later version.
% 
% This program is distributed in the hope that it will be useful,
% but WITHOUT ANY WARRANTY; without even the implied warranty of
% MERCHANTABILITY or FITNESS FOR A PARTICULAR PURPOSE.  See the
% GNU General Public License for more details.
% 
% You should have received a copy of the GNU General Public License
% along with this program.  If not, see <http://www.gnu.org/licenses/>.
%
%
%%%%%%%%%%%%%%%%%%%%%%%%%%%%%%%%%%%%%%%%%%%%%%%%%%%%%%%%%%%%%%%%%%%%%%%%%%%%%
\documentclass[a4paper]{ltxdoc}

\usepackage{makeidx}

% DON't let hyperref overload the format of index and glossary. 
% I want to do that on my own in the stylefiles for makeindex...
\makeatletter
\let\@old@wrindex=\@wrindex
\makeatother

\usepackage{ifpdf}
\ifpdf
	\usepackage[pdftex]{hyperref}
\else
	\usepackage[dvipdfm]{hyperref}
\fi
\hypersetup{pdfborder={0 0 0}}

\makeatletter
\let\@wrindex=\@old@wrindex
\makeatother

% Formatiere Seitennummern im Index:
\newcommand{\indexpageno}[1]{%
	{\bfseries\hyperpage{#1}}%
}


\newcommand{\R}{\mathbb{R}}
\newcommand{\N}{\mathbb{N}}
\newcommand{\Z}{\mathbb{Z}}

\long\def\COMMENTLOWLEVEL#1\ENDCOMMENT{}
\def\ENDCOMMENT{}

\usepackage{textcomp}
\usepackage{booktabs}

\usepackage{calc}
\usepackage[formats]{listings}
%\usepackage{courier} % don't use it - the '^' character can't be copy-pasted in courier

\usepackage{array}
\lstset{%
	basicstyle=\ttfamily,
	language=[LaTeX]tex, % Seems as if \lstset{language=tex} must be invoked BEFORE loading tikz!?
	tabsize=4,
	breaklines=true,
	breakindent=0pt
}

\ifpdf
	\pdfinfo {
		/Author	(Christian Feuersaenger)
	}

\else
	\def\pgfsysdriver{pgfsys-dvipdfm.def}
\fi
%\def\pgfsysdriver{pgfsys-pdftex.def}
\usepackage{pgfplots}

\ifpdf
	\usetikzlibrary{pgfplotsclickable}
\fi

%\usepackage{fp}
% ATTENTION:
% this requires pgf version NEWER than 2.00 :
%\usetikzlibrary{fixedpointarithmetic}

\usetikzlibrary{dateplot}

\usepackage[a4paper,left=2.25cm,right=2.25cm,top=2.5cm,bottom=2.5cm,nohead]{geometry}
\usepackage{amsmath,amssymb}
\usepackage{xxcolor}
\usepackage{pifont}
\usepackage[latin1]{inputenc}
\usepackage{amsmath}
\usepackage{eurosym}
\input{pgfplots-macros}

\pgfqkeys{/codeexample}{%
	every codeexample/.style={
		width=8cm,
		/pgfplots/every axis/.append style={legend style={fill=graphicbackground}}
	},
	tabsize=4,
}
\pgfplotsset{
	every axis/.append style={width=7cm},
	filter discard warning=false,
}

\usetikzlibrary{backgrounds,patterns}
% Global styles:
\tikzset{
  shape example/.style={
    color=black!30,
    draw,
    fill=yellow!30,
    line width=.5cm,
    inner xsep=2.5cm,
    inner ysep=0.5cm}
}

\newcommand{\FIXME}[1]{\textcolor{red}{(FIXME: #1)}}

% fuer endvironment 'sidewaysfigure' bspw
% \usepackage{rotating}

\newcommand\Tikz{Ti\textit kZ}
\newcommand\PGF{\textsc{pgf}}
\newcommand\PGFPlots{\textsc{pgfplots}}
\def\PGFPlotstable{\textsc{PgfplotsTable}}


\makeindex

% Fix overful hboxes automatically:
\tolerance=2000
\emergencystretch=10pt

\tikzset{prefix=gnuplot/pgfplots_} % prefix for 'plot function'

\author{%
	Christian Feuers\"anger\footnote{\url{http://wissrech.ins.uni-bonn.de/people/feuersaenger}}\\%
	Institut f\"ur Numerische Simulation\\
	Universit\"at Bonn, Germany}


%\RequirePackage[german,english,francais]{babel}

\pgfplotsmanualexternalexpensivetrue

%\usetikzlibrary{external}
%\tikzexternalize[prefix=figures/]{pgfplots}

\title{%
	Manual for Package \PGFPlots\\
	{\small Version \pgfplotsversion}\\
	{\small\href{http://sourceforge.net/projects/pgfplots}{http://sourceforge.net/projects/pgfplots}}}

%\includeonly{pgfplots.libs}


\begin{document}

\def\plotcoords{%
\addplot coordinates {
(5,8.312e-02)    (17,2.547e-02)   (49,7.407e-03)
(129,2.102e-03)  (321,5.874e-04)  (769,1.623e-04)
(1793,4.442e-05) (4097,1.207e-05) (9217,3.261e-06)
};

\addplot coordinates{
(7,8.472e-02)    (31,3.044e-02)    (111,1.022e-02)
(351,3.303e-03)  (1023,1.039e-03)  (2815,3.196e-04)
(7423,9.658e-05) (18943,2.873e-05) (47103,8.437e-06)
};

\addplot coordinates{
(9,7.881e-02)     (49,3.243e-02)    (209,1.232e-02)
(769,4.454e-03)   (2561,1.551e-03)  (7937,5.236e-04)
(23297,1.723e-04) (65537,5.545e-05) (178177,1.751e-05)
};

\addplot coordinates{
(11,6.887e-02)    (71,3.177e-02)     (351,1.341e-02)
(1471,5.334e-03)  (5503,2.027e-03)   (18943,7.415e-04)
(61183,2.628e-04) (187903,9.063e-05) (553983,3.053e-05)
};

\addplot coordinates{
(13,5.755e-02)     (97,2.925e-02)     (545,1.351e-02)
(2561,5.842e-03)   (10625,2.397e-03)  (40193,9.414e-04)
(141569,3.564e-04) (471041,1.308e-04) 
(1496065,4.670e-05)
};
}%
\maketitle
\begin{abstract}%
\PGFPlots\ draws high--quality function plots in normal or logarithmic scaling with a user-friendly interface directly in \TeX. The user supplies axis labels, legend entries and the plot coordinates for one or more plots and \PGFPlots\ applies axis scaling, computes any logarithms and axis ticks and draws the plots, supporting line plots, scatter plots, piecewise constant plots, bar plots, area plots, mesh-- and surface plots and some more. It is based on Till Tantau's package \PGF/\Tikz.
\end{abstract}
\tableofcontents
\section{Introduction}
This package provides tools to generate plots and labeled axes easily. It draws normal plots, logplots and semi-logplots, in two and three dimensions. Axis ticks, labels, legends (in case of multiple plots) can be added with key-value options. It can cycle through a set of predefined line/marker/color specifications. In summary, its purpose is to simplify the generation of high-quality function and/or data plots, and solving the problems of
\begin{itemize}
	\item consistency of document and font type and font size,
	\item direct use of \TeX\ math mode in axis descriptions,
	\item consistency of data and figures (no third party tool necessary),
	\item inter document consistency using preamble configurations and styles.
\end{itemize}
Although not necessary, separate |.pdf| or |.eps| graphics can be generated using the |external| library developed as part of \Tikz.

\PGFPlots\ is build completely on \Tikz/\PGF. Knowledge of \Tikz\ will simplify the work with \PGFPlots, although it is not required.

\section[About PGFPlots: Preliminaries]{About {\normalfont\PGFPlots}: Preliminaries}
This section contains information about upgrades, the team, the installation (in case you need to do it manually) and troubleshooting. You may skip it completely except for the upgrade remarks.

However, note that this library requires at least \PGF\ version $2.00$. At the time of this writing, many \TeX-distributions still contain the older \PGF\ version $1.18$, so it may be necessary to install a recent \PGF\ prior to using \PGFPlots.

\subsection{Components}
\PGFPlots\ comes with two components:
\begin{enumerate}
	\item the plotting component (which you are currently reading) and
	\item the \PGFPlotstable\ component which simplifies number formatting and postprocessing of numerical tables. It comes as a separate package and has its own manual \href{file:pgfplotstable.pdf}{pgfplotstable.pdf}.
\end{enumerate}

\subsection{Upgrade remarks}
This release provides a lot of improvements which can be found in all detail in \texttt{ChangeLog} for interested readers. However, some attention is useful with respect to the following changes.

\subsubsection{New Optional Features}
\begin{enumerate}
	\item \PGFPlots\ 1.3 comes with user interface improvements. The technical distinction between ``behavior options'' and ``style options'' of older versions is no longer necessary (although still fully supported).

	\item \PGFPlots\ 1.3 has a new feature which allows to \emph{move axis labels tight to tick labels} automatically.

	Since this affects the spacing, it is not enabled be default.

	Use
\begin{codeexample}[code only]
\usepackage{pgfplots}
\pgfplotsset{compat=1.3}
\end{codeexample}
	in your preamble to benefit from the improved spacing\footnote{The |compat=1.3| setting is necessary for new features which move axis labels around like reversed axis. However, \PGFPlots\ will still work as it does in earlier versions, your documents will remain the same if you don't set it explicitly.}.
	Take a look at the next page for the precise definition of |compat|.

	\item \PGFPlots\ 1.3 now supports reversed axes. It is no longer necessary to use work--arounds with negative units.
\pgfkeys{/pdflinks/search key prefixes in/.add={/pgfplots/,}{}}

	Take a look at the |x dir=reverse| key.

	Existing work--arounds will still function properly. Use |\pgfplotsset{compat=1.3}| together with |x dir=reverse| to switch to the new version.
\end{enumerate}

\subsubsection{Old Features Which May Need Attention}
\begin{enumerate}
	\item The |scatter/classes| feature produces proper legends as of version 1.3. This may change the appearance of existing legends of plots with |scatter/classes|.

	\item Due to a small math bug in \PGF\ $2.00$, you \emph{can't} use the math expression `|-x^2|'. It is necessary to use `|0-x^2|' instead. The same holds for
		`|exp(-x^2)|'; use `|exp(0-x^2)|' instead.

		This will be fixed with \PGF\ $>2.00$.

	\item Starting with \PGFPlots\ $1.1$, |\tikzstyle| should \emph{no longer be used} to set \PGFPlots\ options.
	
	Although |\tikzstyle| is still supported for some older \PGFPlots\ options, you should replace any occurance of |\tikzstyle| with |\pgfplotsset{|\meta{style name}|/.style={|\meta{key-value-list}|}}| or the associated |/.append style| variant. See section~\ref{sec:styles} for more detail.
\end{enumerate}
I apologize for any inconvenience caused by these changes.

\begin{pgfplotskey}{compat=\mchoice{1.3,pre 1.3,default,newest} (initially default)}
	The preamble configuration 
\begin{codeexample}[code only]
\usepackage{pgfplots}
\pgfplotsset{compat=1.3}
\end{codeexample}
	allows to choose between backwards compatibility and most recent features.

	\PGFPlots\ version 1.3 comes with new features which may lead to different spacings compared to earlier versions:
	\begin{enumerate}
		\item Version 1.3 comes with the |xlabel near ticks| feature which places (in this case $x$) axis labels ``near'' the tick labels, taking the size of tick labels into account.
		
		Older versions used |xlabel absolute|, i.e.\ an absolute distance between the axis and axis labels (now matter how large or small tick labels are).

		The initial setting \emph{keeps backwards compatibility}.

		You are encouraged to use
\begin{codeexample}[code only]
\usepackage{pgfplots}
\pgfplotsset{compat=1.3}
\end{codeexample}
		in your preamble.
		
		\item Version 1.3 fixes a bug with |anchor=south west| which occurred mainly for reversed axes: the anchors where upside--down. This is fixed now.

		
		If you have written some work--around and want to keep it going, use

		|\pgfplotsset{compat/anchors=pre 1.3}|\pgfmanualpdflabel{/pgfplots/compat/anchors}{} 

		to disable the bug fix.
	\end{enumerate}

	Use |\pgfplotsset{compat=default}| to restore the factory settings.

	The setting |\pgfplotsset{compat=newest}| will always use features of the most recent version. This might result in small changes in the document's appearance.
\end{pgfplotskey}

\subsection{The Team}
\PGFPlots\ has been written mainly by Christian Feuers�nger with many improvements of Pascal Wolkotte and Nick Papior Andersen as a spare time project. We hope it is useful and provides valuable plots.

If you are interested in writing something but don't know how, consider reading the auxiliary manual \href{file:TeX-programming-notes.pdf}{TeX-programming-notes.pdf} which comes with \PGFPlots. It is far from complete, but maybe it is a good starting point (at least for more literature).

\subsection{Acknowledgements}
I thank God for all hours of enjoyed programming. I thank Pascal Wolkotte and Nick Papior Andersen for their programming efforts and contributions as part of the development team. I thank Stefan Tibus, who contributed the |plot shell| feature. I thank Tom Cashman for the contribution of the |reverse legend| feature. Special thanks go to Stefan Pinnow whose tests of \PGFPlots\ lead to numerous quality improvements. Furthermore, I thank Dr.~Schweitzer for many fruitful discussions and Dr.~Meine for his ideas and suggestions. Thanks as well to the many international contributors who provided feature requests or identified bugs or simply manual improvements!

Last but not least, I thank Till Tantau and Mark Wibrow for their excellent graphics (and more) package \PGF\ and \Tikz\ which is the base of \PGFPlots.

%%%%%%%%%%%%%%%%%%%%%%%%%%%%%%%%%%%%%%%%%%%%%%%%%%%%%%%%%%%%%%%%%%%%%%%%%%%%%
%
% Package pgfplots.sty documentation. 
%
% Copyright 2007/2008 by Christian Feuersaenger.
%
% This program is free software: you can redistribute it and/or modify
% it under the terms of the GNU General Public License as published by
% the Free Software Foundation, either version 3 of the License, or
% (at your option) any later version.
% 
% This program is distributed in the hope that it will be useful,
% but WITHOUT ANY WARRANTY; without even the implied warranty of
% MERCHANTABILITY or FITNESS FOR A PARTICULAR PURPOSE.  See the
% GNU General Public License for more details.
% 
% You should have received a copy of the GNU General Public License
% along with this program.  If not, see <http://www.gnu.org/licenses/>.
%
%
%%%%%%%%%%%%%%%%%%%%%%%%%%%%%%%%%%%%%%%%%%%%%%%%%%%%%%%%%%%%%%%%%%%%%%%%%%%%%
\input{pgfplots.preamble.tex}
%\RequirePackage[german,english,francais]{babel}

\pgfplotsmanualexternalexpensivetrue

%\usetikzlibrary{external}
%\tikzexternalize[prefix=figures/]{pgfplots}

\title{%
	Manual for Package \PGFPlots\\
	{\small Version \pgfplotsversion}\\
	{\small\href{http://sourceforge.net/projects/pgfplots}{http://sourceforge.net/projects/pgfplots}}}

%\includeonly{pgfplots.libs}


\begin{document}

\def\plotcoords{%
\addplot coordinates {
(5,8.312e-02)    (17,2.547e-02)   (49,7.407e-03)
(129,2.102e-03)  (321,5.874e-04)  (769,1.623e-04)
(1793,4.442e-05) (4097,1.207e-05) (9217,3.261e-06)
};

\addplot coordinates{
(7,8.472e-02)    (31,3.044e-02)    (111,1.022e-02)
(351,3.303e-03)  (1023,1.039e-03)  (2815,3.196e-04)
(7423,9.658e-05) (18943,2.873e-05) (47103,8.437e-06)
};

\addplot coordinates{
(9,7.881e-02)     (49,3.243e-02)    (209,1.232e-02)
(769,4.454e-03)   (2561,1.551e-03)  (7937,5.236e-04)
(23297,1.723e-04) (65537,5.545e-05) (178177,1.751e-05)
};

\addplot coordinates{
(11,6.887e-02)    (71,3.177e-02)     (351,1.341e-02)
(1471,5.334e-03)  (5503,2.027e-03)   (18943,7.415e-04)
(61183,2.628e-04) (187903,9.063e-05) (553983,3.053e-05)
};

\addplot coordinates{
(13,5.755e-02)     (97,2.925e-02)     (545,1.351e-02)
(2561,5.842e-03)   (10625,2.397e-03)  (40193,9.414e-04)
(141569,3.564e-04) (471041,1.308e-04) 
(1496065,4.670e-05)
};
}%
\maketitle
\begin{abstract}%
\PGFPlots\ draws high--quality function plots in normal or logarithmic scaling with a user-friendly interface directly in \TeX. The user supplies axis labels, legend entries and the plot coordinates for one or more plots and \PGFPlots\ applies axis scaling, computes any logarithms and axis ticks and draws the plots, supporting line plots, scatter plots, piecewise constant plots, bar plots, area plots, mesh-- and surface plots and some more. It is based on Till Tantau's package \PGF/\Tikz.
\end{abstract}
\tableofcontents
\section{Introduction}
This package provides tools to generate plots and labeled axes easily. It draws normal plots, logplots and semi-logplots, in two and three dimensions. Axis ticks, labels, legends (in case of multiple plots) can be added with key-value options. It can cycle through a set of predefined line/marker/color specifications. In summary, its purpose is to simplify the generation of high-quality function and/or data plots, and solving the problems of
\begin{itemize}
	\item consistency of document and font type and font size,
	\item direct use of \TeX\ math mode in axis descriptions,
	\item consistency of data and figures (no third party tool necessary),
	\item inter document consistency using preamble configurations and styles.
\end{itemize}
Although not necessary, separate |.pdf| or |.eps| graphics can be generated using the |external| library developed as part of \Tikz.

\PGFPlots\ is build completely on \Tikz/\PGF. Knowledge of \Tikz\ will simplify the work with \PGFPlots, although it is not required.

\section[About PGFPlots: Preliminaries]{About {\normalfont\PGFPlots}: Preliminaries}
This section contains information about upgrades, the team, the installation (in case you need to do it manually) and troubleshooting. You may skip it completely except for the upgrade remarks.

However, note that this library requires at least \PGF\ version $2.00$. At the time of this writing, many \TeX-distributions still contain the older \PGF\ version $1.18$, so it may be necessary to install a recent \PGF\ prior to using \PGFPlots.

\subsection{Components}
\PGFPlots\ comes with two components:
\begin{enumerate}
	\item the plotting component (which you are currently reading) and
	\item the \PGFPlotstable\ component which simplifies number formatting and postprocessing of numerical tables. It comes as a separate package and has its own manual \href{file:pgfplotstable.pdf}{pgfplotstable.pdf}.
\end{enumerate}

\subsection{Upgrade remarks}
This release provides a lot of improvements which can be found in all detail in \texttt{ChangeLog} for interested readers. However, some attention is useful with respect to the following changes.

\subsubsection{New Optional Features}
\begin{enumerate}
	\item \PGFPlots\ 1.3 comes with user interface improvements. The technical distinction between ``behavior options'' and ``style options'' of older versions is no longer necessary (although still fully supported).

	\item \PGFPlots\ 1.3 has a new feature which allows to \emph{move axis labels tight to tick labels} automatically.

	Since this affects the spacing, it is not enabled be default.

	Use
\begin{codeexample}[code only]
\usepackage{pgfplots}
\pgfplotsset{compat=1.3}
\end{codeexample}
	in your preamble to benefit from the improved spacing\footnote{The |compat=1.3| setting is necessary for new features which move axis labels around like reversed axis. However, \PGFPlots\ will still work as it does in earlier versions, your documents will remain the same if you don't set it explicitly.}.
	Take a look at the next page for the precise definition of |compat|.

	\item \PGFPlots\ 1.3 now supports reversed axes. It is no longer necessary to use work--arounds with negative units.
\pgfkeys{/pdflinks/search key prefixes in/.add={/pgfplots/,}{}}

	Take a look at the |x dir=reverse| key.

	Existing work--arounds will still function properly. Use |\pgfplotsset{compat=1.3}| together with |x dir=reverse| to switch to the new version.
\end{enumerate}

\subsubsection{Old Features Which May Need Attention}
\begin{enumerate}
	\item The |scatter/classes| feature produces proper legends as of version 1.3. This may change the appearance of existing legends of plots with |scatter/classes|.

	\item Due to a small math bug in \PGF\ $2.00$, you \emph{can't} use the math expression `|-x^2|'. It is necessary to use `|0-x^2|' instead. The same holds for
		`|exp(-x^2)|'; use `|exp(0-x^2)|' instead.

		This will be fixed with \PGF\ $>2.00$.

	\item Starting with \PGFPlots\ $1.1$, |\tikzstyle| should \emph{no longer be used} to set \PGFPlots\ options.
	
	Although |\tikzstyle| is still supported for some older \PGFPlots\ options, you should replace any occurance of |\tikzstyle| with |\pgfplotsset{|\meta{style name}|/.style={|\meta{key-value-list}|}}| or the associated |/.append style| variant. See section~\ref{sec:styles} for more detail.
\end{enumerate}
I apologize for any inconvenience caused by these changes.

\begin{pgfplotskey}{compat=\mchoice{1.3,pre 1.3,default,newest} (initially default)}
	The preamble configuration 
\begin{codeexample}[code only]
\usepackage{pgfplots}
\pgfplotsset{compat=1.3}
\end{codeexample}
	allows to choose between backwards compatibility and most recent features.

	\PGFPlots\ version 1.3 comes with new features which may lead to different spacings compared to earlier versions:
	\begin{enumerate}
		\item Version 1.3 comes with the |xlabel near ticks| feature which places (in this case $x$) axis labels ``near'' the tick labels, taking the size of tick labels into account.
		
		Older versions used |xlabel absolute|, i.e.\ an absolute distance between the axis and axis labels (now matter how large or small tick labels are).

		The initial setting \emph{keeps backwards compatibility}.

		You are encouraged to use
\begin{codeexample}[code only]
\usepackage{pgfplots}
\pgfplotsset{compat=1.3}
\end{codeexample}
		in your preamble.
		
		\item Version 1.3 fixes a bug with |anchor=south west| which occurred mainly for reversed axes: the anchors where upside--down. This is fixed now.

		
		If you have written some work--around and want to keep it going, use

		|\pgfplotsset{compat/anchors=pre 1.3}|\pgfmanualpdflabel{/pgfplots/compat/anchors}{} 

		to disable the bug fix.
	\end{enumerate}

	Use |\pgfplotsset{compat=default}| to restore the factory settings.

	The setting |\pgfplotsset{compat=newest}| will always use features of the most recent version. This might result in small changes in the document's appearance.
\end{pgfplotskey}

\subsection{The Team}
\PGFPlots\ has been written mainly by Christian Feuers�nger with many improvements of Pascal Wolkotte and Nick Papior Andersen as a spare time project. We hope it is useful and provides valuable plots.

If you are interested in writing something but don't know how, consider reading the auxiliary manual \href{file:TeX-programming-notes.pdf}{TeX-programming-notes.pdf} which comes with \PGFPlots. It is far from complete, but maybe it is a good starting point (at least for more literature).

\subsection{Acknowledgements}
I thank God for all hours of enjoyed programming. I thank Pascal Wolkotte and Nick Papior Andersen for their programming efforts and contributions as part of the development team. I thank Stefan Tibus, who contributed the |plot shell| feature. I thank Tom Cashman for the contribution of the |reverse legend| feature. Special thanks go to Stefan Pinnow whose tests of \PGFPlots\ lead to numerous quality improvements. Furthermore, I thank Dr.~Schweitzer for many fruitful discussions and Dr.~Meine for his ideas and suggestions. Thanks as well to the many international contributors who provided feature requests or identified bugs or simply manual improvements!

Last but not least, I thank Till Tantau and Mark Wibrow for their excellent graphics (and more) package \PGF\ and \Tikz\ which is the base of \PGFPlots.

\include{pgfplots.install}%
\include{pgfplots.intro}%
\include{pgfplots.reference}%
\include{pgfplots.libs}%
\include{pgfplots.resources}%
\include{pgfplots.importexport}%
\include{pgfplots.basic.reference}%

\printindex

\bibliographystyle{abbrv} %gerapali} %gerabbrv} %gerunsrt.bst} %gerabbrv}% gerplain}
\nocite{pgfplotstable}
\nocite{programmingnotes}
\bibliography{pgfplots}
\end{document}
%
%%%%%%%%%%%%%%%%%%%%%%%%%%%%%%%%%%%%%%%%%%%%%%%%%%%%%%%%%%%%%%%%%%%%%%%%%%%%%
%
% Package pgfplots.sty documentation. 
%
% Copyright 2007/2008 by Christian Feuersaenger.
%
% This program is free software: you can redistribute it and/or modify
% it under the terms of the GNU General Public License as published by
% the Free Software Foundation, either version 3 of the License, or
% (at your option) any later version.
% 
% This program is distributed in the hope that it will be useful,
% but WITHOUT ANY WARRANTY; without even the implied warranty of
% MERCHANTABILITY or FITNESS FOR A PARTICULAR PURPOSE.  See the
% GNU General Public License for more details.
% 
% You should have received a copy of the GNU General Public License
% along with this program.  If not, see <http://www.gnu.org/licenses/>.
%
%
%%%%%%%%%%%%%%%%%%%%%%%%%%%%%%%%%%%%%%%%%%%%%%%%%%%%%%%%%%%%%%%%%%%%%%%%%%%%%
\input{pgfplots.preamble.tex}
%\RequirePackage[german,english,francais]{babel}

\pgfplotsmanualexternalexpensivetrue

%\usetikzlibrary{external}
%\tikzexternalize[prefix=figures/]{pgfplots}

\title{%
	Manual for Package \PGFPlots\\
	{\small Version \pgfplotsversion}\\
	{\small\href{http://sourceforge.net/projects/pgfplots}{http://sourceforge.net/projects/pgfplots}}}

%\includeonly{pgfplots.libs}


\begin{document}

\def\plotcoords{%
\addplot coordinates {
(5,8.312e-02)    (17,2.547e-02)   (49,7.407e-03)
(129,2.102e-03)  (321,5.874e-04)  (769,1.623e-04)
(1793,4.442e-05) (4097,1.207e-05) (9217,3.261e-06)
};

\addplot coordinates{
(7,8.472e-02)    (31,3.044e-02)    (111,1.022e-02)
(351,3.303e-03)  (1023,1.039e-03)  (2815,3.196e-04)
(7423,9.658e-05) (18943,2.873e-05) (47103,8.437e-06)
};

\addplot coordinates{
(9,7.881e-02)     (49,3.243e-02)    (209,1.232e-02)
(769,4.454e-03)   (2561,1.551e-03)  (7937,5.236e-04)
(23297,1.723e-04) (65537,5.545e-05) (178177,1.751e-05)
};

\addplot coordinates{
(11,6.887e-02)    (71,3.177e-02)     (351,1.341e-02)
(1471,5.334e-03)  (5503,2.027e-03)   (18943,7.415e-04)
(61183,2.628e-04) (187903,9.063e-05) (553983,3.053e-05)
};

\addplot coordinates{
(13,5.755e-02)     (97,2.925e-02)     (545,1.351e-02)
(2561,5.842e-03)   (10625,2.397e-03)  (40193,9.414e-04)
(141569,3.564e-04) (471041,1.308e-04) 
(1496065,4.670e-05)
};
}%
\maketitle
\begin{abstract}%
\PGFPlots\ draws high--quality function plots in normal or logarithmic scaling with a user-friendly interface directly in \TeX. The user supplies axis labels, legend entries and the plot coordinates for one or more plots and \PGFPlots\ applies axis scaling, computes any logarithms and axis ticks and draws the plots, supporting line plots, scatter plots, piecewise constant plots, bar plots, area plots, mesh-- and surface plots and some more. It is based on Till Tantau's package \PGF/\Tikz.
\end{abstract}
\tableofcontents
\section{Introduction}
This package provides tools to generate plots and labeled axes easily. It draws normal plots, logplots and semi-logplots, in two and three dimensions. Axis ticks, labels, legends (in case of multiple plots) can be added with key-value options. It can cycle through a set of predefined line/marker/color specifications. In summary, its purpose is to simplify the generation of high-quality function and/or data plots, and solving the problems of
\begin{itemize}
	\item consistency of document and font type and font size,
	\item direct use of \TeX\ math mode in axis descriptions,
	\item consistency of data and figures (no third party tool necessary),
	\item inter document consistency using preamble configurations and styles.
\end{itemize}
Although not necessary, separate |.pdf| or |.eps| graphics can be generated using the |external| library developed as part of \Tikz.

\PGFPlots\ is build completely on \Tikz/\PGF. Knowledge of \Tikz\ will simplify the work with \PGFPlots, although it is not required.

\section[About PGFPlots: Preliminaries]{About {\normalfont\PGFPlots}: Preliminaries}
This section contains information about upgrades, the team, the installation (in case you need to do it manually) and troubleshooting. You may skip it completely except for the upgrade remarks.

However, note that this library requires at least \PGF\ version $2.00$. At the time of this writing, many \TeX-distributions still contain the older \PGF\ version $1.18$, so it may be necessary to install a recent \PGF\ prior to using \PGFPlots.

\subsection{Components}
\PGFPlots\ comes with two components:
\begin{enumerate}
	\item the plotting component (which you are currently reading) and
	\item the \PGFPlotstable\ component which simplifies number formatting and postprocessing of numerical tables. It comes as a separate package and has its own manual \href{file:pgfplotstable.pdf}{pgfplotstable.pdf}.
\end{enumerate}

\subsection{Upgrade remarks}
This release provides a lot of improvements which can be found in all detail in \texttt{ChangeLog} for interested readers. However, some attention is useful with respect to the following changes.

\subsubsection{New Optional Features}
\begin{enumerate}
	\item \PGFPlots\ 1.3 comes with user interface improvements. The technical distinction between ``behavior options'' and ``style options'' of older versions is no longer necessary (although still fully supported).

	\item \PGFPlots\ 1.3 has a new feature which allows to \emph{move axis labels tight to tick labels} automatically.

	Since this affects the spacing, it is not enabled be default.

	Use
\begin{codeexample}[code only]
\usepackage{pgfplots}
\pgfplotsset{compat=1.3}
\end{codeexample}
	in your preamble to benefit from the improved spacing\footnote{The |compat=1.3| setting is necessary for new features which move axis labels around like reversed axis. However, \PGFPlots\ will still work as it does in earlier versions, your documents will remain the same if you don't set it explicitly.}.
	Take a look at the next page for the precise definition of |compat|.

	\item \PGFPlots\ 1.3 now supports reversed axes. It is no longer necessary to use work--arounds with negative units.
\pgfkeys{/pdflinks/search key prefixes in/.add={/pgfplots/,}{}}

	Take a look at the |x dir=reverse| key.

	Existing work--arounds will still function properly. Use |\pgfplotsset{compat=1.3}| together with |x dir=reverse| to switch to the new version.
\end{enumerate}

\subsubsection{Old Features Which May Need Attention}
\begin{enumerate}
	\item The |scatter/classes| feature produces proper legends as of version 1.3. This may change the appearance of existing legends of plots with |scatter/classes|.

	\item Due to a small math bug in \PGF\ $2.00$, you \emph{can't} use the math expression `|-x^2|'. It is necessary to use `|0-x^2|' instead. The same holds for
		`|exp(-x^2)|'; use `|exp(0-x^2)|' instead.

		This will be fixed with \PGF\ $>2.00$.

	\item Starting with \PGFPlots\ $1.1$, |\tikzstyle| should \emph{no longer be used} to set \PGFPlots\ options.
	
	Although |\tikzstyle| is still supported for some older \PGFPlots\ options, you should replace any occurance of |\tikzstyle| with |\pgfplotsset{|\meta{style name}|/.style={|\meta{key-value-list}|}}| or the associated |/.append style| variant. See section~\ref{sec:styles} for more detail.
\end{enumerate}
I apologize for any inconvenience caused by these changes.

\begin{pgfplotskey}{compat=\mchoice{1.3,pre 1.3,default,newest} (initially default)}
	The preamble configuration 
\begin{codeexample}[code only]
\usepackage{pgfplots}
\pgfplotsset{compat=1.3}
\end{codeexample}
	allows to choose between backwards compatibility and most recent features.

	\PGFPlots\ version 1.3 comes with new features which may lead to different spacings compared to earlier versions:
	\begin{enumerate}
		\item Version 1.3 comes with the |xlabel near ticks| feature which places (in this case $x$) axis labels ``near'' the tick labels, taking the size of tick labels into account.
		
		Older versions used |xlabel absolute|, i.e.\ an absolute distance between the axis and axis labels (now matter how large or small tick labels are).

		The initial setting \emph{keeps backwards compatibility}.

		You are encouraged to use
\begin{codeexample}[code only]
\usepackage{pgfplots}
\pgfplotsset{compat=1.3}
\end{codeexample}
		in your preamble.
		
		\item Version 1.3 fixes a bug with |anchor=south west| which occurred mainly for reversed axes: the anchors where upside--down. This is fixed now.

		
		If you have written some work--around and want to keep it going, use

		|\pgfplotsset{compat/anchors=pre 1.3}|\pgfmanualpdflabel{/pgfplots/compat/anchors}{} 

		to disable the bug fix.
	\end{enumerate}

	Use |\pgfplotsset{compat=default}| to restore the factory settings.

	The setting |\pgfplotsset{compat=newest}| will always use features of the most recent version. This might result in small changes in the document's appearance.
\end{pgfplotskey}

\subsection{The Team}
\PGFPlots\ has been written mainly by Christian Feuers�nger with many improvements of Pascal Wolkotte and Nick Papior Andersen as a spare time project. We hope it is useful and provides valuable plots.

If you are interested in writing something but don't know how, consider reading the auxiliary manual \href{file:TeX-programming-notes.pdf}{TeX-programming-notes.pdf} which comes with \PGFPlots. It is far from complete, but maybe it is a good starting point (at least for more literature).

\subsection{Acknowledgements}
I thank God for all hours of enjoyed programming. I thank Pascal Wolkotte and Nick Papior Andersen for their programming efforts and contributions as part of the development team. I thank Stefan Tibus, who contributed the |plot shell| feature. I thank Tom Cashman for the contribution of the |reverse legend| feature. Special thanks go to Stefan Pinnow whose tests of \PGFPlots\ lead to numerous quality improvements. Furthermore, I thank Dr.~Schweitzer for many fruitful discussions and Dr.~Meine for his ideas and suggestions. Thanks as well to the many international contributors who provided feature requests or identified bugs or simply manual improvements!

Last but not least, I thank Till Tantau and Mark Wibrow for their excellent graphics (and more) package \PGF\ and \Tikz\ which is the base of \PGFPlots.

\include{pgfplots.install}%
\include{pgfplots.intro}%
\include{pgfplots.reference}%
\include{pgfplots.libs}%
\include{pgfplots.resources}%
\include{pgfplots.importexport}%
\include{pgfplots.basic.reference}%

\printindex

\bibliographystyle{abbrv} %gerapali} %gerabbrv} %gerunsrt.bst} %gerabbrv}% gerplain}
\nocite{pgfplotstable}
\nocite{programmingnotes}
\bibliography{pgfplots}
\end{document}
%
%%%%%%%%%%%%%%%%%%%%%%%%%%%%%%%%%%%%%%%%%%%%%%%%%%%%%%%%%%%%%%%%%%%%%%%%%%%%%
%
% Package pgfplots.sty documentation. 
%
% Copyright 2007/2008 by Christian Feuersaenger.
%
% This program is free software: you can redistribute it and/or modify
% it under the terms of the GNU General Public License as published by
% the Free Software Foundation, either version 3 of the License, or
% (at your option) any later version.
% 
% This program is distributed in the hope that it will be useful,
% but WITHOUT ANY WARRANTY; without even the implied warranty of
% MERCHANTABILITY or FITNESS FOR A PARTICULAR PURPOSE.  See the
% GNU General Public License for more details.
% 
% You should have received a copy of the GNU General Public License
% along with this program.  If not, see <http://www.gnu.org/licenses/>.
%
%
%%%%%%%%%%%%%%%%%%%%%%%%%%%%%%%%%%%%%%%%%%%%%%%%%%%%%%%%%%%%%%%%%%%%%%%%%%%%%
\input{pgfplots.preamble.tex}
%\RequirePackage[german,english,francais]{babel}

\pgfplotsmanualexternalexpensivetrue

%\usetikzlibrary{external}
%\tikzexternalize[prefix=figures/]{pgfplots}

\title{%
	Manual for Package \PGFPlots\\
	{\small Version \pgfplotsversion}\\
	{\small\href{http://sourceforge.net/projects/pgfplots}{http://sourceforge.net/projects/pgfplots}}}

%\includeonly{pgfplots.libs}


\begin{document}

\def\plotcoords{%
\addplot coordinates {
(5,8.312e-02)    (17,2.547e-02)   (49,7.407e-03)
(129,2.102e-03)  (321,5.874e-04)  (769,1.623e-04)
(1793,4.442e-05) (4097,1.207e-05) (9217,3.261e-06)
};

\addplot coordinates{
(7,8.472e-02)    (31,3.044e-02)    (111,1.022e-02)
(351,3.303e-03)  (1023,1.039e-03)  (2815,3.196e-04)
(7423,9.658e-05) (18943,2.873e-05) (47103,8.437e-06)
};

\addplot coordinates{
(9,7.881e-02)     (49,3.243e-02)    (209,1.232e-02)
(769,4.454e-03)   (2561,1.551e-03)  (7937,5.236e-04)
(23297,1.723e-04) (65537,5.545e-05) (178177,1.751e-05)
};

\addplot coordinates{
(11,6.887e-02)    (71,3.177e-02)     (351,1.341e-02)
(1471,5.334e-03)  (5503,2.027e-03)   (18943,7.415e-04)
(61183,2.628e-04) (187903,9.063e-05) (553983,3.053e-05)
};

\addplot coordinates{
(13,5.755e-02)     (97,2.925e-02)     (545,1.351e-02)
(2561,5.842e-03)   (10625,2.397e-03)  (40193,9.414e-04)
(141569,3.564e-04) (471041,1.308e-04) 
(1496065,4.670e-05)
};
}%
\maketitle
\begin{abstract}%
\PGFPlots\ draws high--quality function plots in normal or logarithmic scaling with a user-friendly interface directly in \TeX. The user supplies axis labels, legend entries and the plot coordinates for one or more plots and \PGFPlots\ applies axis scaling, computes any logarithms and axis ticks and draws the plots, supporting line plots, scatter plots, piecewise constant plots, bar plots, area plots, mesh-- and surface plots and some more. It is based on Till Tantau's package \PGF/\Tikz.
\end{abstract}
\tableofcontents
\section{Introduction}
This package provides tools to generate plots and labeled axes easily. It draws normal plots, logplots and semi-logplots, in two and three dimensions. Axis ticks, labels, legends (in case of multiple plots) can be added with key-value options. It can cycle through a set of predefined line/marker/color specifications. In summary, its purpose is to simplify the generation of high-quality function and/or data plots, and solving the problems of
\begin{itemize}
	\item consistency of document and font type and font size,
	\item direct use of \TeX\ math mode in axis descriptions,
	\item consistency of data and figures (no third party tool necessary),
	\item inter document consistency using preamble configurations and styles.
\end{itemize}
Although not necessary, separate |.pdf| or |.eps| graphics can be generated using the |external| library developed as part of \Tikz.

\PGFPlots\ is build completely on \Tikz/\PGF. Knowledge of \Tikz\ will simplify the work with \PGFPlots, although it is not required.

\section[About PGFPlots: Preliminaries]{About {\normalfont\PGFPlots}: Preliminaries}
This section contains information about upgrades, the team, the installation (in case you need to do it manually) and troubleshooting. You may skip it completely except for the upgrade remarks.

However, note that this library requires at least \PGF\ version $2.00$. At the time of this writing, many \TeX-distributions still contain the older \PGF\ version $1.18$, so it may be necessary to install a recent \PGF\ prior to using \PGFPlots.

\subsection{Components}
\PGFPlots\ comes with two components:
\begin{enumerate}
	\item the plotting component (which you are currently reading) and
	\item the \PGFPlotstable\ component which simplifies number formatting and postprocessing of numerical tables. It comes as a separate package and has its own manual \href{file:pgfplotstable.pdf}{pgfplotstable.pdf}.
\end{enumerate}

\subsection{Upgrade remarks}
This release provides a lot of improvements which can be found in all detail in \texttt{ChangeLog} for interested readers. However, some attention is useful with respect to the following changes.

\subsubsection{New Optional Features}
\begin{enumerate}
	\item \PGFPlots\ 1.3 comes with user interface improvements. The technical distinction between ``behavior options'' and ``style options'' of older versions is no longer necessary (although still fully supported).

	\item \PGFPlots\ 1.3 has a new feature which allows to \emph{move axis labels tight to tick labels} automatically.

	Since this affects the spacing, it is not enabled be default.

	Use
\begin{codeexample}[code only]
\usepackage{pgfplots}
\pgfplotsset{compat=1.3}
\end{codeexample}
	in your preamble to benefit from the improved spacing\footnote{The |compat=1.3| setting is necessary for new features which move axis labels around like reversed axis. However, \PGFPlots\ will still work as it does in earlier versions, your documents will remain the same if you don't set it explicitly.}.
	Take a look at the next page for the precise definition of |compat|.

	\item \PGFPlots\ 1.3 now supports reversed axes. It is no longer necessary to use work--arounds with negative units.
\pgfkeys{/pdflinks/search key prefixes in/.add={/pgfplots/,}{}}

	Take a look at the |x dir=reverse| key.

	Existing work--arounds will still function properly. Use |\pgfplotsset{compat=1.3}| together with |x dir=reverse| to switch to the new version.
\end{enumerate}

\subsubsection{Old Features Which May Need Attention}
\begin{enumerate}
	\item The |scatter/classes| feature produces proper legends as of version 1.3. This may change the appearance of existing legends of plots with |scatter/classes|.

	\item Due to a small math bug in \PGF\ $2.00$, you \emph{can't} use the math expression `|-x^2|'. It is necessary to use `|0-x^2|' instead. The same holds for
		`|exp(-x^2)|'; use `|exp(0-x^2)|' instead.

		This will be fixed with \PGF\ $>2.00$.

	\item Starting with \PGFPlots\ $1.1$, |\tikzstyle| should \emph{no longer be used} to set \PGFPlots\ options.
	
	Although |\tikzstyle| is still supported for some older \PGFPlots\ options, you should replace any occurance of |\tikzstyle| with |\pgfplotsset{|\meta{style name}|/.style={|\meta{key-value-list}|}}| or the associated |/.append style| variant. See section~\ref{sec:styles} for more detail.
\end{enumerate}
I apologize for any inconvenience caused by these changes.

\begin{pgfplotskey}{compat=\mchoice{1.3,pre 1.3,default,newest} (initially default)}
	The preamble configuration 
\begin{codeexample}[code only]
\usepackage{pgfplots}
\pgfplotsset{compat=1.3}
\end{codeexample}
	allows to choose between backwards compatibility and most recent features.

	\PGFPlots\ version 1.3 comes with new features which may lead to different spacings compared to earlier versions:
	\begin{enumerate}
		\item Version 1.3 comes with the |xlabel near ticks| feature which places (in this case $x$) axis labels ``near'' the tick labels, taking the size of tick labels into account.
		
		Older versions used |xlabel absolute|, i.e.\ an absolute distance between the axis and axis labels (now matter how large or small tick labels are).

		The initial setting \emph{keeps backwards compatibility}.

		You are encouraged to use
\begin{codeexample}[code only]
\usepackage{pgfplots}
\pgfplotsset{compat=1.3}
\end{codeexample}
		in your preamble.
		
		\item Version 1.3 fixes a bug with |anchor=south west| which occurred mainly for reversed axes: the anchors where upside--down. This is fixed now.

		
		If you have written some work--around and want to keep it going, use

		|\pgfplotsset{compat/anchors=pre 1.3}|\pgfmanualpdflabel{/pgfplots/compat/anchors}{} 

		to disable the bug fix.
	\end{enumerate}

	Use |\pgfplotsset{compat=default}| to restore the factory settings.

	The setting |\pgfplotsset{compat=newest}| will always use features of the most recent version. This might result in small changes in the document's appearance.
\end{pgfplotskey}

\subsection{The Team}
\PGFPlots\ has been written mainly by Christian Feuers�nger with many improvements of Pascal Wolkotte and Nick Papior Andersen as a spare time project. We hope it is useful and provides valuable plots.

If you are interested in writing something but don't know how, consider reading the auxiliary manual \href{file:TeX-programming-notes.pdf}{TeX-programming-notes.pdf} which comes with \PGFPlots. It is far from complete, but maybe it is a good starting point (at least for more literature).

\subsection{Acknowledgements}
I thank God for all hours of enjoyed programming. I thank Pascal Wolkotte and Nick Papior Andersen for their programming efforts and contributions as part of the development team. I thank Stefan Tibus, who contributed the |plot shell| feature. I thank Tom Cashman for the contribution of the |reverse legend| feature. Special thanks go to Stefan Pinnow whose tests of \PGFPlots\ lead to numerous quality improvements. Furthermore, I thank Dr.~Schweitzer for many fruitful discussions and Dr.~Meine for his ideas and suggestions. Thanks as well to the many international contributors who provided feature requests or identified bugs or simply manual improvements!

Last but not least, I thank Till Tantau and Mark Wibrow for their excellent graphics (and more) package \PGF\ and \Tikz\ which is the base of \PGFPlots.

\include{pgfplots.install}%
\include{pgfplots.intro}%
\include{pgfplots.reference}%
\include{pgfplots.libs}%
\include{pgfplots.resources}%
\include{pgfplots.importexport}%
\include{pgfplots.basic.reference}%

\printindex

\bibliographystyle{abbrv} %gerapali} %gerabbrv} %gerunsrt.bst} %gerabbrv}% gerplain}
\nocite{pgfplotstable}
\nocite{programmingnotes}
\bibliography{pgfplots}
\end{document}
%
%%%%%%%%%%%%%%%%%%%%%%%%%%%%%%%%%%%%%%%%%%%%%%%%%%%%%%%%%%%%%%%%%%%%%%%%%%%%%
%
% Package pgfplots.sty documentation. 
%
% Copyright 2007/2008 by Christian Feuersaenger.
%
% This program is free software: you can redistribute it and/or modify
% it under the terms of the GNU General Public License as published by
% the Free Software Foundation, either version 3 of the License, or
% (at your option) any later version.
% 
% This program is distributed in the hope that it will be useful,
% but WITHOUT ANY WARRANTY; without even the implied warranty of
% MERCHANTABILITY or FITNESS FOR A PARTICULAR PURPOSE.  See the
% GNU General Public License for more details.
% 
% You should have received a copy of the GNU General Public License
% along with this program.  If not, see <http://www.gnu.org/licenses/>.
%
%
%%%%%%%%%%%%%%%%%%%%%%%%%%%%%%%%%%%%%%%%%%%%%%%%%%%%%%%%%%%%%%%%%%%%%%%%%%%%%
\input{pgfplots.preamble.tex}
%\RequirePackage[german,english,francais]{babel}

\pgfplotsmanualexternalexpensivetrue

%\usetikzlibrary{external}
%\tikzexternalize[prefix=figures/]{pgfplots}

\title{%
	Manual for Package \PGFPlots\\
	{\small Version \pgfplotsversion}\\
	{\small\href{http://sourceforge.net/projects/pgfplots}{http://sourceforge.net/projects/pgfplots}}}

%\includeonly{pgfplots.libs}


\begin{document}

\def\plotcoords{%
\addplot coordinates {
(5,8.312e-02)    (17,2.547e-02)   (49,7.407e-03)
(129,2.102e-03)  (321,5.874e-04)  (769,1.623e-04)
(1793,4.442e-05) (4097,1.207e-05) (9217,3.261e-06)
};

\addplot coordinates{
(7,8.472e-02)    (31,3.044e-02)    (111,1.022e-02)
(351,3.303e-03)  (1023,1.039e-03)  (2815,3.196e-04)
(7423,9.658e-05) (18943,2.873e-05) (47103,8.437e-06)
};

\addplot coordinates{
(9,7.881e-02)     (49,3.243e-02)    (209,1.232e-02)
(769,4.454e-03)   (2561,1.551e-03)  (7937,5.236e-04)
(23297,1.723e-04) (65537,5.545e-05) (178177,1.751e-05)
};

\addplot coordinates{
(11,6.887e-02)    (71,3.177e-02)     (351,1.341e-02)
(1471,5.334e-03)  (5503,2.027e-03)   (18943,7.415e-04)
(61183,2.628e-04) (187903,9.063e-05) (553983,3.053e-05)
};

\addplot coordinates{
(13,5.755e-02)     (97,2.925e-02)     (545,1.351e-02)
(2561,5.842e-03)   (10625,2.397e-03)  (40193,9.414e-04)
(141569,3.564e-04) (471041,1.308e-04) 
(1496065,4.670e-05)
};
}%
\maketitle
\begin{abstract}%
\PGFPlots\ draws high--quality function plots in normal or logarithmic scaling with a user-friendly interface directly in \TeX. The user supplies axis labels, legend entries and the plot coordinates for one or more plots and \PGFPlots\ applies axis scaling, computes any logarithms and axis ticks and draws the plots, supporting line plots, scatter plots, piecewise constant plots, bar plots, area plots, mesh-- and surface plots and some more. It is based on Till Tantau's package \PGF/\Tikz.
\end{abstract}
\tableofcontents
\section{Introduction}
This package provides tools to generate plots and labeled axes easily. It draws normal plots, logplots and semi-logplots, in two and three dimensions. Axis ticks, labels, legends (in case of multiple plots) can be added with key-value options. It can cycle through a set of predefined line/marker/color specifications. In summary, its purpose is to simplify the generation of high-quality function and/or data plots, and solving the problems of
\begin{itemize}
	\item consistency of document and font type and font size,
	\item direct use of \TeX\ math mode in axis descriptions,
	\item consistency of data and figures (no third party tool necessary),
	\item inter document consistency using preamble configurations and styles.
\end{itemize}
Although not necessary, separate |.pdf| or |.eps| graphics can be generated using the |external| library developed as part of \Tikz.

\PGFPlots\ is build completely on \Tikz/\PGF. Knowledge of \Tikz\ will simplify the work with \PGFPlots, although it is not required.

\section[About PGFPlots: Preliminaries]{About {\normalfont\PGFPlots}: Preliminaries}
This section contains information about upgrades, the team, the installation (in case you need to do it manually) and troubleshooting. You may skip it completely except for the upgrade remarks.

However, note that this library requires at least \PGF\ version $2.00$. At the time of this writing, many \TeX-distributions still contain the older \PGF\ version $1.18$, so it may be necessary to install a recent \PGF\ prior to using \PGFPlots.

\subsection{Components}
\PGFPlots\ comes with two components:
\begin{enumerate}
	\item the plotting component (which you are currently reading) and
	\item the \PGFPlotstable\ component which simplifies number formatting and postprocessing of numerical tables. It comes as a separate package and has its own manual \href{file:pgfplotstable.pdf}{pgfplotstable.pdf}.
\end{enumerate}

\subsection{Upgrade remarks}
This release provides a lot of improvements which can be found in all detail in \texttt{ChangeLog} for interested readers. However, some attention is useful with respect to the following changes.

\subsubsection{New Optional Features}
\begin{enumerate}
	\item \PGFPlots\ 1.3 comes with user interface improvements. The technical distinction between ``behavior options'' and ``style options'' of older versions is no longer necessary (although still fully supported).

	\item \PGFPlots\ 1.3 has a new feature which allows to \emph{move axis labels tight to tick labels} automatically.

	Since this affects the spacing, it is not enabled be default.

	Use
\begin{codeexample}[code only]
\usepackage{pgfplots}
\pgfplotsset{compat=1.3}
\end{codeexample}
	in your preamble to benefit from the improved spacing\footnote{The |compat=1.3| setting is necessary for new features which move axis labels around like reversed axis. However, \PGFPlots\ will still work as it does in earlier versions, your documents will remain the same if you don't set it explicitly.}.
	Take a look at the next page for the precise definition of |compat|.

	\item \PGFPlots\ 1.3 now supports reversed axes. It is no longer necessary to use work--arounds with negative units.
\pgfkeys{/pdflinks/search key prefixes in/.add={/pgfplots/,}{}}

	Take a look at the |x dir=reverse| key.

	Existing work--arounds will still function properly. Use |\pgfplotsset{compat=1.3}| together with |x dir=reverse| to switch to the new version.
\end{enumerate}

\subsubsection{Old Features Which May Need Attention}
\begin{enumerate}
	\item The |scatter/classes| feature produces proper legends as of version 1.3. This may change the appearance of existing legends of plots with |scatter/classes|.

	\item Due to a small math bug in \PGF\ $2.00$, you \emph{can't} use the math expression `|-x^2|'. It is necessary to use `|0-x^2|' instead. The same holds for
		`|exp(-x^2)|'; use `|exp(0-x^2)|' instead.

		This will be fixed with \PGF\ $>2.00$.

	\item Starting with \PGFPlots\ $1.1$, |\tikzstyle| should \emph{no longer be used} to set \PGFPlots\ options.
	
	Although |\tikzstyle| is still supported for some older \PGFPlots\ options, you should replace any occurance of |\tikzstyle| with |\pgfplotsset{|\meta{style name}|/.style={|\meta{key-value-list}|}}| or the associated |/.append style| variant. See section~\ref{sec:styles} for more detail.
\end{enumerate}
I apologize for any inconvenience caused by these changes.

\begin{pgfplotskey}{compat=\mchoice{1.3,pre 1.3,default,newest} (initially default)}
	The preamble configuration 
\begin{codeexample}[code only]
\usepackage{pgfplots}
\pgfplotsset{compat=1.3}
\end{codeexample}
	allows to choose between backwards compatibility and most recent features.

	\PGFPlots\ version 1.3 comes with new features which may lead to different spacings compared to earlier versions:
	\begin{enumerate}
		\item Version 1.3 comes with the |xlabel near ticks| feature which places (in this case $x$) axis labels ``near'' the tick labels, taking the size of tick labels into account.
		
		Older versions used |xlabel absolute|, i.e.\ an absolute distance between the axis and axis labels (now matter how large or small tick labels are).

		The initial setting \emph{keeps backwards compatibility}.

		You are encouraged to use
\begin{codeexample}[code only]
\usepackage{pgfplots}
\pgfplotsset{compat=1.3}
\end{codeexample}
		in your preamble.
		
		\item Version 1.3 fixes a bug with |anchor=south west| which occurred mainly for reversed axes: the anchors where upside--down. This is fixed now.

		
		If you have written some work--around and want to keep it going, use

		|\pgfplotsset{compat/anchors=pre 1.3}|\pgfmanualpdflabel{/pgfplots/compat/anchors}{} 

		to disable the bug fix.
	\end{enumerate}

	Use |\pgfplotsset{compat=default}| to restore the factory settings.

	The setting |\pgfplotsset{compat=newest}| will always use features of the most recent version. This might result in small changes in the document's appearance.
\end{pgfplotskey}

\subsection{The Team}
\PGFPlots\ has been written mainly by Christian Feuers�nger with many improvements of Pascal Wolkotte and Nick Papior Andersen as a spare time project. We hope it is useful and provides valuable plots.

If you are interested in writing something but don't know how, consider reading the auxiliary manual \href{file:TeX-programming-notes.pdf}{TeX-programming-notes.pdf} which comes with \PGFPlots. It is far from complete, but maybe it is a good starting point (at least for more literature).

\subsection{Acknowledgements}
I thank God for all hours of enjoyed programming. I thank Pascal Wolkotte and Nick Papior Andersen for their programming efforts and contributions as part of the development team. I thank Stefan Tibus, who contributed the |plot shell| feature. I thank Tom Cashman for the contribution of the |reverse legend| feature. Special thanks go to Stefan Pinnow whose tests of \PGFPlots\ lead to numerous quality improvements. Furthermore, I thank Dr.~Schweitzer for many fruitful discussions and Dr.~Meine for his ideas and suggestions. Thanks as well to the many international contributors who provided feature requests or identified bugs or simply manual improvements!

Last but not least, I thank Till Tantau and Mark Wibrow for their excellent graphics (and more) package \PGF\ and \Tikz\ which is the base of \PGFPlots.

\include{pgfplots.install}%
\include{pgfplots.intro}%
\include{pgfplots.reference}%
\include{pgfplots.libs}%
\include{pgfplots.resources}%
\include{pgfplots.importexport}%
\include{pgfplots.basic.reference}%

\printindex

\bibliographystyle{abbrv} %gerapali} %gerabbrv} %gerunsrt.bst} %gerabbrv}% gerplain}
\nocite{pgfplotstable}
\nocite{programmingnotes}
\bibliography{pgfplots}
\end{document}
%
%%%%%%%%%%%%%%%%%%%%%%%%%%%%%%%%%%%%%%%%%%%%%%%%%%%%%%%%%%%%%%%%%%%%%%%%%%%%%
%
% Package pgfplots.sty documentation. 
%
% Copyright 2007/2008 by Christian Feuersaenger.
%
% This program is free software: you can redistribute it and/or modify
% it under the terms of the GNU General Public License as published by
% the Free Software Foundation, either version 3 of the License, or
% (at your option) any later version.
% 
% This program is distributed in the hope that it will be useful,
% but WITHOUT ANY WARRANTY; without even the implied warranty of
% MERCHANTABILITY or FITNESS FOR A PARTICULAR PURPOSE.  See the
% GNU General Public License for more details.
% 
% You should have received a copy of the GNU General Public License
% along with this program.  If not, see <http://www.gnu.org/licenses/>.
%
%
%%%%%%%%%%%%%%%%%%%%%%%%%%%%%%%%%%%%%%%%%%%%%%%%%%%%%%%%%%%%%%%%%%%%%%%%%%%%%
\input{pgfplots.preamble.tex}
%\RequirePackage[german,english,francais]{babel}

\pgfplotsmanualexternalexpensivetrue

%\usetikzlibrary{external}
%\tikzexternalize[prefix=figures/]{pgfplots}

\title{%
	Manual for Package \PGFPlots\\
	{\small Version \pgfplotsversion}\\
	{\small\href{http://sourceforge.net/projects/pgfplots}{http://sourceforge.net/projects/pgfplots}}}

%\includeonly{pgfplots.libs}


\begin{document}

\def\plotcoords{%
\addplot coordinates {
(5,8.312e-02)    (17,2.547e-02)   (49,7.407e-03)
(129,2.102e-03)  (321,5.874e-04)  (769,1.623e-04)
(1793,4.442e-05) (4097,1.207e-05) (9217,3.261e-06)
};

\addplot coordinates{
(7,8.472e-02)    (31,3.044e-02)    (111,1.022e-02)
(351,3.303e-03)  (1023,1.039e-03)  (2815,3.196e-04)
(7423,9.658e-05) (18943,2.873e-05) (47103,8.437e-06)
};

\addplot coordinates{
(9,7.881e-02)     (49,3.243e-02)    (209,1.232e-02)
(769,4.454e-03)   (2561,1.551e-03)  (7937,5.236e-04)
(23297,1.723e-04) (65537,5.545e-05) (178177,1.751e-05)
};

\addplot coordinates{
(11,6.887e-02)    (71,3.177e-02)     (351,1.341e-02)
(1471,5.334e-03)  (5503,2.027e-03)   (18943,7.415e-04)
(61183,2.628e-04) (187903,9.063e-05) (553983,3.053e-05)
};

\addplot coordinates{
(13,5.755e-02)     (97,2.925e-02)     (545,1.351e-02)
(2561,5.842e-03)   (10625,2.397e-03)  (40193,9.414e-04)
(141569,3.564e-04) (471041,1.308e-04) 
(1496065,4.670e-05)
};
}%
\maketitle
\begin{abstract}%
\PGFPlots\ draws high--quality function plots in normal or logarithmic scaling with a user-friendly interface directly in \TeX. The user supplies axis labels, legend entries and the plot coordinates for one or more plots and \PGFPlots\ applies axis scaling, computes any logarithms and axis ticks and draws the plots, supporting line plots, scatter plots, piecewise constant plots, bar plots, area plots, mesh-- and surface plots and some more. It is based on Till Tantau's package \PGF/\Tikz.
\end{abstract}
\tableofcontents
\section{Introduction}
This package provides tools to generate plots and labeled axes easily. It draws normal plots, logplots and semi-logplots, in two and three dimensions. Axis ticks, labels, legends (in case of multiple plots) can be added with key-value options. It can cycle through a set of predefined line/marker/color specifications. In summary, its purpose is to simplify the generation of high-quality function and/or data plots, and solving the problems of
\begin{itemize}
	\item consistency of document and font type and font size,
	\item direct use of \TeX\ math mode in axis descriptions,
	\item consistency of data and figures (no third party tool necessary),
	\item inter document consistency using preamble configurations and styles.
\end{itemize}
Although not necessary, separate |.pdf| or |.eps| graphics can be generated using the |external| library developed as part of \Tikz.

\PGFPlots\ is build completely on \Tikz/\PGF. Knowledge of \Tikz\ will simplify the work with \PGFPlots, although it is not required.

\section[About PGFPlots: Preliminaries]{About {\normalfont\PGFPlots}: Preliminaries}
This section contains information about upgrades, the team, the installation (in case you need to do it manually) and troubleshooting. You may skip it completely except for the upgrade remarks.

However, note that this library requires at least \PGF\ version $2.00$. At the time of this writing, many \TeX-distributions still contain the older \PGF\ version $1.18$, so it may be necessary to install a recent \PGF\ prior to using \PGFPlots.

\subsection{Components}
\PGFPlots\ comes with two components:
\begin{enumerate}
	\item the plotting component (which you are currently reading) and
	\item the \PGFPlotstable\ component which simplifies number formatting and postprocessing of numerical tables. It comes as a separate package and has its own manual \href{file:pgfplotstable.pdf}{pgfplotstable.pdf}.
\end{enumerate}

\subsection{Upgrade remarks}
This release provides a lot of improvements which can be found in all detail in \texttt{ChangeLog} for interested readers. However, some attention is useful with respect to the following changes.

\subsubsection{New Optional Features}
\begin{enumerate}
	\item \PGFPlots\ 1.3 comes with user interface improvements. The technical distinction between ``behavior options'' and ``style options'' of older versions is no longer necessary (although still fully supported).

	\item \PGFPlots\ 1.3 has a new feature which allows to \emph{move axis labels tight to tick labels} automatically.

	Since this affects the spacing, it is not enabled be default.

	Use
\begin{codeexample}[code only]
\usepackage{pgfplots}
\pgfplotsset{compat=1.3}
\end{codeexample}
	in your preamble to benefit from the improved spacing\footnote{The |compat=1.3| setting is necessary for new features which move axis labels around like reversed axis. However, \PGFPlots\ will still work as it does in earlier versions, your documents will remain the same if you don't set it explicitly.}.
	Take a look at the next page for the precise definition of |compat|.

	\item \PGFPlots\ 1.3 now supports reversed axes. It is no longer necessary to use work--arounds with negative units.
\pgfkeys{/pdflinks/search key prefixes in/.add={/pgfplots/,}{}}

	Take a look at the |x dir=reverse| key.

	Existing work--arounds will still function properly. Use |\pgfplotsset{compat=1.3}| together with |x dir=reverse| to switch to the new version.
\end{enumerate}

\subsubsection{Old Features Which May Need Attention}
\begin{enumerate}
	\item The |scatter/classes| feature produces proper legends as of version 1.3. This may change the appearance of existing legends of plots with |scatter/classes|.

	\item Due to a small math bug in \PGF\ $2.00$, you \emph{can't} use the math expression `|-x^2|'. It is necessary to use `|0-x^2|' instead. The same holds for
		`|exp(-x^2)|'; use `|exp(0-x^2)|' instead.

		This will be fixed with \PGF\ $>2.00$.

	\item Starting with \PGFPlots\ $1.1$, |\tikzstyle| should \emph{no longer be used} to set \PGFPlots\ options.
	
	Although |\tikzstyle| is still supported for some older \PGFPlots\ options, you should replace any occurance of |\tikzstyle| with |\pgfplotsset{|\meta{style name}|/.style={|\meta{key-value-list}|}}| or the associated |/.append style| variant. See section~\ref{sec:styles} for more detail.
\end{enumerate}
I apologize for any inconvenience caused by these changes.

\begin{pgfplotskey}{compat=\mchoice{1.3,pre 1.3,default,newest} (initially default)}
	The preamble configuration 
\begin{codeexample}[code only]
\usepackage{pgfplots}
\pgfplotsset{compat=1.3}
\end{codeexample}
	allows to choose between backwards compatibility and most recent features.

	\PGFPlots\ version 1.3 comes with new features which may lead to different spacings compared to earlier versions:
	\begin{enumerate}
		\item Version 1.3 comes with the |xlabel near ticks| feature which places (in this case $x$) axis labels ``near'' the tick labels, taking the size of tick labels into account.
		
		Older versions used |xlabel absolute|, i.e.\ an absolute distance between the axis and axis labels (now matter how large or small tick labels are).

		The initial setting \emph{keeps backwards compatibility}.

		You are encouraged to use
\begin{codeexample}[code only]
\usepackage{pgfplots}
\pgfplotsset{compat=1.3}
\end{codeexample}
		in your preamble.
		
		\item Version 1.3 fixes a bug with |anchor=south west| which occurred mainly for reversed axes: the anchors where upside--down. This is fixed now.

		
		If you have written some work--around and want to keep it going, use

		|\pgfplotsset{compat/anchors=pre 1.3}|\pgfmanualpdflabel{/pgfplots/compat/anchors}{} 

		to disable the bug fix.
	\end{enumerate}

	Use |\pgfplotsset{compat=default}| to restore the factory settings.

	The setting |\pgfplotsset{compat=newest}| will always use features of the most recent version. This might result in small changes in the document's appearance.
\end{pgfplotskey}

\subsection{The Team}
\PGFPlots\ has been written mainly by Christian Feuers�nger with many improvements of Pascal Wolkotte and Nick Papior Andersen as a spare time project. We hope it is useful and provides valuable plots.

If you are interested in writing something but don't know how, consider reading the auxiliary manual \href{file:TeX-programming-notes.pdf}{TeX-programming-notes.pdf} which comes with \PGFPlots. It is far from complete, but maybe it is a good starting point (at least for more literature).

\subsection{Acknowledgements}
I thank God for all hours of enjoyed programming. I thank Pascal Wolkotte and Nick Papior Andersen for their programming efforts and contributions as part of the development team. I thank Stefan Tibus, who contributed the |plot shell| feature. I thank Tom Cashman for the contribution of the |reverse legend| feature. Special thanks go to Stefan Pinnow whose tests of \PGFPlots\ lead to numerous quality improvements. Furthermore, I thank Dr.~Schweitzer for many fruitful discussions and Dr.~Meine for his ideas and suggestions. Thanks as well to the many international contributors who provided feature requests or identified bugs or simply manual improvements!

Last but not least, I thank Till Tantau and Mark Wibrow for their excellent graphics (and more) package \PGF\ and \Tikz\ which is the base of \PGFPlots.

\include{pgfplots.install}%
\include{pgfplots.intro}%
\include{pgfplots.reference}%
\include{pgfplots.libs}%
\include{pgfplots.resources}%
\include{pgfplots.importexport}%
\include{pgfplots.basic.reference}%

\printindex

\bibliographystyle{abbrv} %gerapali} %gerabbrv} %gerunsrt.bst} %gerabbrv}% gerplain}
\nocite{pgfplotstable}
\nocite{programmingnotes}
\bibliography{pgfplots}
\end{document}
%
%%%%%%%%%%%%%%%%%%%%%%%%%%%%%%%%%%%%%%%%%%%%%%%%%%%%%%%%%%%%%%%%%%%%%%%%%%%%%
%
% Package pgfplots.sty documentation. 
%
% Copyright 2007/2008 by Christian Feuersaenger.
%
% This program is free software: you can redistribute it and/or modify
% it under the terms of the GNU General Public License as published by
% the Free Software Foundation, either version 3 of the License, or
% (at your option) any later version.
% 
% This program is distributed in the hope that it will be useful,
% but WITHOUT ANY WARRANTY; without even the implied warranty of
% MERCHANTABILITY or FITNESS FOR A PARTICULAR PURPOSE.  See the
% GNU General Public License for more details.
% 
% You should have received a copy of the GNU General Public License
% along with this program.  If not, see <http://www.gnu.org/licenses/>.
%
%
%%%%%%%%%%%%%%%%%%%%%%%%%%%%%%%%%%%%%%%%%%%%%%%%%%%%%%%%%%%%%%%%%%%%%%%%%%%%%
\input{pgfplots.preamble.tex}
%\RequirePackage[german,english,francais]{babel}

\pgfplotsmanualexternalexpensivetrue

%\usetikzlibrary{external}
%\tikzexternalize[prefix=figures/]{pgfplots}

\title{%
	Manual for Package \PGFPlots\\
	{\small Version \pgfplotsversion}\\
	{\small\href{http://sourceforge.net/projects/pgfplots}{http://sourceforge.net/projects/pgfplots}}}

%\includeonly{pgfplots.libs}


\begin{document}

\def\plotcoords{%
\addplot coordinates {
(5,8.312e-02)    (17,2.547e-02)   (49,7.407e-03)
(129,2.102e-03)  (321,5.874e-04)  (769,1.623e-04)
(1793,4.442e-05) (4097,1.207e-05) (9217,3.261e-06)
};

\addplot coordinates{
(7,8.472e-02)    (31,3.044e-02)    (111,1.022e-02)
(351,3.303e-03)  (1023,1.039e-03)  (2815,3.196e-04)
(7423,9.658e-05) (18943,2.873e-05) (47103,8.437e-06)
};

\addplot coordinates{
(9,7.881e-02)     (49,3.243e-02)    (209,1.232e-02)
(769,4.454e-03)   (2561,1.551e-03)  (7937,5.236e-04)
(23297,1.723e-04) (65537,5.545e-05) (178177,1.751e-05)
};

\addplot coordinates{
(11,6.887e-02)    (71,3.177e-02)     (351,1.341e-02)
(1471,5.334e-03)  (5503,2.027e-03)   (18943,7.415e-04)
(61183,2.628e-04) (187903,9.063e-05) (553983,3.053e-05)
};

\addplot coordinates{
(13,5.755e-02)     (97,2.925e-02)     (545,1.351e-02)
(2561,5.842e-03)   (10625,2.397e-03)  (40193,9.414e-04)
(141569,3.564e-04) (471041,1.308e-04) 
(1496065,4.670e-05)
};
}%
\maketitle
\begin{abstract}%
\PGFPlots\ draws high--quality function plots in normal or logarithmic scaling with a user-friendly interface directly in \TeX. The user supplies axis labels, legend entries and the plot coordinates for one or more plots and \PGFPlots\ applies axis scaling, computes any logarithms and axis ticks and draws the plots, supporting line plots, scatter plots, piecewise constant plots, bar plots, area plots, mesh-- and surface plots and some more. It is based on Till Tantau's package \PGF/\Tikz.
\end{abstract}
\tableofcontents
\section{Introduction}
This package provides tools to generate plots and labeled axes easily. It draws normal plots, logplots and semi-logplots, in two and three dimensions. Axis ticks, labels, legends (in case of multiple plots) can be added with key-value options. It can cycle through a set of predefined line/marker/color specifications. In summary, its purpose is to simplify the generation of high-quality function and/or data plots, and solving the problems of
\begin{itemize}
	\item consistency of document and font type and font size,
	\item direct use of \TeX\ math mode in axis descriptions,
	\item consistency of data and figures (no third party tool necessary),
	\item inter document consistency using preamble configurations and styles.
\end{itemize}
Although not necessary, separate |.pdf| or |.eps| graphics can be generated using the |external| library developed as part of \Tikz.

\PGFPlots\ is build completely on \Tikz/\PGF. Knowledge of \Tikz\ will simplify the work with \PGFPlots, although it is not required.

\section[About PGFPlots: Preliminaries]{About {\normalfont\PGFPlots}: Preliminaries}
This section contains information about upgrades, the team, the installation (in case you need to do it manually) and troubleshooting. You may skip it completely except for the upgrade remarks.

However, note that this library requires at least \PGF\ version $2.00$. At the time of this writing, many \TeX-distributions still contain the older \PGF\ version $1.18$, so it may be necessary to install a recent \PGF\ prior to using \PGFPlots.

\subsection{Components}
\PGFPlots\ comes with two components:
\begin{enumerate}
	\item the plotting component (which you are currently reading) and
	\item the \PGFPlotstable\ component which simplifies number formatting and postprocessing of numerical tables. It comes as a separate package and has its own manual \href{file:pgfplotstable.pdf}{pgfplotstable.pdf}.
\end{enumerate}

\subsection{Upgrade remarks}
This release provides a lot of improvements which can be found in all detail in \texttt{ChangeLog} for interested readers. However, some attention is useful with respect to the following changes.

\subsubsection{New Optional Features}
\begin{enumerate}
	\item \PGFPlots\ 1.3 comes with user interface improvements. The technical distinction between ``behavior options'' and ``style options'' of older versions is no longer necessary (although still fully supported).

	\item \PGFPlots\ 1.3 has a new feature which allows to \emph{move axis labels tight to tick labels} automatically.

	Since this affects the spacing, it is not enabled be default.

	Use
\begin{codeexample}[code only]
\usepackage{pgfplots}
\pgfplotsset{compat=1.3}
\end{codeexample}
	in your preamble to benefit from the improved spacing\footnote{The |compat=1.3| setting is necessary for new features which move axis labels around like reversed axis. However, \PGFPlots\ will still work as it does in earlier versions, your documents will remain the same if you don't set it explicitly.}.
	Take a look at the next page for the precise definition of |compat|.

	\item \PGFPlots\ 1.3 now supports reversed axes. It is no longer necessary to use work--arounds with negative units.
\pgfkeys{/pdflinks/search key prefixes in/.add={/pgfplots/,}{}}

	Take a look at the |x dir=reverse| key.

	Existing work--arounds will still function properly. Use |\pgfplotsset{compat=1.3}| together with |x dir=reverse| to switch to the new version.
\end{enumerate}

\subsubsection{Old Features Which May Need Attention}
\begin{enumerate}
	\item The |scatter/classes| feature produces proper legends as of version 1.3. This may change the appearance of existing legends of plots with |scatter/classes|.

	\item Due to a small math bug in \PGF\ $2.00$, you \emph{can't} use the math expression `|-x^2|'. It is necessary to use `|0-x^2|' instead. The same holds for
		`|exp(-x^2)|'; use `|exp(0-x^2)|' instead.

		This will be fixed with \PGF\ $>2.00$.

	\item Starting with \PGFPlots\ $1.1$, |\tikzstyle| should \emph{no longer be used} to set \PGFPlots\ options.
	
	Although |\tikzstyle| is still supported for some older \PGFPlots\ options, you should replace any occurance of |\tikzstyle| with |\pgfplotsset{|\meta{style name}|/.style={|\meta{key-value-list}|}}| or the associated |/.append style| variant. See section~\ref{sec:styles} for more detail.
\end{enumerate}
I apologize for any inconvenience caused by these changes.

\begin{pgfplotskey}{compat=\mchoice{1.3,pre 1.3,default,newest} (initially default)}
	The preamble configuration 
\begin{codeexample}[code only]
\usepackage{pgfplots}
\pgfplotsset{compat=1.3}
\end{codeexample}
	allows to choose between backwards compatibility and most recent features.

	\PGFPlots\ version 1.3 comes with new features which may lead to different spacings compared to earlier versions:
	\begin{enumerate}
		\item Version 1.3 comes with the |xlabel near ticks| feature which places (in this case $x$) axis labels ``near'' the tick labels, taking the size of tick labels into account.
		
		Older versions used |xlabel absolute|, i.e.\ an absolute distance between the axis and axis labels (now matter how large or small tick labels are).

		The initial setting \emph{keeps backwards compatibility}.

		You are encouraged to use
\begin{codeexample}[code only]
\usepackage{pgfplots}
\pgfplotsset{compat=1.3}
\end{codeexample}
		in your preamble.
		
		\item Version 1.3 fixes a bug with |anchor=south west| which occurred mainly for reversed axes: the anchors where upside--down. This is fixed now.

		
		If you have written some work--around and want to keep it going, use

		|\pgfplotsset{compat/anchors=pre 1.3}|\pgfmanualpdflabel{/pgfplots/compat/anchors}{} 

		to disable the bug fix.
	\end{enumerate}

	Use |\pgfplotsset{compat=default}| to restore the factory settings.

	The setting |\pgfplotsset{compat=newest}| will always use features of the most recent version. This might result in small changes in the document's appearance.
\end{pgfplotskey}

\subsection{The Team}
\PGFPlots\ has been written mainly by Christian Feuers�nger with many improvements of Pascal Wolkotte and Nick Papior Andersen as a spare time project. We hope it is useful and provides valuable plots.

If you are interested in writing something but don't know how, consider reading the auxiliary manual \href{file:TeX-programming-notes.pdf}{TeX-programming-notes.pdf} which comes with \PGFPlots. It is far from complete, but maybe it is a good starting point (at least for more literature).

\subsection{Acknowledgements}
I thank God for all hours of enjoyed programming. I thank Pascal Wolkotte and Nick Papior Andersen for their programming efforts and contributions as part of the development team. I thank Stefan Tibus, who contributed the |plot shell| feature. I thank Tom Cashman for the contribution of the |reverse legend| feature. Special thanks go to Stefan Pinnow whose tests of \PGFPlots\ lead to numerous quality improvements. Furthermore, I thank Dr.~Schweitzer for many fruitful discussions and Dr.~Meine for his ideas and suggestions. Thanks as well to the many international contributors who provided feature requests or identified bugs or simply manual improvements!

Last but not least, I thank Till Tantau and Mark Wibrow for their excellent graphics (and more) package \PGF\ and \Tikz\ which is the base of \PGFPlots.

\include{pgfplots.install}%
\include{pgfplots.intro}%
\include{pgfplots.reference}%
\include{pgfplots.libs}%
\include{pgfplots.resources}%
\include{pgfplots.importexport}%
\include{pgfplots.basic.reference}%

\printindex

\bibliographystyle{abbrv} %gerapali} %gerabbrv} %gerunsrt.bst} %gerabbrv}% gerplain}
\nocite{pgfplotstable}
\nocite{programmingnotes}
\bibliography{pgfplots}
\end{document}
%
\include{pgfplots.basic.reference}%

\printindex

\bibliographystyle{abbrv} %gerapali} %gerabbrv} %gerunsrt.bst} %gerabbrv}% gerplain}
\nocite{pgfplotstable}
\nocite{programmingnotes}
\bibliography{pgfplots}
\end{document}
%
\include{pgfplots.basic.reference}%

\printindex

\bibliographystyle{abbrv} %gerapali} %gerabbrv} %gerunsrt.bst} %gerabbrv}% gerplain}
\nocite{pgfplotstable}
\nocite{programmingnotes}
\bibliography{pgfplots}
\end{document}
%
%%%%%%%%%%%%%%%%%%%%%%%%%%%%%%%%%%%%%%%%%%%%%%%%%%%%%%%%%%%%%%%%%%%%%%%%%%%%%
%
% Package pgfplots.sty documentation. 
%
% Copyright 2007/2008 by Christian Feuersaenger.
%
% This program is free software: you can redistribute it and/or modify
% it under the terms of the GNU General Public License as published by
% the Free Software Foundation, either version 3 of the License, or
% (at your option) any later version.
% 
% This program is distributed in the hope that it will be useful,
% but WITHOUT ANY WARRANTY; without even the implied warranty of
% MERCHANTABILITY or FITNESS FOR A PARTICULAR PURPOSE.  See the
% GNU General Public License for more details.
% 
% You should have received a copy of the GNU General Public License
% along with this program.  If not, see <http://www.gnu.org/licenses/>.
%
%
%%%%%%%%%%%%%%%%%%%%%%%%%%%%%%%%%%%%%%%%%%%%%%%%%%%%%%%%%%%%%%%%%%%%%%%%%%%%%
%%%%%%%%%%%%%%%%%%%%%%%%%%%%%%%%%%%%%%%%%%%%%%%%%%%%%%%%%%%%%%%%%%%%%%%%%%%%%
%
% Package pgfplots.sty documentation. 
%
% Copyright 2007/2008 by Christian Feuersaenger.
%
% This program is free software: you can redistribute it and/or modify
% it under the terms of the GNU General Public License as published by
% the Free Software Foundation, either version 3 of the License, or
% (at your option) any later version.
% 
% This program is distributed in the hope that it will be useful,
% but WITHOUT ANY WARRANTY; without even the implied warranty of
% MERCHANTABILITY or FITNESS FOR A PARTICULAR PURPOSE.  See the
% GNU General Public License for more details.
% 
% You should have received a copy of the GNU General Public License
% along with this program.  If not, see <http://www.gnu.org/licenses/>.
%
%
%%%%%%%%%%%%%%%%%%%%%%%%%%%%%%%%%%%%%%%%%%%%%%%%%%%%%%%%%%%%%%%%%%%%%%%%%%%%%
\documentclass[a4paper]{ltxdoc}

\usepackage{makeidx}

% DON't let hyperref overload the format of index and glossary. 
% I want to do that on my own in the stylefiles for makeindex...
\makeatletter
\let\@old@wrindex=\@wrindex
\makeatother

\usepackage{ifpdf}
\ifpdf
	\usepackage[pdftex]{hyperref}
\else
	\usepackage[dvipdfm]{hyperref}
\fi
\hypersetup{pdfborder={0 0 0}}

\makeatletter
\let\@wrindex=\@old@wrindex
\makeatother

% Formatiere Seitennummern im Index:
\newcommand{\indexpageno}[1]{%
	{\bfseries\hyperpage{#1}}%
}


\newcommand{\R}{\mathbb{R}}
\newcommand{\N}{\mathbb{N}}
\newcommand{\Z}{\mathbb{Z}}

\long\def\COMMENTLOWLEVEL#1\ENDCOMMENT{}
\def\ENDCOMMENT{}

\usepackage{textcomp}
\usepackage{booktabs}

\usepackage{calc}
\usepackage[formats]{listings}
%\usepackage{courier} % don't use it - the '^' character can't be copy-pasted in courier

\usepackage{array}
\lstset{%
	basicstyle=\ttfamily,
	language=[LaTeX]tex, % Seems as if \lstset{language=tex} must be invoked BEFORE loading tikz!?
	tabsize=4,
	breaklines=true,
	breakindent=0pt
}

\ifpdf
	\pdfinfo {
		/Author	(Christian Feuersaenger)
	}

\else
	\def\pgfsysdriver{pgfsys-dvipdfm.def}
\fi
%\def\pgfsysdriver{pgfsys-pdftex.def}
\usepackage{pgfplots}

\ifpdf
	\usetikzlibrary{pgfplotsclickable}
\fi

%\usepackage{fp}
% ATTENTION:
% this requires pgf version NEWER than 2.00 :
%\usetikzlibrary{fixedpointarithmetic}

\usetikzlibrary{dateplot}

\usepackage[a4paper,left=2.25cm,right=2.25cm,top=2.5cm,bottom=2.5cm,nohead]{geometry}
\usepackage{amsmath,amssymb}
\usepackage{xxcolor}
\usepackage{pifont}
\usepackage[latin1]{inputenc}
\usepackage{amsmath}
\usepackage{eurosym}

\def\eps{\textsc{eps}}

\input pgfmanual-en-macros.tex

\def\pgfmanualbar{\char`\|}
\makeatletter
\newenvironment{addplotoperation}[3][]{
  \begin{pgfmanualentry}
  	{%
	\let\ltxdoc@marg=\marg
	\let\ltxdoc@oarg=\oarg
	\let\ltxdoc@parg=\parg
	\let\ltxdoc@meta=\meta
	\def\marg##1{{\normalfont\ltxdoc@marg{##1}}}%
	\def\oarg##1{{\normalfont\ltxdoc@oarg{##1}}}%
	\def\parg##1{{\normalfont\ltxdoc@parg{##1}}}%
	\def\meta##1{{\normalfont\ltxdoc@meta{##1}}}%
    \pgfmanualentryheadline{\textcolor{gray}{{\ttfamily\char`\\addplot\ }}%
      \declare{\texttt{#2}} \texttt{#3;}}%
	  \unskip
	 \nobreak
    \pgfmanualentryheadline{\textcolor{gray}{{\texttt{\char`\\addplot}\oarg{style options} \texttt{plot}\oarg{behavior options}}\ }%
      \declare{\texttt{#2}} \texttt{#3} \textcolor{gray}{\meta{trailing path commands}}\texttt{;}}%
    \def\pgfmanualtest{#1}%
    \ifx\pgfmanualtest\@empty%
      \index{#2@\protect\textcolor{gray}{\protect\texttt{plot}}\protect\texttt{ #2}}%
      \index{Plot operations!plot #2@\protect\texttt{plot #2}}%
    \fi%
	}%
    \pgfmanualbody
}
{
  \end{pgfmanualentry}
}

\newenvironment{codekey}[1]{%
  \begin{pgfmanualentry}
 	\pgfmanualentryheadline{{\ttfamily\declare{#1}\textcolor{gray}{/.code}=\marg{...}}\hfill}%
	\def\mykey{#1}%
	\def\mypath{}%
	\def\myname{}%
	\firsttimetrue%
	\decompose#1/\nil%
    \pgfmanualbody
}
{
  \end{pgfmanualentry}
}

\newenvironment{pgfplotscodekey}[1]{%
	\begin{codekey}{/pgfplots/#1}%
}
{
  \end{codekey}
}
\newenvironment{pgfplotscodetwokey}[1]{%
  \begin{pgfmanualentry}
 	\pgfmanualentryheadline{{\ttfamily\declare{/pgfplots/#1}\textcolor{gray}{/.code 2 args}=\marg{...}}\hfill}%
	\def\mykey{/pgfplots/#1}%
	\def\mypath{}%
	\def\myname{}%
	\firsttimetrue%
	\decompose/pgfplots/#1/\nil%
    \pgfmanualbody
}
{
  \end{pgfmanualentry}
}

\newenvironment{pgfplotsxycodekeylist}[1]{%
	\begingroup
	\let\oldpgfmanualentryheadline=\pgfmanualentryheadline
	\def\pgfmanualentryheadline##1{%
		\pgfmanualentryheadline@##1\pgfplots@EOI
	}%
	\def\pgfmanualentryheadline@##1\hfill##2\pgfplots@EOI{%
		\oldpgfmanualentryheadline{{\ttfamily\declare{##1}\textcolor{gray}{/.code}=\marg{...}}\hfill}%
	}
	\begin{pgfplotsxykeylist}{#1}%
}
{
	\end{pgfplotsxykeylist}
	\endgroup
}

\newenvironment{pgfplotskey}[1]{%
  \begin{key}{/pgfplots/#1}%
}
{
  \end{key}
}

\def\choicesep{$\vert$}%
\def\choicearg#1{\texttt{#1}}

\newif\iffirstchoice
\newcommand\mchoice[1]{%
	\begingroup
	\firstchoicetrue
	\foreach \mchoice@ in {#1} {%
		\iffirstchoice
			\global\firstchoicefalse
		\else
			\choicesep
		\fi
		\choicearg{\mchoice@}%
	}%
	\endgroup
}%

% \begin{xykey}{/path/\x label=value}
% \end{xykey}
%
% has same features with 'default', 'initially' etc as key environment
\newenvironment{xykey}[2][]{%
	\begin{pgfmanualentry}
    \def\extrakeytext{}
	\insertpathifneeded{#2}{#1}%
	\expandafter\pgfutil@in@\expandafter=\expandafter{\mykey}%
	\ifpgfutil@in@%
		\expandafter\xykey@eq\mykey\@nil
	\else
		\expandafter\xykey@noeq\mykey\@nil
	\fi
	\pgfmanualbody
}{%
	\end{pgfmanualentry}
}%

% \begin{xystylekey}{/path/\x label=value}
% \end{xystylekey}
%
% has same features with 'default', 'initially' etc as key environment
\newenvironment{xystylekey}[2][]{%
	\begin{pgfmanualentry}
    \def\extrakeytext{style, }
	\insertpathifneeded{#2}{#1}%
	\expandafter\pgfutil@in@\expandafter=\expandafter{\mykey}%
	\ifpgfutil@in@%
		\expandafter\xykey@eq\mykey\@nil
	\else
		\expandafter\xykey@noeq\mykey\@nil
	\fi
	\pgfmanualbody
}{%
	\end{pgfmanualentry}
}%

% \insertpathifneeded{a key}{/pgfplots} -> assign mykey={/pgfplots/a key}
% \insertpathifneeded{/tikz/a key}{/pgfplots} -> assign mykey={/tikz/a key}
%
% #1: the key
% #2: a default path (or empty)
\def\insertpathifneeded#1#2{%
	\def\insertpathifneeded@@{#2}%
	\ifx\insertpathifneeded@@\empty
		\def\mykey{#1}%
	\else
		\insertpathifneeded@#2\@nil
		\ifpgfutil@in@
			\def\mykey{#2/#1}%
		\else
			\def\mykey{#1}%
		\fi
	\fi
}%
\def\insertpathifneeded@#1#2\@nil{%
	\def\insertpathifneeded@@{#1}%
	\def\insertpathifneeded@@@{/}%
	\ifx\insertpathifneeded@@\insertpathifneeded@@@
		\pgfutil@in@true
	\else
		\pgfutil@in@false
	\fi
}%

% \begin{keylist}[default path]
% 	{/path/option 1=value,/path/option 2=value2}
% \end{keylist}
\newenvironment{keylist}[2][]{%
	\begin{pgfmanualentry}
    \def\extrakeytext{}%
	\foreach \xx in {#2} {%
		\expandafter\insertpathifneeded\expandafter{\xx}{#1}%
		\expandafter\extractkey\mykey\@nil%
	}%
	\pgfmanualbody
}{%
  \end{pgfmanualentry}
}%

\newenvironment{pgfplotskeylist}[1]{%
	\begin{keylist}[/pgfplots]{#1}%
}{%
	\end{keylist}%
}

% \begin{xykeylist}[default path]
% 	{/path/option \x1=value,/path/option \x2=value2,/path/option \x3=value}
% \end{xykeylist}
\newenvironment{xykeylist}[2][]{%
	\begin{pgfmanualentry}
    \def\extrakeytext{}
	\foreach \xx in {#2} {%
		\expandafter\insertpathifneeded\expandafter{\xx}{#1}%
		\expandafter\pgfutil@in@\expandafter=\expandafter{\mykey}%
		\ifpgfutil@in@%
			\expandafter\xykey@eq\mykey\@nil
		\else
			\expandafter\xykey@noeq\mykey\@nil
		\fi
	}%
	\pgfmanualbody
}{%
  \end{pgfmanualentry}
}%

\newif\ifxykeyfound

\def\xykey@eq#1=#2\@nil{%
	\def\x{x}%
	\xdef\mykey{#1}%
	\def\xykey@@{#1}%
	\ifx\xykey@@\mykey
		\xykeyfoundfalse
	\else
		\xykeyfoundtrue
	\fi
    \expandafter\extractkey\mykey=#2\@nil%
	\ifxykeyfound
		\def\x{y}%
		\xdef\mykey{#1}%
		\expandafter\extractkey\mykey=#2\@nil%
	\fi
}
\def\xykey@noeq#1\@nil{%
	\def\x{x}%
	\xdef\mykey{#1}%
	\def\xykey@@{#1}%
	\ifx\xykey@@\mykey
		\xykeyfoundfalse
	\else
		\xykeyfoundtrue
	\fi
    \expandafter\extractkey\mykey\@nil%
	\ifxykeyfound
		\def\x{y}%
		\xdef\mykey{#1}%
		\expandafter\extractkey\mykey\@nil%
	\fi
}

% \begin{pgfplotsxykey}{\x label=value}
% \end{pgfplotsxykey}
%
% It introduces the path /pgfplots/ automatically.
%
% has same features with 'default', 'initially' etc as key environment
\newenvironment{pgfplotsxykey}[1]{%
	\begin{xykey}[/pgfplots]{#1}%
}{%
	\end{xykey}%
}


\newenvironment{pgfplotsxykeylist}[1]{%
	\begin{xykeylist}[/pgfplots]{#1}%
}{%
	\end{xykeylist}%
}


\newenvironment{plottype}[1]{%
	\begin{key}{/tikz/#1}%
	\end{key}
  \begin{pgfmanualentry}
    \pgfmanualentryheadline{\textcolor{gray}{{\ttfamily\char`\\addplot+[\declare{#1}]}}}%
    \pgfmanualbody
}
{
  \end{pgfmanualentry}
}

\def\index@prologue{\section*{Index}\addcontentsline{toc}{section}{Index}
}

\newenvironment{pgfplotstablecolumnkey}{%
  \begin{pgfmanualentry}
 	\pgfmanualentryheadline{{\ttfamily\textcolor{gray}{/pgfplots/table/}\declare{columns/\meta{column name}}\textcolor{gray}{/.style}=\marg{key-value-list}}\hfill}%
	\pgfplotsmanualkeyindex{/pgfplots/table/columns}%
    \pgfmanualbody
}
{
  \end{pgfmanualentry}
}
\newenvironment{pgfplotstabledisplaycolumnkey}{%
  \begin{pgfmanualentry}
 	\pgfmanualentryheadline{{\ttfamily\textcolor{gray}{/pgfplots/table/}\declare{display columns/\meta{index}}\textcolor{gray}{/.style}=\marg{key-value-list}}\hfill}%
	\pgfplotsmanualkeyindex{/pgfplots/table/display columns}%
    \pgfmanualbody
}
{
  \end{pgfmanualentry}
}
\newenvironment{pgfplotstablealiaskey}{%
  \begin{pgfmanualentry}
 	\pgfmanualentryheadline{{\ttfamily\textcolor{gray}{/pgfplots/table/}\declare{alias/\meta{col name}}\textcolor{gray}{/.initial}=\marg{real col name}}\hfill}%
	\pgfplotsmanualkeyindex{/pgfplots/table/alias}%
    \pgfmanualbody
}
{
  \end{pgfmanualentry}
}


\def\pgfplotsmanualkeyindex#1{%
	\def\mypath{#1}%
	\def\myname{}%
	\firsttimetrue%
	\decompose#1/\nil%
}
\newenvironment{pgfplotstablecreateonusekey}{%
  \begin{pgfmanualentry}
 	\pgfmanualentryheadline{{\ttfamily\textcolor{gray}{/pgfplots/table/}\declare{create on use/\meta{col name}}\textcolor{gray}{/.style}=\marg{create options}}\hfill}%
	\def\mykey{/pgfplots/table/create on use}%
    \pgfmanualbody
	\pgfplotsmanualkeyindex{/pgfplots/table/create on use}%
}
{
  \end{pgfmanualentry}
}

\def\pgfplotsassertcmdkeyexists#1{%
	\pgfkeysifdefined{/pgfplots/#1/.@cmd}\relax{%
		\pgfplots@error{DOCUMENTATION ERROR: command key /pgfplots/#1 does not exist!}%
	}%
}%

{
\catcode`\ =12%
\gdef\makespaceexpandable{\def\ { }}}%

\def\pgfplotsassertXYcmdkeyexists#1{%
	{\makespaceexpandable\def\x{x}\edef\pgfplotsassertXYcmdkeyexists@tmp{#1}%
	\pgfkeysifdefined{/pgfplots/\pgfplotsassertXYcmdkeyexists@tmp/.@cmd}\relax{%
		\pgfplots@error{DOCUMENTATION ERROR: command key /pgfplots/#1 does not exist!}%
	}}%
	{\makespaceexpandable\def\x{y}\edef\pgfplotsassertXYcmdkeyexists@tmp{#1}%
	\pgfkeysifdefined{/pgfplots/\pgfplotsassertXYcmdkeyexists@tmp/.@cmd}\relax{%
		\pgfplots@error{DOCUMENTATION ERROR: command key /pgfplots/#1 does not exist!}%
	}}%
}%

\def\pgfplotsshortstylekey #1=#2\pgfeov{%
	\pgfplotsassertcmdkeyexists{#1}%
	\pgfplotsassertcmdkeyexists{#2}%
	\begin{pgfplotskey}{#1=\marg{key-value-list}}
		An abbreviation for \texttt{#2/.append style=}\marg{key-value-list}.
	\end{pgfplotskey}
}
\def\pgfplotsshortxystylekey #1=#2\pgfeov{%
	\pgfplotsassertXYcmdkeyexists{#1}%
	\pgfplotsassertXYcmdkeyexists{#2}%
	\begin{pgfplotsxykey}{#1=\marg{key-value-list}}
		An abbreviation for {\def\x{x}\texttt{#2/.append style=}}\marg{key-value-list} (or the respective style for $y$, {\def\x{y}\texttt{#2}}).
	\end{pgfplotsxykey}
}
\def\pgfplotsshortstylekeys #1,#2=#3\pgfeov{%
	\pgfplotsassertcmdkeyexists{#1}%
	\pgfplotsassertcmdkeyexists{#2}%
	\pgfplotsassertcmdkeyexists{#3}%
	\begin{pgfplotskeylist}{%
		#1=\marg{key-value-list},
		#2=\marg{key-value-list}}
		Different abbreviations for \texttt{#3/.append style=}\marg{key-value-list}.
	\end{pgfplotskeylist}
}
\def\pgfplotsshortxystylekeys #1,#2=#3\pgfeov{%
	\pgfplotsassertXYcmdkeyexists{#1}%
	\pgfplotsassertXYcmdkeyexists{#2}%
	\pgfplotsassertXYcmdkeyexists{#3}%
	\begin{pgfplotsxykeylist}{%
		#1=\marg{key-value-list},
		#2=\marg{key-value-list}}
		Different abbreviations for {\def\x{x}\texttt{#3/.append style=}}\marg{key-value-list} (or the respective style for $y$, {\def\x{y}\texttt{#3}}).
	\end{pgfplotsxykeylist}
}

%
% Creates and shows a colormap with specification '#1'.
\def\pgfplotsshowcolormapexample#1{%
	\pgfplotscreatecolormap{tempcolormap}{#1}%
	\pgfplotsshowcolormap{tempcolormap}%
}

% Shows the colormap named '#1'.
\def\pgfplotsshowcolormap#1{%
	\pgfplotscolormaptoshadingspec{#1}{8cm}\result
	\def\tempb{\pgfdeclarehorizontalshading{tempshading}{1cm}}%
	\expandafter\tempb\expandafter{\result}%
	\pgfuseshading{tempshading}%
}

\makeatother


\pgfqkeys{/codeexample}{%
	every codeexample/.style={
		width=8cm,
		/pgfplots/every axis/.append style={legend style={fill=graphicbackground}}
	},
	tabsize=4,
}
\pgfplotsset{
	every axis/.append style={width=7cm},
	filter discard warning=false,
}

\usetikzlibrary{backgrounds,patterns}
% Global styles:
\tikzset{
  shape example/.style={
    color=black!30,
    draw,
    fill=yellow!30,
    line width=.5cm,
    inner xsep=2.5cm,
    inner ysep=0.5cm}
}

\newcommand{\FIXME}[1]{\textcolor{red}{(FIXME: #1)}}

% fuer endvironment 'sidewaysfigure' bspw
% \usepackage{rotating}

\newcommand\Tikz{Ti\textit kZ}
\newcommand\PGF{\textsc{pgf}}
\newcommand\PGFPlots{\textsc{pgfplots}}
\def\PGFPlotstable{\textsc{PgfplotsTable}}


\makeindex

% Fix overful hboxes automatically:
\tolerance=2000
\emergencystretch=10pt

\tikzset{prefix=gnuplot/pgfplots_} % prefix for 'plot function'

\author{%
	Christian Feuers\"anger\footnote{\url{http://wissrech.ins.uni-bonn.de/people/feuersaenger}}\\%
	Institut f\"ur Numerische Simulation\\
	Universit\"at Bonn, Germany}


%\RequirePackage[german,english,francais]{babel}

\pgfplotsmanualexternalexpensivetrue

%\usetikzlibrary{external}
%\tikzexternalize[prefix=figures/]{pgfplots}

\title{%
	Manual for Package \PGFPlots\\
	{\small Version \pgfplotsversion}\\
	{\small\href{http://sourceforge.net/projects/pgfplots}{http://sourceforge.net/projects/pgfplots}}}

%\includeonly{pgfplots.libs}


\begin{document}

\def\plotcoords{%
\addplot coordinates {
(5,8.312e-02)    (17,2.547e-02)   (49,7.407e-03)
(129,2.102e-03)  (321,5.874e-04)  (769,1.623e-04)
(1793,4.442e-05) (4097,1.207e-05) (9217,3.261e-06)
};

\addplot coordinates{
(7,8.472e-02)    (31,3.044e-02)    (111,1.022e-02)
(351,3.303e-03)  (1023,1.039e-03)  (2815,3.196e-04)
(7423,9.658e-05) (18943,2.873e-05) (47103,8.437e-06)
};

\addplot coordinates{
(9,7.881e-02)     (49,3.243e-02)    (209,1.232e-02)
(769,4.454e-03)   (2561,1.551e-03)  (7937,5.236e-04)
(23297,1.723e-04) (65537,5.545e-05) (178177,1.751e-05)
};

\addplot coordinates{
(11,6.887e-02)    (71,3.177e-02)     (351,1.341e-02)
(1471,5.334e-03)  (5503,2.027e-03)   (18943,7.415e-04)
(61183,2.628e-04) (187903,9.063e-05) (553983,3.053e-05)
};

\addplot coordinates{
(13,5.755e-02)     (97,2.925e-02)     (545,1.351e-02)
(2561,5.842e-03)   (10625,2.397e-03)  (40193,9.414e-04)
(141569,3.564e-04) (471041,1.308e-04) 
(1496065,4.670e-05)
};
}%
\maketitle
\begin{abstract}%
\PGFPlots\ draws high--quality function plots in normal or logarithmic scaling with a user-friendly interface directly in \TeX. The user supplies axis labels, legend entries and the plot coordinates for one or more plots and \PGFPlots\ applies axis scaling, computes any logarithms and axis ticks and draws the plots, supporting line plots, scatter plots, piecewise constant plots, bar plots, area plots, mesh-- and surface plots and some more. It is based on Till Tantau's package \PGF/\Tikz.
\end{abstract}
\tableofcontents
\section{Introduction}
This package provides tools to generate plots and labeled axes easily. It draws normal plots, logplots and semi-logplots, in two and three dimensions. Axis ticks, labels, legends (in case of multiple plots) can be added with key-value options. It can cycle through a set of predefined line/marker/color specifications. In summary, its purpose is to simplify the generation of high-quality function and/or data plots, and solving the problems of
\begin{itemize}
	\item consistency of document and font type and font size,
	\item direct use of \TeX\ math mode in axis descriptions,
	\item consistency of data and figures (no third party tool necessary),
	\item inter document consistency using preamble configurations and styles.
\end{itemize}
Although not necessary, separate |.pdf| or |.eps| graphics can be generated using the |external| library developed as part of \Tikz.

\PGFPlots\ is build completely on \Tikz/\PGF. Knowledge of \Tikz\ will simplify the work with \PGFPlots, although it is not required.

\section[About PGFPlots: Preliminaries]{About {\normalfont\PGFPlots}: Preliminaries}
This section contains information about upgrades, the team, the installation (in case you need to do it manually) and troubleshooting. You may skip it completely except for the upgrade remarks.

However, note that this library requires at least \PGF\ version $2.00$. At the time of this writing, many \TeX-distributions still contain the older \PGF\ version $1.18$, so it may be necessary to install a recent \PGF\ prior to using \PGFPlots.

\subsection{Components}
\PGFPlots\ comes with two components:
\begin{enumerate}
	\item the plotting component (which you are currently reading) and
	\item the \PGFPlotstable\ component which simplifies number formatting and postprocessing of numerical tables. It comes as a separate package and has its own manual \href{file:pgfplotstable.pdf}{pgfplotstable.pdf}.
\end{enumerate}

\subsection{Upgrade remarks}
This release provides a lot of improvements which can be found in all detail in \texttt{ChangeLog} for interested readers. However, some attention is useful with respect to the following changes.

\subsubsection{New Optional Features}
\begin{enumerate}
	\item \PGFPlots\ 1.3 comes with user interface improvements. The technical distinction between ``behavior options'' and ``style options'' of older versions is no longer necessary (although still fully supported).

	\item \PGFPlots\ 1.3 has a new feature which allows to \emph{move axis labels tight to tick labels} automatically.

	Since this affects the spacing, it is not enabled be default.

	Use
\begin{codeexample}[code only]
\usepackage{pgfplots}
\pgfplotsset{compat=1.3}
\end{codeexample}
	in your preamble to benefit from the improved spacing\footnote{The |compat=1.3| setting is necessary for new features which move axis labels around like reversed axis. However, \PGFPlots\ will still work as it does in earlier versions, your documents will remain the same if you don't set it explicitly.}.
	Take a look at the next page for the precise definition of |compat|.

	\item \PGFPlots\ 1.3 now supports reversed axes. It is no longer necessary to use work--arounds with negative units.
\pgfkeys{/pdflinks/search key prefixes in/.add={/pgfplots/,}{}}

	Take a look at the |x dir=reverse| key.

	Existing work--arounds will still function properly. Use |\pgfplotsset{compat=1.3}| together with |x dir=reverse| to switch to the new version.
\end{enumerate}

\subsubsection{Old Features Which May Need Attention}
\begin{enumerate}
	\item The |scatter/classes| feature produces proper legends as of version 1.3. This may change the appearance of existing legends of plots with |scatter/classes|.

	\item Due to a small math bug in \PGF\ $2.00$, you \emph{can't} use the math expression `|-x^2|'. It is necessary to use `|0-x^2|' instead. The same holds for
		`|exp(-x^2)|'; use `|exp(0-x^2)|' instead.

		This will be fixed with \PGF\ $>2.00$.

	\item Starting with \PGFPlots\ $1.1$, |\tikzstyle| should \emph{no longer be used} to set \PGFPlots\ options.
	
	Although |\tikzstyle| is still supported for some older \PGFPlots\ options, you should replace any occurance of |\tikzstyle| with |\pgfplotsset{|\meta{style name}|/.style={|\meta{key-value-list}|}}| or the associated |/.append style| variant. See section~\ref{sec:styles} for more detail.
\end{enumerate}
I apologize for any inconvenience caused by these changes.

\begin{pgfplotskey}{compat=\mchoice{1.3,pre 1.3,default,newest} (initially default)}
	The preamble configuration 
\begin{codeexample}[code only]
\usepackage{pgfplots}
\pgfplotsset{compat=1.3}
\end{codeexample}
	allows to choose between backwards compatibility and most recent features.

	\PGFPlots\ version 1.3 comes with new features which may lead to different spacings compared to earlier versions:
	\begin{enumerate}
		\item Version 1.3 comes with the |xlabel near ticks| feature which places (in this case $x$) axis labels ``near'' the tick labels, taking the size of tick labels into account.
		
		Older versions used |xlabel absolute|, i.e.\ an absolute distance between the axis and axis labels (now matter how large or small tick labels are).

		The initial setting \emph{keeps backwards compatibility}.

		You are encouraged to use
\begin{codeexample}[code only]
\usepackage{pgfplots}
\pgfplotsset{compat=1.3}
\end{codeexample}
		in your preamble.
		
		\item Version 1.3 fixes a bug with |anchor=south west| which occurred mainly for reversed axes: the anchors where upside--down. This is fixed now.

		
		If you have written some work--around and want to keep it going, use

		|\pgfplotsset{compat/anchors=pre 1.3}|\pgfmanualpdflabel{/pgfplots/compat/anchors}{} 

		to disable the bug fix.
	\end{enumerate}

	Use |\pgfplotsset{compat=default}| to restore the factory settings.

	The setting |\pgfplotsset{compat=newest}| will always use features of the most recent version. This might result in small changes in the document's appearance.
\end{pgfplotskey}

\subsection{The Team}
\PGFPlots\ has been written mainly by Christian Feuers�nger with many improvements of Pascal Wolkotte and Nick Papior Andersen as a spare time project. We hope it is useful and provides valuable plots.

If you are interested in writing something but don't know how, consider reading the auxiliary manual \href{file:TeX-programming-notes.pdf}{TeX-programming-notes.pdf} which comes with \PGFPlots. It is far from complete, but maybe it is a good starting point (at least for more literature).

\subsection{Acknowledgements}
I thank God for all hours of enjoyed programming. I thank Pascal Wolkotte and Nick Papior Andersen for their programming efforts and contributions as part of the development team. I thank Stefan Tibus, who contributed the |plot shell| feature. I thank Tom Cashman for the contribution of the |reverse legend| feature. Special thanks go to Stefan Pinnow whose tests of \PGFPlots\ lead to numerous quality improvements. Furthermore, I thank Dr.~Schweitzer for many fruitful discussions and Dr.~Meine for his ideas and suggestions. Thanks as well to the many international contributors who provided feature requests or identified bugs or simply manual improvements!

Last but not least, I thank Till Tantau and Mark Wibrow for their excellent graphics (and more) package \PGF\ and \Tikz\ which is the base of \PGFPlots.

%%%%%%%%%%%%%%%%%%%%%%%%%%%%%%%%%%%%%%%%%%%%%%%%%%%%%%%%%%%%%%%%%%%%%%%%%%%%%
%
% Package pgfplots.sty documentation. 
%
% Copyright 2007/2008 by Christian Feuersaenger.
%
% This program is free software: you can redistribute it and/or modify
% it under the terms of the GNU General Public License as published by
% the Free Software Foundation, either version 3 of the License, or
% (at your option) any later version.
% 
% This program is distributed in the hope that it will be useful,
% but WITHOUT ANY WARRANTY; without even the implied warranty of
% MERCHANTABILITY or FITNESS FOR A PARTICULAR PURPOSE.  See the
% GNU General Public License for more details.
% 
% You should have received a copy of the GNU General Public License
% along with this program.  If not, see <http://www.gnu.org/licenses/>.
%
%
%%%%%%%%%%%%%%%%%%%%%%%%%%%%%%%%%%%%%%%%%%%%%%%%%%%%%%%%%%%%%%%%%%%%%%%%%%%%%
%%%%%%%%%%%%%%%%%%%%%%%%%%%%%%%%%%%%%%%%%%%%%%%%%%%%%%%%%%%%%%%%%%%%%%%%%%%%%
%
% Package pgfplots.sty documentation. 
%
% Copyright 2007/2008 by Christian Feuersaenger.
%
% This program is free software: you can redistribute it and/or modify
% it under the terms of the GNU General Public License as published by
% the Free Software Foundation, either version 3 of the License, or
% (at your option) any later version.
% 
% This program is distributed in the hope that it will be useful,
% but WITHOUT ANY WARRANTY; without even the implied warranty of
% MERCHANTABILITY or FITNESS FOR A PARTICULAR PURPOSE.  See the
% GNU General Public License for more details.
% 
% You should have received a copy of the GNU General Public License
% along with this program.  If not, see <http://www.gnu.org/licenses/>.
%
%
%%%%%%%%%%%%%%%%%%%%%%%%%%%%%%%%%%%%%%%%%%%%%%%%%%%%%%%%%%%%%%%%%%%%%%%%%%%%%
\documentclass[a4paper]{ltxdoc}

\usepackage{makeidx}

% DON't let hyperref overload the format of index and glossary. 
% I want to do that on my own in the stylefiles for makeindex...
\makeatletter
\let\@old@wrindex=\@wrindex
\makeatother

\usepackage{ifpdf}
\ifpdf
	\usepackage[pdftex]{hyperref}
\else
	\usepackage[dvipdfm]{hyperref}
\fi
\hypersetup{pdfborder={0 0 0}}

\makeatletter
\let\@wrindex=\@old@wrindex
\makeatother

% Formatiere Seitennummern im Index:
\newcommand{\indexpageno}[1]{%
	{\bfseries\hyperpage{#1}}%
}


\newcommand{\R}{\mathbb{R}}
\newcommand{\N}{\mathbb{N}}
\newcommand{\Z}{\mathbb{Z}}

\long\def\COMMENTLOWLEVEL#1\ENDCOMMENT{}
\def\ENDCOMMENT{}

\usepackage{textcomp}
\usepackage{booktabs}

\usepackage{calc}
\usepackage[formats]{listings}
%\usepackage{courier} % don't use it - the '^' character can't be copy-pasted in courier

\usepackage{array}
\lstset{%
	basicstyle=\ttfamily,
	language=[LaTeX]tex, % Seems as if \lstset{language=tex} must be invoked BEFORE loading tikz!?
	tabsize=4,
	breaklines=true,
	breakindent=0pt
}

\ifpdf
	\pdfinfo {
		/Author	(Christian Feuersaenger)
	}

\else
	\def\pgfsysdriver{pgfsys-dvipdfm.def}
\fi
%\def\pgfsysdriver{pgfsys-pdftex.def}
\usepackage{pgfplots}

\ifpdf
	\usetikzlibrary{pgfplotsclickable}
\fi

%\usepackage{fp}
% ATTENTION:
% this requires pgf version NEWER than 2.00 :
%\usetikzlibrary{fixedpointarithmetic}

\usetikzlibrary{dateplot}

\usepackage[a4paper,left=2.25cm,right=2.25cm,top=2.5cm,bottom=2.5cm,nohead]{geometry}
\usepackage{amsmath,amssymb}
\usepackage{xxcolor}
\usepackage{pifont}
\usepackage[latin1]{inputenc}
\usepackage{amsmath}
\usepackage{eurosym}
\input{pgfplots-macros}

\pgfqkeys{/codeexample}{%
	every codeexample/.style={
		width=8cm,
		/pgfplots/every axis/.append style={legend style={fill=graphicbackground}}
	},
	tabsize=4,
}
\pgfplotsset{
	every axis/.append style={width=7cm},
	filter discard warning=false,
}

\usetikzlibrary{backgrounds,patterns}
% Global styles:
\tikzset{
  shape example/.style={
    color=black!30,
    draw,
    fill=yellow!30,
    line width=.5cm,
    inner xsep=2.5cm,
    inner ysep=0.5cm}
}

\newcommand{\FIXME}[1]{\textcolor{red}{(FIXME: #1)}}

% fuer endvironment 'sidewaysfigure' bspw
% \usepackage{rotating}

\newcommand\Tikz{Ti\textit kZ}
\newcommand\PGF{\textsc{pgf}}
\newcommand\PGFPlots{\textsc{pgfplots}}
\def\PGFPlotstable{\textsc{PgfplotsTable}}


\makeindex

% Fix overful hboxes automatically:
\tolerance=2000
\emergencystretch=10pt

\tikzset{prefix=gnuplot/pgfplots_} % prefix for 'plot function'

\author{%
	Christian Feuers\"anger\footnote{\url{http://wissrech.ins.uni-bonn.de/people/feuersaenger}}\\%
	Institut f\"ur Numerische Simulation\\
	Universit\"at Bonn, Germany}


%\RequirePackage[german,english,francais]{babel}

\pgfplotsmanualexternalexpensivetrue

%\usetikzlibrary{external}
%\tikzexternalize[prefix=figures/]{pgfplots}

\title{%
	Manual for Package \PGFPlots\\
	{\small Version \pgfplotsversion}\\
	{\small\href{http://sourceforge.net/projects/pgfplots}{http://sourceforge.net/projects/pgfplots}}}

%\includeonly{pgfplots.libs}


\begin{document}

\def\plotcoords{%
\addplot coordinates {
(5,8.312e-02)    (17,2.547e-02)   (49,7.407e-03)
(129,2.102e-03)  (321,5.874e-04)  (769,1.623e-04)
(1793,4.442e-05) (4097,1.207e-05) (9217,3.261e-06)
};

\addplot coordinates{
(7,8.472e-02)    (31,3.044e-02)    (111,1.022e-02)
(351,3.303e-03)  (1023,1.039e-03)  (2815,3.196e-04)
(7423,9.658e-05) (18943,2.873e-05) (47103,8.437e-06)
};

\addplot coordinates{
(9,7.881e-02)     (49,3.243e-02)    (209,1.232e-02)
(769,4.454e-03)   (2561,1.551e-03)  (7937,5.236e-04)
(23297,1.723e-04) (65537,5.545e-05) (178177,1.751e-05)
};

\addplot coordinates{
(11,6.887e-02)    (71,3.177e-02)     (351,1.341e-02)
(1471,5.334e-03)  (5503,2.027e-03)   (18943,7.415e-04)
(61183,2.628e-04) (187903,9.063e-05) (553983,3.053e-05)
};

\addplot coordinates{
(13,5.755e-02)     (97,2.925e-02)     (545,1.351e-02)
(2561,5.842e-03)   (10625,2.397e-03)  (40193,9.414e-04)
(141569,3.564e-04) (471041,1.308e-04) 
(1496065,4.670e-05)
};
}%
\maketitle
\begin{abstract}%
\PGFPlots\ draws high--quality function plots in normal or logarithmic scaling with a user-friendly interface directly in \TeX. The user supplies axis labels, legend entries and the plot coordinates for one or more plots and \PGFPlots\ applies axis scaling, computes any logarithms and axis ticks and draws the plots, supporting line plots, scatter plots, piecewise constant plots, bar plots, area plots, mesh-- and surface plots and some more. It is based on Till Tantau's package \PGF/\Tikz.
\end{abstract}
\tableofcontents
\section{Introduction}
This package provides tools to generate plots and labeled axes easily. It draws normal plots, logplots and semi-logplots, in two and three dimensions. Axis ticks, labels, legends (in case of multiple plots) can be added with key-value options. It can cycle through a set of predefined line/marker/color specifications. In summary, its purpose is to simplify the generation of high-quality function and/or data plots, and solving the problems of
\begin{itemize}
	\item consistency of document and font type and font size,
	\item direct use of \TeX\ math mode in axis descriptions,
	\item consistency of data and figures (no third party tool necessary),
	\item inter document consistency using preamble configurations and styles.
\end{itemize}
Although not necessary, separate |.pdf| or |.eps| graphics can be generated using the |external| library developed as part of \Tikz.

\PGFPlots\ is build completely on \Tikz/\PGF. Knowledge of \Tikz\ will simplify the work with \PGFPlots, although it is not required.

\section[About PGFPlots: Preliminaries]{About {\normalfont\PGFPlots}: Preliminaries}
This section contains information about upgrades, the team, the installation (in case you need to do it manually) and troubleshooting. You may skip it completely except for the upgrade remarks.

However, note that this library requires at least \PGF\ version $2.00$. At the time of this writing, many \TeX-distributions still contain the older \PGF\ version $1.18$, so it may be necessary to install a recent \PGF\ prior to using \PGFPlots.

\subsection{Components}
\PGFPlots\ comes with two components:
\begin{enumerate}
	\item the plotting component (which you are currently reading) and
	\item the \PGFPlotstable\ component which simplifies number formatting and postprocessing of numerical tables. It comes as a separate package and has its own manual \href{file:pgfplotstable.pdf}{pgfplotstable.pdf}.
\end{enumerate}

\subsection{Upgrade remarks}
This release provides a lot of improvements which can be found in all detail in \texttt{ChangeLog} for interested readers. However, some attention is useful with respect to the following changes.

\subsubsection{New Optional Features}
\begin{enumerate}
	\item \PGFPlots\ 1.3 comes with user interface improvements. The technical distinction between ``behavior options'' and ``style options'' of older versions is no longer necessary (although still fully supported).

	\item \PGFPlots\ 1.3 has a new feature which allows to \emph{move axis labels tight to tick labels} automatically.

	Since this affects the spacing, it is not enabled be default.

	Use
\begin{codeexample}[code only]
\usepackage{pgfplots}
\pgfplotsset{compat=1.3}
\end{codeexample}
	in your preamble to benefit from the improved spacing\footnote{The |compat=1.3| setting is necessary for new features which move axis labels around like reversed axis. However, \PGFPlots\ will still work as it does in earlier versions, your documents will remain the same if you don't set it explicitly.}.
	Take a look at the next page for the precise definition of |compat|.

	\item \PGFPlots\ 1.3 now supports reversed axes. It is no longer necessary to use work--arounds with negative units.
\pgfkeys{/pdflinks/search key prefixes in/.add={/pgfplots/,}{}}

	Take a look at the |x dir=reverse| key.

	Existing work--arounds will still function properly. Use |\pgfplotsset{compat=1.3}| together with |x dir=reverse| to switch to the new version.
\end{enumerate}

\subsubsection{Old Features Which May Need Attention}
\begin{enumerate}
	\item The |scatter/classes| feature produces proper legends as of version 1.3. This may change the appearance of existing legends of plots with |scatter/classes|.

	\item Due to a small math bug in \PGF\ $2.00$, you \emph{can't} use the math expression `|-x^2|'. It is necessary to use `|0-x^2|' instead. The same holds for
		`|exp(-x^2)|'; use `|exp(0-x^2)|' instead.

		This will be fixed with \PGF\ $>2.00$.

	\item Starting with \PGFPlots\ $1.1$, |\tikzstyle| should \emph{no longer be used} to set \PGFPlots\ options.
	
	Although |\tikzstyle| is still supported for some older \PGFPlots\ options, you should replace any occurance of |\tikzstyle| with |\pgfplotsset{|\meta{style name}|/.style={|\meta{key-value-list}|}}| or the associated |/.append style| variant. See section~\ref{sec:styles} for more detail.
\end{enumerate}
I apologize for any inconvenience caused by these changes.

\begin{pgfplotskey}{compat=\mchoice{1.3,pre 1.3,default,newest} (initially default)}
	The preamble configuration 
\begin{codeexample}[code only]
\usepackage{pgfplots}
\pgfplotsset{compat=1.3}
\end{codeexample}
	allows to choose between backwards compatibility and most recent features.

	\PGFPlots\ version 1.3 comes with new features which may lead to different spacings compared to earlier versions:
	\begin{enumerate}
		\item Version 1.3 comes with the |xlabel near ticks| feature which places (in this case $x$) axis labels ``near'' the tick labels, taking the size of tick labels into account.
		
		Older versions used |xlabel absolute|, i.e.\ an absolute distance between the axis and axis labels (now matter how large or small tick labels are).

		The initial setting \emph{keeps backwards compatibility}.

		You are encouraged to use
\begin{codeexample}[code only]
\usepackage{pgfplots}
\pgfplotsset{compat=1.3}
\end{codeexample}
		in your preamble.
		
		\item Version 1.3 fixes a bug with |anchor=south west| which occurred mainly for reversed axes: the anchors where upside--down. This is fixed now.

		
		If you have written some work--around and want to keep it going, use

		|\pgfplotsset{compat/anchors=pre 1.3}|\pgfmanualpdflabel{/pgfplots/compat/anchors}{} 

		to disable the bug fix.
	\end{enumerate}

	Use |\pgfplotsset{compat=default}| to restore the factory settings.

	The setting |\pgfplotsset{compat=newest}| will always use features of the most recent version. This might result in small changes in the document's appearance.
\end{pgfplotskey}

\subsection{The Team}
\PGFPlots\ has been written mainly by Christian Feuers�nger with many improvements of Pascal Wolkotte and Nick Papior Andersen as a spare time project. We hope it is useful and provides valuable plots.

If you are interested in writing something but don't know how, consider reading the auxiliary manual \href{file:TeX-programming-notes.pdf}{TeX-programming-notes.pdf} which comes with \PGFPlots. It is far from complete, but maybe it is a good starting point (at least for more literature).

\subsection{Acknowledgements}
I thank God for all hours of enjoyed programming. I thank Pascal Wolkotte and Nick Papior Andersen for their programming efforts and contributions as part of the development team. I thank Stefan Tibus, who contributed the |plot shell| feature. I thank Tom Cashman for the contribution of the |reverse legend| feature. Special thanks go to Stefan Pinnow whose tests of \PGFPlots\ lead to numerous quality improvements. Furthermore, I thank Dr.~Schweitzer for many fruitful discussions and Dr.~Meine for his ideas and suggestions. Thanks as well to the many international contributors who provided feature requests or identified bugs or simply manual improvements!

Last but not least, I thank Till Tantau and Mark Wibrow for their excellent graphics (and more) package \PGF\ and \Tikz\ which is the base of \PGFPlots.

%%%%%%%%%%%%%%%%%%%%%%%%%%%%%%%%%%%%%%%%%%%%%%%%%%%%%%%%%%%%%%%%%%%%%%%%%%%%%
%
% Package pgfplots.sty documentation. 
%
% Copyright 2007/2008 by Christian Feuersaenger.
%
% This program is free software: you can redistribute it and/or modify
% it under the terms of the GNU General Public License as published by
% the Free Software Foundation, either version 3 of the License, or
% (at your option) any later version.
% 
% This program is distributed in the hope that it will be useful,
% but WITHOUT ANY WARRANTY; without even the implied warranty of
% MERCHANTABILITY or FITNESS FOR A PARTICULAR PURPOSE.  See the
% GNU General Public License for more details.
% 
% You should have received a copy of the GNU General Public License
% along with this program.  If not, see <http://www.gnu.org/licenses/>.
%
%
%%%%%%%%%%%%%%%%%%%%%%%%%%%%%%%%%%%%%%%%%%%%%%%%%%%%%%%%%%%%%%%%%%%%%%%%%%%%%
\input{pgfplots.preamble.tex}
%\RequirePackage[german,english,francais]{babel}

\pgfplotsmanualexternalexpensivetrue

%\usetikzlibrary{external}
%\tikzexternalize[prefix=figures/]{pgfplots}

\title{%
	Manual for Package \PGFPlots\\
	{\small Version \pgfplotsversion}\\
	{\small\href{http://sourceforge.net/projects/pgfplots}{http://sourceforge.net/projects/pgfplots}}}

%\includeonly{pgfplots.libs}


\begin{document}

\def\plotcoords{%
\addplot coordinates {
(5,8.312e-02)    (17,2.547e-02)   (49,7.407e-03)
(129,2.102e-03)  (321,5.874e-04)  (769,1.623e-04)
(1793,4.442e-05) (4097,1.207e-05) (9217,3.261e-06)
};

\addplot coordinates{
(7,8.472e-02)    (31,3.044e-02)    (111,1.022e-02)
(351,3.303e-03)  (1023,1.039e-03)  (2815,3.196e-04)
(7423,9.658e-05) (18943,2.873e-05) (47103,8.437e-06)
};

\addplot coordinates{
(9,7.881e-02)     (49,3.243e-02)    (209,1.232e-02)
(769,4.454e-03)   (2561,1.551e-03)  (7937,5.236e-04)
(23297,1.723e-04) (65537,5.545e-05) (178177,1.751e-05)
};

\addplot coordinates{
(11,6.887e-02)    (71,3.177e-02)     (351,1.341e-02)
(1471,5.334e-03)  (5503,2.027e-03)   (18943,7.415e-04)
(61183,2.628e-04) (187903,9.063e-05) (553983,3.053e-05)
};

\addplot coordinates{
(13,5.755e-02)     (97,2.925e-02)     (545,1.351e-02)
(2561,5.842e-03)   (10625,2.397e-03)  (40193,9.414e-04)
(141569,3.564e-04) (471041,1.308e-04) 
(1496065,4.670e-05)
};
}%
\maketitle
\begin{abstract}%
\PGFPlots\ draws high--quality function plots in normal or logarithmic scaling with a user-friendly interface directly in \TeX. The user supplies axis labels, legend entries and the plot coordinates for one or more plots and \PGFPlots\ applies axis scaling, computes any logarithms and axis ticks and draws the plots, supporting line plots, scatter plots, piecewise constant plots, bar plots, area plots, mesh-- and surface plots and some more. It is based on Till Tantau's package \PGF/\Tikz.
\end{abstract}
\tableofcontents
\section{Introduction}
This package provides tools to generate plots and labeled axes easily. It draws normal plots, logplots and semi-logplots, in two and three dimensions. Axis ticks, labels, legends (in case of multiple plots) can be added with key-value options. It can cycle through a set of predefined line/marker/color specifications. In summary, its purpose is to simplify the generation of high-quality function and/or data plots, and solving the problems of
\begin{itemize}
	\item consistency of document and font type and font size,
	\item direct use of \TeX\ math mode in axis descriptions,
	\item consistency of data and figures (no third party tool necessary),
	\item inter document consistency using preamble configurations and styles.
\end{itemize}
Although not necessary, separate |.pdf| or |.eps| graphics can be generated using the |external| library developed as part of \Tikz.

\PGFPlots\ is build completely on \Tikz/\PGF. Knowledge of \Tikz\ will simplify the work with \PGFPlots, although it is not required.

\section[About PGFPlots: Preliminaries]{About {\normalfont\PGFPlots}: Preliminaries}
This section contains information about upgrades, the team, the installation (in case you need to do it manually) and troubleshooting. You may skip it completely except for the upgrade remarks.

However, note that this library requires at least \PGF\ version $2.00$. At the time of this writing, many \TeX-distributions still contain the older \PGF\ version $1.18$, so it may be necessary to install a recent \PGF\ prior to using \PGFPlots.

\subsection{Components}
\PGFPlots\ comes with two components:
\begin{enumerate}
	\item the plotting component (which you are currently reading) and
	\item the \PGFPlotstable\ component which simplifies number formatting and postprocessing of numerical tables. It comes as a separate package and has its own manual \href{file:pgfplotstable.pdf}{pgfplotstable.pdf}.
\end{enumerate}

\subsection{Upgrade remarks}
This release provides a lot of improvements which can be found in all detail in \texttt{ChangeLog} for interested readers. However, some attention is useful with respect to the following changes.

\subsubsection{New Optional Features}
\begin{enumerate}
	\item \PGFPlots\ 1.3 comes with user interface improvements. The technical distinction between ``behavior options'' and ``style options'' of older versions is no longer necessary (although still fully supported).

	\item \PGFPlots\ 1.3 has a new feature which allows to \emph{move axis labels tight to tick labels} automatically.

	Since this affects the spacing, it is not enabled be default.

	Use
\begin{codeexample}[code only]
\usepackage{pgfplots}
\pgfplotsset{compat=1.3}
\end{codeexample}
	in your preamble to benefit from the improved spacing\footnote{The |compat=1.3| setting is necessary for new features which move axis labels around like reversed axis. However, \PGFPlots\ will still work as it does in earlier versions, your documents will remain the same if you don't set it explicitly.}.
	Take a look at the next page for the precise definition of |compat|.

	\item \PGFPlots\ 1.3 now supports reversed axes. It is no longer necessary to use work--arounds with negative units.
\pgfkeys{/pdflinks/search key prefixes in/.add={/pgfplots/,}{}}

	Take a look at the |x dir=reverse| key.

	Existing work--arounds will still function properly. Use |\pgfplotsset{compat=1.3}| together with |x dir=reverse| to switch to the new version.
\end{enumerate}

\subsubsection{Old Features Which May Need Attention}
\begin{enumerate}
	\item The |scatter/classes| feature produces proper legends as of version 1.3. This may change the appearance of existing legends of plots with |scatter/classes|.

	\item Due to a small math bug in \PGF\ $2.00$, you \emph{can't} use the math expression `|-x^2|'. It is necessary to use `|0-x^2|' instead. The same holds for
		`|exp(-x^2)|'; use `|exp(0-x^2)|' instead.

		This will be fixed with \PGF\ $>2.00$.

	\item Starting with \PGFPlots\ $1.1$, |\tikzstyle| should \emph{no longer be used} to set \PGFPlots\ options.
	
	Although |\tikzstyle| is still supported for some older \PGFPlots\ options, you should replace any occurance of |\tikzstyle| with |\pgfplotsset{|\meta{style name}|/.style={|\meta{key-value-list}|}}| or the associated |/.append style| variant. See section~\ref{sec:styles} for more detail.
\end{enumerate}
I apologize for any inconvenience caused by these changes.

\begin{pgfplotskey}{compat=\mchoice{1.3,pre 1.3,default,newest} (initially default)}
	The preamble configuration 
\begin{codeexample}[code only]
\usepackage{pgfplots}
\pgfplotsset{compat=1.3}
\end{codeexample}
	allows to choose between backwards compatibility and most recent features.

	\PGFPlots\ version 1.3 comes with new features which may lead to different spacings compared to earlier versions:
	\begin{enumerate}
		\item Version 1.3 comes with the |xlabel near ticks| feature which places (in this case $x$) axis labels ``near'' the tick labels, taking the size of tick labels into account.
		
		Older versions used |xlabel absolute|, i.e.\ an absolute distance between the axis and axis labels (now matter how large or small tick labels are).

		The initial setting \emph{keeps backwards compatibility}.

		You are encouraged to use
\begin{codeexample}[code only]
\usepackage{pgfplots}
\pgfplotsset{compat=1.3}
\end{codeexample}
		in your preamble.
		
		\item Version 1.3 fixes a bug with |anchor=south west| which occurred mainly for reversed axes: the anchors where upside--down. This is fixed now.

		
		If you have written some work--around and want to keep it going, use

		|\pgfplotsset{compat/anchors=pre 1.3}|\pgfmanualpdflabel{/pgfplots/compat/anchors}{} 

		to disable the bug fix.
	\end{enumerate}

	Use |\pgfplotsset{compat=default}| to restore the factory settings.

	The setting |\pgfplotsset{compat=newest}| will always use features of the most recent version. This might result in small changes in the document's appearance.
\end{pgfplotskey}

\subsection{The Team}
\PGFPlots\ has been written mainly by Christian Feuers�nger with many improvements of Pascal Wolkotte and Nick Papior Andersen as a spare time project. We hope it is useful and provides valuable plots.

If you are interested in writing something but don't know how, consider reading the auxiliary manual \href{file:TeX-programming-notes.pdf}{TeX-programming-notes.pdf} which comes with \PGFPlots. It is far from complete, but maybe it is a good starting point (at least for more literature).

\subsection{Acknowledgements}
I thank God for all hours of enjoyed programming. I thank Pascal Wolkotte and Nick Papior Andersen for their programming efforts and contributions as part of the development team. I thank Stefan Tibus, who contributed the |plot shell| feature. I thank Tom Cashman for the contribution of the |reverse legend| feature. Special thanks go to Stefan Pinnow whose tests of \PGFPlots\ lead to numerous quality improvements. Furthermore, I thank Dr.~Schweitzer for many fruitful discussions and Dr.~Meine for his ideas and suggestions. Thanks as well to the many international contributors who provided feature requests or identified bugs or simply manual improvements!

Last but not least, I thank Till Tantau and Mark Wibrow for their excellent graphics (and more) package \PGF\ and \Tikz\ which is the base of \PGFPlots.

\include{pgfplots.install}%
\include{pgfplots.intro}%
\include{pgfplots.reference}%
\include{pgfplots.libs}%
\include{pgfplots.resources}%
\include{pgfplots.importexport}%
\include{pgfplots.basic.reference}%

\printindex

\bibliographystyle{abbrv} %gerapali} %gerabbrv} %gerunsrt.bst} %gerabbrv}% gerplain}
\nocite{pgfplotstable}
\nocite{programmingnotes}
\bibliography{pgfplots}
\end{document}
%
%%%%%%%%%%%%%%%%%%%%%%%%%%%%%%%%%%%%%%%%%%%%%%%%%%%%%%%%%%%%%%%%%%%%%%%%%%%%%
%
% Package pgfplots.sty documentation. 
%
% Copyright 2007/2008 by Christian Feuersaenger.
%
% This program is free software: you can redistribute it and/or modify
% it under the terms of the GNU General Public License as published by
% the Free Software Foundation, either version 3 of the License, or
% (at your option) any later version.
% 
% This program is distributed in the hope that it will be useful,
% but WITHOUT ANY WARRANTY; without even the implied warranty of
% MERCHANTABILITY or FITNESS FOR A PARTICULAR PURPOSE.  See the
% GNU General Public License for more details.
% 
% You should have received a copy of the GNU General Public License
% along with this program.  If not, see <http://www.gnu.org/licenses/>.
%
%
%%%%%%%%%%%%%%%%%%%%%%%%%%%%%%%%%%%%%%%%%%%%%%%%%%%%%%%%%%%%%%%%%%%%%%%%%%%%%
\input{pgfplots.preamble.tex}
%\RequirePackage[german,english,francais]{babel}

\pgfplotsmanualexternalexpensivetrue

%\usetikzlibrary{external}
%\tikzexternalize[prefix=figures/]{pgfplots}

\title{%
	Manual for Package \PGFPlots\\
	{\small Version \pgfplotsversion}\\
	{\small\href{http://sourceforge.net/projects/pgfplots}{http://sourceforge.net/projects/pgfplots}}}

%\includeonly{pgfplots.libs}


\begin{document}

\def\plotcoords{%
\addplot coordinates {
(5,8.312e-02)    (17,2.547e-02)   (49,7.407e-03)
(129,2.102e-03)  (321,5.874e-04)  (769,1.623e-04)
(1793,4.442e-05) (4097,1.207e-05) (9217,3.261e-06)
};

\addplot coordinates{
(7,8.472e-02)    (31,3.044e-02)    (111,1.022e-02)
(351,3.303e-03)  (1023,1.039e-03)  (2815,3.196e-04)
(7423,9.658e-05) (18943,2.873e-05) (47103,8.437e-06)
};

\addplot coordinates{
(9,7.881e-02)     (49,3.243e-02)    (209,1.232e-02)
(769,4.454e-03)   (2561,1.551e-03)  (7937,5.236e-04)
(23297,1.723e-04) (65537,5.545e-05) (178177,1.751e-05)
};

\addplot coordinates{
(11,6.887e-02)    (71,3.177e-02)     (351,1.341e-02)
(1471,5.334e-03)  (5503,2.027e-03)   (18943,7.415e-04)
(61183,2.628e-04) (187903,9.063e-05) (553983,3.053e-05)
};

\addplot coordinates{
(13,5.755e-02)     (97,2.925e-02)     (545,1.351e-02)
(2561,5.842e-03)   (10625,2.397e-03)  (40193,9.414e-04)
(141569,3.564e-04) (471041,1.308e-04) 
(1496065,4.670e-05)
};
}%
\maketitle
\begin{abstract}%
\PGFPlots\ draws high--quality function plots in normal or logarithmic scaling with a user-friendly interface directly in \TeX. The user supplies axis labels, legend entries and the plot coordinates for one or more plots and \PGFPlots\ applies axis scaling, computes any logarithms and axis ticks and draws the plots, supporting line plots, scatter plots, piecewise constant plots, bar plots, area plots, mesh-- and surface plots and some more. It is based on Till Tantau's package \PGF/\Tikz.
\end{abstract}
\tableofcontents
\section{Introduction}
This package provides tools to generate plots and labeled axes easily. It draws normal plots, logplots and semi-logplots, in two and three dimensions. Axis ticks, labels, legends (in case of multiple plots) can be added with key-value options. It can cycle through a set of predefined line/marker/color specifications. In summary, its purpose is to simplify the generation of high-quality function and/or data plots, and solving the problems of
\begin{itemize}
	\item consistency of document and font type and font size,
	\item direct use of \TeX\ math mode in axis descriptions,
	\item consistency of data and figures (no third party tool necessary),
	\item inter document consistency using preamble configurations and styles.
\end{itemize}
Although not necessary, separate |.pdf| or |.eps| graphics can be generated using the |external| library developed as part of \Tikz.

\PGFPlots\ is build completely on \Tikz/\PGF. Knowledge of \Tikz\ will simplify the work with \PGFPlots, although it is not required.

\section[About PGFPlots: Preliminaries]{About {\normalfont\PGFPlots}: Preliminaries}
This section contains information about upgrades, the team, the installation (in case you need to do it manually) and troubleshooting. You may skip it completely except for the upgrade remarks.

However, note that this library requires at least \PGF\ version $2.00$. At the time of this writing, many \TeX-distributions still contain the older \PGF\ version $1.18$, so it may be necessary to install a recent \PGF\ prior to using \PGFPlots.

\subsection{Components}
\PGFPlots\ comes with two components:
\begin{enumerate}
	\item the plotting component (which you are currently reading) and
	\item the \PGFPlotstable\ component which simplifies number formatting and postprocessing of numerical tables. It comes as a separate package and has its own manual \href{file:pgfplotstable.pdf}{pgfplotstable.pdf}.
\end{enumerate}

\subsection{Upgrade remarks}
This release provides a lot of improvements which can be found in all detail in \texttt{ChangeLog} for interested readers. However, some attention is useful with respect to the following changes.

\subsubsection{New Optional Features}
\begin{enumerate}
	\item \PGFPlots\ 1.3 comes with user interface improvements. The technical distinction between ``behavior options'' and ``style options'' of older versions is no longer necessary (although still fully supported).

	\item \PGFPlots\ 1.3 has a new feature which allows to \emph{move axis labels tight to tick labels} automatically.

	Since this affects the spacing, it is not enabled be default.

	Use
\begin{codeexample}[code only]
\usepackage{pgfplots}
\pgfplotsset{compat=1.3}
\end{codeexample}
	in your preamble to benefit from the improved spacing\footnote{The |compat=1.3| setting is necessary for new features which move axis labels around like reversed axis. However, \PGFPlots\ will still work as it does in earlier versions, your documents will remain the same if you don't set it explicitly.}.
	Take a look at the next page for the precise definition of |compat|.

	\item \PGFPlots\ 1.3 now supports reversed axes. It is no longer necessary to use work--arounds with negative units.
\pgfkeys{/pdflinks/search key prefixes in/.add={/pgfplots/,}{}}

	Take a look at the |x dir=reverse| key.

	Existing work--arounds will still function properly. Use |\pgfplotsset{compat=1.3}| together with |x dir=reverse| to switch to the new version.
\end{enumerate}

\subsubsection{Old Features Which May Need Attention}
\begin{enumerate}
	\item The |scatter/classes| feature produces proper legends as of version 1.3. This may change the appearance of existing legends of plots with |scatter/classes|.

	\item Due to a small math bug in \PGF\ $2.00$, you \emph{can't} use the math expression `|-x^2|'. It is necessary to use `|0-x^2|' instead. The same holds for
		`|exp(-x^2)|'; use `|exp(0-x^2)|' instead.

		This will be fixed with \PGF\ $>2.00$.

	\item Starting with \PGFPlots\ $1.1$, |\tikzstyle| should \emph{no longer be used} to set \PGFPlots\ options.
	
	Although |\tikzstyle| is still supported for some older \PGFPlots\ options, you should replace any occurance of |\tikzstyle| with |\pgfplotsset{|\meta{style name}|/.style={|\meta{key-value-list}|}}| or the associated |/.append style| variant. See section~\ref{sec:styles} for more detail.
\end{enumerate}
I apologize for any inconvenience caused by these changes.

\begin{pgfplotskey}{compat=\mchoice{1.3,pre 1.3,default,newest} (initially default)}
	The preamble configuration 
\begin{codeexample}[code only]
\usepackage{pgfplots}
\pgfplotsset{compat=1.3}
\end{codeexample}
	allows to choose between backwards compatibility and most recent features.

	\PGFPlots\ version 1.3 comes with new features which may lead to different spacings compared to earlier versions:
	\begin{enumerate}
		\item Version 1.3 comes with the |xlabel near ticks| feature which places (in this case $x$) axis labels ``near'' the tick labels, taking the size of tick labels into account.
		
		Older versions used |xlabel absolute|, i.e.\ an absolute distance between the axis and axis labels (now matter how large or small tick labels are).

		The initial setting \emph{keeps backwards compatibility}.

		You are encouraged to use
\begin{codeexample}[code only]
\usepackage{pgfplots}
\pgfplotsset{compat=1.3}
\end{codeexample}
		in your preamble.
		
		\item Version 1.3 fixes a bug with |anchor=south west| which occurred mainly for reversed axes: the anchors where upside--down. This is fixed now.

		
		If you have written some work--around and want to keep it going, use

		|\pgfplotsset{compat/anchors=pre 1.3}|\pgfmanualpdflabel{/pgfplots/compat/anchors}{} 

		to disable the bug fix.
	\end{enumerate}

	Use |\pgfplotsset{compat=default}| to restore the factory settings.

	The setting |\pgfplotsset{compat=newest}| will always use features of the most recent version. This might result in small changes in the document's appearance.
\end{pgfplotskey}

\subsection{The Team}
\PGFPlots\ has been written mainly by Christian Feuers�nger with many improvements of Pascal Wolkotte and Nick Papior Andersen as a spare time project. We hope it is useful and provides valuable plots.

If you are interested in writing something but don't know how, consider reading the auxiliary manual \href{file:TeX-programming-notes.pdf}{TeX-programming-notes.pdf} which comes with \PGFPlots. It is far from complete, but maybe it is a good starting point (at least for more literature).

\subsection{Acknowledgements}
I thank God for all hours of enjoyed programming. I thank Pascal Wolkotte and Nick Papior Andersen for their programming efforts and contributions as part of the development team. I thank Stefan Tibus, who contributed the |plot shell| feature. I thank Tom Cashman for the contribution of the |reverse legend| feature. Special thanks go to Stefan Pinnow whose tests of \PGFPlots\ lead to numerous quality improvements. Furthermore, I thank Dr.~Schweitzer for many fruitful discussions and Dr.~Meine for his ideas and suggestions. Thanks as well to the many international contributors who provided feature requests or identified bugs or simply manual improvements!

Last but not least, I thank Till Tantau and Mark Wibrow for their excellent graphics (and more) package \PGF\ and \Tikz\ which is the base of \PGFPlots.

\include{pgfplots.install}%
\include{pgfplots.intro}%
\include{pgfplots.reference}%
\include{pgfplots.libs}%
\include{pgfplots.resources}%
\include{pgfplots.importexport}%
\include{pgfplots.basic.reference}%

\printindex

\bibliographystyle{abbrv} %gerapali} %gerabbrv} %gerunsrt.bst} %gerabbrv}% gerplain}
\nocite{pgfplotstable}
\nocite{programmingnotes}
\bibliography{pgfplots}
\end{document}
%
%%%%%%%%%%%%%%%%%%%%%%%%%%%%%%%%%%%%%%%%%%%%%%%%%%%%%%%%%%%%%%%%%%%%%%%%%%%%%
%
% Package pgfplots.sty documentation. 
%
% Copyright 2007/2008 by Christian Feuersaenger.
%
% This program is free software: you can redistribute it and/or modify
% it under the terms of the GNU General Public License as published by
% the Free Software Foundation, either version 3 of the License, or
% (at your option) any later version.
% 
% This program is distributed in the hope that it will be useful,
% but WITHOUT ANY WARRANTY; without even the implied warranty of
% MERCHANTABILITY or FITNESS FOR A PARTICULAR PURPOSE.  See the
% GNU General Public License for more details.
% 
% You should have received a copy of the GNU General Public License
% along with this program.  If not, see <http://www.gnu.org/licenses/>.
%
%
%%%%%%%%%%%%%%%%%%%%%%%%%%%%%%%%%%%%%%%%%%%%%%%%%%%%%%%%%%%%%%%%%%%%%%%%%%%%%
\input{pgfplots.preamble.tex}
%\RequirePackage[german,english,francais]{babel}

\pgfplotsmanualexternalexpensivetrue

%\usetikzlibrary{external}
%\tikzexternalize[prefix=figures/]{pgfplots}

\title{%
	Manual for Package \PGFPlots\\
	{\small Version \pgfplotsversion}\\
	{\small\href{http://sourceforge.net/projects/pgfplots}{http://sourceforge.net/projects/pgfplots}}}

%\includeonly{pgfplots.libs}


\begin{document}

\def\plotcoords{%
\addplot coordinates {
(5,8.312e-02)    (17,2.547e-02)   (49,7.407e-03)
(129,2.102e-03)  (321,5.874e-04)  (769,1.623e-04)
(1793,4.442e-05) (4097,1.207e-05) (9217,3.261e-06)
};

\addplot coordinates{
(7,8.472e-02)    (31,3.044e-02)    (111,1.022e-02)
(351,3.303e-03)  (1023,1.039e-03)  (2815,3.196e-04)
(7423,9.658e-05) (18943,2.873e-05) (47103,8.437e-06)
};

\addplot coordinates{
(9,7.881e-02)     (49,3.243e-02)    (209,1.232e-02)
(769,4.454e-03)   (2561,1.551e-03)  (7937,5.236e-04)
(23297,1.723e-04) (65537,5.545e-05) (178177,1.751e-05)
};

\addplot coordinates{
(11,6.887e-02)    (71,3.177e-02)     (351,1.341e-02)
(1471,5.334e-03)  (5503,2.027e-03)   (18943,7.415e-04)
(61183,2.628e-04) (187903,9.063e-05) (553983,3.053e-05)
};

\addplot coordinates{
(13,5.755e-02)     (97,2.925e-02)     (545,1.351e-02)
(2561,5.842e-03)   (10625,2.397e-03)  (40193,9.414e-04)
(141569,3.564e-04) (471041,1.308e-04) 
(1496065,4.670e-05)
};
}%
\maketitle
\begin{abstract}%
\PGFPlots\ draws high--quality function plots in normal or logarithmic scaling with a user-friendly interface directly in \TeX. The user supplies axis labels, legend entries and the plot coordinates for one or more plots and \PGFPlots\ applies axis scaling, computes any logarithms and axis ticks and draws the plots, supporting line plots, scatter plots, piecewise constant plots, bar plots, area plots, mesh-- and surface plots and some more. It is based on Till Tantau's package \PGF/\Tikz.
\end{abstract}
\tableofcontents
\section{Introduction}
This package provides tools to generate plots and labeled axes easily. It draws normal plots, logplots and semi-logplots, in two and three dimensions. Axis ticks, labels, legends (in case of multiple plots) can be added with key-value options. It can cycle through a set of predefined line/marker/color specifications. In summary, its purpose is to simplify the generation of high-quality function and/or data plots, and solving the problems of
\begin{itemize}
	\item consistency of document and font type and font size,
	\item direct use of \TeX\ math mode in axis descriptions,
	\item consistency of data and figures (no third party tool necessary),
	\item inter document consistency using preamble configurations and styles.
\end{itemize}
Although not necessary, separate |.pdf| or |.eps| graphics can be generated using the |external| library developed as part of \Tikz.

\PGFPlots\ is build completely on \Tikz/\PGF. Knowledge of \Tikz\ will simplify the work with \PGFPlots, although it is not required.

\section[About PGFPlots: Preliminaries]{About {\normalfont\PGFPlots}: Preliminaries}
This section contains information about upgrades, the team, the installation (in case you need to do it manually) and troubleshooting. You may skip it completely except for the upgrade remarks.

However, note that this library requires at least \PGF\ version $2.00$. At the time of this writing, many \TeX-distributions still contain the older \PGF\ version $1.18$, so it may be necessary to install a recent \PGF\ prior to using \PGFPlots.

\subsection{Components}
\PGFPlots\ comes with two components:
\begin{enumerate}
	\item the plotting component (which you are currently reading) and
	\item the \PGFPlotstable\ component which simplifies number formatting and postprocessing of numerical tables. It comes as a separate package and has its own manual \href{file:pgfplotstable.pdf}{pgfplotstable.pdf}.
\end{enumerate}

\subsection{Upgrade remarks}
This release provides a lot of improvements which can be found in all detail in \texttt{ChangeLog} for interested readers. However, some attention is useful with respect to the following changes.

\subsubsection{New Optional Features}
\begin{enumerate}
	\item \PGFPlots\ 1.3 comes with user interface improvements. The technical distinction between ``behavior options'' and ``style options'' of older versions is no longer necessary (although still fully supported).

	\item \PGFPlots\ 1.3 has a new feature which allows to \emph{move axis labels tight to tick labels} automatically.

	Since this affects the spacing, it is not enabled be default.

	Use
\begin{codeexample}[code only]
\usepackage{pgfplots}
\pgfplotsset{compat=1.3}
\end{codeexample}
	in your preamble to benefit from the improved spacing\footnote{The |compat=1.3| setting is necessary for new features which move axis labels around like reversed axis. However, \PGFPlots\ will still work as it does in earlier versions, your documents will remain the same if you don't set it explicitly.}.
	Take a look at the next page for the precise definition of |compat|.

	\item \PGFPlots\ 1.3 now supports reversed axes. It is no longer necessary to use work--arounds with negative units.
\pgfkeys{/pdflinks/search key prefixes in/.add={/pgfplots/,}{}}

	Take a look at the |x dir=reverse| key.

	Existing work--arounds will still function properly. Use |\pgfplotsset{compat=1.3}| together with |x dir=reverse| to switch to the new version.
\end{enumerate}

\subsubsection{Old Features Which May Need Attention}
\begin{enumerate}
	\item The |scatter/classes| feature produces proper legends as of version 1.3. This may change the appearance of existing legends of plots with |scatter/classes|.

	\item Due to a small math bug in \PGF\ $2.00$, you \emph{can't} use the math expression `|-x^2|'. It is necessary to use `|0-x^2|' instead. The same holds for
		`|exp(-x^2)|'; use `|exp(0-x^2)|' instead.

		This will be fixed with \PGF\ $>2.00$.

	\item Starting with \PGFPlots\ $1.1$, |\tikzstyle| should \emph{no longer be used} to set \PGFPlots\ options.
	
	Although |\tikzstyle| is still supported for some older \PGFPlots\ options, you should replace any occurance of |\tikzstyle| with |\pgfplotsset{|\meta{style name}|/.style={|\meta{key-value-list}|}}| or the associated |/.append style| variant. See section~\ref{sec:styles} for more detail.
\end{enumerate}
I apologize for any inconvenience caused by these changes.

\begin{pgfplotskey}{compat=\mchoice{1.3,pre 1.3,default,newest} (initially default)}
	The preamble configuration 
\begin{codeexample}[code only]
\usepackage{pgfplots}
\pgfplotsset{compat=1.3}
\end{codeexample}
	allows to choose between backwards compatibility and most recent features.

	\PGFPlots\ version 1.3 comes with new features which may lead to different spacings compared to earlier versions:
	\begin{enumerate}
		\item Version 1.3 comes with the |xlabel near ticks| feature which places (in this case $x$) axis labels ``near'' the tick labels, taking the size of tick labels into account.
		
		Older versions used |xlabel absolute|, i.e.\ an absolute distance between the axis and axis labels (now matter how large or small tick labels are).

		The initial setting \emph{keeps backwards compatibility}.

		You are encouraged to use
\begin{codeexample}[code only]
\usepackage{pgfplots}
\pgfplotsset{compat=1.3}
\end{codeexample}
		in your preamble.
		
		\item Version 1.3 fixes a bug with |anchor=south west| which occurred mainly for reversed axes: the anchors where upside--down. This is fixed now.

		
		If you have written some work--around and want to keep it going, use

		|\pgfplotsset{compat/anchors=pre 1.3}|\pgfmanualpdflabel{/pgfplots/compat/anchors}{} 

		to disable the bug fix.
	\end{enumerate}

	Use |\pgfplotsset{compat=default}| to restore the factory settings.

	The setting |\pgfplotsset{compat=newest}| will always use features of the most recent version. This might result in small changes in the document's appearance.
\end{pgfplotskey}

\subsection{The Team}
\PGFPlots\ has been written mainly by Christian Feuers�nger with many improvements of Pascal Wolkotte and Nick Papior Andersen as a spare time project. We hope it is useful and provides valuable plots.

If you are interested in writing something but don't know how, consider reading the auxiliary manual \href{file:TeX-programming-notes.pdf}{TeX-programming-notes.pdf} which comes with \PGFPlots. It is far from complete, but maybe it is a good starting point (at least for more literature).

\subsection{Acknowledgements}
I thank God for all hours of enjoyed programming. I thank Pascal Wolkotte and Nick Papior Andersen for their programming efforts and contributions as part of the development team. I thank Stefan Tibus, who contributed the |plot shell| feature. I thank Tom Cashman for the contribution of the |reverse legend| feature. Special thanks go to Stefan Pinnow whose tests of \PGFPlots\ lead to numerous quality improvements. Furthermore, I thank Dr.~Schweitzer for many fruitful discussions and Dr.~Meine for his ideas and suggestions. Thanks as well to the many international contributors who provided feature requests or identified bugs or simply manual improvements!

Last but not least, I thank Till Tantau and Mark Wibrow for their excellent graphics (and more) package \PGF\ and \Tikz\ which is the base of \PGFPlots.

\include{pgfplots.install}%
\include{pgfplots.intro}%
\include{pgfplots.reference}%
\include{pgfplots.libs}%
\include{pgfplots.resources}%
\include{pgfplots.importexport}%
\include{pgfplots.basic.reference}%

\printindex

\bibliographystyle{abbrv} %gerapali} %gerabbrv} %gerunsrt.bst} %gerabbrv}% gerplain}
\nocite{pgfplotstable}
\nocite{programmingnotes}
\bibliography{pgfplots}
\end{document}
%
%%%%%%%%%%%%%%%%%%%%%%%%%%%%%%%%%%%%%%%%%%%%%%%%%%%%%%%%%%%%%%%%%%%%%%%%%%%%%
%
% Package pgfplots.sty documentation. 
%
% Copyright 2007/2008 by Christian Feuersaenger.
%
% This program is free software: you can redistribute it and/or modify
% it under the terms of the GNU General Public License as published by
% the Free Software Foundation, either version 3 of the License, or
% (at your option) any later version.
% 
% This program is distributed in the hope that it will be useful,
% but WITHOUT ANY WARRANTY; without even the implied warranty of
% MERCHANTABILITY or FITNESS FOR A PARTICULAR PURPOSE.  See the
% GNU General Public License for more details.
% 
% You should have received a copy of the GNU General Public License
% along with this program.  If not, see <http://www.gnu.org/licenses/>.
%
%
%%%%%%%%%%%%%%%%%%%%%%%%%%%%%%%%%%%%%%%%%%%%%%%%%%%%%%%%%%%%%%%%%%%%%%%%%%%%%
\input{pgfplots.preamble.tex}
%\RequirePackage[german,english,francais]{babel}

\pgfplotsmanualexternalexpensivetrue

%\usetikzlibrary{external}
%\tikzexternalize[prefix=figures/]{pgfplots}

\title{%
	Manual for Package \PGFPlots\\
	{\small Version \pgfplotsversion}\\
	{\small\href{http://sourceforge.net/projects/pgfplots}{http://sourceforge.net/projects/pgfplots}}}

%\includeonly{pgfplots.libs}


\begin{document}

\def\plotcoords{%
\addplot coordinates {
(5,8.312e-02)    (17,2.547e-02)   (49,7.407e-03)
(129,2.102e-03)  (321,5.874e-04)  (769,1.623e-04)
(1793,4.442e-05) (4097,1.207e-05) (9217,3.261e-06)
};

\addplot coordinates{
(7,8.472e-02)    (31,3.044e-02)    (111,1.022e-02)
(351,3.303e-03)  (1023,1.039e-03)  (2815,3.196e-04)
(7423,9.658e-05) (18943,2.873e-05) (47103,8.437e-06)
};

\addplot coordinates{
(9,7.881e-02)     (49,3.243e-02)    (209,1.232e-02)
(769,4.454e-03)   (2561,1.551e-03)  (7937,5.236e-04)
(23297,1.723e-04) (65537,5.545e-05) (178177,1.751e-05)
};

\addplot coordinates{
(11,6.887e-02)    (71,3.177e-02)     (351,1.341e-02)
(1471,5.334e-03)  (5503,2.027e-03)   (18943,7.415e-04)
(61183,2.628e-04) (187903,9.063e-05) (553983,3.053e-05)
};

\addplot coordinates{
(13,5.755e-02)     (97,2.925e-02)     (545,1.351e-02)
(2561,5.842e-03)   (10625,2.397e-03)  (40193,9.414e-04)
(141569,3.564e-04) (471041,1.308e-04) 
(1496065,4.670e-05)
};
}%
\maketitle
\begin{abstract}%
\PGFPlots\ draws high--quality function plots in normal or logarithmic scaling with a user-friendly interface directly in \TeX. The user supplies axis labels, legend entries and the plot coordinates for one or more plots and \PGFPlots\ applies axis scaling, computes any logarithms and axis ticks and draws the plots, supporting line plots, scatter plots, piecewise constant plots, bar plots, area plots, mesh-- and surface plots and some more. It is based on Till Tantau's package \PGF/\Tikz.
\end{abstract}
\tableofcontents
\section{Introduction}
This package provides tools to generate plots and labeled axes easily. It draws normal plots, logplots and semi-logplots, in two and three dimensions. Axis ticks, labels, legends (in case of multiple plots) can be added with key-value options. It can cycle through a set of predefined line/marker/color specifications. In summary, its purpose is to simplify the generation of high-quality function and/or data plots, and solving the problems of
\begin{itemize}
	\item consistency of document and font type and font size,
	\item direct use of \TeX\ math mode in axis descriptions,
	\item consistency of data and figures (no third party tool necessary),
	\item inter document consistency using preamble configurations and styles.
\end{itemize}
Although not necessary, separate |.pdf| or |.eps| graphics can be generated using the |external| library developed as part of \Tikz.

\PGFPlots\ is build completely on \Tikz/\PGF. Knowledge of \Tikz\ will simplify the work with \PGFPlots, although it is not required.

\section[About PGFPlots: Preliminaries]{About {\normalfont\PGFPlots}: Preliminaries}
This section contains information about upgrades, the team, the installation (in case you need to do it manually) and troubleshooting. You may skip it completely except for the upgrade remarks.

However, note that this library requires at least \PGF\ version $2.00$. At the time of this writing, many \TeX-distributions still contain the older \PGF\ version $1.18$, so it may be necessary to install a recent \PGF\ prior to using \PGFPlots.

\subsection{Components}
\PGFPlots\ comes with two components:
\begin{enumerate}
	\item the plotting component (which you are currently reading) and
	\item the \PGFPlotstable\ component which simplifies number formatting and postprocessing of numerical tables. It comes as a separate package and has its own manual \href{file:pgfplotstable.pdf}{pgfplotstable.pdf}.
\end{enumerate}

\subsection{Upgrade remarks}
This release provides a lot of improvements which can be found in all detail in \texttt{ChangeLog} for interested readers. However, some attention is useful with respect to the following changes.

\subsubsection{New Optional Features}
\begin{enumerate}
	\item \PGFPlots\ 1.3 comes with user interface improvements. The technical distinction between ``behavior options'' and ``style options'' of older versions is no longer necessary (although still fully supported).

	\item \PGFPlots\ 1.3 has a new feature which allows to \emph{move axis labels tight to tick labels} automatically.

	Since this affects the spacing, it is not enabled be default.

	Use
\begin{codeexample}[code only]
\usepackage{pgfplots}
\pgfplotsset{compat=1.3}
\end{codeexample}
	in your preamble to benefit from the improved spacing\footnote{The |compat=1.3| setting is necessary for new features which move axis labels around like reversed axis. However, \PGFPlots\ will still work as it does in earlier versions, your documents will remain the same if you don't set it explicitly.}.
	Take a look at the next page for the precise definition of |compat|.

	\item \PGFPlots\ 1.3 now supports reversed axes. It is no longer necessary to use work--arounds with negative units.
\pgfkeys{/pdflinks/search key prefixes in/.add={/pgfplots/,}{}}

	Take a look at the |x dir=reverse| key.

	Existing work--arounds will still function properly. Use |\pgfplotsset{compat=1.3}| together with |x dir=reverse| to switch to the new version.
\end{enumerate}

\subsubsection{Old Features Which May Need Attention}
\begin{enumerate}
	\item The |scatter/classes| feature produces proper legends as of version 1.3. This may change the appearance of existing legends of plots with |scatter/classes|.

	\item Due to a small math bug in \PGF\ $2.00$, you \emph{can't} use the math expression `|-x^2|'. It is necessary to use `|0-x^2|' instead. The same holds for
		`|exp(-x^2)|'; use `|exp(0-x^2)|' instead.

		This will be fixed with \PGF\ $>2.00$.

	\item Starting with \PGFPlots\ $1.1$, |\tikzstyle| should \emph{no longer be used} to set \PGFPlots\ options.
	
	Although |\tikzstyle| is still supported for some older \PGFPlots\ options, you should replace any occurance of |\tikzstyle| with |\pgfplotsset{|\meta{style name}|/.style={|\meta{key-value-list}|}}| or the associated |/.append style| variant. See section~\ref{sec:styles} for more detail.
\end{enumerate}
I apologize for any inconvenience caused by these changes.

\begin{pgfplotskey}{compat=\mchoice{1.3,pre 1.3,default,newest} (initially default)}
	The preamble configuration 
\begin{codeexample}[code only]
\usepackage{pgfplots}
\pgfplotsset{compat=1.3}
\end{codeexample}
	allows to choose between backwards compatibility and most recent features.

	\PGFPlots\ version 1.3 comes with new features which may lead to different spacings compared to earlier versions:
	\begin{enumerate}
		\item Version 1.3 comes with the |xlabel near ticks| feature which places (in this case $x$) axis labels ``near'' the tick labels, taking the size of tick labels into account.
		
		Older versions used |xlabel absolute|, i.e.\ an absolute distance between the axis and axis labels (now matter how large or small tick labels are).

		The initial setting \emph{keeps backwards compatibility}.

		You are encouraged to use
\begin{codeexample}[code only]
\usepackage{pgfplots}
\pgfplotsset{compat=1.3}
\end{codeexample}
		in your preamble.
		
		\item Version 1.3 fixes a bug with |anchor=south west| which occurred mainly for reversed axes: the anchors where upside--down. This is fixed now.

		
		If you have written some work--around and want to keep it going, use

		|\pgfplotsset{compat/anchors=pre 1.3}|\pgfmanualpdflabel{/pgfplots/compat/anchors}{} 

		to disable the bug fix.
	\end{enumerate}

	Use |\pgfplotsset{compat=default}| to restore the factory settings.

	The setting |\pgfplotsset{compat=newest}| will always use features of the most recent version. This might result in small changes in the document's appearance.
\end{pgfplotskey}

\subsection{The Team}
\PGFPlots\ has been written mainly by Christian Feuers�nger with many improvements of Pascal Wolkotte and Nick Papior Andersen as a spare time project. We hope it is useful and provides valuable plots.

If you are interested in writing something but don't know how, consider reading the auxiliary manual \href{file:TeX-programming-notes.pdf}{TeX-programming-notes.pdf} which comes with \PGFPlots. It is far from complete, but maybe it is a good starting point (at least for more literature).

\subsection{Acknowledgements}
I thank God for all hours of enjoyed programming. I thank Pascal Wolkotte and Nick Papior Andersen for their programming efforts and contributions as part of the development team. I thank Stefan Tibus, who contributed the |plot shell| feature. I thank Tom Cashman for the contribution of the |reverse legend| feature. Special thanks go to Stefan Pinnow whose tests of \PGFPlots\ lead to numerous quality improvements. Furthermore, I thank Dr.~Schweitzer for many fruitful discussions and Dr.~Meine for his ideas and suggestions. Thanks as well to the many international contributors who provided feature requests or identified bugs or simply manual improvements!

Last but not least, I thank Till Tantau and Mark Wibrow for their excellent graphics (and more) package \PGF\ and \Tikz\ which is the base of \PGFPlots.

\include{pgfplots.install}%
\include{pgfplots.intro}%
\include{pgfplots.reference}%
\include{pgfplots.libs}%
\include{pgfplots.resources}%
\include{pgfplots.importexport}%
\include{pgfplots.basic.reference}%

\printindex

\bibliographystyle{abbrv} %gerapali} %gerabbrv} %gerunsrt.bst} %gerabbrv}% gerplain}
\nocite{pgfplotstable}
\nocite{programmingnotes}
\bibliography{pgfplots}
\end{document}
%
%%%%%%%%%%%%%%%%%%%%%%%%%%%%%%%%%%%%%%%%%%%%%%%%%%%%%%%%%%%%%%%%%%%%%%%%%%%%%
%
% Package pgfplots.sty documentation. 
%
% Copyright 2007/2008 by Christian Feuersaenger.
%
% This program is free software: you can redistribute it and/or modify
% it under the terms of the GNU General Public License as published by
% the Free Software Foundation, either version 3 of the License, or
% (at your option) any later version.
% 
% This program is distributed in the hope that it will be useful,
% but WITHOUT ANY WARRANTY; without even the implied warranty of
% MERCHANTABILITY or FITNESS FOR A PARTICULAR PURPOSE.  See the
% GNU General Public License for more details.
% 
% You should have received a copy of the GNU General Public License
% along with this program.  If not, see <http://www.gnu.org/licenses/>.
%
%
%%%%%%%%%%%%%%%%%%%%%%%%%%%%%%%%%%%%%%%%%%%%%%%%%%%%%%%%%%%%%%%%%%%%%%%%%%%%%
\input{pgfplots.preamble.tex}
%\RequirePackage[german,english,francais]{babel}

\pgfplotsmanualexternalexpensivetrue

%\usetikzlibrary{external}
%\tikzexternalize[prefix=figures/]{pgfplots}

\title{%
	Manual for Package \PGFPlots\\
	{\small Version \pgfplotsversion}\\
	{\small\href{http://sourceforge.net/projects/pgfplots}{http://sourceforge.net/projects/pgfplots}}}

%\includeonly{pgfplots.libs}


\begin{document}

\def\plotcoords{%
\addplot coordinates {
(5,8.312e-02)    (17,2.547e-02)   (49,7.407e-03)
(129,2.102e-03)  (321,5.874e-04)  (769,1.623e-04)
(1793,4.442e-05) (4097,1.207e-05) (9217,3.261e-06)
};

\addplot coordinates{
(7,8.472e-02)    (31,3.044e-02)    (111,1.022e-02)
(351,3.303e-03)  (1023,1.039e-03)  (2815,3.196e-04)
(7423,9.658e-05) (18943,2.873e-05) (47103,8.437e-06)
};

\addplot coordinates{
(9,7.881e-02)     (49,3.243e-02)    (209,1.232e-02)
(769,4.454e-03)   (2561,1.551e-03)  (7937,5.236e-04)
(23297,1.723e-04) (65537,5.545e-05) (178177,1.751e-05)
};

\addplot coordinates{
(11,6.887e-02)    (71,3.177e-02)     (351,1.341e-02)
(1471,5.334e-03)  (5503,2.027e-03)   (18943,7.415e-04)
(61183,2.628e-04) (187903,9.063e-05) (553983,3.053e-05)
};

\addplot coordinates{
(13,5.755e-02)     (97,2.925e-02)     (545,1.351e-02)
(2561,5.842e-03)   (10625,2.397e-03)  (40193,9.414e-04)
(141569,3.564e-04) (471041,1.308e-04) 
(1496065,4.670e-05)
};
}%
\maketitle
\begin{abstract}%
\PGFPlots\ draws high--quality function plots in normal or logarithmic scaling with a user-friendly interface directly in \TeX. The user supplies axis labels, legend entries and the plot coordinates for one or more plots and \PGFPlots\ applies axis scaling, computes any logarithms and axis ticks and draws the plots, supporting line plots, scatter plots, piecewise constant plots, bar plots, area plots, mesh-- and surface plots and some more. It is based on Till Tantau's package \PGF/\Tikz.
\end{abstract}
\tableofcontents
\section{Introduction}
This package provides tools to generate plots and labeled axes easily. It draws normal plots, logplots and semi-logplots, in two and three dimensions. Axis ticks, labels, legends (in case of multiple plots) can be added with key-value options. It can cycle through a set of predefined line/marker/color specifications. In summary, its purpose is to simplify the generation of high-quality function and/or data plots, and solving the problems of
\begin{itemize}
	\item consistency of document and font type and font size,
	\item direct use of \TeX\ math mode in axis descriptions,
	\item consistency of data and figures (no third party tool necessary),
	\item inter document consistency using preamble configurations and styles.
\end{itemize}
Although not necessary, separate |.pdf| or |.eps| graphics can be generated using the |external| library developed as part of \Tikz.

\PGFPlots\ is build completely on \Tikz/\PGF. Knowledge of \Tikz\ will simplify the work with \PGFPlots, although it is not required.

\section[About PGFPlots: Preliminaries]{About {\normalfont\PGFPlots}: Preliminaries}
This section contains information about upgrades, the team, the installation (in case you need to do it manually) and troubleshooting. You may skip it completely except for the upgrade remarks.

However, note that this library requires at least \PGF\ version $2.00$. At the time of this writing, many \TeX-distributions still contain the older \PGF\ version $1.18$, so it may be necessary to install a recent \PGF\ prior to using \PGFPlots.

\subsection{Components}
\PGFPlots\ comes with two components:
\begin{enumerate}
	\item the plotting component (which you are currently reading) and
	\item the \PGFPlotstable\ component which simplifies number formatting and postprocessing of numerical tables. It comes as a separate package and has its own manual \href{file:pgfplotstable.pdf}{pgfplotstable.pdf}.
\end{enumerate}

\subsection{Upgrade remarks}
This release provides a lot of improvements which can be found in all detail in \texttt{ChangeLog} for interested readers. However, some attention is useful with respect to the following changes.

\subsubsection{New Optional Features}
\begin{enumerate}
	\item \PGFPlots\ 1.3 comes with user interface improvements. The technical distinction between ``behavior options'' and ``style options'' of older versions is no longer necessary (although still fully supported).

	\item \PGFPlots\ 1.3 has a new feature which allows to \emph{move axis labels tight to tick labels} automatically.

	Since this affects the spacing, it is not enabled be default.

	Use
\begin{codeexample}[code only]
\usepackage{pgfplots}
\pgfplotsset{compat=1.3}
\end{codeexample}
	in your preamble to benefit from the improved spacing\footnote{The |compat=1.3| setting is necessary for new features which move axis labels around like reversed axis. However, \PGFPlots\ will still work as it does in earlier versions, your documents will remain the same if you don't set it explicitly.}.
	Take a look at the next page for the precise definition of |compat|.

	\item \PGFPlots\ 1.3 now supports reversed axes. It is no longer necessary to use work--arounds with negative units.
\pgfkeys{/pdflinks/search key prefixes in/.add={/pgfplots/,}{}}

	Take a look at the |x dir=reverse| key.

	Existing work--arounds will still function properly. Use |\pgfplotsset{compat=1.3}| together with |x dir=reverse| to switch to the new version.
\end{enumerate}

\subsubsection{Old Features Which May Need Attention}
\begin{enumerate}
	\item The |scatter/classes| feature produces proper legends as of version 1.3. This may change the appearance of existing legends of plots with |scatter/classes|.

	\item Due to a small math bug in \PGF\ $2.00$, you \emph{can't} use the math expression `|-x^2|'. It is necessary to use `|0-x^2|' instead. The same holds for
		`|exp(-x^2)|'; use `|exp(0-x^2)|' instead.

		This will be fixed with \PGF\ $>2.00$.

	\item Starting with \PGFPlots\ $1.1$, |\tikzstyle| should \emph{no longer be used} to set \PGFPlots\ options.
	
	Although |\tikzstyle| is still supported for some older \PGFPlots\ options, you should replace any occurance of |\tikzstyle| with |\pgfplotsset{|\meta{style name}|/.style={|\meta{key-value-list}|}}| or the associated |/.append style| variant. See section~\ref{sec:styles} for more detail.
\end{enumerate}
I apologize for any inconvenience caused by these changes.

\begin{pgfplotskey}{compat=\mchoice{1.3,pre 1.3,default,newest} (initially default)}
	The preamble configuration 
\begin{codeexample}[code only]
\usepackage{pgfplots}
\pgfplotsset{compat=1.3}
\end{codeexample}
	allows to choose between backwards compatibility and most recent features.

	\PGFPlots\ version 1.3 comes with new features which may lead to different spacings compared to earlier versions:
	\begin{enumerate}
		\item Version 1.3 comes with the |xlabel near ticks| feature which places (in this case $x$) axis labels ``near'' the tick labels, taking the size of tick labels into account.
		
		Older versions used |xlabel absolute|, i.e.\ an absolute distance between the axis and axis labels (now matter how large or small tick labels are).

		The initial setting \emph{keeps backwards compatibility}.

		You are encouraged to use
\begin{codeexample}[code only]
\usepackage{pgfplots}
\pgfplotsset{compat=1.3}
\end{codeexample}
		in your preamble.
		
		\item Version 1.3 fixes a bug with |anchor=south west| which occurred mainly for reversed axes: the anchors where upside--down. This is fixed now.

		
		If you have written some work--around and want to keep it going, use

		|\pgfplotsset{compat/anchors=pre 1.3}|\pgfmanualpdflabel{/pgfplots/compat/anchors}{} 

		to disable the bug fix.
	\end{enumerate}

	Use |\pgfplotsset{compat=default}| to restore the factory settings.

	The setting |\pgfplotsset{compat=newest}| will always use features of the most recent version. This might result in small changes in the document's appearance.
\end{pgfplotskey}

\subsection{The Team}
\PGFPlots\ has been written mainly by Christian Feuers�nger with many improvements of Pascal Wolkotte and Nick Papior Andersen as a spare time project. We hope it is useful and provides valuable plots.

If you are interested in writing something but don't know how, consider reading the auxiliary manual \href{file:TeX-programming-notes.pdf}{TeX-programming-notes.pdf} which comes with \PGFPlots. It is far from complete, but maybe it is a good starting point (at least for more literature).

\subsection{Acknowledgements}
I thank God for all hours of enjoyed programming. I thank Pascal Wolkotte and Nick Papior Andersen for their programming efforts and contributions as part of the development team. I thank Stefan Tibus, who contributed the |plot shell| feature. I thank Tom Cashman for the contribution of the |reverse legend| feature. Special thanks go to Stefan Pinnow whose tests of \PGFPlots\ lead to numerous quality improvements. Furthermore, I thank Dr.~Schweitzer for many fruitful discussions and Dr.~Meine for his ideas and suggestions. Thanks as well to the many international contributors who provided feature requests or identified bugs or simply manual improvements!

Last but not least, I thank Till Tantau and Mark Wibrow for their excellent graphics (and more) package \PGF\ and \Tikz\ which is the base of \PGFPlots.

\include{pgfplots.install}%
\include{pgfplots.intro}%
\include{pgfplots.reference}%
\include{pgfplots.libs}%
\include{pgfplots.resources}%
\include{pgfplots.importexport}%
\include{pgfplots.basic.reference}%

\printindex

\bibliographystyle{abbrv} %gerapali} %gerabbrv} %gerunsrt.bst} %gerabbrv}% gerplain}
\nocite{pgfplotstable}
\nocite{programmingnotes}
\bibliography{pgfplots}
\end{document}
%
%%%%%%%%%%%%%%%%%%%%%%%%%%%%%%%%%%%%%%%%%%%%%%%%%%%%%%%%%%%%%%%%%%%%%%%%%%%%%
%
% Package pgfplots.sty documentation. 
%
% Copyright 2007/2008 by Christian Feuersaenger.
%
% This program is free software: you can redistribute it and/or modify
% it under the terms of the GNU General Public License as published by
% the Free Software Foundation, either version 3 of the License, or
% (at your option) any later version.
% 
% This program is distributed in the hope that it will be useful,
% but WITHOUT ANY WARRANTY; without even the implied warranty of
% MERCHANTABILITY or FITNESS FOR A PARTICULAR PURPOSE.  See the
% GNU General Public License for more details.
% 
% You should have received a copy of the GNU General Public License
% along with this program.  If not, see <http://www.gnu.org/licenses/>.
%
%
%%%%%%%%%%%%%%%%%%%%%%%%%%%%%%%%%%%%%%%%%%%%%%%%%%%%%%%%%%%%%%%%%%%%%%%%%%%%%
\input{pgfplots.preamble.tex}
%\RequirePackage[german,english,francais]{babel}

\pgfplotsmanualexternalexpensivetrue

%\usetikzlibrary{external}
%\tikzexternalize[prefix=figures/]{pgfplots}

\title{%
	Manual for Package \PGFPlots\\
	{\small Version \pgfplotsversion}\\
	{\small\href{http://sourceforge.net/projects/pgfplots}{http://sourceforge.net/projects/pgfplots}}}

%\includeonly{pgfplots.libs}


\begin{document}

\def\plotcoords{%
\addplot coordinates {
(5,8.312e-02)    (17,2.547e-02)   (49,7.407e-03)
(129,2.102e-03)  (321,5.874e-04)  (769,1.623e-04)
(1793,4.442e-05) (4097,1.207e-05) (9217,3.261e-06)
};

\addplot coordinates{
(7,8.472e-02)    (31,3.044e-02)    (111,1.022e-02)
(351,3.303e-03)  (1023,1.039e-03)  (2815,3.196e-04)
(7423,9.658e-05) (18943,2.873e-05) (47103,8.437e-06)
};

\addplot coordinates{
(9,7.881e-02)     (49,3.243e-02)    (209,1.232e-02)
(769,4.454e-03)   (2561,1.551e-03)  (7937,5.236e-04)
(23297,1.723e-04) (65537,5.545e-05) (178177,1.751e-05)
};

\addplot coordinates{
(11,6.887e-02)    (71,3.177e-02)     (351,1.341e-02)
(1471,5.334e-03)  (5503,2.027e-03)   (18943,7.415e-04)
(61183,2.628e-04) (187903,9.063e-05) (553983,3.053e-05)
};

\addplot coordinates{
(13,5.755e-02)     (97,2.925e-02)     (545,1.351e-02)
(2561,5.842e-03)   (10625,2.397e-03)  (40193,9.414e-04)
(141569,3.564e-04) (471041,1.308e-04) 
(1496065,4.670e-05)
};
}%
\maketitle
\begin{abstract}%
\PGFPlots\ draws high--quality function plots in normal or logarithmic scaling with a user-friendly interface directly in \TeX. The user supplies axis labels, legend entries and the plot coordinates for one or more plots and \PGFPlots\ applies axis scaling, computes any logarithms and axis ticks and draws the plots, supporting line plots, scatter plots, piecewise constant plots, bar plots, area plots, mesh-- and surface plots and some more. It is based on Till Tantau's package \PGF/\Tikz.
\end{abstract}
\tableofcontents
\section{Introduction}
This package provides tools to generate plots and labeled axes easily. It draws normal plots, logplots and semi-logplots, in two and three dimensions. Axis ticks, labels, legends (in case of multiple plots) can be added with key-value options. It can cycle through a set of predefined line/marker/color specifications. In summary, its purpose is to simplify the generation of high-quality function and/or data plots, and solving the problems of
\begin{itemize}
	\item consistency of document and font type and font size,
	\item direct use of \TeX\ math mode in axis descriptions,
	\item consistency of data and figures (no third party tool necessary),
	\item inter document consistency using preamble configurations and styles.
\end{itemize}
Although not necessary, separate |.pdf| or |.eps| graphics can be generated using the |external| library developed as part of \Tikz.

\PGFPlots\ is build completely on \Tikz/\PGF. Knowledge of \Tikz\ will simplify the work with \PGFPlots, although it is not required.

\section[About PGFPlots: Preliminaries]{About {\normalfont\PGFPlots}: Preliminaries}
This section contains information about upgrades, the team, the installation (in case you need to do it manually) and troubleshooting. You may skip it completely except for the upgrade remarks.

However, note that this library requires at least \PGF\ version $2.00$. At the time of this writing, many \TeX-distributions still contain the older \PGF\ version $1.18$, so it may be necessary to install a recent \PGF\ prior to using \PGFPlots.

\subsection{Components}
\PGFPlots\ comes with two components:
\begin{enumerate}
	\item the plotting component (which you are currently reading) and
	\item the \PGFPlotstable\ component which simplifies number formatting and postprocessing of numerical tables. It comes as a separate package and has its own manual \href{file:pgfplotstable.pdf}{pgfplotstable.pdf}.
\end{enumerate}

\subsection{Upgrade remarks}
This release provides a lot of improvements which can be found in all detail in \texttt{ChangeLog} for interested readers. However, some attention is useful with respect to the following changes.

\subsubsection{New Optional Features}
\begin{enumerate}
	\item \PGFPlots\ 1.3 comes with user interface improvements. The technical distinction between ``behavior options'' and ``style options'' of older versions is no longer necessary (although still fully supported).

	\item \PGFPlots\ 1.3 has a new feature which allows to \emph{move axis labels tight to tick labels} automatically.

	Since this affects the spacing, it is not enabled be default.

	Use
\begin{codeexample}[code only]
\usepackage{pgfplots}
\pgfplotsset{compat=1.3}
\end{codeexample}
	in your preamble to benefit from the improved spacing\footnote{The |compat=1.3| setting is necessary for new features which move axis labels around like reversed axis. However, \PGFPlots\ will still work as it does in earlier versions, your documents will remain the same if you don't set it explicitly.}.
	Take a look at the next page for the precise definition of |compat|.

	\item \PGFPlots\ 1.3 now supports reversed axes. It is no longer necessary to use work--arounds with negative units.
\pgfkeys{/pdflinks/search key prefixes in/.add={/pgfplots/,}{}}

	Take a look at the |x dir=reverse| key.

	Existing work--arounds will still function properly. Use |\pgfplotsset{compat=1.3}| together with |x dir=reverse| to switch to the new version.
\end{enumerate}

\subsubsection{Old Features Which May Need Attention}
\begin{enumerate}
	\item The |scatter/classes| feature produces proper legends as of version 1.3. This may change the appearance of existing legends of plots with |scatter/classes|.

	\item Due to a small math bug in \PGF\ $2.00$, you \emph{can't} use the math expression `|-x^2|'. It is necessary to use `|0-x^2|' instead. The same holds for
		`|exp(-x^2)|'; use `|exp(0-x^2)|' instead.

		This will be fixed with \PGF\ $>2.00$.

	\item Starting with \PGFPlots\ $1.1$, |\tikzstyle| should \emph{no longer be used} to set \PGFPlots\ options.
	
	Although |\tikzstyle| is still supported for some older \PGFPlots\ options, you should replace any occurance of |\tikzstyle| with |\pgfplotsset{|\meta{style name}|/.style={|\meta{key-value-list}|}}| or the associated |/.append style| variant. See section~\ref{sec:styles} for more detail.
\end{enumerate}
I apologize for any inconvenience caused by these changes.

\begin{pgfplotskey}{compat=\mchoice{1.3,pre 1.3,default,newest} (initially default)}
	The preamble configuration 
\begin{codeexample}[code only]
\usepackage{pgfplots}
\pgfplotsset{compat=1.3}
\end{codeexample}
	allows to choose between backwards compatibility and most recent features.

	\PGFPlots\ version 1.3 comes with new features which may lead to different spacings compared to earlier versions:
	\begin{enumerate}
		\item Version 1.3 comes with the |xlabel near ticks| feature which places (in this case $x$) axis labels ``near'' the tick labels, taking the size of tick labels into account.
		
		Older versions used |xlabel absolute|, i.e.\ an absolute distance between the axis and axis labels (now matter how large or small tick labels are).

		The initial setting \emph{keeps backwards compatibility}.

		You are encouraged to use
\begin{codeexample}[code only]
\usepackage{pgfplots}
\pgfplotsset{compat=1.3}
\end{codeexample}
		in your preamble.
		
		\item Version 1.3 fixes a bug with |anchor=south west| which occurred mainly for reversed axes: the anchors where upside--down. This is fixed now.

		
		If you have written some work--around and want to keep it going, use

		|\pgfplotsset{compat/anchors=pre 1.3}|\pgfmanualpdflabel{/pgfplots/compat/anchors}{} 

		to disable the bug fix.
	\end{enumerate}

	Use |\pgfplotsset{compat=default}| to restore the factory settings.

	The setting |\pgfplotsset{compat=newest}| will always use features of the most recent version. This might result in small changes in the document's appearance.
\end{pgfplotskey}

\subsection{The Team}
\PGFPlots\ has been written mainly by Christian Feuers�nger with many improvements of Pascal Wolkotte and Nick Papior Andersen as a spare time project. We hope it is useful and provides valuable plots.

If you are interested in writing something but don't know how, consider reading the auxiliary manual \href{file:TeX-programming-notes.pdf}{TeX-programming-notes.pdf} which comes with \PGFPlots. It is far from complete, but maybe it is a good starting point (at least for more literature).

\subsection{Acknowledgements}
I thank God for all hours of enjoyed programming. I thank Pascal Wolkotte and Nick Papior Andersen for their programming efforts and contributions as part of the development team. I thank Stefan Tibus, who contributed the |plot shell| feature. I thank Tom Cashman for the contribution of the |reverse legend| feature. Special thanks go to Stefan Pinnow whose tests of \PGFPlots\ lead to numerous quality improvements. Furthermore, I thank Dr.~Schweitzer for many fruitful discussions and Dr.~Meine for his ideas and suggestions. Thanks as well to the many international contributors who provided feature requests or identified bugs or simply manual improvements!

Last but not least, I thank Till Tantau and Mark Wibrow for their excellent graphics (and more) package \PGF\ and \Tikz\ which is the base of \PGFPlots.

\include{pgfplots.install}%
\include{pgfplots.intro}%
\include{pgfplots.reference}%
\include{pgfplots.libs}%
\include{pgfplots.resources}%
\include{pgfplots.importexport}%
\include{pgfplots.basic.reference}%

\printindex

\bibliographystyle{abbrv} %gerapali} %gerabbrv} %gerunsrt.bst} %gerabbrv}% gerplain}
\nocite{pgfplotstable}
\nocite{programmingnotes}
\bibliography{pgfplots}
\end{document}
%
\include{pgfplots.basic.reference}%

\printindex

\bibliographystyle{abbrv} %gerapali} %gerabbrv} %gerunsrt.bst} %gerabbrv}% gerplain}
\nocite{pgfplotstable}
\nocite{programmingnotes}
\bibliography{pgfplots}
\end{document}
%
%%%%%%%%%%%%%%%%%%%%%%%%%%%%%%%%%%%%%%%%%%%%%%%%%%%%%%%%%%%%%%%%%%%%%%%%%%%%%
%
% Package pgfplots.sty documentation. 
%
% Copyright 2007/2008 by Christian Feuersaenger.
%
% This program is free software: you can redistribute it and/or modify
% it under the terms of the GNU General Public License as published by
% the Free Software Foundation, either version 3 of the License, or
% (at your option) any later version.
% 
% This program is distributed in the hope that it will be useful,
% but WITHOUT ANY WARRANTY; without even the implied warranty of
% MERCHANTABILITY or FITNESS FOR A PARTICULAR PURPOSE.  See the
% GNU General Public License for more details.
% 
% You should have received a copy of the GNU General Public License
% along with this program.  If not, see <http://www.gnu.org/licenses/>.
%
%
%%%%%%%%%%%%%%%%%%%%%%%%%%%%%%%%%%%%%%%%%%%%%%%%%%%%%%%%%%%%%%%%%%%%%%%%%%%%%
%%%%%%%%%%%%%%%%%%%%%%%%%%%%%%%%%%%%%%%%%%%%%%%%%%%%%%%%%%%%%%%%%%%%%%%%%%%%%
%
% Package pgfplots.sty documentation. 
%
% Copyright 2007/2008 by Christian Feuersaenger.
%
% This program is free software: you can redistribute it and/or modify
% it under the terms of the GNU General Public License as published by
% the Free Software Foundation, either version 3 of the License, or
% (at your option) any later version.
% 
% This program is distributed in the hope that it will be useful,
% but WITHOUT ANY WARRANTY; without even the implied warranty of
% MERCHANTABILITY or FITNESS FOR A PARTICULAR PURPOSE.  See the
% GNU General Public License for more details.
% 
% You should have received a copy of the GNU General Public License
% along with this program.  If not, see <http://www.gnu.org/licenses/>.
%
%
%%%%%%%%%%%%%%%%%%%%%%%%%%%%%%%%%%%%%%%%%%%%%%%%%%%%%%%%%%%%%%%%%%%%%%%%%%%%%
\documentclass[a4paper]{ltxdoc}

\usepackage{makeidx}

% DON't let hyperref overload the format of index and glossary. 
% I want to do that on my own in the stylefiles for makeindex...
\makeatletter
\let\@old@wrindex=\@wrindex
\makeatother

\usepackage{ifpdf}
\ifpdf
	\usepackage[pdftex]{hyperref}
\else
	\usepackage[dvipdfm]{hyperref}
\fi
\hypersetup{pdfborder={0 0 0}}

\makeatletter
\let\@wrindex=\@old@wrindex
\makeatother

% Formatiere Seitennummern im Index:
\newcommand{\indexpageno}[1]{%
	{\bfseries\hyperpage{#1}}%
}


\newcommand{\R}{\mathbb{R}}
\newcommand{\N}{\mathbb{N}}
\newcommand{\Z}{\mathbb{Z}}

\long\def\COMMENTLOWLEVEL#1\ENDCOMMENT{}
\def\ENDCOMMENT{}

\usepackage{textcomp}
\usepackage{booktabs}

\usepackage{calc}
\usepackage[formats]{listings}
%\usepackage{courier} % don't use it - the '^' character can't be copy-pasted in courier

\usepackage{array}
\lstset{%
	basicstyle=\ttfamily,
	language=[LaTeX]tex, % Seems as if \lstset{language=tex} must be invoked BEFORE loading tikz!?
	tabsize=4,
	breaklines=true,
	breakindent=0pt
}

\ifpdf
	\pdfinfo {
		/Author	(Christian Feuersaenger)
	}

\else
	\def\pgfsysdriver{pgfsys-dvipdfm.def}
\fi
%\def\pgfsysdriver{pgfsys-pdftex.def}
\usepackage{pgfplots}

\ifpdf
	\usetikzlibrary{pgfplotsclickable}
\fi

%\usepackage{fp}
% ATTENTION:
% this requires pgf version NEWER than 2.00 :
%\usetikzlibrary{fixedpointarithmetic}

\usetikzlibrary{dateplot}

\usepackage[a4paper,left=2.25cm,right=2.25cm,top=2.5cm,bottom=2.5cm,nohead]{geometry}
\usepackage{amsmath,amssymb}
\usepackage{xxcolor}
\usepackage{pifont}
\usepackage[latin1]{inputenc}
\usepackage{amsmath}
\usepackage{eurosym}
\input{pgfplots-macros}

\pgfqkeys{/codeexample}{%
	every codeexample/.style={
		width=8cm,
		/pgfplots/every axis/.append style={legend style={fill=graphicbackground}}
	},
	tabsize=4,
}
\pgfplotsset{
	every axis/.append style={width=7cm},
	filter discard warning=false,
}

\usetikzlibrary{backgrounds,patterns}
% Global styles:
\tikzset{
  shape example/.style={
    color=black!30,
    draw,
    fill=yellow!30,
    line width=.5cm,
    inner xsep=2.5cm,
    inner ysep=0.5cm}
}

\newcommand{\FIXME}[1]{\textcolor{red}{(FIXME: #1)}}

% fuer endvironment 'sidewaysfigure' bspw
% \usepackage{rotating}

\newcommand\Tikz{Ti\textit kZ}
\newcommand\PGF{\textsc{pgf}}
\newcommand\PGFPlots{\textsc{pgfplots}}
\def\PGFPlotstable{\textsc{PgfplotsTable}}


\makeindex

% Fix overful hboxes automatically:
\tolerance=2000
\emergencystretch=10pt

\tikzset{prefix=gnuplot/pgfplots_} % prefix for 'plot function'

\author{%
	Christian Feuers\"anger\footnote{\url{http://wissrech.ins.uni-bonn.de/people/feuersaenger}}\\%
	Institut f\"ur Numerische Simulation\\
	Universit\"at Bonn, Germany}


%\RequirePackage[german,english,francais]{babel}

\pgfplotsmanualexternalexpensivetrue

%\usetikzlibrary{external}
%\tikzexternalize[prefix=figures/]{pgfplots}

\title{%
	Manual for Package \PGFPlots\\
	{\small Version \pgfplotsversion}\\
	{\small\href{http://sourceforge.net/projects/pgfplots}{http://sourceforge.net/projects/pgfplots}}}

%\includeonly{pgfplots.libs}


\begin{document}

\def\plotcoords{%
\addplot coordinates {
(5,8.312e-02)    (17,2.547e-02)   (49,7.407e-03)
(129,2.102e-03)  (321,5.874e-04)  (769,1.623e-04)
(1793,4.442e-05) (4097,1.207e-05) (9217,3.261e-06)
};

\addplot coordinates{
(7,8.472e-02)    (31,3.044e-02)    (111,1.022e-02)
(351,3.303e-03)  (1023,1.039e-03)  (2815,3.196e-04)
(7423,9.658e-05) (18943,2.873e-05) (47103,8.437e-06)
};

\addplot coordinates{
(9,7.881e-02)     (49,3.243e-02)    (209,1.232e-02)
(769,4.454e-03)   (2561,1.551e-03)  (7937,5.236e-04)
(23297,1.723e-04) (65537,5.545e-05) (178177,1.751e-05)
};

\addplot coordinates{
(11,6.887e-02)    (71,3.177e-02)     (351,1.341e-02)
(1471,5.334e-03)  (5503,2.027e-03)   (18943,7.415e-04)
(61183,2.628e-04) (187903,9.063e-05) (553983,3.053e-05)
};

\addplot coordinates{
(13,5.755e-02)     (97,2.925e-02)     (545,1.351e-02)
(2561,5.842e-03)   (10625,2.397e-03)  (40193,9.414e-04)
(141569,3.564e-04) (471041,1.308e-04) 
(1496065,4.670e-05)
};
}%
\maketitle
\begin{abstract}%
\PGFPlots\ draws high--quality function plots in normal or logarithmic scaling with a user-friendly interface directly in \TeX. The user supplies axis labels, legend entries and the plot coordinates for one or more plots and \PGFPlots\ applies axis scaling, computes any logarithms and axis ticks and draws the plots, supporting line plots, scatter plots, piecewise constant plots, bar plots, area plots, mesh-- and surface plots and some more. It is based on Till Tantau's package \PGF/\Tikz.
\end{abstract}
\tableofcontents
\section{Introduction}
This package provides tools to generate plots and labeled axes easily. It draws normal plots, logplots and semi-logplots, in two and three dimensions. Axis ticks, labels, legends (in case of multiple plots) can be added with key-value options. It can cycle through a set of predefined line/marker/color specifications. In summary, its purpose is to simplify the generation of high-quality function and/or data plots, and solving the problems of
\begin{itemize}
	\item consistency of document and font type and font size,
	\item direct use of \TeX\ math mode in axis descriptions,
	\item consistency of data and figures (no third party tool necessary),
	\item inter document consistency using preamble configurations and styles.
\end{itemize}
Although not necessary, separate |.pdf| or |.eps| graphics can be generated using the |external| library developed as part of \Tikz.

\PGFPlots\ is build completely on \Tikz/\PGF. Knowledge of \Tikz\ will simplify the work with \PGFPlots, although it is not required.

\section[About PGFPlots: Preliminaries]{About {\normalfont\PGFPlots}: Preliminaries}
This section contains information about upgrades, the team, the installation (in case you need to do it manually) and troubleshooting. You may skip it completely except for the upgrade remarks.

However, note that this library requires at least \PGF\ version $2.00$. At the time of this writing, many \TeX-distributions still contain the older \PGF\ version $1.18$, so it may be necessary to install a recent \PGF\ prior to using \PGFPlots.

\subsection{Components}
\PGFPlots\ comes with two components:
\begin{enumerate}
	\item the plotting component (which you are currently reading) and
	\item the \PGFPlotstable\ component which simplifies number formatting and postprocessing of numerical tables. It comes as a separate package and has its own manual \href{file:pgfplotstable.pdf}{pgfplotstable.pdf}.
\end{enumerate}

\subsection{Upgrade remarks}
This release provides a lot of improvements which can be found in all detail in \texttt{ChangeLog} for interested readers. However, some attention is useful with respect to the following changes.

\subsubsection{New Optional Features}
\begin{enumerate}
	\item \PGFPlots\ 1.3 comes with user interface improvements. The technical distinction between ``behavior options'' and ``style options'' of older versions is no longer necessary (although still fully supported).

	\item \PGFPlots\ 1.3 has a new feature which allows to \emph{move axis labels tight to tick labels} automatically.

	Since this affects the spacing, it is not enabled be default.

	Use
\begin{codeexample}[code only]
\usepackage{pgfplots}
\pgfplotsset{compat=1.3}
\end{codeexample}
	in your preamble to benefit from the improved spacing\footnote{The |compat=1.3| setting is necessary for new features which move axis labels around like reversed axis. However, \PGFPlots\ will still work as it does in earlier versions, your documents will remain the same if you don't set it explicitly.}.
	Take a look at the next page for the precise definition of |compat|.

	\item \PGFPlots\ 1.3 now supports reversed axes. It is no longer necessary to use work--arounds with negative units.
\pgfkeys{/pdflinks/search key prefixes in/.add={/pgfplots/,}{}}

	Take a look at the |x dir=reverse| key.

	Existing work--arounds will still function properly. Use |\pgfplotsset{compat=1.3}| together with |x dir=reverse| to switch to the new version.
\end{enumerate}

\subsubsection{Old Features Which May Need Attention}
\begin{enumerate}
	\item The |scatter/classes| feature produces proper legends as of version 1.3. This may change the appearance of existing legends of plots with |scatter/classes|.

	\item Due to a small math bug in \PGF\ $2.00$, you \emph{can't} use the math expression `|-x^2|'. It is necessary to use `|0-x^2|' instead. The same holds for
		`|exp(-x^2)|'; use `|exp(0-x^2)|' instead.

		This will be fixed with \PGF\ $>2.00$.

	\item Starting with \PGFPlots\ $1.1$, |\tikzstyle| should \emph{no longer be used} to set \PGFPlots\ options.
	
	Although |\tikzstyle| is still supported for some older \PGFPlots\ options, you should replace any occurance of |\tikzstyle| with |\pgfplotsset{|\meta{style name}|/.style={|\meta{key-value-list}|}}| or the associated |/.append style| variant. See section~\ref{sec:styles} for more detail.
\end{enumerate}
I apologize for any inconvenience caused by these changes.

\begin{pgfplotskey}{compat=\mchoice{1.3,pre 1.3,default,newest} (initially default)}
	The preamble configuration 
\begin{codeexample}[code only]
\usepackage{pgfplots}
\pgfplotsset{compat=1.3}
\end{codeexample}
	allows to choose between backwards compatibility and most recent features.

	\PGFPlots\ version 1.3 comes with new features which may lead to different spacings compared to earlier versions:
	\begin{enumerate}
		\item Version 1.3 comes with the |xlabel near ticks| feature which places (in this case $x$) axis labels ``near'' the tick labels, taking the size of tick labels into account.
		
		Older versions used |xlabel absolute|, i.e.\ an absolute distance between the axis and axis labels (now matter how large or small tick labels are).

		The initial setting \emph{keeps backwards compatibility}.

		You are encouraged to use
\begin{codeexample}[code only]
\usepackage{pgfplots}
\pgfplotsset{compat=1.3}
\end{codeexample}
		in your preamble.
		
		\item Version 1.3 fixes a bug with |anchor=south west| which occurred mainly for reversed axes: the anchors where upside--down. This is fixed now.

		
		If you have written some work--around and want to keep it going, use

		|\pgfplotsset{compat/anchors=pre 1.3}|\pgfmanualpdflabel{/pgfplots/compat/anchors}{} 

		to disable the bug fix.
	\end{enumerate}

	Use |\pgfplotsset{compat=default}| to restore the factory settings.

	The setting |\pgfplotsset{compat=newest}| will always use features of the most recent version. This might result in small changes in the document's appearance.
\end{pgfplotskey}

\subsection{The Team}
\PGFPlots\ has been written mainly by Christian Feuers�nger with many improvements of Pascal Wolkotte and Nick Papior Andersen as a spare time project. We hope it is useful and provides valuable plots.

If you are interested in writing something but don't know how, consider reading the auxiliary manual \href{file:TeX-programming-notes.pdf}{TeX-programming-notes.pdf} which comes with \PGFPlots. It is far from complete, but maybe it is a good starting point (at least for more literature).

\subsection{Acknowledgements}
I thank God for all hours of enjoyed programming. I thank Pascal Wolkotte and Nick Papior Andersen for their programming efforts and contributions as part of the development team. I thank Stefan Tibus, who contributed the |plot shell| feature. I thank Tom Cashman for the contribution of the |reverse legend| feature. Special thanks go to Stefan Pinnow whose tests of \PGFPlots\ lead to numerous quality improvements. Furthermore, I thank Dr.~Schweitzer for many fruitful discussions and Dr.~Meine for his ideas and suggestions. Thanks as well to the many international contributors who provided feature requests or identified bugs or simply manual improvements!

Last but not least, I thank Till Tantau and Mark Wibrow for their excellent graphics (and more) package \PGF\ and \Tikz\ which is the base of \PGFPlots.

%%%%%%%%%%%%%%%%%%%%%%%%%%%%%%%%%%%%%%%%%%%%%%%%%%%%%%%%%%%%%%%%%%%%%%%%%%%%%
%
% Package pgfplots.sty documentation. 
%
% Copyright 2007/2008 by Christian Feuersaenger.
%
% This program is free software: you can redistribute it and/or modify
% it under the terms of the GNU General Public License as published by
% the Free Software Foundation, either version 3 of the License, or
% (at your option) any later version.
% 
% This program is distributed in the hope that it will be useful,
% but WITHOUT ANY WARRANTY; without even the implied warranty of
% MERCHANTABILITY or FITNESS FOR A PARTICULAR PURPOSE.  See the
% GNU General Public License for more details.
% 
% You should have received a copy of the GNU General Public License
% along with this program.  If not, see <http://www.gnu.org/licenses/>.
%
%
%%%%%%%%%%%%%%%%%%%%%%%%%%%%%%%%%%%%%%%%%%%%%%%%%%%%%%%%%%%%%%%%%%%%%%%%%%%%%
\input{pgfplots.preamble.tex}
%\RequirePackage[german,english,francais]{babel}

\pgfplotsmanualexternalexpensivetrue

%\usetikzlibrary{external}
%\tikzexternalize[prefix=figures/]{pgfplots}

\title{%
	Manual for Package \PGFPlots\\
	{\small Version \pgfplotsversion}\\
	{\small\href{http://sourceforge.net/projects/pgfplots}{http://sourceforge.net/projects/pgfplots}}}

%\includeonly{pgfplots.libs}


\begin{document}

\def\plotcoords{%
\addplot coordinates {
(5,8.312e-02)    (17,2.547e-02)   (49,7.407e-03)
(129,2.102e-03)  (321,5.874e-04)  (769,1.623e-04)
(1793,4.442e-05) (4097,1.207e-05) (9217,3.261e-06)
};

\addplot coordinates{
(7,8.472e-02)    (31,3.044e-02)    (111,1.022e-02)
(351,3.303e-03)  (1023,1.039e-03)  (2815,3.196e-04)
(7423,9.658e-05) (18943,2.873e-05) (47103,8.437e-06)
};

\addplot coordinates{
(9,7.881e-02)     (49,3.243e-02)    (209,1.232e-02)
(769,4.454e-03)   (2561,1.551e-03)  (7937,5.236e-04)
(23297,1.723e-04) (65537,5.545e-05) (178177,1.751e-05)
};

\addplot coordinates{
(11,6.887e-02)    (71,3.177e-02)     (351,1.341e-02)
(1471,5.334e-03)  (5503,2.027e-03)   (18943,7.415e-04)
(61183,2.628e-04) (187903,9.063e-05) (553983,3.053e-05)
};

\addplot coordinates{
(13,5.755e-02)     (97,2.925e-02)     (545,1.351e-02)
(2561,5.842e-03)   (10625,2.397e-03)  (40193,9.414e-04)
(141569,3.564e-04) (471041,1.308e-04) 
(1496065,4.670e-05)
};
}%
\maketitle
\begin{abstract}%
\PGFPlots\ draws high--quality function plots in normal or logarithmic scaling with a user-friendly interface directly in \TeX. The user supplies axis labels, legend entries and the plot coordinates for one or more plots and \PGFPlots\ applies axis scaling, computes any logarithms and axis ticks and draws the plots, supporting line plots, scatter plots, piecewise constant plots, bar plots, area plots, mesh-- and surface plots and some more. It is based on Till Tantau's package \PGF/\Tikz.
\end{abstract}
\tableofcontents
\section{Introduction}
This package provides tools to generate plots and labeled axes easily. It draws normal plots, logplots and semi-logplots, in two and three dimensions. Axis ticks, labels, legends (in case of multiple plots) can be added with key-value options. It can cycle through a set of predefined line/marker/color specifications. In summary, its purpose is to simplify the generation of high-quality function and/or data plots, and solving the problems of
\begin{itemize}
	\item consistency of document and font type and font size,
	\item direct use of \TeX\ math mode in axis descriptions,
	\item consistency of data and figures (no third party tool necessary),
	\item inter document consistency using preamble configurations and styles.
\end{itemize}
Although not necessary, separate |.pdf| or |.eps| graphics can be generated using the |external| library developed as part of \Tikz.

\PGFPlots\ is build completely on \Tikz/\PGF. Knowledge of \Tikz\ will simplify the work with \PGFPlots, although it is not required.

\section[About PGFPlots: Preliminaries]{About {\normalfont\PGFPlots}: Preliminaries}
This section contains information about upgrades, the team, the installation (in case you need to do it manually) and troubleshooting. You may skip it completely except for the upgrade remarks.

However, note that this library requires at least \PGF\ version $2.00$. At the time of this writing, many \TeX-distributions still contain the older \PGF\ version $1.18$, so it may be necessary to install a recent \PGF\ prior to using \PGFPlots.

\subsection{Components}
\PGFPlots\ comes with two components:
\begin{enumerate}
	\item the plotting component (which you are currently reading) and
	\item the \PGFPlotstable\ component which simplifies number formatting and postprocessing of numerical tables. It comes as a separate package and has its own manual \href{file:pgfplotstable.pdf}{pgfplotstable.pdf}.
\end{enumerate}

\subsection{Upgrade remarks}
This release provides a lot of improvements which can be found in all detail in \texttt{ChangeLog} for interested readers. However, some attention is useful with respect to the following changes.

\subsubsection{New Optional Features}
\begin{enumerate}
	\item \PGFPlots\ 1.3 comes with user interface improvements. The technical distinction between ``behavior options'' and ``style options'' of older versions is no longer necessary (although still fully supported).

	\item \PGFPlots\ 1.3 has a new feature which allows to \emph{move axis labels tight to tick labels} automatically.

	Since this affects the spacing, it is not enabled be default.

	Use
\begin{codeexample}[code only]
\usepackage{pgfplots}
\pgfplotsset{compat=1.3}
\end{codeexample}
	in your preamble to benefit from the improved spacing\footnote{The |compat=1.3| setting is necessary for new features which move axis labels around like reversed axis. However, \PGFPlots\ will still work as it does in earlier versions, your documents will remain the same if you don't set it explicitly.}.
	Take a look at the next page for the precise definition of |compat|.

	\item \PGFPlots\ 1.3 now supports reversed axes. It is no longer necessary to use work--arounds with negative units.
\pgfkeys{/pdflinks/search key prefixes in/.add={/pgfplots/,}{}}

	Take a look at the |x dir=reverse| key.

	Existing work--arounds will still function properly. Use |\pgfplotsset{compat=1.3}| together with |x dir=reverse| to switch to the new version.
\end{enumerate}

\subsubsection{Old Features Which May Need Attention}
\begin{enumerate}
	\item The |scatter/classes| feature produces proper legends as of version 1.3. This may change the appearance of existing legends of plots with |scatter/classes|.

	\item Due to a small math bug in \PGF\ $2.00$, you \emph{can't} use the math expression `|-x^2|'. It is necessary to use `|0-x^2|' instead. The same holds for
		`|exp(-x^2)|'; use `|exp(0-x^2)|' instead.

		This will be fixed with \PGF\ $>2.00$.

	\item Starting with \PGFPlots\ $1.1$, |\tikzstyle| should \emph{no longer be used} to set \PGFPlots\ options.
	
	Although |\tikzstyle| is still supported for some older \PGFPlots\ options, you should replace any occurance of |\tikzstyle| with |\pgfplotsset{|\meta{style name}|/.style={|\meta{key-value-list}|}}| or the associated |/.append style| variant. See section~\ref{sec:styles} for more detail.
\end{enumerate}
I apologize for any inconvenience caused by these changes.

\begin{pgfplotskey}{compat=\mchoice{1.3,pre 1.3,default,newest} (initially default)}
	The preamble configuration 
\begin{codeexample}[code only]
\usepackage{pgfplots}
\pgfplotsset{compat=1.3}
\end{codeexample}
	allows to choose between backwards compatibility and most recent features.

	\PGFPlots\ version 1.3 comes with new features which may lead to different spacings compared to earlier versions:
	\begin{enumerate}
		\item Version 1.3 comes with the |xlabel near ticks| feature which places (in this case $x$) axis labels ``near'' the tick labels, taking the size of tick labels into account.
		
		Older versions used |xlabel absolute|, i.e.\ an absolute distance between the axis and axis labels (now matter how large or small tick labels are).

		The initial setting \emph{keeps backwards compatibility}.

		You are encouraged to use
\begin{codeexample}[code only]
\usepackage{pgfplots}
\pgfplotsset{compat=1.3}
\end{codeexample}
		in your preamble.
		
		\item Version 1.3 fixes a bug with |anchor=south west| which occurred mainly for reversed axes: the anchors where upside--down. This is fixed now.

		
		If you have written some work--around and want to keep it going, use

		|\pgfplotsset{compat/anchors=pre 1.3}|\pgfmanualpdflabel{/pgfplots/compat/anchors}{} 

		to disable the bug fix.
	\end{enumerate}

	Use |\pgfplotsset{compat=default}| to restore the factory settings.

	The setting |\pgfplotsset{compat=newest}| will always use features of the most recent version. This might result in small changes in the document's appearance.
\end{pgfplotskey}

\subsection{The Team}
\PGFPlots\ has been written mainly by Christian Feuers�nger with many improvements of Pascal Wolkotte and Nick Papior Andersen as a spare time project. We hope it is useful and provides valuable plots.

If you are interested in writing something but don't know how, consider reading the auxiliary manual \href{file:TeX-programming-notes.pdf}{TeX-programming-notes.pdf} which comes with \PGFPlots. It is far from complete, but maybe it is a good starting point (at least for more literature).

\subsection{Acknowledgements}
I thank God for all hours of enjoyed programming. I thank Pascal Wolkotte and Nick Papior Andersen for their programming efforts and contributions as part of the development team. I thank Stefan Tibus, who contributed the |plot shell| feature. I thank Tom Cashman for the contribution of the |reverse legend| feature. Special thanks go to Stefan Pinnow whose tests of \PGFPlots\ lead to numerous quality improvements. Furthermore, I thank Dr.~Schweitzer for many fruitful discussions and Dr.~Meine for his ideas and suggestions. Thanks as well to the many international contributors who provided feature requests or identified bugs or simply manual improvements!

Last but not least, I thank Till Tantau and Mark Wibrow for their excellent graphics (and more) package \PGF\ and \Tikz\ which is the base of \PGFPlots.

\include{pgfplots.install}%
\include{pgfplots.intro}%
\include{pgfplots.reference}%
\include{pgfplots.libs}%
\include{pgfplots.resources}%
\include{pgfplots.importexport}%
\include{pgfplots.basic.reference}%

\printindex

\bibliographystyle{abbrv} %gerapali} %gerabbrv} %gerunsrt.bst} %gerabbrv}% gerplain}
\nocite{pgfplotstable}
\nocite{programmingnotes}
\bibliography{pgfplots}
\end{document}
%
%%%%%%%%%%%%%%%%%%%%%%%%%%%%%%%%%%%%%%%%%%%%%%%%%%%%%%%%%%%%%%%%%%%%%%%%%%%%%
%
% Package pgfplots.sty documentation. 
%
% Copyright 2007/2008 by Christian Feuersaenger.
%
% This program is free software: you can redistribute it and/or modify
% it under the terms of the GNU General Public License as published by
% the Free Software Foundation, either version 3 of the License, or
% (at your option) any later version.
% 
% This program is distributed in the hope that it will be useful,
% but WITHOUT ANY WARRANTY; without even the implied warranty of
% MERCHANTABILITY or FITNESS FOR A PARTICULAR PURPOSE.  See the
% GNU General Public License for more details.
% 
% You should have received a copy of the GNU General Public License
% along with this program.  If not, see <http://www.gnu.org/licenses/>.
%
%
%%%%%%%%%%%%%%%%%%%%%%%%%%%%%%%%%%%%%%%%%%%%%%%%%%%%%%%%%%%%%%%%%%%%%%%%%%%%%
\input{pgfplots.preamble.tex}
%\RequirePackage[german,english,francais]{babel}

\pgfplotsmanualexternalexpensivetrue

%\usetikzlibrary{external}
%\tikzexternalize[prefix=figures/]{pgfplots}

\title{%
	Manual for Package \PGFPlots\\
	{\small Version \pgfplotsversion}\\
	{\small\href{http://sourceforge.net/projects/pgfplots}{http://sourceforge.net/projects/pgfplots}}}

%\includeonly{pgfplots.libs}


\begin{document}

\def\plotcoords{%
\addplot coordinates {
(5,8.312e-02)    (17,2.547e-02)   (49,7.407e-03)
(129,2.102e-03)  (321,5.874e-04)  (769,1.623e-04)
(1793,4.442e-05) (4097,1.207e-05) (9217,3.261e-06)
};

\addplot coordinates{
(7,8.472e-02)    (31,3.044e-02)    (111,1.022e-02)
(351,3.303e-03)  (1023,1.039e-03)  (2815,3.196e-04)
(7423,9.658e-05) (18943,2.873e-05) (47103,8.437e-06)
};

\addplot coordinates{
(9,7.881e-02)     (49,3.243e-02)    (209,1.232e-02)
(769,4.454e-03)   (2561,1.551e-03)  (7937,5.236e-04)
(23297,1.723e-04) (65537,5.545e-05) (178177,1.751e-05)
};

\addplot coordinates{
(11,6.887e-02)    (71,3.177e-02)     (351,1.341e-02)
(1471,5.334e-03)  (5503,2.027e-03)   (18943,7.415e-04)
(61183,2.628e-04) (187903,9.063e-05) (553983,3.053e-05)
};

\addplot coordinates{
(13,5.755e-02)     (97,2.925e-02)     (545,1.351e-02)
(2561,5.842e-03)   (10625,2.397e-03)  (40193,9.414e-04)
(141569,3.564e-04) (471041,1.308e-04) 
(1496065,4.670e-05)
};
}%
\maketitle
\begin{abstract}%
\PGFPlots\ draws high--quality function plots in normal or logarithmic scaling with a user-friendly interface directly in \TeX. The user supplies axis labels, legend entries and the plot coordinates for one or more plots and \PGFPlots\ applies axis scaling, computes any logarithms and axis ticks and draws the plots, supporting line plots, scatter plots, piecewise constant plots, bar plots, area plots, mesh-- and surface plots and some more. It is based on Till Tantau's package \PGF/\Tikz.
\end{abstract}
\tableofcontents
\section{Introduction}
This package provides tools to generate plots and labeled axes easily. It draws normal plots, logplots and semi-logplots, in two and three dimensions. Axis ticks, labels, legends (in case of multiple plots) can be added with key-value options. It can cycle through a set of predefined line/marker/color specifications. In summary, its purpose is to simplify the generation of high-quality function and/or data plots, and solving the problems of
\begin{itemize}
	\item consistency of document and font type and font size,
	\item direct use of \TeX\ math mode in axis descriptions,
	\item consistency of data and figures (no third party tool necessary),
	\item inter document consistency using preamble configurations and styles.
\end{itemize}
Although not necessary, separate |.pdf| or |.eps| graphics can be generated using the |external| library developed as part of \Tikz.

\PGFPlots\ is build completely on \Tikz/\PGF. Knowledge of \Tikz\ will simplify the work with \PGFPlots, although it is not required.

\section[About PGFPlots: Preliminaries]{About {\normalfont\PGFPlots}: Preliminaries}
This section contains information about upgrades, the team, the installation (in case you need to do it manually) and troubleshooting. You may skip it completely except for the upgrade remarks.

However, note that this library requires at least \PGF\ version $2.00$. At the time of this writing, many \TeX-distributions still contain the older \PGF\ version $1.18$, so it may be necessary to install a recent \PGF\ prior to using \PGFPlots.

\subsection{Components}
\PGFPlots\ comes with two components:
\begin{enumerate}
	\item the plotting component (which you are currently reading) and
	\item the \PGFPlotstable\ component which simplifies number formatting and postprocessing of numerical tables. It comes as a separate package and has its own manual \href{file:pgfplotstable.pdf}{pgfplotstable.pdf}.
\end{enumerate}

\subsection{Upgrade remarks}
This release provides a lot of improvements which can be found in all detail in \texttt{ChangeLog} for interested readers. However, some attention is useful with respect to the following changes.

\subsubsection{New Optional Features}
\begin{enumerate}
	\item \PGFPlots\ 1.3 comes with user interface improvements. The technical distinction between ``behavior options'' and ``style options'' of older versions is no longer necessary (although still fully supported).

	\item \PGFPlots\ 1.3 has a new feature which allows to \emph{move axis labels tight to tick labels} automatically.

	Since this affects the spacing, it is not enabled be default.

	Use
\begin{codeexample}[code only]
\usepackage{pgfplots}
\pgfplotsset{compat=1.3}
\end{codeexample}
	in your preamble to benefit from the improved spacing\footnote{The |compat=1.3| setting is necessary for new features which move axis labels around like reversed axis. However, \PGFPlots\ will still work as it does in earlier versions, your documents will remain the same if you don't set it explicitly.}.
	Take a look at the next page for the precise definition of |compat|.

	\item \PGFPlots\ 1.3 now supports reversed axes. It is no longer necessary to use work--arounds with negative units.
\pgfkeys{/pdflinks/search key prefixes in/.add={/pgfplots/,}{}}

	Take a look at the |x dir=reverse| key.

	Existing work--arounds will still function properly. Use |\pgfplotsset{compat=1.3}| together with |x dir=reverse| to switch to the new version.
\end{enumerate}

\subsubsection{Old Features Which May Need Attention}
\begin{enumerate}
	\item The |scatter/classes| feature produces proper legends as of version 1.3. This may change the appearance of existing legends of plots with |scatter/classes|.

	\item Due to a small math bug in \PGF\ $2.00$, you \emph{can't} use the math expression `|-x^2|'. It is necessary to use `|0-x^2|' instead. The same holds for
		`|exp(-x^2)|'; use `|exp(0-x^2)|' instead.

		This will be fixed with \PGF\ $>2.00$.

	\item Starting with \PGFPlots\ $1.1$, |\tikzstyle| should \emph{no longer be used} to set \PGFPlots\ options.
	
	Although |\tikzstyle| is still supported for some older \PGFPlots\ options, you should replace any occurance of |\tikzstyle| with |\pgfplotsset{|\meta{style name}|/.style={|\meta{key-value-list}|}}| or the associated |/.append style| variant. See section~\ref{sec:styles} for more detail.
\end{enumerate}
I apologize for any inconvenience caused by these changes.

\begin{pgfplotskey}{compat=\mchoice{1.3,pre 1.3,default,newest} (initially default)}
	The preamble configuration 
\begin{codeexample}[code only]
\usepackage{pgfplots}
\pgfplotsset{compat=1.3}
\end{codeexample}
	allows to choose between backwards compatibility and most recent features.

	\PGFPlots\ version 1.3 comes with new features which may lead to different spacings compared to earlier versions:
	\begin{enumerate}
		\item Version 1.3 comes with the |xlabel near ticks| feature which places (in this case $x$) axis labels ``near'' the tick labels, taking the size of tick labels into account.
		
		Older versions used |xlabel absolute|, i.e.\ an absolute distance between the axis and axis labels (now matter how large or small tick labels are).

		The initial setting \emph{keeps backwards compatibility}.

		You are encouraged to use
\begin{codeexample}[code only]
\usepackage{pgfplots}
\pgfplotsset{compat=1.3}
\end{codeexample}
		in your preamble.
		
		\item Version 1.3 fixes a bug with |anchor=south west| which occurred mainly for reversed axes: the anchors where upside--down. This is fixed now.

		
		If you have written some work--around and want to keep it going, use

		|\pgfplotsset{compat/anchors=pre 1.3}|\pgfmanualpdflabel{/pgfplots/compat/anchors}{} 

		to disable the bug fix.
	\end{enumerate}

	Use |\pgfplotsset{compat=default}| to restore the factory settings.

	The setting |\pgfplotsset{compat=newest}| will always use features of the most recent version. This might result in small changes in the document's appearance.
\end{pgfplotskey}

\subsection{The Team}
\PGFPlots\ has been written mainly by Christian Feuers�nger with many improvements of Pascal Wolkotte and Nick Papior Andersen as a spare time project. We hope it is useful and provides valuable plots.

If you are interested in writing something but don't know how, consider reading the auxiliary manual \href{file:TeX-programming-notes.pdf}{TeX-programming-notes.pdf} which comes with \PGFPlots. It is far from complete, but maybe it is a good starting point (at least for more literature).

\subsection{Acknowledgements}
I thank God for all hours of enjoyed programming. I thank Pascal Wolkotte and Nick Papior Andersen for their programming efforts and contributions as part of the development team. I thank Stefan Tibus, who contributed the |plot shell| feature. I thank Tom Cashman for the contribution of the |reverse legend| feature. Special thanks go to Stefan Pinnow whose tests of \PGFPlots\ lead to numerous quality improvements. Furthermore, I thank Dr.~Schweitzer for many fruitful discussions and Dr.~Meine for his ideas and suggestions. Thanks as well to the many international contributors who provided feature requests or identified bugs or simply manual improvements!

Last but not least, I thank Till Tantau and Mark Wibrow for their excellent graphics (and more) package \PGF\ and \Tikz\ which is the base of \PGFPlots.

\include{pgfplots.install}%
\include{pgfplots.intro}%
\include{pgfplots.reference}%
\include{pgfplots.libs}%
\include{pgfplots.resources}%
\include{pgfplots.importexport}%
\include{pgfplots.basic.reference}%

\printindex

\bibliographystyle{abbrv} %gerapali} %gerabbrv} %gerunsrt.bst} %gerabbrv}% gerplain}
\nocite{pgfplotstable}
\nocite{programmingnotes}
\bibliography{pgfplots}
\end{document}
%
%%%%%%%%%%%%%%%%%%%%%%%%%%%%%%%%%%%%%%%%%%%%%%%%%%%%%%%%%%%%%%%%%%%%%%%%%%%%%
%
% Package pgfplots.sty documentation. 
%
% Copyright 2007/2008 by Christian Feuersaenger.
%
% This program is free software: you can redistribute it and/or modify
% it under the terms of the GNU General Public License as published by
% the Free Software Foundation, either version 3 of the License, or
% (at your option) any later version.
% 
% This program is distributed in the hope that it will be useful,
% but WITHOUT ANY WARRANTY; without even the implied warranty of
% MERCHANTABILITY or FITNESS FOR A PARTICULAR PURPOSE.  See the
% GNU General Public License for more details.
% 
% You should have received a copy of the GNU General Public License
% along with this program.  If not, see <http://www.gnu.org/licenses/>.
%
%
%%%%%%%%%%%%%%%%%%%%%%%%%%%%%%%%%%%%%%%%%%%%%%%%%%%%%%%%%%%%%%%%%%%%%%%%%%%%%
\input{pgfplots.preamble.tex}
%\RequirePackage[german,english,francais]{babel}

\pgfplotsmanualexternalexpensivetrue

%\usetikzlibrary{external}
%\tikzexternalize[prefix=figures/]{pgfplots}

\title{%
	Manual for Package \PGFPlots\\
	{\small Version \pgfplotsversion}\\
	{\small\href{http://sourceforge.net/projects/pgfplots}{http://sourceforge.net/projects/pgfplots}}}

%\includeonly{pgfplots.libs}


\begin{document}

\def\plotcoords{%
\addplot coordinates {
(5,8.312e-02)    (17,2.547e-02)   (49,7.407e-03)
(129,2.102e-03)  (321,5.874e-04)  (769,1.623e-04)
(1793,4.442e-05) (4097,1.207e-05) (9217,3.261e-06)
};

\addplot coordinates{
(7,8.472e-02)    (31,3.044e-02)    (111,1.022e-02)
(351,3.303e-03)  (1023,1.039e-03)  (2815,3.196e-04)
(7423,9.658e-05) (18943,2.873e-05) (47103,8.437e-06)
};

\addplot coordinates{
(9,7.881e-02)     (49,3.243e-02)    (209,1.232e-02)
(769,4.454e-03)   (2561,1.551e-03)  (7937,5.236e-04)
(23297,1.723e-04) (65537,5.545e-05) (178177,1.751e-05)
};

\addplot coordinates{
(11,6.887e-02)    (71,3.177e-02)     (351,1.341e-02)
(1471,5.334e-03)  (5503,2.027e-03)   (18943,7.415e-04)
(61183,2.628e-04) (187903,9.063e-05) (553983,3.053e-05)
};

\addplot coordinates{
(13,5.755e-02)     (97,2.925e-02)     (545,1.351e-02)
(2561,5.842e-03)   (10625,2.397e-03)  (40193,9.414e-04)
(141569,3.564e-04) (471041,1.308e-04) 
(1496065,4.670e-05)
};
}%
\maketitle
\begin{abstract}%
\PGFPlots\ draws high--quality function plots in normal or logarithmic scaling with a user-friendly interface directly in \TeX. The user supplies axis labels, legend entries and the plot coordinates for one or more plots and \PGFPlots\ applies axis scaling, computes any logarithms and axis ticks and draws the plots, supporting line plots, scatter plots, piecewise constant plots, bar plots, area plots, mesh-- and surface plots and some more. It is based on Till Tantau's package \PGF/\Tikz.
\end{abstract}
\tableofcontents
\section{Introduction}
This package provides tools to generate plots and labeled axes easily. It draws normal plots, logplots and semi-logplots, in two and three dimensions. Axis ticks, labels, legends (in case of multiple plots) can be added with key-value options. It can cycle through a set of predefined line/marker/color specifications. In summary, its purpose is to simplify the generation of high-quality function and/or data plots, and solving the problems of
\begin{itemize}
	\item consistency of document and font type and font size,
	\item direct use of \TeX\ math mode in axis descriptions,
	\item consistency of data and figures (no third party tool necessary),
	\item inter document consistency using preamble configurations and styles.
\end{itemize}
Although not necessary, separate |.pdf| or |.eps| graphics can be generated using the |external| library developed as part of \Tikz.

\PGFPlots\ is build completely on \Tikz/\PGF. Knowledge of \Tikz\ will simplify the work with \PGFPlots, although it is not required.

\section[About PGFPlots: Preliminaries]{About {\normalfont\PGFPlots}: Preliminaries}
This section contains information about upgrades, the team, the installation (in case you need to do it manually) and troubleshooting. You may skip it completely except for the upgrade remarks.

However, note that this library requires at least \PGF\ version $2.00$. At the time of this writing, many \TeX-distributions still contain the older \PGF\ version $1.18$, so it may be necessary to install a recent \PGF\ prior to using \PGFPlots.

\subsection{Components}
\PGFPlots\ comes with two components:
\begin{enumerate}
	\item the plotting component (which you are currently reading) and
	\item the \PGFPlotstable\ component which simplifies number formatting and postprocessing of numerical tables. It comes as a separate package and has its own manual \href{file:pgfplotstable.pdf}{pgfplotstable.pdf}.
\end{enumerate}

\subsection{Upgrade remarks}
This release provides a lot of improvements which can be found in all detail in \texttt{ChangeLog} for interested readers. However, some attention is useful with respect to the following changes.

\subsubsection{New Optional Features}
\begin{enumerate}
	\item \PGFPlots\ 1.3 comes with user interface improvements. The technical distinction between ``behavior options'' and ``style options'' of older versions is no longer necessary (although still fully supported).

	\item \PGFPlots\ 1.3 has a new feature which allows to \emph{move axis labels tight to tick labels} automatically.

	Since this affects the spacing, it is not enabled be default.

	Use
\begin{codeexample}[code only]
\usepackage{pgfplots}
\pgfplotsset{compat=1.3}
\end{codeexample}
	in your preamble to benefit from the improved spacing\footnote{The |compat=1.3| setting is necessary for new features which move axis labels around like reversed axis. However, \PGFPlots\ will still work as it does in earlier versions, your documents will remain the same if you don't set it explicitly.}.
	Take a look at the next page for the precise definition of |compat|.

	\item \PGFPlots\ 1.3 now supports reversed axes. It is no longer necessary to use work--arounds with negative units.
\pgfkeys{/pdflinks/search key prefixes in/.add={/pgfplots/,}{}}

	Take a look at the |x dir=reverse| key.

	Existing work--arounds will still function properly. Use |\pgfplotsset{compat=1.3}| together with |x dir=reverse| to switch to the new version.
\end{enumerate}

\subsubsection{Old Features Which May Need Attention}
\begin{enumerate}
	\item The |scatter/classes| feature produces proper legends as of version 1.3. This may change the appearance of existing legends of plots with |scatter/classes|.

	\item Due to a small math bug in \PGF\ $2.00$, you \emph{can't} use the math expression `|-x^2|'. It is necessary to use `|0-x^2|' instead. The same holds for
		`|exp(-x^2)|'; use `|exp(0-x^2)|' instead.

		This will be fixed with \PGF\ $>2.00$.

	\item Starting with \PGFPlots\ $1.1$, |\tikzstyle| should \emph{no longer be used} to set \PGFPlots\ options.
	
	Although |\tikzstyle| is still supported for some older \PGFPlots\ options, you should replace any occurance of |\tikzstyle| with |\pgfplotsset{|\meta{style name}|/.style={|\meta{key-value-list}|}}| or the associated |/.append style| variant. See section~\ref{sec:styles} for more detail.
\end{enumerate}
I apologize for any inconvenience caused by these changes.

\begin{pgfplotskey}{compat=\mchoice{1.3,pre 1.3,default,newest} (initially default)}
	The preamble configuration 
\begin{codeexample}[code only]
\usepackage{pgfplots}
\pgfplotsset{compat=1.3}
\end{codeexample}
	allows to choose between backwards compatibility and most recent features.

	\PGFPlots\ version 1.3 comes with new features which may lead to different spacings compared to earlier versions:
	\begin{enumerate}
		\item Version 1.3 comes with the |xlabel near ticks| feature which places (in this case $x$) axis labels ``near'' the tick labels, taking the size of tick labels into account.
		
		Older versions used |xlabel absolute|, i.e.\ an absolute distance between the axis and axis labels (now matter how large or small tick labels are).

		The initial setting \emph{keeps backwards compatibility}.

		You are encouraged to use
\begin{codeexample}[code only]
\usepackage{pgfplots}
\pgfplotsset{compat=1.3}
\end{codeexample}
		in your preamble.
		
		\item Version 1.3 fixes a bug with |anchor=south west| which occurred mainly for reversed axes: the anchors where upside--down. This is fixed now.

		
		If you have written some work--around and want to keep it going, use

		|\pgfplotsset{compat/anchors=pre 1.3}|\pgfmanualpdflabel{/pgfplots/compat/anchors}{} 

		to disable the bug fix.
	\end{enumerate}

	Use |\pgfplotsset{compat=default}| to restore the factory settings.

	The setting |\pgfplotsset{compat=newest}| will always use features of the most recent version. This might result in small changes in the document's appearance.
\end{pgfplotskey}

\subsection{The Team}
\PGFPlots\ has been written mainly by Christian Feuers�nger with many improvements of Pascal Wolkotte and Nick Papior Andersen as a spare time project. We hope it is useful and provides valuable plots.

If you are interested in writing something but don't know how, consider reading the auxiliary manual \href{file:TeX-programming-notes.pdf}{TeX-programming-notes.pdf} which comes with \PGFPlots. It is far from complete, but maybe it is a good starting point (at least for more literature).

\subsection{Acknowledgements}
I thank God for all hours of enjoyed programming. I thank Pascal Wolkotte and Nick Papior Andersen for their programming efforts and contributions as part of the development team. I thank Stefan Tibus, who contributed the |plot shell| feature. I thank Tom Cashman for the contribution of the |reverse legend| feature. Special thanks go to Stefan Pinnow whose tests of \PGFPlots\ lead to numerous quality improvements. Furthermore, I thank Dr.~Schweitzer for many fruitful discussions and Dr.~Meine for his ideas and suggestions. Thanks as well to the many international contributors who provided feature requests or identified bugs or simply manual improvements!

Last but not least, I thank Till Tantau and Mark Wibrow for their excellent graphics (and more) package \PGF\ and \Tikz\ which is the base of \PGFPlots.

\include{pgfplots.install}%
\include{pgfplots.intro}%
\include{pgfplots.reference}%
\include{pgfplots.libs}%
\include{pgfplots.resources}%
\include{pgfplots.importexport}%
\include{pgfplots.basic.reference}%

\printindex

\bibliographystyle{abbrv} %gerapali} %gerabbrv} %gerunsrt.bst} %gerabbrv}% gerplain}
\nocite{pgfplotstable}
\nocite{programmingnotes}
\bibliography{pgfplots}
\end{document}
%
%%%%%%%%%%%%%%%%%%%%%%%%%%%%%%%%%%%%%%%%%%%%%%%%%%%%%%%%%%%%%%%%%%%%%%%%%%%%%
%
% Package pgfplots.sty documentation. 
%
% Copyright 2007/2008 by Christian Feuersaenger.
%
% This program is free software: you can redistribute it and/or modify
% it under the terms of the GNU General Public License as published by
% the Free Software Foundation, either version 3 of the License, or
% (at your option) any later version.
% 
% This program is distributed in the hope that it will be useful,
% but WITHOUT ANY WARRANTY; without even the implied warranty of
% MERCHANTABILITY or FITNESS FOR A PARTICULAR PURPOSE.  See the
% GNU General Public License for more details.
% 
% You should have received a copy of the GNU General Public License
% along with this program.  If not, see <http://www.gnu.org/licenses/>.
%
%
%%%%%%%%%%%%%%%%%%%%%%%%%%%%%%%%%%%%%%%%%%%%%%%%%%%%%%%%%%%%%%%%%%%%%%%%%%%%%
\input{pgfplots.preamble.tex}
%\RequirePackage[german,english,francais]{babel}

\pgfplotsmanualexternalexpensivetrue

%\usetikzlibrary{external}
%\tikzexternalize[prefix=figures/]{pgfplots}

\title{%
	Manual for Package \PGFPlots\\
	{\small Version \pgfplotsversion}\\
	{\small\href{http://sourceforge.net/projects/pgfplots}{http://sourceforge.net/projects/pgfplots}}}

%\includeonly{pgfplots.libs}


\begin{document}

\def\plotcoords{%
\addplot coordinates {
(5,8.312e-02)    (17,2.547e-02)   (49,7.407e-03)
(129,2.102e-03)  (321,5.874e-04)  (769,1.623e-04)
(1793,4.442e-05) (4097,1.207e-05) (9217,3.261e-06)
};

\addplot coordinates{
(7,8.472e-02)    (31,3.044e-02)    (111,1.022e-02)
(351,3.303e-03)  (1023,1.039e-03)  (2815,3.196e-04)
(7423,9.658e-05) (18943,2.873e-05) (47103,8.437e-06)
};

\addplot coordinates{
(9,7.881e-02)     (49,3.243e-02)    (209,1.232e-02)
(769,4.454e-03)   (2561,1.551e-03)  (7937,5.236e-04)
(23297,1.723e-04) (65537,5.545e-05) (178177,1.751e-05)
};

\addplot coordinates{
(11,6.887e-02)    (71,3.177e-02)     (351,1.341e-02)
(1471,5.334e-03)  (5503,2.027e-03)   (18943,7.415e-04)
(61183,2.628e-04) (187903,9.063e-05) (553983,3.053e-05)
};

\addplot coordinates{
(13,5.755e-02)     (97,2.925e-02)     (545,1.351e-02)
(2561,5.842e-03)   (10625,2.397e-03)  (40193,9.414e-04)
(141569,3.564e-04) (471041,1.308e-04) 
(1496065,4.670e-05)
};
}%
\maketitle
\begin{abstract}%
\PGFPlots\ draws high--quality function plots in normal or logarithmic scaling with a user-friendly interface directly in \TeX. The user supplies axis labels, legend entries and the plot coordinates for one or more plots and \PGFPlots\ applies axis scaling, computes any logarithms and axis ticks and draws the plots, supporting line plots, scatter plots, piecewise constant plots, bar plots, area plots, mesh-- and surface plots and some more. It is based on Till Tantau's package \PGF/\Tikz.
\end{abstract}
\tableofcontents
\section{Introduction}
This package provides tools to generate plots and labeled axes easily. It draws normal plots, logplots and semi-logplots, in two and three dimensions. Axis ticks, labels, legends (in case of multiple plots) can be added with key-value options. It can cycle through a set of predefined line/marker/color specifications. In summary, its purpose is to simplify the generation of high-quality function and/or data plots, and solving the problems of
\begin{itemize}
	\item consistency of document and font type and font size,
	\item direct use of \TeX\ math mode in axis descriptions,
	\item consistency of data and figures (no third party tool necessary),
	\item inter document consistency using preamble configurations and styles.
\end{itemize}
Although not necessary, separate |.pdf| or |.eps| graphics can be generated using the |external| library developed as part of \Tikz.

\PGFPlots\ is build completely on \Tikz/\PGF. Knowledge of \Tikz\ will simplify the work with \PGFPlots, although it is not required.

\section[About PGFPlots: Preliminaries]{About {\normalfont\PGFPlots}: Preliminaries}
This section contains information about upgrades, the team, the installation (in case you need to do it manually) and troubleshooting. You may skip it completely except for the upgrade remarks.

However, note that this library requires at least \PGF\ version $2.00$. At the time of this writing, many \TeX-distributions still contain the older \PGF\ version $1.18$, so it may be necessary to install a recent \PGF\ prior to using \PGFPlots.

\subsection{Components}
\PGFPlots\ comes with two components:
\begin{enumerate}
	\item the plotting component (which you are currently reading) and
	\item the \PGFPlotstable\ component which simplifies number formatting and postprocessing of numerical tables. It comes as a separate package and has its own manual \href{file:pgfplotstable.pdf}{pgfplotstable.pdf}.
\end{enumerate}

\subsection{Upgrade remarks}
This release provides a lot of improvements which can be found in all detail in \texttt{ChangeLog} for interested readers. However, some attention is useful with respect to the following changes.

\subsubsection{New Optional Features}
\begin{enumerate}
	\item \PGFPlots\ 1.3 comes with user interface improvements. The technical distinction between ``behavior options'' and ``style options'' of older versions is no longer necessary (although still fully supported).

	\item \PGFPlots\ 1.3 has a new feature which allows to \emph{move axis labels tight to tick labels} automatically.

	Since this affects the spacing, it is not enabled be default.

	Use
\begin{codeexample}[code only]
\usepackage{pgfplots}
\pgfplotsset{compat=1.3}
\end{codeexample}
	in your preamble to benefit from the improved spacing\footnote{The |compat=1.3| setting is necessary for new features which move axis labels around like reversed axis. However, \PGFPlots\ will still work as it does in earlier versions, your documents will remain the same if you don't set it explicitly.}.
	Take a look at the next page for the precise definition of |compat|.

	\item \PGFPlots\ 1.3 now supports reversed axes. It is no longer necessary to use work--arounds with negative units.
\pgfkeys{/pdflinks/search key prefixes in/.add={/pgfplots/,}{}}

	Take a look at the |x dir=reverse| key.

	Existing work--arounds will still function properly. Use |\pgfplotsset{compat=1.3}| together with |x dir=reverse| to switch to the new version.
\end{enumerate}

\subsubsection{Old Features Which May Need Attention}
\begin{enumerate}
	\item The |scatter/classes| feature produces proper legends as of version 1.3. This may change the appearance of existing legends of plots with |scatter/classes|.

	\item Due to a small math bug in \PGF\ $2.00$, you \emph{can't} use the math expression `|-x^2|'. It is necessary to use `|0-x^2|' instead. The same holds for
		`|exp(-x^2)|'; use `|exp(0-x^2)|' instead.

		This will be fixed with \PGF\ $>2.00$.

	\item Starting with \PGFPlots\ $1.1$, |\tikzstyle| should \emph{no longer be used} to set \PGFPlots\ options.
	
	Although |\tikzstyle| is still supported for some older \PGFPlots\ options, you should replace any occurance of |\tikzstyle| with |\pgfplotsset{|\meta{style name}|/.style={|\meta{key-value-list}|}}| or the associated |/.append style| variant. See section~\ref{sec:styles} for more detail.
\end{enumerate}
I apologize for any inconvenience caused by these changes.

\begin{pgfplotskey}{compat=\mchoice{1.3,pre 1.3,default,newest} (initially default)}
	The preamble configuration 
\begin{codeexample}[code only]
\usepackage{pgfplots}
\pgfplotsset{compat=1.3}
\end{codeexample}
	allows to choose between backwards compatibility and most recent features.

	\PGFPlots\ version 1.3 comes with new features which may lead to different spacings compared to earlier versions:
	\begin{enumerate}
		\item Version 1.3 comes with the |xlabel near ticks| feature which places (in this case $x$) axis labels ``near'' the tick labels, taking the size of tick labels into account.
		
		Older versions used |xlabel absolute|, i.e.\ an absolute distance between the axis and axis labels (now matter how large or small tick labels are).

		The initial setting \emph{keeps backwards compatibility}.

		You are encouraged to use
\begin{codeexample}[code only]
\usepackage{pgfplots}
\pgfplotsset{compat=1.3}
\end{codeexample}
		in your preamble.
		
		\item Version 1.3 fixes a bug with |anchor=south west| which occurred mainly for reversed axes: the anchors where upside--down. This is fixed now.

		
		If you have written some work--around and want to keep it going, use

		|\pgfplotsset{compat/anchors=pre 1.3}|\pgfmanualpdflabel{/pgfplots/compat/anchors}{} 

		to disable the bug fix.
	\end{enumerate}

	Use |\pgfplotsset{compat=default}| to restore the factory settings.

	The setting |\pgfplotsset{compat=newest}| will always use features of the most recent version. This might result in small changes in the document's appearance.
\end{pgfplotskey}

\subsection{The Team}
\PGFPlots\ has been written mainly by Christian Feuers�nger with many improvements of Pascal Wolkotte and Nick Papior Andersen as a spare time project. We hope it is useful and provides valuable plots.

If you are interested in writing something but don't know how, consider reading the auxiliary manual \href{file:TeX-programming-notes.pdf}{TeX-programming-notes.pdf} which comes with \PGFPlots. It is far from complete, but maybe it is a good starting point (at least for more literature).

\subsection{Acknowledgements}
I thank God for all hours of enjoyed programming. I thank Pascal Wolkotte and Nick Papior Andersen for their programming efforts and contributions as part of the development team. I thank Stefan Tibus, who contributed the |plot shell| feature. I thank Tom Cashman for the contribution of the |reverse legend| feature. Special thanks go to Stefan Pinnow whose tests of \PGFPlots\ lead to numerous quality improvements. Furthermore, I thank Dr.~Schweitzer for many fruitful discussions and Dr.~Meine for his ideas and suggestions. Thanks as well to the many international contributors who provided feature requests or identified bugs or simply manual improvements!

Last but not least, I thank Till Tantau and Mark Wibrow for their excellent graphics (and more) package \PGF\ and \Tikz\ which is the base of \PGFPlots.

\include{pgfplots.install}%
\include{pgfplots.intro}%
\include{pgfplots.reference}%
\include{pgfplots.libs}%
\include{pgfplots.resources}%
\include{pgfplots.importexport}%
\include{pgfplots.basic.reference}%

\printindex

\bibliographystyle{abbrv} %gerapali} %gerabbrv} %gerunsrt.bst} %gerabbrv}% gerplain}
\nocite{pgfplotstable}
\nocite{programmingnotes}
\bibliography{pgfplots}
\end{document}
%
%%%%%%%%%%%%%%%%%%%%%%%%%%%%%%%%%%%%%%%%%%%%%%%%%%%%%%%%%%%%%%%%%%%%%%%%%%%%%
%
% Package pgfplots.sty documentation. 
%
% Copyright 2007/2008 by Christian Feuersaenger.
%
% This program is free software: you can redistribute it and/or modify
% it under the terms of the GNU General Public License as published by
% the Free Software Foundation, either version 3 of the License, or
% (at your option) any later version.
% 
% This program is distributed in the hope that it will be useful,
% but WITHOUT ANY WARRANTY; without even the implied warranty of
% MERCHANTABILITY or FITNESS FOR A PARTICULAR PURPOSE.  See the
% GNU General Public License for more details.
% 
% You should have received a copy of the GNU General Public License
% along with this program.  If not, see <http://www.gnu.org/licenses/>.
%
%
%%%%%%%%%%%%%%%%%%%%%%%%%%%%%%%%%%%%%%%%%%%%%%%%%%%%%%%%%%%%%%%%%%%%%%%%%%%%%
\input{pgfplots.preamble.tex}
%\RequirePackage[german,english,francais]{babel}

\pgfplotsmanualexternalexpensivetrue

%\usetikzlibrary{external}
%\tikzexternalize[prefix=figures/]{pgfplots}

\title{%
	Manual for Package \PGFPlots\\
	{\small Version \pgfplotsversion}\\
	{\small\href{http://sourceforge.net/projects/pgfplots}{http://sourceforge.net/projects/pgfplots}}}

%\includeonly{pgfplots.libs}


\begin{document}

\def\plotcoords{%
\addplot coordinates {
(5,8.312e-02)    (17,2.547e-02)   (49,7.407e-03)
(129,2.102e-03)  (321,5.874e-04)  (769,1.623e-04)
(1793,4.442e-05) (4097,1.207e-05) (9217,3.261e-06)
};

\addplot coordinates{
(7,8.472e-02)    (31,3.044e-02)    (111,1.022e-02)
(351,3.303e-03)  (1023,1.039e-03)  (2815,3.196e-04)
(7423,9.658e-05) (18943,2.873e-05) (47103,8.437e-06)
};

\addplot coordinates{
(9,7.881e-02)     (49,3.243e-02)    (209,1.232e-02)
(769,4.454e-03)   (2561,1.551e-03)  (7937,5.236e-04)
(23297,1.723e-04) (65537,5.545e-05) (178177,1.751e-05)
};

\addplot coordinates{
(11,6.887e-02)    (71,3.177e-02)     (351,1.341e-02)
(1471,5.334e-03)  (5503,2.027e-03)   (18943,7.415e-04)
(61183,2.628e-04) (187903,9.063e-05) (553983,3.053e-05)
};

\addplot coordinates{
(13,5.755e-02)     (97,2.925e-02)     (545,1.351e-02)
(2561,5.842e-03)   (10625,2.397e-03)  (40193,9.414e-04)
(141569,3.564e-04) (471041,1.308e-04) 
(1496065,4.670e-05)
};
}%
\maketitle
\begin{abstract}%
\PGFPlots\ draws high--quality function plots in normal or logarithmic scaling with a user-friendly interface directly in \TeX. The user supplies axis labels, legend entries and the plot coordinates for one or more plots and \PGFPlots\ applies axis scaling, computes any logarithms and axis ticks and draws the plots, supporting line plots, scatter plots, piecewise constant plots, bar plots, area plots, mesh-- and surface plots and some more. It is based on Till Tantau's package \PGF/\Tikz.
\end{abstract}
\tableofcontents
\section{Introduction}
This package provides tools to generate plots and labeled axes easily. It draws normal plots, logplots and semi-logplots, in two and three dimensions. Axis ticks, labels, legends (in case of multiple plots) can be added with key-value options. It can cycle through a set of predefined line/marker/color specifications. In summary, its purpose is to simplify the generation of high-quality function and/or data plots, and solving the problems of
\begin{itemize}
	\item consistency of document and font type and font size,
	\item direct use of \TeX\ math mode in axis descriptions,
	\item consistency of data and figures (no third party tool necessary),
	\item inter document consistency using preamble configurations and styles.
\end{itemize}
Although not necessary, separate |.pdf| or |.eps| graphics can be generated using the |external| library developed as part of \Tikz.

\PGFPlots\ is build completely on \Tikz/\PGF. Knowledge of \Tikz\ will simplify the work with \PGFPlots, although it is not required.

\section[About PGFPlots: Preliminaries]{About {\normalfont\PGFPlots}: Preliminaries}
This section contains information about upgrades, the team, the installation (in case you need to do it manually) and troubleshooting. You may skip it completely except for the upgrade remarks.

However, note that this library requires at least \PGF\ version $2.00$. At the time of this writing, many \TeX-distributions still contain the older \PGF\ version $1.18$, so it may be necessary to install a recent \PGF\ prior to using \PGFPlots.

\subsection{Components}
\PGFPlots\ comes with two components:
\begin{enumerate}
	\item the plotting component (which you are currently reading) and
	\item the \PGFPlotstable\ component which simplifies number formatting and postprocessing of numerical tables. It comes as a separate package and has its own manual \href{file:pgfplotstable.pdf}{pgfplotstable.pdf}.
\end{enumerate}

\subsection{Upgrade remarks}
This release provides a lot of improvements which can be found in all detail in \texttt{ChangeLog} for interested readers. However, some attention is useful with respect to the following changes.

\subsubsection{New Optional Features}
\begin{enumerate}
	\item \PGFPlots\ 1.3 comes with user interface improvements. The technical distinction between ``behavior options'' and ``style options'' of older versions is no longer necessary (although still fully supported).

	\item \PGFPlots\ 1.3 has a new feature which allows to \emph{move axis labels tight to tick labels} automatically.

	Since this affects the spacing, it is not enabled be default.

	Use
\begin{codeexample}[code only]
\usepackage{pgfplots}
\pgfplotsset{compat=1.3}
\end{codeexample}
	in your preamble to benefit from the improved spacing\footnote{The |compat=1.3| setting is necessary for new features which move axis labels around like reversed axis. However, \PGFPlots\ will still work as it does in earlier versions, your documents will remain the same if you don't set it explicitly.}.
	Take a look at the next page for the precise definition of |compat|.

	\item \PGFPlots\ 1.3 now supports reversed axes. It is no longer necessary to use work--arounds with negative units.
\pgfkeys{/pdflinks/search key prefixes in/.add={/pgfplots/,}{}}

	Take a look at the |x dir=reverse| key.

	Existing work--arounds will still function properly. Use |\pgfplotsset{compat=1.3}| together with |x dir=reverse| to switch to the new version.
\end{enumerate}

\subsubsection{Old Features Which May Need Attention}
\begin{enumerate}
	\item The |scatter/classes| feature produces proper legends as of version 1.3. This may change the appearance of existing legends of plots with |scatter/classes|.

	\item Due to a small math bug in \PGF\ $2.00$, you \emph{can't} use the math expression `|-x^2|'. It is necessary to use `|0-x^2|' instead. The same holds for
		`|exp(-x^2)|'; use `|exp(0-x^2)|' instead.

		This will be fixed with \PGF\ $>2.00$.

	\item Starting with \PGFPlots\ $1.1$, |\tikzstyle| should \emph{no longer be used} to set \PGFPlots\ options.
	
	Although |\tikzstyle| is still supported for some older \PGFPlots\ options, you should replace any occurance of |\tikzstyle| with |\pgfplotsset{|\meta{style name}|/.style={|\meta{key-value-list}|}}| or the associated |/.append style| variant. See section~\ref{sec:styles} for more detail.
\end{enumerate}
I apologize for any inconvenience caused by these changes.

\begin{pgfplotskey}{compat=\mchoice{1.3,pre 1.3,default,newest} (initially default)}
	The preamble configuration 
\begin{codeexample}[code only]
\usepackage{pgfplots}
\pgfplotsset{compat=1.3}
\end{codeexample}
	allows to choose between backwards compatibility and most recent features.

	\PGFPlots\ version 1.3 comes with new features which may lead to different spacings compared to earlier versions:
	\begin{enumerate}
		\item Version 1.3 comes with the |xlabel near ticks| feature which places (in this case $x$) axis labels ``near'' the tick labels, taking the size of tick labels into account.
		
		Older versions used |xlabel absolute|, i.e.\ an absolute distance between the axis and axis labels (now matter how large or small tick labels are).

		The initial setting \emph{keeps backwards compatibility}.

		You are encouraged to use
\begin{codeexample}[code only]
\usepackage{pgfplots}
\pgfplotsset{compat=1.3}
\end{codeexample}
		in your preamble.
		
		\item Version 1.3 fixes a bug with |anchor=south west| which occurred mainly for reversed axes: the anchors where upside--down. This is fixed now.

		
		If you have written some work--around and want to keep it going, use

		|\pgfplotsset{compat/anchors=pre 1.3}|\pgfmanualpdflabel{/pgfplots/compat/anchors}{} 

		to disable the bug fix.
	\end{enumerate}

	Use |\pgfplotsset{compat=default}| to restore the factory settings.

	The setting |\pgfplotsset{compat=newest}| will always use features of the most recent version. This might result in small changes in the document's appearance.
\end{pgfplotskey}

\subsection{The Team}
\PGFPlots\ has been written mainly by Christian Feuers�nger with many improvements of Pascal Wolkotte and Nick Papior Andersen as a spare time project. We hope it is useful and provides valuable plots.

If you are interested in writing something but don't know how, consider reading the auxiliary manual \href{file:TeX-programming-notes.pdf}{TeX-programming-notes.pdf} which comes with \PGFPlots. It is far from complete, but maybe it is a good starting point (at least for more literature).

\subsection{Acknowledgements}
I thank God for all hours of enjoyed programming. I thank Pascal Wolkotte and Nick Papior Andersen for their programming efforts and contributions as part of the development team. I thank Stefan Tibus, who contributed the |plot shell| feature. I thank Tom Cashman for the contribution of the |reverse legend| feature. Special thanks go to Stefan Pinnow whose tests of \PGFPlots\ lead to numerous quality improvements. Furthermore, I thank Dr.~Schweitzer for many fruitful discussions and Dr.~Meine for his ideas and suggestions. Thanks as well to the many international contributors who provided feature requests or identified bugs or simply manual improvements!

Last but not least, I thank Till Tantau and Mark Wibrow for their excellent graphics (and more) package \PGF\ and \Tikz\ which is the base of \PGFPlots.

\include{pgfplots.install}%
\include{pgfplots.intro}%
\include{pgfplots.reference}%
\include{pgfplots.libs}%
\include{pgfplots.resources}%
\include{pgfplots.importexport}%
\include{pgfplots.basic.reference}%

\printindex

\bibliographystyle{abbrv} %gerapali} %gerabbrv} %gerunsrt.bst} %gerabbrv}% gerplain}
\nocite{pgfplotstable}
\nocite{programmingnotes}
\bibliography{pgfplots}
\end{document}
%
%%%%%%%%%%%%%%%%%%%%%%%%%%%%%%%%%%%%%%%%%%%%%%%%%%%%%%%%%%%%%%%%%%%%%%%%%%%%%
%
% Package pgfplots.sty documentation. 
%
% Copyright 2007/2008 by Christian Feuersaenger.
%
% This program is free software: you can redistribute it and/or modify
% it under the terms of the GNU General Public License as published by
% the Free Software Foundation, either version 3 of the License, or
% (at your option) any later version.
% 
% This program is distributed in the hope that it will be useful,
% but WITHOUT ANY WARRANTY; without even the implied warranty of
% MERCHANTABILITY or FITNESS FOR A PARTICULAR PURPOSE.  See the
% GNU General Public License for more details.
% 
% You should have received a copy of the GNU General Public License
% along with this program.  If not, see <http://www.gnu.org/licenses/>.
%
%
%%%%%%%%%%%%%%%%%%%%%%%%%%%%%%%%%%%%%%%%%%%%%%%%%%%%%%%%%%%%%%%%%%%%%%%%%%%%%
\input{pgfplots.preamble.tex}
%\RequirePackage[german,english,francais]{babel}

\pgfplotsmanualexternalexpensivetrue

%\usetikzlibrary{external}
%\tikzexternalize[prefix=figures/]{pgfplots}

\title{%
	Manual for Package \PGFPlots\\
	{\small Version \pgfplotsversion}\\
	{\small\href{http://sourceforge.net/projects/pgfplots}{http://sourceforge.net/projects/pgfplots}}}

%\includeonly{pgfplots.libs}


\begin{document}

\def\plotcoords{%
\addplot coordinates {
(5,8.312e-02)    (17,2.547e-02)   (49,7.407e-03)
(129,2.102e-03)  (321,5.874e-04)  (769,1.623e-04)
(1793,4.442e-05) (4097,1.207e-05) (9217,3.261e-06)
};

\addplot coordinates{
(7,8.472e-02)    (31,3.044e-02)    (111,1.022e-02)
(351,3.303e-03)  (1023,1.039e-03)  (2815,3.196e-04)
(7423,9.658e-05) (18943,2.873e-05) (47103,8.437e-06)
};

\addplot coordinates{
(9,7.881e-02)     (49,3.243e-02)    (209,1.232e-02)
(769,4.454e-03)   (2561,1.551e-03)  (7937,5.236e-04)
(23297,1.723e-04) (65537,5.545e-05) (178177,1.751e-05)
};

\addplot coordinates{
(11,6.887e-02)    (71,3.177e-02)     (351,1.341e-02)
(1471,5.334e-03)  (5503,2.027e-03)   (18943,7.415e-04)
(61183,2.628e-04) (187903,9.063e-05) (553983,3.053e-05)
};

\addplot coordinates{
(13,5.755e-02)     (97,2.925e-02)     (545,1.351e-02)
(2561,5.842e-03)   (10625,2.397e-03)  (40193,9.414e-04)
(141569,3.564e-04) (471041,1.308e-04) 
(1496065,4.670e-05)
};
}%
\maketitle
\begin{abstract}%
\PGFPlots\ draws high--quality function plots in normal or logarithmic scaling with a user-friendly interface directly in \TeX. The user supplies axis labels, legend entries and the plot coordinates for one or more plots and \PGFPlots\ applies axis scaling, computes any logarithms and axis ticks and draws the plots, supporting line plots, scatter plots, piecewise constant plots, bar plots, area plots, mesh-- and surface plots and some more. It is based on Till Tantau's package \PGF/\Tikz.
\end{abstract}
\tableofcontents
\section{Introduction}
This package provides tools to generate plots and labeled axes easily. It draws normal plots, logplots and semi-logplots, in two and three dimensions. Axis ticks, labels, legends (in case of multiple plots) can be added with key-value options. It can cycle through a set of predefined line/marker/color specifications. In summary, its purpose is to simplify the generation of high-quality function and/or data plots, and solving the problems of
\begin{itemize}
	\item consistency of document and font type and font size,
	\item direct use of \TeX\ math mode in axis descriptions,
	\item consistency of data and figures (no third party tool necessary),
	\item inter document consistency using preamble configurations and styles.
\end{itemize}
Although not necessary, separate |.pdf| or |.eps| graphics can be generated using the |external| library developed as part of \Tikz.

\PGFPlots\ is build completely on \Tikz/\PGF. Knowledge of \Tikz\ will simplify the work with \PGFPlots, although it is not required.

\section[About PGFPlots: Preliminaries]{About {\normalfont\PGFPlots}: Preliminaries}
This section contains information about upgrades, the team, the installation (in case you need to do it manually) and troubleshooting. You may skip it completely except for the upgrade remarks.

However, note that this library requires at least \PGF\ version $2.00$. At the time of this writing, many \TeX-distributions still contain the older \PGF\ version $1.18$, so it may be necessary to install a recent \PGF\ prior to using \PGFPlots.

\subsection{Components}
\PGFPlots\ comes with two components:
\begin{enumerate}
	\item the plotting component (which you are currently reading) and
	\item the \PGFPlotstable\ component which simplifies number formatting and postprocessing of numerical tables. It comes as a separate package and has its own manual \href{file:pgfplotstable.pdf}{pgfplotstable.pdf}.
\end{enumerate}

\subsection{Upgrade remarks}
This release provides a lot of improvements which can be found in all detail in \texttt{ChangeLog} for interested readers. However, some attention is useful with respect to the following changes.

\subsubsection{New Optional Features}
\begin{enumerate}
	\item \PGFPlots\ 1.3 comes with user interface improvements. The technical distinction between ``behavior options'' and ``style options'' of older versions is no longer necessary (although still fully supported).

	\item \PGFPlots\ 1.3 has a new feature which allows to \emph{move axis labels tight to tick labels} automatically.

	Since this affects the spacing, it is not enabled be default.

	Use
\begin{codeexample}[code only]
\usepackage{pgfplots}
\pgfplotsset{compat=1.3}
\end{codeexample}
	in your preamble to benefit from the improved spacing\footnote{The |compat=1.3| setting is necessary for new features which move axis labels around like reversed axis. However, \PGFPlots\ will still work as it does in earlier versions, your documents will remain the same if you don't set it explicitly.}.
	Take a look at the next page for the precise definition of |compat|.

	\item \PGFPlots\ 1.3 now supports reversed axes. It is no longer necessary to use work--arounds with negative units.
\pgfkeys{/pdflinks/search key prefixes in/.add={/pgfplots/,}{}}

	Take a look at the |x dir=reverse| key.

	Existing work--arounds will still function properly. Use |\pgfplotsset{compat=1.3}| together with |x dir=reverse| to switch to the new version.
\end{enumerate}

\subsubsection{Old Features Which May Need Attention}
\begin{enumerate}
	\item The |scatter/classes| feature produces proper legends as of version 1.3. This may change the appearance of existing legends of plots with |scatter/classes|.

	\item Due to a small math bug in \PGF\ $2.00$, you \emph{can't} use the math expression `|-x^2|'. It is necessary to use `|0-x^2|' instead. The same holds for
		`|exp(-x^2)|'; use `|exp(0-x^2)|' instead.

		This will be fixed with \PGF\ $>2.00$.

	\item Starting with \PGFPlots\ $1.1$, |\tikzstyle| should \emph{no longer be used} to set \PGFPlots\ options.
	
	Although |\tikzstyle| is still supported for some older \PGFPlots\ options, you should replace any occurance of |\tikzstyle| with |\pgfplotsset{|\meta{style name}|/.style={|\meta{key-value-list}|}}| or the associated |/.append style| variant. See section~\ref{sec:styles} for more detail.
\end{enumerate}
I apologize for any inconvenience caused by these changes.

\begin{pgfplotskey}{compat=\mchoice{1.3,pre 1.3,default,newest} (initially default)}
	The preamble configuration 
\begin{codeexample}[code only]
\usepackage{pgfplots}
\pgfplotsset{compat=1.3}
\end{codeexample}
	allows to choose between backwards compatibility and most recent features.

	\PGFPlots\ version 1.3 comes with new features which may lead to different spacings compared to earlier versions:
	\begin{enumerate}
		\item Version 1.3 comes with the |xlabel near ticks| feature which places (in this case $x$) axis labels ``near'' the tick labels, taking the size of tick labels into account.
		
		Older versions used |xlabel absolute|, i.e.\ an absolute distance between the axis and axis labels (now matter how large or small tick labels are).

		The initial setting \emph{keeps backwards compatibility}.

		You are encouraged to use
\begin{codeexample}[code only]
\usepackage{pgfplots}
\pgfplotsset{compat=1.3}
\end{codeexample}
		in your preamble.
		
		\item Version 1.3 fixes a bug with |anchor=south west| which occurred mainly for reversed axes: the anchors where upside--down. This is fixed now.

		
		If you have written some work--around and want to keep it going, use

		|\pgfplotsset{compat/anchors=pre 1.3}|\pgfmanualpdflabel{/pgfplots/compat/anchors}{} 

		to disable the bug fix.
	\end{enumerate}

	Use |\pgfplotsset{compat=default}| to restore the factory settings.

	The setting |\pgfplotsset{compat=newest}| will always use features of the most recent version. This might result in small changes in the document's appearance.
\end{pgfplotskey}

\subsection{The Team}
\PGFPlots\ has been written mainly by Christian Feuers�nger with many improvements of Pascal Wolkotte and Nick Papior Andersen as a spare time project. We hope it is useful and provides valuable plots.

If you are interested in writing something but don't know how, consider reading the auxiliary manual \href{file:TeX-programming-notes.pdf}{TeX-programming-notes.pdf} which comes with \PGFPlots. It is far from complete, but maybe it is a good starting point (at least for more literature).

\subsection{Acknowledgements}
I thank God for all hours of enjoyed programming. I thank Pascal Wolkotte and Nick Papior Andersen for their programming efforts and contributions as part of the development team. I thank Stefan Tibus, who contributed the |plot shell| feature. I thank Tom Cashman for the contribution of the |reverse legend| feature. Special thanks go to Stefan Pinnow whose tests of \PGFPlots\ lead to numerous quality improvements. Furthermore, I thank Dr.~Schweitzer for many fruitful discussions and Dr.~Meine for his ideas and suggestions. Thanks as well to the many international contributors who provided feature requests or identified bugs or simply manual improvements!

Last but not least, I thank Till Tantau and Mark Wibrow for their excellent graphics (and more) package \PGF\ and \Tikz\ which is the base of \PGFPlots.

\include{pgfplots.install}%
\include{pgfplots.intro}%
\include{pgfplots.reference}%
\include{pgfplots.libs}%
\include{pgfplots.resources}%
\include{pgfplots.importexport}%
\include{pgfplots.basic.reference}%

\printindex

\bibliographystyle{abbrv} %gerapali} %gerabbrv} %gerunsrt.bst} %gerabbrv}% gerplain}
\nocite{pgfplotstable}
\nocite{programmingnotes}
\bibliography{pgfplots}
\end{document}
%
\include{pgfplots.basic.reference}%

\printindex

\bibliographystyle{abbrv} %gerapali} %gerabbrv} %gerunsrt.bst} %gerabbrv}% gerplain}
\nocite{pgfplotstable}
\nocite{programmingnotes}
\bibliography{pgfplots}
\end{document}
%
%%%%%%%%%%%%%%%%%%%%%%%%%%%%%%%%%%%%%%%%%%%%%%%%%%%%%%%%%%%%%%%%%%%%%%%%%%%%%
%
% Package pgfplots.sty documentation. 
%
% Copyright 2007/2008 by Christian Feuersaenger.
%
% This program is free software: you can redistribute it and/or modify
% it under the terms of the GNU General Public License as published by
% the Free Software Foundation, either version 3 of the License, or
% (at your option) any later version.
% 
% This program is distributed in the hope that it will be useful,
% but WITHOUT ANY WARRANTY; without even the implied warranty of
% MERCHANTABILITY or FITNESS FOR A PARTICULAR PURPOSE.  See the
% GNU General Public License for more details.
% 
% You should have received a copy of the GNU General Public License
% along with this program.  If not, see <http://www.gnu.org/licenses/>.
%
%
%%%%%%%%%%%%%%%%%%%%%%%%%%%%%%%%%%%%%%%%%%%%%%%%%%%%%%%%%%%%%%%%%%%%%%%%%%%%%
%%%%%%%%%%%%%%%%%%%%%%%%%%%%%%%%%%%%%%%%%%%%%%%%%%%%%%%%%%%%%%%%%%%%%%%%%%%%%
%
% Package pgfplots.sty documentation. 
%
% Copyright 2007/2008 by Christian Feuersaenger.
%
% This program is free software: you can redistribute it and/or modify
% it under the terms of the GNU General Public License as published by
% the Free Software Foundation, either version 3 of the License, or
% (at your option) any later version.
% 
% This program is distributed in the hope that it will be useful,
% but WITHOUT ANY WARRANTY; without even the implied warranty of
% MERCHANTABILITY or FITNESS FOR A PARTICULAR PURPOSE.  See the
% GNU General Public License for more details.
% 
% You should have received a copy of the GNU General Public License
% along with this program.  If not, see <http://www.gnu.org/licenses/>.
%
%
%%%%%%%%%%%%%%%%%%%%%%%%%%%%%%%%%%%%%%%%%%%%%%%%%%%%%%%%%%%%%%%%%%%%%%%%%%%%%
\documentclass[a4paper]{ltxdoc}

\usepackage{makeidx}

% DON't let hyperref overload the format of index and glossary. 
% I want to do that on my own in the stylefiles for makeindex...
\makeatletter
\let\@old@wrindex=\@wrindex
\makeatother

\usepackage{ifpdf}
\ifpdf
	\usepackage[pdftex]{hyperref}
\else
	\usepackage[dvipdfm]{hyperref}
\fi
\hypersetup{pdfborder={0 0 0}}

\makeatletter
\let\@wrindex=\@old@wrindex
\makeatother

% Formatiere Seitennummern im Index:
\newcommand{\indexpageno}[1]{%
	{\bfseries\hyperpage{#1}}%
}


\newcommand{\R}{\mathbb{R}}
\newcommand{\N}{\mathbb{N}}
\newcommand{\Z}{\mathbb{Z}}

\long\def\COMMENTLOWLEVEL#1\ENDCOMMENT{}
\def\ENDCOMMENT{}

\usepackage{textcomp}
\usepackage{booktabs}

\usepackage{calc}
\usepackage[formats]{listings}
%\usepackage{courier} % don't use it - the '^' character can't be copy-pasted in courier

\usepackage{array}
\lstset{%
	basicstyle=\ttfamily,
	language=[LaTeX]tex, % Seems as if \lstset{language=tex} must be invoked BEFORE loading tikz!?
	tabsize=4,
	breaklines=true,
	breakindent=0pt
}

\ifpdf
	\pdfinfo {
		/Author	(Christian Feuersaenger)
	}

\else
	\def\pgfsysdriver{pgfsys-dvipdfm.def}
\fi
%\def\pgfsysdriver{pgfsys-pdftex.def}
\usepackage{pgfplots}

\ifpdf
	\usetikzlibrary{pgfplotsclickable}
\fi

%\usepackage{fp}
% ATTENTION:
% this requires pgf version NEWER than 2.00 :
%\usetikzlibrary{fixedpointarithmetic}

\usetikzlibrary{dateplot}

\usepackage[a4paper,left=2.25cm,right=2.25cm,top=2.5cm,bottom=2.5cm,nohead]{geometry}
\usepackage{amsmath,amssymb}
\usepackage{xxcolor}
\usepackage{pifont}
\usepackage[latin1]{inputenc}
\usepackage{amsmath}
\usepackage{eurosym}
\input{pgfplots-macros}

\pgfqkeys{/codeexample}{%
	every codeexample/.style={
		width=8cm,
		/pgfplots/every axis/.append style={legend style={fill=graphicbackground}}
	},
	tabsize=4,
}
\pgfplotsset{
	every axis/.append style={width=7cm},
	filter discard warning=false,
}

\usetikzlibrary{backgrounds,patterns}
% Global styles:
\tikzset{
  shape example/.style={
    color=black!30,
    draw,
    fill=yellow!30,
    line width=.5cm,
    inner xsep=2.5cm,
    inner ysep=0.5cm}
}

\newcommand{\FIXME}[1]{\textcolor{red}{(FIXME: #1)}}

% fuer endvironment 'sidewaysfigure' bspw
% \usepackage{rotating}

\newcommand\Tikz{Ti\textit kZ}
\newcommand\PGF{\textsc{pgf}}
\newcommand\PGFPlots{\textsc{pgfplots}}
\def\PGFPlotstable{\textsc{PgfplotsTable}}


\makeindex

% Fix overful hboxes automatically:
\tolerance=2000
\emergencystretch=10pt

\tikzset{prefix=gnuplot/pgfplots_} % prefix for 'plot function'

\author{%
	Christian Feuers\"anger\footnote{\url{http://wissrech.ins.uni-bonn.de/people/feuersaenger}}\\%
	Institut f\"ur Numerische Simulation\\
	Universit\"at Bonn, Germany}


%\RequirePackage[german,english,francais]{babel}

\pgfplotsmanualexternalexpensivetrue

%\usetikzlibrary{external}
%\tikzexternalize[prefix=figures/]{pgfplots}

\title{%
	Manual for Package \PGFPlots\\
	{\small Version \pgfplotsversion}\\
	{\small\href{http://sourceforge.net/projects/pgfplots}{http://sourceforge.net/projects/pgfplots}}}

%\includeonly{pgfplots.libs}


\begin{document}

\def\plotcoords{%
\addplot coordinates {
(5,8.312e-02)    (17,2.547e-02)   (49,7.407e-03)
(129,2.102e-03)  (321,5.874e-04)  (769,1.623e-04)
(1793,4.442e-05) (4097,1.207e-05) (9217,3.261e-06)
};

\addplot coordinates{
(7,8.472e-02)    (31,3.044e-02)    (111,1.022e-02)
(351,3.303e-03)  (1023,1.039e-03)  (2815,3.196e-04)
(7423,9.658e-05) (18943,2.873e-05) (47103,8.437e-06)
};

\addplot coordinates{
(9,7.881e-02)     (49,3.243e-02)    (209,1.232e-02)
(769,4.454e-03)   (2561,1.551e-03)  (7937,5.236e-04)
(23297,1.723e-04) (65537,5.545e-05) (178177,1.751e-05)
};

\addplot coordinates{
(11,6.887e-02)    (71,3.177e-02)     (351,1.341e-02)
(1471,5.334e-03)  (5503,2.027e-03)   (18943,7.415e-04)
(61183,2.628e-04) (187903,9.063e-05) (553983,3.053e-05)
};

\addplot coordinates{
(13,5.755e-02)     (97,2.925e-02)     (545,1.351e-02)
(2561,5.842e-03)   (10625,2.397e-03)  (40193,9.414e-04)
(141569,3.564e-04) (471041,1.308e-04) 
(1496065,4.670e-05)
};
}%
\maketitle
\begin{abstract}%
\PGFPlots\ draws high--quality function plots in normal or logarithmic scaling with a user-friendly interface directly in \TeX. The user supplies axis labels, legend entries and the plot coordinates for one or more plots and \PGFPlots\ applies axis scaling, computes any logarithms and axis ticks and draws the plots, supporting line plots, scatter plots, piecewise constant plots, bar plots, area plots, mesh-- and surface plots and some more. It is based on Till Tantau's package \PGF/\Tikz.
\end{abstract}
\tableofcontents
\section{Introduction}
This package provides tools to generate plots and labeled axes easily. It draws normal plots, logplots and semi-logplots, in two and three dimensions. Axis ticks, labels, legends (in case of multiple plots) can be added with key-value options. It can cycle through a set of predefined line/marker/color specifications. In summary, its purpose is to simplify the generation of high-quality function and/or data plots, and solving the problems of
\begin{itemize}
	\item consistency of document and font type and font size,
	\item direct use of \TeX\ math mode in axis descriptions,
	\item consistency of data and figures (no third party tool necessary),
	\item inter document consistency using preamble configurations and styles.
\end{itemize}
Although not necessary, separate |.pdf| or |.eps| graphics can be generated using the |external| library developed as part of \Tikz.

\PGFPlots\ is build completely on \Tikz/\PGF. Knowledge of \Tikz\ will simplify the work with \PGFPlots, although it is not required.

\section[About PGFPlots: Preliminaries]{About {\normalfont\PGFPlots}: Preliminaries}
This section contains information about upgrades, the team, the installation (in case you need to do it manually) and troubleshooting. You may skip it completely except for the upgrade remarks.

However, note that this library requires at least \PGF\ version $2.00$. At the time of this writing, many \TeX-distributions still contain the older \PGF\ version $1.18$, so it may be necessary to install a recent \PGF\ prior to using \PGFPlots.

\subsection{Components}
\PGFPlots\ comes with two components:
\begin{enumerate}
	\item the plotting component (which you are currently reading) and
	\item the \PGFPlotstable\ component which simplifies number formatting and postprocessing of numerical tables. It comes as a separate package and has its own manual \href{file:pgfplotstable.pdf}{pgfplotstable.pdf}.
\end{enumerate}

\subsection{Upgrade remarks}
This release provides a lot of improvements which can be found in all detail in \texttt{ChangeLog} for interested readers. However, some attention is useful with respect to the following changes.

\subsubsection{New Optional Features}
\begin{enumerate}
	\item \PGFPlots\ 1.3 comes with user interface improvements. The technical distinction between ``behavior options'' and ``style options'' of older versions is no longer necessary (although still fully supported).

	\item \PGFPlots\ 1.3 has a new feature which allows to \emph{move axis labels tight to tick labels} automatically.

	Since this affects the spacing, it is not enabled be default.

	Use
\begin{codeexample}[code only]
\usepackage{pgfplots}
\pgfplotsset{compat=1.3}
\end{codeexample}
	in your preamble to benefit from the improved spacing\footnote{The |compat=1.3| setting is necessary for new features which move axis labels around like reversed axis. However, \PGFPlots\ will still work as it does in earlier versions, your documents will remain the same if you don't set it explicitly.}.
	Take a look at the next page for the precise definition of |compat|.

	\item \PGFPlots\ 1.3 now supports reversed axes. It is no longer necessary to use work--arounds with negative units.
\pgfkeys{/pdflinks/search key prefixes in/.add={/pgfplots/,}{}}

	Take a look at the |x dir=reverse| key.

	Existing work--arounds will still function properly. Use |\pgfplotsset{compat=1.3}| together with |x dir=reverse| to switch to the new version.
\end{enumerate}

\subsubsection{Old Features Which May Need Attention}
\begin{enumerate}
	\item The |scatter/classes| feature produces proper legends as of version 1.3. This may change the appearance of existing legends of plots with |scatter/classes|.

	\item Due to a small math bug in \PGF\ $2.00$, you \emph{can't} use the math expression `|-x^2|'. It is necessary to use `|0-x^2|' instead. The same holds for
		`|exp(-x^2)|'; use `|exp(0-x^2)|' instead.

		This will be fixed with \PGF\ $>2.00$.

	\item Starting with \PGFPlots\ $1.1$, |\tikzstyle| should \emph{no longer be used} to set \PGFPlots\ options.
	
	Although |\tikzstyle| is still supported for some older \PGFPlots\ options, you should replace any occurance of |\tikzstyle| with |\pgfplotsset{|\meta{style name}|/.style={|\meta{key-value-list}|}}| or the associated |/.append style| variant. See section~\ref{sec:styles} for more detail.
\end{enumerate}
I apologize for any inconvenience caused by these changes.

\begin{pgfplotskey}{compat=\mchoice{1.3,pre 1.3,default,newest} (initially default)}
	The preamble configuration 
\begin{codeexample}[code only]
\usepackage{pgfplots}
\pgfplotsset{compat=1.3}
\end{codeexample}
	allows to choose between backwards compatibility and most recent features.

	\PGFPlots\ version 1.3 comes with new features which may lead to different spacings compared to earlier versions:
	\begin{enumerate}
		\item Version 1.3 comes with the |xlabel near ticks| feature which places (in this case $x$) axis labels ``near'' the tick labels, taking the size of tick labels into account.
		
		Older versions used |xlabel absolute|, i.e.\ an absolute distance between the axis and axis labels (now matter how large or small tick labels are).

		The initial setting \emph{keeps backwards compatibility}.

		You are encouraged to use
\begin{codeexample}[code only]
\usepackage{pgfplots}
\pgfplotsset{compat=1.3}
\end{codeexample}
		in your preamble.
		
		\item Version 1.3 fixes a bug with |anchor=south west| which occurred mainly for reversed axes: the anchors where upside--down. This is fixed now.

		
		If you have written some work--around and want to keep it going, use

		|\pgfplotsset{compat/anchors=pre 1.3}|\pgfmanualpdflabel{/pgfplots/compat/anchors}{} 

		to disable the bug fix.
	\end{enumerate}

	Use |\pgfplotsset{compat=default}| to restore the factory settings.

	The setting |\pgfplotsset{compat=newest}| will always use features of the most recent version. This might result in small changes in the document's appearance.
\end{pgfplotskey}

\subsection{The Team}
\PGFPlots\ has been written mainly by Christian Feuers�nger with many improvements of Pascal Wolkotte and Nick Papior Andersen as a spare time project. We hope it is useful and provides valuable plots.

If you are interested in writing something but don't know how, consider reading the auxiliary manual \href{file:TeX-programming-notes.pdf}{TeX-programming-notes.pdf} which comes with \PGFPlots. It is far from complete, but maybe it is a good starting point (at least for more literature).

\subsection{Acknowledgements}
I thank God for all hours of enjoyed programming. I thank Pascal Wolkotte and Nick Papior Andersen for their programming efforts and contributions as part of the development team. I thank Stefan Tibus, who contributed the |plot shell| feature. I thank Tom Cashman for the contribution of the |reverse legend| feature. Special thanks go to Stefan Pinnow whose tests of \PGFPlots\ lead to numerous quality improvements. Furthermore, I thank Dr.~Schweitzer for many fruitful discussions and Dr.~Meine for his ideas and suggestions. Thanks as well to the many international contributors who provided feature requests or identified bugs or simply manual improvements!

Last but not least, I thank Till Tantau and Mark Wibrow for their excellent graphics (and more) package \PGF\ and \Tikz\ which is the base of \PGFPlots.

%%%%%%%%%%%%%%%%%%%%%%%%%%%%%%%%%%%%%%%%%%%%%%%%%%%%%%%%%%%%%%%%%%%%%%%%%%%%%
%
% Package pgfplots.sty documentation. 
%
% Copyright 2007/2008 by Christian Feuersaenger.
%
% This program is free software: you can redistribute it and/or modify
% it under the terms of the GNU General Public License as published by
% the Free Software Foundation, either version 3 of the License, or
% (at your option) any later version.
% 
% This program is distributed in the hope that it will be useful,
% but WITHOUT ANY WARRANTY; without even the implied warranty of
% MERCHANTABILITY or FITNESS FOR A PARTICULAR PURPOSE.  See the
% GNU General Public License for more details.
% 
% You should have received a copy of the GNU General Public License
% along with this program.  If not, see <http://www.gnu.org/licenses/>.
%
%
%%%%%%%%%%%%%%%%%%%%%%%%%%%%%%%%%%%%%%%%%%%%%%%%%%%%%%%%%%%%%%%%%%%%%%%%%%%%%
\input{pgfplots.preamble.tex}
%\RequirePackage[german,english,francais]{babel}

\pgfplotsmanualexternalexpensivetrue

%\usetikzlibrary{external}
%\tikzexternalize[prefix=figures/]{pgfplots}

\title{%
	Manual for Package \PGFPlots\\
	{\small Version \pgfplotsversion}\\
	{\small\href{http://sourceforge.net/projects/pgfplots}{http://sourceforge.net/projects/pgfplots}}}

%\includeonly{pgfplots.libs}


\begin{document}

\def\plotcoords{%
\addplot coordinates {
(5,8.312e-02)    (17,2.547e-02)   (49,7.407e-03)
(129,2.102e-03)  (321,5.874e-04)  (769,1.623e-04)
(1793,4.442e-05) (4097,1.207e-05) (9217,3.261e-06)
};

\addplot coordinates{
(7,8.472e-02)    (31,3.044e-02)    (111,1.022e-02)
(351,3.303e-03)  (1023,1.039e-03)  (2815,3.196e-04)
(7423,9.658e-05) (18943,2.873e-05) (47103,8.437e-06)
};

\addplot coordinates{
(9,7.881e-02)     (49,3.243e-02)    (209,1.232e-02)
(769,4.454e-03)   (2561,1.551e-03)  (7937,5.236e-04)
(23297,1.723e-04) (65537,5.545e-05) (178177,1.751e-05)
};

\addplot coordinates{
(11,6.887e-02)    (71,3.177e-02)     (351,1.341e-02)
(1471,5.334e-03)  (5503,2.027e-03)   (18943,7.415e-04)
(61183,2.628e-04) (187903,9.063e-05) (553983,3.053e-05)
};

\addplot coordinates{
(13,5.755e-02)     (97,2.925e-02)     (545,1.351e-02)
(2561,5.842e-03)   (10625,2.397e-03)  (40193,9.414e-04)
(141569,3.564e-04) (471041,1.308e-04) 
(1496065,4.670e-05)
};
}%
\maketitle
\begin{abstract}%
\PGFPlots\ draws high--quality function plots in normal or logarithmic scaling with a user-friendly interface directly in \TeX. The user supplies axis labels, legend entries and the plot coordinates for one or more plots and \PGFPlots\ applies axis scaling, computes any logarithms and axis ticks and draws the plots, supporting line plots, scatter plots, piecewise constant plots, bar plots, area plots, mesh-- and surface plots and some more. It is based on Till Tantau's package \PGF/\Tikz.
\end{abstract}
\tableofcontents
\section{Introduction}
This package provides tools to generate plots and labeled axes easily. It draws normal plots, logplots and semi-logplots, in two and three dimensions. Axis ticks, labels, legends (in case of multiple plots) can be added with key-value options. It can cycle through a set of predefined line/marker/color specifications. In summary, its purpose is to simplify the generation of high-quality function and/or data plots, and solving the problems of
\begin{itemize}
	\item consistency of document and font type and font size,
	\item direct use of \TeX\ math mode in axis descriptions,
	\item consistency of data and figures (no third party tool necessary),
	\item inter document consistency using preamble configurations and styles.
\end{itemize}
Although not necessary, separate |.pdf| or |.eps| graphics can be generated using the |external| library developed as part of \Tikz.

\PGFPlots\ is build completely on \Tikz/\PGF. Knowledge of \Tikz\ will simplify the work with \PGFPlots, although it is not required.

\section[About PGFPlots: Preliminaries]{About {\normalfont\PGFPlots}: Preliminaries}
This section contains information about upgrades, the team, the installation (in case you need to do it manually) and troubleshooting. You may skip it completely except for the upgrade remarks.

However, note that this library requires at least \PGF\ version $2.00$. At the time of this writing, many \TeX-distributions still contain the older \PGF\ version $1.18$, so it may be necessary to install a recent \PGF\ prior to using \PGFPlots.

\subsection{Components}
\PGFPlots\ comes with two components:
\begin{enumerate}
	\item the plotting component (which you are currently reading) and
	\item the \PGFPlotstable\ component which simplifies number formatting and postprocessing of numerical tables. It comes as a separate package and has its own manual \href{file:pgfplotstable.pdf}{pgfplotstable.pdf}.
\end{enumerate}

\subsection{Upgrade remarks}
This release provides a lot of improvements which can be found in all detail in \texttt{ChangeLog} for interested readers. However, some attention is useful with respect to the following changes.

\subsubsection{New Optional Features}
\begin{enumerate}
	\item \PGFPlots\ 1.3 comes with user interface improvements. The technical distinction between ``behavior options'' and ``style options'' of older versions is no longer necessary (although still fully supported).

	\item \PGFPlots\ 1.3 has a new feature which allows to \emph{move axis labels tight to tick labels} automatically.

	Since this affects the spacing, it is not enabled be default.

	Use
\begin{codeexample}[code only]
\usepackage{pgfplots}
\pgfplotsset{compat=1.3}
\end{codeexample}
	in your preamble to benefit from the improved spacing\footnote{The |compat=1.3| setting is necessary for new features which move axis labels around like reversed axis. However, \PGFPlots\ will still work as it does in earlier versions, your documents will remain the same if you don't set it explicitly.}.
	Take a look at the next page for the precise definition of |compat|.

	\item \PGFPlots\ 1.3 now supports reversed axes. It is no longer necessary to use work--arounds with negative units.
\pgfkeys{/pdflinks/search key prefixes in/.add={/pgfplots/,}{}}

	Take a look at the |x dir=reverse| key.

	Existing work--arounds will still function properly. Use |\pgfplotsset{compat=1.3}| together with |x dir=reverse| to switch to the new version.
\end{enumerate}

\subsubsection{Old Features Which May Need Attention}
\begin{enumerate}
	\item The |scatter/classes| feature produces proper legends as of version 1.3. This may change the appearance of existing legends of plots with |scatter/classes|.

	\item Due to a small math bug in \PGF\ $2.00$, you \emph{can't} use the math expression `|-x^2|'. It is necessary to use `|0-x^2|' instead. The same holds for
		`|exp(-x^2)|'; use `|exp(0-x^2)|' instead.

		This will be fixed with \PGF\ $>2.00$.

	\item Starting with \PGFPlots\ $1.1$, |\tikzstyle| should \emph{no longer be used} to set \PGFPlots\ options.
	
	Although |\tikzstyle| is still supported for some older \PGFPlots\ options, you should replace any occurance of |\tikzstyle| with |\pgfplotsset{|\meta{style name}|/.style={|\meta{key-value-list}|}}| or the associated |/.append style| variant. See section~\ref{sec:styles} for more detail.
\end{enumerate}
I apologize for any inconvenience caused by these changes.

\begin{pgfplotskey}{compat=\mchoice{1.3,pre 1.3,default,newest} (initially default)}
	The preamble configuration 
\begin{codeexample}[code only]
\usepackage{pgfplots}
\pgfplotsset{compat=1.3}
\end{codeexample}
	allows to choose between backwards compatibility and most recent features.

	\PGFPlots\ version 1.3 comes with new features which may lead to different spacings compared to earlier versions:
	\begin{enumerate}
		\item Version 1.3 comes with the |xlabel near ticks| feature which places (in this case $x$) axis labels ``near'' the tick labels, taking the size of tick labels into account.
		
		Older versions used |xlabel absolute|, i.e.\ an absolute distance between the axis and axis labels (now matter how large or small tick labels are).

		The initial setting \emph{keeps backwards compatibility}.

		You are encouraged to use
\begin{codeexample}[code only]
\usepackage{pgfplots}
\pgfplotsset{compat=1.3}
\end{codeexample}
		in your preamble.
		
		\item Version 1.3 fixes a bug with |anchor=south west| which occurred mainly for reversed axes: the anchors where upside--down. This is fixed now.

		
		If you have written some work--around and want to keep it going, use

		|\pgfplotsset{compat/anchors=pre 1.3}|\pgfmanualpdflabel{/pgfplots/compat/anchors}{} 

		to disable the bug fix.
	\end{enumerate}

	Use |\pgfplotsset{compat=default}| to restore the factory settings.

	The setting |\pgfplotsset{compat=newest}| will always use features of the most recent version. This might result in small changes in the document's appearance.
\end{pgfplotskey}

\subsection{The Team}
\PGFPlots\ has been written mainly by Christian Feuers�nger with many improvements of Pascal Wolkotte and Nick Papior Andersen as a spare time project. We hope it is useful and provides valuable plots.

If you are interested in writing something but don't know how, consider reading the auxiliary manual \href{file:TeX-programming-notes.pdf}{TeX-programming-notes.pdf} which comes with \PGFPlots. It is far from complete, but maybe it is a good starting point (at least for more literature).

\subsection{Acknowledgements}
I thank God for all hours of enjoyed programming. I thank Pascal Wolkotte and Nick Papior Andersen for their programming efforts and contributions as part of the development team. I thank Stefan Tibus, who contributed the |plot shell| feature. I thank Tom Cashman for the contribution of the |reverse legend| feature. Special thanks go to Stefan Pinnow whose tests of \PGFPlots\ lead to numerous quality improvements. Furthermore, I thank Dr.~Schweitzer for many fruitful discussions and Dr.~Meine for his ideas and suggestions. Thanks as well to the many international contributors who provided feature requests or identified bugs or simply manual improvements!

Last but not least, I thank Till Tantau and Mark Wibrow for their excellent graphics (and more) package \PGF\ and \Tikz\ which is the base of \PGFPlots.

\include{pgfplots.install}%
\include{pgfplots.intro}%
\include{pgfplots.reference}%
\include{pgfplots.libs}%
\include{pgfplots.resources}%
\include{pgfplots.importexport}%
\include{pgfplots.basic.reference}%

\printindex

\bibliographystyle{abbrv} %gerapali} %gerabbrv} %gerunsrt.bst} %gerabbrv}% gerplain}
\nocite{pgfplotstable}
\nocite{programmingnotes}
\bibliography{pgfplots}
\end{document}
%
%%%%%%%%%%%%%%%%%%%%%%%%%%%%%%%%%%%%%%%%%%%%%%%%%%%%%%%%%%%%%%%%%%%%%%%%%%%%%
%
% Package pgfplots.sty documentation. 
%
% Copyright 2007/2008 by Christian Feuersaenger.
%
% This program is free software: you can redistribute it and/or modify
% it under the terms of the GNU General Public License as published by
% the Free Software Foundation, either version 3 of the License, or
% (at your option) any later version.
% 
% This program is distributed in the hope that it will be useful,
% but WITHOUT ANY WARRANTY; without even the implied warranty of
% MERCHANTABILITY or FITNESS FOR A PARTICULAR PURPOSE.  See the
% GNU General Public License for more details.
% 
% You should have received a copy of the GNU General Public License
% along with this program.  If not, see <http://www.gnu.org/licenses/>.
%
%
%%%%%%%%%%%%%%%%%%%%%%%%%%%%%%%%%%%%%%%%%%%%%%%%%%%%%%%%%%%%%%%%%%%%%%%%%%%%%
\input{pgfplots.preamble.tex}
%\RequirePackage[german,english,francais]{babel}

\pgfplotsmanualexternalexpensivetrue

%\usetikzlibrary{external}
%\tikzexternalize[prefix=figures/]{pgfplots}

\title{%
	Manual for Package \PGFPlots\\
	{\small Version \pgfplotsversion}\\
	{\small\href{http://sourceforge.net/projects/pgfplots}{http://sourceforge.net/projects/pgfplots}}}

%\includeonly{pgfplots.libs}


\begin{document}

\def\plotcoords{%
\addplot coordinates {
(5,8.312e-02)    (17,2.547e-02)   (49,7.407e-03)
(129,2.102e-03)  (321,5.874e-04)  (769,1.623e-04)
(1793,4.442e-05) (4097,1.207e-05) (9217,3.261e-06)
};

\addplot coordinates{
(7,8.472e-02)    (31,3.044e-02)    (111,1.022e-02)
(351,3.303e-03)  (1023,1.039e-03)  (2815,3.196e-04)
(7423,9.658e-05) (18943,2.873e-05) (47103,8.437e-06)
};

\addplot coordinates{
(9,7.881e-02)     (49,3.243e-02)    (209,1.232e-02)
(769,4.454e-03)   (2561,1.551e-03)  (7937,5.236e-04)
(23297,1.723e-04) (65537,5.545e-05) (178177,1.751e-05)
};

\addplot coordinates{
(11,6.887e-02)    (71,3.177e-02)     (351,1.341e-02)
(1471,5.334e-03)  (5503,2.027e-03)   (18943,7.415e-04)
(61183,2.628e-04) (187903,9.063e-05) (553983,3.053e-05)
};

\addplot coordinates{
(13,5.755e-02)     (97,2.925e-02)     (545,1.351e-02)
(2561,5.842e-03)   (10625,2.397e-03)  (40193,9.414e-04)
(141569,3.564e-04) (471041,1.308e-04) 
(1496065,4.670e-05)
};
}%
\maketitle
\begin{abstract}%
\PGFPlots\ draws high--quality function plots in normal or logarithmic scaling with a user-friendly interface directly in \TeX. The user supplies axis labels, legend entries and the plot coordinates for one or more plots and \PGFPlots\ applies axis scaling, computes any logarithms and axis ticks and draws the plots, supporting line plots, scatter plots, piecewise constant plots, bar plots, area plots, mesh-- and surface plots and some more. It is based on Till Tantau's package \PGF/\Tikz.
\end{abstract}
\tableofcontents
\section{Introduction}
This package provides tools to generate plots and labeled axes easily. It draws normal plots, logplots and semi-logplots, in two and three dimensions. Axis ticks, labels, legends (in case of multiple plots) can be added with key-value options. It can cycle through a set of predefined line/marker/color specifications. In summary, its purpose is to simplify the generation of high-quality function and/or data plots, and solving the problems of
\begin{itemize}
	\item consistency of document and font type and font size,
	\item direct use of \TeX\ math mode in axis descriptions,
	\item consistency of data and figures (no third party tool necessary),
	\item inter document consistency using preamble configurations and styles.
\end{itemize}
Although not necessary, separate |.pdf| or |.eps| graphics can be generated using the |external| library developed as part of \Tikz.

\PGFPlots\ is build completely on \Tikz/\PGF. Knowledge of \Tikz\ will simplify the work with \PGFPlots, although it is not required.

\section[About PGFPlots: Preliminaries]{About {\normalfont\PGFPlots}: Preliminaries}
This section contains information about upgrades, the team, the installation (in case you need to do it manually) and troubleshooting. You may skip it completely except for the upgrade remarks.

However, note that this library requires at least \PGF\ version $2.00$. At the time of this writing, many \TeX-distributions still contain the older \PGF\ version $1.18$, so it may be necessary to install a recent \PGF\ prior to using \PGFPlots.

\subsection{Components}
\PGFPlots\ comes with two components:
\begin{enumerate}
	\item the plotting component (which you are currently reading) and
	\item the \PGFPlotstable\ component which simplifies number formatting and postprocessing of numerical tables. It comes as a separate package and has its own manual \href{file:pgfplotstable.pdf}{pgfplotstable.pdf}.
\end{enumerate}

\subsection{Upgrade remarks}
This release provides a lot of improvements which can be found in all detail in \texttt{ChangeLog} for interested readers. However, some attention is useful with respect to the following changes.

\subsubsection{New Optional Features}
\begin{enumerate}
	\item \PGFPlots\ 1.3 comes with user interface improvements. The technical distinction between ``behavior options'' and ``style options'' of older versions is no longer necessary (although still fully supported).

	\item \PGFPlots\ 1.3 has a new feature which allows to \emph{move axis labels tight to tick labels} automatically.

	Since this affects the spacing, it is not enabled be default.

	Use
\begin{codeexample}[code only]
\usepackage{pgfplots}
\pgfplotsset{compat=1.3}
\end{codeexample}
	in your preamble to benefit from the improved spacing\footnote{The |compat=1.3| setting is necessary for new features which move axis labels around like reversed axis. However, \PGFPlots\ will still work as it does in earlier versions, your documents will remain the same if you don't set it explicitly.}.
	Take a look at the next page for the precise definition of |compat|.

	\item \PGFPlots\ 1.3 now supports reversed axes. It is no longer necessary to use work--arounds with negative units.
\pgfkeys{/pdflinks/search key prefixes in/.add={/pgfplots/,}{}}

	Take a look at the |x dir=reverse| key.

	Existing work--arounds will still function properly. Use |\pgfplotsset{compat=1.3}| together with |x dir=reverse| to switch to the new version.
\end{enumerate}

\subsubsection{Old Features Which May Need Attention}
\begin{enumerate}
	\item The |scatter/classes| feature produces proper legends as of version 1.3. This may change the appearance of existing legends of plots with |scatter/classes|.

	\item Due to a small math bug in \PGF\ $2.00$, you \emph{can't} use the math expression `|-x^2|'. It is necessary to use `|0-x^2|' instead. The same holds for
		`|exp(-x^2)|'; use `|exp(0-x^2)|' instead.

		This will be fixed with \PGF\ $>2.00$.

	\item Starting with \PGFPlots\ $1.1$, |\tikzstyle| should \emph{no longer be used} to set \PGFPlots\ options.
	
	Although |\tikzstyle| is still supported for some older \PGFPlots\ options, you should replace any occurance of |\tikzstyle| with |\pgfplotsset{|\meta{style name}|/.style={|\meta{key-value-list}|}}| or the associated |/.append style| variant. See section~\ref{sec:styles} for more detail.
\end{enumerate}
I apologize for any inconvenience caused by these changes.

\begin{pgfplotskey}{compat=\mchoice{1.3,pre 1.3,default,newest} (initially default)}
	The preamble configuration 
\begin{codeexample}[code only]
\usepackage{pgfplots}
\pgfplotsset{compat=1.3}
\end{codeexample}
	allows to choose between backwards compatibility and most recent features.

	\PGFPlots\ version 1.3 comes with new features which may lead to different spacings compared to earlier versions:
	\begin{enumerate}
		\item Version 1.3 comes with the |xlabel near ticks| feature which places (in this case $x$) axis labels ``near'' the tick labels, taking the size of tick labels into account.
		
		Older versions used |xlabel absolute|, i.e.\ an absolute distance between the axis and axis labels (now matter how large or small tick labels are).

		The initial setting \emph{keeps backwards compatibility}.

		You are encouraged to use
\begin{codeexample}[code only]
\usepackage{pgfplots}
\pgfplotsset{compat=1.3}
\end{codeexample}
		in your preamble.
		
		\item Version 1.3 fixes a bug with |anchor=south west| which occurred mainly for reversed axes: the anchors where upside--down. This is fixed now.

		
		If you have written some work--around and want to keep it going, use

		|\pgfplotsset{compat/anchors=pre 1.3}|\pgfmanualpdflabel{/pgfplots/compat/anchors}{} 

		to disable the bug fix.
	\end{enumerate}

	Use |\pgfplotsset{compat=default}| to restore the factory settings.

	The setting |\pgfplotsset{compat=newest}| will always use features of the most recent version. This might result in small changes in the document's appearance.
\end{pgfplotskey}

\subsection{The Team}
\PGFPlots\ has been written mainly by Christian Feuers�nger with many improvements of Pascal Wolkotte and Nick Papior Andersen as a spare time project. We hope it is useful and provides valuable plots.

If you are interested in writing something but don't know how, consider reading the auxiliary manual \href{file:TeX-programming-notes.pdf}{TeX-programming-notes.pdf} which comes with \PGFPlots. It is far from complete, but maybe it is a good starting point (at least for more literature).

\subsection{Acknowledgements}
I thank God for all hours of enjoyed programming. I thank Pascal Wolkotte and Nick Papior Andersen for their programming efforts and contributions as part of the development team. I thank Stefan Tibus, who contributed the |plot shell| feature. I thank Tom Cashman for the contribution of the |reverse legend| feature. Special thanks go to Stefan Pinnow whose tests of \PGFPlots\ lead to numerous quality improvements. Furthermore, I thank Dr.~Schweitzer for many fruitful discussions and Dr.~Meine for his ideas and suggestions. Thanks as well to the many international contributors who provided feature requests or identified bugs or simply manual improvements!

Last but not least, I thank Till Tantau and Mark Wibrow for their excellent graphics (and more) package \PGF\ and \Tikz\ which is the base of \PGFPlots.

\include{pgfplots.install}%
\include{pgfplots.intro}%
\include{pgfplots.reference}%
\include{pgfplots.libs}%
\include{pgfplots.resources}%
\include{pgfplots.importexport}%
\include{pgfplots.basic.reference}%

\printindex

\bibliographystyle{abbrv} %gerapali} %gerabbrv} %gerunsrt.bst} %gerabbrv}% gerplain}
\nocite{pgfplotstable}
\nocite{programmingnotes}
\bibliography{pgfplots}
\end{document}
%
%%%%%%%%%%%%%%%%%%%%%%%%%%%%%%%%%%%%%%%%%%%%%%%%%%%%%%%%%%%%%%%%%%%%%%%%%%%%%
%
% Package pgfplots.sty documentation. 
%
% Copyright 2007/2008 by Christian Feuersaenger.
%
% This program is free software: you can redistribute it and/or modify
% it under the terms of the GNU General Public License as published by
% the Free Software Foundation, either version 3 of the License, or
% (at your option) any later version.
% 
% This program is distributed in the hope that it will be useful,
% but WITHOUT ANY WARRANTY; without even the implied warranty of
% MERCHANTABILITY or FITNESS FOR A PARTICULAR PURPOSE.  See the
% GNU General Public License for more details.
% 
% You should have received a copy of the GNU General Public License
% along with this program.  If not, see <http://www.gnu.org/licenses/>.
%
%
%%%%%%%%%%%%%%%%%%%%%%%%%%%%%%%%%%%%%%%%%%%%%%%%%%%%%%%%%%%%%%%%%%%%%%%%%%%%%
\input{pgfplots.preamble.tex}
%\RequirePackage[german,english,francais]{babel}

\pgfplotsmanualexternalexpensivetrue

%\usetikzlibrary{external}
%\tikzexternalize[prefix=figures/]{pgfplots}

\title{%
	Manual for Package \PGFPlots\\
	{\small Version \pgfplotsversion}\\
	{\small\href{http://sourceforge.net/projects/pgfplots}{http://sourceforge.net/projects/pgfplots}}}

%\includeonly{pgfplots.libs}


\begin{document}

\def\plotcoords{%
\addplot coordinates {
(5,8.312e-02)    (17,2.547e-02)   (49,7.407e-03)
(129,2.102e-03)  (321,5.874e-04)  (769,1.623e-04)
(1793,4.442e-05) (4097,1.207e-05) (9217,3.261e-06)
};

\addplot coordinates{
(7,8.472e-02)    (31,3.044e-02)    (111,1.022e-02)
(351,3.303e-03)  (1023,1.039e-03)  (2815,3.196e-04)
(7423,9.658e-05) (18943,2.873e-05) (47103,8.437e-06)
};

\addplot coordinates{
(9,7.881e-02)     (49,3.243e-02)    (209,1.232e-02)
(769,4.454e-03)   (2561,1.551e-03)  (7937,5.236e-04)
(23297,1.723e-04) (65537,5.545e-05) (178177,1.751e-05)
};

\addplot coordinates{
(11,6.887e-02)    (71,3.177e-02)     (351,1.341e-02)
(1471,5.334e-03)  (5503,2.027e-03)   (18943,7.415e-04)
(61183,2.628e-04) (187903,9.063e-05) (553983,3.053e-05)
};

\addplot coordinates{
(13,5.755e-02)     (97,2.925e-02)     (545,1.351e-02)
(2561,5.842e-03)   (10625,2.397e-03)  (40193,9.414e-04)
(141569,3.564e-04) (471041,1.308e-04) 
(1496065,4.670e-05)
};
}%
\maketitle
\begin{abstract}%
\PGFPlots\ draws high--quality function plots in normal or logarithmic scaling with a user-friendly interface directly in \TeX. The user supplies axis labels, legend entries and the plot coordinates for one or more plots and \PGFPlots\ applies axis scaling, computes any logarithms and axis ticks and draws the plots, supporting line plots, scatter plots, piecewise constant plots, bar plots, area plots, mesh-- and surface plots and some more. It is based on Till Tantau's package \PGF/\Tikz.
\end{abstract}
\tableofcontents
\section{Introduction}
This package provides tools to generate plots and labeled axes easily. It draws normal plots, logplots and semi-logplots, in two and three dimensions. Axis ticks, labels, legends (in case of multiple plots) can be added with key-value options. It can cycle through a set of predefined line/marker/color specifications. In summary, its purpose is to simplify the generation of high-quality function and/or data plots, and solving the problems of
\begin{itemize}
	\item consistency of document and font type and font size,
	\item direct use of \TeX\ math mode in axis descriptions,
	\item consistency of data and figures (no third party tool necessary),
	\item inter document consistency using preamble configurations and styles.
\end{itemize}
Although not necessary, separate |.pdf| or |.eps| graphics can be generated using the |external| library developed as part of \Tikz.

\PGFPlots\ is build completely on \Tikz/\PGF. Knowledge of \Tikz\ will simplify the work with \PGFPlots, although it is not required.

\section[About PGFPlots: Preliminaries]{About {\normalfont\PGFPlots}: Preliminaries}
This section contains information about upgrades, the team, the installation (in case you need to do it manually) and troubleshooting. You may skip it completely except for the upgrade remarks.

However, note that this library requires at least \PGF\ version $2.00$. At the time of this writing, many \TeX-distributions still contain the older \PGF\ version $1.18$, so it may be necessary to install a recent \PGF\ prior to using \PGFPlots.

\subsection{Components}
\PGFPlots\ comes with two components:
\begin{enumerate}
	\item the plotting component (which you are currently reading) and
	\item the \PGFPlotstable\ component which simplifies number formatting and postprocessing of numerical tables. It comes as a separate package and has its own manual \href{file:pgfplotstable.pdf}{pgfplotstable.pdf}.
\end{enumerate}

\subsection{Upgrade remarks}
This release provides a lot of improvements which can be found in all detail in \texttt{ChangeLog} for interested readers. However, some attention is useful with respect to the following changes.

\subsubsection{New Optional Features}
\begin{enumerate}
	\item \PGFPlots\ 1.3 comes with user interface improvements. The technical distinction between ``behavior options'' and ``style options'' of older versions is no longer necessary (although still fully supported).

	\item \PGFPlots\ 1.3 has a new feature which allows to \emph{move axis labels tight to tick labels} automatically.

	Since this affects the spacing, it is not enabled be default.

	Use
\begin{codeexample}[code only]
\usepackage{pgfplots}
\pgfplotsset{compat=1.3}
\end{codeexample}
	in your preamble to benefit from the improved spacing\footnote{The |compat=1.3| setting is necessary for new features which move axis labels around like reversed axis. However, \PGFPlots\ will still work as it does in earlier versions, your documents will remain the same if you don't set it explicitly.}.
	Take a look at the next page for the precise definition of |compat|.

	\item \PGFPlots\ 1.3 now supports reversed axes. It is no longer necessary to use work--arounds with negative units.
\pgfkeys{/pdflinks/search key prefixes in/.add={/pgfplots/,}{}}

	Take a look at the |x dir=reverse| key.

	Existing work--arounds will still function properly. Use |\pgfplotsset{compat=1.3}| together with |x dir=reverse| to switch to the new version.
\end{enumerate}

\subsubsection{Old Features Which May Need Attention}
\begin{enumerate}
	\item The |scatter/classes| feature produces proper legends as of version 1.3. This may change the appearance of existing legends of plots with |scatter/classes|.

	\item Due to a small math bug in \PGF\ $2.00$, you \emph{can't} use the math expression `|-x^2|'. It is necessary to use `|0-x^2|' instead. The same holds for
		`|exp(-x^2)|'; use `|exp(0-x^2)|' instead.

		This will be fixed with \PGF\ $>2.00$.

	\item Starting with \PGFPlots\ $1.1$, |\tikzstyle| should \emph{no longer be used} to set \PGFPlots\ options.
	
	Although |\tikzstyle| is still supported for some older \PGFPlots\ options, you should replace any occurance of |\tikzstyle| with |\pgfplotsset{|\meta{style name}|/.style={|\meta{key-value-list}|}}| or the associated |/.append style| variant. See section~\ref{sec:styles} for more detail.
\end{enumerate}
I apologize for any inconvenience caused by these changes.

\begin{pgfplotskey}{compat=\mchoice{1.3,pre 1.3,default,newest} (initially default)}
	The preamble configuration 
\begin{codeexample}[code only]
\usepackage{pgfplots}
\pgfplotsset{compat=1.3}
\end{codeexample}
	allows to choose between backwards compatibility and most recent features.

	\PGFPlots\ version 1.3 comes with new features which may lead to different spacings compared to earlier versions:
	\begin{enumerate}
		\item Version 1.3 comes with the |xlabel near ticks| feature which places (in this case $x$) axis labels ``near'' the tick labels, taking the size of tick labels into account.
		
		Older versions used |xlabel absolute|, i.e.\ an absolute distance between the axis and axis labels (now matter how large or small tick labels are).

		The initial setting \emph{keeps backwards compatibility}.

		You are encouraged to use
\begin{codeexample}[code only]
\usepackage{pgfplots}
\pgfplotsset{compat=1.3}
\end{codeexample}
		in your preamble.
		
		\item Version 1.3 fixes a bug with |anchor=south west| which occurred mainly for reversed axes: the anchors where upside--down. This is fixed now.

		
		If you have written some work--around and want to keep it going, use

		|\pgfplotsset{compat/anchors=pre 1.3}|\pgfmanualpdflabel{/pgfplots/compat/anchors}{} 

		to disable the bug fix.
	\end{enumerate}

	Use |\pgfplotsset{compat=default}| to restore the factory settings.

	The setting |\pgfplotsset{compat=newest}| will always use features of the most recent version. This might result in small changes in the document's appearance.
\end{pgfplotskey}

\subsection{The Team}
\PGFPlots\ has been written mainly by Christian Feuers�nger with many improvements of Pascal Wolkotte and Nick Papior Andersen as a spare time project. We hope it is useful and provides valuable plots.

If you are interested in writing something but don't know how, consider reading the auxiliary manual \href{file:TeX-programming-notes.pdf}{TeX-programming-notes.pdf} which comes with \PGFPlots. It is far from complete, but maybe it is a good starting point (at least for more literature).

\subsection{Acknowledgements}
I thank God for all hours of enjoyed programming. I thank Pascal Wolkotte and Nick Papior Andersen for their programming efforts and contributions as part of the development team. I thank Stefan Tibus, who contributed the |plot shell| feature. I thank Tom Cashman for the contribution of the |reverse legend| feature. Special thanks go to Stefan Pinnow whose tests of \PGFPlots\ lead to numerous quality improvements. Furthermore, I thank Dr.~Schweitzer for many fruitful discussions and Dr.~Meine for his ideas and suggestions. Thanks as well to the many international contributors who provided feature requests or identified bugs or simply manual improvements!

Last but not least, I thank Till Tantau and Mark Wibrow for their excellent graphics (and more) package \PGF\ and \Tikz\ which is the base of \PGFPlots.

\include{pgfplots.install}%
\include{pgfplots.intro}%
\include{pgfplots.reference}%
\include{pgfplots.libs}%
\include{pgfplots.resources}%
\include{pgfplots.importexport}%
\include{pgfplots.basic.reference}%

\printindex

\bibliographystyle{abbrv} %gerapali} %gerabbrv} %gerunsrt.bst} %gerabbrv}% gerplain}
\nocite{pgfplotstable}
\nocite{programmingnotes}
\bibliography{pgfplots}
\end{document}
%
%%%%%%%%%%%%%%%%%%%%%%%%%%%%%%%%%%%%%%%%%%%%%%%%%%%%%%%%%%%%%%%%%%%%%%%%%%%%%
%
% Package pgfplots.sty documentation. 
%
% Copyright 2007/2008 by Christian Feuersaenger.
%
% This program is free software: you can redistribute it and/or modify
% it under the terms of the GNU General Public License as published by
% the Free Software Foundation, either version 3 of the License, or
% (at your option) any later version.
% 
% This program is distributed in the hope that it will be useful,
% but WITHOUT ANY WARRANTY; without even the implied warranty of
% MERCHANTABILITY or FITNESS FOR A PARTICULAR PURPOSE.  See the
% GNU General Public License for more details.
% 
% You should have received a copy of the GNU General Public License
% along with this program.  If not, see <http://www.gnu.org/licenses/>.
%
%
%%%%%%%%%%%%%%%%%%%%%%%%%%%%%%%%%%%%%%%%%%%%%%%%%%%%%%%%%%%%%%%%%%%%%%%%%%%%%
\input{pgfplots.preamble.tex}
%\RequirePackage[german,english,francais]{babel}

\pgfplotsmanualexternalexpensivetrue

%\usetikzlibrary{external}
%\tikzexternalize[prefix=figures/]{pgfplots}

\title{%
	Manual for Package \PGFPlots\\
	{\small Version \pgfplotsversion}\\
	{\small\href{http://sourceforge.net/projects/pgfplots}{http://sourceforge.net/projects/pgfplots}}}

%\includeonly{pgfplots.libs}


\begin{document}

\def\plotcoords{%
\addplot coordinates {
(5,8.312e-02)    (17,2.547e-02)   (49,7.407e-03)
(129,2.102e-03)  (321,5.874e-04)  (769,1.623e-04)
(1793,4.442e-05) (4097,1.207e-05) (9217,3.261e-06)
};

\addplot coordinates{
(7,8.472e-02)    (31,3.044e-02)    (111,1.022e-02)
(351,3.303e-03)  (1023,1.039e-03)  (2815,3.196e-04)
(7423,9.658e-05) (18943,2.873e-05) (47103,8.437e-06)
};

\addplot coordinates{
(9,7.881e-02)     (49,3.243e-02)    (209,1.232e-02)
(769,4.454e-03)   (2561,1.551e-03)  (7937,5.236e-04)
(23297,1.723e-04) (65537,5.545e-05) (178177,1.751e-05)
};

\addplot coordinates{
(11,6.887e-02)    (71,3.177e-02)     (351,1.341e-02)
(1471,5.334e-03)  (5503,2.027e-03)   (18943,7.415e-04)
(61183,2.628e-04) (187903,9.063e-05) (553983,3.053e-05)
};

\addplot coordinates{
(13,5.755e-02)     (97,2.925e-02)     (545,1.351e-02)
(2561,5.842e-03)   (10625,2.397e-03)  (40193,9.414e-04)
(141569,3.564e-04) (471041,1.308e-04) 
(1496065,4.670e-05)
};
}%
\maketitle
\begin{abstract}%
\PGFPlots\ draws high--quality function plots in normal or logarithmic scaling with a user-friendly interface directly in \TeX. The user supplies axis labels, legend entries and the plot coordinates for one or more plots and \PGFPlots\ applies axis scaling, computes any logarithms and axis ticks and draws the plots, supporting line plots, scatter plots, piecewise constant plots, bar plots, area plots, mesh-- and surface plots and some more. It is based on Till Tantau's package \PGF/\Tikz.
\end{abstract}
\tableofcontents
\section{Introduction}
This package provides tools to generate plots and labeled axes easily. It draws normal plots, logplots and semi-logplots, in two and three dimensions. Axis ticks, labels, legends (in case of multiple plots) can be added with key-value options. It can cycle through a set of predefined line/marker/color specifications. In summary, its purpose is to simplify the generation of high-quality function and/or data plots, and solving the problems of
\begin{itemize}
	\item consistency of document and font type and font size,
	\item direct use of \TeX\ math mode in axis descriptions,
	\item consistency of data and figures (no third party tool necessary),
	\item inter document consistency using preamble configurations and styles.
\end{itemize}
Although not necessary, separate |.pdf| or |.eps| graphics can be generated using the |external| library developed as part of \Tikz.

\PGFPlots\ is build completely on \Tikz/\PGF. Knowledge of \Tikz\ will simplify the work with \PGFPlots, although it is not required.

\section[About PGFPlots: Preliminaries]{About {\normalfont\PGFPlots}: Preliminaries}
This section contains information about upgrades, the team, the installation (in case you need to do it manually) and troubleshooting. You may skip it completely except for the upgrade remarks.

However, note that this library requires at least \PGF\ version $2.00$. At the time of this writing, many \TeX-distributions still contain the older \PGF\ version $1.18$, so it may be necessary to install a recent \PGF\ prior to using \PGFPlots.

\subsection{Components}
\PGFPlots\ comes with two components:
\begin{enumerate}
	\item the plotting component (which you are currently reading) and
	\item the \PGFPlotstable\ component which simplifies number formatting and postprocessing of numerical tables. It comes as a separate package and has its own manual \href{file:pgfplotstable.pdf}{pgfplotstable.pdf}.
\end{enumerate}

\subsection{Upgrade remarks}
This release provides a lot of improvements which can be found in all detail in \texttt{ChangeLog} for interested readers. However, some attention is useful with respect to the following changes.

\subsubsection{New Optional Features}
\begin{enumerate}
	\item \PGFPlots\ 1.3 comes with user interface improvements. The technical distinction between ``behavior options'' and ``style options'' of older versions is no longer necessary (although still fully supported).

	\item \PGFPlots\ 1.3 has a new feature which allows to \emph{move axis labels tight to tick labels} automatically.

	Since this affects the spacing, it is not enabled be default.

	Use
\begin{codeexample}[code only]
\usepackage{pgfplots}
\pgfplotsset{compat=1.3}
\end{codeexample}
	in your preamble to benefit from the improved spacing\footnote{The |compat=1.3| setting is necessary for new features which move axis labels around like reversed axis. However, \PGFPlots\ will still work as it does in earlier versions, your documents will remain the same if you don't set it explicitly.}.
	Take a look at the next page for the precise definition of |compat|.

	\item \PGFPlots\ 1.3 now supports reversed axes. It is no longer necessary to use work--arounds with negative units.
\pgfkeys{/pdflinks/search key prefixes in/.add={/pgfplots/,}{}}

	Take a look at the |x dir=reverse| key.

	Existing work--arounds will still function properly. Use |\pgfplotsset{compat=1.3}| together with |x dir=reverse| to switch to the new version.
\end{enumerate}

\subsubsection{Old Features Which May Need Attention}
\begin{enumerate}
	\item The |scatter/classes| feature produces proper legends as of version 1.3. This may change the appearance of existing legends of plots with |scatter/classes|.

	\item Due to a small math bug in \PGF\ $2.00$, you \emph{can't} use the math expression `|-x^2|'. It is necessary to use `|0-x^2|' instead. The same holds for
		`|exp(-x^2)|'; use `|exp(0-x^2)|' instead.

		This will be fixed with \PGF\ $>2.00$.

	\item Starting with \PGFPlots\ $1.1$, |\tikzstyle| should \emph{no longer be used} to set \PGFPlots\ options.
	
	Although |\tikzstyle| is still supported for some older \PGFPlots\ options, you should replace any occurance of |\tikzstyle| with |\pgfplotsset{|\meta{style name}|/.style={|\meta{key-value-list}|}}| or the associated |/.append style| variant. See section~\ref{sec:styles} for more detail.
\end{enumerate}
I apologize for any inconvenience caused by these changes.

\begin{pgfplotskey}{compat=\mchoice{1.3,pre 1.3,default,newest} (initially default)}
	The preamble configuration 
\begin{codeexample}[code only]
\usepackage{pgfplots}
\pgfplotsset{compat=1.3}
\end{codeexample}
	allows to choose between backwards compatibility and most recent features.

	\PGFPlots\ version 1.3 comes with new features which may lead to different spacings compared to earlier versions:
	\begin{enumerate}
		\item Version 1.3 comes with the |xlabel near ticks| feature which places (in this case $x$) axis labels ``near'' the tick labels, taking the size of tick labels into account.
		
		Older versions used |xlabel absolute|, i.e.\ an absolute distance between the axis and axis labels (now matter how large or small tick labels are).

		The initial setting \emph{keeps backwards compatibility}.

		You are encouraged to use
\begin{codeexample}[code only]
\usepackage{pgfplots}
\pgfplotsset{compat=1.3}
\end{codeexample}
		in your preamble.
		
		\item Version 1.3 fixes a bug with |anchor=south west| which occurred mainly for reversed axes: the anchors where upside--down. This is fixed now.

		
		If you have written some work--around and want to keep it going, use

		|\pgfplotsset{compat/anchors=pre 1.3}|\pgfmanualpdflabel{/pgfplots/compat/anchors}{} 

		to disable the bug fix.
	\end{enumerate}

	Use |\pgfplotsset{compat=default}| to restore the factory settings.

	The setting |\pgfplotsset{compat=newest}| will always use features of the most recent version. This might result in small changes in the document's appearance.
\end{pgfplotskey}

\subsection{The Team}
\PGFPlots\ has been written mainly by Christian Feuers�nger with many improvements of Pascal Wolkotte and Nick Papior Andersen as a spare time project. We hope it is useful and provides valuable plots.

If you are interested in writing something but don't know how, consider reading the auxiliary manual \href{file:TeX-programming-notes.pdf}{TeX-programming-notes.pdf} which comes with \PGFPlots. It is far from complete, but maybe it is a good starting point (at least for more literature).

\subsection{Acknowledgements}
I thank God for all hours of enjoyed programming. I thank Pascal Wolkotte and Nick Papior Andersen for their programming efforts and contributions as part of the development team. I thank Stefan Tibus, who contributed the |plot shell| feature. I thank Tom Cashman for the contribution of the |reverse legend| feature. Special thanks go to Stefan Pinnow whose tests of \PGFPlots\ lead to numerous quality improvements. Furthermore, I thank Dr.~Schweitzer for many fruitful discussions and Dr.~Meine for his ideas and suggestions. Thanks as well to the many international contributors who provided feature requests or identified bugs or simply manual improvements!

Last but not least, I thank Till Tantau and Mark Wibrow for their excellent graphics (and more) package \PGF\ and \Tikz\ which is the base of \PGFPlots.

\include{pgfplots.install}%
\include{pgfplots.intro}%
\include{pgfplots.reference}%
\include{pgfplots.libs}%
\include{pgfplots.resources}%
\include{pgfplots.importexport}%
\include{pgfplots.basic.reference}%

\printindex

\bibliographystyle{abbrv} %gerapali} %gerabbrv} %gerunsrt.bst} %gerabbrv}% gerplain}
\nocite{pgfplotstable}
\nocite{programmingnotes}
\bibliography{pgfplots}
\end{document}
%
%%%%%%%%%%%%%%%%%%%%%%%%%%%%%%%%%%%%%%%%%%%%%%%%%%%%%%%%%%%%%%%%%%%%%%%%%%%%%
%
% Package pgfplots.sty documentation. 
%
% Copyright 2007/2008 by Christian Feuersaenger.
%
% This program is free software: you can redistribute it and/or modify
% it under the terms of the GNU General Public License as published by
% the Free Software Foundation, either version 3 of the License, or
% (at your option) any later version.
% 
% This program is distributed in the hope that it will be useful,
% but WITHOUT ANY WARRANTY; without even the implied warranty of
% MERCHANTABILITY or FITNESS FOR A PARTICULAR PURPOSE.  See the
% GNU General Public License for more details.
% 
% You should have received a copy of the GNU General Public License
% along with this program.  If not, see <http://www.gnu.org/licenses/>.
%
%
%%%%%%%%%%%%%%%%%%%%%%%%%%%%%%%%%%%%%%%%%%%%%%%%%%%%%%%%%%%%%%%%%%%%%%%%%%%%%
\input{pgfplots.preamble.tex}
%\RequirePackage[german,english,francais]{babel}

\pgfplotsmanualexternalexpensivetrue

%\usetikzlibrary{external}
%\tikzexternalize[prefix=figures/]{pgfplots}

\title{%
	Manual for Package \PGFPlots\\
	{\small Version \pgfplotsversion}\\
	{\small\href{http://sourceforge.net/projects/pgfplots}{http://sourceforge.net/projects/pgfplots}}}

%\includeonly{pgfplots.libs}


\begin{document}

\def\plotcoords{%
\addplot coordinates {
(5,8.312e-02)    (17,2.547e-02)   (49,7.407e-03)
(129,2.102e-03)  (321,5.874e-04)  (769,1.623e-04)
(1793,4.442e-05) (4097,1.207e-05) (9217,3.261e-06)
};

\addplot coordinates{
(7,8.472e-02)    (31,3.044e-02)    (111,1.022e-02)
(351,3.303e-03)  (1023,1.039e-03)  (2815,3.196e-04)
(7423,9.658e-05) (18943,2.873e-05) (47103,8.437e-06)
};

\addplot coordinates{
(9,7.881e-02)     (49,3.243e-02)    (209,1.232e-02)
(769,4.454e-03)   (2561,1.551e-03)  (7937,5.236e-04)
(23297,1.723e-04) (65537,5.545e-05) (178177,1.751e-05)
};

\addplot coordinates{
(11,6.887e-02)    (71,3.177e-02)     (351,1.341e-02)
(1471,5.334e-03)  (5503,2.027e-03)   (18943,7.415e-04)
(61183,2.628e-04) (187903,9.063e-05) (553983,3.053e-05)
};

\addplot coordinates{
(13,5.755e-02)     (97,2.925e-02)     (545,1.351e-02)
(2561,5.842e-03)   (10625,2.397e-03)  (40193,9.414e-04)
(141569,3.564e-04) (471041,1.308e-04) 
(1496065,4.670e-05)
};
}%
\maketitle
\begin{abstract}%
\PGFPlots\ draws high--quality function plots in normal or logarithmic scaling with a user-friendly interface directly in \TeX. The user supplies axis labels, legend entries and the plot coordinates for one or more plots and \PGFPlots\ applies axis scaling, computes any logarithms and axis ticks and draws the plots, supporting line plots, scatter plots, piecewise constant plots, bar plots, area plots, mesh-- and surface plots and some more. It is based on Till Tantau's package \PGF/\Tikz.
\end{abstract}
\tableofcontents
\section{Introduction}
This package provides tools to generate plots and labeled axes easily. It draws normal plots, logplots and semi-logplots, in two and three dimensions. Axis ticks, labels, legends (in case of multiple plots) can be added with key-value options. It can cycle through a set of predefined line/marker/color specifications. In summary, its purpose is to simplify the generation of high-quality function and/or data plots, and solving the problems of
\begin{itemize}
	\item consistency of document and font type and font size,
	\item direct use of \TeX\ math mode in axis descriptions,
	\item consistency of data and figures (no third party tool necessary),
	\item inter document consistency using preamble configurations and styles.
\end{itemize}
Although not necessary, separate |.pdf| or |.eps| graphics can be generated using the |external| library developed as part of \Tikz.

\PGFPlots\ is build completely on \Tikz/\PGF. Knowledge of \Tikz\ will simplify the work with \PGFPlots, although it is not required.

\section[About PGFPlots: Preliminaries]{About {\normalfont\PGFPlots}: Preliminaries}
This section contains information about upgrades, the team, the installation (in case you need to do it manually) and troubleshooting. You may skip it completely except for the upgrade remarks.

However, note that this library requires at least \PGF\ version $2.00$. At the time of this writing, many \TeX-distributions still contain the older \PGF\ version $1.18$, so it may be necessary to install a recent \PGF\ prior to using \PGFPlots.

\subsection{Components}
\PGFPlots\ comes with two components:
\begin{enumerate}
	\item the plotting component (which you are currently reading) and
	\item the \PGFPlotstable\ component which simplifies number formatting and postprocessing of numerical tables. It comes as a separate package and has its own manual \href{file:pgfplotstable.pdf}{pgfplotstable.pdf}.
\end{enumerate}

\subsection{Upgrade remarks}
This release provides a lot of improvements which can be found in all detail in \texttt{ChangeLog} for interested readers. However, some attention is useful with respect to the following changes.

\subsubsection{New Optional Features}
\begin{enumerate}
	\item \PGFPlots\ 1.3 comes with user interface improvements. The technical distinction between ``behavior options'' and ``style options'' of older versions is no longer necessary (although still fully supported).

	\item \PGFPlots\ 1.3 has a new feature which allows to \emph{move axis labels tight to tick labels} automatically.

	Since this affects the spacing, it is not enabled be default.

	Use
\begin{codeexample}[code only]
\usepackage{pgfplots}
\pgfplotsset{compat=1.3}
\end{codeexample}
	in your preamble to benefit from the improved spacing\footnote{The |compat=1.3| setting is necessary for new features which move axis labels around like reversed axis. However, \PGFPlots\ will still work as it does in earlier versions, your documents will remain the same if you don't set it explicitly.}.
	Take a look at the next page for the precise definition of |compat|.

	\item \PGFPlots\ 1.3 now supports reversed axes. It is no longer necessary to use work--arounds with negative units.
\pgfkeys{/pdflinks/search key prefixes in/.add={/pgfplots/,}{}}

	Take a look at the |x dir=reverse| key.

	Existing work--arounds will still function properly. Use |\pgfplotsset{compat=1.3}| together with |x dir=reverse| to switch to the new version.
\end{enumerate}

\subsubsection{Old Features Which May Need Attention}
\begin{enumerate}
	\item The |scatter/classes| feature produces proper legends as of version 1.3. This may change the appearance of existing legends of plots with |scatter/classes|.

	\item Due to a small math bug in \PGF\ $2.00$, you \emph{can't} use the math expression `|-x^2|'. It is necessary to use `|0-x^2|' instead. The same holds for
		`|exp(-x^2)|'; use `|exp(0-x^2)|' instead.

		This will be fixed with \PGF\ $>2.00$.

	\item Starting with \PGFPlots\ $1.1$, |\tikzstyle| should \emph{no longer be used} to set \PGFPlots\ options.
	
	Although |\tikzstyle| is still supported for some older \PGFPlots\ options, you should replace any occurance of |\tikzstyle| with |\pgfplotsset{|\meta{style name}|/.style={|\meta{key-value-list}|}}| or the associated |/.append style| variant. See section~\ref{sec:styles} for more detail.
\end{enumerate}
I apologize for any inconvenience caused by these changes.

\begin{pgfplotskey}{compat=\mchoice{1.3,pre 1.3,default,newest} (initially default)}
	The preamble configuration 
\begin{codeexample}[code only]
\usepackage{pgfplots}
\pgfplotsset{compat=1.3}
\end{codeexample}
	allows to choose between backwards compatibility and most recent features.

	\PGFPlots\ version 1.3 comes with new features which may lead to different spacings compared to earlier versions:
	\begin{enumerate}
		\item Version 1.3 comes with the |xlabel near ticks| feature which places (in this case $x$) axis labels ``near'' the tick labels, taking the size of tick labels into account.
		
		Older versions used |xlabel absolute|, i.e.\ an absolute distance between the axis and axis labels (now matter how large or small tick labels are).

		The initial setting \emph{keeps backwards compatibility}.

		You are encouraged to use
\begin{codeexample}[code only]
\usepackage{pgfplots}
\pgfplotsset{compat=1.3}
\end{codeexample}
		in your preamble.
		
		\item Version 1.3 fixes a bug with |anchor=south west| which occurred mainly for reversed axes: the anchors where upside--down. This is fixed now.

		
		If you have written some work--around and want to keep it going, use

		|\pgfplotsset{compat/anchors=pre 1.3}|\pgfmanualpdflabel{/pgfplots/compat/anchors}{} 

		to disable the bug fix.
	\end{enumerate}

	Use |\pgfplotsset{compat=default}| to restore the factory settings.

	The setting |\pgfplotsset{compat=newest}| will always use features of the most recent version. This might result in small changes in the document's appearance.
\end{pgfplotskey}

\subsection{The Team}
\PGFPlots\ has been written mainly by Christian Feuers�nger with many improvements of Pascal Wolkotte and Nick Papior Andersen as a spare time project. We hope it is useful and provides valuable plots.

If you are interested in writing something but don't know how, consider reading the auxiliary manual \href{file:TeX-programming-notes.pdf}{TeX-programming-notes.pdf} which comes with \PGFPlots. It is far from complete, but maybe it is a good starting point (at least for more literature).

\subsection{Acknowledgements}
I thank God for all hours of enjoyed programming. I thank Pascal Wolkotte and Nick Papior Andersen for their programming efforts and contributions as part of the development team. I thank Stefan Tibus, who contributed the |plot shell| feature. I thank Tom Cashman for the contribution of the |reverse legend| feature. Special thanks go to Stefan Pinnow whose tests of \PGFPlots\ lead to numerous quality improvements. Furthermore, I thank Dr.~Schweitzer for many fruitful discussions and Dr.~Meine for his ideas and suggestions. Thanks as well to the many international contributors who provided feature requests or identified bugs or simply manual improvements!

Last but not least, I thank Till Tantau and Mark Wibrow for their excellent graphics (and more) package \PGF\ and \Tikz\ which is the base of \PGFPlots.

\include{pgfplots.install}%
\include{pgfplots.intro}%
\include{pgfplots.reference}%
\include{pgfplots.libs}%
\include{pgfplots.resources}%
\include{pgfplots.importexport}%
\include{pgfplots.basic.reference}%

\printindex

\bibliographystyle{abbrv} %gerapali} %gerabbrv} %gerunsrt.bst} %gerabbrv}% gerplain}
\nocite{pgfplotstable}
\nocite{programmingnotes}
\bibliography{pgfplots}
\end{document}
%
%%%%%%%%%%%%%%%%%%%%%%%%%%%%%%%%%%%%%%%%%%%%%%%%%%%%%%%%%%%%%%%%%%%%%%%%%%%%%
%
% Package pgfplots.sty documentation. 
%
% Copyright 2007/2008 by Christian Feuersaenger.
%
% This program is free software: you can redistribute it and/or modify
% it under the terms of the GNU General Public License as published by
% the Free Software Foundation, either version 3 of the License, or
% (at your option) any later version.
% 
% This program is distributed in the hope that it will be useful,
% but WITHOUT ANY WARRANTY; without even the implied warranty of
% MERCHANTABILITY or FITNESS FOR A PARTICULAR PURPOSE.  See the
% GNU General Public License for more details.
% 
% You should have received a copy of the GNU General Public License
% along with this program.  If not, see <http://www.gnu.org/licenses/>.
%
%
%%%%%%%%%%%%%%%%%%%%%%%%%%%%%%%%%%%%%%%%%%%%%%%%%%%%%%%%%%%%%%%%%%%%%%%%%%%%%
\input{pgfplots.preamble.tex}
%\RequirePackage[german,english,francais]{babel}

\pgfplotsmanualexternalexpensivetrue

%\usetikzlibrary{external}
%\tikzexternalize[prefix=figures/]{pgfplots}

\title{%
	Manual for Package \PGFPlots\\
	{\small Version \pgfplotsversion}\\
	{\small\href{http://sourceforge.net/projects/pgfplots}{http://sourceforge.net/projects/pgfplots}}}

%\includeonly{pgfplots.libs}


\begin{document}

\def\plotcoords{%
\addplot coordinates {
(5,8.312e-02)    (17,2.547e-02)   (49,7.407e-03)
(129,2.102e-03)  (321,5.874e-04)  (769,1.623e-04)
(1793,4.442e-05) (4097,1.207e-05) (9217,3.261e-06)
};

\addplot coordinates{
(7,8.472e-02)    (31,3.044e-02)    (111,1.022e-02)
(351,3.303e-03)  (1023,1.039e-03)  (2815,3.196e-04)
(7423,9.658e-05) (18943,2.873e-05) (47103,8.437e-06)
};

\addplot coordinates{
(9,7.881e-02)     (49,3.243e-02)    (209,1.232e-02)
(769,4.454e-03)   (2561,1.551e-03)  (7937,5.236e-04)
(23297,1.723e-04) (65537,5.545e-05) (178177,1.751e-05)
};

\addplot coordinates{
(11,6.887e-02)    (71,3.177e-02)     (351,1.341e-02)
(1471,5.334e-03)  (5503,2.027e-03)   (18943,7.415e-04)
(61183,2.628e-04) (187903,9.063e-05) (553983,3.053e-05)
};

\addplot coordinates{
(13,5.755e-02)     (97,2.925e-02)     (545,1.351e-02)
(2561,5.842e-03)   (10625,2.397e-03)  (40193,9.414e-04)
(141569,3.564e-04) (471041,1.308e-04) 
(1496065,4.670e-05)
};
}%
\maketitle
\begin{abstract}%
\PGFPlots\ draws high--quality function plots in normal or logarithmic scaling with a user-friendly interface directly in \TeX. The user supplies axis labels, legend entries and the plot coordinates for one or more plots and \PGFPlots\ applies axis scaling, computes any logarithms and axis ticks and draws the plots, supporting line plots, scatter plots, piecewise constant plots, bar plots, area plots, mesh-- and surface plots and some more. It is based on Till Tantau's package \PGF/\Tikz.
\end{abstract}
\tableofcontents
\section{Introduction}
This package provides tools to generate plots and labeled axes easily. It draws normal plots, logplots and semi-logplots, in two and three dimensions. Axis ticks, labels, legends (in case of multiple plots) can be added with key-value options. It can cycle through a set of predefined line/marker/color specifications. In summary, its purpose is to simplify the generation of high-quality function and/or data plots, and solving the problems of
\begin{itemize}
	\item consistency of document and font type and font size,
	\item direct use of \TeX\ math mode in axis descriptions,
	\item consistency of data and figures (no third party tool necessary),
	\item inter document consistency using preamble configurations and styles.
\end{itemize}
Although not necessary, separate |.pdf| or |.eps| graphics can be generated using the |external| library developed as part of \Tikz.

\PGFPlots\ is build completely on \Tikz/\PGF. Knowledge of \Tikz\ will simplify the work with \PGFPlots, although it is not required.

\section[About PGFPlots: Preliminaries]{About {\normalfont\PGFPlots}: Preliminaries}
This section contains information about upgrades, the team, the installation (in case you need to do it manually) and troubleshooting. You may skip it completely except for the upgrade remarks.

However, note that this library requires at least \PGF\ version $2.00$. At the time of this writing, many \TeX-distributions still contain the older \PGF\ version $1.18$, so it may be necessary to install a recent \PGF\ prior to using \PGFPlots.

\subsection{Components}
\PGFPlots\ comes with two components:
\begin{enumerate}
	\item the plotting component (which you are currently reading) and
	\item the \PGFPlotstable\ component which simplifies number formatting and postprocessing of numerical tables. It comes as a separate package and has its own manual \href{file:pgfplotstable.pdf}{pgfplotstable.pdf}.
\end{enumerate}

\subsection{Upgrade remarks}
This release provides a lot of improvements which can be found in all detail in \texttt{ChangeLog} for interested readers. However, some attention is useful with respect to the following changes.

\subsubsection{New Optional Features}
\begin{enumerate}
	\item \PGFPlots\ 1.3 comes with user interface improvements. The technical distinction between ``behavior options'' and ``style options'' of older versions is no longer necessary (although still fully supported).

	\item \PGFPlots\ 1.3 has a new feature which allows to \emph{move axis labels tight to tick labels} automatically.

	Since this affects the spacing, it is not enabled be default.

	Use
\begin{codeexample}[code only]
\usepackage{pgfplots}
\pgfplotsset{compat=1.3}
\end{codeexample}
	in your preamble to benefit from the improved spacing\footnote{The |compat=1.3| setting is necessary for new features which move axis labels around like reversed axis. However, \PGFPlots\ will still work as it does in earlier versions, your documents will remain the same if you don't set it explicitly.}.
	Take a look at the next page for the precise definition of |compat|.

	\item \PGFPlots\ 1.3 now supports reversed axes. It is no longer necessary to use work--arounds with negative units.
\pgfkeys{/pdflinks/search key prefixes in/.add={/pgfplots/,}{}}

	Take a look at the |x dir=reverse| key.

	Existing work--arounds will still function properly. Use |\pgfplotsset{compat=1.3}| together with |x dir=reverse| to switch to the new version.
\end{enumerate}

\subsubsection{Old Features Which May Need Attention}
\begin{enumerate}
	\item The |scatter/classes| feature produces proper legends as of version 1.3. This may change the appearance of existing legends of plots with |scatter/classes|.

	\item Due to a small math bug in \PGF\ $2.00$, you \emph{can't} use the math expression `|-x^2|'. It is necessary to use `|0-x^2|' instead. The same holds for
		`|exp(-x^2)|'; use `|exp(0-x^2)|' instead.

		This will be fixed with \PGF\ $>2.00$.

	\item Starting with \PGFPlots\ $1.1$, |\tikzstyle| should \emph{no longer be used} to set \PGFPlots\ options.
	
	Although |\tikzstyle| is still supported for some older \PGFPlots\ options, you should replace any occurance of |\tikzstyle| with |\pgfplotsset{|\meta{style name}|/.style={|\meta{key-value-list}|}}| or the associated |/.append style| variant. See section~\ref{sec:styles} for more detail.
\end{enumerate}
I apologize for any inconvenience caused by these changes.

\begin{pgfplotskey}{compat=\mchoice{1.3,pre 1.3,default,newest} (initially default)}
	The preamble configuration 
\begin{codeexample}[code only]
\usepackage{pgfplots}
\pgfplotsset{compat=1.3}
\end{codeexample}
	allows to choose between backwards compatibility and most recent features.

	\PGFPlots\ version 1.3 comes with new features which may lead to different spacings compared to earlier versions:
	\begin{enumerate}
		\item Version 1.3 comes with the |xlabel near ticks| feature which places (in this case $x$) axis labels ``near'' the tick labels, taking the size of tick labels into account.
		
		Older versions used |xlabel absolute|, i.e.\ an absolute distance between the axis and axis labels (now matter how large or small tick labels are).

		The initial setting \emph{keeps backwards compatibility}.

		You are encouraged to use
\begin{codeexample}[code only]
\usepackage{pgfplots}
\pgfplotsset{compat=1.3}
\end{codeexample}
		in your preamble.
		
		\item Version 1.3 fixes a bug with |anchor=south west| which occurred mainly for reversed axes: the anchors where upside--down. This is fixed now.

		
		If you have written some work--around and want to keep it going, use

		|\pgfplotsset{compat/anchors=pre 1.3}|\pgfmanualpdflabel{/pgfplots/compat/anchors}{} 

		to disable the bug fix.
	\end{enumerate}

	Use |\pgfplotsset{compat=default}| to restore the factory settings.

	The setting |\pgfplotsset{compat=newest}| will always use features of the most recent version. This might result in small changes in the document's appearance.
\end{pgfplotskey}

\subsection{The Team}
\PGFPlots\ has been written mainly by Christian Feuers�nger with many improvements of Pascal Wolkotte and Nick Papior Andersen as a spare time project. We hope it is useful and provides valuable plots.

If you are interested in writing something but don't know how, consider reading the auxiliary manual \href{file:TeX-programming-notes.pdf}{TeX-programming-notes.pdf} which comes with \PGFPlots. It is far from complete, but maybe it is a good starting point (at least for more literature).

\subsection{Acknowledgements}
I thank God for all hours of enjoyed programming. I thank Pascal Wolkotte and Nick Papior Andersen for their programming efforts and contributions as part of the development team. I thank Stefan Tibus, who contributed the |plot shell| feature. I thank Tom Cashman for the contribution of the |reverse legend| feature. Special thanks go to Stefan Pinnow whose tests of \PGFPlots\ lead to numerous quality improvements. Furthermore, I thank Dr.~Schweitzer for many fruitful discussions and Dr.~Meine for his ideas and suggestions. Thanks as well to the many international contributors who provided feature requests or identified bugs or simply manual improvements!

Last but not least, I thank Till Tantau and Mark Wibrow for their excellent graphics (and more) package \PGF\ and \Tikz\ which is the base of \PGFPlots.

\include{pgfplots.install}%
\include{pgfplots.intro}%
\include{pgfplots.reference}%
\include{pgfplots.libs}%
\include{pgfplots.resources}%
\include{pgfplots.importexport}%
\include{pgfplots.basic.reference}%

\printindex

\bibliographystyle{abbrv} %gerapali} %gerabbrv} %gerunsrt.bst} %gerabbrv}% gerplain}
\nocite{pgfplotstable}
\nocite{programmingnotes}
\bibliography{pgfplots}
\end{document}
%
\include{pgfplots.basic.reference}%

\printindex

\bibliographystyle{abbrv} %gerapali} %gerabbrv} %gerunsrt.bst} %gerabbrv}% gerplain}
\nocite{pgfplotstable}
\nocite{programmingnotes}
\bibliography{pgfplots}
\end{document}
%
%%%%%%%%%%%%%%%%%%%%%%%%%%%%%%%%%%%%%%%%%%%%%%%%%%%%%%%%%%%%%%%%%%%%%%%%%%%%%
%
% Package pgfplots.sty documentation. 
%
% Copyright 2007/2008 by Christian Feuersaenger.
%
% This program is free software: you can redistribute it and/or modify
% it under the terms of the GNU General Public License as published by
% the Free Software Foundation, either version 3 of the License, or
% (at your option) any later version.
% 
% This program is distributed in the hope that it will be useful,
% but WITHOUT ANY WARRANTY; without even the implied warranty of
% MERCHANTABILITY or FITNESS FOR A PARTICULAR PURPOSE.  See the
% GNU General Public License for more details.
% 
% You should have received a copy of the GNU General Public License
% along with this program.  If not, see <http://www.gnu.org/licenses/>.
%
%
%%%%%%%%%%%%%%%%%%%%%%%%%%%%%%%%%%%%%%%%%%%%%%%%%%%%%%%%%%%%%%%%%%%%%%%%%%%%%
%%%%%%%%%%%%%%%%%%%%%%%%%%%%%%%%%%%%%%%%%%%%%%%%%%%%%%%%%%%%%%%%%%%%%%%%%%%%%
%
% Package pgfplots.sty documentation. 
%
% Copyright 2007/2008 by Christian Feuersaenger.
%
% This program is free software: you can redistribute it and/or modify
% it under the terms of the GNU General Public License as published by
% the Free Software Foundation, either version 3 of the License, or
% (at your option) any later version.
% 
% This program is distributed in the hope that it will be useful,
% but WITHOUT ANY WARRANTY; without even the implied warranty of
% MERCHANTABILITY or FITNESS FOR A PARTICULAR PURPOSE.  See the
% GNU General Public License for more details.
% 
% You should have received a copy of the GNU General Public License
% along with this program.  If not, see <http://www.gnu.org/licenses/>.
%
%
%%%%%%%%%%%%%%%%%%%%%%%%%%%%%%%%%%%%%%%%%%%%%%%%%%%%%%%%%%%%%%%%%%%%%%%%%%%%%
\documentclass[a4paper]{ltxdoc}

\usepackage{makeidx}

% DON't let hyperref overload the format of index and glossary. 
% I want to do that on my own in the stylefiles for makeindex...
\makeatletter
\let\@old@wrindex=\@wrindex
\makeatother

\usepackage{ifpdf}
\ifpdf
	\usepackage[pdftex]{hyperref}
\else
	\usepackage[dvipdfm]{hyperref}
\fi
\hypersetup{pdfborder={0 0 0}}

\makeatletter
\let\@wrindex=\@old@wrindex
\makeatother

% Formatiere Seitennummern im Index:
\newcommand{\indexpageno}[1]{%
	{\bfseries\hyperpage{#1}}%
}


\newcommand{\R}{\mathbb{R}}
\newcommand{\N}{\mathbb{N}}
\newcommand{\Z}{\mathbb{Z}}

\long\def\COMMENTLOWLEVEL#1\ENDCOMMENT{}
\def\ENDCOMMENT{}

\usepackage{textcomp}
\usepackage{booktabs}

\usepackage{calc}
\usepackage[formats]{listings}
%\usepackage{courier} % don't use it - the '^' character can't be copy-pasted in courier

\usepackage{array}
\lstset{%
	basicstyle=\ttfamily,
	language=[LaTeX]tex, % Seems as if \lstset{language=tex} must be invoked BEFORE loading tikz!?
	tabsize=4,
	breaklines=true,
	breakindent=0pt
}

\ifpdf
	\pdfinfo {
		/Author	(Christian Feuersaenger)
	}

\else
	\def\pgfsysdriver{pgfsys-dvipdfm.def}
\fi
%\def\pgfsysdriver{pgfsys-pdftex.def}
\usepackage{pgfplots}

\ifpdf
	\usetikzlibrary{pgfplotsclickable}
\fi

%\usepackage{fp}
% ATTENTION:
% this requires pgf version NEWER than 2.00 :
%\usetikzlibrary{fixedpointarithmetic}

\usetikzlibrary{dateplot}

\usepackage[a4paper,left=2.25cm,right=2.25cm,top=2.5cm,bottom=2.5cm,nohead]{geometry}
\usepackage{amsmath,amssymb}
\usepackage{xxcolor}
\usepackage{pifont}
\usepackage[latin1]{inputenc}
\usepackage{amsmath}
\usepackage{eurosym}
\input{pgfplots-macros}

\pgfqkeys{/codeexample}{%
	every codeexample/.style={
		width=8cm,
		/pgfplots/every axis/.append style={legend style={fill=graphicbackground}}
	},
	tabsize=4,
}
\pgfplotsset{
	every axis/.append style={width=7cm},
	filter discard warning=false,
}

\usetikzlibrary{backgrounds,patterns}
% Global styles:
\tikzset{
  shape example/.style={
    color=black!30,
    draw,
    fill=yellow!30,
    line width=.5cm,
    inner xsep=2.5cm,
    inner ysep=0.5cm}
}

\newcommand{\FIXME}[1]{\textcolor{red}{(FIXME: #1)}}

% fuer endvironment 'sidewaysfigure' bspw
% \usepackage{rotating}

\newcommand\Tikz{Ti\textit kZ}
\newcommand\PGF{\textsc{pgf}}
\newcommand\PGFPlots{\textsc{pgfplots}}
\def\PGFPlotstable{\textsc{PgfplotsTable}}


\makeindex

% Fix overful hboxes automatically:
\tolerance=2000
\emergencystretch=10pt

\tikzset{prefix=gnuplot/pgfplots_} % prefix for 'plot function'

\author{%
	Christian Feuers\"anger\footnote{\url{http://wissrech.ins.uni-bonn.de/people/feuersaenger}}\\%
	Institut f\"ur Numerische Simulation\\
	Universit\"at Bonn, Germany}


%\RequirePackage[german,english,francais]{babel}

\pgfplotsmanualexternalexpensivetrue

%\usetikzlibrary{external}
%\tikzexternalize[prefix=figures/]{pgfplots}

\title{%
	Manual for Package \PGFPlots\\
	{\small Version \pgfplotsversion}\\
	{\small\href{http://sourceforge.net/projects/pgfplots}{http://sourceforge.net/projects/pgfplots}}}

%\includeonly{pgfplots.libs}


\begin{document}

\def\plotcoords{%
\addplot coordinates {
(5,8.312e-02)    (17,2.547e-02)   (49,7.407e-03)
(129,2.102e-03)  (321,5.874e-04)  (769,1.623e-04)
(1793,4.442e-05) (4097,1.207e-05) (9217,3.261e-06)
};

\addplot coordinates{
(7,8.472e-02)    (31,3.044e-02)    (111,1.022e-02)
(351,3.303e-03)  (1023,1.039e-03)  (2815,3.196e-04)
(7423,9.658e-05) (18943,2.873e-05) (47103,8.437e-06)
};

\addplot coordinates{
(9,7.881e-02)     (49,3.243e-02)    (209,1.232e-02)
(769,4.454e-03)   (2561,1.551e-03)  (7937,5.236e-04)
(23297,1.723e-04) (65537,5.545e-05) (178177,1.751e-05)
};

\addplot coordinates{
(11,6.887e-02)    (71,3.177e-02)     (351,1.341e-02)
(1471,5.334e-03)  (5503,2.027e-03)   (18943,7.415e-04)
(61183,2.628e-04) (187903,9.063e-05) (553983,3.053e-05)
};

\addplot coordinates{
(13,5.755e-02)     (97,2.925e-02)     (545,1.351e-02)
(2561,5.842e-03)   (10625,2.397e-03)  (40193,9.414e-04)
(141569,3.564e-04) (471041,1.308e-04) 
(1496065,4.670e-05)
};
}%
\maketitle
\begin{abstract}%
\PGFPlots\ draws high--quality function plots in normal or logarithmic scaling with a user-friendly interface directly in \TeX. The user supplies axis labels, legend entries and the plot coordinates for one or more plots and \PGFPlots\ applies axis scaling, computes any logarithms and axis ticks and draws the plots, supporting line plots, scatter plots, piecewise constant plots, bar plots, area plots, mesh-- and surface plots and some more. It is based on Till Tantau's package \PGF/\Tikz.
\end{abstract}
\tableofcontents
\section{Introduction}
This package provides tools to generate plots and labeled axes easily. It draws normal plots, logplots and semi-logplots, in two and three dimensions. Axis ticks, labels, legends (in case of multiple plots) can be added with key-value options. It can cycle through a set of predefined line/marker/color specifications. In summary, its purpose is to simplify the generation of high-quality function and/or data plots, and solving the problems of
\begin{itemize}
	\item consistency of document and font type and font size,
	\item direct use of \TeX\ math mode in axis descriptions,
	\item consistency of data and figures (no third party tool necessary),
	\item inter document consistency using preamble configurations and styles.
\end{itemize}
Although not necessary, separate |.pdf| or |.eps| graphics can be generated using the |external| library developed as part of \Tikz.

\PGFPlots\ is build completely on \Tikz/\PGF. Knowledge of \Tikz\ will simplify the work with \PGFPlots, although it is not required.

\section[About PGFPlots: Preliminaries]{About {\normalfont\PGFPlots}: Preliminaries}
This section contains information about upgrades, the team, the installation (in case you need to do it manually) and troubleshooting. You may skip it completely except for the upgrade remarks.

However, note that this library requires at least \PGF\ version $2.00$. At the time of this writing, many \TeX-distributions still contain the older \PGF\ version $1.18$, so it may be necessary to install a recent \PGF\ prior to using \PGFPlots.

\subsection{Components}
\PGFPlots\ comes with two components:
\begin{enumerate}
	\item the plotting component (which you are currently reading) and
	\item the \PGFPlotstable\ component which simplifies number formatting and postprocessing of numerical tables. It comes as a separate package and has its own manual \href{file:pgfplotstable.pdf}{pgfplotstable.pdf}.
\end{enumerate}

\subsection{Upgrade remarks}
This release provides a lot of improvements which can be found in all detail in \texttt{ChangeLog} for interested readers. However, some attention is useful with respect to the following changes.

\subsubsection{New Optional Features}
\begin{enumerate}
	\item \PGFPlots\ 1.3 comes with user interface improvements. The technical distinction between ``behavior options'' and ``style options'' of older versions is no longer necessary (although still fully supported).

	\item \PGFPlots\ 1.3 has a new feature which allows to \emph{move axis labels tight to tick labels} automatically.

	Since this affects the spacing, it is not enabled be default.

	Use
\begin{codeexample}[code only]
\usepackage{pgfplots}
\pgfplotsset{compat=1.3}
\end{codeexample}
	in your preamble to benefit from the improved spacing\footnote{The |compat=1.3| setting is necessary for new features which move axis labels around like reversed axis. However, \PGFPlots\ will still work as it does in earlier versions, your documents will remain the same if you don't set it explicitly.}.
	Take a look at the next page for the precise definition of |compat|.

	\item \PGFPlots\ 1.3 now supports reversed axes. It is no longer necessary to use work--arounds with negative units.
\pgfkeys{/pdflinks/search key prefixes in/.add={/pgfplots/,}{}}

	Take a look at the |x dir=reverse| key.

	Existing work--arounds will still function properly. Use |\pgfplotsset{compat=1.3}| together with |x dir=reverse| to switch to the new version.
\end{enumerate}

\subsubsection{Old Features Which May Need Attention}
\begin{enumerate}
	\item The |scatter/classes| feature produces proper legends as of version 1.3. This may change the appearance of existing legends of plots with |scatter/classes|.

	\item Due to a small math bug in \PGF\ $2.00$, you \emph{can't} use the math expression `|-x^2|'. It is necessary to use `|0-x^2|' instead. The same holds for
		`|exp(-x^2)|'; use `|exp(0-x^2)|' instead.

		This will be fixed with \PGF\ $>2.00$.

	\item Starting with \PGFPlots\ $1.1$, |\tikzstyle| should \emph{no longer be used} to set \PGFPlots\ options.
	
	Although |\tikzstyle| is still supported for some older \PGFPlots\ options, you should replace any occurance of |\tikzstyle| with |\pgfplotsset{|\meta{style name}|/.style={|\meta{key-value-list}|}}| or the associated |/.append style| variant. See section~\ref{sec:styles} for more detail.
\end{enumerate}
I apologize for any inconvenience caused by these changes.

\begin{pgfplotskey}{compat=\mchoice{1.3,pre 1.3,default,newest} (initially default)}
	The preamble configuration 
\begin{codeexample}[code only]
\usepackage{pgfplots}
\pgfplotsset{compat=1.3}
\end{codeexample}
	allows to choose between backwards compatibility and most recent features.

	\PGFPlots\ version 1.3 comes with new features which may lead to different spacings compared to earlier versions:
	\begin{enumerate}
		\item Version 1.3 comes with the |xlabel near ticks| feature which places (in this case $x$) axis labels ``near'' the tick labels, taking the size of tick labels into account.
		
		Older versions used |xlabel absolute|, i.e.\ an absolute distance between the axis and axis labels (now matter how large or small tick labels are).

		The initial setting \emph{keeps backwards compatibility}.

		You are encouraged to use
\begin{codeexample}[code only]
\usepackage{pgfplots}
\pgfplotsset{compat=1.3}
\end{codeexample}
		in your preamble.
		
		\item Version 1.3 fixes a bug with |anchor=south west| which occurred mainly for reversed axes: the anchors where upside--down. This is fixed now.

		
		If you have written some work--around and want to keep it going, use

		|\pgfplotsset{compat/anchors=pre 1.3}|\pgfmanualpdflabel{/pgfplots/compat/anchors}{} 

		to disable the bug fix.
	\end{enumerate}

	Use |\pgfplotsset{compat=default}| to restore the factory settings.

	The setting |\pgfplotsset{compat=newest}| will always use features of the most recent version. This might result in small changes in the document's appearance.
\end{pgfplotskey}

\subsection{The Team}
\PGFPlots\ has been written mainly by Christian Feuers�nger with many improvements of Pascal Wolkotte and Nick Papior Andersen as a spare time project. We hope it is useful and provides valuable plots.

If you are interested in writing something but don't know how, consider reading the auxiliary manual \href{file:TeX-programming-notes.pdf}{TeX-programming-notes.pdf} which comes with \PGFPlots. It is far from complete, but maybe it is a good starting point (at least for more literature).

\subsection{Acknowledgements}
I thank God for all hours of enjoyed programming. I thank Pascal Wolkotte and Nick Papior Andersen for their programming efforts and contributions as part of the development team. I thank Stefan Tibus, who contributed the |plot shell| feature. I thank Tom Cashman for the contribution of the |reverse legend| feature. Special thanks go to Stefan Pinnow whose tests of \PGFPlots\ lead to numerous quality improvements. Furthermore, I thank Dr.~Schweitzer for many fruitful discussions and Dr.~Meine for his ideas and suggestions. Thanks as well to the many international contributors who provided feature requests or identified bugs or simply manual improvements!

Last but not least, I thank Till Tantau and Mark Wibrow for their excellent graphics (and more) package \PGF\ and \Tikz\ which is the base of \PGFPlots.

%%%%%%%%%%%%%%%%%%%%%%%%%%%%%%%%%%%%%%%%%%%%%%%%%%%%%%%%%%%%%%%%%%%%%%%%%%%%%
%
% Package pgfplots.sty documentation. 
%
% Copyright 2007/2008 by Christian Feuersaenger.
%
% This program is free software: you can redistribute it and/or modify
% it under the terms of the GNU General Public License as published by
% the Free Software Foundation, either version 3 of the License, or
% (at your option) any later version.
% 
% This program is distributed in the hope that it will be useful,
% but WITHOUT ANY WARRANTY; without even the implied warranty of
% MERCHANTABILITY or FITNESS FOR A PARTICULAR PURPOSE.  See the
% GNU General Public License for more details.
% 
% You should have received a copy of the GNU General Public License
% along with this program.  If not, see <http://www.gnu.org/licenses/>.
%
%
%%%%%%%%%%%%%%%%%%%%%%%%%%%%%%%%%%%%%%%%%%%%%%%%%%%%%%%%%%%%%%%%%%%%%%%%%%%%%
\input{pgfplots.preamble.tex}
%\RequirePackage[german,english,francais]{babel}

\pgfplotsmanualexternalexpensivetrue

%\usetikzlibrary{external}
%\tikzexternalize[prefix=figures/]{pgfplots}

\title{%
	Manual for Package \PGFPlots\\
	{\small Version \pgfplotsversion}\\
	{\small\href{http://sourceforge.net/projects/pgfplots}{http://sourceforge.net/projects/pgfplots}}}

%\includeonly{pgfplots.libs}


\begin{document}

\def\plotcoords{%
\addplot coordinates {
(5,8.312e-02)    (17,2.547e-02)   (49,7.407e-03)
(129,2.102e-03)  (321,5.874e-04)  (769,1.623e-04)
(1793,4.442e-05) (4097,1.207e-05) (9217,3.261e-06)
};

\addplot coordinates{
(7,8.472e-02)    (31,3.044e-02)    (111,1.022e-02)
(351,3.303e-03)  (1023,1.039e-03)  (2815,3.196e-04)
(7423,9.658e-05) (18943,2.873e-05) (47103,8.437e-06)
};

\addplot coordinates{
(9,7.881e-02)     (49,3.243e-02)    (209,1.232e-02)
(769,4.454e-03)   (2561,1.551e-03)  (7937,5.236e-04)
(23297,1.723e-04) (65537,5.545e-05) (178177,1.751e-05)
};

\addplot coordinates{
(11,6.887e-02)    (71,3.177e-02)     (351,1.341e-02)
(1471,5.334e-03)  (5503,2.027e-03)   (18943,7.415e-04)
(61183,2.628e-04) (187903,9.063e-05) (553983,3.053e-05)
};

\addplot coordinates{
(13,5.755e-02)     (97,2.925e-02)     (545,1.351e-02)
(2561,5.842e-03)   (10625,2.397e-03)  (40193,9.414e-04)
(141569,3.564e-04) (471041,1.308e-04) 
(1496065,4.670e-05)
};
}%
\maketitle
\begin{abstract}%
\PGFPlots\ draws high--quality function plots in normal or logarithmic scaling with a user-friendly interface directly in \TeX. The user supplies axis labels, legend entries and the plot coordinates for one or more plots and \PGFPlots\ applies axis scaling, computes any logarithms and axis ticks and draws the plots, supporting line plots, scatter plots, piecewise constant plots, bar plots, area plots, mesh-- and surface plots and some more. It is based on Till Tantau's package \PGF/\Tikz.
\end{abstract}
\tableofcontents
\section{Introduction}
This package provides tools to generate plots and labeled axes easily. It draws normal plots, logplots and semi-logplots, in two and three dimensions. Axis ticks, labels, legends (in case of multiple plots) can be added with key-value options. It can cycle through a set of predefined line/marker/color specifications. In summary, its purpose is to simplify the generation of high-quality function and/or data plots, and solving the problems of
\begin{itemize}
	\item consistency of document and font type and font size,
	\item direct use of \TeX\ math mode in axis descriptions,
	\item consistency of data and figures (no third party tool necessary),
	\item inter document consistency using preamble configurations and styles.
\end{itemize}
Although not necessary, separate |.pdf| or |.eps| graphics can be generated using the |external| library developed as part of \Tikz.

\PGFPlots\ is build completely on \Tikz/\PGF. Knowledge of \Tikz\ will simplify the work with \PGFPlots, although it is not required.

\section[About PGFPlots: Preliminaries]{About {\normalfont\PGFPlots}: Preliminaries}
This section contains information about upgrades, the team, the installation (in case you need to do it manually) and troubleshooting. You may skip it completely except for the upgrade remarks.

However, note that this library requires at least \PGF\ version $2.00$. At the time of this writing, many \TeX-distributions still contain the older \PGF\ version $1.18$, so it may be necessary to install a recent \PGF\ prior to using \PGFPlots.

\subsection{Components}
\PGFPlots\ comes with two components:
\begin{enumerate}
	\item the plotting component (which you are currently reading) and
	\item the \PGFPlotstable\ component which simplifies number formatting and postprocessing of numerical tables. It comes as a separate package and has its own manual \href{file:pgfplotstable.pdf}{pgfplotstable.pdf}.
\end{enumerate}

\subsection{Upgrade remarks}
This release provides a lot of improvements which can be found in all detail in \texttt{ChangeLog} for interested readers. However, some attention is useful with respect to the following changes.

\subsubsection{New Optional Features}
\begin{enumerate}
	\item \PGFPlots\ 1.3 comes with user interface improvements. The technical distinction between ``behavior options'' and ``style options'' of older versions is no longer necessary (although still fully supported).

	\item \PGFPlots\ 1.3 has a new feature which allows to \emph{move axis labels tight to tick labels} automatically.

	Since this affects the spacing, it is not enabled be default.

	Use
\begin{codeexample}[code only]
\usepackage{pgfplots}
\pgfplotsset{compat=1.3}
\end{codeexample}
	in your preamble to benefit from the improved spacing\footnote{The |compat=1.3| setting is necessary for new features which move axis labels around like reversed axis. However, \PGFPlots\ will still work as it does in earlier versions, your documents will remain the same if you don't set it explicitly.}.
	Take a look at the next page for the precise definition of |compat|.

	\item \PGFPlots\ 1.3 now supports reversed axes. It is no longer necessary to use work--arounds with negative units.
\pgfkeys{/pdflinks/search key prefixes in/.add={/pgfplots/,}{}}

	Take a look at the |x dir=reverse| key.

	Existing work--arounds will still function properly. Use |\pgfplotsset{compat=1.3}| together with |x dir=reverse| to switch to the new version.
\end{enumerate}

\subsubsection{Old Features Which May Need Attention}
\begin{enumerate}
	\item The |scatter/classes| feature produces proper legends as of version 1.3. This may change the appearance of existing legends of plots with |scatter/classes|.

	\item Due to a small math bug in \PGF\ $2.00$, you \emph{can't} use the math expression `|-x^2|'. It is necessary to use `|0-x^2|' instead. The same holds for
		`|exp(-x^2)|'; use `|exp(0-x^2)|' instead.

		This will be fixed with \PGF\ $>2.00$.

	\item Starting with \PGFPlots\ $1.1$, |\tikzstyle| should \emph{no longer be used} to set \PGFPlots\ options.
	
	Although |\tikzstyle| is still supported for some older \PGFPlots\ options, you should replace any occurance of |\tikzstyle| with |\pgfplotsset{|\meta{style name}|/.style={|\meta{key-value-list}|}}| or the associated |/.append style| variant. See section~\ref{sec:styles} for more detail.
\end{enumerate}
I apologize for any inconvenience caused by these changes.

\begin{pgfplotskey}{compat=\mchoice{1.3,pre 1.3,default,newest} (initially default)}
	The preamble configuration 
\begin{codeexample}[code only]
\usepackage{pgfplots}
\pgfplotsset{compat=1.3}
\end{codeexample}
	allows to choose between backwards compatibility and most recent features.

	\PGFPlots\ version 1.3 comes with new features which may lead to different spacings compared to earlier versions:
	\begin{enumerate}
		\item Version 1.3 comes with the |xlabel near ticks| feature which places (in this case $x$) axis labels ``near'' the tick labels, taking the size of tick labels into account.
		
		Older versions used |xlabel absolute|, i.e.\ an absolute distance between the axis and axis labels (now matter how large or small tick labels are).

		The initial setting \emph{keeps backwards compatibility}.

		You are encouraged to use
\begin{codeexample}[code only]
\usepackage{pgfplots}
\pgfplotsset{compat=1.3}
\end{codeexample}
		in your preamble.
		
		\item Version 1.3 fixes a bug with |anchor=south west| which occurred mainly for reversed axes: the anchors where upside--down. This is fixed now.

		
		If you have written some work--around and want to keep it going, use

		|\pgfplotsset{compat/anchors=pre 1.3}|\pgfmanualpdflabel{/pgfplots/compat/anchors}{} 

		to disable the bug fix.
	\end{enumerate}

	Use |\pgfplotsset{compat=default}| to restore the factory settings.

	The setting |\pgfplotsset{compat=newest}| will always use features of the most recent version. This might result in small changes in the document's appearance.
\end{pgfplotskey}

\subsection{The Team}
\PGFPlots\ has been written mainly by Christian Feuers�nger with many improvements of Pascal Wolkotte and Nick Papior Andersen as a spare time project. We hope it is useful and provides valuable plots.

If you are interested in writing something but don't know how, consider reading the auxiliary manual \href{file:TeX-programming-notes.pdf}{TeX-programming-notes.pdf} which comes with \PGFPlots. It is far from complete, but maybe it is a good starting point (at least for more literature).

\subsection{Acknowledgements}
I thank God for all hours of enjoyed programming. I thank Pascal Wolkotte and Nick Papior Andersen for their programming efforts and contributions as part of the development team. I thank Stefan Tibus, who contributed the |plot shell| feature. I thank Tom Cashman for the contribution of the |reverse legend| feature. Special thanks go to Stefan Pinnow whose tests of \PGFPlots\ lead to numerous quality improvements. Furthermore, I thank Dr.~Schweitzer for many fruitful discussions and Dr.~Meine for his ideas and suggestions. Thanks as well to the many international contributors who provided feature requests or identified bugs or simply manual improvements!

Last but not least, I thank Till Tantau and Mark Wibrow for their excellent graphics (and more) package \PGF\ and \Tikz\ which is the base of \PGFPlots.

\include{pgfplots.install}%
\include{pgfplots.intro}%
\include{pgfplots.reference}%
\include{pgfplots.libs}%
\include{pgfplots.resources}%
\include{pgfplots.importexport}%
\include{pgfplots.basic.reference}%

\printindex

\bibliographystyle{abbrv} %gerapali} %gerabbrv} %gerunsrt.bst} %gerabbrv}% gerplain}
\nocite{pgfplotstable}
\nocite{programmingnotes}
\bibliography{pgfplots}
\end{document}
%
%%%%%%%%%%%%%%%%%%%%%%%%%%%%%%%%%%%%%%%%%%%%%%%%%%%%%%%%%%%%%%%%%%%%%%%%%%%%%
%
% Package pgfplots.sty documentation. 
%
% Copyright 2007/2008 by Christian Feuersaenger.
%
% This program is free software: you can redistribute it and/or modify
% it under the terms of the GNU General Public License as published by
% the Free Software Foundation, either version 3 of the License, or
% (at your option) any later version.
% 
% This program is distributed in the hope that it will be useful,
% but WITHOUT ANY WARRANTY; without even the implied warranty of
% MERCHANTABILITY or FITNESS FOR A PARTICULAR PURPOSE.  See the
% GNU General Public License for more details.
% 
% You should have received a copy of the GNU General Public License
% along with this program.  If not, see <http://www.gnu.org/licenses/>.
%
%
%%%%%%%%%%%%%%%%%%%%%%%%%%%%%%%%%%%%%%%%%%%%%%%%%%%%%%%%%%%%%%%%%%%%%%%%%%%%%
\input{pgfplots.preamble.tex}
%\RequirePackage[german,english,francais]{babel}

\pgfplotsmanualexternalexpensivetrue

%\usetikzlibrary{external}
%\tikzexternalize[prefix=figures/]{pgfplots}

\title{%
	Manual for Package \PGFPlots\\
	{\small Version \pgfplotsversion}\\
	{\small\href{http://sourceforge.net/projects/pgfplots}{http://sourceforge.net/projects/pgfplots}}}

%\includeonly{pgfplots.libs}


\begin{document}

\def\plotcoords{%
\addplot coordinates {
(5,8.312e-02)    (17,2.547e-02)   (49,7.407e-03)
(129,2.102e-03)  (321,5.874e-04)  (769,1.623e-04)
(1793,4.442e-05) (4097,1.207e-05) (9217,3.261e-06)
};

\addplot coordinates{
(7,8.472e-02)    (31,3.044e-02)    (111,1.022e-02)
(351,3.303e-03)  (1023,1.039e-03)  (2815,3.196e-04)
(7423,9.658e-05) (18943,2.873e-05) (47103,8.437e-06)
};

\addplot coordinates{
(9,7.881e-02)     (49,3.243e-02)    (209,1.232e-02)
(769,4.454e-03)   (2561,1.551e-03)  (7937,5.236e-04)
(23297,1.723e-04) (65537,5.545e-05) (178177,1.751e-05)
};

\addplot coordinates{
(11,6.887e-02)    (71,3.177e-02)     (351,1.341e-02)
(1471,5.334e-03)  (5503,2.027e-03)   (18943,7.415e-04)
(61183,2.628e-04) (187903,9.063e-05) (553983,3.053e-05)
};

\addplot coordinates{
(13,5.755e-02)     (97,2.925e-02)     (545,1.351e-02)
(2561,5.842e-03)   (10625,2.397e-03)  (40193,9.414e-04)
(141569,3.564e-04) (471041,1.308e-04) 
(1496065,4.670e-05)
};
}%
\maketitle
\begin{abstract}%
\PGFPlots\ draws high--quality function plots in normal or logarithmic scaling with a user-friendly interface directly in \TeX. The user supplies axis labels, legend entries and the plot coordinates for one or more plots and \PGFPlots\ applies axis scaling, computes any logarithms and axis ticks and draws the plots, supporting line plots, scatter plots, piecewise constant plots, bar plots, area plots, mesh-- and surface plots and some more. It is based on Till Tantau's package \PGF/\Tikz.
\end{abstract}
\tableofcontents
\section{Introduction}
This package provides tools to generate plots and labeled axes easily. It draws normal plots, logplots and semi-logplots, in two and three dimensions. Axis ticks, labels, legends (in case of multiple plots) can be added with key-value options. It can cycle through a set of predefined line/marker/color specifications. In summary, its purpose is to simplify the generation of high-quality function and/or data plots, and solving the problems of
\begin{itemize}
	\item consistency of document and font type and font size,
	\item direct use of \TeX\ math mode in axis descriptions,
	\item consistency of data and figures (no third party tool necessary),
	\item inter document consistency using preamble configurations and styles.
\end{itemize}
Although not necessary, separate |.pdf| or |.eps| graphics can be generated using the |external| library developed as part of \Tikz.

\PGFPlots\ is build completely on \Tikz/\PGF. Knowledge of \Tikz\ will simplify the work with \PGFPlots, although it is not required.

\section[About PGFPlots: Preliminaries]{About {\normalfont\PGFPlots}: Preliminaries}
This section contains information about upgrades, the team, the installation (in case you need to do it manually) and troubleshooting. You may skip it completely except for the upgrade remarks.

However, note that this library requires at least \PGF\ version $2.00$. At the time of this writing, many \TeX-distributions still contain the older \PGF\ version $1.18$, so it may be necessary to install a recent \PGF\ prior to using \PGFPlots.

\subsection{Components}
\PGFPlots\ comes with two components:
\begin{enumerate}
	\item the plotting component (which you are currently reading) and
	\item the \PGFPlotstable\ component which simplifies number formatting and postprocessing of numerical tables. It comes as a separate package and has its own manual \href{file:pgfplotstable.pdf}{pgfplotstable.pdf}.
\end{enumerate}

\subsection{Upgrade remarks}
This release provides a lot of improvements which can be found in all detail in \texttt{ChangeLog} for interested readers. However, some attention is useful with respect to the following changes.

\subsubsection{New Optional Features}
\begin{enumerate}
	\item \PGFPlots\ 1.3 comes with user interface improvements. The technical distinction between ``behavior options'' and ``style options'' of older versions is no longer necessary (although still fully supported).

	\item \PGFPlots\ 1.3 has a new feature which allows to \emph{move axis labels tight to tick labels} automatically.

	Since this affects the spacing, it is not enabled be default.

	Use
\begin{codeexample}[code only]
\usepackage{pgfplots}
\pgfplotsset{compat=1.3}
\end{codeexample}
	in your preamble to benefit from the improved spacing\footnote{The |compat=1.3| setting is necessary for new features which move axis labels around like reversed axis. However, \PGFPlots\ will still work as it does in earlier versions, your documents will remain the same if you don't set it explicitly.}.
	Take a look at the next page for the precise definition of |compat|.

	\item \PGFPlots\ 1.3 now supports reversed axes. It is no longer necessary to use work--arounds with negative units.
\pgfkeys{/pdflinks/search key prefixes in/.add={/pgfplots/,}{}}

	Take a look at the |x dir=reverse| key.

	Existing work--arounds will still function properly. Use |\pgfplotsset{compat=1.3}| together with |x dir=reverse| to switch to the new version.
\end{enumerate}

\subsubsection{Old Features Which May Need Attention}
\begin{enumerate}
	\item The |scatter/classes| feature produces proper legends as of version 1.3. This may change the appearance of existing legends of plots with |scatter/classes|.

	\item Due to a small math bug in \PGF\ $2.00$, you \emph{can't} use the math expression `|-x^2|'. It is necessary to use `|0-x^2|' instead. The same holds for
		`|exp(-x^2)|'; use `|exp(0-x^2)|' instead.

		This will be fixed with \PGF\ $>2.00$.

	\item Starting with \PGFPlots\ $1.1$, |\tikzstyle| should \emph{no longer be used} to set \PGFPlots\ options.
	
	Although |\tikzstyle| is still supported for some older \PGFPlots\ options, you should replace any occurance of |\tikzstyle| with |\pgfplotsset{|\meta{style name}|/.style={|\meta{key-value-list}|}}| or the associated |/.append style| variant. See section~\ref{sec:styles} for more detail.
\end{enumerate}
I apologize for any inconvenience caused by these changes.

\begin{pgfplotskey}{compat=\mchoice{1.3,pre 1.3,default,newest} (initially default)}
	The preamble configuration 
\begin{codeexample}[code only]
\usepackage{pgfplots}
\pgfplotsset{compat=1.3}
\end{codeexample}
	allows to choose between backwards compatibility and most recent features.

	\PGFPlots\ version 1.3 comes with new features which may lead to different spacings compared to earlier versions:
	\begin{enumerate}
		\item Version 1.3 comes with the |xlabel near ticks| feature which places (in this case $x$) axis labels ``near'' the tick labels, taking the size of tick labels into account.
		
		Older versions used |xlabel absolute|, i.e.\ an absolute distance between the axis and axis labels (now matter how large or small tick labels are).

		The initial setting \emph{keeps backwards compatibility}.

		You are encouraged to use
\begin{codeexample}[code only]
\usepackage{pgfplots}
\pgfplotsset{compat=1.3}
\end{codeexample}
		in your preamble.
		
		\item Version 1.3 fixes a bug with |anchor=south west| which occurred mainly for reversed axes: the anchors where upside--down. This is fixed now.

		
		If you have written some work--around and want to keep it going, use

		|\pgfplotsset{compat/anchors=pre 1.3}|\pgfmanualpdflabel{/pgfplots/compat/anchors}{} 

		to disable the bug fix.
	\end{enumerate}

	Use |\pgfplotsset{compat=default}| to restore the factory settings.

	The setting |\pgfplotsset{compat=newest}| will always use features of the most recent version. This might result in small changes in the document's appearance.
\end{pgfplotskey}

\subsection{The Team}
\PGFPlots\ has been written mainly by Christian Feuers�nger with many improvements of Pascal Wolkotte and Nick Papior Andersen as a spare time project. We hope it is useful and provides valuable plots.

If you are interested in writing something but don't know how, consider reading the auxiliary manual \href{file:TeX-programming-notes.pdf}{TeX-programming-notes.pdf} which comes with \PGFPlots. It is far from complete, but maybe it is a good starting point (at least for more literature).

\subsection{Acknowledgements}
I thank God for all hours of enjoyed programming. I thank Pascal Wolkotte and Nick Papior Andersen for their programming efforts and contributions as part of the development team. I thank Stefan Tibus, who contributed the |plot shell| feature. I thank Tom Cashman for the contribution of the |reverse legend| feature. Special thanks go to Stefan Pinnow whose tests of \PGFPlots\ lead to numerous quality improvements. Furthermore, I thank Dr.~Schweitzer for many fruitful discussions and Dr.~Meine for his ideas and suggestions. Thanks as well to the many international contributors who provided feature requests or identified bugs or simply manual improvements!

Last but not least, I thank Till Tantau and Mark Wibrow for their excellent graphics (and more) package \PGF\ and \Tikz\ which is the base of \PGFPlots.

\include{pgfplots.install}%
\include{pgfplots.intro}%
\include{pgfplots.reference}%
\include{pgfplots.libs}%
\include{pgfplots.resources}%
\include{pgfplots.importexport}%
\include{pgfplots.basic.reference}%

\printindex

\bibliographystyle{abbrv} %gerapali} %gerabbrv} %gerunsrt.bst} %gerabbrv}% gerplain}
\nocite{pgfplotstable}
\nocite{programmingnotes}
\bibliography{pgfplots}
\end{document}
%
%%%%%%%%%%%%%%%%%%%%%%%%%%%%%%%%%%%%%%%%%%%%%%%%%%%%%%%%%%%%%%%%%%%%%%%%%%%%%
%
% Package pgfplots.sty documentation. 
%
% Copyright 2007/2008 by Christian Feuersaenger.
%
% This program is free software: you can redistribute it and/or modify
% it under the terms of the GNU General Public License as published by
% the Free Software Foundation, either version 3 of the License, or
% (at your option) any later version.
% 
% This program is distributed in the hope that it will be useful,
% but WITHOUT ANY WARRANTY; without even the implied warranty of
% MERCHANTABILITY or FITNESS FOR A PARTICULAR PURPOSE.  See the
% GNU General Public License for more details.
% 
% You should have received a copy of the GNU General Public License
% along with this program.  If not, see <http://www.gnu.org/licenses/>.
%
%
%%%%%%%%%%%%%%%%%%%%%%%%%%%%%%%%%%%%%%%%%%%%%%%%%%%%%%%%%%%%%%%%%%%%%%%%%%%%%
\input{pgfplots.preamble.tex}
%\RequirePackage[german,english,francais]{babel}

\pgfplotsmanualexternalexpensivetrue

%\usetikzlibrary{external}
%\tikzexternalize[prefix=figures/]{pgfplots}

\title{%
	Manual for Package \PGFPlots\\
	{\small Version \pgfplotsversion}\\
	{\small\href{http://sourceforge.net/projects/pgfplots}{http://sourceforge.net/projects/pgfplots}}}

%\includeonly{pgfplots.libs}


\begin{document}

\def\plotcoords{%
\addplot coordinates {
(5,8.312e-02)    (17,2.547e-02)   (49,7.407e-03)
(129,2.102e-03)  (321,5.874e-04)  (769,1.623e-04)
(1793,4.442e-05) (4097,1.207e-05) (9217,3.261e-06)
};

\addplot coordinates{
(7,8.472e-02)    (31,3.044e-02)    (111,1.022e-02)
(351,3.303e-03)  (1023,1.039e-03)  (2815,3.196e-04)
(7423,9.658e-05) (18943,2.873e-05) (47103,8.437e-06)
};

\addplot coordinates{
(9,7.881e-02)     (49,3.243e-02)    (209,1.232e-02)
(769,4.454e-03)   (2561,1.551e-03)  (7937,5.236e-04)
(23297,1.723e-04) (65537,5.545e-05) (178177,1.751e-05)
};

\addplot coordinates{
(11,6.887e-02)    (71,3.177e-02)     (351,1.341e-02)
(1471,5.334e-03)  (5503,2.027e-03)   (18943,7.415e-04)
(61183,2.628e-04) (187903,9.063e-05) (553983,3.053e-05)
};

\addplot coordinates{
(13,5.755e-02)     (97,2.925e-02)     (545,1.351e-02)
(2561,5.842e-03)   (10625,2.397e-03)  (40193,9.414e-04)
(141569,3.564e-04) (471041,1.308e-04) 
(1496065,4.670e-05)
};
}%
\maketitle
\begin{abstract}%
\PGFPlots\ draws high--quality function plots in normal or logarithmic scaling with a user-friendly interface directly in \TeX. The user supplies axis labels, legend entries and the plot coordinates for one or more plots and \PGFPlots\ applies axis scaling, computes any logarithms and axis ticks and draws the plots, supporting line plots, scatter plots, piecewise constant plots, bar plots, area plots, mesh-- and surface plots and some more. It is based on Till Tantau's package \PGF/\Tikz.
\end{abstract}
\tableofcontents
\section{Introduction}
This package provides tools to generate plots and labeled axes easily. It draws normal plots, logplots and semi-logplots, in two and three dimensions. Axis ticks, labels, legends (in case of multiple plots) can be added with key-value options. It can cycle through a set of predefined line/marker/color specifications. In summary, its purpose is to simplify the generation of high-quality function and/or data plots, and solving the problems of
\begin{itemize}
	\item consistency of document and font type and font size,
	\item direct use of \TeX\ math mode in axis descriptions,
	\item consistency of data and figures (no third party tool necessary),
	\item inter document consistency using preamble configurations and styles.
\end{itemize}
Although not necessary, separate |.pdf| or |.eps| graphics can be generated using the |external| library developed as part of \Tikz.

\PGFPlots\ is build completely on \Tikz/\PGF. Knowledge of \Tikz\ will simplify the work with \PGFPlots, although it is not required.

\section[About PGFPlots: Preliminaries]{About {\normalfont\PGFPlots}: Preliminaries}
This section contains information about upgrades, the team, the installation (in case you need to do it manually) and troubleshooting. You may skip it completely except for the upgrade remarks.

However, note that this library requires at least \PGF\ version $2.00$. At the time of this writing, many \TeX-distributions still contain the older \PGF\ version $1.18$, so it may be necessary to install a recent \PGF\ prior to using \PGFPlots.

\subsection{Components}
\PGFPlots\ comes with two components:
\begin{enumerate}
	\item the plotting component (which you are currently reading) and
	\item the \PGFPlotstable\ component which simplifies number formatting and postprocessing of numerical tables. It comes as a separate package and has its own manual \href{file:pgfplotstable.pdf}{pgfplotstable.pdf}.
\end{enumerate}

\subsection{Upgrade remarks}
This release provides a lot of improvements which can be found in all detail in \texttt{ChangeLog} for interested readers. However, some attention is useful with respect to the following changes.

\subsubsection{New Optional Features}
\begin{enumerate}
	\item \PGFPlots\ 1.3 comes with user interface improvements. The technical distinction between ``behavior options'' and ``style options'' of older versions is no longer necessary (although still fully supported).

	\item \PGFPlots\ 1.3 has a new feature which allows to \emph{move axis labels tight to tick labels} automatically.

	Since this affects the spacing, it is not enabled be default.

	Use
\begin{codeexample}[code only]
\usepackage{pgfplots}
\pgfplotsset{compat=1.3}
\end{codeexample}
	in your preamble to benefit from the improved spacing\footnote{The |compat=1.3| setting is necessary for new features which move axis labels around like reversed axis. However, \PGFPlots\ will still work as it does in earlier versions, your documents will remain the same if you don't set it explicitly.}.
	Take a look at the next page for the precise definition of |compat|.

	\item \PGFPlots\ 1.3 now supports reversed axes. It is no longer necessary to use work--arounds with negative units.
\pgfkeys{/pdflinks/search key prefixes in/.add={/pgfplots/,}{}}

	Take a look at the |x dir=reverse| key.

	Existing work--arounds will still function properly. Use |\pgfplotsset{compat=1.3}| together with |x dir=reverse| to switch to the new version.
\end{enumerate}

\subsubsection{Old Features Which May Need Attention}
\begin{enumerate}
	\item The |scatter/classes| feature produces proper legends as of version 1.3. This may change the appearance of existing legends of plots with |scatter/classes|.

	\item Due to a small math bug in \PGF\ $2.00$, you \emph{can't} use the math expression `|-x^2|'. It is necessary to use `|0-x^2|' instead. The same holds for
		`|exp(-x^2)|'; use `|exp(0-x^2)|' instead.

		This will be fixed with \PGF\ $>2.00$.

	\item Starting with \PGFPlots\ $1.1$, |\tikzstyle| should \emph{no longer be used} to set \PGFPlots\ options.
	
	Although |\tikzstyle| is still supported for some older \PGFPlots\ options, you should replace any occurance of |\tikzstyle| with |\pgfplotsset{|\meta{style name}|/.style={|\meta{key-value-list}|}}| or the associated |/.append style| variant. See section~\ref{sec:styles} for more detail.
\end{enumerate}
I apologize for any inconvenience caused by these changes.

\begin{pgfplotskey}{compat=\mchoice{1.3,pre 1.3,default,newest} (initially default)}
	The preamble configuration 
\begin{codeexample}[code only]
\usepackage{pgfplots}
\pgfplotsset{compat=1.3}
\end{codeexample}
	allows to choose between backwards compatibility and most recent features.

	\PGFPlots\ version 1.3 comes with new features which may lead to different spacings compared to earlier versions:
	\begin{enumerate}
		\item Version 1.3 comes with the |xlabel near ticks| feature which places (in this case $x$) axis labels ``near'' the tick labels, taking the size of tick labels into account.
		
		Older versions used |xlabel absolute|, i.e.\ an absolute distance between the axis and axis labels (now matter how large or small tick labels are).

		The initial setting \emph{keeps backwards compatibility}.

		You are encouraged to use
\begin{codeexample}[code only]
\usepackage{pgfplots}
\pgfplotsset{compat=1.3}
\end{codeexample}
		in your preamble.
		
		\item Version 1.3 fixes a bug with |anchor=south west| which occurred mainly for reversed axes: the anchors where upside--down. This is fixed now.

		
		If you have written some work--around and want to keep it going, use

		|\pgfplotsset{compat/anchors=pre 1.3}|\pgfmanualpdflabel{/pgfplots/compat/anchors}{} 

		to disable the bug fix.
	\end{enumerate}

	Use |\pgfplotsset{compat=default}| to restore the factory settings.

	The setting |\pgfplotsset{compat=newest}| will always use features of the most recent version. This might result in small changes in the document's appearance.
\end{pgfplotskey}

\subsection{The Team}
\PGFPlots\ has been written mainly by Christian Feuers�nger with many improvements of Pascal Wolkotte and Nick Papior Andersen as a spare time project. We hope it is useful and provides valuable plots.

If you are interested in writing something but don't know how, consider reading the auxiliary manual \href{file:TeX-programming-notes.pdf}{TeX-programming-notes.pdf} which comes with \PGFPlots. It is far from complete, but maybe it is a good starting point (at least for more literature).

\subsection{Acknowledgements}
I thank God for all hours of enjoyed programming. I thank Pascal Wolkotte and Nick Papior Andersen for their programming efforts and contributions as part of the development team. I thank Stefan Tibus, who contributed the |plot shell| feature. I thank Tom Cashman for the contribution of the |reverse legend| feature. Special thanks go to Stefan Pinnow whose tests of \PGFPlots\ lead to numerous quality improvements. Furthermore, I thank Dr.~Schweitzer for many fruitful discussions and Dr.~Meine for his ideas and suggestions. Thanks as well to the many international contributors who provided feature requests or identified bugs or simply manual improvements!

Last but not least, I thank Till Tantau and Mark Wibrow for their excellent graphics (and more) package \PGF\ and \Tikz\ which is the base of \PGFPlots.

\include{pgfplots.install}%
\include{pgfplots.intro}%
\include{pgfplots.reference}%
\include{pgfplots.libs}%
\include{pgfplots.resources}%
\include{pgfplots.importexport}%
\include{pgfplots.basic.reference}%

\printindex

\bibliographystyle{abbrv} %gerapali} %gerabbrv} %gerunsrt.bst} %gerabbrv}% gerplain}
\nocite{pgfplotstable}
\nocite{programmingnotes}
\bibliography{pgfplots}
\end{document}
%
%%%%%%%%%%%%%%%%%%%%%%%%%%%%%%%%%%%%%%%%%%%%%%%%%%%%%%%%%%%%%%%%%%%%%%%%%%%%%
%
% Package pgfplots.sty documentation. 
%
% Copyright 2007/2008 by Christian Feuersaenger.
%
% This program is free software: you can redistribute it and/or modify
% it under the terms of the GNU General Public License as published by
% the Free Software Foundation, either version 3 of the License, or
% (at your option) any later version.
% 
% This program is distributed in the hope that it will be useful,
% but WITHOUT ANY WARRANTY; without even the implied warranty of
% MERCHANTABILITY or FITNESS FOR A PARTICULAR PURPOSE.  See the
% GNU General Public License for more details.
% 
% You should have received a copy of the GNU General Public License
% along with this program.  If not, see <http://www.gnu.org/licenses/>.
%
%
%%%%%%%%%%%%%%%%%%%%%%%%%%%%%%%%%%%%%%%%%%%%%%%%%%%%%%%%%%%%%%%%%%%%%%%%%%%%%
\input{pgfplots.preamble.tex}
%\RequirePackage[german,english,francais]{babel}

\pgfplotsmanualexternalexpensivetrue

%\usetikzlibrary{external}
%\tikzexternalize[prefix=figures/]{pgfplots}

\title{%
	Manual for Package \PGFPlots\\
	{\small Version \pgfplotsversion}\\
	{\small\href{http://sourceforge.net/projects/pgfplots}{http://sourceforge.net/projects/pgfplots}}}

%\includeonly{pgfplots.libs}


\begin{document}

\def\plotcoords{%
\addplot coordinates {
(5,8.312e-02)    (17,2.547e-02)   (49,7.407e-03)
(129,2.102e-03)  (321,5.874e-04)  (769,1.623e-04)
(1793,4.442e-05) (4097,1.207e-05) (9217,3.261e-06)
};

\addplot coordinates{
(7,8.472e-02)    (31,3.044e-02)    (111,1.022e-02)
(351,3.303e-03)  (1023,1.039e-03)  (2815,3.196e-04)
(7423,9.658e-05) (18943,2.873e-05) (47103,8.437e-06)
};

\addplot coordinates{
(9,7.881e-02)     (49,3.243e-02)    (209,1.232e-02)
(769,4.454e-03)   (2561,1.551e-03)  (7937,5.236e-04)
(23297,1.723e-04) (65537,5.545e-05) (178177,1.751e-05)
};

\addplot coordinates{
(11,6.887e-02)    (71,3.177e-02)     (351,1.341e-02)
(1471,5.334e-03)  (5503,2.027e-03)   (18943,7.415e-04)
(61183,2.628e-04) (187903,9.063e-05) (553983,3.053e-05)
};

\addplot coordinates{
(13,5.755e-02)     (97,2.925e-02)     (545,1.351e-02)
(2561,5.842e-03)   (10625,2.397e-03)  (40193,9.414e-04)
(141569,3.564e-04) (471041,1.308e-04) 
(1496065,4.670e-05)
};
}%
\maketitle
\begin{abstract}%
\PGFPlots\ draws high--quality function plots in normal or logarithmic scaling with a user-friendly interface directly in \TeX. The user supplies axis labels, legend entries and the plot coordinates for one or more plots and \PGFPlots\ applies axis scaling, computes any logarithms and axis ticks and draws the plots, supporting line plots, scatter plots, piecewise constant plots, bar plots, area plots, mesh-- and surface plots and some more. It is based on Till Tantau's package \PGF/\Tikz.
\end{abstract}
\tableofcontents
\section{Introduction}
This package provides tools to generate plots and labeled axes easily. It draws normal plots, logplots and semi-logplots, in two and three dimensions. Axis ticks, labels, legends (in case of multiple plots) can be added with key-value options. It can cycle through a set of predefined line/marker/color specifications. In summary, its purpose is to simplify the generation of high-quality function and/or data plots, and solving the problems of
\begin{itemize}
	\item consistency of document and font type and font size,
	\item direct use of \TeX\ math mode in axis descriptions,
	\item consistency of data and figures (no third party tool necessary),
	\item inter document consistency using preamble configurations and styles.
\end{itemize}
Although not necessary, separate |.pdf| or |.eps| graphics can be generated using the |external| library developed as part of \Tikz.

\PGFPlots\ is build completely on \Tikz/\PGF. Knowledge of \Tikz\ will simplify the work with \PGFPlots, although it is not required.

\section[About PGFPlots: Preliminaries]{About {\normalfont\PGFPlots}: Preliminaries}
This section contains information about upgrades, the team, the installation (in case you need to do it manually) and troubleshooting. You may skip it completely except for the upgrade remarks.

However, note that this library requires at least \PGF\ version $2.00$. At the time of this writing, many \TeX-distributions still contain the older \PGF\ version $1.18$, so it may be necessary to install a recent \PGF\ prior to using \PGFPlots.

\subsection{Components}
\PGFPlots\ comes with two components:
\begin{enumerate}
	\item the plotting component (which you are currently reading) and
	\item the \PGFPlotstable\ component which simplifies number formatting and postprocessing of numerical tables. It comes as a separate package and has its own manual \href{file:pgfplotstable.pdf}{pgfplotstable.pdf}.
\end{enumerate}

\subsection{Upgrade remarks}
This release provides a lot of improvements which can be found in all detail in \texttt{ChangeLog} for interested readers. However, some attention is useful with respect to the following changes.

\subsubsection{New Optional Features}
\begin{enumerate}
	\item \PGFPlots\ 1.3 comes with user interface improvements. The technical distinction between ``behavior options'' and ``style options'' of older versions is no longer necessary (although still fully supported).

	\item \PGFPlots\ 1.3 has a new feature which allows to \emph{move axis labels tight to tick labels} automatically.

	Since this affects the spacing, it is not enabled be default.

	Use
\begin{codeexample}[code only]
\usepackage{pgfplots}
\pgfplotsset{compat=1.3}
\end{codeexample}
	in your preamble to benefit from the improved spacing\footnote{The |compat=1.3| setting is necessary for new features which move axis labels around like reversed axis. However, \PGFPlots\ will still work as it does in earlier versions, your documents will remain the same if you don't set it explicitly.}.
	Take a look at the next page for the precise definition of |compat|.

	\item \PGFPlots\ 1.3 now supports reversed axes. It is no longer necessary to use work--arounds with negative units.
\pgfkeys{/pdflinks/search key prefixes in/.add={/pgfplots/,}{}}

	Take a look at the |x dir=reverse| key.

	Existing work--arounds will still function properly. Use |\pgfplotsset{compat=1.3}| together with |x dir=reverse| to switch to the new version.
\end{enumerate}

\subsubsection{Old Features Which May Need Attention}
\begin{enumerate}
	\item The |scatter/classes| feature produces proper legends as of version 1.3. This may change the appearance of existing legends of plots with |scatter/classes|.

	\item Due to a small math bug in \PGF\ $2.00$, you \emph{can't} use the math expression `|-x^2|'. It is necessary to use `|0-x^2|' instead. The same holds for
		`|exp(-x^2)|'; use `|exp(0-x^2)|' instead.

		This will be fixed with \PGF\ $>2.00$.

	\item Starting with \PGFPlots\ $1.1$, |\tikzstyle| should \emph{no longer be used} to set \PGFPlots\ options.
	
	Although |\tikzstyle| is still supported for some older \PGFPlots\ options, you should replace any occurance of |\tikzstyle| with |\pgfplotsset{|\meta{style name}|/.style={|\meta{key-value-list}|}}| or the associated |/.append style| variant. See section~\ref{sec:styles} for more detail.
\end{enumerate}
I apologize for any inconvenience caused by these changes.

\begin{pgfplotskey}{compat=\mchoice{1.3,pre 1.3,default,newest} (initially default)}
	The preamble configuration 
\begin{codeexample}[code only]
\usepackage{pgfplots}
\pgfplotsset{compat=1.3}
\end{codeexample}
	allows to choose between backwards compatibility and most recent features.

	\PGFPlots\ version 1.3 comes with new features which may lead to different spacings compared to earlier versions:
	\begin{enumerate}
		\item Version 1.3 comes with the |xlabel near ticks| feature which places (in this case $x$) axis labels ``near'' the tick labels, taking the size of tick labels into account.
		
		Older versions used |xlabel absolute|, i.e.\ an absolute distance between the axis and axis labels (now matter how large or small tick labels are).

		The initial setting \emph{keeps backwards compatibility}.

		You are encouraged to use
\begin{codeexample}[code only]
\usepackage{pgfplots}
\pgfplotsset{compat=1.3}
\end{codeexample}
		in your preamble.
		
		\item Version 1.3 fixes a bug with |anchor=south west| which occurred mainly for reversed axes: the anchors where upside--down. This is fixed now.

		
		If you have written some work--around and want to keep it going, use

		|\pgfplotsset{compat/anchors=pre 1.3}|\pgfmanualpdflabel{/pgfplots/compat/anchors}{} 

		to disable the bug fix.
	\end{enumerate}

	Use |\pgfplotsset{compat=default}| to restore the factory settings.

	The setting |\pgfplotsset{compat=newest}| will always use features of the most recent version. This might result in small changes in the document's appearance.
\end{pgfplotskey}

\subsection{The Team}
\PGFPlots\ has been written mainly by Christian Feuers�nger with many improvements of Pascal Wolkotte and Nick Papior Andersen as a spare time project. We hope it is useful and provides valuable plots.

If you are interested in writing something but don't know how, consider reading the auxiliary manual \href{file:TeX-programming-notes.pdf}{TeX-programming-notes.pdf} which comes with \PGFPlots. It is far from complete, but maybe it is a good starting point (at least for more literature).

\subsection{Acknowledgements}
I thank God for all hours of enjoyed programming. I thank Pascal Wolkotte and Nick Papior Andersen for their programming efforts and contributions as part of the development team. I thank Stefan Tibus, who contributed the |plot shell| feature. I thank Tom Cashman for the contribution of the |reverse legend| feature. Special thanks go to Stefan Pinnow whose tests of \PGFPlots\ lead to numerous quality improvements. Furthermore, I thank Dr.~Schweitzer for many fruitful discussions and Dr.~Meine for his ideas and suggestions. Thanks as well to the many international contributors who provided feature requests or identified bugs or simply manual improvements!

Last but not least, I thank Till Tantau and Mark Wibrow for their excellent graphics (and more) package \PGF\ and \Tikz\ which is the base of \PGFPlots.

\include{pgfplots.install}%
\include{pgfplots.intro}%
\include{pgfplots.reference}%
\include{pgfplots.libs}%
\include{pgfplots.resources}%
\include{pgfplots.importexport}%
\include{pgfplots.basic.reference}%

\printindex

\bibliographystyle{abbrv} %gerapali} %gerabbrv} %gerunsrt.bst} %gerabbrv}% gerplain}
\nocite{pgfplotstable}
\nocite{programmingnotes}
\bibliography{pgfplots}
\end{document}
%
%%%%%%%%%%%%%%%%%%%%%%%%%%%%%%%%%%%%%%%%%%%%%%%%%%%%%%%%%%%%%%%%%%%%%%%%%%%%%
%
% Package pgfplots.sty documentation. 
%
% Copyright 2007/2008 by Christian Feuersaenger.
%
% This program is free software: you can redistribute it and/or modify
% it under the terms of the GNU General Public License as published by
% the Free Software Foundation, either version 3 of the License, or
% (at your option) any later version.
% 
% This program is distributed in the hope that it will be useful,
% but WITHOUT ANY WARRANTY; without even the implied warranty of
% MERCHANTABILITY or FITNESS FOR A PARTICULAR PURPOSE.  See the
% GNU General Public License for more details.
% 
% You should have received a copy of the GNU General Public License
% along with this program.  If not, see <http://www.gnu.org/licenses/>.
%
%
%%%%%%%%%%%%%%%%%%%%%%%%%%%%%%%%%%%%%%%%%%%%%%%%%%%%%%%%%%%%%%%%%%%%%%%%%%%%%
\input{pgfplots.preamble.tex}
%\RequirePackage[german,english,francais]{babel}

\pgfplotsmanualexternalexpensivetrue

%\usetikzlibrary{external}
%\tikzexternalize[prefix=figures/]{pgfplots}

\title{%
	Manual for Package \PGFPlots\\
	{\small Version \pgfplotsversion}\\
	{\small\href{http://sourceforge.net/projects/pgfplots}{http://sourceforge.net/projects/pgfplots}}}

%\includeonly{pgfplots.libs}


\begin{document}

\def\plotcoords{%
\addplot coordinates {
(5,8.312e-02)    (17,2.547e-02)   (49,7.407e-03)
(129,2.102e-03)  (321,5.874e-04)  (769,1.623e-04)
(1793,4.442e-05) (4097,1.207e-05) (9217,3.261e-06)
};

\addplot coordinates{
(7,8.472e-02)    (31,3.044e-02)    (111,1.022e-02)
(351,3.303e-03)  (1023,1.039e-03)  (2815,3.196e-04)
(7423,9.658e-05) (18943,2.873e-05) (47103,8.437e-06)
};

\addplot coordinates{
(9,7.881e-02)     (49,3.243e-02)    (209,1.232e-02)
(769,4.454e-03)   (2561,1.551e-03)  (7937,5.236e-04)
(23297,1.723e-04) (65537,5.545e-05) (178177,1.751e-05)
};

\addplot coordinates{
(11,6.887e-02)    (71,3.177e-02)     (351,1.341e-02)
(1471,5.334e-03)  (5503,2.027e-03)   (18943,7.415e-04)
(61183,2.628e-04) (187903,9.063e-05) (553983,3.053e-05)
};

\addplot coordinates{
(13,5.755e-02)     (97,2.925e-02)     (545,1.351e-02)
(2561,5.842e-03)   (10625,2.397e-03)  (40193,9.414e-04)
(141569,3.564e-04) (471041,1.308e-04) 
(1496065,4.670e-05)
};
}%
\maketitle
\begin{abstract}%
\PGFPlots\ draws high--quality function plots in normal or logarithmic scaling with a user-friendly interface directly in \TeX. The user supplies axis labels, legend entries and the plot coordinates for one or more plots and \PGFPlots\ applies axis scaling, computes any logarithms and axis ticks and draws the plots, supporting line plots, scatter plots, piecewise constant plots, bar plots, area plots, mesh-- and surface plots and some more. It is based on Till Tantau's package \PGF/\Tikz.
\end{abstract}
\tableofcontents
\section{Introduction}
This package provides tools to generate plots and labeled axes easily. It draws normal plots, logplots and semi-logplots, in two and three dimensions. Axis ticks, labels, legends (in case of multiple plots) can be added with key-value options. It can cycle through a set of predefined line/marker/color specifications. In summary, its purpose is to simplify the generation of high-quality function and/or data plots, and solving the problems of
\begin{itemize}
	\item consistency of document and font type and font size,
	\item direct use of \TeX\ math mode in axis descriptions,
	\item consistency of data and figures (no third party tool necessary),
	\item inter document consistency using preamble configurations and styles.
\end{itemize}
Although not necessary, separate |.pdf| or |.eps| graphics can be generated using the |external| library developed as part of \Tikz.

\PGFPlots\ is build completely on \Tikz/\PGF. Knowledge of \Tikz\ will simplify the work with \PGFPlots, although it is not required.

\section[About PGFPlots: Preliminaries]{About {\normalfont\PGFPlots}: Preliminaries}
This section contains information about upgrades, the team, the installation (in case you need to do it manually) and troubleshooting. You may skip it completely except for the upgrade remarks.

However, note that this library requires at least \PGF\ version $2.00$. At the time of this writing, many \TeX-distributions still contain the older \PGF\ version $1.18$, so it may be necessary to install a recent \PGF\ prior to using \PGFPlots.

\subsection{Components}
\PGFPlots\ comes with two components:
\begin{enumerate}
	\item the plotting component (which you are currently reading) and
	\item the \PGFPlotstable\ component which simplifies number formatting and postprocessing of numerical tables. It comes as a separate package and has its own manual \href{file:pgfplotstable.pdf}{pgfplotstable.pdf}.
\end{enumerate}

\subsection{Upgrade remarks}
This release provides a lot of improvements which can be found in all detail in \texttt{ChangeLog} for interested readers. However, some attention is useful with respect to the following changes.

\subsubsection{New Optional Features}
\begin{enumerate}
	\item \PGFPlots\ 1.3 comes with user interface improvements. The technical distinction between ``behavior options'' and ``style options'' of older versions is no longer necessary (although still fully supported).

	\item \PGFPlots\ 1.3 has a new feature which allows to \emph{move axis labels tight to tick labels} automatically.

	Since this affects the spacing, it is not enabled be default.

	Use
\begin{codeexample}[code only]
\usepackage{pgfplots}
\pgfplotsset{compat=1.3}
\end{codeexample}
	in your preamble to benefit from the improved spacing\footnote{The |compat=1.3| setting is necessary for new features which move axis labels around like reversed axis. However, \PGFPlots\ will still work as it does in earlier versions, your documents will remain the same if you don't set it explicitly.}.
	Take a look at the next page for the precise definition of |compat|.

	\item \PGFPlots\ 1.3 now supports reversed axes. It is no longer necessary to use work--arounds with negative units.
\pgfkeys{/pdflinks/search key prefixes in/.add={/pgfplots/,}{}}

	Take a look at the |x dir=reverse| key.

	Existing work--arounds will still function properly. Use |\pgfplotsset{compat=1.3}| together with |x dir=reverse| to switch to the new version.
\end{enumerate}

\subsubsection{Old Features Which May Need Attention}
\begin{enumerate}
	\item The |scatter/classes| feature produces proper legends as of version 1.3. This may change the appearance of existing legends of plots with |scatter/classes|.

	\item Due to a small math bug in \PGF\ $2.00$, you \emph{can't} use the math expression `|-x^2|'. It is necessary to use `|0-x^2|' instead. The same holds for
		`|exp(-x^2)|'; use `|exp(0-x^2)|' instead.

		This will be fixed with \PGF\ $>2.00$.

	\item Starting with \PGFPlots\ $1.1$, |\tikzstyle| should \emph{no longer be used} to set \PGFPlots\ options.
	
	Although |\tikzstyle| is still supported for some older \PGFPlots\ options, you should replace any occurance of |\tikzstyle| with |\pgfplotsset{|\meta{style name}|/.style={|\meta{key-value-list}|}}| or the associated |/.append style| variant. See section~\ref{sec:styles} for more detail.
\end{enumerate}
I apologize for any inconvenience caused by these changes.

\begin{pgfplotskey}{compat=\mchoice{1.3,pre 1.3,default,newest} (initially default)}
	The preamble configuration 
\begin{codeexample}[code only]
\usepackage{pgfplots}
\pgfplotsset{compat=1.3}
\end{codeexample}
	allows to choose between backwards compatibility and most recent features.

	\PGFPlots\ version 1.3 comes with new features which may lead to different spacings compared to earlier versions:
	\begin{enumerate}
		\item Version 1.3 comes with the |xlabel near ticks| feature which places (in this case $x$) axis labels ``near'' the tick labels, taking the size of tick labels into account.
		
		Older versions used |xlabel absolute|, i.e.\ an absolute distance between the axis and axis labels (now matter how large or small tick labels are).

		The initial setting \emph{keeps backwards compatibility}.

		You are encouraged to use
\begin{codeexample}[code only]
\usepackage{pgfplots}
\pgfplotsset{compat=1.3}
\end{codeexample}
		in your preamble.
		
		\item Version 1.3 fixes a bug with |anchor=south west| which occurred mainly for reversed axes: the anchors where upside--down. This is fixed now.

		
		If you have written some work--around and want to keep it going, use

		|\pgfplotsset{compat/anchors=pre 1.3}|\pgfmanualpdflabel{/pgfplots/compat/anchors}{} 

		to disable the bug fix.
	\end{enumerate}

	Use |\pgfplotsset{compat=default}| to restore the factory settings.

	The setting |\pgfplotsset{compat=newest}| will always use features of the most recent version. This might result in small changes in the document's appearance.
\end{pgfplotskey}

\subsection{The Team}
\PGFPlots\ has been written mainly by Christian Feuers�nger with many improvements of Pascal Wolkotte and Nick Papior Andersen as a spare time project. We hope it is useful and provides valuable plots.

If you are interested in writing something but don't know how, consider reading the auxiliary manual \href{file:TeX-programming-notes.pdf}{TeX-programming-notes.pdf} which comes with \PGFPlots. It is far from complete, but maybe it is a good starting point (at least for more literature).

\subsection{Acknowledgements}
I thank God for all hours of enjoyed programming. I thank Pascal Wolkotte and Nick Papior Andersen for their programming efforts and contributions as part of the development team. I thank Stefan Tibus, who contributed the |plot shell| feature. I thank Tom Cashman for the contribution of the |reverse legend| feature. Special thanks go to Stefan Pinnow whose tests of \PGFPlots\ lead to numerous quality improvements. Furthermore, I thank Dr.~Schweitzer for many fruitful discussions and Dr.~Meine for his ideas and suggestions. Thanks as well to the many international contributors who provided feature requests or identified bugs or simply manual improvements!

Last but not least, I thank Till Tantau and Mark Wibrow for their excellent graphics (and more) package \PGF\ and \Tikz\ which is the base of \PGFPlots.

\include{pgfplots.install}%
\include{pgfplots.intro}%
\include{pgfplots.reference}%
\include{pgfplots.libs}%
\include{pgfplots.resources}%
\include{pgfplots.importexport}%
\include{pgfplots.basic.reference}%

\printindex

\bibliographystyle{abbrv} %gerapali} %gerabbrv} %gerunsrt.bst} %gerabbrv}% gerplain}
\nocite{pgfplotstable}
\nocite{programmingnotes}
\bibliography{pgfplots}
\end{document}
%
%%%%%%%%%%%%%%%%%%%%%%%%%%%%%%%%%%%%%%%%%%%%%%%%%%%%%%%%%%%%%%%%%%%%%%%%%%%%%
%
% Package pgfplots.sty documentation. 
%
% Copyright 2007/2008 by Christian Feuersaenger.
%
% This program is free software: you can redistribute it and/or modify
% it under the terms of the GNU General Public License as published by
% the Free Software Foundation, either version 3 of the License, or
% (at your option) any later version.
% 
% This program is distributed in the hope that it will be useful,
% but WITHOUT ANY WARRANTY; without even the implied warranty of
% MERCHANTABILITY or FITNESS FOR A PARTICULAR PURPOSE.  See the
% GNU General Public License for more details.
% 
% You should have received a copy of the GNU General Public License
% along with this program.  If not, see <http://www.gnu.org/licenses/>.
%
%
%%%%%%%%%%%%%%%%%%%%%%%%%%%%%%%%%%%%%%%%%%%%%%%%%%%%%%%%%%%%%%%%%%%%%%%%%%%%%
\input{pgfplots.preamble.tex}
%\RequirePackage[german,english,francais]{babel}

\pgfplotsmanualexternalexpensivetrue

%\usetikzlibrary{external}
%\tikzexternalize[prefix=figures/]{pgfplots}

\title{%
	Manual for Package \PGFPlots\\
	{\small Version \pgfplotsversion}\\
	{\small\href{http://sourceforge.net/projects/pgfplots}{http://sourceforge.net/projects/pgfplots}}}

%\includeonly{pgfplots.libs}


\begin{document}

\def\plotcoords{%
\addplot coordinates {
(5,8.312e-02)    (17,2.547e-02)   (49,7.407e-03)
(129,2.102e-03)  (321,5.874e-04)  (769,1.623e-04)
(1793,4.442e-05) (4097,1.207e-05) (9217,3.261e-06)
};

\addplot coordinates{
(7,8.472e-02)    (31,3.044e-02)    (111,1.022e-02)
(351,3.303e-03)  (1023,1.039e-03)  (2815,3.196e-04)
(7423,9.658e-05) (18943,2.873e-05) (47103,8.437e-06)
};

\addplot coordinates{
(9,7.881e-02)     (49,3.243e-02)    (209,1.232e-02)
(769,4.454e-03)   (2561,1.551e-03)  (7937,5.236e-04)
(23297,1.723e-04) (65537,5.545e-05) (178177,1.751e-05)
};

\addplot coordinates{
(11,6.887e-02)    (71,3.177e-02)     (351,1.341e-02)
(1471,5.334e-03)  (5503,2.027e-03)   (18943,7.415e-04)
(61183,2.628e-04) (187903,9.063e-05) (553983,3.053e-05)
};

\addplot coordinates{
(13,5.755e-02)     (97,2.925e-02)     (545,1.351e-02)
(2561,5.842e-03)   (10625,2.397e-03)  (40193,9.414e-04)
(141569,3.564e-04) (471041,1.308e-04) 
(1496065,4.670e-05)
};
}%
\maketitle
\begin{abstract}%
\PGFPlots\ draws high--quality function plots in normal or logarithmic scaling with a user-friendly interface directly in \TeX. The user supplies axis labels, legend entries and the plot coordinates for one or more plots and \PGFPlots\ applies axis scaling, computes any logarithms and axis ticks and draws the plots, supporting line plots, scatter plots, piecewise constant plots, bar plots, area plots, mesh-- and surface plots and some more. It is based on Till Tantau's package \PGF/\Tikz.
\end{abstract}
\tableofcontents
\section{Introduction}
This package provides tools to generate plots and labeled axes easily. It draws normal plots, logplots and semi-logplots, in two and three dimensions. Axis ticks, labels, legends (in case of multiple plots) can be added with key-value options. It can cycle through a set of predefined line/marker/color specifications. In summary, its purpose is to simplify the generation of high-quality function and/or data plots, and solving the problems of
\begin{itemize}
	\item consistency of document and font type and font size,
	\item direct use of \TeX\ math mode in axis descriptions,
	\item consistency of data and figures (no third party tool necessary),
	\item inter document consistency using preamble configurations and styles.
\end{itemize}
Although not necessary, separate |.pdf| or |.eps| graphics can be generated using the |external| library developed as part of \Tikz.

\PGFPlots\ is build completely on \Tikz/\PGF. Knowledge of \Tikz\ will simplify the work with \PGFPlots, although it is not required.

\section[About PGFPlots: Preliminaries]{About {\normalfont\PGFPlots}: Preliminaries}
This section contains information about upgrades, the team, the installation (in case you need to do it manually) and troubleshooting. You may skip it completely except for the upgrade remarks.

However, note that this library requires at least \PGF\ version $2.00$. At the time of this writing, many \TeX-distributions still contain the older \PGF\ version $1.18$, so it may be necessary to install a recent \PGF\ prior to using \PGFPlots.

\subsection{Components}
\PGFPlots\ comes with two components:
\begin{enumerate}
	\item the plotting component (which you are currently reading) and
	\item the \PGFPlotstable\ component which simplifies number formatting and postprocessing of numerical tables. It comes as a separate package and has its own manual \href{file:pgfplotstable.pdf}{pgfplotstable.pdf}.
\end{enumerate}

\subsection{Upgrade remarks}
This release provides a lot of improvements which can be found in all detail in \texttt{ChangeLog} for interested readers. However, some attention is useful with respect to the following changes.

\subsubsection{New Optional Features}
\begin{enumerate}
	\item \PGFPlots\ 1.3 comes with user interface improvements. The technical distinction between ``behavior options'' and ``style options'' of older versions is no longer necessary (although still fully supported).

	\item \PGFPlots\ 1.3 has a new feature which allows to \emph{move axis labels tight to tick labels} automatically.

	Since this affects the spacing, it is not enabled be default.

	Use
\begin{codeexample}[code only]
\usepackage{pgfplots}
\pgfplotsset{compat=1.3}
\end{codeexample}
	in your preamble to benefit from the improved spacing\footnote{The |compat=1.3| setting is necessary for new features which move axis labels around like reversed axis. However, \PGFPlots\ will still work as it does in earlier versions, your documents will remain the same if you don't set it explicitly.}.
	Take a look at the next page for the precise definition of |compat|.

	\item \PGFPlots\ 1.3 now supports reversed axes. It is no longer necessary to use work--arounds with negative units.
\pgfkeys{/pdflinks/search key prefixes in/.add={/pgfplots/,}{}}

	Take a look at the |x dir=reverse| key.

	Existing work--arounds will still function properly. Use |\pgfplotsset{compat=1.3}| together with |x dir=reverse| to switch to the new version.
\end{enumerate}

\subsubsection{Old Features Which May Need Attention}
\begin{enumerate}
	\item The |scatter/classes| feature produces proper legends as of version 1.3. This may change the appearance of existing legends of plots with |scatter/classes|.

	\item Due to a small math bug in \PGF\ $2.00$, you \emph{can't} use the math expression `|-x^2|'. It is necessary to use `|0-x^2|' instead. The same holds for
		`|exp(-x^2)|'; use `|exp(0-x^2)|' instead.

		This will be fixed with \PGF\ $>2.00$.

	\item Starting with \PGFPlots\ $1.1$, |\tikzstyle| should \emph{no longer be used} to set \PGFPlots\ options.
	
	Although |\tikzstyle| is still supported for some older \PGFPlots\ options, you should replace any occurance of |\tikzstyle| with |\pgfplotsset{|\meta{style name}|/.style={|\meta{key-value-list}|}}| or the associated |/.append style| variant. See section~\ref{sec:styles} for more detail.
\end{enumerate}
I apologize for any inconvenience caused by these changes.

\begin{pgfplotskey}{compat=\mchoice{1.3,pre 1.3,default,newest} (initially default)}
	The preamble configuration 
\begin{codeexample}[code only]
\usepackage{pgfplots}
\pgfplotsset{compat=1.3}
\end{codeexample}
	allows to choose between backwards compatibility and most recent features.

	\PGFPlots\ version 1.3 comes with new features which may lead to different spacings compared to earlier versions:
	\begin{enumerate}
		\item Version 1.3 comes with the |xlabel near ticks| feature which places (in this case $x$) axis labels ``near'' the tick labels, taking the size of tick labels into account.
		
		Older versions used |xlabel absolute|, i.e.\ an absolute distance between the axis and axis labels (now matter how large or small tick labels are).

		The initial setting \emph{keeps backwards compatibility}.

		You are encouraged to use
\begin{codeexample}[code only]
\usepackage{pgfplots}
\pgfplotsset{compat=1.3}
\end{codeexample}
		in your preamble.
		
		\item Version 1.3 fixes a bug with |anchor=south west| which occurred mainly for reversed axes: the anchors where upside--down. This is fixed now.

		
		If you have written some work--around and want to keep it going, use

		|\pgfplotsset{compat/anchors=pre 1.3}|\pgfmanualpdflabel{/pgfplots/compat/anchors}{} 

		to disable the bug fix.
	\end{enumerate}

	Use |\pgfplotsset{compat=default}| to restore the factory settings.

	The setting |\pgfplotsset{compat=newest}| will always use features of the most recent version. This might result in small changes in the document's appearance.
\end{pgfplotskey}

\subsection{The Team}
\PGFPlots\ has been written mainly by Christian Feuers�nger with many improvements of Pascal Wolkotte and Nick Papior Andersen as a spare time project. We hope it is useful and provides valuable plots.

If you are interested in writing something but don't know how, consider reading the auxiliary manual \href{file:TeX-programming-notes.pdf}{TeX-programming-notes.pdf} which comes with \PGFPlots. It is far from complete, but maybe it is a good starting point (at least for more literature).

\subsection{Acknowledgements}
I thank God for all hours of enjoyed programming. I thank Pascal Wolkotte and Nick Papior Andersen for their programming efforts and contributions as part of the development team. I thank Stefan Tibus, who contributed the |plot shell| feature. I thank Tom Cashman for the contribution of the |reverse legend| feature. Special thanks go to Stefan Pinnow whose tests of \PGFPlots\ lead to numerous quality improvements. Furthermore, I thank Dr.~Schweitzer for many fruitful discussions and Dr.~Meine for his ideas and suggestions. Thanks as well to the many international contributors who provided feature requests or identified bugs or simply manual improvements!

Last but not least, I thank Till Tantau and Mark Wibrow for their excellent graphics (and more) package \PGF\ and \Tikz\ which is the base of \PGFPlots.

\include{pgfplots.install}%
\include{pgfplots.intro}%
\include{pgfplots.reference}%
\include{pgfplots.libs}%
\include{pgfplots.resources}%
\include{pgfplots.importexport}%
\include{pgfplots.basic.reference}%

\printindex

\bibliographystyle{abbrv} %gerapali} %gerabbrv} %gerunsrt.bst} %gerabbrv}% gerplain}
\nocite{pgfplotstable}
\nocite{programmingnotes}
\bibliography{pgfplots}
\end{document}
%
\include{pgfplots.basic.reference}%

\printindex

\bibliographystyle{abbrv} %gerapali} %gerabbrv} %gerunsrt.bst} %gerabbrv}% gerplain}
\nocite{pgfplotstable}
\nocite{programmingnotes}
\bibliography{pgfplots}
\end{document}
%
%%%%%%%%%%%%%%%%%%%%%%%%%%%%%%%%%%%%%%%%%%%%%%%%%%%%%%%%%%%%%%%%%%%%%%%%%%%%%
%
% Package pgfplots.sty documentation. 
%
% Copyright 2007/2008 by Christian Feuersaenger.
%
% This program is free software: you can redistribute it and/or modify
% it under the terms of the GNU General Public License as published by
% the Free Software Foundation, either version 3 of the License, or
% (at your option) any later version.
% 
% This program is distributed in the hope that it will be useful,
% but WITHOUT ANY WARRANTY; without even the implied warranty of
% MERCHANTABILITY or FITNESS FOR A PARTICULAR PURPOSE.  See the
% GNU General Public License for more details.
% 
% You should have received a copy of the GNU General Public License
% along with this program.  If not, see <http://www.gnu.org/licenses/>.
%
%
%%%%%%%%%%%%%%%%%%%%%%%%%%%%%%%%%%%%%%%%%%%%%%%%%%%%%%%%%%%%%%%%%%%%%%%%%%%%%
%%%%%%%%%%%%%%%%%%%%%%%%%%%%%%%%%%%%%%%%%%%%%%%%%%%%%%%%%%%%%%%%%%%%%%%%%%%%%
%
% Package pgfplots.sty documentation. 
%
% Copyright 2007/2008 by Christian Feuersaenger.
%
% This program is free software: you can redistribute it and/or modify
% it under the terms of the GNU General Public License as published by
% the Free Software Foundation, either version 3 of the License, or
% (at your option) any later version.
% 
% This program is distributed in the hope that it will be useful,
% but WITHOUT ANY WARRANTY; without even the implied warranty of
% MERCHANTABILITY or FITNESS FOR A PARTICULAR PURPOSE.  See the
% GNU General Public License for more details.
% 
% You should have received a copy of the GNU General Public License
% along with this program.  If not, see <http://www.gnu.org/licenses/>.
%
%
%%%%%%%%%%%%%%%%%%%%%%%%%%%%%%%%%%%%%%%%%%%%%%%%%%%%%%%%%%%%%%%%%%%%%%%%%%%%%
\documentclass[a4paper]{ltxdoc}

\usepackage{makeidx}

% DON't let hyperref overload the format of index and glossary. 
% I want to do that on my own in the stylefiles for makeindex...
\makeatletter
\let\@old@wrindex=\@wrindex
\makeatother

\usepackage{ifpdf}
\ifpdf
	\usepackage[pdftex]{hyperref}
\else
	\usepackage[dvipdfm]{hyperref}
\fi
\hypersetup{pdfborder={0 0 0}}

\makeatletter
\let\@wrindex=\@old@wrindex
\makeatother

% Formatiere Seitennummern im Index:
\newcommand{\indexpageno}[1]{%
	{\bfseries\hyperpage{#1}}%
}


\newcommand{\R}{\mathbb{R}}
\newcommand{\N}{\mathbb{N}}
\newcommand{\Z}{\mathbb{Z}}

\long\def\COMMENTLOWLEVEL#1\ENDCOMMENT{}
\def\ENDCOMMENT{}

\usepackage{textcomp}
\usepackage{booktabs}

\usepackage{calc}
\usepackage[formats]{listings}
%\usepackage{courier} % don't use it - the '^' character can't be copy-pasted in courier

\usepackage{array}
\lstset{%
	basicstyle=\ttfamily,
	language=[LaTeX]tex, % Seems as if \lstset{language=tex} must be invoked BEFORE loading tikz!?
	tabsize=4,
	breaklines=true,
	breakindent=0pt
}

\ifpdf
	\pdfinfo {
		/Author	(Christian Feuersaenger)
	}

\else
	\def\pgfsysdriver{pgfsys-dvipdfm.def}
\fi
%\def\pgfsysdriver{pgfsys-pdftex.def}
\usepackage{pgfplots}

\ifpdf
	\usetikzlibrary{pgfplotsclickable}
\fi

%\usepackage{fp}
% ATTENTION:
% this requires pgf version NEWER than 2.00 :
%\usetikzlibrary{fixedpointarithmetic}

\usetikzlibrary{dateplot}

\usepackage[a4paper,left=2.25cm,right=2.25cm,top=2.5cm,bottom=2.5cm,nohead]{geometry}
\usepackage{amsmath,amssymb}
\usepackage{xxcolor}
\usepackage{pifont}
\usepackage[latin1]{inputenc}
\usepackage{amsmath}
\usepackage{eurosym}
\input{pgfplots-macros}

\pgfqkeys{/codeexample}{%
	every codeexample/.style={
		width=8cm,
		/pgfplots/every axis/.append style={legend style={fill=graphicbackground}}
	},
	tabsize=4,
}
\pgfplotsset{
	every axis/.append style={width=7cm},
	filter discard warning=false,
}

\usetikzlibrary{backgrounds,patterns}
% Global styles:
\tikzset{
  shape example/.style={
    color=black!30,
    draw,
    fill=yellow!30,
    line width=.5cm,
    inner xsep=2.5cm,
    inner ysep=0.5cm}
}

\newcommand{\FIXME}[1]{\textcolor{red}{(FIXME: #1)}}

% fuer endvironment 'sidewaysfigure' bspw
% \usepackage{rotating}

\newcommand\Tikz{Ti\textit kZ}
\newcommand\PGF{\textsc{pgf}}
\newcommand\PGFPlots{\textsc{pgfplots}}
\def\PGFPlotstable{\textsc{PgfplotsTable}}


\makeindex

% Fix overful hboxes automatically:
\tolerance=2000
\emergencystretch=10pt

\tikzset{prefix=gnuplot/pgfplots_} % prefix for 'plot function'

\author{%
	Christian Feuers\"anger\footnote{\url{http://wissrech.ins.uni-bonn.de/people/feuersaenger}}\\%
	Institut f\"ur Numerische Simulation\\
	Universit\"at Bonn, Germany}


%\RequirePackage[german,english,francais]{babel}

\pgfplotsmanualexternalexpensivetrue

%\usetikzlibrary{external}
%\tikzexternalize[prefix=figures/]{pgfplots}

\title{%
	Manual for Package \PGFPlots\\
	{\small Version \pgfplotsversion}\\
	{\small\href{http://sourceforge.net/projects/pgfplots}{http://sourceforge.net/projects/pgfplots}}}

%\includeonly{pgfplots.libs}


\begin{document}

\def\plotcoords{%
\addplot coordinates {
(5,8.312e-02)    (17,2.547e-02)   (49,7.407e-03)
(129,2.102e-03)  (321,5.874e-04)  (769,1.623e-04)
(1793,4.442e-05) (4097,1.207e-05) (9217,3.261e-06)
};

\addplot coordinates{
(7,8.472e-02)    (31,3.044e-02)    (111,1.022e-02)
(351,3.303e-03)  (1023,1.039e-03)  (2815,3.196e-04)
(7423,9.658e-05) (18943,2.873e-05) (47103,8.437e-06)
};

\addplot coordinates{
(9,7.881e-02)     (49,3.243e-02)    (209,1.232e-02)
(769,4.454e-03)   (2561,1.551e-03)  (7937,5.236e-04)
(23297,1.723e-04) (65537,5.545e-05) (178177,1.751e-05)
};

\addplot coordinates{
(11,6.887e-02)    (71,3.177e-02)     (351,1.341e-02)
(1471,5.334e-03)  (5503,2.027e-03)   (18943,7.415e-04)
(61183,2.628e-04) (187903,9.063e-05) (553983,3.053e-05)
};

\addplot coordinates{
(13,5.755e-02)     (97,2.925e-02)     (545,1.351e-02)
(2561,5.842e-03)   (10625,2.397e-03)  (40193,9.414e-04)
(141569,3.564e-04) (471041,1.308e-04) 
(1496065,4.670e-05)
};
}%
\maketitle
\begin{abstract}%
\PGFPlots\ draws high--quality function plots in normal or logarithmic scaling with a user-friendly interface directly in \TeX. The user supplies axis labels, legend entries and the plot coordinates for one or more plots and \PGFPlots\ applies axis scaling, computes any logarithms and axis ticks and draws the plots, supporting line plots, scatter plots, piecewise constant plots, bar plots, area plots, mesh-- and surface plots and some more. It is based on Till Tantau's package \PGF/\Tikz.
\end{abstract}
\tableofcontents
\section{Introduction}
This package provides tools to generate plots and labeled axes easily. It draws normal plots, logplots and semi-logplots, in two and three dimensions. Axis ticks, labels, legends (in case of multiple plots) can be added with key-value options. It can cycle through a set of predefined line/marker/color specifications. In summary, its purpose is to simplify the generation of high-quality function and/or data plots, and solving the problems of
\begin{itemize}
	\item consistency of document and font type and font size,
	\item direct use of \TeX\ math mode in axis descriptions,
	\item consistency of data and figures (no third party tool necessary),
	\item inter document consistency using preamble configurations and styles.
\end{itemize}
Although not necessary, separate |.pdf| or |.eps| graphics can be generated using the |external| library developed as part of \Tikz.

\PGFPlots\ is build completely on \Tikz/\PGF. Knowledge of \Tikz\ will simplify the work with \PGFPlots, although it is not required.

\section[About PGFPlots: Preliminaries]{About {\normalfont\PGFPlots}: Preliminaries}
This section contains information about upgrades, the team, the installation (in case you need to do it manually) and troubleshooting. You may skip it completely except for the upgrade remarks.

However, note that this library requires at least \PGF\ version $2.00$. At the time of this writing, many \TeX-distributions still contain the older \PGF\ version $1.18$, so it may be necessary to install a recent \PGF\ prior to using \PGFPlots.

\subsection{Components}
\PGFPlots\ comes with two components:
\begin{enumerate}
	\item the plotting component (which you are currently reading) and
	\item the \PGFPlotstable\ component which simplifies number formatting and postprocessing of numerical tables. It comes as a separate package and has its own manual \href{file:pgfplotstable.pdf}{pgfplotstable.pdf}.
\end{enumerate}

\subsection{Upgrade remarks}
This release provides a lot of improvements which can be found in all detail in \texttt{ChangeLog} for interested readers. However, some attention is useful with respect to the following changes.

\subsubsection{New Optional Features}
\begin{enumerate}
	\item \PGFPlots\ 1.3 comes with user interface improvements. The technical distinction between ``behavior options'' and ``style options'' of older versions is no longer necessary (although still fully supported).

	\item \PGFPlots\ 1.3 has a new feature which allows to \emph{move axis labels tight to tick labels} automatically.

	Since this affects the spacing, it is not enabled be default.

	Use
\begin{codeexample}[code only]
\usepackage{pgfplots}
\pgfplotsset{compat=1.3}
\end{codeexample}
	in your preamble to benefit from the improved spacing\footnote{The |compat=1.3| setting is necessary for new features which move axis labels around like reversed axis. However, \PGFPlots\ will still work as it does in earlier versions, your documents will remain the same if you don't set it explicitly.}.
	Take a look at the next page for the precise definition of |compat|.

	\item \PGFPlots\ 1.3 now supports reversed axes. It is no longer necessary to use work--arounds with negative units.
\pgfkeys{/pdflinks/search key prefixes in/.add={/pgfplots/,}{}}

	Take a look at the |x dir=reverse| key.

	Existing work--arounds will still function properly. Use |\pgfplotsset{compat=1.3}| together with |x dir=reverse| to switch to the new version.
\end{enumerate}

\subsubsection{Old Features Which May Need Attention}
\begin{enumerate}
	\item The |scatter/classes| feature produces proper legends as of version 1.3. This may change the appearance of existing legends of plots with |scatter/classes|.

	\item Due to a small math bug in \PGF\ $2.00$, you \emph{can't} use the math expression `|-x^2|'. It is necessary to use `|0-x^2|' instead. The same holds for
		`|exp(-x^2)|'; use `|exp(0-x^2)|' instead.

		This will be fixed with \PGF\ $>2.00$.

	\item Starting with \PGFPlots\ $1.1$, |\tikzstyle| should \emph{no longer be used} to set \PGFPlots\ options.
	
	Although |\tikzstyle| is still supported for some older \PGFPlots\ options, you should replace any occurance of |\tikzstyle| with |\pgfplotsset{|\meta{style name}|/.style={|\meta{key-value-list}|}}| or the associated |/.append style| variant. See section~\ref{sec:styles} for more detail.
\end{enumerate}
I apologize for any inconvenience caused by these changes.

\begin{pgfplotskey}{compat=\mchoice{1.3,pre 1.3,default,newest} (initially default)}
	The preamble configuration 
\begin{codeexample}[code only]
\usepackage{pgfplots}
\pgfplotsset{compat=1.3}
\end{codeexample}
	allows to choose between backwards compatibility and most recent features.

	\PGFPlots\ version 1.3 comes with new features which may lead to different spacings compared to earlier versions:
	\begin{enumerate}
		\item Version 1.3 comes with the |xlabel near ticks| feature which places (in this case $x$) axis labels ``near'' the tick labels, taking the size of tick labels into account.
		
		Older versions used |xlabel absolute|, i.e.\ an absolute distance between the axis and axis labels (now matter how large or small tick labels are).

		The initial setting \emph{keeps backwards compatibility}.

		You are encouraged to use
\begin{codeexample}[code only]
\usepackage{pgfplots}
\pgfplotsset{compat=1.3}
\end{codeexample}
		in your preamble.
		
		\item Version 1.3 fixes a bug with |anchor=south west| which occurred mainly for reversed axes: the anchors where upside--down. This is fixed now.

		
		If you have written some work--around and want to keep it going, use

		|\pgfplotsset{compat/anchors=pre 1.3}|\pgfmanualpdflabel{/pgfplots/compat/anchors}{} 

		to disable the bug fix.
	\end{enumerate}

	Use |\pgfplotsset{compat=default}| to restore the factory settings.

	The setting |\pgfplotsset{compat=newest}| will always use features of the most recent version. This might result in small changes in the document's appearance.
\end{pgfplotskey}

\subsection{The Team}
\PGFPlots\ has been written mainly by Christian Feuers�nger with many improvements of Pascal Wolkotte and Nick Papior Andersen as a spare time project. We hope it is useful and provides valuable plots.

If you are interested in writing something but don't know how, consider reading the auxiliary manual \href{file:TeX-programming-notes.pdf}{TeX-programming-notes.pdf} which comes with \PGFPlots. It is far from complete, but maybe it is a good starting point (at least for more literature).

\subsection{Acknowledgements}
I thank God for all hours of enjoyed programming. I thank Pascal Wolkotte and Nick Papior Andersen for their programming efforts and contributions as part of the development team. I thank Stefan Tibus, who contributed the |plot shell| feature. I thank Tom Cashman for the contribution of the |reverse legend| feature. Special thanks go to Stefan Pinnow whose tests of \PGFPlots\ lead to numerous quality improvements. Furthermore, I thank Dr.~Schweitzer for many fruitful discussions and Dr.~Meine for his ideas and suggestions. Thanks as well to the many international contributors who provided feature requests or identified bugs or simply manual improvements!

Last but not least, I thank Till Tantau and Mark Wibrow for their excellent graphics (and more) package \PGF\ and \Tikz\ which is the base of \PGFPlots.

%%%%%%%%%%%%%%%%%%%%%%%%%%%%%%%%%%%%%%%%%%%%%%%%%%%%%%%%%%%%%%%%%%%%%%%%%%%%%
%
% Package pgfplots.sty documentation. 
%
% Copyright 2007/2008 by Christian Feuersaenger.
%
% This program is free software: you can redistribute it and/or modify
% it under the terms of the GNU General Public License as published by
% the Free Software Foundation, either version 3 of the License, or
% (at your option) any later version.
% 
% This program is distributed in the hope that it will be useful,
% but WITHOUT ANY WARRANTY; without even the implied warranty of
% MERCHANTABILITY or FITNESS FOR A PARTICULAR PURPOSE.  See the
% GNU General Public License for more details.
% 
% You should have received a copy of the GNU General Public License
% along with this program.  If not, see <http://www.gnu.org/licenses/>.
%
%
%%%%%%%%%%%%%%%%%%%%%%%%%%%%%%%%%%%%%%%%%%%%%%%%%%%%%%%%%%%%%%%%%%%%%%%%%%%%%
\input{pgfplots.preamble.tex}
%\RequirePackage[german,english,francais]{babel}

\pgfplotsmanualexternalexpensivetrue

%\usetikzlibrary{external}
%\tikzexternalize[prefix=figures/]{pgfplots}

\title{%
	Manual for Package \PGFPlots\\
	{\small Version \pgfplotsversion}\\
	{\small\href{http://sourceforge.net/projects/pgfplots}{http://sourceforge.net/projects/pgfplots}}}

%\includeonly{pgfplots.libs}


\begin{document}

\def\plotcoords{%
\addplot coordinates {
(5,8.312e-02)    (17,2.547e-02)   (49,7.407e-03)
(129,2.102e-03)  (321,5.874e-04)  (769,1.623e-04)
(1793,4.442e-05) (4097,1.207e-05) (9217,3.261e-06)
};

\addplot coordinates{
(7,8.472e-02)    (31,3.044e-02)    (111,1.022e-02)
(351,3.303e-03)  (1023,1.039e-03)  (2815,3.196e-04)
(7423,9.658e-05) (18943,2.873e-05) (47103,8.437e-06)
};

\addplot coordinates{
(9,7.881e-02)     (49,3.243e-02)    (209,1.232e-02)
(769,4.454e-03)   (2561,1.551e-03)  (7937,5.236e-04)
(23297,1.723e-04) (65537,5.545e-05) (178177,1.751e-05)
};

\addplot coordinates{
(11,6.887e-02)    (71,3.177e-02)     (351,1.341e-02)
(1471,5.334e-03)  (5503,2.027e-03)   (18943,7.415e-04)
(61183,2.628e-04) (187903,9.063e-05) (553983,3.053e-05)
};

\addplot coordinates{
(13,5.755e-02)     (97,2.925e-02)     (545,1.351e-02)
(2561,5.842e-03)   (10625,2.397e-03)  (40193,9.414e-04)
(141569,3.564e-04) (471041,1.308e-04) 
(1496065,4.670e-05)
};
}%
\maketitle
\begin{abstract}%
\PGFPlots\ draws high--quality function plots in normal or logarithmic scaling with a user-friendly interface directly in \TeX. The user supplies axis labels, legend entries and the plot coordinates for one or more plots and \PGFPlots\ applies axis scaling, computes any logarithms and axis ticks and draws the plots, supporting line plots, scatter plots, piecewise constant plots, bar plots, area plots, mesh-- and surface plots and some more. It is based on Till Tantau's package \PGF/\Tikz.
\end{abstract}
\tableofcontents
\section{Introduction}
This package provides tools to generate plots and labeled axes easily. It draws normal plots, logplots and semi-logplots, in two and three dimensions. Axis ticks, labels, legends (in case of multiple plots) can be added with key-value options. It can cycle through a set of predefined line/marker/color specifications. In summary, its purpose is to simplify the generation of high-quality function and/or data plots, and solving the problems of
\begin{itemize}
	\item consistency of document and font type and font size,
	\item direct use of \TeX\ math mode in axis descriptions,
	\item consistency of data and figures (no third party tool necessary),
	\item inter document consistency using preamble configurations and styles.
\end{itemize}
Although not necessary, separate |.pdf| or |.eps| graphics can be generated using the |external| library developed as part of \Tikz.

\PGFPlots\ is build completely on \Tikz/\PGF. Knowledge of \Tikz\ will simplify the work with \PGFPlots, although it is not required.

\section[About PGFPlots: Preliminaries]{About {\normalfont\PGFPlots}: Preliminaries}
This section contains information about upgrades, the team, the installation (in case you need to do it manually) and troubleshooting. You may skip it completely except for the upgrade remarks.

However, note that this library requires at least \PGF\ version $2.00$. At the time of this writing, many \TeX-distributions still contain the older \PGF\ version $1.18$, so it may be necessary to install a recent \PGF\ prior to using \PGFPlots.

\subsection{Components}
\PGFPlots\ comes with two components:
\begin{enumerate}
	\item the plotting component (which you are currently reading) and
	\item the \PGFPlotstable\ component which simplifies number formatting and postprocessing of numerical tables. It comes as a separate package and has its own manual \href{file:pgfplotstable.pdf}{pgfplotstable.pdf}.
\end{enumerate}

\subsection{Upgrade remarks}
This release provides a lot of improvements which can be found in all detail in \texttt{ChangeLog} for interested readers. However, some attention is useful with respect to the following changes.

\subsubsection{New Optional Features}
\begin{enumerate}
	\item \PGFPlots\ 1.3 comes with user interface improvements. The technical distinction between ``behavior options'' and ``style options'' of older versions is no longer necessary (although still fully supported).

	\item \PGFPlots\ 1.3 has a new feature which allows to \emph{move axis labels tight to tick labels} automatically.

	Since this affects the spacing, it is not enabled be default.

	Use
\begin{codeexample}[code only]
\usepackage{pgfplots}
\pgfplotsset{compat=1.3}
\end{codeexample}
	in your preamble to benefit from the improved spacing\footnote{The |compat=1.3| setting is necessary for new features which move axis labels around like reversed axis. However, \PGFPlots\ will still work as it does in earlier versions, your documents will remain the same if you don't set it explicitly.}.
	Take a look at the next page for the precise definition of |compat|.

	\item \PGFPlots\ 1.3 now supports reversed axes. It is no longer necessary to use work--arounds with negative units.
\pgfkeys{/pdflinks/search key prefixes in/.add={/pgfplots/,}{}}

	Take a look at the |x dir=reverse| key.

	Existing work--arounds will still function properly. Use |\pgfplotsset{compat=1.3}| together with |x dir=reverse| to switch to the new version.
\end{enumerate}

\subsubsection{Old Features Which May Need Attention}
\begin{enumerate}
	\item The |scatter/classes| feature produces proper legends as of version 1.3. This may change the appearance of existing legends of plots with |scatter/classes|.

	\item Due to a small math bug in \PGF\ $2.00$, you \emph{can't} use the math expression `|-x^2|'. It is necessary to use `|0-x^2|' instead. The same holds for
		`|exp(-x^2)|'; use `|exp(0-x^2)|' instead.

		This will be fixed with \PGF\ $>2.00$.

	\item Starting with \PGFPlots\ $1.1$, |\tikzstyle| should \emph{no longer be used} to set \PGFPlots\ options.
	
	Although |\tikzstyle| is still supported for some older \PGFPlots\ options, you should replace any occurance of |\tikzstyle| with |\pgfplotsset{|\meta{style name}|/.style={|\meta{key-value-list}|}}| or the associated |/.append style| variant. See section~\ref{sec:styles} for more detail.
\end{enumerate}
I apologize for any inconvenience caused by these changes.

\begin{pgfplotskey}{compat=\mchoice{1.3,pre 1.3,default,newest} (initially default)}
	The preamble configuration 
\begin{codeexample}[code only]
\usepackage{pgfplots}
\pgfplotsset{compat=1.3}
\end{codeexample}
	allows to choose between backwards compatibility and most recent features.

	\PGFPlots\ version 1.3 comes with new features which may lead to different spacings compared to earlier versions:
	\begin{enumerate}
		\item Version 1.3 comes with the |xlabel near ticks| feature which places (in this case $x$) axis labels ``near'' the tick labels, taking the size of tick labels into account.
		
		Older versions used |xlabel absolute|, i.e.\ an absolute distance between the axis and axis labels (now matter how large or small tick labels are).

		The initial setting \emph{keeps backwards compatibility}.

		You are encouraged to use
\begin{codeexample}[code only]
\usepackage{pgfplots}
\pgfplotsset{compat=1.3}
\end{codeexample}
		in your preamble.
		
		\item Version 1.3 fixes a bug with |anchor=south west| which occurred mainly for reversed axes: the anchors where upside--down. This is fixed now.

		
		If you have written some work--around and want to keep it going, use

		|\pgfplotsset{compat/anchors=pre 1.3}|\pgfmanualpdflabel{/pgfplots/compat/anchors}{} 

		to disable the bug fix.
	\end{enumerate}

	Use |\pgfplotsset{compat=default}| to restore the factory settings.

	The setting |\pgfplotsset{compat=newest}| will always use features of the most recent version. This might result in small changes in the document's appearance.
\end{pgfplotskey}

\subsection{The Team}
\PGFPlots\ has been written mainly by Christian Feuers�nger with many improvements of Pascal Wolkotte and Nick Papior Andersen as a spare time project. We hope it is useful and provides valuable plots.

If you are interested in writing something but don't know how, consider reading the auxiliary manual \href{file:TeX-programming-notes.pdf}{TeX-programming-notes.pdf} which comes with \PGFPlots. It is far from complete, but maybe it is a good starting point (at least for more literature).

\subsection{Acknowledgements}
I thank God for all hours of enjoyed programming. I thank Pascal Wolkotte and Nick Papior Andersen for their programming efforts and contributions as part of the development team. I thank Stefan Tibus, who contributed the |plot shell| feature. I thank Tom Cashman for the contribution of the |reverse legend| feature. Special thanks go to Stefan Pinnow whose tests of \PGFPlots\ lead to numerous quality improvements. Furthermore, I thank Dr.~Schweitzer for many fruitful discussions and Dr.~Meine for his ideas and suggestions. Thanks as well to the many international contributors who provided feature requests or identified bugs or simply manual improvements!

Last but not least, I thank Till Tantau and Mark Wibrow for their excellent graphics (and more) package \PGF\ and \Tikz\ which is the base of \PGFPlots.

\include{pgfplots.install}%
\include{pgfplots.intro}%
\include{pgfplots.reference}%
\include{pgfplots.libs}%
\include{pgfplots.resources}%
\include{pgfplots.importexport}%
\include{pgfplots.basic.reference}%

\printindex

\bibliographystyle{abbrv} %gerapali} %gerabbrv} %gerunsrt.bst} %gerabbrv}% gerplain}
\nocite{pgfplotstable}
\nocite{programmingnotes}
\bibliography{pgfplots}
\end{document}
%
%%%%%%%%%%%%%%%%%%%%%%%%%%%%%%%%%%%%%%%%%%%%%%%%%%%%%%%%%%%%%%%%%%%%%%%%%%%%%
%
% Package pgfplots.sty documentation. 
%
% Copyright 2007/2008 by Christian Feuersaenger.
%
% This program is free software: you can redistribute it and/or modify
% it under the terms of the GNU General Public License as published by
% the Free Software Foundation, either version 3 of the License, or
% (at your option) any later version.
% 
% This program is distributed in the hope that it will be useful,
% but WITHOUT ANY WARRANTY; without even the implied warranty of
% MERCHANTABILITY or FITNESS FOR A PARTICULAR PURPOSE.  See the
% GNU General Public License for more details.
% 
% You should have received a copy of the GNU General Public License
% along with this program.  If not, see <http://www.gnu.org/licenses/>.
%
%
%%%%%%%%%%%%%%%%%%%%%%%%%%%%%%%%%%%%%%%%%%%%%%%%%%%%%%%%%%%%%%%%%%%%%%%%%%%%%
\input{pgfplots.preamble.tex}
%\RequirePackage[german,english,francais]{babel}

\pgfplotsmanualexternalexpensivetrue

%\usetikzlibrary{external}
%\tikzexternalize[prefix=figures/]{pgfplots}

\title{%
	Manual for Package \PGFPlots\\
	{\small Version \pgfplotsversion}\\
	{\small\href{http://sourceforge.net/projects/pgfplots}{http://sourceforge.net/projects/pgfplots}}}

%\includeonly{pgfplots.libs}


\begin{document}

\def\plotcoords{%
\addplot coordinates {
(5,8.312e-02)    (17,2.547e-02)   (49,7.407e-03)
(129,2.102e-03)  (321,5.874e-04)  (769,1.623e-04)
(1793,4.442e-05) (4097,1.207e-05) (9217,3.261e-06)
};

\addplot coordinates{
(7,8.472e-02)    (31,3.044e-02)    (111,1.022e-02)
(351,3.303e-03)  (1023,1.039e-03)  (2815,3.196e-04)
(7423,9.658e-05) (18943,2.873e-05) (47103,8.437e-06)
};

\addplot coordinates{
(9,7.881e-02)     (49,3.243e-02)    (209,1.232e-02)
(769,4.454e-03)   (2561,1.551e-03)  (7937,5.236e-04)
(23297,1.723e-04) (65537,5.545e-05) (178177,1.751e-05)
};

\addplot coordinates{
(11,6.887e-02)    (71,3.177e-02)     (351,1.341e-02)
(1471,5.334e-03)  (5503,2.027e-03)   (18943,7.415e-04)
(61183,2.628e-04) (187903,9.063e-05) (553983,3.053e-05)
};

\addplot coordinates{
(13,5.755e-02)     (97,2.925e-02)     (545,1.351e-02)
(2561,5.842e-03)   (10625,2.397e-03)  (40193,9.414e-04)
(141569,3.564e-04) (471041,1.308e-04) 
(1496065,4.670e-05)
};
}%
\maketitle
\begin{abstract}%
\PGFPlots\ draws high--quality function plots in normal or logarithmic scaling with a user-friendly interface directly in \TeX. The user supplies axis labels, legend entries and the plot coordinates for one or more plots and \PGFPlots\ applies axis scaling, computes any logarithms and axis ticks and draws the plots, supporting line plots, scatter plots, piecewise constant plots, bar plots, area plots, mesh-- and surface plots and some more. It is based on Till Tantau's package \PGF/\Tikz.
\end{abstract}
\tableofcontents
\section{Introduction}
This package provides tools to generate plots and labeled axes easily. It draws normal plots, logplots and semi-logplots, in two and three dimensions. Axis ticks, labels, legends (in case of multiple plots) can be added with key-value options. It can cycle through a set of predefined line/marker/color specifications. In summary, its purpose is to simplify the generation of high-quality function and/or data plots, and solving the problems of
\begin{itemize}
	\item consistency of document and font type and font size,
	\item direct use of \TeX\ math mode in axis descriptions,
	\item consistency of data and figures (no third party tool necessary),
	\item inter document consistency using preamble configurations and styles.
\end{itemize}
Although not necessary, separate |.pdf| or |.eps| graphics can be generated using the |external| library developed as part of \Tikz.

\PGFPlots\ is build completely on \Tikz/\PGF. Knowledge of \Tikz\ will simplify the work with \PGFPlots, although it is not required.

\section[About PGFPlots: Preliminaries]{About {\normalfont\PGFPlots}: Preliminaries}
This section contains information about upgrades, the team, the installation (in case you need to do it manually) and troubleshooting. You may skip it completely except for the upgrade remarks.

However, note that this library requires at least \PGF\ version $2.00$. At the time of this writing, many \TeX-distributions still contain the older \PGF\ version $1.18$, so it may be necessary to install a recent \PGF\ prior to using \PGFPlots.

\subsection{Components}
\PGFPlots\ comes with two components:
\begin{enumerate}
	\item the plotting component (which you are currently reading) and
	\item the \PGFPlotstable\ component which simplifies number formatting and postprocessing of numerical tables. It comes as a separate package and has its own manual \href{file:pgfplotstable.pdf}{pgfplotstable.pdf}.
\end{enumerate}

\subsection{Upgrade remarks}
This release provides a lot of improvements which can be found in all detail in \texttt{ChangeLog} for interested readers. However, some attention is useful with respect to the following changes.

\subsubsection{New Optional Features}
\begin{enumerate}
	\item \PGFPlots\ 1.3 comes with user interface improvements. The technical distinction between ``behavior options'' and ``style options'' of older versions is no longer necessary (although still fully supported).

	\item \PGFPlots\ 1.3 has a new feature which allows to \emph{move axis labels tight to tick labels} automatically.

	Since this affects the spacing, it is not enabled be default.

	Use
\begin{codeexample}[code only]
\usepackage{pgfplots}
\pgfplotsset{compat=1.3}
\end{codeexample}
	in your preamble to benefit from the improved spacing\footnote{The |compat=1.3| setting is necessary for new features which move axis labels around like reversed axis. However, \PGFPlots\ will still work as it does in earlier versions, your documents will remain the same if you don't set it explicitly.}.
	Take a look at the next page for the precise definition of |compat|.

	\item \PGFPlots\ 1.3 now supports reversed axes. It is no longer necessary to use work--arounds with negative units.
\pgfkeys{/pdflinks/search key prefixes in/.add={/pgfplots/,}{}}

	Take a look at the |x dir=reverse| key.

	Existing work--arounds will still function properly. Use |\pgfplotsset{compat=1.3}| together with |x dir=reverse| to switch to the new version.
\end{enumerate}

\subsubsection{Old Features Which May Need Attention}
\begin{enumerate}
	\item The |scatter/classes| feature produces proper legends as of version 1.3. This may change the appearance of existing legends of plots with |scatter/classes|.

	\item Due to a small math bug in \PGF\ $2.00$, you \emph{can't} use the math expression `|-x^2|'. It is necessary to use `|0-x^2|' instead. The same holds for
		`|exp(-x^2)|'; use `|exp(0-x^2)|' instead.

		This will be fixed with \PGF\ $>2.00$.

	\item Starting with \PGFPlots\ $1.1$, |\tikzstyle| should \emph{no longer be used} to set \PGFPlots\ options.
	
	Although |\tikzstyle| is still supported for some older \PGFPlots\ options, you should replace any occurance of |\tikzstyle| with |\pgfplotsset{|\meta{style name}|/.style={|\meta{key-value-list}|}}| or the associated |/.append style| variant. See section~\ref{sec:styles} for more detail.
\end{enumerate}
I apologize for any inconvenience caused by these changes.

\begin{pgfplotskey}{compat=\mchoice{1.3,pre 1.3,default,newest} (initially default)}
	The preamble configuration 
\begin{codeexample}[code only]
\usepackage{pgfplots}
\pgfplotsset{compat=1.3}
\end{codeexample}
	allows to choose between backwards compatibility and most recent features.

	\PGFPlots\ version 1.3 comes with new features which may lead to different spacings compared to earlier versions:
	\begin{enumerate}
		\item Version 1.3 comes with the |xlabel near ticks| feature which places (in this case $x$) axis labels ``near'' the tick labels, taking the size of tick labels into account.
		
		Older versions used |xlabel absolute|, i.e.\ an absolute distance between the axis and axis labels (now matter how large or small tick labels are).

		The initial setting \emph{keeps backwards compatibility}.

		You are encouraged to use
\begin{codeexample}[code only]
\usepackage{pgfplots}
\pgfplotsset{compat=1.3}
\end{codeexample}
		in your preamble.
		
		\item Version 1.3 fixes a bug with |anchor=south west| which occurred mainly for reversed axes: the anchors where upside--down. This is fixed now.

		
		If you have written some work--around and want to keep it going, use

		|\pgfplotsset{compat/anchors=pre 1.3}|\pgfmanualpdflabel{/pgfplots/compat/anchors}{} 

		to disable the bug fix.
	\end{enumerate}

	Use |\pgfplotsset{compat=default}| to restore the factory settings.

	The setting |\pgfplotsset{compat=newest}| will always use features of the most recent version. This might result in small changes in the document's appearance.
\end{pgfplotskey}

\subsection{The Team}
\PGFPlots\ has been written mainly by Christian Feuers�nger with many improvements of Pascal Wolkotte and Nick Papior Andersen as a spare time project. We hope it is useful and provides valuable plots.

If you are interested in writing something but don't know how, consider reading the auxiliary manual \href{file:TeX-programming-notes.pdf}{TeX-programming-notes.pdf} which comes with \PGFPlots. It is far from complete, but maybe it is a good starting point (at least for more literature).

\subsection{Acknowledgements}
I thank God for all hours of enjoyed programming. I thank Pascal Wolkotte and Nick Papior Andersen for their programming efforts and contributions as part of the development team. I thank Stefan Tibus, who contributed the |plot shell| feature. I thank Tom Cashman for the contribution of the |reverse legend| feature. Special thanks go to Stefan Pinnow whose tests of \PGFPlots\ lead to numerous quality improvements. Furthermore, I thank Dr.~Schweitzer for many fruitful discussions and Dr.~Meine for his ideas and suggestions. Thanks as well to the many international contributors who provided feature requests or identified bugs or simply manual improvements!

Last but not least, I thank Till Tantau and Mark Wibrow for their excellent graphics (and more) package \PGF\ and \Tikz\ which is the base of \PGFPlots.

\include{pgfplots.install}%
\include{pgfplots.intro}%
\include{pgfplots.reference}%
\include{pgfplots.libs}%
\include{pgfplots.resources}%
\include{pgfplots.importexport}%
\include{pgfplots.basic.reference}%

\printindex

\bibliographystyle{abbrv} %gerapali} %gerabbrv} %gerunsrt.bst} %gerabbrv}% gerplain}
\nocite{pgfplotstable}
\nocite{programmingnotes}
\bibliography{pgfplots}
\end{document}
%
%%%%%%%%%%%%%%%%%%%%%%%%%%%%%%%%%%%%%%%%%%%%%%%%%%%%%%%%%%%%%%%%%%%%%%%%%%%%%
%
% Package pgfplots.sty documentation. 
%
% Copyright 2007/2008 by Christian Feuersaenger.
%
% This program is free software: you can redistribute it and/or modify
% it under the terms of the GNU General Public License as published by
% the Free Software Foundation, either version 3 of the License, or
% (at your option) any later version.
% 
% This program is distributed in the hope that it will be useful,
% but WITHOUT ANY WARRANTY; without even the implied warranty of
% MERCHANTABILITY or FITNESS FOR A PARTICULAR PURPOSE.  See the
% GNU General Public License for more details.
% 
% You should have received a copy of the GNU General Public License
% along with this program.  If not, see <http://www.gnu.org/licenses/>.
%
%
%%%%%%%%%%%%%%%%%%%%%%%%%%%%%%%%%%%%%%%%%%%%%%%%%%%%%%%%%%%%%%%%%%%%%%%%%%%%%
\input{pgfplots.preamble.tex}
%\RequirePackage[german,english,francais]{babel}

\pgfplotsmanualexternalexpensivetrue

%\usetikzlibrary{external}
%\tikzexternalize[prefix=figures/]{pgfplots}

\title{%
	Manual for Package \PGFPlots\\
	{\small Version \pgfplotsversion}\\
	{\small\href{http://sourceforge.net/projects/pgfplots}{http://sourceforge.net/projects/pgfplots}}}

%\includeonly{pgfplots.libs}


\begin{document}

\def\plotcoords{%
\addplot coordinates {
(5,8.312e-02)    (17,2.547e-02)   (49,7.407e-03)
(129,2.102e-03)  (321,5.874e-04)  (769,1.623e-04)
(1793,4.442e-05) (4097,1.207e-05) (9217,3.261e-06)
};

\addplot coordinates{
(7,8.472e-02)    (31,3.044e-02)    (111,1.022e-02)
(351,3.303e-03)  (1023,1.039e-03)  (2815,3.196e-04)
(7423,9.658e-05) (18943,2.873e-05) (47103,8.437e-06)
};

\addplot coordinates{
(9,7.881e-02)     (49,3.243e-02)    (209,1.232e-02)
(769,4.454e-03)   (2561,1.551e-03)  (7937,5.236e-04)
(23297,1.723e-04) (65537,5.545e-05) (178177,1.751e-05)
};

\addplot coordinates{
(11,6.887e-02)    (71,3.177e-02)     (351,1.341e-02)
(1471,5.334e-03)  (5503,2.027e-03)   (18943,7.415e-04)
(61183,2.628e-04) (187903,9.063e-05) (553983,3.053e-05)
};

\addplot coordinates{
(13,5.755e-02)     (97,2.925e-02)     (545,1.351e-02)
(2561,5.842e-03)   (10625,2.397e-03)  (40193,9.414e-04)
(141569,3.564e-04) (471041,1.308e-04) 
(1496065,4.670e-05)
};
}%
\maketitle
\begin{abstract}%
\PGFPlots\ draws high--quality function plots in normal or logarithmic scaling with a user-friendly interface directly in \TeX. The user supplies axis labels, legend entries and the plot coordinates for one or more plots and \PGFPlots\ applies axis scaling, computes any logarithms and axis ticks and draws the plots, supporting line plots, scatter plots, piecewise constant plots, bar plots, area plots, mesh-- and surface plots and some more. It is based on Till Tantau's package \PGF/\Tikz.
\end{abstract}
\tableofcontents
\section{Introduction}
This package provides tools to generate plots and labeled axes easily. It draws normal plots, logplots and semi-logplots, in two and three dimensions. Axis ticks, labels, legends (in case of multiple plots) can be added with key-value options. It can cycle through a set of predefined line/marker/color specifications. In summary, its purpose is to simplify the generation of high-quality function and/or data plots, and solving the problems of
\begin{itemize}
	\item consistency of document and font type and font size,
	\item direct use of \TeX\ math mode in axis descriptions,
	\item consistency of data and figures (no third party tool necessary),
	\item inter document consistency using preamble configurations and styles.
\end{itemize}
Although not necessary, separate |.pdf| or |.eps| graphics can be generated using the |external| library developed as part of \Tikz.

\PGFPlots\ is build completely on \Tikz/\PGF. Knowledge of \Tikz\ will simplify the work with \PGFPlots, although it is not required.

\section[About PGFPlots: Preliminaries]{About {\normalfont\PGFPlots}: Preliminaries}
This section contains information about upgrades, the team, the installation (in case you need to do it manually) and troubleshooting. You may skip it completely except for the upgrade remarks.

However, note that this library requires at least \PGF\ version $2.00$. At the time of this writing, many \TeX-distributions still contain the older \PGF\ version $1.18$, so it may be necessary to install a recent \PGF\ prior to using \PGFPlots.

\subsection{Components}
\PGFPlots\ comes with two components:
\begin{enumerate}
	\item the plotting component (which you are currently reading) and
	\item the \PGFPlotstable\ component which simplifies number formatting and postprocessing of numerical tables. It comes as a separate package and has its own manual \href{file:pgfplotstable.pdf}{pgfplotstable.pdf}.
\end{enumerate}

\subsection{Upgrade remarks}
This release provides a lot of improvements which can be found in all detail in \texttt{ChangeLog} for interested readers. However, some attention is useful with respect to the following changes.

\subsubsection{New Optional Features}
\begin{enumerate}
	\item \PGFPlots\ 1.3 comes with user interface improvements. The technical distinction between ``behavior options'' and ``style options'' of older versions is no longer necessary (although still fully supported).

	\item \PGFPlots\ 1.3 has a new feature which allows to \emph{move axis labels tight to tick labels} automatically.

	Since this affects the spacing, it is not enabled be default.

	Use
\begin{codeexample}[code only]
\usepackage{pgfplots}
\pgfplotsset{compat=1.3}
\end{codeexample}
	in your preamble to benefit from the improved spacing\footnote{The |compat=1.3| setting is necessary for new features which move axis labels around like reversed axis. However, \PGFPlots\ will still work as it does in earlier versions, your documents will remain the same if you don't set it explicitly.}.
	Take a look at the next page for the precise definition of |compat|.

	\item \PGFPlots\ 1.3 now supports reversed axes. It is no longer necessary to use work--arounds with negative units.
\pgfkeys{/pdflinks/search key prefixes in/.add={/pgfplots/,}{}}

	Take a look at the |x dir=reverse| key.

	Existing work--arounds will still function properly. Use |\pgfplotsset{compat=1.3}| together with |x dir=reverse| to switch to the new version.
\end{enumerate}

\subsubsection{Old Features Which May Need Attention}
\begin{enumerate}
	\item The |scatter/classes| feature produces proper legends as of version 1.3. This may change the appearance of existing legends of plots with |scatter/classes|.

	\item Due to a small math bug in \PGF\ $2.00$, you \emph{can't} use the math expression `|-x^2|'. It is necessary to use `|0-x^2|' instead. The same holds for
		`|exp(-x^2)|'; use `|exp(0-x^2)|' instead.

		This will be fixed with \PGF\ $>2.00$.

	\item Starting with \PGFPlots\ $1.1$, |\tikzstyle| should \emph{no longer be used} to set \PGFPlots\ options.
	
	Although |\tikzstyle| is still supported for some older \PGFPlots\ options, you should replace any occurance of |\tikzstyle| with |\pgfplotsset{|\meta{style name}|/.style={|\meta{key-value-list}|}}| or the associated |/.append style| variant. See section~\ref{sec:styles} for more detail.
\end{enumerate}
I apologize for any inconvenience caused by these changes.

\begin{pgfplotskey}{compat=\mchoice{1.3,pre 1.3,default,newest} (initially default)}
	The preamble configuration 
\begin{codeexample}[code only]
\usepackage{pgfplots}
\pgfplotsset{compat=1.3}
\end{codeexample}
	allows to choose between backwards compatibility and most recent features.

	\PGFPlots\ version 1.3 comes with new features which may lead to different spacings compared to earlier versions:
	\begin{enumerate}
		\item Version 1.3 comes with the |xlabel near ticks| feature which places (in this case $x$) axis labels ``near'' the tick labels, taking the size of tick labels into account.
		
		Older versions used |xlabel absolute|, i.e.\ an absolute distance between the axis and axis labels (now matter how large or small tick labels are).

		The initial setting \emph{keeps backwards compatibility}.

		You are encouraged to use
\begin{codeexample}[code only]
\usepackage{pgfplots}
\pgfplotsset{compat=1.3}
\end{codeexample}
		in your preamble.
		
		\item Version 1.3 fixes a bug with |anchor=south west| which occurred mainly for reversed axes: the anchors where upside--down. This is fixed now.

		
		If you have written some work--around and want to keep it going, use

		|\pgfplotsset{compat/anchors=pre 1.3}|\pgfmanualpdflabel{/pgfplots/compat/anchors}{} 

		to disable the bug fix.
	\end{enumerate}

	Use |\pgfplotsset{compat=default}| to restore the factory settings.

	The setting |\pgfplotsset{compat=newest}| will always use features of the most recent version. This might result in small changes in the document's appearance.
\end{pgfplotskey}

\subsection{The Team}
\PGFPlots\ has been written mainly by Christian Feuers�nger with many improvements of Pascal Wolkotte and Nick Papior Andersen as a spare time project. We hope it is useful and provides valuable plots.

If you are interested in writing something but don't know how, consider reading the auxiliary manual \href{file:TeX-programming-notes.pdf}{TeX-programming-notes.pdf} which comes with \PGFPlots. It is far from complete, but maybe it is a good starting point (at least for more literature).

\subsection{Acknowledgements}
I thank God for all hours of enjoyed programming. I thank Pascal Wolkotte and Nick Papior Andersen for their programming efforts and contributions as part of the development team. I thank Stefan Tibus, who contributed the |plot shell| feature. I thank Tom Cashman for the contribution of the |reverse legend| feature. Special thanks go to Stefan Pinnow whose tests of \PGFPlots\ lead to numerous quality improvements. Furthermore, I thank Dr.~Schweitzer for many fruitful discussions and Dr.~Meine for his ideas and suggestions. Thanks as well to the many international contributors who provided feature requests or identified bugs or simply manual improvements!

Last but not least, I thank Till Tantau and Mark Wibrow for their excellent graphics (and more) package \PGF\ and \Tikz\ which is the base of \PGFPlots.

\include{pgfplots.install}%
\include{pgfplots.intro}%
\include{pgfplots.reference}%
\include{pgfplots.libs}%
\include{pgfplots.resources}%
\include{pgfplots.importexport}%
\include{pgfplots.basic.reference}%

\printindex

\bibliographystyle{abbrv} %gerapali} %gerabbrv} %gerunsrt.bst} %gerabbrv}% gerplain}
\nocite{pgfplotstable}
\nocite{programmingnotes}
\bibliography{pgfplots}
\end{document}
%
%%%%%%%%%%%%%%%%%%%%%%%%%%%%%%%%%%%%%%%%%%%%%%%%%%%%%%%%%%%%%%%%%%%%%%%%%%%%%
%
% Package pgfplots.sty documentation. 
%
% Copyright 2007/2008 by Christian Feuersaenger.
%
% This program is free software: you can redistribute it and/or modify
% it under the terms of the GNU General Public License as published by
% the Free Software Foundation, either version 3 of the License, or
% (at your option) any later version.
% 
% This program is distributed in the hope that it will be useful,
% but WITHOUT ANY WARRANTY; without even the implied warranty of
% MERCHANTABILITY or FITNESS FOR A PARTICULAR PURPOSE.  See the
% GNU General Public License for more details.
% 
% You should have received a copy of the GNU General Public License
% along with this program.  If not, see <http://www.gnu.org/licenses/>.
%
%
%%%%%%%%%%%%%%%%%%%%%%%%%%%%%%%%%%%%%%%%%%%%%%%%%%%%%%%%%%%%%%%%%%%%%%%%%%%%%
\input{pgfplots.preamble.tex}
%\RequirePackage[german,english,francais]{babel}

\pgfplotsmanualexternalexpensivetrue

%\usetikzlibrary{external}
%\tikzexternalize[prefix=figures/]{pgfplots}

\title{%
	Manual for Package \PGFPlots\\
	{\small Version \pgfplotsversion}\\
	{\small\href{http://sourceforge.net/projects/pgfplots}{http://sourceforge.net/projects/pgfplots}}}

%\includeonly{pgfplots.libs}


\begin{document}

\def\plotcoords{%
\addplot coordinates {
(5,8.312e-02)    (17,2.547e-02)   (49,7.407e-03)
(129,2.102e-03)  (321,5.874e-04)  (769,1.623e-04)
(1793,4.442e-05) (4097,1.207e-05) (9217,3.261e-06)
};

\addplot coordinates{
(7,8.472e-02)    (31,3.044e-02)    (111,1.022e-02)
(351,3.303e-03)  (1023,1.039e-03)  (2815,3.196e-04)
(7423,9.658e-05) (18943,2.873e-05) (47103,8.437e-06)
};

\addplot coordinates{
(9,7.881e-02)     (49,3.243e-02)    (209,1.232e-02)
(769,4.454e-03)   (2561,1.551e-03)  (7937,5.236e-04)
(23297,1.723e-04) (65537,5.545e-05) (178177,1.751e-05)
};

\addplot coordinates{
(11,6.887e-02)    (71,3.177e-02)     (351,1.341e-02)
(1471,5.334e-03)  (5503,2.027e-03)   (18943,7.415e-04)
(61183,2.628e-04) (187903,9.063e-05) (553983,3.053e-05)
};

\addplot coordinates{
(13,5.755e-02)     (97,2.925e-02)     (545,1.351e-02)
(2561,5.842e-03)   (10625,2.397e-03)  (40193,9.414e-04)
(141569,3.564e-04) (471041,1.308e-04) 
(1496065,4.670e-05)
};
}%
\maketitle
\begin{abstract}%
\PGFPlots\ draws high--quality function plots in normal or logarithmic scaling with a user-friendly interface directly in \TeX. The user supplies axis labels, legend entries and the plot coordinates for one or more plots and \PGFPlots\ applies axis scaling, computes any logarithms and axis ticks and draws the plots, supporting line plots, scatter plots, piecewise constant plots, bar plots, area plots, mesh-- and surface plots and some more. It is based on Till Tantau's package \PGF/\Tikz.
\end{abstract}
\tableofcontents
\section{Introduction}
This package provides tools to generate plots and labeled axes easily. It draws normal plots, logplots and semi-logplots, in two and three dimensions. Axis ticks, labels, legends (in case of multiple plots) can be added with key-value options. It can cycle through a set of predefined line/marker/color specifications. In summary, its purpose is to simplify the generation of high-quality function and/or data plots, and solving the problems of
\begin{itemize}
	\item consistency of document and font type and font size,
	\item direct use of \TeX\ math mode in axis descriptions,
	\item consistency of data and figures (no third party tool necessary),
	\item inter document consistency using preamble configurations and styles.
\end{itemize}
Although not necessary, separate |.pdf| or |.eps| graphics can be generated using the |external| library developed as part of \Tikz.

\PGFPlots\ is build completely on \Tikz/\PGF. Knowledge of \Tikz\ will simplify the work with \PGFPlots, although it is not required.

\section[About PGFPlots: Preliminaries]{About {\normalfont\PGFPlots}: Preliminaries}
This section contains information about upgrades, the team, the installation (in case you need to do it manually) and troubleshooting. You may skip it completely except for the upgrade remarks.

However, note that this library requires at least \PGF\ version $2.00$. At the time of this writing, many \TeX-distributions still contain the older \PGF\ version $1.18$, so it may be necessary to install a recent \PGF\ prior to using \PGFPlots.

\subsection{Components}
\PGFPlots\ comes with two components:
\begin{enumerate}
	\item the plotting component (which you are currently reading) and
	\item the \PGFPlotstable\ component which simplifies number formatting and postprocessing of numerical tables. It comes as a separate package and has its own manual \href{file:pgfplotstable.pdf}{pgfplotstable.pdf}.
\end{enumerate}

\subsection{Upgrade remarks}
This release provides a lot of improvements which can be found in all detail in \texttt{ChangeLog} for interested readers. However, some attention is useful with respect to the following changes.

\subsubsection{New Optional Features}
\begin{enumerate}
	\item \PGFPlots\ 1.3 comes with user interface improvements. The technical distinction between ``behavior options'' and ``style options'' of older versions is no longer necessary (although still fully supported).

	\item \PGFPlots\ 1.3 has a new feature which allows to \emph{move axis labels tight to tick labels} automatically.

	Since this affects the spacing, it is not enabled be default.

	Use
\begin{codeexample}[code only]
\usepackage{pgfplots}
\pgfplotsset{compat=1.3}
\end{codeexample}
	in your preamble to benefit from the improved spacing\footnote{The |compat=1.3| setting is necessary for new features which move axis labels around like reversed axis. However, \PGFPlots\ will still work as it does in earlier versions, your documents will remain the same if you don't set it explicitly.}.
	Take a look at the next page for the precise definition of |compat|.

	\item \PGFPlots\ 1.3 now supports reversed axes. It is no longer necessary to use work--arounds with negative units.
\pgfkeys{/pdflinks/search key prefixes in/.add={/pgfplots/,}{}}

	Take a look at the |x dir=reverse| key.

	Existing work--arounds will still function properly. Use |\pgfplotsset{compat=1.3}| together with |x dir=reverse| to switch to the new version.
\end{enumerate}

\subsubsection{Old Features Which May Need Attention}
\begin{enumerate}
	\item The |scatter/classes| feature produces proper legends as of version 1.3. This may change the appearance of existing legends of plots with |scatter/classes|.

	\item Due to a small math bug in \PGF\ $2.00$, you \emph{can't} use the math expression `|-x^2|'. It is necessary to use `|0-x^2|' instead. The same holds for
		`|exp(-x^2)|'; use `|exp(0-x^2)|' instead.

		This will be fixed with \PGF\ $>2.00$.

	\item Starting with \PGFPlots\ $1.1$, |\tikzstyle| should \emph{no longer be used} to set \PGFPlots\ options.
	
	Although |\tikzstyle| is still supported for some older \PGFPlots\ options, you should replace any occurance of |\tikzstyle| with |\pgfplotsset{|\meta{style name}|/.style={|\meta{key-value-list}|}}| or the associated |/.append style| variant. See section~\ref{sec:styles} for more detail.
\end{enumerate}
I apologize for any inconvenience caused by these changes.

\begin{pgfplotskey}{compat=\mchoice{1.3,pre 1.3,default,newest} (initially default)}
	The preamble configuration 
\begin{codeexample}[code only]
\usepackage{pgfplots}
\pgfplotsset{compat=1.3}
\end{codeexample}
	allows to choose between backwards compatibility and most recent features.

	\PGFPlots\ version 1.3 comes with new features which may lead to different spacings compared to earlier versions:
	\begin{enumerate}
		\item Version 1.3 comes with the |xlabel near ticks| feature which places (in this case $x$) axis labels ``near'' the tick labels, taking the size of tick labels into account.
		
		Older versions used |xlabel absolute|, i.e.\ an absolute distance between the axis and axis labels (now matter how large or small tick labels are).

		The initial setting \emph{keeps backwards compatibility}.

		You are encouraged to use
\begin{codeexample}[code only]
\usepackage{pgfplots}
\pgfplotsset{compat=1.3}
\end{codeexample}
		in your preamble.
		
		\item Version 1.3 fixes a bug with |anchor=south west| which occurred mainly for reversed axes: the anchors where upside--down. This is fixed now.

		
		If you have written some work--around and want to keep it going, use

		|\pgfplotsset{compat/anchors=pre 1.3}|\pgfmanualpdflabel{/pgfplots/compat/anchors}{} 

		to disable the bug fix.
	\end{enumerate}

	Use |\pgfplotsset{compat=default}| to restore the factory settings.

	The setting |\pgfplotsset{compat=newest}| will always use features of the most recent version. This might result in small changes in the document's appearance.
\end{pgfplotskey}

\subsection{The Team}
\PGFPlots\ has been written mainly by Christian Feuers�nger with many improvements of Pascal Wolkotte and Nick Papior Andersen as a spare time project. We hope it is useful and provides valuable plots.

If you are interested in writing something but don't know how, consider reading the auxiliary manual \href{file:TeX-programming-notes.pdf}{TeX-programming-notes.pdf} which comes with \PGFPlots. It is far from complete, but maybe it is a good starting point (at least for more literature).

\subsection{Acknowledgements}
I thank God for all hours of enjoyed programming. I thank Pascal Wolkotte and Nick Papior Andersen for their programming efforts and contributions as part of the development team. I thank Stefan Tibus, who contributed the |plot shell| feature. I thank Tom Cashman for the contribution of the |reverse legend| feature. Special thanks go to Stefan Pinnow whose tests of \PGFPlots\ lead to numerous quality improvements. Furthermore, I thank Dr.~Schweitzer for many fruitful discussions and Dr.~Meine for his ideas and suggestions. Thanks as well to the many international contributors who provided feature requests or identified bugs or simply manual improvements!

Last but not least, I thank Till Tantau and Mark Wibrow for their excellent graphics (and more) package \PGF\ and \Tikz\ which is the base of \PGFPlots.

\include{pgfplots.install}%
\include{pgfplots.intro}%
\include{pgfplots.reference}%
\include{pgfplots.libs}%
\include{pgfplots.resources}%
\include{pgfplots.importexport}%
\include{pgfplots.basic.reference}%

\printindex

\bibliographystyle{abbrv} %gerapali} %gerabbrv} %gerunsrt.bst} %gerabbrv}% gerplain}
\nocite{pgfplotstable}
\nocite{programmingnotes}
\bibliography{pgfplots}
\end{document}
%
%%%%%%%%%%%%%%%%%%%%%%%%%%%%%%%%%%%%%%%%%%%%%%%%%%%%%%%%%%%%%%%%%%%%%%%%%%%%%
%
% Package pgfplots.sty documentation. 
%
% Copyright 2007/2008 by Christian Feuersaenger.
%
% This program is free software: you can redistribute it and/or modify
% it under the terms of the GNU General Public License as published by
% the Free Software Foundation, either version 3 of the License, or
% (at your option) any later version.
% 
% This program is distributed in the hope that it will be useful,
% but WITHOUT ANY WARRANTY; without even the implied warranty of
% MERCHANTABILITY or FITNESS FOR A PARTICULAR PURPOSE.  See the
% GNU General Public License for more details.
% 
% You should have received a copy of the GNU General Public License
% along with this program.  If not, see <http://www.gnu.org/licenses/>.
%
%
%%%%%%%%%%%%%%%%%%%%%%%%%%%%%%%%%%%%%%%%%%%%%%%%%%%%%%%%%%%%%%%%%%%%%%%%%%%%%
\input{pgfplots.preamble.tex}
%\RequirePackage[german,english,francais]{babel}

\pgfplotsmanualexternalexpensivetrue

%\usetikzlibrary{external}
%\tikzexternalize[prefix=figures/]{pgfplots}

\title{%
	Manual for Package \PGFPlots\\
	{\small Version \pgfplotsversion}\\
	{\small\href{http://sourceforge.net/projects/pgfplots}{http://sourceforge.net/projects/pgfplots}}}

%\includeonly{pgfplots.libs}


\begin{document}

\def\plotcoords{%
\addplot coordinates {
(5,8.312e-02)    (17,2.547e-02)   (49,7.407e-03)
(129,2.102e-03)  (321,5.874e-04)  (769,1.623e-04)
(1793,4.442e-05) (4097,1.207e-05) (9217,3.261e-06)
};

\addplot coordinates{
(7,8.472e-02)    (31,3.044e-02)    (111,1.022e-02)
(351,3.303e-03)  (1023,1.039e-03)  (2815,3.196e-04)
(7423,9.658e-05) (18943,2.873e-05) (47103,8.437e-06)
};

\addplot coordinates{
(9,7.881e-02)     (49,3.243e-02)    (209,1.232e-02)
(769,4.454e-03)   (2561,1.551e-03)  (7937,5.236e-04)
(23297,1.723e-04) (65537,5.545e-05) (178177,1.751e-05)
};

\addplot coordinates{
(11,6.887e-02)    (71,3.177e-02)     (351,1.341e-02)
(1471,5.334e-03)  (5503,2.027e-03)   (18943,7.415e-04)
(61183,2.628e-04) (187903,9.063e-05) (553983,3.053e-05)
};

\addplot coordinates{
(13,5.755e-02)     (97,2.925e-02)     (545,1.351e-02)
(2561,5.842e-03)   (10625,2.397e-03)  (40193,9.414e-04)
(141569,3.564e-04) (471041,1.308e-04) 
(1496065,4.670e-05)
};
}%
\maketitle
\begin{abstract}%
\PGFPlots\ draws high--quality function plots in normal or logarithmic scaling with a user-friendly interface directly in \TeX. The user supplies axis labels, legend entries and the plot coordinates for one or more plots and \PGFPlots\ applies axis scaling, computes any logarithms and axis ticks and draws the plots, supporting line plots, scatter plots, piecewise constant plots, bar plots, area plots, mesh-- and surface plots and some more. It is based on Till Tantau's package \PGF/\Tikz.
\end{abstract}
\tableofcontents
\section{Introduction}
This package provides tools to generate plots and labeled axes easily. It draws normal plots, logplots and semi-logplots, in two and three dimensions. Axis ticks, labels, legends (in case of multiple plots) can be added with key-value options. It can cycle through a set of predefined line/marker/color specifications. In summary, its purpose is to simplify the generation of high-quality function and/or data plots, and solving the problems of
\begin{itemize}
	\item consistency of document and font type and font size,
	\item direct use of \TeX\ math mode in axis descriptions,
	\item consistency of data and figures (no third party tool necessary),
	\item inter document consistency using preamble configurations and styles.
\end{itemize}
Although not necessary, separate |.pdf| or |.eps| graphics can be generated using the |external| library developed as part of \Tikz.

\PGFPlots\ is build completely on \Tikz/\PGF. Knowledge of \Tikz\ will simplify the work with \PGFPlots, although it is not required.

\section[About PGFPlots: Preliminaries]{About {\normalfont\PGFPlots}: Preliminaries}
This section contains information about upgrades, the team, the installation (in case you need to do it manually) and troubleshooting. You may skip it completely except for the upgrade remarks.

However, note that this library requires at least \PGF\ version $2.00$. At the time of this writing, many \TeX-distributions still contain the older \PGF\ version $1.18$, so it may be necessary to install a recent \PGF\ prior to using \PGFPlots.

\subsection{Components}
\PGFPlots\ comes with two components:
\begin{enumerate}
	\item the plotting component (which you are currently reading) and
	\item the \PGFPlotstable\ component which simplifies number formatting and postprocessing of numerical tables. It comes as a separate package and has its own manual \href{file:pgfplotstable.pdf}{pgfplotstable.pdf}.
\end{enumerate}

\subsection{Upgrade remarks}
This release provides a lot of improvements which can be found in all detail in \texttt{ChangeLog} for interested readers. However, some attention is useful with respect to the following changes.

\subsubsection{New Optional Features}
\begin{enumerate}
	\item \PGFPlots\ 1.3 comes with user interface improvements. The technical distinction between ``behavior options'' and ``style options'' of older versions is no longer necessary (although still fully supported).

	\item \PGFPlots\ 1.3 has a new feature which allows to \emph{move axis labels tight to tick labels} automatically.

	Since this affects the spacing, it is not enabled be default.

	Use
\begin{codeexample}[code only]
\usepackage{pgfplots}
\pgfplotsset{compat=1.3}
\end{codeexample}
	in your preamble to benefit from the improved spacing\footnote{The |compat=1.3| setting is necessary for new features which move axis labels around like reversed axis. However, \PGFPlots\ will still work as it does in earlier versions, your documents will remain the same if you don't set it explicitly.}.
	Take a look at the next page for the precise definition of |compat|.

	\item \PGFPlots\ 1.3 now supports reversed axes. It is no longer necessary to use work--arounds with negative units.
\pgfkeys{/pdflinks/search key prefixes in/.add={/pgfplots/,}{}}

	Take a look at the |x dir=reverse| key.

	Existing work--arounds will still function properly. Use |\pgfplotsset{compat=1.3}| together with |x dir=reverse| to switch to the new version.
\end{enumerate}

\subsubsection{Old Features Which May Need Attention}
\begin{enumerate}
	\item The |scatter/classes| feature produces proper legends as of version 1.3. This may change the appearance of existing legends of plots with |scatter/classes|.

	\item Due to a small math bug in \PGF\ $2.00$, you \emph{can't} use the math expression `|-x^2|'. It is necessary to use `|0-x^2|' instead. The same holds for
		`|exp(-x^2)|'; use `|exp(0-x^2)|' instead.

		This will be fixed with \PGF\ $>2.00$.

	\item Starting with \PGFPlots\ $1.1$, |\tikzstyle| should \emph{no longer be used} to set \PGFPlots\ options.
	
	Although |\tikzstyle| is still supported for some older \PGFPlots\ options, you should replace any occurance of |\tikzstyle| with |\pgfplotsset{|\meta{style name}|/.style={|\meta{key-value-list}|}}| or the associated |/.append style| variant. See section~\ref{sec:styles} for more detail.
\end{enumerate}
I apologize for any inconvenience caused by these changes.

\begin{pgfplotskey}{compat=\mchoice{1.3,pre 1.3,default,newest} (initially default)}
	The preamble configuration 
\begin{codeexample}[code only]
\usepackage{pgfplots}
\pgfplotsset{compat=1.3}
\end{codeexample}
	allows to choose between backwards compatibility and most recent features.

	\PGFPlots\ version 1.3 comes with new features which may lead to different spacings compared to earlier versions:
	\begin{enumerate}
		\item Version 1.3 comes with the |xlabel near ticks| feature which places (in this case $x$) axis labels ``near'' the tick labels, taking the size of tick labels into account.
		
		Older versions used |xlabel absolute|, i.e.\ an absolute distance between the axis and axis labels (now matter how large or small tick labels are).

		The initial setting \emph{keeps backwards compatibility}.

		You are encouraged to use
\begin{codeexample}[code only]
\usepackage{pgfplots}
\pgfplotsset{compat=1.3}
\end{codeexample}
		in your preamble.
		
		\item Version 1.3 fixes a bug with |anchor=south west| which occurred mainly for reversed axes: the anchors where upside--down. This is fixed now.

		
		If you have written some work--around and want to keep it going, use

		|\pgfplotsset{compat/anchors=pre 1.3}|\pgfmanualpdflabel{/pgfplots/compat/anchors}{} 

		to disable the bug fix.
	\end{enumerate}

	Use |\pgfplotsset{compat=default}| to restore the factory settings.

	The setting |\pgfplotsset{compat=newest}| will always use features of the most recent version. This might result in small changes in the document's appearance.
\end{pgfplotskey}

\subsection{The Team}
\PGFPlots\ has been written mainly by Christian Feuers�nger with many improvements of Pascal Wolkotte and Nick Papior Andersen as a spare time project. We hope it is useful and provides valuable plots.

If you are interested in writing something but don't know how, consider reading the auxiliary manual \href{file:TeX-programming-notes.pdf}{TeX-programming-notes.pdf} which comes with \PGFPlots. It is far from complete, but maybe it is a good starting point (at least for more literature).

\subsection{Acknowledgements}
I thank God for all hours of enjoyed programming. I thank Pascal Wolkotte and Nick Papior Andersen for their programming efforts and contributions as part of the development team. I thank Stefan Tibus, who contributed the |plot shell| feature. I thank Tom Cashman for the contribution of the |reverse legend| feature. Special thanks go to Stefan Pinnow whose tests of \PGFPlots\ lead to numerous quality improvements. Furthermore, I thank Dr.~Schweitzer for many fruitful discussions and Dr.~Meine for his ideas and suggestions. Thanks as well to the many international contributors who provided feature requests or identified bugs or simply manual improvements!

Last but not least, I thank Till Tantau and Mark Wibrow for their excellent graphics (and more) package \PGF\ and \Tikz\ which is the base of \PGFPlots.

\include{pgfplots.install}%
\include{pgfplots.intro}%
\include{pgfplots.reference}%
\include{pgfplots.libs}%
\include{pgfplots.resources}%
\include{pgfplots.importexport}%
\include{pgfplots.basic.reference}%

\printindex

\bibliographystyle{abbrv} %gerapali} %gerabbrv} %gerunsrt.bst} %gerabbrv}% gerplain}
\nocite{pgfplotstable}
\nocite{programmingnotes}
\bibliography{pgfplots}
\end{document}
%
%%%%%%%%%%%%%%%%%%%%%%%%%%%%%%%%%%%%%%%%%%%%%%%%%%%%%%%%%%%%%%%%%%%%%%%%%%%%%
%
% Package pgfplots.sty documentation. 
%
% Copyright 2007/2008 by Christian Feuersaenger.
%
% This program is free software: you can redistribute it and/or modify
% it under the terms of the GNU General Public License as published by
% the Free Software Foundation, either version 3 of the License, or
% (at your option) any later version.
% 
% This program is distributed in the hope that it will be useful,
% but WITHOUT ANY WARRANTY; without even the implied warranty of
% MERCHANTABILITY or FITNESS FOR A PARTICULAR PURPOSE.  See the
% GNU General Public License for more details.
% 
% You should have received a copy of the GNU General Public License
% along with this program.  If not, see <http://www.gnu.org/licenses/>.
%
%
%%%%%%%%%%%%%%%%%%%%%%%%%%%%%%%%%%%%%%%%%%%%%%%%%%%%%%%%%%%%%%%%%%%%%%%%%%%%%
\input{pgfplots.preamble.tex}
%\RequirePackage[german,english,francais]{babel}

\pgfplotsmanualexternalexpensivetrue

%\usetikzlibrary{external}
%\tikzexternalize[prefix=figures/]{pgfplots}

\title{%
	Manual for Package \PGFPlots\\
	{\small Version \pgfplotsversion}\\
	{\small\href{http://sourceforge.net/projects/pgfplots}{http://sourceforge.net/projects/pgfplots}}}

%\includeonly{pgfplots.libs}


\begin{document}

\def\plotcoords{%
\addplot coordinates {
(5,8.312e-02)    (17,2.547e-02)   (49,7.407e-03)
(129,2.102e-03)  (321,5.874e-04)  (769,1.623e-04)
(1793,4.442e-05) (4097,1.207e-05) (9217,3.261e-06)
};

\addplot coordinates{
(7,8.472e-02)    (31,3.044e-02)    (111,1.022e-02)
(351,3.303e-03)  (1023,1.039e-03)  (2815,3.196e-04)
(7423,9.658e-05) (18943,2.873e-05) (47103,8.437e-06)
};

\addplot coordinates{
(9,7.881e-02)     (49,3.243e-02)    (209,1.232e-02)
(769,4.454e-03)   (2561,1.551e-03)  (7937,5.236e-04)
(23297,1.723e-04) (65537,5.545e-05) (178177,1.751e-05)
};

\addplot coordinates{
(11,6.887e-02)    (71,3.177e-02)     (351,1.341e-02)
(1471,5.334e-03)  (5503,2.027e-03)   (18943,7.415e-04)
(61183,2.628e-04) (187903,9.063e-05) (553983,3.053e-05)
};

\addplot coordinates{
(13,5.755e-02)     (97,2.925e-02)     (545,1.351e-02)
(2561,5.842e-03)   (10625,2.397e-03)  (40193,9.414e-04)
(141569,3.564e-04) (471041,1.308e-04) 
(1496065,4.670e-05)
};
}%
\maketitle
\begin{abstract}%
\PGFPlots\ draws high--quality function plots in normal or logarithmic scaling with a user-friendly interface directly in \TeX. The user supplies axis labels, legend entries and the plot coordinates for one or more plots and \PGFPlots\ applies axis scaling, computes any logarithms and axis ticks and draws the plots, supporting line plots, scatter plots, piecewise constant plots, bar plots, area plots, mesh-- and surface plots and some more. It is based on Till Tantau's package \PGF/\Tikz.
\end{abstract}
\tableofcontents
\section{Introduction}
This package provides tools to generate plots and labeled axes easily. It draws normal plots, logplots and semi-logplots, in two and three dimensions. Axis ticks, labels, legends (in case of multiple plots) can be added with key-value options. It can cycle through a set of predefined line/marker/color specifications. In summary, its purpose is to simplify the generation of high-quality function and/or data plots, and solving the problems of
\begin{itemize}
	\item consistency of document and font type and font size,
	\item direct use of \TeX\ math mode in axis descriptions,
	\item consistency of data and figures (no third party tool necessary),
	\item inter document consistency using preamble configurations and styles.
\end{itemize}
Although not necessary, separate |.pdf| or |.eps| graphics can be generated using the |external| library developed as part of \Tikz.

\PGFPlots\ is build completely on \Tikz/\PGF. Knowledge of \Tikz\ will simplify the work with \PGFPlots, although it is not required.

\section[About PGFPlots: Preliminaries]{About {\normalfont\PGFPlots}: Preliminaries}
This section contains information about upgrades, the team, the installation (in case you need to do it manually) and troubleshooting. You may skip it completely except for the upgrade remarks.

However, note that this library requires at least \PGF\ version $2.00$. At the time of this writing, many \TeX-distributions still contain the older \PGF\ version $1.18$, so it may be necessary to install a recent \PGF\ prior to using \PGFPlots.

\subsection{Components}
\PGFPlots\ comes with two components:
\begin{enumerate}
	\item the plotting component (which you are currently reading) and
	\item the \PGFPlotstable\ component which simplifies number formatting and postprocessing of numerical tables. It comes as a separate package and has its own manual \href{file:pgfplotstable.pdf}{pgfplotstable.pdf}.
\end{enumerate}

\subsection{Upgrade remarks}
This release provides a lot of improvements which can be found in all detail in \texttt{ChangeLog} for interested readers. However, some attention is useful with respect to the following changes.

\subsubsection{New Optional Features}
\begin{enumerate}
	\item \PGFPlots\ 1.3 comes with user interface improvements. The technical distinction between ``behavior options'' and ``style options'' of older versions is no longer necessary (although still fully supported).

	\item \PGFPlots\ 1.3 has a new feature which allows to \emph{move axis labels tight to tick labels} automatically.

	Since this affects the spacing, it is not enabled be default.

	Use
\begin{codeexample}[code only]
\usepackage{pgfplots}
\pgfplotsset{compat=1.3}
\end{codeexample}
	in your preamble to benefit from the improved spacing\footnote{The |compat=1.3| setting is necessary for new features which move axis labels around like reversed axis. However, \PGFPlots\ will still work as it does in earlier versions, your documents will remain the same if you don't set it explicitly.}.
	Take a look at the next page for the precise definition of |compat|.

	\item \PGFPlots\ 1.3 now supports reversed axes. It is no longer necessary to use work--arounds with negative units.
\pgfkeys{/pdflinks/search key prefixes in/.add={/pgfplots/,}{}}

	Take a look at the |x dir=reverse| key.

	Existing work--arounds will still function properly. Use |\pgfplotsset{compat=1.3}| together with |x dir=reverse| to switch to the new version.
\end{enumerate}

\subsubsection{Old Features Which May Need Attention}
\begin{enumerate}
	\item The |scatter/classes| feature produces proper legends as of version 1.3. This may change the appearance of existing legends of plots with |scatter/classes|.

	\item Due to a small math bug in \PGF\ $2.00$, you \emph{can't} use the math expression `|-x^2|'. It is necessary to use `|0-x^2|' instead. The same holds for
		`|exp(-x^2)|'; use `|exp(0-x^2)|' instead.

		This will be fixed with \PGF\ $>2.00$.

	\item Starting with \PGFPlots\ $1.1$, |\tikzstyle| should \emph{no longer be used} to set \PGFPlots\ options.
	
	Although |\tikzstyle| is still supported for some older \PGFPlots\ options, you should replace any occurance of |\tikzstyle| with |\pgfplotsset{|\meta{style name}|/.style={|\meta{key-value-list}|}}| or the associated |/.append style| variant. See section~\ref{sec:styles} for more detail.
\end{enumerate}
I apologize for any inconvenience caused by these changes.

\begin{pgfplotskey}{compat=\mchoice{1.3,pre 1.3,default,newest} (initially default)}
	The preamble configuration 
\begin{codeexample}[code only]
\usepackage{pgfplots}
\pgfplotsset{compat=1.3}
\end{codeexample}
	allows to choose between backwards compatibility and most recent features.

	\PGFPlots\ version 1.3 comes with new features which may lead to different spacings compared to earlier versions:
	\begin{enumerate}
		\item Version 1.3 comes with the |xlabel near ticks| feature which places (in this case $x$) axis labels ``near'' the tick labels, taking the size of tick labels into account.
		
		Older versions used |xlabel absolute|, i.e.\ an absolute distance between the axis and axis labels (now matter how large or small tick labels are).

		The initial setting \emph{keeps backwards compatibility}.

		You are encouraged to use
\begin{codeexample}[code only]
\usepackage{pgfplots}
\pgfplotsset{compat=1.3}
\end{codeexample}
		in your preamble.
		
		\item Version 1.3 fixes a bug with |anchor=south west| which occurred mainly for reversed axes: the anchors where upside--down. This is fixed now.

		
		If you have written some work--around and want to keep it going, use

		|\pgfplotsset{compat/anchors=pre 1.3}|\pgfmanualpdflabel{/pgfplots/compat/anchors}{} 

		to disable the bug fix.
	\end{enumerate}

	Use |\pgfplotsset{compat=default}| to restore the factory settings.

	The setting |\pgfplotsset{compat=newest}| will always use features of the most recent version. This might result in small changes in the document's appearance.
\end{pgfplotskey}

\subsection{The Team}
\PGFPlots\ has been written mainly by Christian Feuers�nger with many improvements of Pascal Wolkotte and Nick Papior Andersen as a spare time project. We hope it is useful and provides valuable plots.

If you are interested in writing something but don't know how, consider reading the auxiliary manual \href{file:TeX-programming-notes.pdf}{TeX-programming-notes.pdf} which comes with \PGFPlots. It is far from complete, but maybe it is a good starting point (at least for more literature).

\subsection{Acknowledgements}
I thank God for all hours of enjoyed programming. I thank Pascal Wolkotte and Nick Papior Andersen for their programming efforts and contributions as part of the development team. I thank Stefan Tibus, who contributed the |plot shell| feature. I thank Tom Cashman for the contribution of the |reverse legend| feature. Special thanks go to Stefan Pinnow whose tests of \PGFPlots\ lead to numerous quality improvements. Furthermore, I thank Dr.~Schweitzer for many fruitful discussions and Dr.~Meine for his ideas and suggestions. Thanks as well to the many international contributors who provided feature requests or identified bugs or simply manual improvements!

Last but not least, I thank Till Tantau and Mark Wibrow for their excellent graphics (and more) package \PGF\ and \Tikz\ which is the base of \PGFPlots.

\include{pgfplots.install}%
\include{pgfplots.intro}%
\include{pgfplots.reference}%
\include{pgfplots.libs}%
\include{pgfplots.resources}%
\include{pgfplots.importexport}%
\include{pgfplots.basic.reference}%

\printindex

\bibliographystyle{abbrv} %gerapali} %gerabbrv} %gerunsrt.bst} %gerabbrv}% gerplain}
\nocite{pgfplotstable}
\nocite{programmingnotes}
\bibliography{pgfplots}
\end{document}
%
\include{pgfplots.basic.reference}%

\printindex

\bibliographystyle{abbrv} %gerapali} %gerabbrv} %gerunsrt.bst} %gerabbrv}% gerplain}
\nocite{pgfplotstable}
\nocite{programmingnotes}
\bibliography{pgfplots}
\end{document}
%
%%%%%%%%%%%%%%%%%%%%%%%%%%%%%%%%%%%%%%%%%%%%%%%%%%%%%%%%%%%%%%%%%%%%%%%%%%%%%
%
% Package pgfplots.sty documentation. 
%
% Copyright 2007/2008 by Christian Feuersaenger.
%
% This program is free software: you can redistribute it and/or modify
% it under the terms of the GNU General Public License as published by
% the Free Software Foundation, either version 3 of the License, or
% (at your option) any later version.
% 
% This program is distributed in the hope that it will be useful,
% but WITHOUT ANY WARRANTY; without even the implied warranty of
% MERCHANTABILITY or FITNESS FOR A PARTICULAR PURPOSE.  See the
% GNU General Public License for more details.
% 
% You should have received a copy of the GNU General Public License
% along with this program.  If not, see <http://www.gnu.org/licenses/>.
%
%
%%%%%%%%%%%%%%%%%%%%%%%%%%%%%%%%%%%%%%%%%%%%%%%%%%%%%%%%%%%%%%%%%%%%%%%%%%%%%
%%%%%%%%%%%%%%%%%%%%%%%%%%%%%%%%%%%%%%%%%%%%%%%%%%%%%%%%%%%%%%%%%%%%%%%%%%%%%
%
% Package pgfplots.sty documentation. 
%
% Copyright 2007/2008 by Christian Feuersaenger.
%
% This program is free software: you can redistribute it and/or modify
% it under the terms of the GNU General Public License as published by
% the Free Software Foundation, either version 3 of the License, or
% (at your option) any later version.
% 
% This program is distributed in the hope that it will be useful,
% but WITHOUT ANY WARRANTY; without even the implied warranty of
% MERCHANTABILITY or FITNESS FOR A PARTICULAR PURPOSE.  See the
% GNU General Public License for more details.
% 
% You should have received a copy of the GNU General Public License
% along with this program.  If not, see <http://www.gnu.org/licenses/>.
%
%
%%%%%%%%%%%%%%%%%%%%%%%%%%%%%%%%%%%%%%%%%%%%%%%%%%%%%%%%%%%%%%%%%%%%%%%%%%%%%
\documentclass[a4paper]{ltxdoc}

\usepackage{makeidx}

% DON't let hyperref overload the format of index and glossary. 
% I want to do that on my own in the stylefiles for makeindex...
\makeatletter
\let\@old@wrindex=\@wrindex
\makeatother

\usepackage{ifpdf}
\ifpdf
	\usepackage[pdftex]{hyperref}
\else
	\usepackage[dvipdfm]{hyperref}
\fi
\hypersetup{pdfborder={0 0 0}}

\makeatletter
\let\@wrindex=\@old@wrindex
\makeatother

% Formatiere Seitennummern im Index:
\newcommand{\indexpageno}[1]{%
	{\bfseries\hyperpage{#1}}%
}


\newcommand{\R}{\mathbb{R}}
\newcommand{\N}{\mathbb{N}}
\newcommand{\Z}{\mathbb{Z}}

\long\def\COMMENTLOWLEVEL#1\ENDCOMMENT{}
\def\ENDCOMMENT{}

\usepackage{textcomp}
\usepackage{booktabs}

\usepackage{calc}
\usepackage[formats]{listings}
%\usepackage{courier} % don't use it - the '^' character can't be copy-pasted in courier

\usepackage{array}
\lstset{%
	basicstyle=\ttfamily,
	language=[LaTeX]tex, % Seems as if \lstset{language=tex} must be invoked BEFORE loading tikz!?
	tabsize=4,
	breaklines=true,
	breakindent=0pt
}

\ifpdf
	\pdfinfo {
		/Author	(Christian Feuersaenger)
	}

\else
	\def\pgfsysdriver{pgfsys-dvipdfm.def}
\fi
%\def\pgfsysdriver{pgfsys-pdftex.def}
\usepackage{pgfplots}

\ifpdf
	\usetikzlibrary{pgfplotsclickable}
\fi

%\usepackage{fp}
% ATTENTION:
% this requires pgf version NEWER than 2.00 :
%\usetikzlibrary{fixedpointarithmetic}

\usetikzlibrary{dateplot}

\usepackage[a4paper,left=2.25cm,right=2.25cm,top=2.5cm,bottom=2.5cm,nohead]{geometry}
\usepackage{amsmath,amssymb}
\usepackage{xxcolor}
\usepackage{pifont}
\usepackage[latin1]{inputenc}
\usepackage{amsmath}
\usepackage{eurosym}
\input{pgfplots-macros}

\pgfqkeys{/codeexample}{%
	every codeexample/.style={
		width=8cm,
		/pgfplots/every axis/.append style={legend style={fill=graphicbackground}}
	},
	tabsize=4,
}
\pgfplotsset{
	every axis/.append style={width=7cm},
	filter discard warning=false,
}

\usetikzlibrary{backgrounds,patterns}
% Global styles:
\tikzset{
  shape example/.style={
    color=black!30,
    draw,
    fill=yellow!30,
    line width=.5cm,
    inner xsep=2.5cm,
    inner ysep=0.5cm}
}

\newcommand{\FIXME}[1]{\textcolor{red}{(FIXME: #1)}}

% fuer endvironment 'sidewaysfigure' bspw
% \usepackage{rotating}

\newcommand\Tikz{Ti\textit kZ}
\newcommand\PGF{\textsc{pgf}}
\newcommand\PGFPlots{\textsc{pgfplots}}
\def\PGFPlotstable{\textsc{PgfplotsTable}}


\makeindex

% Fix overful hboxes automatically:
\tolerance=2000
\emergencystretch=10pt

\tikzset{prefix=gnuplot/pgfplots_} % prefix for 'plot function'

\author{%
	Christian Feuers\"anger\footnote{\url{http://wissrech.ins.uni-bonn.de/people/feuersaenger}}\\%
	Institut f\"ur Numerische Simulation\\
	Universit\"at Bonn, Germany}


%\RequirePackage[german,english,francais]{babel}

\pgfplotsmanualexternalexpensivetrue

%\usetikzlibrary{external}
%\tikzexternalize[prefix=figures/]{pgfplots}

\title{%
	Manual for Package \PGFPlots\\
	{\small Version \pgfplotsversion}\\
	{\small\href{http://sourceforge.net/projects/pgfplots}{http://sourceforge.net/projects/pgfplots}}}

%\includeonly{pgfplots.libs}


\begin{document}

\def\plotcoords{%
\addplot coordinates {
(5,8.312e-02)    (17,2.547e-02)   (49,7.407e-03)
(129,2.102e-03)  (321,5.874e-04)  (769,1.623e-04)
(1793,4.442e-05) (4097,1.207e-05) (9217,3.261e-06)
};

\addplot coordinates{
(7,8.472e-02)    (31,3.044e-02)    (111,1.022e-02)
(351,3.303e-03)  (1023,1.039e-03)  (2815,3.196e-04)
(7423,9.658e-05) (18943,2.873e-05) (47103,8.437e-06)
};

\addplot coordinates{
(9,7.881e-02)     (49,3.243e-02)    (209,1.232e-02)
(769,4.454e-03)   (2561,1.551e-03)  (7937,5.236e-04)
(23297,1.723e-04) (65537,5.545e-05) (178177,1.751e-05)
};

\addplot coordinates{
(11,6.887e-02)    (71,3.177e-02)     (351,1.341e-02)
(1471,5.334e-03)  (5503,2.027e-03)   (18943,7.415e-04)
(61183,2.628e-04) (187903,9.063e-05) (553983,3.053e-05)
};

\addplot coordinates{
(13,5.755e-02)     (97,2.925e-02)     (545,1.351e-02)
(2561,5.842e-03)   (10625,2.397e-03)  (40193,9.414e-04)
(141569,3.564e-04) (471041,1.308e-04) 
(1496065,4.670e-05)
};
}%
\maketitle
\begin{abstract}%
\PGFPlots\ draws high--quality function plots in normal or logarithmic scaling with a user-friendly interface directly in \TeX. The user supplies axis labels, legend entries and the plot coordinates for one or more plots and \PGFPlots\ applies axis scaling, computes any logarithms and axis ticks and draws the plots, supporting line plots, scatter plots, piecewise constant plots, bar plots, area plots, mesh-- and surface plots and some more. It is based on Till Tantau's package \PGF/\Tikz.
\end{abstract}
\tableofcontents
\section{Introduction}
This package provides tools to generate plots and labeled axes easily. It draws normal plots, logplots and semi-logplots, in two and three dimensions. Axis ticks, labels, legends (in case of multiple plots) can be added with key-value options. It can cycle through a set of predefined line/marker/color specifications. In summary, its purpose is to simplify the generation of high-quality function and/or data plots, and solving the problems of
\begin{itemize}
	\item consistency of document and font type and font size,
	\item direct use of \TeX\ math mode in axis descriptions,
	\item consistency of data and figures (no third party tool necessary),
	\item inter document consistency using preamble configurations and styles.
\end{itemize}
Although not necessary, separate |.pdf| or |.eps| graphics can be generated using the |external| library developed as part of \Tikz.

\PGFPlots\ is build completely on \Tikz/\PGF. Knowledge of \Tikz\ will simplify the work with \PGFPlots, although it is not required.

\section[About PGFPlots: Preliminaries]{About {\normalfont\PGFPlots}: Preliminaries}
This section contains information about upgrades, the team, the installation (in case you need to do it manually) and troubleshooting. You may skip it completely except for the upgrade remarks.

However, note that this library requires at least \PGF\ version $2.00$. At the time of this writing, many \TeX-distributions still contain the older \PGF\ version $1.18$, so it may be necessary to install a recent \PGF\ prior to using \PGFPlots.

\subsection{Components}
\PGFPlots\ comes with two components:
\begin{enumerate}
	\item the plotting component (which you are currently reading) and
	\item the \PGFPlotstable\ component which simplifies number formatting and postprocessing of numerical tables. It comes as a separate package and has its own manual \href{file:pgfplotstable.pdf}{pgfplotstable.pdf}.
\end{enumerate}

\subsection{Upgrade remarks}
This release provides a lot of improvements which can be found in all detail in \texttt{ChangeLog} for interested readers. However, some attention is useful with respect to the following changes.

\subsubsection{New Optional Features}
\begin{enumerate}
	\item \PGFPlots\ 1.3 comes with user interface improvements. The technical distinction between ``behavior options'' and ``style options'' of older versions is no longer necessary (although still fully supported).

	\item \PGFPlots\ 1.3 has a new feature which allows to \emph{move axis labels tight to tick labels} automatically.

	Since this affects the spacing, it is not enabled be default.

	Use
\begin{codeexample}[code only]
\usepackage{pgfplots}
\pgfplotsset{compat=1.3}
\end{codeexample}
	in your preamble to benefit from the improved spacing\footnote{The |compat=1.3| setting is necessary for new features which move axis labels around like reversed axis. However, \PGFPlots\ will still work as it does in earlier versions, your documents will remain the same if you don't set it explicitly.}.
	Take a look at the next page for the precise definition of |compat|.

	\item \PGFPlots\ 1.3 now supports reversed axes. It is no longer necessary to use work--arounds with negative units.
\pgfkeys{/pdflinks/search key prefixes in/.add={/pgfplots/,}{}}

	Take a look at the |x dir=reverse| key.

	Existing work--arounds will still function properly. Use |\pgfplotsset{compat=1.3}| together with |x dir=reverse| to switch to the new version.
\end{enumerate}

\subsubsection{Old Features Which May Need Attention}
\begin{enumerate}
	\item The |scatter/classes| feature produces proper legends as of version 1.3. This may change the appearance of existing legends of plots with |scatter/classes|.

	\item Due to a small math bug in \PGF\ $2.00$, you \emph{can't} use the math expression `|-x^2|'. It is necessary to use `|0-x^2|' instead. The same holds for
		`|exp(-x^2)|'; use `|exp(0-x^2)|' instead.

		This will be fixed with \PGF\ $>2.00$.

	\item Starting with \PGFPlots\ $1.1$, |\tikzstyle| should \emph{no longer be used} to set \PGFPlots\ options.
	
	Although |\tikzstyle| is still supported for some older \PGFPlots\ options, you should replace any occurance of |\tikzstyle| with |\pgfplotsset{|\meta{style name}|/.style={|\meta{key-value-list}|}}| or the associated |/.append style| variant. See section~\ref{sec:styles} for more detail.
\end{enumerate}
I apologize for any inconvenience caused by these changes.

\begin{pgfplotskey}{compat=\mchoice{1.3,pre 1.3,default,newest} (initially default)}
	The preamble configuration 
\begin{codeexample}[code only]
\usepackage{pgfplots}
\pgfplotsset{compat=1.3}
\end{codeexample}
	allows to choose between backwards compatibility and most recent features.

	\PGFPlots\ version 1.3 comes with new features which may lead to different spacings compared to earlier versions:
	\begin{enumerate}
		\item Version 1.3 comes with the |xlabel near ticks| feature which places (in this case $x$) axis labels ``near'' the tick labels, taking the size of tick labels into account.
		
		Older versions used |xlabel absolute|, i.e.\ an absolute distance between the axis and axis labels (now matter how large or small tick labels are).

		The initial setting \emph{keeps backwards compatibility}.

		You are encouraged to use
\begin{codeexample}[code only]
\usepackage{pgfplots}
\pgfplotsset{compat=1.3}
\end{codeexample}
		in your preamble.
		
		\item Version 1.3 fixes a bug with |anchor=south west| which occurred mainly for reversed axes: the anchors where upside--down. This is fixed now.

		
		If you have written some work--around and want to keep it going, use

		|\pgfplotsset{compat/anchors=pre 1.3}|\pgfmanualpdflabel{/pgfplots/compat/anchors}{} 

		to disable the bug fix.
	\end{enumerate}

	Use |\pgfplotsset{compat=default}| to restore the factory settings.

	The setting |\pgfplotsset{compat=newest}| will always use features of the most recent version. This might result in small changes in the document's appearance.
\end{pgfplotskey}

\subsection{The Team}
\PGFPlots\ has been written mainly by Christian Feuers�nger with many improvements of Pascal Wolkotte and Nick Papior Andersen as a spare time project. We hope it is useful and provides valuable plots.

If you are interested in writing something but don't know how, consider reading the auxiliary manual \href{file:TeX-programming-notes.pdf}{TeX-programming-notes.pdf} which comes with \PGFPlots. It is far from complete, but maybe it is a good starting point (at least for more literature).

\subsection{Acknowledgements}
I thank God for all hours of enjoyed programming. I thank Pascal Wolkotte and Nick Papior Andersen for their programming efforts and contributions as part of the development team. I thank Stefan Tibus, who contributed the |plot shell| feature. I thank Tom Cashman for the contribution of the |reverse legend| feature. Special thanks go to Stefan Pinnow whose tests of \PGFPlots\ lead to numerous quality improvements. Furthermore, I thank Dr.~Schweitzer for many fruitful discussions and Dr.~Meine for his ideas and suggestions. Thanks as well to the many international contributors who provided feature requests or identified bugs or simply manual improvements!

Last but not least, I thank Till Tantau and Mark Wibrow for their excellent graphics (and more) package \PGF\ and \Tikz\ which is the base of \PGFPlots.

%%%%%%%%%%%%%%%%%%%%%%%%%%%%%%%%%%%%%%%%%%%%%%%%%%%%%%%%%%%%%%%%%%%%%%%%%%%%%
%
% Package pgfplots.sty documentation. 
%
% Copyright 2007/2008 by Christian Feuersaenger.
%
% This program is free software: you can redistribute it and/or modify
% it under the terms of the GNU General Public License as published by
% the Free Software Foundation, either version 3 of the License, or
% (at your option) any later version.
% 
% This program is distributed in the hope that it will be useful,
% but WITHOUT ANY WARRANTY; without even the implied warranty of
% MERCHANTABILITY or FITNESS FOR A PARTICULAR PURPOSE.  See the
% GNU General Public License for more details.
% 
% You should have received a copy of the GNU General Public License
% along with this program.  If not, see <http://www.gnu.org/licenses/>.
%
%
%%%%%%%%%%%%%%%%%%%%%%%%%%%%%%%%%%%%%%%%%%%%%%%%%%%%%%%%%%%%%%%%%%%%%%%%%%%%%
\input{pgfplots.preamble.tex}
%\RequirePackage[german,english,francais]{babel}

\pgfplotsmanualexternalexpensivetrue

%\usetikzlibrary{external}
%\tikzexternalize[prefix=figures/]{pgfplots}

\title{%
	Manual for Package \PGFPlots\\
	{\small Version \pgfplotsversion}\\
	{\small\href{http://sourceforge.net/projects/pgfplots}{http://sourceforge.net/projects/pgfplots}}}

%\includeonly{pgfplots.libs}


\begin{document}

\def\plotcoords{%
\addplot coordinates {
(5,8.312e-02)    (17,2.547e-02)   (49,7.407e-03)
(129,2.102e-03)  (321,5.874e-04)  (769,1.623e-04)
(1793,4.442e-05) (4097,1.207e-05) (9217,3.261e-06)
};

\addplot coordinates{
(7,8.472e-02)    (31,3.044e-02)    (111,1.022e-02)
(351,3.303e-03)  (1023,1.039e-03)  (2815,3.196e-04)
(7423,9.658e-05) (18943,2.873e-05) (47103,8.437e-06)
};

\addplot coordinates{
(9,7.881e-02)     (49,3.243e-02)    (209,1.232e-02)
(769,4.454e-03)   (2561,1.551e-03)  (7937,5.236e-04)
(23297,1.723e-04) (65537,5.545e-05) (178177,1.751e-05)
};

\addplot coordinates{
(11,6.887e-02)    (71,3.177e-02)     (351,1.341e-02)
(1471,5.334e-03)  (5503,2.027e-03)   (18943,7.415e-04)
(61183,2.628e-04) (187903,9.063e-05) (553983,3.053e-05)
};

\addplot coordinates{
(13,5.755e-02)     (97,2.925e-02)     (545,1.351e-02)
(2561,5.842e-03)   (10625,2.397e-03)  (40193,9.414e-04)
(141569,3.564e-04) (471041,1.308e-04) 
(1496065,4.670e-05)
};
}%
\maketitle
\begin{abstract}%
\PGFPlots\ draws high--quality function plots in normal or logarithmic scaling with a user-friendly interface directly in \TeX. The user supplies axis labels, legend entries and the plot coordinates for one or more plots and \PGFPlots\ applies axis scaling, computes any logarithms and axis ticks and draws the plots, supporting line plots, scatter plots, piecewise constant plots, bar plots, area plots, mesh-- and surface plots and some more. It is based on Till Tantau's package \PGF/\Tikz.
\end{abstract}
\tableofcontents
\section{Introduction}
This package provides tools to generate plots and labeled axes easily. It draws normal plots, logplots and semi-logplots, in two and three dimensions. Axis ticks, labels, legends (in case of multiple plots) can be added with key-value options. It can cycle through a set of predefined line/marker/color specifications. In summary, its purpose is to simplify the generation of high-quality function and/or data plots, and solving the problems of
\begin{itemize}
	\item consistency of document and font type and font size,
	\item direct use of \TeX\ math mode in axis descriptions,
	\item consistency of data and figures (no third party tool necessary),
	\item inter document consistency using preamble configurations and styles.
\end{itemize}
Although not necessary, separate |.pdf| or |.eps| graphics can be generated using the |external| library developed as part of \Tikz.

\PGFPlots\ is build completely on \Tikz/\PGF. Knowledge of \Tikz\ will simplify the work with \PGFPlots, although it is not required.

\section[About PGFPlots: Preliminaries]{About {\normalfont\PGFPlots}: Preliminaries}
This section contains information about upgrades, the team, the installation (in case you need to do it manually) and troubleshooting. You may skip it completely except for the upgrade remarks.

However, note that this library requires at least \PGF\ version $2.00$. At the time of this writing, many \TeX-distributions still contain the older \PGF\ version $1.18$, so it may be necessary to install a recent \PGF\ prior to using \PGFPlots.

\subsection{Components}
\PGFPlots\ comes with two components:
\begin{enumerate}
	\item the plotting component (which you are currently reading) and
	\item the \PGFPlotstable\ component which simplifies number formatting and postprocessing of numerical tables. It comes as a separate package and has its own manual \href{file:pgfplotstable.pdf}{pgfplotstable.pdf}.
\end{enumerate}

\subsection{Upgrade remarks}
This release provides a lot of improvements which can be found in all detail in \texttt{ChangeLog} for interested readers. However, some attention is useful with respect to the following changes.

\subsubsection{New Optional Features}
\begin{enumerate}
	\item \PGFPlots\ 1.3 comes with user interface improvements. The technical distinction between ``behavior options'' and ``style options'' of older versions is no longer necessary (although still fully supported).

	\item \PGFPlots\ 1.3 has a new feature which allows to \emph{move axis labels tight to tick labels} automatically.

	Since this affects the spacing, it is not enabled be default.

	Use
\begin{codeexample}[code only]
\usepackage{pgfplots}
\pgfplotsset{compat=1.3}
\end{codeexample}
	in your preamble to benefit from the improved spacing\footnote{The |compat=1.3| setting is necessary for new features which move axis labels around like reversed axis. However, \PGFPlots\ will still work as it does in earlier versions, your documents will remain the same if you don't set it explicitly.}.
	Take a look at the next page for the precise definition of |compat|.

	\item \PGFPlots\ 1.3 now supports reversed axes. It is no longer necessary to use work--arounds with negative units.
\pgfkeys{/pdflinks/search key prefixes in/.add={/pgfplots/,}{}}

	Take a look at the |x dir=reverse| key.

	Existing work--arounds will still function properly. Use |\pgfplotsset{compat=1.3}| together with |x dir=reverse| to switch to the new version.
\end{enumerate}

\subsubsection{Old Features Which May Need Attention}
\begin{enumerate}
	\item The |scatter/classes| feature produces proper legends as of version 1.3. This may change the appearance of existing legends of plots with |scatter/classes|.

	\item Due to a small math bug in \PGF\ $2.00$, you \emph{can't} use the math expression `|-x^2|'. It is necessary to use `|0-x^2|' instead. The same holds for
		`|exp(-x^2)|'; use `|exp(0-x^2)|' instead.

		This will be fixed with \PGF\ $>2.00$.

	\item Starting with \PGFPlots\ $1.1$, |\tikzstyle| should \emph{no longer be used} to set \PGFPlots\ options.
	
	Although |\tikzstyle| is still supported for some older \PGFPlots\ options, you should replace any occurance of |\tikzstyle| with |\pgfplotsset{|\meta{style name}|/.style={|\meta{key-value-list}|}}| or the associated |/.append style| variant. See section~\ref{sec:styles} for more detail.
\end{enumerate}
I apologize for any inconvenience caused by these changes.

\begin{pgfplotskey}{compat=\mchoice{1.3,pre 1.3,default,newest} (initially default)}
	The preamble configuration 
\begin{codeexample}[code only]
\usepackage{pgfplots}
\pgfplotsset{compat=1.3}
\end{codeexample}
	allows to choose between backwards compatibility and most recent features.

	\PGFPlots\ version 1.3 comes with new features which may lead to different spacings compared to earlier versions:
	\begin{enumerate}
		\item Version 1.3 comes with the |xlabel near ticks| feature which places (in this case $x$) axis labels ``near'' the tick labels, taking the size of tick labels into account.
		
		Older versions used |xlabel absolute|, i.e.\ an absolute distance between the axis and axis labels (now matter how large or small tick labels are).

		The initial setting \emph{keeps backwards compatibility}.

		You are encouraged to use
\begin{codeexample}[code only]
\usepackage{pgfplots}
\pgfplotsset{compat=1.3}
\end{codeexample}
		in your preamble.
		
		\item Version 1.3 fixes a bug with |anchor=south west| which occurred mainly for reversed axes: the anchors where upside--down. This is fixed now.

		
		If you have written some work--around and want to keep it going, use

		|\pgfplotsset{compat/anchors=pre 1.3}|\pgfmanualpdflabel{/pgfplots/compat/anchors}{} 

		to disable the bug fix.
	\end{enumerate}

	Use |\pgfplotsset{compat=default}| to restore the factory settings.

	The setting |\pgfplotsset{compat=newest}| will always use features of the most recent version. This might result in small changes in the document's appearance.
\end{pgfplotskey}

\subsection{The Team}
\PGFPlots\ has been written mainly by Christian Feuers�nger with many improvements of Pascal Wolkotte and Nick Papior Andersen as a spare time project. We hope it is useful and provides valuable plots.

If you are interested in writing something but don't know how, consider reading the auxiliary manual \href{file:TeX-programming-notes.pdf}{TeX-programming-notes.pdf} which comes with \PGFPlots. It is far from complete, but maybe it is a good starting point (at least for more literature).

\subsection{Acknowledgements}
I thank God for all hours of enjoyed programming. I thank Pascal Wolkotte and Nick Papior Andersen for their programming efforts and contributions as part of the development team. I thank Stefan Tibus, who contributed the |plot shell| feature. I thank Tom Cashman for the contribution of the |reverse legend| feature. Special thanks go to Stefan Pinnow whose tests of \PGFPlots\ lead to numerous quality improvements. Furthermore, I thank Dr.~Schweitzer for many fruitful discussions and Dr.~Meine for his ideas and suggestions. Thanks as well to the many international contributors who provided feature requests or identified bugs or simply manual improvements!

Last but not least, I thank Till Tantau and Mark Wibrow for their excellent graphics (and more) package \PGF\ and \Tikz\ which is the base of \PGFPlots.

\include{pgfplots.install}%
\include{pgfplots.intro}%
\include{pgfplots.reference}%
\include{pgfplots.libs}%
\include{pgfplots.resources}%
\include{pgfplots.importexport}%
\include{pgfplots.basic.reference}%

\printindex

\bibliographystyle{abbrv} %gerapali} %gerabbrv} %gerunsrt.bst} %gerabbrv}% gerplain}
\nocite{pgfplotstable}
\nocite{programmingnotes}
\bibliography{pgfplots}
\end{document}
%
%%%%%%%%%%%%%%%%%%%%%%%%%%%%%%%%%%%%%%%%%%%%%%%%%%%%%%%%%%%%%%%%%%%%%%%%%%%%%
%
% Package pgfplots.sty documentation. 
%
% Copyright 2007/2008 by Christian Feuersaenger.
%
% This program is free software: you can redistribute it and/or modify
% it under the terms of the GNU General Public License as published by
% the Free Software Foundation, either version 3 of the License, or
% (at your option) any later version.
% 
% This program is distributed in the hope that it will be useful,
% but WITHOUT ANY WARRANTY; without even the implied warranty of
% MERCHANTABILITY or FITNESS FOR A PARTICULAR PURPOSE.  See the
% GNU General Public License for more details.
% 
% You should have received a copy of the GNU General Public License
% along with this program.  If not, see <http://www.gnu.org/licenses/>.
%
%
%%%%%%%%%%%%%%%%%%%%%%%%%%%%%%%%%%%%%%%%%%%%%%%%%%%%%%%%%%%%%%%%%%%%%%%%%%%%%
\input{pgfplots.preamble.tex}
%\RequirePackage[german,english,francais]{babel}

\pgfplotsmanualexternalexpensivetrue

%\usetikzlibrary{external}
%\tikzexternalize[prefix=figures/]{pgfplots}

\title{%
	Manual for Package \PGFPlots\\
	{\small Version \pgfplotsversion}\\
	{\small\href{http://sourceforge.net/projects/pgfplots}{http://sourceforge.net/projects/pgfplots}}}

%\includeonly{pgfplots.libs}


\begin{document}

\def\plotcoords{%
\addplot coordinates {
(5,8.312e-02)    (17,2.547e-02)   (49,7.407e-03)
(129,2.102e-03)  (321,5.874e-04)  (769,1.623e-04)
(1793,4.442e-05) (4097,1.207e-05) (9217,3.261e-06)
};

\addplot coordinates{
(7,8.472e-02)    (31,3.044e-02)    (111,1.022e-02)
(351,3.303e-03)  (1023,1.039e-03)  (2815,3.196e-04)
(7423,9.658e-05) (18943,2.873e-05) (47103,8.437e-06)
};

\addplot coordinates{
(9,7.881e-02)     (49,3.243e-02)    (209,1.232e-02)
(769,4.454e-03)   (2561,1.551e-03)  (7937,5.236e-04)
(23297,1.723e-04) (65537,5.545e-05) (178177,1.751e-05)
};

\addplot coordinates{
(11,6.887e-02)    (71,3.177e-02)     (351,1.341e-02)
(1471,5.334e-03)  (5503,2.027e-03)   (18943,7.415e-04)
(61183,2.628e-04) (187903,9.063e-05) (553983,3.053e-05)
};

\addplot coordinates{
(13,5.755e-02)     (97,2.925e-02)     (545,1.351e-02)
(2561,5.842e-03)   (10625,2.397e-03)  (40193,9.414e-04)
(141569,3.564e-04) (471041,1.308e-04) 
(1496065,4.670e-05)
};
}%
\maketitle
\begin{abstract}%
\PGFPlots\ draws high--quality function plots in normal or logarithmic scaling with a user-friendly interface directly in \TeX. The user supplies axis labels, legend entries and the plot coordinates for one or more plots and \PGFPlots\ applies axis scaling, computes any logarithms and axis ticks and draws the plots, supporting line plots, scatter plots, piecewise constant plots, bar plots, area plots, mesh-- and surface plots and some more. It is based on Till Tantau's package \PGF/\Tikz.
\end{abstract}
\tableofcontents
\section{Introduction}
This package provides tools to generate plots and labeled axes easily. It draws normal plots, logplots and semi-logplots, in two and three dimensions. Axis ticks, labels, legends (in case of multiple plots) can be added with key-value options. It can cycle through a set of predefined line/marker/color specifications. In summary, its purpose is to simplify the generation of high-quality function and/or data plots, and solving the problems of
\begin{itemize}
	\item consistency of document and font type and font size,
	\item direct use of \TeX\ math mode in axis descriptions,
	\item consistency of data and figures (no third party tool necessary),
	\item inter document consistency using preamble configurations and styles.
\end{itemize}
Although not necessary, separate |.pdf| or |.eps| graphics can be generated using the |external| library developed as part of \Tikz.

\PGFPlots\ is build completely on \Tikz/\PGF. Knowledge of \Tikz\ will simplify the work with \PGFPlots, although it is not required.

\section[About PGFPlots: Preliminaries]{About {\normalfont\PGFPlots}: Preliminaries}
This section contains information about upgrades, the team, the installation (in case you need to do it manually) and troubleshooting. You may skip it completely except for the upgrade remarks.

However, note that this library requires at least \PGF\ version $2.00$. At the time of this writing, many \TeX-distributions still contain the older \PGF\ version $1.18$, so it may be necessary to install a recent \PGF\ prior to using \PGFPlots.

\subsection{Components}
\PGFPlots\ comes with two components:
\begin{enumerate}
	\item the plotting component (which you are currently reading) and
	\item the \PGFPlotstable\ component which simplifies number formatting and postprocessing of numerical tables. It comes as a separate package and has its own manual \href{file:pgfplotstable.pdf}{pgfplotstable.pdf}.
\end{enumerate}

\subsection{Upgrade remarks}
This release provides a lot of improvements which can be found in all detail in \texttt{ChangeLog} for interested readers. However, some attention is useful with respect to the following changes.

\subsubsection{New Optional Features}
\begin{enumerate}
	\item \PGFPlots\ 1.3 comes with user interface improvements. The technical distinction between ``behavior options'' and ``style options'' of older versions is no longer necessary (although still fully supported).

	\item \PGFPlots\ 1.3 has a new feature which allows to \emph{move axis labels tight to tick labels} automatically.

	Since this affects the spacing, it is not enabled be default.

	Use
\begin{codeexample}[code only]
\usepackage{pgfplots}
\pgfplotsset{compat=1.3}
\end{codeexample}
	in your preamble to benefit from the improved spacing\footnote{The |compat=1.3| setting is necessary for new features which move axis labels around like reversed axis. However, \PGFPlots\ will still work as it does in earlier versions, your documents will remain the same if you don't set it explicitly.}.
	Take a look at the next page for the precise definition of |compat|.

	\item \PGFPlots\ 1.3 now supports reversed axes. It is no longer necessary to use work--arounds with negative units.
\pgfkeys{/pdflinks/search key prefixes in/.add={/pgfplots/,}{}}

	Take a look at the |x dir=reverse| key.

	Existing work--arounds will still function properly. Use |\pgfplotsset{compat=1.3}| together with |x dir=reverse| to switch to the new version.
\end{enumerate}

\subsubsection{Old Features Which May Need Attention}
\begin{enumerate}
	\item The |scatter/classes| feature produces proper legends as of version 1.3. This may change the appearance of existing legends of plots with |scatter/classes|.

	\item Due to a small math bug in \PGF\ $2.00$, you \emph{can't} use the math expression `|-x^2|'. It is necessary to use `|0-x^2|' instead. The same holds for
		`|exp(-x^2)|'; use `|exp(0-x^2)|' instead.

		This will be fixed with \PGF\ $>2.00$.

	\item Starting with \PGFPlots\ $1.1$, |\tikzstyle| should \emph{no longer be used} to set \PGFPlots\ options.
	
	Although |\tikzstyle| is still supported for some older \PGFPlots\ options, you should replace any occurance of |\tikzstyle| with |\pgfplotsset{|\meta{style name}|/.style={|\meta{key-value-list}|}}| or the associated |/.append style| variant. See section~\ref{sec:styles} for more detail.
\end{enumerate}
I apologize for any inconvenience caused by these changes.

\begin{pgfplotskey}{compat=\mchoice{1.3,pre 1.3,default,newest} (initially default)}
	The preamble configuration 
\begin{codeexample}[code only]
\usepackage{pgfplots}
\pgfplotsset{compat=1.3}
\end{codeexample}
	allows to choose between backwards compatibility and most recent features.

	\PGFPlots\ version 1.3 comes with new features which may lead to different spacings compared to earlier versions:
	\begin{enumerate}
		\item Version 1.3 comes with the |xlabel near ticks| feature which places (in this case $x$) axis labels ``near'' the tick labels, taking the size of tick labels into account.
		
		Older versions used |xlabel absolute|, i.e.\ an absolute distance between the axis and axis labels (now matter how large or small tick labels are).

		The initial setting \emph{keeps backwards compatibility}.

		You are encouraged to use
\begin{codeexample}[code only]
\usepackage{pgfplots}
\pgfplotsset{compat=1.3}
\end{codeexample}
		in your preamble.
		
		\item Version 1.3 fixes a bug with |anchor=south west| which occurred mainly for reversed axes: the anchors where upside--down. This is fixed now.

		
		If you have written some work--around and want to keep it going, use

		|\pgfplotsset{compat/anchors=pre 1.3}|\pgfmanualpdflabel{/pgfplots/compat/anchors}{} 

		to disable the bug fix.
	\end{enumerate}

	Use |\pgfplotsset{compat=default}| to restore the factory settings.

	The setting |\pgfplotsset{compat=newest}| will always use features of the most recent version. This might result in small changes in the document's appearance.
\end{pgfplotskey}

\subsection{The Team}
\PGFPlots\ has been written mainly by Christian Feuers�nger with many improvements of Pascal Wolkotte and Nick Papior Andersen as a spare time project. We hope it is useful and provides valuable plots.

If you are interested in writing something but don't know how, consider reading the auxiliary manual \href{file:TeX-programming-notes.pdf}{TeX-programming-notes.pdf} which comes with \PGFPlots. It is far from complete, but maybe it is a good starting point (at least for more literature).

\subsection{Acknowledgements}
I thank God for all hours of enjoyed programming. I thank Pascal Wolkotte and Nick Papior Andersen for their programming efforts and contributions as part of the development team. I thank Stefan Tibus, who contributed the |plot shell| feature. I thank Tom Cashman for the contribution of the |reverse legend| feature. Special thanks go to Stefan Pinnow whose tests of \PGFPlots\ lead to numerous quality improvements. Furthermore, I thank Dr.~Schweitzer for many fruitful discussions and Dr.~Meine for his ideas and suggestions. Thanks as well to the many international contributors who provided feature requests or identified bugs or simply manual improvements!

Last but not least, I thank Till Tantau and Mark Wibrow for their excellent graphics (and more) package \PGF\ and \Tikz\ which is the base of \PGFPlots.

\include{pgfplots.install}%
\include{pgfplots.intro}%
\include{pgfplots.reference}%
\include{pgfplots.libs}%
\include{pgfplots.resources}%
\include{pgfplots.importexport}%
\include{pgfplots.basic.reference}%

\printindex

\bibliographystyle{abbrv} %gerapali} %gerabbrv} %gerunsrt.bst} %gerabbrv}% gerplain}
\nocite{pgfplotstable}
\nocite{programmingnotes}
\bibliography{pgfplots}
\end{document}
%
%%%%%%%%%%%%%%%%%%%%%%%%%%%%%%%%%%%%%%%%%%%%%%%%%%%%%%%%%%%%%%%%%%%%%%%%%%%%%
%
% Package pgfplots.sty documentation. 
%
% Copyright 2007/2008 by Christian Feuersaenger.
%
% This program is free software: you can redistribute it and/or modify
% it under the terms of the GNU General Public License as published by
% the Free Software Foundation, either version 3 of the License, or
% (at your option) any later version.
% 
% This program is distributed in the hope that it will be useful,
% but WITHOUT ANY WARRANTY; without even the implied warranty of
% MERCHANTABILITY or FITNESS FOR A PARTICULAR PURPOSE.  See the
% GNU General Public License for more details.
% 
% You should have received a copy of the GNU General Public License
% along with this program.  If not, see <http://www.gnu.org/licenses/>.
%
%
%%%%%%%%%%%%%%%%%%%%%%%%%%%%%%%%%%%%%%%%%%%%%%%%%%%%%%%%%%%%%%%%%%%%%%%%%%%%%
\input{pgfplots.preamble.tex}
%\RequirePackage[german,english,francais]{babel}

\pgfplotsmanualexternalexpensivetrue

%\usetikzlibrary{external}
%\tikzexternalize[prefix=figures/]{pgfplots}

\title{%
	Manual for Package \PGFPlots\\
	{\small Version \pgfplotsversion}\\
	{\small\href{http://sourceforge.net/projects/pgfplots}{http://sourceforge.net/projects/pgfplots}}}

%\includeonly{pgfplots.libs}


\begin{document}

\def\plotcoords{%
\addplot coordinates {
(5,8.312e-02)    (17,2.547e-02)   (49,7.407e-03)
(129,2.102e-03)  (321,5.874e-04)  (769,1.623e-04)
(1793,4.442e-05) (4097,1.207e-05) (9217,3.261e-06)
};

\addplot coordinates{
(7,8.472e-02)    (31,3.044e-02)    (111,1.022e-02)
(351,3.303e-03)  (1023,1.039e-03)  (2815,3.196e-04)
(7423,9.658e-05) (18943,2.873e-05) (47103,8.437e-06)
};

\addplot coordinates{
(9,7.881e-02)     (49,3.243e-02)    (209,1.232e-02)
(769,4.454e-03)   (2561,1.551e-03)  (7937,5.236e-04)
(23297,1.723e-04) (65537,5.545e-05) (178177,1.751e-05)
};

\addplot coordinates{
(11,6.887e-02)    (71,3.177e-02)     (351,1.341e-02)
(1471,5.334e-03)  (5503,2.027e-03)   (18943,7.415e-04)
(61183,2.628e-04) (187903,9.063e-05) (553983,3.053e-05)
};

\addplot coordinates{
(13,5.755e-02)     (97,2.925e-02)     (545,1.351e-02)
(2561,5.842e-03)   (10625,2.397e-03)  (40193,9.414e-04)
(141569,3.564e-04) (471041,1.308e-04) 
(1496065,4.670e-05)
};
}%
\maketitle
\begin{abstract}%
\PGFPlots\ draws high--quality function plots in normal or logarithmic scaling with a user-friendly interface directly in \TeX. The user supplies axis labels, legend entries and the plot coordinates for one or more plots and \PGFPlots\ applies axis scaling, computes any logarithms and axis ticks and draws the plots, supporting line plots, scatter plots, piecewise constant plots, bar plots, area plots, mesh-- and surface plots and some more. It is based on Till Tantau's package \PGF/\Tikz.
\end{abstract}
\tableofcontents
\section{Introduction}
This package provides tools to generate plots and labeled axes easily. It draws normal plots, logplots and semi-logplots, in two and three dimensions. Axis ticks, labels, legends (in case of multiple plots) can be added with key-value options. It can cycle through a set of predefined line/marker/color specifications. In summary, its purpose is to simplify the generation of high-quality function and/or data plots, and solving the problems of
\begin{itemize}
	\item consistency of document and font type and font size,
	\item direct use of \TeX\ math mode in axis descriptions,
	\item consistency of data and figures (no third party tool necessary),
	\item inter document consistency using preamble configurations and styles.
\end{itemize}
Although not necessary, separate |.pdf| or |.eps| graphics can be generated using the |external| library developed as part of \Tikz.

\PGFPlots\ is build completely on \Tikz/\PGF. Knowledge of \Tikz\ will simplify the work with \PGFPlots, although it is not required.

\section[About PGFPlots: Preliminaries]{About {\normalfont\PGFPlots}: Preliminaries}
This section contains information about upgrades, the team, the installation (in case you need to do it manually) and troubleshooting. You may skip it completely except for the upgrade remarks.

However, note that this library requires at least \PGF\ version $2.00$. At the time of this writing, many \TeX-distributions still contain the older \PGF\ version $1.18$, so it may be necessary to install a recent \PGF\ prior to using \PGFPlots.

\subsection{Components}
\PGFPlots\ comes with two components:
\begin{enumerate}
	\item the plotting component (which you are currently reading) and
	\item the \PGFPlotstable\ component which simplifies number formatting and postprocessing of numerical tables. It comes as a separate package and has its own manual \href{file:pgfplotstable.pdf}{pgfplotstable.pdf}.
\end{enumerate}

\subsection{Upgrade remarks}
This release provides a lot of improvements which can be found in all detail in \texttt{ChangeLog} for interested readers. However, some attention is useful with respect to the following changes.

\subsubsection{New Optional Features}
\begin{enumerate}
	\item \PGFPlots\ 1.3 comes with user interface improvements. The technical distinction between ``behavior options'' and ``style options'' of older versions is no longer necessary (although still fully supported).

	\item \PGFPlots\ 1.3 has a new feature which allows to \emph{move axis labels tight to tick labels} automatically.

	Since this affects the spacing, it is not enabled be default.

	Use
\begin{codeexample}[code only]
\usepackage{pgfplots}
\pgfplotsset{compat=1.3}
\end{codeexample}
	in your preamble to benefit from the improved spacing\footnote{The |compat=1.3| setting is necessary for new features which move axis labels around like reversed axis. However, \PGFPlots\ will still work as it does in earlier versions, your documents will remain the same if you don't set it explicitly.}.
	Take a look at the next page for the precise definition of |compat|.

	\item \PGFPlots\ 1.3 now supports reversed axes. It is no longer necessary to use work--arounds with negative units.
\pgfkeys{/pdflinks/search key prefixes in/.add={/pgfplots/,}{}}

	Take a look at the |x dir=reverse| key.

	Existing work--arounds will still function properly. Use |\pgfplotsset{compat=1.3}| together with |x dir=reverse| to switch to the new version.
\end{enumerate}

\subsubsection{Old Features Which May Need Attention}
\begin{enumerate}
	\item The |scatter/classes| feature produces proper legends as of version 1.3. This may change the appearance of existing legends of plots with |scatter/classes|.

	\item Due to a small math bug in \PGF\ $2.00$, you \emph{can't} use the math expression `|-x^2|'. It is necessary to use `|0-x^2|' instead. The same holds for
		`|exp(-x^2)|'; use `|exp(0-x^2)|' instead.

		This will be fixed with \PGF\ $>2.00$.

	\item Starting with \PGFPlots\ $1.1$, |\tikzstyle| should \emph{no longer be used} to set \PGFPlots\ options.
	
	Although |\tikzstyle| is still supported for some older \PGFPlots\ options, you should replace any occurance of |\tikzstyle| with |\pgfplotsset{|\meta{style name}|/.style={|\meta{key-value-list}|}}| or the associated |/.append style| variant. See section~\ref{sec:styles} for more detail.
\end{enumerate}
I apologize for any inconvenience caused by these changes.

\begin{pgfplotskey}{compat=\mchoice{1.3,pre 1.3,default,newest} (initially default)}
	The preamble configuration 
\begin{codeexample}[code only]
\usepackage{pgfplots}
\pgfplotsset{compat=1.3}
\end{codeexample}
	allows to choose between backwards compatibility and most recent features.

	\PGFPlots\ version 1.3 comes with new features which may lead to different spacings compared to earlier versions:
	\begin{enumerate}
		\item Version 1.3 comes with the |xlabel near ticks| feature which places (in this case $x$) axis labels ``near'' the tick labels, taking the size of tick labels into account.
		
		Older versions used |xlabel absolute|, i.e.\ an absolute distance between the axis and axis labels (now matter how large or small tick labels are).

		The initial setting \emph{keeps backwards compatibility}.

		You are encouraged to use
\begin{codeexample}[code only]
\usepackage{pgfplots}
\pgfplotsset{compat=1.3}
\end{codeexample}
		in your preamble.
		
		\item Version 1.3 fixes a bug with |anchor=south west| which occurred mainly for reversed axes: the anchors where upside--down. This is fixed now.

		
		If you have written some work--around and want to keep it going, use

		|\pgfplotsset{compat/anchors=pre 1.3}|\pgfmanualpdflabel{/pgfplots/compat/anchors}{} 

		to disable the bug fix.
	\end{enumerate}

	Use |\pgfplotsset{compat=default}| to restore the factory settings.

	The setting |\pgfplotsset{compat=newest}| will always use features of the most recent version. This might result in small changes in the document's appearance.
\end{pgfplotskey}

\subsection{The Team}
\PGFPlots\ has been written mainly by Christian Feuers�nger with many improvements of Pascal Wolkotte and Nick Papior Andersen as a spare time project. We hope it is useful and provides valuable plots.

If you are interested in writing something but don't know how, consider reading the auxiliary manual \href{file:TeX-programming-notes.pdf}{TeX-programming-notes.pdf} which comes with \PGFPlots. It is far from complete, but maybe it is a good starting point (at least for more literature).

\subsection{Acknowledgements}
I thank God for all hours of enjoyed programming. I thank Pascal Wolkotte and Nick Papior Andersen for their programming efforts and contributions as part of the development team. I thank Stefan Tibus, who contributed the |plot shell| feature. I thank Tom Cashman for the contribution of the |reverse legend| feature. Special thanks go to Stefan Pinnow whose tests of \PGFPlots\ lead to numerous quality improvements. Furthermore, I thank Dr.~Schweitzer for many fruitful discussions and Dr.~Meine for his ideas and suggestions. Thanks as well to the many international contributors who provided feature requests or identified bugs or simply manual improvements!

Last but not least, I thank Till Tantau and Mark Wibrow for their excellent graphics (and more) package \PGF\ and \Tikz\ which is the base of \PGFPlots.

\include{pgfplots.install}%
\include{pgfplots.intro}%
\include{pgfplots.reference}%
\include{pgfplots.libs}%
\include{pgfplots.resources}%
\include{pgfplots.importexport}%
\include{pgfplots.basic.reference}%

\printindex

\bibliographystyle{abbrv} %gerapali} %gerabbrv} %gerunsrt.bst} %gerabbrv}% gerplain}
\nocite{pgfplotstable}
\nocite{programmingnotes}
\bibliography{pgfplots}
\end{document}
%
%%%%%%%%%%%%%%%%%%%%%%%%%%%%%%%%%%%%%%%%%%%%%%%%%%%%%%%%%%%%%%%%%%%%%%%%%%%%%
%
% Package pgfplots.sty documentation. 
%
% Copyright 2007/2008 by Christian Feuersaenger.
%
% This program is free software: you can redistribute it and/or modify
% it under the terms of the GNU General Public License as published by
% the Free Software Foundation, either version 3 of the License, or
% (at your option) any later version.
% 
% This program is distributed in the hope that it will be useful,
% but WITHOUT ANY WARRANTY; without even the implied warranty of
% MERCHANTABILITY or FITNESS FOR A PARTICULAR PURPOSE.  See the
% GNU General Public License for more details.
% 
% You should have received a copy of the GNU General Public License
% along with this program.  If not, see <http://www.gnu.org/licenses/>.
%
%
%%%%%%%%%%%%%%%%%%%%%%%%%%%%%%%%%%%%%%%%%%%%%%%%%%%%%%%%%%%%%%%%%%%%%%%%%%%%%
\input{pgfplots.preamble.tex}
%\RequirePackage[german,english,francais]{babel}

\pgfplotsmanualexternalexpensivetrue

%\usetikzlibrary{external}
%\tikzexternalize[prefix=figures/]{pgfplots}

\title{%
	Manual for Package \PGFPlots\\
	{\small Version \pgfplotsversion}\\
	{\small\href{http://sourceforge.net/projects/pgfplots}{http://sourceforge.net/projects/pgfplots}}}

%\includeonly{pgfplots.libs}


\begin{document}

\def\plotcoords{%
\addplot coordinates {
(5,8.312e-02)    (17,2.547e-02)   (49,7.407e-03)
(129,2.102e-03)  (321,5.874e-04)  (769,1.623e-04)
(1793,4.442e-05) (4097,1.207e-05) (9217,3.261e-06)
};

\addplot coordinates{
(7,8.472e-02)    (31,3.044e-02)    (111,1.022e-02)
(351,3.303e-03)  (1023,1.039e-03)  (2815,3.196e-04)
(7423,9.658e-05) (18943,2.873e-05) (47103,8.437e-06)
};

\addplot coordinates{
(9,7.881e-02)     (49,3.243e-02)    (209,1.232e-02)
(769,4.454e-03)   (2561,1.551e-03)  (7937,5.236e-04)
(23297,1.723e-04) (65537,5.545e-05) (178177,1.751e-05)
};

\addplot coordinates{
(11,6.887e-02)    (71,3.177e-02)     (351,1.341e-02)
(1471,5.334e-03)  (5503,2.027e-03)   (18943,7.415e-04)
(61183,2.628e-04) (187903,9.063e-05) (553983,3.053e-05)
};

\addplot coordinates{
(13,5.755e-02)     (97,2.925e-02)     (545,1.351e-02)
(2561,5.842e-03)   (10625,2.397e-03)  (40193,9.414e-04)
(141569,3.564e-04) (471041,1.308e-04) 
(1496065,4.670e-05)
};
}%
\maketitle
\begin{abstract}%
\PGFPlots\ draws high--quality function plots in normal or logarithmic scaling with a user-friendly interface directly in \TeX. The user supplies axis labels, legend entries and the plot coordinates for one or more plots and \PGFPlots\ applies axis scaling, computes any logarithms and axis ticks and draws the plots, supporting line plots, scatter plots, piecewise constant plots, bar plots, area plots, mesh-- and surface plots and some more. It is based on Till Tantau's package \PGF/\Tikz.
\end{abstract}
\tableofcontents
\section{Introduction}
This package provides tools to generate plots and labeled axes easily. It draws normal plots, logplots and semi-logplots, in two and three dimensions. Axis ticks, labels, legends (in case of multiple plots) can be added with key-value options. It can cycle through a set of predefined line/marker/color specifications. In summary, its purpose is to simplify the generation of high-quality function and/or data plots, and solving the problems of
\begin{itemize}
	\item consistency of document and font type and font size,
	\item direct use of \TeX\ math mode in axis descriptions,
	\item consistency of data and figures (no third party tool necessary),
	\item inter document consistency using preamble configurations and styles.
\end{itemize}
Although not necessary, separate |.pdf| or |.eps| graphics can be generated using the |external| library developed as part of \Tikz.

\PGFPlots\ is build completely on \Tikz/\PGF. Knowledge of \Tikz\ will simplify the work with \PGFPlots, although it is not required.

\section[About PGFPlots: Preliminaries]{About {\normalfont\PGFPlots}: Preliminaries}
This section contains information about upgrades, the team, the installation (in case you need to do it manually) and troubleshooting. You may skip it completely except for the upgrade remarks.

However, note that this library requires at least \PGF\ version $2.00$. At the time of this writing, many \TeX-distributions still contain the older \PGF\ version $1.18$, so it may be necessary to install a recent \PGF\ prior to using \PGFPlots.

\subsection{Components}
\PGFPlots\ comes with two components:
\begin{enumerate}
	\item the plotting component (which you are currently reading) and
	\item the \PGFPlotstable\ component which simplifies number formatting and postprocessing of numerical tables. It comes as a separate package and has its own manual \href{file:pgfplotstable.pdf}{pgfplotstable.pdf}.
\end{enumerate}

\subsection{Upgrade remarks}
This release provides a lot of improvements which can be found in all detail in \texttt{ChangeLog} for interested readers. However, some attention is useful with respect to the following changes.

\subsubsection{New Optional Features}
\begin{enumerate}
	\item \PGFPlots\ 1.3 comes with user interface improvements. The technical distinction between ``behavior options'' and ``style options'' of older versions is no longer necessary (although still fully supported).

	\item \PGFPlots\ 1.3 has a new feature which allows to \emph{move axis labels tight to tick labels} automatically.

	Since this affects the spacing, it is not enabled be default.

	Use
\begin{codeexample}[code only]
\usepackage{pgfplots}
\pgfplotsset{compat=1.3}
\end{codeexample}
	in your preamble to benefit from the improved spacing\footnote{The |compat=1.3| setting is necessary for new features which move axis labels around like reversed axis. However, \PGFPlots\ will still work as it does in earlier versions, your documents will remain the same if you don't set it explicitly.}.
	Take a look at the next page for the precise definition of |compat|.

	\item \PGFPlots\ 1.3 now supports reversed axes. It is no longer necessary to use work--arounds with negative units.
\pgfkeys{/pdflinks/search key prefixes in/.add={/pgfplots/,}{}}

	Take a look at the |x dir=reverse| key.

	Existing work--arounds will still function properly. Use |\pgfplotsset{compat=1.3}| together with |x dir=reverse| to switch to the new version.
\end{enumerate}

\subsubsection{Old Features Which May Need Attention}
\begin{enumerate}
	\item The |scatter/classes| feature produces proper legends as of version 1.3. This may change the appearance of existing legends of plots with |scatter/classes|.

	\item Due to a small math bug in \PGF\ $2.00$, you \emph{can't} use the math expression `|-x^2|'. It is necessary to use `|0-x^2|' instead. The same holds for
		`|exp(-x^2)|'; use `|exp(0-x^2)|' instead.

		This will be fixed with \PGF\ $>2.00$.

	\item Starting with \PGFPlots\ $1.1$, |\tikzstyle| should \emph{no longer be used} to set \PGFPlots\ options.
	
	Although |\tikzstyle| is still supported for some older \PGFPlots\ options, you should replace any occurance of |\tikzstyle| with |\pgfplotsset{|\meta{style name}|/.style={|\meta{key-value-list}|}}| or the associated |/.append style| variant. See section~\ref{sec:styles} for more detail.
\end{enumerate}
I apologize for any inconvenience caused by these changes.

\begin{pgfplotskey}{compat=\mchoice{1.3,pre 1.3,default,newest} (initially default)}
	The preamble configuration 
\begin{codeexample}[code only]
\usepackage{pgfplots}
\pgfplotsset{compat=1.3}
\end{codeexample}
	allows to choose between backwards compatibility and most recent features.

	\PGFPlots\ version 1.3 comes with new features which may lead to different spacings compared to earlier versions:
	\begin{enumerate}
		\item Version 1.3 comes with the |xlabel near ticks| feature which places (in this case $x$) axis labels ``near'' the tick labels, taking the size of tick labels into account.
		
		Older versions used |xlabel absolute|, i.e.\ an absolute distance between the axis and axis labels (now matter how large or small tick labels are).

		The initial setting \emph{keeps backwards compatibility}.

		You are encouraged to use
\begin{codeexample}[code only]
\usepackage{pgfplots}
\pgfplotsset{compat=1.3}
\end{codeexample}
		in your preamble.
		
		\item Version 1.3 fixes a bug with |anchor=south west| which occurred mainly for reversed axes: the anchors where upside--down. This is fixed now.

		
		If you have written some work--around and want to keep it going, use

		|\pgfplotsset{compat/anchors=pre 1.3}|\pgfmanualpdflabel{/pgfplots/compat/anchors}{} 

		to disable the bug fix.
	\end{enumerate}

	Use |\pgfplotsset{compat=default}| to restore the factory settings.

	The setting |\pgfplotsset{compat=newest}| will always use features of the most recent version. This might result in small changes in the document's appearance.
\end{pgfplotskey}

\subsection{The Team}
\PGFPlots\ has been written mainly by Christian Feuers�nger with many improvements of Pascal Wolkotte and Nick Papior Andersen as a spare time project. We hope it is useful and provides valuable plots.

If you are interested in writing something but don't know how, consider reading the auxiliary manual \href{file:TeX-programming-notes.pdf}{TeX-programming-notes.pdf} which comes with \PGFPlots. It is far from complete, but maybe it is a good starting point (at least for more literature).

\subsection{Acknowledgements}
I thank God for all hours of enjoyed programming. I thank Pascal Wolkotte and Nick Papior Andersen for their programming efforts and contributions as part of the development team. I thank Stefan Tibus, who contributed the |plot shell| feature. I thank Tom Cashman for the contribution of the |reverse legend| feature. Special thanks go to Stefan Pinnow whose tests of \PGFPlots\ lead to numerous quality improvements. Furthermore, I thank Dr.~Schweitzer for many fruitful discussions and Dr.~Meine for his ideas and suggestions. Thanks as well to the many international contributors who provided feature requests or identified bugs or simply manual improvements!

Last but not least, I thank Till Tantau and Mark Wibrow for their excellent graphics (and more) package \PGF\ and \Tikz\ which is the base of \PGFPlots.

\include{pgfplots.install}%
\include{pgfplots.intro}%
\include{pgfplots.reference}%
\include{pgfplots.libs}%
\include{pgfplots.resources}%
\include{pgfplots.importexport}%
\include{pgfplots.basic.reference}%

\printindex

\bibliographystyle{abbrv} %gerapali} %gerabbrv} %gerunsrt.bst} %gerabbrv}% gerplain}
\nocite{pgfplotstable}
\nocite{programmingnotes}
\bibliography{pgfplots}
\end{document}
%
%%%%%%%%%%%%%%%%%%%%%%%%%%%%%%%%%%%%%%%%%%%%%%%%%%%%%%%%%%%%%%%%%%%%%%%%%%%%%
%
% Package pgfplots.sty documentation. 
%
% Copyright 2007/2008 by Christian Feuersaenger.
%
% This program is free software: you can redistribute it and/or modify
% it under the terms of the GNU General Public License as published by
% the Free Software Foundation, either version 3 of the License, or
% (at your option) any later version.
% 
% This program is distributed in the hope that it will be useful,
% but WITHOUT ANY WARRANTY; without even the implied warranty of
% MERCHANTABILITY or FITNESS FOR A PARTICULAR PURPOSE.  See the
% GNU General Public License for more details.
% 
% You should have received a copy of the GNU General Public License
% along with this program.  If not, see <http://www.gnu.org/licenses/>.
%
%
%%%%%%%%%%%%%%%%%%%%%%%%%%%%%%%%%%%%%%%%%%%%%%%%%%%%%%%%%%%%%%%%%%%%%%%%%%%%%
\input{pgfplots.preamble.tex}
%\RequirePackage[german,english,francais]{babel}

\pgfplotsmanualexternalexpensivetrue

%\usetikzlibrary{external}
%\tikzexternalize[prefix=figures/]{pgfplots}

\title{%
	Manual for Package \PGFPlots\\
	{\small Version \pgfplotsversion}\\
	{\small\href{http://sourceforge.net/projects/pgfplots}{http://sourceforge.net/projects/pgfplots}}}

%\includeonly{pgfplots.libs}


\begin{document}

\def\plotcoords{%
\addplot coordinates {
(5,8.312e-02)    (17,2.547e-02)   (49,7.407e-03)
(129,2.102e-03)  (321,5.874e-04)  (769,1.623e-04)
(1793,4.442e-05) (4097,1.207e-05) (9217,3.261e-06)
};

\addplot coordinates{
(7,8.472e-02)    (31,3.044e-02)    (111,1.022e-02)
(351,3.303e-03)  (1023,1.039e-03)  (2815,3.196e-04)
(7423,9.658e-05) (18943,2.873e-05) (47103,8.437e-06)
};

\addplot coordinates{
(9,7.881e-02)     (49,3.243e-02)    (209,1.232e-02)
(769,4.454e-03)   (2561,1.551e-03)  (7937,5.236e-04)
(23297,1.723e-04) (65537,5.545e-05) (178177,1.751e-05)
};

\addplot coordinates{
(11,6.887e-02)    (71,3.177e-02)     (351,1.341e-02)
(1471,5.334e-03)  (5503,2.027e-03)   (18943,7.415e-04)
(61183,2.628e-04) (187903,9.063e-05) (553983,3.053e-05)
};

\addplot coordinates{
(13,5.755e-02)     (97,2.925e-02)     (545,1.351e-02)
(2561,5.842e-03)   (10625,2.397e-03)  (40193,9.414e-04)
(141569,3.564e-04) (471041,1.308e-04) 
(1496065,4.670e-05)
};
}%
\maketitle
\begin{abstract}%
\PGFPlots\ draws high--quality function plots in normal or logarithmic scaling with a user-friendly interface directly in \TeX. The user supplies axis labels, legend entries and the plot coordinates for one or more plots and \PGFPlots\ applies axis scaling, computes any logarithms and axis ticks and draws the plots, supporting line plots, scatter plots, piecewise constant plots, bar plots, area plots, mesh-- and surface plots and some more. It is based on Till Tantau's package \PGF/\Tikz.
\end{abstract}
\tableofcontents
\section{Introduction}
This package provides tools to generate plots and labeled axes easily. It draws normal plots, logplots and semi-logplots, in two and three dimensions. Axis ticks, labels, legends (in case of multiple plots) can be added with key-value options. It can cycle through a set of predefined line/marker/color specifications. In summary, its purpose is to simplify the generation of high-quality function and/or data plots, and solving the problems of
\begin{itemize}
	\item consistency of document and font type and font size,
	\item direct use of \TeX\ math mode in axis descriptions,
	\item consistency of data and figures (no third party tool necessary),
	\item inter document consistency using preamble configurations and styles.
\end{itemize}
Although not necessary, separate |.pdf| or |.eps| graphics can be generated using the |external| library developed as part of \Tikz.

\PGFPlots\ is build completely on \Tikz/\PGF. Knowledge of \Tikz\ will simplify the work with \PGFPlots, although it is not required.

\section[About PGFPlots: Preliminaries]{About {\normalfont\PGFPlots}: Preliminaries}
This section contains information about upgrades, the team, the installation (in case you need to do it manually) and troubleshooting. You may skip it completely except for the upgrade remarks.

However, note that this library requires at least \PGF\ version $2.00$. At the time of this writing, many \TeX-distributions still contain the older \PGF\ version $1.18$, so it may be necessary to install a recent \PGF\ prior to using \PGFPlots.

\subsection{Components}
\PGFPlots\ comes with two components:
\begin{enumerate}
	\item the plotting component (which you are currently reading) and
	\item the \PGFPlotstable\ component which simplifies number formatting and postprocessing of numerical tables. It comes as a separate package and has its own manual \href{file:pgfplotstable.pdf}{pgfplotstable.pdf}.
\end{enumerate}

\subsection{Upgrade remarks}
This release provides a lot of improvements which can be found in all detail in \texttt{ChangeLog} for interested readers. However, some attention is useful with respect to the following changes.

\subsubsection{New Optional Features}
\begin{enumerate}
	\item \PGFPlots\ 1.3 comes with user interface improvements. The technical distinction between ``behavior options'' and ``style options'' of older versions is no longer necessary (although still fully supported).

	\item \PGFPlots\ 1.3 has a new feature which allows to \emph{move axis labels tight to tick labels} automatically.

	Since this affects the spacing, it is not enabled be default.

	Use
\begin{codeexample}[code only]
\usepackage{pgfplots}
\pgfplotsset{compat=1.3}
\end{codeexample}
	in your preamble to benefit from the improved spacing\footnote{The |compat=1.3| setting is necessary for new features which move axis labels around like reversed axis. However, \PGFPlots\ will still work as it does in earlier versions, your documents will remain the same if you don't set it explicitly.}.
	Take a look at the next page for the precise definition of |compat|.

	\item \PGFPlots\ 1.3 now supports reversed axes. It is no longer necessary to use work--arounds with negative units.
\pgfkeys{/pdflinks/search key prefixes in/.add={/pgfplots/,}{}}

	Take a look at the |x dir=reverse| key.

	Existing work--arounds will still function properly. Use |\pgfplotsset{compat=1.3}| together with |x dir=reverse| to switch to the new version.
\end{enumerate}

\subsubsection{Old Features Which May Need Attention}
\begin{enumerate}
	\item The |scatter/classes| feature produces proper legends as of version 1.3. This may change the appearance of existing legends of plots with |scatter/classes|.

	\item Due to a small math bug in \PGF\ $2.00$, you \emph{can't} use the math expression `|-x^2|'. It is necessary to use `|0-x^2|' instead. The same holds for
		`|exp(-x^2)|'; use `|exp(0-x^2)|' instead.

		This will be fixed with \PGF\ $>2.00$.

	\item Starting with \PGFPlots\ $1.1$, |\tikzstyle| should \emph{no longer be used} to set \PGFPlots\ options.
	
	Although |\tikzstyle| is still supported for some older \PGFPlots\ options, you should replace any occurance of |\tikzstyle| with |\pgfplotsset{|\meta{style name}|/.style={|\meta{key-value-list}|}}| or the associated |/.append style| variant. See section~\ref{sec:styles} for more detail.
\end{enumerate}
I apologize for any inconvenience caused by these changes.

\begin{pgfplotskey}{compat=\mchoice{1.3,pre 1.3,default,newest} (initially default)}
	The preamble configuration 
\begin{codeexample}[code only]
\usepackage{pgfplots}
\pgfplotsset{compat=1.3}
\end{codeexample}
	allows to choose between backwards compatibility and most recent features.

	\PGFPlots\ version 1.3 comes with new features which may lead to different spacings compared to earlier versions:
	\begin{enumerate}
		\item Version 1.3 comes with the |xlabel near ticks| feature which places (in this case $x$) axis labels ``near'' the tick labels, taking the size of tick labels into account.
		
		Older versions used |xlabel absolute|, i.e.\ an absolute distance between the axis and axis labels (now matter how large or small tick labels are).

		The initial setting \emph{keeps backwards compatibility}.

		You are encouraged to use
\begin{codeexample}[code only]
\usepackage{pgfplots}
\pgfplotsset{compat=1.3}
\end{codeexample}
		in your preamble.
		
		\item Version 1.3 fixes a bug with |anchor=south west| which occurred mainly for reversed axes: the anchors where upside--down. This is fixed now.

		
		If you have written some work--around and want to keep it going, use

		|\pgfplotsset{compat/anchors=pre 1.3}|\pgfmanualpdflabel{/pgfplots/compat/anchors}{} 

		to disable the bug fix.
	\end{enumerate}

	Use |\pgfplotsset{compat=default}| to restore the factory settings.

	The setting |\pgfplotsset{compat=newest}| will always use features of the most recent version. This might result in small changes in the document's appearance.
\end{pgfplotskey}

\subsection{The Team}
\PGFPlots\ has been written mainly by Christian Feuers�nger with many improvements of Pascal Wolkotte and Nick Papior Andersen as a spare time project. We hope it is useful and provides valuable plots.

If you are interested in writing something but don't know how, consider reading the auxiliary manual \href{file:TeX-programming-notes.pdf}{TeX-programming-notes.pdf} which comes with \PGFPlots. It is far from complete, but maybe it is a good starting point (at least for more literature).

\subsection{Acknowledgements}
I thank God for all hours of enjoyed programming. I thank Pascal Wolkotte and Nick Papior Andersen for their programming efforts and contributions as part of the development team. I thank Stefan Tibus, who contributed the |plot shell| feature. I thank Tom Cashman for the contribution of the |reverse legend| feature. Special thanks go to Stefan Pinnow whose tests of \PGFPlots\ lead to numerous quality improvements. Furthermore, I thank Dr.~Schweitzer for many fruitful discussions and Dr.~Meine for his ideas and suggestions. Thanks as well to the many international contributors who provided feature requests or identified bugs or simply manual improvements!

Last but not least, I thank Till Tantau and Mark Wibrow for their excellent graphics (and more) package \PGF\ and \Tikz\ which is the base of \PGFPlots.

\include{pgfplots.install}%
\include{pgfplots.intro}%
\include{pgfplots.reference}%
\include{pgfplots.libs}%
\include{pgfplots.resources}%
\include{pgfplots.importexport}%
\include{pgfplots.basic.reference}%

\printindex

\bibliographystyle{abbrv} %gerapali} %gerabbrv} %gerunsrt.bst} %gerabbrv}% gerplain}
\nocite{pgfplotstable}
\nocite{programmingnotes}
\bibliography{pgfplots}
\end{document}
%
%%%%%%%%%%%%%%%%%%%%%%%%%%%%%%%%%%%%%%%%%%%%%%%%%%%%%%%%%%%%%%%%%%%%%%%%%%%%%
%
% Package pgfplots.sty documentation. 
%
% Copyright 2007/2008 by Christian Feuersaenger.
%
% This program is free software: you can redistribute it and/or modify
% it under the terms of the GNU General Public License as published by
% the Free Software Foundation, either version 3 of the License, or
% (at your option) any later version.
% 
% This program is distributed in the hope that it will be useful,
% but WITHOUT ANY WARRANTY; without even the implied warranty of
% MERCHANTABILITY or FITNESS FOR A PARTICULAR PURPOSE.  See the
% GNU General Public License for more details.
% 
% You should have received a copy of the GNU General Public License
% along with this program.  If not, see <http://www.gnu.org/licenses/>.
%
%
%%%%%%%%%%%%%%%%%%%%%%%%%%%%%%%%%%%%%%%%%%%%%%%%%%%%%%%%%%%%%%%%%%%%%%%%%%%%%
\input{pgfplots.preamble.tex}
%\RequirePackage[german,english,francais]{babel}

\pgfplotsmanualexternalexpensivetrue

%\usetikzlibrary{external}
%\tikzexternalize[prefix=figures/]{pgfplots}

\title{%
	Manual for Package \PGFPlots\\
	{\small Version \pgfplotsversion}\\
	{\small\href{http://sourceforge.net/projects/pgfplots}{http://sourceforge.net/projects/pgfplots}}}

%\includeonly{pgfplots.libs}


\begin{document}

\def\plotcoords{%
\addplot coordinates {
(5,8.312e-02)    (17,2.547e-02)   (49,7.407e-03)
(129,2.102e-03)  (321,5.874e-04)  (769,1.623e-04)
(1793,4.442e-05) (4097,1.207e-05) (9217,3.261e-06)
};

\addplot coordinates{
(7,8.472e-02)    (31,3.044e-02)    (111,1.022e-02)
(351,3.303e-03)  (1023,1.039e-03)  (2815,3.196e-04)
(7423,9.658e-05) (18943,2.873e-05) (47103,8.437e-06)
};

\addplot coordinates{
(9,7.881e-02)     (49,3.243e-02)    (209,1.232e-02)
(769,4.454e-03)   (2561,1.551e-03)  (7937,5.236e-04)
(23297,1.723e-04) (65537,5.545e-05) (178177,1.751e-05)
};

\addplot coordinates{
(11,6.887e-02)    (71,3.177e-02)     (351,1.341e-02)
(1471,5.334e-03)  (5503,2.027e-03)   (18943,7.415e-04)
(61183,2.628e-04) (187903,9.063e-05) (553983,3.053e-05)
};

\addplot coordinates{
(13,5.755e-02)     (97,2.925e-02)     (545,1.351e-02)
(2561,5.842e-03)   (10625,2.397e-03)  (40193,9.414e-04)
(141569,3.564e-04) (471041,1.308e-04) 
(1496065,4.670e-05)
};
}%
\maketitle
\begin{abstract}%
\PGFPlots\ draws high--quality function plots in normal or logarithmic scaling with a user-friendly interface directly in \TeX. The user supplies axis labels, legend entries and the plot coordinates for one or more plots and \PGFPlots\ applies axis scaling, computes any logarithms and axis ticks and draws the plots, supporting line plots, scatter plots, piecewise constant plots, bar plots, area plots, mesh-- and surface plots and some more. It is based on Till Tantau's package \PGF/\Tikz.
\end{abstract}
\tableofcontents
\section{Introduction}
This package provides tools to generate plots and labeled axes easily. It draws normal plots, logplots and semi-logplots, in two and three dimensions. Axis ticks, labels, legends (in case of multiple plots) can be added with key-value options. It can cycle through a set of predefined line/marker/color specifications. In summary, its purpose is to simplify the generation of high-quality function and/or data plots, and solving the problems of
\begin{itemize}
	\item consistency of document and font type and font size,
	\item direct use of \TeX\ math mode in axis descriptions,
	\item consistency of data and figures (no third party tool necessary),
	\item inter document consistency using preamble configurations and styles.
\end{itemize}
Although not necessary, separate |.pdf| or |.eps| graphics can be generated using the |external| library developed as part of \Tikz.

\PGFPlots\ is build completely on \Tikz/\PGF. Knowledge of \Tikz\ will simplify the work with \PGFPlots, although it is not required.

\section[About PGFPlots: Preliminaries]{About {\normalfont\PGFPlots}: Preliminaries}
This section contains information about upgrades, the team, the installation (in case you need to do it manually) and troubleshooting. You may skip it completely except for the upgrade remarks.

However, note that this library requires at least \PGF\ version $2.00$. At the time of this writing, many \TeX-distributions still contain the older \PGF\ version $1.18$, so it may be necessary to install a recent \PGF\ prior to using \PGFPlots.

\subsection{Components}
\PGFPlots\ comes with two components:
\begin{enumerate}
	\item the plotting component (which you are currently reading) and
	\item the \PGFPlotstable\ component which simplifies number formatting and postprocessing of numerical tables. It comes as a separate package and has its own manual \href{file:pgfplotstable.pdf}{pgfplotstable.pdf}.
\end{enumerate}

\subsection{Upgrade remarks}
This release provides a lot of improvements which can be found in all detail in \texttt{ChangeLog} for interested readers. However, some attention is useful with respect to the following changes.

\subsubsection{New Optional Features}
\begin{enumerate}
	\item \PGFPlots\ 1.3 comes with user interface improvements. The technical distinction between ``behavior options'' and ``style options'' of older versions is no longer necessary (although still fully supported).

	\item \PGFPlots\ 1.3 has a new feature which allows to \emph{move axis labels tight to tick labels} automatically.

	Since this affects the spacing, it is not enabled be default.

	Use
\begin{codeexample}[code only]
\usepackage{pgfplots}
\pgfplotsset{compat=1.3}
\end{codeexample}
	in your preamble to benefit from the improved spacing\footnote{The |compat=1.3| setting is necessary for new features which move axis labels around like reversed axis. However, \PGFPlots\ will still work as it does in earlier versions, your documents will remain the same if you don't set it explicitly.}.
	Take a look at the next page for the precise definition of |compat|.

	\item \PGFPlots\ 1.3 now supports reversed axes. It is no longer necessary to use work--arounds with negative units.
\pgfkeys{/pdflinks/search key prefixes in/.add={/pgfplots/,}{}}

	Take a look at the |x dir=reverse| key.

	Existing work--arounds will still function properly. Use |\pgfplotsset{compat=1.3}| together with |x dir=reverse| to switch to the new version.
\end{enumerate}

\subsubsection{Old Features Which May Need Attention}
\begin{enumerate}
	\item The |scatter/classes| feature produces proper legends as of version 1.3. This may change the appearance of existing legends of plots with |scatter/classes|.

	\item Due to a small math bug in \PGF\ $2.00$, you \emph{can't} use the math expression `|-x^2|'. It is necessary to use `|0-x^2|' instead. The same holds for
		`|exp(-x^2)|'; use `|exp(0-x^2)|' instead.

		This will be fixed with \PGF\ $>2.00$.

	\item Starting with \PGFPlots\ $1.1$, |\tikzstyle| should \emph{no longer be used} to set \PGFPlots\ options.
	
	Although |\tikzstyle| is still supported for some older \PGFPlots\ options, you should replace any occurance of |\tikzstyle| with |\pgfplotsset{|\meta{style name}|/.style={|\meta{key-value-list}|}}| or the associated |/.append style| variant. See section~\ref{sec:styles} for more detail.
\end{enumerate}
I apologize for any inconvenience caused by these changes.

\begin{pgfplotskey}{compat=\mchoice{1.3,pre 1.3,default,newest} (initially default)}
	The preamble configuration 
\begin{codeexample}[code only]
\usepackage{pgfplots}
\pgfplotsset{compat=1.3}
\end{codeexample}
	allows to choose between backwards compatibility and most recent features.

	\PGFPlots\ version 1.3 comes with new features which may lead to different spacings compared to earlier versions:
	\begin{enumerate}
		\item Version 1.3 comes with the |xlabel near ticks| feature which places (in this case $x$) axis labels ``near'' the tick labels, taking the size of tick labels into account.
		
		Older versions used |xlabel absolute|, i.e.\ an absolute distance between the axis and axis labels (now matter how large or small tick labels are).

		The initial setting \emph{keeps backwards compatibility}.

		You are encouraged to use
\begin{codeexample}[code only]
\usepackage{pgfplots}
\pgfplotsset{compat=1.3}
\end{codeexample}
		in your preamble.
		
		\item Version 1.3 fixes a bug with |anchor=south west| which occurred mainly for reversed axes: the anchors where upside--down. This is fixed now.

		
		If you have written some work--around and want to keep it going, use

		|\pgfplotsset{compat/anchors=pre 1.3}|\pgfmanualpdflabel{/pgfplots/compat/anchors}{} 

		to disable the bug fix.
	\end{enumerate}

	Use |\pgfplotsset{compat=default}| to restore the factory settings.

	The setting |\pgfplotsset{compat=newest}| will always use features of the most recent version. This might result in small changes in the document's appearance.
\end{pgfplotskey}

\subsection{The Team}
\PGFPlots\ has been written mainly by Christian Feuers�nger with many improvements of Pascal Wolkotte and Nick Papior Andersen as a spare time project. We hope it is useful and provides valuable plots.

If you are interested in writing something but don't know how, consider reading the auxiliary manual \href{file:TeX-programming-notes.pdf}{TeX-programming-notes.pdf} which comes with \PGFPlots. It is far from complete, but maybe it is a good starting point (at least for more literature).

\subsection{Acknowledgements}
I thank God for all hours of enjoyed programming. I thank Pascal Wolkotte and Nick Papior Andersen for their programming efforts and contributions as part of the development team. I thank Stefan Tibus, who contributed the |plot shell| feature. I thank Tom Cashman for the contribution of the |reverse legend| feature. Special thanks go to Stefan Pinnow whose tests of \PGFPlots\ lead to numerous quality improvements. Furthermore, I thank Dr.~Schweitzer for many fruitful discussions and Dr.~Meine for his ideas and suggestions. Thanks as well to the many international contributors who provided feature requests or identified bugs or simply manual improvements!

Last but not least, I thank Till Tantau and Mark Wibrow for their excellent graphics (and more) package \PGF\ and \Tikz\ which is the base of \PGFPlots.

\include{pgfplots.install}%
\include{pgfplots.intro}%
\include{pgfplots.reference}%
\include{pgfplots.libs}%
\include{pgfplots.resources}%
\include{pgfplots.importexport}%
\include{pgfplots.basic.reference}%

\printindex

\bibliographystyle{abbrv} %gerapali} %gerabbrv} %gerunsrt.bst} %gerabbrv}% gerplain}
\nocite{pgfplotstable}
\nocite{programmingnotes}
\bibliography{pgfplots}
\end{document}
%
\include{pgfplots.basic.reference}%

\printindex

\bibliographystyle{abbrv} %gerapali} %gerabbrv} %gerunsrt.bst} %gerabbrv}% gerplain}
\nocite{pgfplotstable}
\nocite{programmingnotes}
\bibliography{pgfplots}
\end{document}
%
\include{pgfplots.basic.reference}%

\printindex

\bibliographystyle{abbrv} %gerapali} %gerabbrv} %gerunsrt.bst} %gerabbrv}% gerplain}
\nocite{pgfplotstable}
\nocite{programmingnotes}
\bibliography{pgfplots}
\end{document}
%
%%%%%%%%%%%%%%%%%%%%%%%%%%%%%%%%%%%%%%%%%%%%%%%%%%%%%%%%%%%%%%%%%%%%%%%%%%%%%
%
% Package pgfplots.sty documentation. 
%
% Copyright 2007/2008 by Christian Feuersaenger.
%
% This program is free software: you can redistribute it and/or modify
% it under the terms of the GNU General Public License as published by
% the Free Software Foundation, either version 3 of the License, or
% (at your option) any later version.
% 
% This program is distributed in the hope that it will be useful,
% but WITHOUT ANY WARRANTY; without even the implied warranty of
% MERCHANTABILITY or FITNESS FOR A PARTICULAR PURPOSE.  See the
% GNU General Public License for more details.
% 
% You should have received a copy of the GNU General Public License
% along with this program.  If not, see <http://www.gnu.org/licenses/>.
%
%
%%%%%%%%%%%%%%%%%%%%%%%%%%%%%%%%%%%%%%%%%%%%%%%%%%%%%%%%%%%%%%%%%%%%%%%%%%%%%
%%%%%%%%%%%%%%%%%%%%%%%%%%%%%%%%%%%%%%%%%%%%%%%%%%%%%%%%%%%%%%%%%%%%%%%%%%%%%
%
% Package pgfplots.sty documentation. 
%
% Copyright 2007/2008 by Christian Feuersaenger.
%
% This program is free software: you can redistribute it and/or modify
% it under the terms of the GNU General Public License as published by
% the Free Software Foundation, either version 3 of the License, or
% (at your option) any later version.
% 
% This program is distributed in the hope that it will be useful,
% but WITHOUT ANY WARRANTY; without even the implied warranty of
% MERCHANTABILITY or FITNESS FOR A PARTICULAR PURPOSE.  See the
% GNU General Public License for more details.
% 
% You should have received a copy of the GNU General Public License
% along with this program.  If not, see <http://www.gnu.org/licenses/>.
%
%
%%%%%%%%%%%%%%%%%%%%%%%%%%%%%%%%%%%%%%%%%%%%%%%%%%%%%%%%%%%%%%%%%%%%%%%%%%%%%
\documentclass[a4paper]{ltxdoc}

\usepackage{makeidx}

% DON't let hyperref overload the format of index and glossary. 
% I want to do that on my own in the stylefiles for makeindex...
\makeatletter
\let\@old@wrindex=\@wrindex
\makeatother

\usepackage{ifpdf}
\ifpdf
	\usepackage[pdftex]{hyperref}
\else
	\usepackage[dvipdfm]{hyperref}
\fi
\hypersetup{pdfborder={0 0 0}}

\makeatletter
\let\@wrindex=\@old@wrindex
\makeatother

% Formatiere Seitennummern im Index:
\newcommand{\indexpageno}[1]{%
	{\bfseries\hyperpage{#1}}%
}


\newcommand{\R}{\mathbb{R}}
\newcommand{\N}{\mathbb{N}}
\newcommand{\Z}{\mathbb{Z}}

\long\def\COMMENTLOWLEVEL#1\ENDCOMMENT{}
\def\ENDCOMMENT{}

\usepackage{textcomp}
\usepackage{booktabs}

\usepackage{calc}
\usepackage[formats]{listings}
%\usepackage{courier} % don't use it - the '^' character can't be copy-pasted in courier

\usepackage{array}
\lstset{%
	basicstyle=\ttfamily,
	language=[LaTeX]tex, % Seems as if \lstset{language=tex} must be invoked BEFORE loading tikz!?
	tabsize=4,
	breaklines=true,
	breakindent=0pt
}

\ifpdf
	\pdfinfo {
		/Author	(Christian Feuersaenger)
	}

\else
	\def\pgfsysdriver{pgfsys-dvipdfm.def}
\fi
%\def\pgfsysdriver{pgfsys-pdftex.def}
\usepackage{pgfplots}

\ifpdf
	\usetikzlibrary{pgfplotsclickable}
\fi

%\usepackage{fp}
% ATTENTION:
% this requires pgf version NEWER than 2.00 :
%\usetikzlibrary{fixedpointarithmetic}

\usetikzlibrary{dateplot}

\usepackage[a4paper,left=2.25cm,right=2.25cm,top=2.5cm,bottom=2.5cm,nohead]{geometry}
\usepackage{amsmath,amssymb}
\usepackage{xxcolor}
\usepackage{pifont}
\usepackage[latin1]{inputenc}
\usepackage{amsmath}
\usepackage{eurosym}

\def\eps{\textsc{eps}}

\input pgfmanual-en-macros.tex

\def\pgfmanualbar{\char`\|}
\makeatletter
\newenvironment{addplotoperation}[3][]{
  \begin{pgfmanualentry}
  	{%
	\let\ltxdoc@marg=\marg
	\let\ltxdoc@oarg=\oarg
	\let\ltxdoc@parg=\parg
	\let\ltxdoc@meta=\meta
	\def\marg##1{{\normalfont\ltxdoc@marg{##1}}}%
	\def\oarg##1{{\normalfont\ltxdoc@oarg{##1}}}%
	\def\parg##1{{\normalfont\ltxdoc@parg{##1}}}%
	\def\meta##1{{\normalfont\ltxdoc@meta{##1}}}%
    \pgfmanualentryheadline{\textcolor{gray}{{\ttfamily\char`\\addplot\ }}%
      \declare{\texttt{#2}} \texttt{#3;}}%
	  \unskip
	 \nobreak
    \pgfmanualentryheadline{\textcolor{gray}{{\texttt{\char`\\addplot}\oarg{style options} \texttt{plot}\oarg{behavior options}}\ }%
      \declare{\texttt{#2}} \texttt{#3} \textcolor{gray}{\meta{trailing path commands}}\texttt{;}}%
    \def\pgfmanualtest{#1}%
    \ifx\pgfmanualtest\@empty%
      \index{#2@\protect\textcolor{gray}{\protect\texttt{plot}}\protect\texttt{ #2}}%
      \index{Plot operations!plot #2@\protect\texttt{plot #2}}%
    \fi%
	}%
    \pgfmanualbody
}
{
  \end{pgfmanualentry}
}

\newenvironment{codekey}[1]{%
  \begin{pgfmanualentry}
 	\pgfmanualentryheadline{{\ttfamily\declare{#1}\textcolor{gray}{/.code}=\marg{...}}\hfill}%
	\def\mykey{#1}%
	\def\mypath{}%
	\def\myname{}%
	\firsttimetrue%
	\decompose#1/\nil%
    \pgfmanualbody
}
{
  \end{pgfmanualentry}
}

\newenvironment{pgfplotscodekey}[1]{%
	\begin{codekey}{/pgfplots/#1}%
}
{
  \end{codekey}
}
\newenvironment{pgfplotscodetwokey}[1]{%
  \begin{pgfmanualentry}
 	\pgfmanualentryheadline{{\ttfamily\declare{/pgfplots/#1}\textcolor{gray}{/.code 2 args}=\marg{...}}\hfill}%
	\def\mykey{/pgfplots/#1}%
	\def\mypath{}%
	\def\myname{}%
	\firsttimetrue%
	\decompose/pgfplots/#1/\nil%
    \pgfmanualbody
}
{
  \end{pgfmanualentry}
}

\newenvironment{pgfplotsxycodekeylist}[1]{%
	\begingroup
	\let\oldpgfmanualentryheadline=\pgfmanualentryheadline
	\def\pgfmanualentryheadline##1{%
		\pgfmanualentryheadline@##1\pgfplots@EOI
	}%
	\def\pgfmanualentryheadline@##1\hfill##2\pgfplots@EOI{%
		\oldpgfmanualentryheadline{{\ttfamily\declare{##1}\textcolor{gray}{/.code}=\marg{...}}\hfill}%
	}
	\begin{pgfplotsxykeylist}{#1}%
}
{
	\end{pgfplotsxykeylist}
	\endgroup
}

\newenvironment{pgfplotskey}[1]{%
  \begin{key}{/pgfplots/#1}%
}
{
  \end{key}
}

\def\choicesep{$\vert$}%
\def\choicearg#1{\texttt{#1}}

\newif\iffirstchoice
\newcommand\mchoice[1]{%
	\begingroup
	\firstchoicetrue
	\foreach \mchoice@ in {#1} {%
		\iffirstchoice
			\global\firstchoicefalse
		\else
			\choicesep
		\fi
		\choicearg{\mchoice@}%
	}%
	\endgroup
}%

% \begin{xykey}{/path/\x label=value}
% \end{xykey}
%
% has same features with 'default', 'initially' etc as key environment
\newenvironment{xykey}[2][]{%
	\begin{pgfmanualentry}
    \def\extrakeytext{}
	\insertpathifneeded{#2}{#1}%
	\expandafter\pgfutil@in@\expandafter=\expandafter{\mykey}%
	\ifpgfutil@in@%
		\expandafter\xykey@eq\mykey\@nil
	\else
		\expandafter\xykey@noeq\mykey\@nil
	\fi
	\pgfmanualbody
}{%
	\end{pgfmanualentry}
}%

% \begin{xystylekey}{/path/\x label=value}
% \end{xystylekey}
%
% has same features with 'default', 'initially' etc as key environment
\newenvironment{xystylekey}[2][]{%
	\begin{pgfmanualentry}
    \def\extrakeytext{style, }
	\insertpathifneeded{#2}{#1}%
	\expandafter\pgfutil@in@\expandafter=\expandafter{\mykey}%
	\ifpgfutil@in@%
		\expandafter\xykey@eq\mykey\@nil
	\else
		\expandafter\xykey@noeq\mykey\@nil
	\fi
	\pgfmanualbody
}{%
	\end{pgfmanualentry}
}%

% \insertpathifneeded{a key}{/pgfplots} -> assign mykey={/pgfplots/a key}
% \insertpathifneeded{/tikz/a key}{/pgfplots} -> assign mykey={/tikz/a key}
%
% #1: the key
% #2: a default path (or empty)
\def\insertpathifneeded#1#2{%
	\def\insertpathifneeded@@{#2}%
	\ifx\insertpathifneeded@@\empty
		\def\mykey{#1}%
	\else
		\insertpathifneeded@#2\@nil
		\ifpgfutil@in@
			\def\mykey{#2/#1}%
		\else
			\def\mykey{#1}%
		\fi
	\fi
}%
\def\insertpathifneeded@#1#2\@nil{%
	\def\insertpathifneeded@@{#1}%
	\def\insertpathifneeded@@@{/}%
	\ifx\insertpathifneeded@@\insertpathifneeded@@@
		\pgfutil@in@true
	\else
		\pgfutil@in@false
	\fi
}%

% \begin{keylist}[default path]
% 	{/path/option 1=value,/path/option 2=value2}
% \end{keylist}
\newenvironment{keylist}[2][]{%
	\begin{pgfmanualentry}
    \def\extrakeytext{}%
	\foreach \xx in {#2} {%
		\expandafter\insertpathifneeded\expandafter{\xx}{#1}%
		\expandafter\extractkey\mykey\@nil%
	}%
	\pgfmanualbody
}{%
  \end{pgfmanualentry}
}%

\newenvironment{pgfplotskeylist}[1]{%
	\begin{keylist}[/pgfplots]{#1}%
}{%
	\end{keylist}%
}

% \begin{xykeylist}[default path]
% 	{/path/option \x1=value,/path/option \x2=value2,/path/option \x3=value}
% \end{xykeylist}
\newenvironment{xykeylist}[2][]{%
	\begin{pgfmanualentry}
    \def\extrakeytext{}
	\foreach \xx in {#2} {%
		\expandafter\insertpathifneeded\expandafter{\xx}{#1}%
		\expandafter\pgfutil@in@\expandafter=\expandafter{\mykey}%
		\ifpgfutil@in@%
			\expandafter\xykey@eq\mykey\@nil
		\else
			\expandafter\xykey@noeq\mykey\@nil
		\fi
	}%
	\pgfmanualbody
}{%
  \end{pgfmanualentry}
}%

\newif\ifxykeyfound

\def\xykey@eq#1=#2\@nil{%
	\def\x{x}%
	\xdef\mykey{#1}%
	\def\xykey@@{#1}%
	\ifx\xykey@@\mykey
		\xykeyfoundfalse
	\else
		\xykeyfoundtrue
	\fi
    \expandafter\extractkey\mykey=#2\@nil%
	\ifxykeyfound
		\def\x{y}%
		\xdef\mykey{#1}%
		\expandafter\extractkey\mykey=#2\@nil%
	\fi
}
\def\xykey@noeq#1\@nil{%
	\def\x{x}%
	\xdef\mykey{#1}%
	\def\xykey@@{#1}%
	\ifx\xykey@@\mykey
		\xykeyfoundfalse
	\else
		\xykeyfoundtrue
	\fi
    \expandafter\extractkey\mykey\@nil%
	\ifxykeyfound
		\def\x{y}%
		\xdef\mykey{#1}%
		\expandafter\extractkey\mykey\@nil%
	\fi
}

% \begin{pgfplotsxykey}{\x label=value}
% \end{pgfplotsxykey}
%
% It introduces the path /pgfplots/ automatically.
%
% has same features with 'default', 'initially' etc as key environment
\newenvironment{pgfplotsxykey}[1]{%
	\begin{xykey}[/pgfplots]{#1}%
}{%
	\end{xykey}%
}


\newenvironment{pgfplotsxykeylist}[1]{%
	\begin{xykeylist}[/pgfplots]{#1}%
}{%
	\end{xykeylist}%
}


\newenvironment{plottype}[1]{%
	\begin{key}{/tikz/#1}%
	\end{key}
  \begin{pgfmanualentry}
    \pgfmanualentryheadline{\textcolor{gray}{{\ttfamily\char`\\addplot+[\declare{#1}]}}}%
    \pgfmanualbody
}
{
  \end{pgfmanualentry}
}

\def\index@prologue{\section*{Index}\addcontentsline{toc}{section}{Index}
}

\newenvironment{pgfplotstablecolumnkey}{%
  \begin{pgfmanualentry}
 	\pgfmanualentryheadline{{\ttfamily\textcolor{gray}{/pgfplots/table/}\declare{columns/\meta{column name}}\textcolor{gray}{/.style}=\marg{key-value-list}}\hfill}%
	\pgfplotsmanualkeyindex{/pgfplots/table/columns}%
    \pgfmanualbody
}
{
  \end{pgfmanualentry}
}
\newenvironment{pgfplotstabledisplaycolumnkey}{%
  \begin{pgfmanualentry}
 	\pgfmanualentryheadline{{\ttfamily\textcolor{gray}{/pgfplots/table/}\declare{display columns/\meta{index}}\textcolor{gray}{/.style}=\marg{key-value-list}}\hfill}%
	\pgfplotsmanualkeyindex{/pgfplots/table/display columns}%
    \pgfmanualbody
}
{
  \end{pgfmanualentry}
}
\newenvironment{pgfplotstablealiaskey}{%
  \begin{pgfmanualentry}
 	\pgfmanualentryheadline{{\ttfamily\textcolor{gray}{/pgfplots/table/}\declare{alias/\meta{col name}}\textcolor{gray}{/.initial}=\marg{real col name}}\hfill}%
	\pgfplotsmanualkeyindex{/pgfplots/table/alias}%
    \pgfmanualbody
}
{
  \end{pgfmanualentry}
}


\def\pgfplotsmanualkeyindex#1{%
	\def\mypath{#1}%
	\def\myname{}%
	\firsttimetrue%
	\decompose#1/\nil%
}
\newenvironment{pgfplotstablecreateonusekey}{%
  \begin{pgfmanualentry}
 	\pgfmanualentryheadline{{\ttfamily\textcolor{gray}{/pgfplots/table/}\declare{create on use/\meta{col name}}\textcolor{gray}{/.style}=\marg{create options}}\hfill}%
	\def\mykey{/pgfplots/table/create on use}%
    \pgfmanualbody
	\pgfplotsmanualkeyindex{/pgfplots/table/create on use}%
}
{
  \end{pgfmanualentry}
}

\def\pgfplotsassertcmdkeyexists#1{%
	\pgfkeysifdefined{/pgfplots/#1/.@cmd}\relax{%
		\pgfplots@error{DOCUMENTATION ERROR: command key /pgfplots/#1 does not exist!}%
	}%
}%

{
\catcode`\ =12%
\gdef\makespaceexpandable{\def\ { }}}%

\def\pgfplotsassertXYcmdkeyexists#1{%
	{\makespaceexpandable\def\x{x}\edef\pgfplotsassertXYcmdkeyexists@tmp{#1}%
	\pgfkeysifdefined{/pgfplots/\pgfplotsassertXYcmdkeyexists@tmp/.@cmd}\relax{%
		\pgfplots@error{DOCUMENTATION ERROR: command key /pgfplots/#1 does not exist!}%
	}}%
	{\makespaceexpandable\def\x{y}\edef\pgfplotsassertXYcmdkeyexists@tmp{#1}%
	\pgfkeysifdefined{/pgfplots/\pgfplotsassertXYcmdkeyexists@tmp/.@cmd}\relax{%
		\pgfplots@error{DOCUMENTATION ERROR: command key /pgfplots/#1 does not exist!}%
	}}%
}%

\def\pgfplotsshortstylekey #1=#2\pgfeov{%
	\pgfplotsassertcmdkeyexists{#1}%
	\pgfplotsassertcmdkeyexists{#2}%
	\begin{pgfplotskey}{#1=\marg{key-value-list}}
		An abbreviation for \texttt{#2/.append style=}\marg{key-value-list}.
	\end{pgfplotskey}
}
\def\pgfplotsshortxystylekey #1=#2\pgfeov{%
	\pgfplotsassertXYcmdkeyexists{#1}%
	\pgfplotsassertXYcmdkeyexists{#2}%
	\begin{pgfplotsxykey}{#1=\marg{key-value-list}}
		An abbreviation for {\def\x{x}\texttt{#2/.append style=}}\marg{key-value-list} (or the respective style for $y$, {\def\x{y}\texttt{#2}}).
	\end{pgfplotsxykey}
}
\def\pgfplotsshortstylekeys #1,#2=#3\pgfeov{%
	\pgfplotsassertcmdkeyexists{#1}%
	\pgfplotsassertcmdkeyexists{#2}%
	\pgfplotsassertcmdkeyexists{#3}%
	\begin{pgfplotskeylist}{%
		#1=\marg{key-value-list},
		#2=\marg{key-value-list}}
		Different abbreviations for \texttt{#3/.append style=}\marg{key-value-list}.
	\end{pgfplotskeylist}
}
\def\pgfplotsshortxystylekeys #1,#2=#3\pgfeov{%
	\pgfplotsassertXYcmdkeyexists{#1}%
	\pgfplotsassertXYcmdkeyexists{#2}%
	\pgfplotsassertXYcmdkeyexists{#3}%
	\begin{pgfplotsxykeylist}{%
		#1=\marg{key-value-list},
		#2=\marg{key-value-list}}
		Different abbreviations for {\def\x{x}\texttt{#3/.append style=}}\marg{key-value-list} (or the respective style for $y$, {\def\x{y}\texttt{#3}}).
	\end{pgfplotsxykeylist}
}

%
% Creates and shows a colormap with specification '#1'.
\def\pgfplotsshowcolormapexample#1{%
	\pgfplotscreatecolormap{tempcolormap}{#1}%
	\pgfplotsshowcolormap{tempcolormap}%
}

% Shows the colormap named '#1'.
\def\pgfplotsshowcolormap#1{%
	\pgfplotscolormaptoshadingspec{#1}{8cm}\result
	\def\tempb{\pgfdeclarehorizontalshading{tempshading}{1cm}}%
	\expandafter\tempb\expandafter{\result}%
	\pgfuseshading{tempshading}%
}

\makeatother


\pgfqkeys{/codeexample}{%
	every codeexample/.style={
		width=8cm,
		/pgfplots/every axis/.append style={legend style={fill=graphicbackground}}
	},
	tabsize=4,
}
\pgfplotsset{
	every axis/.append style={width=7cm},
	filter discard warning=false,
}

\usetikzlibrary{backgrounds,patterns}
% Global styles:
\tikzset{
  shape example/.style={
    color=black!30,
    draw,
    fill=yellow!30,
    line width=.5cm,
    inner xsep=2.5cm,
    inner ysep=0.5cm}
}

\newcommand{\FIXME}[1]{\textcolor{red}{(FIXME: #1)}}

% fuer endvironment 'sidewaysfigure' bspw
% \usepackage{rotating}

\newcommand\Tikz{Ti\textit kZ}
\newcommand\PGF{\textsc{pgf}}
\newcommand\PGFPlots{\textsc{pgfplots}}
\def\PGFPlotstable{\textsc{PgfplotsTable}}


\makeindex

% Fix overful hboxes automatically:
\tolerance=2000
\emergencystretch=10pt

\tikzset{prefix=gnuplot/pgfplots_} % prefix for 'plot function'

\author{%
	Christian Feuers\"anger\footnote{\url{http://wissrech.ins.uni-bonn.de/people/feuersaenger}}\\%
	Institut f\"ur Numerische Simulation\\
	Universit\"at Bonn, Germany}


%\RequirePackage[german,english,francais]{babel}

\pgfplotsmanualexternalexpensivetrue

%\usetikzlibrary{external}
%\tikzexternalize[prefix=figures/]{pgfplots}

\title{%
	Manual for Package \PGFPlots\\
	{\small Version \pgfplotsversion}\\
	{\small\href{http://sourceforge.net/projects/pgfplots}{http://sourceforge.net/projects/pgfplots}}}

%\includeonly{pgfplots.libs}


\begin{document}

\def\plotcoords{%
\addplot coordinates {
(5,8.312e-02)    (17,2.547e-02)   (49,7.407e-03)
(129,2.102e-03)  (321,5.874e-04)  (769,1.623e-04)
(1793,4.442e-05) (4097,1.207e-05) (9217,3.261e-06)
};

\addplot coordinates{
(7,8.472e-02)    (31,3.044e-02)    (111,1.022e-02)
(351,3.303e-03)  (1023,1.039e-03)  (2815,3.196e-04)
(7423,9.658e-05) (18943,2.873e-05) (47103,8.437e-06)
};

\addplot coordinates{
(9,7.881e-02)     (49,3.243e-02)    (209,1.232e-02)
(769,4.454e-03)   (2561,1.551e-03)  (7937,5.236e-04)
(23297,1.723e-04) (65537,5.545e-05) (178177,1.751e-05)
};

\addplot coordinates{
(11,6.887e-02)    (71,3.177e-02)     (351,1.341e-02)
(1471,5.334e-03)  (5503,2.027e-03)   (18943,7.415e-04)
(61183,2.628e-04) (187903,9.063e-05) (553983,3.053e-05)
};

\addplot coordinates{
(13,5.755e-02)     (97,2.925e-02)     (545,1.351e-02)
(2561,5.842e-03)   (10625,2.397e-03)  (40193,9.414e-04)
(141569,3.564e-04) (471041,1.308e-04) 
(1496065,4.670e-05)
};
}%
\maketitle
\begin{abstract}%
\PGFPlots\ draws high--quality function plots in normal or logarithmic scaling with a user-friendly interface directly in \TeX. The user supplies axis labels, legend entries and the plot coordinates for one or more plots and \PGFPlots\ applies axis scaling, computes any logarithms and axis ticks and draws the plots, supporting line plots, scatter plots, piecewise constant plots, bar plots, area plots, mesh-- and surface plots and some more. It is based on Till Tantau's package \PGF/\Tikz.
\end{abstract}
\tableofcontents
\section{Introduction}
This package provides tools to generate plots and labeled axes easily. It draws normal plots, logplots and semi-logplots, in two and three dimensions. Axis ticks, labels, legends (in case of multiple plots) can be added with key-value options. It can cycle through a set of predefined line/marker/color specifications. In summary, its purpose is to simplify the generation of high-quality function and/or data plots, and solving the problems of
\begin{itemize}
	\item consistency of document and font type and font size,
	\item direct use of \TeX\ math mode in axis descriptions,
	\item consistency of data and figures (no third party tool necessary),
	\item inter document consistency using preamble configurations and styles.
\end{itemize}
Although not necessary, separate |.pdf| or |.eps| graphics can be generated using the |external| library developed as part of \Tikz.

\PGFPlots\ is build completely on \Tikz/\PGF. Knowledge of \Tikz\ will simplify the work with \PGFPlots, although it is not required.

\section[About PGFPlots: Preliminaries]{About {\normalfont\PGFPlots}: Preliminaries}
This section contains information about upgrades, the team, the installation (in case you need to do it manually) and troubleshooting. You may skip it completely except for the upgrade remarks.

However, note that this library requires at least \PGF\ version $2.00$. At the time of this writing, many \TeX-distributions still contain the older \PGF\ version $1.18$, so it may be necessary to install a recent \PGF\ prior to using \PGFPlots.

\subsection{Components}
\PGFPlots\ comes with two components:
\begin{enumerate}
	\item the plotting component (which you are currently reading) and
	\item the \PGFPlotstable\ component which simplifies number formatting and postprocessing of numerical tables. It comes as a separate package and has its own manual \href{file:pgfplotstable.pdf}{pgfplotstable.pdf}.
\end{enumerate}

\subsection{Upgrade remarks}
This release provides a lot of improvements which can be found in all detail in \texttt{ChangeLog} for interested readers. However, some attention is useful with respect to the following changes.

\subsubsection{New Optional Features}
\begin{enumerate}
	\item \PGFPlots\ 1.3 comes with user interface improvements. The technical distinction between ``behavior options'' and ``style options'' of older versions is no longer necessary (although still fully supported).

	\item \PGFPlots\ 1.3 has a new feature which allows to \emph{move axis labels tight to tick labels} automatically.

	Since this affects the spacing, it is not enabled be default.

	Use
\begin{codeexample}[code only]
\usepackage{pgfplots}
\pgfplotsset{compat=1.3}
\end{codeexample}
	in your preamble to benefit from the improved spacing\footnote{The |compat=1.3| setting is necessary for new features which move axis labels around like reversed axis. However, \PGFPlots\ will still work as it does in earlier versions, your documents will remain the same if you don't set it explicitly.}.
	Take a look at the next page for the precise definition of |compat|.

	\item \PGFPlots\ 1.3 now supports reversed axes. It is no longer necessary to use work--arounds with negative units.
\pgfkeys{/pdflinks/search key prefixes in/.add={/pgfplots/,}{}}

	Take a look at the |x dir=reverse| key.

	Existing work--arounds will still function properly. Use |\pgfplotsset{compat=1.3}| together with |x dir=reverse| to switch to the new version.
\end{enumerate}

\subsubsection{Old Features Which May Need Attention}
\begin{enumerate}
	\item The |scatter/classes| feature produces proper legends as of version 1.3. This may change the appearance of existing legends of plots with |scatter/classes|.

	\item Due to a small math bug in \PGF\ $2.00$, you \emph{can't} use the math expression `|-x^2|'. It is necessary to use `|0-x^2|' instead. The same holds for
		`|exp(-x^2)|'; use `|exp(0-x^2)|' instead.

		This will be fixed with \PGF\ $>2.00$.

	\item Starting with \PGFPlots\ $1.1$, |\tikzstyle| should \emph{no longer be used} to set \PGFPlots\ options.
	
	Although |\tikzstyle| is still supported for some older \PGFPlots\ options, you should replace any occurance of |\tikzstyle| with |\pgfplotsset{|\meta{style name}|/.style={|\meta{key-value-list}|}}| or the associated |/.append style| variant. See section~\ref{sec:styles} for more detail.
\end{enumerate}
I apologize for any inconvenience caused by these changes.

\begin{pgfplotskey}{compat=\mchoice{1.3,pre 1.3,default,newest} (initially default)}
	The preamble configuration 
\begin{codeexample}[code only]
\usepackage{pgfplots}
\pgfplotsset{compat=1.3}
\end{codeexample}
	allows to choose between backwards compatibility and most recent features.

	\PGFPlots\ version 1.3 comes with new features which may lead to different spacings compared to earlier versions:
	\begin{enumerate}
		\item Version 1.3 comes with the |xlabel near ticks| feature which places (in this case $x$) axis labels ``near'' the tick labels, taking the size of tick labels into account.
		
		Older versions used |xlabel absolute|, i.e.\ an absolute distance between the axis and axis labels (now matter how large or small tick labels are).

		The initial setting \emph{keeps backwards compatibility}.

		You are encouraged to use
\begin{codeexample}[code only]
\usepackage{pgfplots}
\pgfplotsset{compat=1.3}
\end{codeexample}
		in your preamble.
		
		\item Version 1.3 fixes a bug with |anchor=south west| which occurred mainly for reversed axes: the anchors where upside--down. This is fixed now.

		
		If you have written some work--around and want to keep it going, use

		|\pgfplotsset{compat/anchors=pre 1.3}|\pgfmanualpdflabel{/pgfplots/compat/anchors}{} 

		to disable the bug fix.
	\end{enumerate}

	Use |\pgfplotsset{compat=default}| to restore the factory settings.

	The setting |\pgfplotsset{compat=newest}| will always use features of the most recent version. This might result in small changes in the document's appearance.
\end{pgfplotskey}

\subsection{The Team}
\PGFPlots\ has been written mainly by Christian Feuers�nger with many improvements of Pascal Wolkotte and Nick Papior Andersen as a spare time project. We hope it is useful and provides valuable plots.

If you are interested in writing something but don't know how, consider reading the auxiliary manual \href{file:TeX-programming-notes.pdf}{TeX-programming-notes.pdf} which comes with \PGFPlots. It is far from complete, but maybe it is a good starting point (at least for more literature).

\subsection{Acknowledgements}
I thank God for all hours of enjoyed programming. I thank Pascal Wolkotte and Nick Papior Andersen for their programming efforts and contributions as part of the development team. I thank Stefan Tibus, who contributed the |plot shell| feature. I thank Tom Cashman for the contribution of the |reverse legend| feature. Special thanks go to Stefan Pinnow whose tests of \PGFPlots\ lead to numerous quality improvements. Furthermore, I thank Dr.~Schweitzer for many fruitful discussions and Dr.~Meine for his ideas and suggestions. Thanks as well to the many international contributors who provided feature requests or identified bugs or simply manual improvements!

Last but not least, I thank Till Tantau and Mark Wibrow for their excellent graphics (and more) package \PGF\ and \Tikz\ which is the base of \PGFPlots.

%%%%%%%%%%%%%%%%%%%%%%%%%%%%%%%%%%%%%%%%%%%%%%%%%%%%%%%%%%%%%%%%%%%%%%%%%%%%%
%
% Package pgfplots.sty documentation. 
%
% Copyright 2007/2008 by Christian Feuersaenger.
%
% This program is free software: you can redistribute it and/or modify
% it under the terms of the GNU General Public License as published by
% the Free Software Foundation, either version 3 of the License, or
% (at your option) any later version.
% 
% This program is distributed in the hope that it will be useful,
% but WITHOUT ANY WARRANTY; without even the implied warranty of
% MERCHANTABILITY or FITNESS FOR A PARTICULAR PURPOSE.  See the
% GNU General Public License for more details.
% 
% You should have received a copy of the GNU General Public License
% along with this program.  If not, see <http://www.gnu.org/licenses/>.
%
%
%%%%%%%%%%%%%%%%%%%%%%%%%%%%%%%%%%%%%%%%%%%%%%%%%%%%%%%%%%%%%%%%%%%%%%%%%%%%%
%%%%%%%%%%%%%%%%%%%%%%%%%%%%%%%%%%%%%%%%%%%%%%%%%%%%%%%%%%%%%%%%%%%%%%%%%%%%%
%
% Package pgfplots.sty documentation. 
%
% Copyright 2007/2008 by Christian Feuersaenger.
%
% This program is free software: you can redistribute it and/or modify
% it under the terms of the GNU General Public License as published by
% the Free Software Foundation, either version 3 of the License, or
% (at your option) any later version.
% 
% This program is distributed in the hope that it will be useful,
% but WITHOUT ANY WARRANTY; without even the implied warranty of
% MERCHANTABILITY or FITNESS FOR A PARTICULAR PURPOSE.  See the
% GNU General Public License for more details.
% 
% You should have received a copy of the GNU General Public License
% along with this program.  If not, see <http://www.gnu.org/licenses/>.
%
%
%%%%%%%%%%%%%%%%%%%%%%%%%%%%%%%%%%%%%%%%%%%%%%%%%%%%%%%%%%%%%%%%%%%%%%%%%%%%%
\documentclass[a4paper]{ltxdoc}

\usepackage{makeidx}

% DON't let hyperref overload the format of index and glossary. 
% I want to do that on my own in the stylefiles for makeindex...
\makeatletter
\let\@old@wrindex=\@wrindex
\makeatother

\usepackage{ifpdf}
\ifpdf
	\usepackage[pdftex]{hyperref}
\else
	\usepackage[dvipdfm]{hyperref}
\fi
\hypersetup{pdfborder={0 0 0}}

\makeatletter
\let\@wrindex=\@old@wrindex
\makeatother

% Formatiere Seitennummern im Index:
\newcommand{\indexpageno}[1]{%
	{\bfseries\hyperpage{#1}}%
}


\newcommand{\R}{\mathbb{R}}
\newcommand{\N}{\mathbb{N}}
\newcommand{\Z}{\mathbb{Z}}

\long\def\COMMENTLOWLEVEL#1\ENDCOMMENT{}
\def\ENDCOMMENT{}

\usepackage{textcomp}
\usepackage{booktabs}

\usepackage{calc}
\usepackage[formats]{listings}
%\usepackage{courier} % don't use it - the '^' character can't be copy-pasted in courier

\usepackage{array}
\lstset{%
	basicstyle=\ttfamily,
	language=[LaTeX]tex, % Seems as if \lstset{language=tex} must be invoked BEFORE loading tikz!?
	tabsize=4,
	breaklines=true,
	breakindent=0pt
}

\ifpdf
	\pdfinfo {
		/Author	(Christian Feuersaenger)
	}

\else
	\def\pgfsysdriver{pgfsys-dvipdfm.def}
\fi
%\def\pgfsysdriver{pgfsys-pdftex.def}
\usepackage{pgfplots}

\ifpdf
	\usetikzlibrary{pgfplotsclickable}
\fi

%\usepackage{fp}
% ATTENTION:
% this requires pgf version NEWER than 2.00 :
%\usetikzlibrary{fixedpointarithmetic}

\usetikzlibrary{dateplot}

\usepackage[a4paper,left=2.25cm,right=2.25cm,top=2.5cm,bottom=2.5cm,nohead]{geometry}
\usepackage{amsmath,amssymb}
\usepackage{xxcolor}
\usepackage{pifont}
\usepackage[latin1]{inputenc}
\usepackage{amsmath}
\usepackage{eurosym}
\input{pgfplots-macros}

\pgfqkeys{/codeexample}{%
	every codeexample/.style={
		width=8cm,
		/pgfplots/every axis/.append style={legend style={fill=graphicbackground}}
	},
	tabsize=4,
}
\pgfplotsset{
	every axis/.append style={width=7cm},
	filter discard warning=false,
}

\usetikzlibrary{backgrounds,patterns}
% Global styles:
\tikzset{
  shape example/.style={
    color=black!30,
    draw,
    fill=yellow!30,
    line width=.5cm,
    inner xsep=2.5cm,
    inner ysep=0.5cm}
}

\newcommand{\FIXME}[1]{\textcolor{red}{(FIXME: #1)}}

% fuer endvironment 'sidewaysfigure' bspw
% \usepackage{rotating}

\newcommand\Tikz{Ti\textit kZ}
\newcommand\PGF{\textsc{pgf}}
\newcommand\PGFPlots{\textsc{pgfplots}}
\def\PGFPlotstable{\textsc{PgfplotsTable}}


\makeindex

% Fix overful hboxes automatically:
\tolerance=2000
\emergencystretch=10pt

\tikzset{prefix=gnuplot/pgfplots_} % prefix for 'plot function'

\author{%
	Christian Feuers\"anger\footnote{\url{http://wissrech.ins.uni-bonn.de/people/feuersaenger}}\\%
	Institut f\"ur Numerische Simulation\\
	Universit\"at Bonn, Germany}


%\RequirePackage[german,english,francais]{babel}

\pgfplotsmanualexternalexpensivetrue

%\usetikzlibrary{external}
%\tikzexternalize[prefix=figures/]{pgfplots}

\title{%
	Manual for Package \PGFPlots\\
	{\small Version \pgfplotsversion}\\
	{\small\href{http://sourceforge.net/projects/pgfplots}{http://sourceforge.net/projects/pgfplots}}}

%\includeonly{pgfplots.libs}


\begin{document}

\def\plotcoords{%
\addplot coordinates {
(5,8.312e-02)    (17,2.547e-02)   (49,7.407e-03)
(129,2.102e-03)  (321,5.874e-04)  (769,1.623e-04)
(1793,4.442e-05) (4097,1.207e-05) (9217,3.261e-06)
};

\addplot coordinates{
(7,8.472e-02)    (31,3.044e-02)    (111,1.022e-02)
(351,3.303e-03)  (1023,1.039e-03)  (2815,3.196e-04)
(7423,9.658e-05) (18943,2.873e-05) (47103,8.437e-06)
};

\addplot coordinates{
(9,7.881e-02)     (49,3.243e-02)    (209,1.232e-02)
(769,4.454e-03)   (2561,1.551e-03)  (7937,5.236e-04)
(23297,1.723e-04) (65537,5.545e-05) (178177,1.751e-05)
};

\addplot coordinates{
(11,6.887e-02)    (71,3.177e-02)     (351,1.341e-02)
(1471,5.334e-03)  (5503,2.027e-03)   (18943,7.415e-04)
(61183,2.628e-04) (187903,9.063e-05) (553983,3.053e-05)
};

\addplot coordinates{
(13,5.755e-02)     (97,2.925e-02)     (545,1.351e-02)
(2561,5.842e-03)   (10625,2.397e-03)  (40193,9.414e-04)
(141569,3.564e-04) (471041,1.308e-04) 
(1496065,4.670e-05)
};
}%
\maketitle
\begin{abstract}%
\PGFPlots\ draws high--quality function plots in normal or logarithmic scaling with a user-friendly interface directly in \TeX. The user supplies axis labels, legend entries and the plot coordinates for one or more plots and \PGFPlots\ applies axis scaling, computes any logarithms and axis ticks and draws the plots, supporting line plots, scatter plots, piecewise constant plots, bar plots, area plots, mesh-- and surface plots and some more. It is based on Till Tantau's package \PGF/\Tikz.
\end{abstract}
\tableofcontents
\section{Introduction}
This package provides tools to generate plots and labeled axes easily. It draws normal plots, logplots and semi-logplots, in two and three dimensions. Axis ticks, labels, legends (in case of multiple plots) can be added with key-value options. It can cycle through a set of predefined line/marker/color specifications. In summary, its purpose is to simplify the generation of high-quality function and/or data plots, and solving the problems of
\begin{itemize}
	\item consistency of document and font type and font size,
	\item direct use of \TeX\ math mode in axis descriptions,
	\item consistency of data and figures (no third party tool necessary),
	\item inter document consistency using preamble configurations and styles.
\end{itemize}
Although not necessary, separate |.pdf| or |.eps| graphics can be generated using the |external| library developed as part of \Tikz.

\PGFPlots\ is build completely on \Tikz/\PGF. Knowledge of \Tikz\ will simplify the work with \PGFPlots, although it is not required.

\section[About PGFPlots: Preliminaries]{About {\normalfont\PGFPlots}: Preliminaries}
This section contains information about upgrades, the team, the installation (in case you need to do it manually) and troubleshooting. You may skip it completely except for the upgrade remarks.

However, note that this library requires at least \PGF\ version $2.00$. At the time of this writing, many \TeX-distributions still contain the older \PGF\ version $1.18$, so it may be necessary to install a recent \PGF\ prior to using \PGFPlots.

\subsection{Components}
\PGFPlots\ comes with two components:
\begin{enumerate}
	\item the plotting component (which you are currently reading) and
	\item the \PGFPlotstable\ component which simplifies number formatting and postprocessing of numerical tables. It comes as a separate package and has its own manual \href{file:pgfplotstable.pdf}{pgfplotstable.pdf}.
\end{enumerate}

\subsection{Upgrade remarks}
This release provides a lot of improvements which can be found in all detail in \texttt{ChangeLog} for interested readers. However, some attention is useful with respect to the following changes.

\subsubsection{New Optional Features}
\begin{enumerate}
	\item \PGFPlots\ 1.3 comes with user interface improvements. The technical distinction between ``behavior options'' and ``style options'' of older versions is no longer necessary (although still fully supported).

	\item \PGFPlots\ 1.3 has a new feature which allows to \emph{move axis labels tight to tick labels} automatically.

	Since this affects the spacing, it is not enabled be default.

	Use
\begin{codeexample}[code only]
\usepackage{pgfplots}
\pgfplotsset{compat=1.3}
\end{codeexample}
	in your preamble to benefit from the improved spacing\footnote{The |compat=1.3| setting is necessary for new features which move axis labels around like reversed axis. However, \PGFPlots\ will still work as it does in earlier versions, your documents will remain the same if you don't set it explicitly.}.
	Take a look at the next page for the precise definition of |compat|.

	\item \PGFPlots\ 1.3 now supports reversed axes. It is no longer necessary to use work--arounds with negative units.
\pgfkeys{/pdflinks/search key prefixes in/.add={/pgfplots/,}{}}

	Take a look at the |x dir=reverse| key.

	Existing work--arounds will still function properly. Use |\pgfplotsset{compat=1.3}| together with |x dir=reverse| to switch to the new version.
\end{enumerate}

\subsubsection{Old Features Which May Need Attention}
\begin{enumerate}
	\item The |scatter/classes| feature produces proper legends as of version 1.3. This may change the appearance of existing legends of plots with |scatter/classes|.

	\item Due to a small math bug in \PGF\ $2.00$, you \emph{can't} use the math expression `|-x^2|'. It is necessary to use `|0-x^2|' instead. The same holds for
		`|exp(-x^2)|'; use `|exp(0-x^2)|' instead.

		This will be fixed with \PGF\ $>2.00$.

	\item Starting with \PGFPlots\ $1.1$, |\tikzstyle| should \emph{no longer be used} to set \PGFPlots\ options.
	
	Although |\tikzstyle| is still supported for some older \PGFPlots\ options, you should replace any occurance of |\tikzstyle| with |\pgfplotsset{|\meta{style name}|/.style={|\meta{key-value-list}|}}| or the associated |/.append style| variant. See section~\ref{sec:styles} for more detail.
\end{enumerate}
I apologize for any inconvenience caused by these changes.

\begin{pgfplotskey}{compat=\mchoice{1.3,pre 1.3,default,newest} (initially default)}
	The preamble configuration 
\begin{codeexample}[code only]
\usepackage{pgfplots}
\pgfplotsset{compat=1.3}
\end{codeexample}
	allows to choose between backwards compatibility and most recent features.

	\PGFPlots\ version 1.3 comes with new features which may lead to different spacings compared to earlier versions:
	\begin{enumerate}
		\item Version 1.3 comes with the |xlabel near ticks| feature which places (in this case $x$) axis labels ``near'' the tick labels, taking the size of tick labels into account.
		
		Older versions used |xlabel absolute|, i.e.\ an absolute distance between the axis and axis labels (now matter how large or small tick labels are).

		The initial setting \emph{keeps backwards compatibility}.

		You are encouraged to use
\begin{codeexample}[code only]
\usepackage{pgfplots}
\pgfplotsset{compat=1.3}
\end{codeexample}
		in your preamble.
		
		\item Version 1.3 fixes a bug with |anchor=south west| which occurred mainly for reversed axes: the anchors where upside--down. This is fixed now.

		
		If you have written some work--around and want to keep it going, use

		|\pgfplotsset{compat/anchors=pre 1.3}|\pgfmanualpdflabel{/pgfplots/compat/anchors}{} 

		to disable the bug fix.
	\end{enumerate}

	Use |\pgfplotsset{compat=default}| to restore the factory settings.

	The setting |\pgfplotsset{compat=newest}| will always use features of the most recent version. This might result in small changes in the document's appearance.
\end{pgfplotskey}

\subsection{The Team}
\PGFPlots\ has been written mainly by Christian Feuers�nger with many improvements of Pascal Wolkotte and Nick Papior Andersen as a spare time project. We hope it is useful and provides valuable plots.

If you are interested in writing something but don't know how, consider reading the auxiliary manual \href{file:TeX-programming-notes.pdf}{TeX-programming-notes.pdf} which comes with \PGFPlots. It is far from complete, but maybe it is a good starting point (at least for more literature).

\subsection{Acknowledgements}
I thank God for all hours of enjoyed programming. I thank Pascal Wolkotte and Nick Papior Andersen for their programming efforts and contributions as part of the development team. I thank Stefan Tibus, who contributed the |plot shell| feature. I thank Tom Cashman for the contribution of the |reverse legend| feature. Special thanks go to Stefan Pinnow whose tests of \PGFPlots\ lead to numerous quality improvements. Furthermore, I thank Dr.~Schweitzer for many fruitful discussions and Dr.~Meine for his ideas and suggestions. Thanks as well to the many international contributors who provided feature requests or identified bugs or simply manual improvements!

Last but not least, I thank Till Tantau and Mark Wibrow for their excellent graphics (and more) package \PGF\ and \Tikz\ which is the base of \PGFPlots.

%%%%%%%%%%%%%%%%%%%%%%%%%%%%%%%%%%%%%%%%%%%%%%%%%%%%%%%%%%%%%%%%%%%%%%%%%%%%%
%
% Package pgfplots.sty documentation. 
%
% Copyright 2007/2008 by Christian Feuersaenger.
%
% This program is free software: you can redistribute it and/or modify
% it under the terms of the GNU General Public License as published by
% the Free Software Foundation, either version 3 of the License, or
% (at your option) any later version.
% 
% This program is distributed in the hope that it will be useful,
% but WITHOUT ANY WARRANTY; without even the implied warranty of
% MERCHANTABILITY or FITNESS FOR A PARTICULAR PURPOSE.  See the
% GNU General Public License for more details.
% 
% You should have received a copy of the GNU General Public License
% along with this program.  If not, see <http://www.gnu.org/licenses/>.
%
%
%%%%%%%%%%%%%%%%%%%%%%%%%%%%%%%%%%%%%%%%%%%%%%%%%%%%%%%%%%%%%%%%%%%%%%%%%%%%%
\input{pgfplots.preamble.tex}
%\RequirePackage[german,english,francais]{babel}

\pgfplotsmanualexternalexpensivetrue

%\usetikzlibrary{external}
%\tikzexternalize[prefix=figures/]{pgfplots}

\title{%
	Manual for Package \PGFPlots\\
	{\small Version \pgfplotsversion}\\
	{\small\href{http://sourceforge.net/projects/pgfplots}{http://sourceforge.net/projects/pgfplots}}}

%\includeonly{pgfplots.libs}


\begin{document}

\def\plotcoords{%
\addplot coordinates {
(5,8.312e-02)    (17,2.547e-02)   (49,7.407e-03)
(129,2.102e-03)  (321,5.874e-04)  (769,1.623e-04)
(1793,4.442e-05) (4097,1.207e-05) (9217,3.261e-06)
};

\addplot coordinates{
(7,8.472e-02)    (31,3.044e-02)    (111,1.022e-02)
(351,3.303e-03)  (1023,1.039e-03)  (2815,3.196e-04)
(7423,9.658e-05) (18943,2.873e-05) (47103,8.437e-06)
};

\addplot coordinates{
(9,7.881e-02)     (49,3.243e-02)    (209,1.232e-02)
(769,4.454e-03)   (2561,1.551e-03)  (7937,5.236e-04)
(23297,1.723e-04) (65537,5.545e-05) (178177,1.751e-05)
};

\addplot coordinates{
(11,6.887e-02)    (71,3.177e-02)     (351,1.341e-02)
(1471,5.334e-03)  (5503,2.027e-03)   (18943,7.415e-04)
(61183,2.628e-04) (187903,9.063e-05) (553983,3.053e-05)
};

\addplot coordinates{
(13,5.755e-02)     (97,2.925e-02)     (545,1.351e-02)
(2561,5.842e-03)   (10625,2.397e-03)  (40193,9.414e-04)
(141569,3.564e-04) (471041,1.308e-04) 
(1496065,4.670e-05)
};
}%
\maketitle
\begin{abstract}%
\PGFPlots\ draws high--quality function plots in normal or logarithmic scaling with a user-friendly interface directly in \TeX. The user supplies axis labels, legend entries and the plot coordinates for one or more plots and \PGFPlots\ applies axis scaling, computes any logarithms and axis ticks and draws the plots, supporting line plots, scatter plots, piecewise constant plots, bar plots, area plots, mesh-- and surface plots and some more. It is based on Till Tantau's package \PGF/\Tikz.
\end{abstract}
\tableofcontents
\section{Introduction}
This package provides tools to generate plots and labeled axes easily. It draws normal plots, logplots and semi-logplots, in two and three dimensions. Axis ticks, labels, legends (in case of multiple plots) can be added with key-value options. It can cycle through a set of predefined line/marker/color specifications. In summary, its purpose is to simplify the generation of high-quality function and/or data plots, and solving the problems of
\begin{itemize}
	\item consistency of document and font type and font size,
	\item direct use of \TeX\ math mode in axis descriptions,
	\item consistency of data and figures (no third party tool necessary),
	\item inter document consistency using preamble configurations and styles.
\end{itemize}
Although not necessary, separate |.pdf| or |.eps| graphics can be generated using the |external| library developed as part of \Tikz.

\PGFPlots\ is build completely on \Tikz/\PGF. Knowledge of \Tikz\ will simplify the work with \PGFPlots, although it is not required.

\section[About PGFPlots: Preliminaries]{About {\normalfont\PGFPlots}: Preliminaries}
This section contains information about upgrades, the team, the installation (in case you need to do it manually) and troubleshooting. You may skip it completely except for the upgrade remarks.

However, note that this library requires at least \PGF\ version $2.00$. At the time of this writing, many \TeX-distributions still contain the older \PGF\ version $1.18$, so it may be necessary to install a recent \PGF\ prior to using \PGFPlots.

\subsection{Components}
\PGFPlots\ comes with two components:
\begin{enumerate}
	\item the plotting component (which you are currently reading) and
	\item the \PGFPlotstable\ component which simplifies number formatting and postprocessing of numerical tables. It comes as a separate package and has its own manual \href{file:pgfplotstable.pdf}{pgfplotstable.pdf}.
\end{enumerate}

\subsection{Upgrade remarks}
This release provides a lot of improvements which can be found in all detail in \texttt{ChangeLog} for interested readers. However, some attention is useful with respect to the following changes.

\subsubsection{New Optional Features}
\begin{enumerate}
	\item \PGFPlots\ 1.3 comes with user interface improvements. The technical distinction between ``behavior options'' and ``style options'' of older versions is no longer necessary (although still fully supported).

	\item \PGFPlots\ 1.3 has a new feature which allows to \emph{move axis labels tight to tick labels} automatically.

	Since this affects the spacing, it is not enabled be default.

	Use
\begin{codeexample}[code only]
\usepackage{pgfplots}
\pgfplotsset{compat=1.3}
\end{codeexample}
	in your preamble to benefit from the improved spacing\footnote{The |compat=1.3| setting is necessary for new features which move axis labels around like reversed axis. However, \PGFPlots\ will still work as it does in earlier versions, your documents will remain the same if you don't set it explicitly.}.
	Take a look at the next page for the precise definition of |compat|.

	\item \PGFPlots\ 1.3 now supports reversed axes. It is no longer necessary to use work--arounds with negative units.
\pgfkeys{/pdflinks/search key prefixes in/.add={/pgfplots/,}{}}

	Take a look at the |x dir=reverse| key.

	Existing work--arounds will still function properly. Use |\pgfplotsset{compat=1.3}| together with |x dir=reverse| to switch to the new version.
\end{enumerate}

\subsubsection{Old Features Which May Need Attention}
\begin{enumerate}
	\item The |scatter/classes| feature produces proper legends as of version 1.3. This may change the appearance of existing legends of plots with |scatter/classes|.

	\item Due to a small math bug in \PGF\ $2.00$, you \emph{can't} use the math expression `|-x^2|'. It is necessary to use `|0-x^2|' instead. The same holds for
		`|exp(-x^2)|'; use `|exp(0-x^2)|' instead.

		This will be fixed with \PGF\ $>2.00$.

	\item Starting with \PGFPlots\ $1.1$, |\tikzstyle| should \emph{no longer be used} to set \PGFPlots\ options.
	
	Although |\tikzstyle| is still supported for some older \PGFPlots\ options, you should replace any occurance of |\tikzstyle| with |\pgfplotsset{|\meta{style name}|/.style={|\meta{key-value-list}|}}| or the associated |/.append style| variant. See section~\ref{sec:styles} for more detail.
\end{enumerate}
I apologize for any inconvenience caused by these changes.

\begin{pgfplotskey}{compat=\mchoice{1.3,pre 1.3,default,newest} (initially default)}
	The preamble configuration 
\begin{codeexample}[code only]
\usepackage{pgfplots}
\pgfplotsset{compat=1.3}
\end{codeexample}
	allows to choose between backwards compatibility and most recent features.

	\PGFPlots\ version 1.3 comes with new features which may lead to different spacings compared to earlier versions:
	\begin{enumerate}
		\item Version 1.3 comes with the |xlabel near ticks| feature which places (in this case $x$) axis labels ``near'' the tick labels, taking the size of tick labels into account.
		
		Older versions used |xlabel absolute|, i.e.\ an absolute distance between the axis and axis labels (now matter how large or small tick labels are).

		The initial setting \emph{keeps backwards compatibility}.

		You are encouraged to use
\begin{codeexample}[code only]
\usepackage{pgfplots}
\pgfplotsset{compat=1.3}
\end{codeexample}
		in your preamble.
		
		\item Version 1.3 fixes a bug with |anchor=south west| which occurred mainly for reversed axes: the anchors where upside--down. This is fixed now.

		
		If you have written some work--around and want to keep it going, use

		|\pgfplotsset{compat/anchors=pre 1.3}|\pgfmanualpdflabel{/pgfplots/compat/anchors}{} 

		to disable the bug fix.
	\end{enumerate}

	Use |\pgfplotsset{compat=default}| to restore the factory settings.

	The setting |\pgfplotsset{compat=newest}| will always use features of the most recent version. This might result in small changes in the document's appearance.
\end{pgfplotskey}

\subsection{The Team}
\PGFPlots\ has been written mainly by Christian Feuers�nger with many improvements of Pascal Wolkotte and Nick Papior Andersen as a spare time project. We hope it is useful and provides valuable plots.

If you are interested in writing something but don't know how, consider reading the auxiliary manual \href{file:TeX-programming-notes.pdf}{TeX-programming-notes.pdf} which comes with \PGFPlots. It is far from complete, but maybe it is a good starting point (at least for more literature).

\subsection{Acknowledgements}
I thank God for all hours of enjoyed programming. I thank Pascal Wolkotte and Nick Papior Andersen for their programming efforts and contributions as part of the development team. I thank Stefan Tibus, who contributed the |plot shell| feature. I thank Tom Cashman for the contribution of the |reverse legend| feature. Special thanks go to Stefan Pinnow whose tests of \PGFPlots\ lead to numerous quality improvements. Furthermore, I thank Dr.~Schweitzer for many fruitful discussions and Dr.~Meine for his ideas and suggestions. Thanks as well to the many international contributors who provided feature requests or identified bugs or simply manual improvements!

Last but not least, I thank Till Tantau and Mark Wibrow for their excellent graphics (and more) package \PGF\ and \Tikz\ which is the base of \PGFPlots.

\include{pgfplots.install}%
\include{pgfplots.intro}%
\include{pgfplots.reference}%
\include{pgfplots.libs}%
\include{pgfplots.resources}%
\include{pgfplots.importexport}%
\include{pgfplots.basic.reference}%

\printindex

\bibliographystyle{abbrv} %gerapali} %gerabbrv} %gerunsrt.bst} %gerabbrv}% gerplain}
\nocite{pgfplotstable}
\nocite{programmingnotes}
\bibliography{pgfplots}
\end{document}
%
%%%%%%%%%%%%%%%%%%%%%%%%%%%%%%%%%%%%%%%%%%%%%%%%%%%%%%%%%%%%%%%%%%%%%%%%%%%%%
%
% Package pgfplots.sty documentation. 
%
% Copyright 2007/2008 by Christian Feuersaenger.
%
% This program is free software: you can redistribute it and/or modify
% it under the terms of the GNU General Public License as published by
% the Free Software Foundation, either version 3 of the License, or
% (at your option) any later version.
% 
% This program is distributed in the hope that it will be useful,
% but WITHOUT ANY WARRANTY; without even the implied warranty of
% MERCHANTABILITY or FITNESS FOR A PARTICULAR PURPOSE.  See the
% GNU General Public License for more details.
% 
% You should have received a copy of the GNU General Public License
% along with this program.  If not, see <http://www.gnu.org/licenses/>.
%
%
%%%%%%%%%%%%%%%%%%%%%%%%%%%%%%%%%%%%%%%%%%%%%%%%%%%%%%%%%%%%%%%%%%%%%%%%%%%%%
\input{pgfplots.preamble.tex}
%\RequirePackage[german,english,francais]{babel}

\pgfplotsmanualexternalexpensivetrue

%\usetikzlibrary{external}
%\tikzexternalize[prefix=figures/]{pgfplots}

\title{%
	Manual for Package \PGFPlots\\
	{\small Version \pgfplotsversion}\\
	{\small\href{http://sourceforge.net/projects/pgfplots}{http://sourceforge.net/projects/pgfplots}}}

%\includeonly{pgfplots.libs}


\begin{document}

\def\plotcoords{%
\addplot coordinates {
(5,8.312e-02)    (17,2.547e-02)   (49,7.407e-03)
(129,2.102e-03)  (321,5.874e-04)  (769,1.623e-04)
(1793,4.442e-05) (4097,1.207e-05) (9217,3.261e-06)
};

\addplot coordinates{
(7,8.472e-02)    (31,3.044e-02)    (111,1.022e-02)
(351,3.303e-03)  (1023,1.039e-03)  (2815,3.196e-04)
(7423,9.658e-05) (18943,2.873e-05) (47103,8.437e-06)
};

\addplot coordinates{
(9,7.881e-02)     (49,3.243e-02)    (209,1.232e-02)
(769,4.454e-03)   (2561,1.551e-03)  (7937,5.236e-04)
(23297,1.723e-04) (65537,5.545e-05) (178177,1.751e-05)
};

\addplot coordinates{
(11,6.887e-02)    (71,3.177e-02)     (351,1.341e-02)
(1471,5.334e-03)  (5503,2.027e-03)   (18943,7.415e-04)
(61183,2.628e-04) (187903,9.063e-05) (553983,3.053e-05)
};

\addplot coordinates{
(13,5.755e-02)     (97,2.925e-02)     (545,1.351e-02)
(2561,5.842e-03)   (10625,2.397e-03)  (40193,9.414e-04)
(141569,3.564e-04) (471041,1.308e-04) 
(1496065,4.670e-05)
};
}%
\maketitle
\begin{abstract}%
\PGFPlots\ draws high--quality function plots in normal or logarithmic scaling with a user-friendly interface directly in \TeX. The user supplies axis labels, legend entries and the plot coordinates for one or more plots and \PGFPlots\ applies axis scaling, computes any logarithms and axis ticks and draws the plots, supporting line plots, scatter plots, piecewise constant plots, bar plots, area plots, mesh-- and surface plots and some more. It is based on Till Tantau's package \PGF/\Tikz.
\end{abstract}
\tableofcontents
\section{Introduction}
This package provides tools to generate plots and labeled axes easily. It draws normal plots, logplots and semi-logplots, in two and three dimensions. Axis ticks, labels, legends (in case of multiple plots) can be added with key-value options. It can cycle through a set of predefined line/marker/color specifications. In summary, its purpose is to simplify the generation of high-quality function and/or data plots, and solving the problems of
\begin{itemize}
	\item consistency of document and font type and font size,
	\item direct use of \TeX\ math mode in axis descriptions,
	\item consistency of data and figures (no third party tool necessary),
	\item inter document consistency using preamble configurations and styles.
\end{itemize}
Although not necessary, separate |.pdf| or |.eps| graphics can be generated using the |external| library developed as part of \Tikz.

\PGFPlots\ is build completely on \Tikz/\PGF. Knowledge of \Tikz\ will simplify the work with \PGFPlots, although it is not required.

\section[About PGFPlots: Preliminaries]{About {\normalfont\PGFPlots}: Preliminaries}
This section contains information about upgrades, the team, the installation (in case you need to do it manually) and troubleshooting. You may skip it completely except for the upgrade remarks.

However, note that this library requires at least \PGF\ version $2.00$. At the time of this writing, many \TeX-distributions still contain the older \PGF\ version $1.18$, so it may be necessary to install a recent \PGF\ prior to using \PGFPlots.

\subsection{Components}
\PGFPlots\ comes with two components:
\begin{enumerate}
	\item the plotting component (which you are currently reading) and
	\item the \PGFPlotstable\ component which simplifies number formatting and postprocessing of numerical tables. It comes as a separate package and has its own manual \href{file:pgfplotstable.pdf}{pgfplotstable.pdf}.
\end{enumerate}

\subsection{Upgrade remarks}
This release provides a lot of improvements which can be found in all detail in \texttt{ChangeLog} for interested readers. However, some attention is useful with respect to the following changes.

\subsubsection{New Optional Features}
\begin{enumerate}
	\item \PGFPlots\ 1.3 comes with user interface improvements. The technical distinction between ``behavior options'' and ``style options'' of older versions is no longer necessary (although still fully supported).

	\item \PGFPlots\ 1.3 has a new feature which allows to \emph{move axis labels tight to tick labels} automatically.

	Since this affects the spacing, it is not enabled be default.

	Use
\begin{codeexample}[code only]
\usepackage{pgfplots}
\pgfplotsset{compat=1.3}
\end{codeexample}
	in your preamble to benefit from the improved spacing\footnote{The |compat=1.3| setting is necessary for new features which move axis labels around like reversed axis. However, \PGFPlots\ will still work as it does in earlier versions, your documents will remain the same if you don't set it explicitly.}.
	Take a look at the next page for the precise definition of |compat|.

	\item \PGFPlots\ 1.3 now supports reversed axes. It is no longer necessary to use work--arounds with negative units.
\pgfkeys{/pdflinks/search key prefixes in/.add={/pgfplots/,}{}}

	Take a look at the |x dir=reverse| key.

	Existing work--arounds will still function properly. Use |\pgfplotsset{compat=1.3}| together with |x dir=reverse| to switch to the new version.
\end{enumerate}

\subsubsection{Old Features Which May Need Attention}
\begin{enumerate}
	\item The |scatter/classes| feature produces proper legends as of version 1.3. This may change the appearance of existing legends of plots with |scatter/classes|.

	\item Due to a small math bug in \PGF\ $2.00$, you \emph{can't} use the math expression `|-x^2|'. It is necessary to use `|0-x^2|' instead. The same holds for
		`|exp(-x^2)|'; use `|exp(0-x^2)|' instead.

		This will be fixed with \PGF\ $>2.00$.

	\item Starting with \PGFPlots\ $1.1$, |\tikzstyle| should \emph{no longer be used} to set \PGFPlots\ options.
	
	Although |\tikzstyle| is still supported for some older \PGFPlots\ options, you should replace any occurance of |\tikzstyle| with |\pgfplotsset{|\meta{style name}|/.style={|\meta{key-value-list}|}}| or the associated |/.append style| variant. See section~\ref{sec:styles} for more detail.
\end{enumerate}
I apologize for any inconvenience caused by these changes.

\begin{pgfplotskey}{compat=\mchoice{1.3,pre 1.3,default,newest} (initially default)}
	The preamble configuration 
\begin{codeexample}[code only]
\usepackage{pgfplots}
\pgfplotsset{compat=1.3}
\end{codeexample}
	allows to choose between backwards compatibility and most recent features.

	\PGFPlots\ version 1.3 comes with new features which may lead to different spacings compared to earlier versions:
	\begin{enumerate}
		\item Version 1.3 comes with the |xlabel near ticks| feature which places (in this case $x$) axis labels ``near'' the tick labels, taking the size of tick labels into account.
		
		Older versions used |xlabel absolute|, i.e.\ an absolute distance between the axis and axis labels (now matter how large or small tick labels are).

		The initial setting \emph{keeps backwards compatibility}.

		You are encouraged to use
\begin{codeexample}[code only]
\usepackage{pgfplots}
\pgfplotsset{compat=1.3}
\end{codeexample}
		in your preamble.
		
		\item Version 1.3 fixes a bug with |anchor=south west| which occurred mainly for reversed axes: the anchors where upside--down. This is fixed now.

		
		If you have written some work--around and want to keep it going, use

		|\pgfplotsset{compat/anchors=pre 1.3}|\pgfmanualpdflabel{/pgfplots/compat/anchors}{} 

		to disable the bug fix.
	\end{enumerate}

	Use |\pgfplotsset{compat=default}| to restore the factory settings.

	The setting |\pgfplotsset{compat=newest}| will always use features of the most recent version. This might result in small changes in the document's appearance.
\end{pgfplotskey}

\subsection{The Team}
\PGFPlots\ has been written mainly by Christian Feuers�nger with many improvements of Pascal Wolkotte and Nick Papior Andersen as a spare time project. We hope it is useful and provides valuable plots.

If you are interested in writing something but don't know how, consider reading the auxiliary manual \href{file:TeX-programming-notes.pdf}{TeX-programming-notes.pdf} which comes with \PGFPlots. It is far from complete, but maybe it is a good starting point (at least for more literature).

\subsection{Acknowledgements}
I thank God for all hours of enjoyed programming. I thank Pascal Wolkotte and Nick Papior Andersen for their programming efforts and contributions as part of the development team. I thank Stefan Tibus, who contributed the |plot shell| feature. I thank Tom Cashman for the contribution of the |reverse legend| feature. Special thanks go to Stefan Pinnow whose tests of \PGFPlots\ lead to numerous quality improvements. Furthermore, I thank Dr.~Schweitzer for many fruitful discussions and Dr.~Meine for his ideas and suggestions. Thanks as well to the many international contributors who provided feature requests or identified bugs or simply manual improvements!

Last but not least, I thank Till Tantau and Mark Wibrow for their excellent graphics (and more) package \PGF\ and \Tikz\ which is the base of \PGFPlots.

\include{pgfplots.install}%
\include{pgfplots.intro}%
\include{pgfplots.reference}%
\include{pgfplots.libs}%
\include{pgfplots.resources}%
\include{pgfplots.importexport}%
\include{pgfplots.basic.reference}%

\printindex

\bibliographystyle{abbrv} %gerapali} %gerabbrv} %gerunsrt.bst} %gerabbrv}% gerplain}
\nocite{pgfplotstable}
\nocite{programmingnotes}
\bibliography{pgfplots}
\end{document}
%
%%%%%%%%%%%%%%%%%%%%%%%%%%%%%%%%%%%%%%%%%%%%%%%%%%%%%%%%%%%%%%%%%%%%%%%%%%%%%
%
% Package pgfplots.sty documentation. 
%
% Copyright 2007/2008 by Christian Feuersaenger.
%
% This program is free software: you can redistribute it and/or modify
% it under the terms of the GNU General Public License as published by
% the Free Software Foundation, either version 3 of the License, or
% (at your option) any later version.
% 
% This program is distributed in the hope that it will be useful,
% but WITHOUT ANY WARRANTY; without even the implied warranty of
% MERCHANTABILITY or FITNESS FOR A PARTICULAR PURPOSE.  See the
% GNU General Public License for more details.
% 
% You should have received a copy of the GNU General Public License
% along with this program.  If not, see <http://www.gnu.org/licenses/>.
%
%
%%%%%%%%%%%%%%%%%%%%%%%%%%%%%%%%%%%%%%%%%%%%%%%%%%%%%%%%%%%%%%%%%%%%%%%%%%%%%
\input{pgfplots.preamble.tex}
%\RequirePackage[german,english,francais]{babel}

\pgfplotsmanualexternalexpensivetrue

%\usetikzlibrary{external}
%\tikzexternalize[prefix=figures/]{pgfplots}

\title{%
	Manual for Package \PGFPlots\\
	{\small Version \pgfplotsversion}\\
	{\small\href{http://sourceforge.net/projects/pgfplots}{http://sourceforge.net/projects/pgfplots}}}

%\includeonly{pgfplots.libs}


\begin{document}

\def\plotcoords{%
\addplot coordinates {
(5,8.312e-02)    (17,2.547e-02)   (49,7.407e-03)
(129,2.102e-03)  (321,5.874e-04)  (769,1.623e-04)
(1793,4.442e-05) (4097,1.207e-05) (9217,3.261e-06)
};

\addplot coordinates{
(7,8.472e-02)    (31,3.044e-02)    (111,1.022e-02)
(351,3.303e-03)  (1023,1.039e-03)  (2815,3.196e-04)
(7423,9.658e-05) (18943,2.873e-05) (47103,8.437e-06)
};

\addplot coordinates{
(9,7.881e-02)     (49,3.243e-02)    (209,1.232e-02)
(769,4.454e-03)   (2561,1.551e-03)  (7937,5.236e-04)
(23297,1.723e-04) (65537,5.545e-05) (178177,1.751e-05)
};

\addplot coordinates{
(11,6.887e-02)    (71,3.177e-02)     (351,1.341e-02)
(1471,5.334e-03)  (5503,2.027e-03)   (18943,7.415e-04)
(61183,2.628e-04) (187903,9.063e-05) (553983,3.053e-05)
};

\addplot coordinates{
(13,5.755e-02)     (97,2.925e-02)     (545,1.351e-02)
(2561,5.842e-03)   (10625,2.397e-03)  (40193,9.414e-04)
(141569,3.564e-04) (471041,1.308e-04) 
(1496065,4.670e-05)
};
}%
\maketitle
\begin{abstract}%
\PGFPlots\ draws high--quality function plots in normal or logarithmic scaling with a user-friendly interface directly in \TeX. The user supplies axis labels, legend entries and the plot coordinates for one or more plots and \PGFPlots\ applies axis scaling, computes any logarithms and axis ticks and draws the plots, supporting line plots, scatter plots, piecewise constant plots, bar plots, area plots, mesh-- and surface plots and some more. It is based on Till Tantau's package \PGF/\Tikz.
\end{abstract}
\tableofcontents
\section{Introduction}
This package provides tools to generate plots and labeled axes easily. It draws normal plots, logplots and semi-logplots, in two and three dimensions. Axis ticks, labels, legends (in case of multiple plots) can be added with key-value options. It can cycle through a set of predefined line/marker/color specifications. In summary, its purpose is to simplify the generation of high-quality function and/or data plots, and solving the problems of
\begin{itemize}
	\item consistency of document and font type and font size,
	\item direct use of \TeX\ math mode in axis descriptions,
	\item consistency of data and figures (no third party tool necessary),
	\item inter document consistency using preamble configurations and styles.
\end{itemize}
Although not necessary, separate |.pdf| or |.eps| graphics can be generated using the |external| library developed as part of \Tikz.

\PGFPlots\ is build completely on \Tikz/\PGF. Knowledge of \Tikz\ will simplify the work with \PGFPlots, although it is not required.

\section[About PGFPlots: Preliminaries]{About {\normalfont\PGFPlots}: Preliminaries}
This section contains information about upgrades, the team, the installation (in case you need to do it manually) and troubleshooting. You may skip it completely except for the upgrade remarks.

However, note that this library requires at least \PGF\ version $2.00$. At the time of this writing, many \TeX-distributions still contain the older \PGF\ version $1.18$, so it may be necessary to install a recent \PGF\ prior to using \PGFPlots.

\subsection{Components}
\PGFPlots\ comes with two components:
\begin{enumerate}
	\item the plotting component (which you are currently reading) and
	\item the \PGFPlotstable\ component which simplifies number formatting and postprocessing of numerical tables. It comes as a separate package and has its own manual \href{file:pgfplotstable.pdf}{pgfplotstable.pdf}.
\end{enumerate}

\subsection{Upgrade remarks}
This release provides a lot of improvements which can be found in all detail in \texttt{ChangeLog} for interested readers. However, some attention is useful with respect to the following changes.

\subsubsection{New Optional Features}
\begin{enumerate}
	\item \PGFPlots\ 1.3 comes with user interface improvements. The technical distinction between ``behavior options'' and ``style options'' of older versions is no longer necessary (although still fully supported).

	\item \PGFPlots\ 1.3 has a new feature which allows to \emph{move axis labels tight to tick labels} automatically.

	Since this affects the spacing, it is not enabled be default.

	Use
\begin{codeexample}[code only]
\usepackage{pgfplots}
\pgfplotsset{compat=1.3}
\end{codeexample}
	in your preamble to benefit from the improved spacing\footnote{The |compat=1.3| setting is necessary for new features which move axis labels around like reversed axis. However, \PGFPlots\ will still work as it does in earlier versions, your documents will remain the same if you don't set it explicitly.}.
	Take a look at the next page for the precise definition of |compat|.

	\item \PGFPlots\ 1.3 now supports reversed axes. It is no longer necessary to use work--arounds with negative units.
\pgfkeys{/pdflinks/search key prefixes in/.add={/pgfplots/,}{}}

	Take a look at the |x dir=reverse| key.

	Existing work--arounds will still function properly. Use |\pgfplotsset{compat=1.3}| together with |x dir=reverse| to switch to the new version.
\end{enumerate}

\subsubsection{Old Features Which May Need Attention}
\begin{enumerate}
	\item The |scatter/classes| feature produces proper legends as of version 1.3. This may change the appearance of existing legends of plots with |scatter/classes|.

	\item Due to a small math bug in \PGF\ $2.00$, you \emph{can't} use the math expression `|-x^2|'. It is necessary to use `|0-x^2|' instead. The same holds for
		`|exp(-x^2)|'; use `|exp(0-x^2)|' instead.

		This will be fixed with \PGF\ $>2.00$.

	\item Starting with \PGFPlots\ $1.1$, |\tikzstyle| should \emph{no longer be used} to set \PGFPlots\ options.
	
	Although |\tikzstyle| is still supported for some older \PGFPlots\ options, you should replace any occurance of |\tikzstyle| with |\pgfplotsset{|\meta{style name}|/.style={|\meta{key-value-list}|}}| or the associated |/.append style| variant. See section~\ref{sec:styles} for more detail.
\end{enumerate}
I apologize for any inconvenience caused by these changes.

\begin{pgfplotskey}{compat=\mchoice{1.3,pre 1.3,default,newest} (initially default)}
	The preamble configuration 
\begin{codeexample}[code only]
\usepackage{pgfplots}
\pgfplotsset{compat=1.3}
\end{codeexample}
	allows to choose between backwards compatibility and most recent features.

	\PGFPlots\ version 1.3 comes with new features which may lead to different spacings compared to earlier versions:
	\begin{enumerate}
		\item Version 1.3 comes with the |xlabel near ticks| feature which places (in this case $x$) axis labels ``near'' the tick labels, taking the size of tick labels into account.
		
		Older versions used |xlabel absolute|, i.e.\ an absolute distance between the axis and axis labels (now matter how large or small tick labels are).

		The initial setting \emph{keeps backwards compatibility}.

		You are encouraged to use
\begin{codeexample}[code only]
\usepackage{pgfplots}
\pgfplotsset{compat=1.3}
\end{codeexample}
		in your preamble.
		
		\item Version 1.3 fixes a bug with |anchor=south west| which occurred mainly for reversed axes: the anchors where upside--down. This is fixed now.

		
		If you have written some work--around and want to keep it going, use

		|\pgfplotsset{compat/anchors=pre 1.3}|\pgfmanualpdflabel{/pgfplots/compat/anchors}{} 

		to disable the bug fix.
	\end{enumerate}

	Use |\pgfplotsset{compat=default}| to restore the factory settings.

	The setting |\pgfplotsset{compat=newest}| will always use features of the most recent version. This might result in small changes in the document's appearance.
\end{pgfplotskey}

\subsection{The Team}
\PGFPlots\ has been written mainly by Christian Feuers�nger with many improvements of Pascal Wolkotte and Nick Papior Andersen as a spare time project. We hope it is useful and provides valuable plots.

If you are interested in writing something but don't know how, consider reading the auxiliary manual \href{file:TeX-programming-notes.pdf}{TeX-programming-notes.pdf} which comes with \PGFPlots. It is far from complete, but maybe it is a good starting point (at least for more literature).

\subsection{Acknowledgements}
I thank God for all hours of enjoyed programming. I thank Pascal Wolkotte and Nick Papior Andersen for their programming efforts and contributions as part of the development team. I thank Stefan Tibus, who contributed the |plot shell| feature. I thank Tom Cashman for the contribution of the |reverse legend| feature. Special thanks go to Stefan Pinnow whose tests of \PGFPlots\ lead to numerous quality improvements. Furthermore, I thank Dr.~Schweitzer for many fruitful discussions and Dr.~Meine for his ideas and suggestions. Thanks as well to the many international contributors who provided feature requests or identified bugs or simply manual improvements!

Last but not least, I thank Till Tantau and Mark Wibrow for their excellent graphics (and more) package \PGF\ and \Tikz\ which is the base of \PGFPlots.

\include{pgfplots.install}%
\include{pgfplots.intro}%
\include{pgfplots.reference}%
\include{pgfplots.libs}%
\include{pgfplots.resources}%
\include{pgfplots.importexport}%
\include{pgfplots.basic.reference}%

\printindex

\bibliographystyle{abbrv} %gerapali} %gerabbrv} %gerunsrt.bst} %gerabbrv}% gerplain}
\nocite{pgfplotstable}
\nocite{programmingnotes}
\bibliography{pgfplots}
\end{document}
%
%%%%%%%%%%%%%%%%%%%%%%%%%%%%%%%%%%%%%%%%%%%%%%%%%%%%%%%%%%%%%%%%%%%%%%%%%%%%%
%
% Package pgfplots.sty documentation. 
%
% Copyright 2007/2008 by Christian Feuersaenger.
%
% This program is free software: you can redistribute it and/or modify
% it under the terms of the GNU General Public License as published by
% the Free Software Foundation, either version 3 of the License, or
% (at your option) any later version.
% 
% This program is distributed in the hope that it will be useful,
% but WITHOUT ANY WARRANTY; without even the implied warranty of
% MERCHANTABILITY or FITNESS FOR A PARTICULAR PURPOSE.  See the
% GNU General Public License for more details.
% 
% You should have received a copy of the GNU General Public License
% along with this program.  If not, see <http://www.gnu.org/licenses/>.
%
%
%%%%%%%%%%%%%%%%%%%%%%%%%%%%%%%%%%%%%%%%%%%%%%%%%%%%%%%%%%%%%%%%%%%%%%%%%%%%%
\input{pgfplots.preamble.tex}
%\RequirePackage[german,english,francais]{babel}

\pgfplotsmanualexternalexpensivetrue

%\usetikzlibrary{external}
%\tikzexternalize[prefix=figures/]{pgfplots}

\title{%
	Manual for Package \PGFPlots\\
	{\small Version \pgfplotsversion}\\
	{\small\href{http://sourceforge.net/projects/pgfplots}{http://sourceforge.net/projects/pgfplots}}}

%\includeonly{pgfplots.libs}


\begin{document}

\def\plotcoords{%
\addplot coordinates {
(5,8.312e-02)    (17,2.547e-02)   (49,7.407e-03)
(129,2.102e-03)  (321,5.874e-04)  (769,1.623e-04)
(1793,4.442e-05) (4097,1.207e-05) (9217,3.261e-06)
};

\addplot coordinates{
(7,8.472e-02)    (31,3.044e-02)    (111,1.022e-02)
(351,3.303e-03)  (1023,1.039e-03)  (2815,3.196e-04)
(7423,9.658e-05) (18943,2.873e-05) (47103,8.437e-06)
};

\addplot coordinates{
(9,7.881e-02)     (49,3.243e-02)    (209,1.232e-02)
(769,4.454e-03)   (2561,1.551e-03)  (7937,5.236e-04)
(23297,1.723e-04) (65537,5.545e-05) (178177,1.751e-05)
};

\addplot coordinates{
(11,6.887e-02)    (71,3.177e-02)     (351,1.341e-02)
(1471,5.334e-03)  (5503,2.027e-03)   (18943,7.415e-04)
(61183,2.628e-04) (187903,9.063e-05) (553983,3.053e-05)
};

\addplot coordinates{
(13,5.755e-02)     (97,2.925e-02)     (545,1.351e-02)
(2561,5.842e-03)   (10625,2.397e-03)  (40193,9.414e-04)
(141569,3.564e-04) (471041,1.308e-04) 
(1496065,4.670e-05)
};
}%
\maketitle
\begin{abstract}%
\PGFPlots\ draws high--quality function plots in normal or logarithmic scaling with a user-friendly interface directly in \TeX. The user supplies axis labels, legend entries and the plot coordinates for one or more plots and \PGFPlots\ applies axis scaling, computes any logarithms and axis ticks and draws the plots, supporting line plots, scatter plots, piecewise constant plots, bar plots, area plots, mesh-- and surface plots and some more. It is based on Till Tantau's package \PGF/\Tikz.
\end{abstract}
\tableofcontents
\section{Introduction}
This package provides tools to generate plots and labeled axes easily. It draws normal plots, logplots and semi-logplots, in two and three dimensions. Axis ticks, labels, legends (in case of multiple plots) can be added with key-value options. It can cycle through a set of predefined line/marker/color specifications. In summary, its purpose is to simplify the generation of high-quality function and/or data plots, and solving the problems of
\begin{itemize}
	\item consistency of document and font type and font size,
	\item direct use of \TeX\ math mode in axis descriptions,
	\item consistency of data and figures (no third party tool necessary),
	\item inter document consistency using preamble configurations and styles.
\end{itemize}
Although not necessary, separate |.pdf| or |.eps| graphics can be generated using the |external| library developed as part of \Tikz.

\PGFPlots\ is build completely on \Tikz/\PGF. Knowledge of \Tikz\ will simplify the work with \PGFPlots, although it is not required.

\section[About PGFPlots: Preliminaries]{About {\normalfont\PGFPlots}: Preliminaries}
This section contains information about upgrades, the team, the installation (in case you need to do it manually) and troubleshooting. You may skip it completely except for the upgrade remarks.

However, note that this library requires at least \PGF\ version $2.00$. At the time of this writing, many \TeX-distributions still contain the older \PGF\ version $1.18$, so it may be necessary to install a recent \PGF\ prior to using \PGFPlots.

\subsection{Components}
\PGFPlots\ comes with two components:
\begin{enumerate}
	\item the plotting component (which you are currently reading) and
	\item the \PGFPlotstable\ component which simplifies number formatting and postprocessing of numerical tables. It comes as a separate package and has its own manual \href{file:pgfplotstable.pdf}{pgfplotstable.pdf}.
\end{enumerate}

\subsection{Upgrade remarks}
This release provides a lot of improvements which can be found in all detail in \texttt{ChangeLog} for interested readers. However, some attention is useful with respect to the following changes.

\subsubsection{New Optional Features}
\begin{enumerate}
	\item \PGFPlots\ 1.3 comes with user interface improvements. The technical distinction between ``behavior options'' and ``style options'' of older versions is no longer necessary (although still fully supported).

	\item \PGFPlots\ 1.3 has a new feature which allows to \emph{move axis labels tight to tick labels} automatically.

	Since this affects the spacing, it is not enabled be default.

	Use
\begin{codeexample}[code only]
\usepackage{pgfplots}
\pgfplotsset{compat=1.3}
\end{codeexample}
	in your preamble to benefit from the improved spacing\footnote{The |compat=1.3| setting is necessary for new features which move axis labels around like reversed axis. However, \PGFPlots\ will still work as it does in earlier versions, your documents will remain the same if you don't set it explicitly.}.
	Take a look at the next page for the precise definition of |compat|.

	\item \PGFPlots\ 1.3 now supports reversed axes. It is no longer necessary to use work--arounds with negative units.
\pgfkeys{/pdflinks/search key prefixes in/.add={/pgfplots/,}{}}

	Take a look at the |x dir=reverse| key.

	Existing work--arounds will still function properly. Use |\pgfplotsset{compat=1.3}| together with |x dir=reverse| to switch to the new version.
\end{enumerate}

\subsubsection{Old Features Which May Need Attention}
\begin{enumerate}
	\item The |scatter/classes| feature produces proper legends as of version 1.3. This may change the appearance of existing legends of plots with |scatter/classes|.

	\item Due to a small math bug in \PGF\ $2.00$, you \emph{can't} use the math expression `|-x^2|'. It is necessary to use `|0-x^2|' instead. The same holds for
		`|exp(-x^2)|'; use `|exp(0-x^2)|' instead.

		This will be fixed with \PGF\ $>2.00$.

	\item Starting with \PGFPlots\ $1.1$, |\tikzstyle| should \emph{no longer be used} to set \PGFPlots\ options.
	
	Although |\tikzstyle| is still supported for some older \PGFPlots\ options, you should replace any occurance of |\tikzstyle| with |\pgfplotsset{|\meta{style name}|/.style={|\meta{key-value-list}|}}| or the associated |/.append style| variant. See section~\ref{sec:styles} for more detail.
\end{enumerate}
I apologize for any inconvenience caused by these changes.

\begin{pgfplotskey}{compat=\mchoice{1.3,pre 1.3,default,newest} (initially default)}
	The preamble configuration 
\begin{codeexample}[code only]
\usepackage{pgfplots}
\pgfplotsset{compat=1.3}
\end{codeexample}
	allows to choose between backwards compatibility and most recent features.

	\PGFPlots\ version 1.3 comes with new features which may lead to different spacings compared to earlier versions:
	\begin{enumerate}
		\item Version 1.3 comes with the |xlabel near ticks| feature which places (in this case $x$) axis labels ``near'' the tick labels, taking the size of tick labels into account.
		
		Older versions used |xlabel absolute|, i.e.\ an absolute distance between the axis and axis labels (now matter how large or small tick labels are).

		The initial setting \emph{keeps backwards compatibility}.

		You are encouraged to use
\begin{codeexample}[code only]
\usepackage{pgfplots}
\pgfplotsset{compat=1.3}
\end{codeexample}
		in your preamble.
		
		\item Version 1.3 fixes a bug with |anchor=south west| which occurred mainly for reversed axes: the anchors where upside--down. This is fixed now.

		
		If you have written some work--around and want to keep it going, use

		|\pgfplotsset{compat/anchors=pre 1.3}|\pgfmanualpdflabel{/pgfplots/compat/anchors}{} 

		to disable the bug fix.
	\end{enumerate}

	Use |\pgfplotsset{compat=default}| to restore the factory settings.

	The setting |\pgfplotsset{compat=newest}| will always use features of the most recent version. This might result in small changes in the document's appearance.
\end{pgfplotskey}

\subsection{The Team}
\PGFPlots\ has been written mainly by Christian Feuers�nger with many improvements of Pascal Wolkotte and Nick Papior Andersen as a spare time project. We hope it is useful and provides valuable plots.

If you are interested in writing something but don't know how, consider reading the auxiliary manual \href{file:TeX-programming-notes.pdf}{TeX-programming-notes.pdf} which comes with \PGFPlots. It is far from complete, but maybe it is a good starting point (at least for more literature).

\subsection{Acknowledgements}
I thank God for all hours of enjoyed programming. I thank Pascal Wolkotte and Nick Papior Andersen for their programming efforts and contributions as part of the development team. I thank Stefan Tibus, who contributed the |plot shell| feature. I thank Tom Cashman for the contribution of the |reverse legend| feature. Special thanks go to Stefan Pinnow whose tests of \PGFPlots\ lead to numerous quality improvements. Furthermore, I thank Dr.~Schweitzer for many fruitful discussions and Dr.~Meine for his ideas and suggestions. Thanks as well to the many international contributors who provided feature requests or identified bugs or simply manual improvements!

Last but not least, I thank Till Tantau and Mark Wibrow for their excellent graphics (and more) package \PGF\ and \Tikz\ which is the base of \PGFPlots.

\include{pgfplots.install}%
\include{pgfplots.intro}%
\include{pgfplots.reference}%
\include{pgfplots.libs}%
\include{pgfplots.resources}%
\include{pgfplots.importexport}%
\include{pgfplots.basic.reference}%

\printindex

\bibliographystyle{abbrv} %gerapali} %gerabbrv} %gerunsrt.bst} %gerabbrv}% gerplain}
\nocite{pgfplotstable}
\nocite{programmingnotes}
\bibliography{pgfplots}
\end{document}
%
%%%%%%%%%%%%%%%%%%%%%%%%%%%%%%%%%%%%%%%%%%%%%%%%%%%%%%%%%%%%%%%%%%%%%%%%%%%%%
%
% Package pgfplots.sty documentation. 
%
% Copyright 2007/2008 by Christian Feuersaenger.
%
% This program is free software: you can redistribute it and/or modify
% it under the terms of the GNU General Public License as published by
% the Free Software Foundation, either version 3 of the License, or
% (at your option) any later version.
% 
% This program is distributed in the hope that it will be useful,
% but WITHOUT ANY WARRANTY; without even the implied warranty of
% MERCHANTABILITY or FITNESS FOR A PARTICULAR PURPOSE.  See the
% GNU General Public License for more details.
% 
% You should have received a copy of the GNU General Public License
% along with this program.  If not, see <http://www.gnu.org/licenses/>.
%
%
%%%%%%%%%%%%%%%%%%%%%%%%%%%%%%%%%%%%%%%%%%%%%%%%%%%%%%%%%%%%%%%%%%%%%%%%%%%%%
\input{pgfplots.preamble.tex}
%\RequirePackage[german,english,francais]{babel}

\pgfplotsmanualexternalexpensivetrue

%\usetikzlibrary{external}
%\tikzexternalize[prefix=figures/]{pgfplots}

\title{%
	Manual for Package \PGFPlots\\
	{\small Version \pgfplotsversion}\\
	{\small\href{http://sourceforge.net/projects/pgfplots}{http://sourceforge.net/projects/pgfplots}}}

%\includeonly{pgfplots.libs}


\begin{document}

\def\plotcoords{%
\addplot coordinates {
(5,8.312e-02)    (17,2.547e-02)   (49,7.407e-03)
(129,2.102e-03)  (321,5.874e-04)  (769,1.623e-04)
(1793,4.442e-05) (4097,1.207e-05) (9217,3.261e-06)
};

\addplot coordinates{
(7,8.472e-02)    (31,3.044e-02)    (111,1.022e-02)
(351,3.303e-03)  (1023,1.039e-03)  (2815,3.196e-04)
(7423,9.658e-05) (18943,2.873e-05) (47103,8.437e-06)
};

\addplot coordinates{
(9,7.881e-02)     (49,3.243e-02)    (209,1.232e-02)
(769,4.454e-03)   (2561,1.551e-03)  (7937,5.236e-04)
(23297,1.723e-04) (65537,5.545e-05) (178177,1.751e-05)
};

\addplot coordinates{
(11,6.887e-02)    (71,3.177e-02)     (351,1.341e-02)
(1471,5.334e-03)  (5503,2.027e-03)   (18943,7.415e-04)
(61183,2.628e-04) (187903,9.063e-05) (553983,3.053e-05)
};

\addplot coordinates{
(13,5.755e-02)     (97,2.925e-02)     (545,1.351e-02)
(2561,5.842e-03)   (10625,2.397e-03)  (40193,9.414e-04)
(141569,3.564e-04) (471041,1.308e-04) 
(1496065,4.670e-05)
};
}%
\maketitle
\begin{abstract}%
\PGFPlots\ draws high--quality function plots in normal or logarithmic scaling with a user-friendly interface directly in \TeX. The user supplies axis labels, legend entries and the plot coordinates for one or more plots and \PGFPlots\ applies axis scaling, computes any logarithms and axis ticks and draws the plots, supporting line plots, scatter plots, piecewise constant plots, bar plots, area plots, mesh-- and surface plots and some more. It is based on Till Tantau's package \PGF/\Tikz.
\end{abstract}
\tableofcontents
\section{Introduction}
This package provides tools to generate plots and labeled axes easily. It draws normal plots, logplots and semi-logplots, in two and three dimensions. Axis ticks, labels, legends (in case of multiple plots) can be added with key-value options. It can cycle through a set of predefined line/marker/color specifications. In summary, its purpose is to simplify the generation of high-quality function and/or data plots, and solving the problems of
\begin{itemize}
	\item consistency of document and font type and font size,
	\item direct use of \TeX\ math mode in axis descriptions,
	\item consistency of data and figures (no third party tool necessary),
	\item inter document consistency using preamble configurations and styles.
\end{itemize}
Although not necessary, separate |.pdf| or |.eps| graphics can be generated using the |external| library developed as part of \Tikz.

\PGFPlots\ is build completely on \Tikz/\PGF. Knowledge of \Tikz\ will simplify the work with \PGFPlots, although it is not required.

\section[About PGFPlots: Preliminaries]{About {\normalfont\PGFPlots}: Preliminaries}
This section contains information about upgrades, the team, the installation (in case you need to do it manually) and troubleshooting. You may skip it completely except for the upgrade remarks.

However, note that this library requires at least \PGF\ version $2.00$. At the time of this writing, many \TeX-distributions still contain the older \PGF\ version $1.18$, so it may be necessary to install a recent \PGF\ prior to using \PGFPlots.

\subsection{Components}
\PGFPlots\ comes with two components:
\begin{enumerate}
	\item the plotting component (which you are currently reading) and
	\item the \PGFPlotstable\ component which simplifies number formatting and postprocessing of numerical tables. It comes as a separate package and has its own manual \href{file:pgfplotstable.pdf}{pgfplotstable.pdf}.
\end{enumerate}

\subsection{Upgrade remarks}
This release provides a lot of improvements which can be found in all detail in \texttt{ChangeLog} for interested readers. However, some attention is useful with respect to the following changes.

\subsubsection{New Optional Features}
\begin{enumerate}
	\item \PGFPlots\ 1.3 comes with user interface improvements. The technical distinction between ``behavior options'' and ``style options'' of older versions is no longer necessary (although still fully supported).

	\item \PGFPlots\ 1.3 has a new feature which allows to \emph{move axis labels tight to tick labels} automatically.

	Since this affects the spacing, it is not enabled be default.

	Use
\begin{codeexample}[code only]
\usepackage{pgfplots}
\pgfplotsset{compat=1.3}
\end{codeexample}
	in your preamble to benefit from the improved spacing\footnote{The |compat=1.3| setting is necessary for new features which move axis labels around like reversed axis. However, \PGFPlots\ will still work as it does in earlier versions, your documents will remain the same if you don't set it explicitly.}.
	Take a look at the next page for the precise definition of |compat|.

	\item \PGFPlots\ 1.3 now supports reversed axes. It is no longer necessary to use work--arounds with negative units.
\pgfkeys{/pdflinks/search key prefixes in/.add={/pgfplots/,}{}}

	Take a look at the |x dir=reverse| key.

	Existing work--arounds will still function properly. Use |\pgfplotsset{compat=1.3}| together with |x dir=reverse| to switch to the new version.
\end{enumerate}

\subsubsection{Old Features Which May Need Attention}
\begin{enumerate}
	\item The |scatter/classes| feature produces proper legends as of version 1.3. This may change the appearance of existing legends of plots with |scatter/classes|.

	\item Due to a small math bug in \PGF\ $2.00$, you \emph{can't} use the math expression `|-x^2|'. It is necessary to use `|0-x^2|' instead. The same holds for
		`|exp(-x^2)|'; use `|exp(0-x^2)|' instead.

		This will be fixed with \PGF\ $>2.00$.

	\item Starting with \PGFPlots\ $1.1$, |\tikzstyle| should \emph{no longer be used} to set \PGFPlots\ options.
	
	Although |\tikzstyle| is still supported for some older \PGFPlots\ options, you should replace any occurance of |\tikzstyle| with |\pgfplotsset{|\meta{style name}|/.style={|\meta{key-value-list}|}}| or the associated |/.append style| variant. See section~\ref{sec:styles} for more detail.
\end{enumerate}
I apologize for any inconvenience caused by these changes.

\begin{pgfplotskey}{compat=\mchoice{1.3,pre 1.3,default,newest} (initially default)}
	The preamble configuration 
\begin{codeexample}[code only]
\usepackage{pgfplots}
\pgfplotsset{compat=1.3}
\end{codeexample}
	allows to choose between backwards compatibility and most recent features.

	\PGFPlots\ version 1.3 comes with new features which may lead to different spacings compared to earlier versions:
	\begin{enumerate}
		\item Version 1.3 comes with the |xlabel near ticks| feature which places (in this case $x$) axis labels ``near'' the tick labels, taking the size of tick labels into account.
		
		Older versions used |xlabel absolute|, i.e.\ an absolute distance between the axis and axis labels (now matter how large or small tick labels are).

		The initial setting \emph{keeps backwards compatibility}.

		You are encouraged to use
\begin{codeexample}[code only]
\usepackage{pgfplots}
\pgfplotsset{compat=1.3}
\end{codeexample}
		in your preamble.
		
		\item Version 1.3 fixes a bug with |anchor=south west| which occurred mainly for reversed axes: the anchors where upside--down. This is fixed now.

		
		If you have written some work--around and want to keep it going, use

		|\pgfplotsset{compat/anchors=pre 1.3}|\pgfmanualpdflabel{/pgfplots/compat/anchors}{} 

		to disable the bug fix.
	\end{enumerate}

	Use |\pgfplotsset{compat=default}| to restore the factory settings.

	The setting |\pgfplotsset{compat=newest}| will always use features of the most recent version. This might result in small changes in the document's appearance.
\end{pgfplotskey}

\subsection{The Team}
\PGFPlots\ has been written mainly by Christian Feuers�nger with many improvements of Pascal Wolkotte and Nick Papior Andersen as a spare time project. We hope it is useful and provides valuable plots.

If you are interested in writing something but don't know how, consider reading the auxiliary manual \href{file:TeX-programming-notes.pdf}{TeX-programming-notes.pdf} which comes with \PGFPlots. It is far from complete, but maybe it is a good starting point (at least for more literature).

\subsection{Acknowledgements}
I thank God for all hours of enjoyed programming. I thank Pascal Wolkotte and Nick Papior Andersen for their programming efforts and contributions as part of the development team. I thank Stefan Tibus, who contributed the |plot shell| feature. I thank Tom Cashman for the contribution of the |reverse legend| feature. Special thanks go to Stefan Pinnow whose tests of \PGFPlots\ lead to numerous quality improvements. Furthermore, I thank Dr.~Schweitzer for many fruitful discussions and Dr.~Meine for his ideas and suggestions. Thanks as well to the many international contributors who provided feature requests or identified bugs or simply manual improvements!

Last but not least, I thank Till Tantau and Mark Wibrow for their excellent graphics (and more) package \PGF\ and \Tikz\ which is the base of \PGFPlots.

\include{pgfplots.install}%
\include{pgfplots.intro}%
\include{pgfplots.reference}%
\include{pgfplots.libs}%
\include{pgfplots.resources}%
\include{pgfplots.importexport}%
\include{pgfplots.basic.reference}%

\printindex

\bibliographystyle{abbrv} %gerapali} %gerabbrv} %gerunsrt.bst} %gerabbrv}% gerplain}
\nocite{pgfplotstable}
\nocite{programmingnotes}
\bibliography{pgfplots}
\end{document}
%
%%%%%%%%%%%%%%%%%%%%%%%%%%%%%%%%%%%%%%%%%%%%%%%%%%%%%%%%%%%%%%%%%%%%%%%%%%%%%
%
% Package pgfplots.sty documentation. 
%
% Copyright 2007/2008 by Christian Feuersaenger.
%
% This program is free software: you can redistribute it and/or modify
% it under the terms of the GNU General Public License as published by
% the Free Software Foundation, either version 3 of the License, or
% (at your option) any later version.
% 
% This program is distributed in the hope that it will be useful,
% but WITHOUT ANY WARRANTY; without even the implied warranty of
% MERCHANTABILITY or FITNESS FOR A PARTICULAR PURPOSE.  See the
% GNU General Public License for more details.
% 
% You should have received a copy of the GNU General Public License
% along with this program.  If not, see <http://www.gnu.org/licenses/>.
%
%
%%%%%%%%%%%%%%%%%%%%%%%%%%%%%%%%%%%%%%%%%%%%%%%%%%%%%%%%%%%%%%%%%%%%%%%%%%%%%
\input{pgfplots.preamble.tex}
%\RequirePackage[german,english,francais]{babel}

\pgfplotsmanualexternalexpensivetrue

%\usetikzlibrary{external}
%\tikzexternalize[prefix=figures/]{pgfplots}

\title{%
	Manual for Package \PGFPlots\\
	{\small Version \pgfplotsversion}\\
	{\small\href{http://sourceforge.net/projects/pgfplots}{http://sourceforge.net/projects/pgfplots}}}

%\includeonly{pgfplots.libs}


\begin{document}

\def\plotcoords{%
\addplot coordinates {
(5,8.312e-02)    (17,2.547e-02)   (49,7.407e-03)
(129,2.102e-03)  (321,5.874e-04)  (769,1.623e-04)
(1793,4.442e-05) (4097,1.207e-05) (9217,3.261e-06)
};

\addplot coordinates{
(7,8.472e-02)    (31,3.044e-02)    (111,1.022e-02)
(351,3.303e-03)  (1023,1.039e-03)  (2815,3.196e-04)
(7423,9.658e-05) (18943,2.873e-05) (47103,8.437e-06)
};

\addplot coordinates{
(9,7.881e-02)     (49,3.243e-02)    (209,1.232e-02)
(769,4.454e-03)   (2561,1.551e-03)  (7937,5.236e-04)
(23297,1.723e-04) (65537,5.545e-05) (178177,1.751e-05)
};

\addplot coordinates{
(11,6.887e-02)    (71,3.177e-02)     (351,1.341e-02)
(1471,5.334e-03)  (5503,2.027e-03)   (18943,7.415e-04)
(61183,2.628e-04) (187903,9.063e-05) (553983,3.053e-05)
};

\addplot coordinates{
(13,5.755e-02)     (97,2.925e-02)     (545,1.351e-02)
(2561,5.842e-03)   (10625,2.397e-03)  (40193,9.414e-04)
(141569,3.564e-04) (471041,1.308e-04) 
(1496065,4.670e-05)
};
}%
\maketitle
\begin{abstract}%
\PGFPlots\ draws high--quality function plots in normal or logarithmic scaling with a user-friendly interface directly in \TeX. The user supplies axis labels, legend entries and the plot coordinates for one or more plots and \PGFPlots\ applies axis scaling, computes any logarithms and axis ticks and draws the plots, supporting line plots, scatter plots, piecewise constant plots, bar plots, area plots, mesh-- and surface plots and some more. It is based on Till Tantau's package \PGF/\Tikz.
\end{abstract}
\tableofcontents
\section{Introduction}
This package provides tools to generate plots and labeled axes easily. It draws normal plots, logplots and semi-logplots, in two and three dimensions. Axis ticks, labels, legends (in case of multiple plots) can be added with key-value options. It can cycle through a set of predefined line/marker/color specifications. In summary, its purpose is to simplify the generation of high-quality function and/or data plots, and solving the problems of
\begin{itemize}
	\item consistency of document and font type and font size,
	\item direct use of \TeX\ math mode in axis descriptions,
	\item consistency of data and figures (no third party tool necessary),
	\item inter document consistency using preamble configurations and styles.
\end{itemize}
Although not necessary, separate |.pdf| or |.eps| graphics can be generated using the |external| library developed as part of \Tikz.

\PGFPlots\ is build completely on \Tikz/\PGF. Knowledge of \Tikz\ will simplify the work with \PGFPlots, although it is not required.

\section[About PGFPlots: Preliminaries]{About {\normalfont\PGFPlots}: Preliminaries}
This section contains information about upgrades, the team, the installation (in case you need to do it manually) and troubleshooting. You may skip it completely except for the upgrade remarks.

However, note that this library requires at least \PGF\ version $2.00$. At the time of this writing, many \TeX-distributions still contain the older \PGF\ version $1.18$, so it may be necessary to install a recent \PGF\ prior to using \PGFPlots.

\subsection{Components}
\PGFPlots\ comes with two components:
\begin{enumerate}
	\item the plotting component (which you are currently reading) and
	\item the \PGFPlotstable\ component which simplifies number formatting and postprocessing of numerical tables. It comes as a separate package and has its own manual \href{file:pgfplotstable.pdf}{pgfplotstable.pdf}.
\end{enumerate}

\subsection{Upgrade remarks}
This release provides a lot of improvements which can be found in all detail in \texttt{ChangeLog} for interested readers. However, some attention is useful with respect to the following changes.

\subsubsection{New Optional Features}
\begin{enumerate}
	\item \PGFPlots\ 1.3 comes with user interface improvements. The technical distinction between ``behavior options'' and ``style options'' of older versions is no longer necessary (although still fully supported).

	\item \PGFPlots\ 1.3 has a new feature which allows to \emph{move axis labels tight to tick labels} automatically.

	Since this affects the spacing, it is not enabled be default.

	Use
\begin{codeexample}[code only]
\usepackage{pgfplots}
\pgfplotsset{compat=1.3}
\end{codeexample}
	in your preamble to benefit from the improved spacing\footnote{The |compat=1.3| setting is necessary for new features which move axis labels around like reversed axis. However, \PGFPlots\ will still work as it does in earlier versions, your documents will remain the same if you don't set it explicitly.}.
	Take a look at the next page for the precise definition of |compat|.

	\item \PGFPlots\ 1.3 now supports reversed axes. It is no longer necessary to use work--arounds with negative units.
\pgfkeys{/pdflinks/search key prefixes in/.add={/pgfplots/,}{}}

	Take a look at the |x dir=reverse| key.

	Existing work--arounds will still function properly. Use |\pgfplotsset{compat=1.3}| together with |x dir=reverse| to switch to the new version.
\end{enumerate}

\subsubsection{Old Features Which May Need Attention}
\begin{enumerate}
	\item The |scatter/classes| feature produces proper legends as of version 1.3. This may change the appearance of existing legends of plots with |scatter/classes|.

	\item Due to a small math bug in \PGF\ $2.00$, you \emph{can't} use the math expression `|-x^2|'. It is necessary to use `|0-x^2|' instead. The same holds for
		`|exp(-x^2)|'; use `|exp(0-x^2)|' instead.

		This will be fixed with \PGF\ $>2.00$.

	\item Starting with \PGFPlots\ $1.1$, |\tikzstyle| should \emph{no longer be used} to set \PGFPlots\ options.
	
	Although |\tikzstyle| is still supported for some older \PGFPlots\ options, you should replace any occurance of |\tikzstyle| with |\pgfplotsset{|\meta{style name}|/.style={|\meta{key-value-list}|}}| or the associated |/.append style| variant. See section~\ref{sec:styles} for more detail.
\end{enumerate}
I apologize for any inconvenience caused by these changes.

\begin{pgfplotskey}{compat=\mchoice{1.3,pre 1.3,default,newest} (initially default)}
	The preamble configuration 
\begin{codeexample}[code only]
\usepackage{pgfplots}
\pgfplotsset{compat=1.3}
\end{codeexample}
	allows to choose between backwards compatibility and most recent features.

	\PGFPlots\ version 1.3 comes with new features which may lead to different spacings compared to earlier versions:
	\begin{enumerate}
		\item Version 1.3 comes with the |xlabel near ticks| feature which places (in this case $x$) axis labels ``near'' the tick labels, taking the size of tick labels into account.
		
		Older versions used |xlabel absolute|, i.e.\ an absolute distance between the axis and axis labels (now matter how large or small tick labels are).

		The initial setting \emph{keeps backwards compatibility}.

		You are encouraged to use
\begin{codeexample}[code only]
\usepackage{pgfplots}
\pgfplotsset{compat=1.3}
\end{codeexample}
		in your preamble.
		
		\item Version 1.3 fixes a bug with |anchor=south west| which occurred mainly for reversed axes: the anchors where upside--down. This is fixed now.

		
		If you have written some work--around and want to keep it going, use

		|\pgfplotsset{compat/anchors=pre 1.3}|\pgfmanualpdflabel{/pgfplots/compat/anchors}{} 

		to disable the bug fix.
	\end{enumerate}

	Use |\pgfplotsset{compat=default}| to restore the factory settings.

	The setting |\pgfplotsset{compat=newest}| will always use features of the most recent version. This might result in small changes in the document's appearance.
\end{pgfplotskey}

\subsection{The Team}
\PGFPlots\ has been written mainly by Christian Feuers�nger with many improvements of Pascal Wolkotte and Nick Papior Andersen as a spare time project. We hope it is useful and provides valuable plots.

If you are interested in writing something but don't know how, consider reading the auxiliary manual \href{file:TeX-programming-notes.pdf}{TeX-programming-notes.pdf} which comes with \PGFPlots. It is far from complete, but maybe it is a good starting point (at least for more literature).

\subsection{Acknowledgements}
I thank God for all hours of enjoyed programming. I thank Pascal Wolkotte and Nick Papior Andersen for their programming efforts and contributions as part of the development team. I thank Stefan Tibus, who contributed the |plot shell| feature. I thank Tom Cashman for the contribution of the |reverse legend| feature. Special thanks go to Stefan Pinnow whose tests of \PGFPlots\ lead to numerous quality improvements. Furthermore, I thank Dr.~Schweitzer for many fruitful discussions and Dr.~Meine for his ideas and suggestions. Thanks as well to the many international contributors who provided feature requests or identified bugs or simply manual improvements!

Last but not least, I thank Till Tantau and Mark Wibrow for their excellent graphics (and more) package \PGF\ and \Tikz\ which is the base of \PGFPlots.

\include{pgfplots.install}%
\include{pgfplots.intro}%
\include{pgfplots.reference}%
\include{pgfplots.libs}%
\include{pgfplots.resources}%
\include{pgfplots.importexport}%
\include{pgfplots.basic.reference}%

\printindex

\bibliographystyle{abbrv} %gerapali} %gerabbrv} %gerunsrt.bst} %gerabbrv}% gerplain}
\nocite{pgfplotstable}
\nocite{programmingnotes}
\bibliography{pgfplots}
\end{document}
%
\include{pgfplots.basic.reference}%

\printindex

\bibliographystyle{abbrv} %gerapali} %gerabbrv} %gerunsrt.bst} %gerabbrv}% gerplain}
\nocite{pgfplotstable}
\nocite{programmingnotes}
\bibliography{pgfplots}
\end{document}
%
%%%%%%%%%%%%%%%%%%%%%%%%%%%%%%%%%%%%%%%%%%%%%%%%%%%%%%%%%%%%%%%%%%%%%%%%%%%%%
%
% Package pgfplots.sty documentation. 
%
% Copyright 2007/2008 by Christian Feuersaenger.
%
% This program is free software: you can redistribute it and/or modify
% it under the terms of the GNU General Public License as published by
% the Free Software Foundation, either version 3 of the License, or
% (at your option) any later version.
% 
% This program is distributed in the hope that it will be useful,
% but WITHOUT ANY WARRANTY; without even the implied warranty of
% MERCHANTABILITY or FITNESS FOR A PARTICULAR PURPOSE.  See the
% GNU General Public License for more details.
% 
% You should have received a copy of the GNU General Public License
% along with this program.  If not, see <http://www.gnu.org/licenses/>.
%
%
%%%%%%%%%%%%%%%%%%%%%%%%%%%%%%%%%%%%%%%%%%%%%%%%%%%%%%%%%%%%%%%%%%%%%%%%%%%%%
%%%%%%%%%%%%%%%%%%%%%%%%%%%%%%%%%%%%%%%%%%%%%%%%%%%%%%%%%%%%%%%%%%%%%%%%%%%%%
%
% Package pgfplots.sty documentation. 
%
% Copyright 2007/2008 by Christian Feuersaenger.
%
% This program is free software: you can redistribute it and/or modify
% it under the terms of the GNU General Public License as published by
% the Free Software Foundation, either version 3 of the License, or
% (at your option) any later version.
% 
% This program is distributed in the hope that it will be useful,
% but WITHOUT ANY WARRANTY; without even the implied warranty of
% MERCHANTABILITY or FITNESS FOR A PARTICULAR PURPOSE.  See the
% GNU General Public License for more details.
% 
% You should have received a copy of the GNU General Public License
% along with this program.  If not, see <http://www.gnu.org/licenses/>.
%
%
%%%%%%%%%%%%%%%%%%%%%%%%%%%%%%%%%%%%%%%%%%%%%%%%%%%%%%%%%%%%%%%%%%%%%%%%%%%%%
\documentclass[a4paper]{ltxdoc}

\usepackage{makeidx}

% DON't let hyperref overload the format of index and glossary. 
% I want to do that on my own in the stylefiles for makeindex...
\makeatletter
\let\@old@wrindex=\@wrindex
\makeatother

\usepackage{ifpdf}
\ifpdf
	\usepackage[pdftex]{hyperref}
\else
	\usepackage[dvipdfm]{hyperref}
\fi
\hypersetup{pdfborder={0 0 0}}

\makeatletter
\let\@wrindex=\@old@wrindex
\makeatother

% Formatiere Seitennummern im Index:
\newcommand{\indexpageno}[1]{%
	{\bfseries\hyperpage{#1}}%
}


\newcommand{\R}{\mathbb{R}}
\newcommand{\N}{\mathbb{N}}
\newcommand{\Z}{\mathbb{Z}}

\long\def\COMMENTLOWLEVEL#1\ENDCOMMENT{}
\def\ENDCOMMENT{}

\usepackage{textcomp}
\usepackage{booktabs}

\usepackage{calc}
\usepackage[formats]{listings}
%\usepackage{courier} % don't use it - the '^' character can't be copy-pasted in courier

\usepackage{array}
\lstset{%
	basicstyle=\ttfamily,
	language=[LaTeX]tex, % Seems as if \lstset{language=tex} must be invoked BEFORE loading tikz!?
	tabsize=4,
	breaklines=true,
	breakindent=0pt
}

\ifpdf
	\pdfinfo {
		/Author	(Christian Feuersaenger)
	}

\else
	\def\pgfsysdriver{pgfsys-dvipdfm.def}
\fi
%\def\pgfsysdriver{pgfsys-pdftex.def}
\usepackage{pgfplots}

\ifpdf
	\usetikzlibrary{pgfplotsclickable}
\fi

%\usepackage{fp}
% ATTENTION:
% this requires pgf version NEWER than 2.00 :
%\usetikzlibrary{fixedpointarithmetic}

\usetikzlibrary{dateplot}

\usepackage[a4paper,left=2.25cm,right=2.25cm,top=2.5cm,bottom=2.5cm,nohead]{geometry}
\usepackage{amsmath,amssymb}
\usepackage{xxcolor}
\usepackage{pifont}
\usepackage[latin1]{inputenc}
\usepackage{amsmath}
\usepackage{eurosym}
\input{pgfplots-macros}

\pgfqkeys{/codeexample}{%
	every codeexample/.style={
		width=8cm,
		/pgfplots/every axis/.append style={legend style={fill=graphicbackground}}
	},
	tabsize=4,
}
\pgfplotsset{
	every axis/.append style={width=7cm},
	filter discard warning=false,
}

\usetikzlibrary{backgrounds,patterns}
% Global styles:
\tikzset{
  shape example/.style={
    color=black!30,
    draw,
    fill=yellow!30,
    line width=.5cm,
    inner xsep=2.5cm,
    inner ysep=0.5cm}
}

\newcommand{\FIXME}[1]{\textcolor{red}{(FIXME: #1)}}

% fuer endvironment 'sidewaysfigure' bspw
% \usepackage{rotating}

\newcommand\Tikz{Ti\textit kZ}
\newcommand\PGF{\textsc{pgf}}
\newcommand\PGFPlots{\textsc{pgfplots}}
\def\PGFPlotstable{\textsc{PgfplotsTable}}


\makeindex

% Fix overful hboxes automatically:
\tolerance=2000
\emergencystretch=10pt

\tikzset{prefix=gnuplot/pgfplots_} % prefix for 'plot function'

\author{%
	Christian Feuers\"anger\footnote{\url{http://wissrech.ins.uni-bonn.de/people/feuersaenger}}\\%
	Institut f\"ur Numerische Simulation\\
	Universit\"at Bonn, Germany}


%\RequirePackage[german,english,francais]{babel}

\pgfplotsmanualexternalexpensivetrue

%\usetikzlibrary{external}
%\tikzexternalize[prefix=figures/]{pgfplots}

\title{%
	Manual for Package \PGFPlots\\
	{\small Version \pgfplotsversion}\\
	{\small\href{http://sourceforge.net/projects/pgfplots}{http://sourceforge.net/projects/pgfplots}}}

%\includeonly{pgfplots.libs}


\begin{document}

\def\plotcoords{%
\addplot coordinates {
(5,8.312e-02)    (17,2.547e-02)   (49,7.407e-03)
(129,2.102e-03)  (321,5.874e-04)  (769,1.623e-04)
(1793,4.442e-05) (4097,1.207e-05) (9217,3.261e-06)
};

\addplot coordinates{
(7,8.472e-02)    (31,3.044e-02)    (111,1.022e-02)
(351,3.303e-03)  (1023,1.039e-03)  (2815,3.196e-04)
(7423,9.658e-05) (18943,2.873e-05) (47103,8.437e-06)
};

\addplot coordinates{
(9,7.881e-02)     (49,3.243e-02)    (209,1.232e-02)
(769,4.454e-03)   (2561,1.551e-03)  (7937,5.236e-04)
(23297,1.723e-04) (65537,5.545e-05) (178177,1.751e-05)
};

\addplot coordinates{
(11,6.887e-02)    (71,3.177e-02)     (351,1.341e-02)
(1471,5.334e-03)  (5503,2.027e-03)   (18943,7.415e-04)
(61183,2.628e-04) (187903,9.063e-05) (553983,3.053e-05)
};

\addplot coordinates{
(13,5.755e-02)     (97,2.925e-02)     (545,1.351e-02)
(2561,5.842e-03)   (10625,2.397e-03)  (40193,9.414e-04)
(141569,3.564e-04) (471041,1.308e-04) 
(1496065,4.670e-05)
};
}%
\maketitle
\begin{abstract}%
\PGFPlots\ draws high--quality function plots in normal or logarithmic scaling with a user-friendly interface directly in \TeX. The user supplies axis labels, legend entries and the plot coordinates for one or more plots and \PGFPlots\ applies axis scaling, computes any logarithms and axis ticks and draws the plots, supporting line plots, scatter plots, piecewise constant plots, bar plots, area plots, mesh-- and surface plots and some more. It is based on Till Tantau's package \PGF/\Tikz.
\end{abstract}
\tableofcontents
\section{Introduction}
This package provides tools to generate plots and labeled axes easily. It draws normal plots, logplots and semi-logplots, in two and three dimensions. Axis ticks, labels, legends (in case of multiple plots) can be added with key-value options. It can cycle through a set of predefined line/marker/color specifications. In summary, its purpose is to simplify the generation of high-quality function and/or data plots, and solving the problems of
\begin{itemize}
	\item consistency of document and font type and font size,
	\item direct use of \TeX\ math mode in axis descriptions,
	\item consistency of data and figures (no third party tool necessary),
	\item inter document consistency using preamble configurations and styles.
\end{itemize}
Although not necessary, separate |.pdf| or |.eps| graphics can be generated using the |external| library developed as part of \Tikz.

\PGFPlots\ is build completely on \Tikz/\PGF. Knowledge of \Tikz\ will simplify the work with \PGFPlots, although it is not required.

\section[About PGFPlots: Preliminaries]{About {\normalfont\PGFPlots}: Preliminaries}
This section contains information about upgrades, the team, the installation (in case you need to do it manually) and troubleshooting. You may skip it completely except for the upgrade remarks.

However, note that this library requires at least \PGF\ version $2.00$. At the time of this writing, many \TeX-distributions still contain the older \PGF\ version $1.18$, so it may be necessary to install a recent \PGF\ prior to using \PGFPlots.

\subsection{Components}
\PGFPlots\ comes with two components:
\begin{enumerate}
	\item the plotting component (which you are currently reading) and
	\item the \PGFPlotstable\ component which simplifies number formatting and postprocessing of numerical tables. It comes as a separate package and has its own manual \href{file:pgfplotstable.pdf}{pgfplotstable.pdf}.
\end{enumerate}

\subsection{Upgrade remarks}
This release provides a lot of improvements which can be found in all detail in \texttt{ChangeLog} for interested readers. However, some attention is useful with respect to the following changes.

\subsubsection{New Optional Features}
\begin{enumerate}
	\item \PGFPlots\ 1.3 comes with user interface improvements. The technical distinction between ``behavior options'' and ``style options'' of older versions is no longer necessary (although still fully supported).

	\item \PGFPlots\ 1.3 has a new feature which allows to \emph{move axis labels tight to tick labels} automatically.

	Since this affects the spacing, it is not enabled be default.

	Use
\begin{codeexample}[code only]
\usepackage{pgfplots}
\pgfplotsset{compat=1.3}
\end{codeexample}
	in your preamble to benefit from the improved spacing\footnote{The |compat=1.3| setting is necessary for new features which move axis labels around like reversed axis. However, \PGFPlots\ will still work as it does in earlier versions, your documents will remain the same if you don't set it explicitly.}.
	Take a look at the next page for the precise definition of |compat|.

	\item \PGFPlots\ 1.3 now supports reversed axes. It is no longer necessary to use work--arounds with negative units.
\pgfkeys{/pdflinks/search key prefixes in/.add={/pgfplots/,}{}}

	Take a look at the |x dir=reverse| key.

	Existing work--arounds will still function properly. Use |\pgfplotsset{compat=1.3}| together with |x dir=reverse| to switch to the new version.
\end{enumerate}

\subsubsection{Old Features Which May Need Attention}
\begin{enumerate}
	\item The |scatter/classes| feature produces proper legends as of version 1.3. This may change the appearance of existing legends of plots with |scatter/classes|.

	\item Due to a small math bug in \PGF\ $2.00$, you \emph{can't} use the math expression `|-x^2|'. It is necessary to use `|0-x^2|' instead. The same holds for
		`|exp(-x^2)|'; use `|exp(0-x^2)|' instead.

		This will be fixed with \PGF\ $>2.00$.

	\item Starting with \PGFPlots\ $1.1$, |\tikzstyle| should \emph{no longer be used} to set \PGFPlots\ options.
	
	Although |\tikzstyle| is still supported for some older \PGFPlots\ options, you should replace any occurance of |\tikzstyle| with |\pgfplotsset{|\meta{style name}|/.style={|\meta{key-value-list}|}}| or the associated |/.append style| variant. See section~\ref{sec:styles} for more detail.
\end{enumerate}
I apologize for any inconvenience caused by these changes.

\begin{pgfplotskey}{compat=\mchoice{1.3,pre 1.3,default,newest} (initially default)}
	The preamble configuration 
\begin{codeexample}[code only]
\usepackage{pgfplots}
\pgfplotsset{compat=1.3}
\end{codeexample}
	allows to choose between backwards compatibility and most recent features.

	\PGFPlots\ version 1.3 comes with new features which may lead to different spacings compared to earlier versions:
	\begin{enumerate}
		\item Version 1.3 comes with the |xlabel near ticks| feature which places (in this case $x$) axis labels ``near'' the tick labels, taking the size of tick labels into account.
		
		Older versions used |xlabel absolute|, i.e.\ an absolute distance between the axis and axis labels (now matter how large or small tick labels are).

		The initial setting \emph{keeps backwards compatibility}.

		You are encouraged to use
\begin{codeexample}[code only]
\usepackage{pgfplots}
\pgfplotsset{compat=1.3}
\end{codeexample}
		in your preamble.
		
		\item Version 1.3 fixes a bug with |anchor=south west| which occurred mainly for reversed axes: the anchors where upside--down. This is fixed now.

		
		If you have written some work--around and want to keep it going, use

		|\pgfplotsset{compat/anchors=pre 1.3}|\pgfmanualpdflabel{/pgfplots/compat/anchors}{} 

		to disable the bug fix.
	\end{enumerate}

	Use |\pgfplotsset{compat=default}| to restore the factory settings.

	The setting |\pgfplotsset{compat=newest}| will always use features of the most recent version. This might result in small changes in the document's appearance.
\end{pgfplotskey}

\subsection{The Team}
\PGFPlots\ has been written mainly by Christian Feuers�nger with many improvements of Pascal Wolkotte and Nick Papior Andersen as a spare time project. We hope it is useful and provides valuable plots.

If you are interested in writing something but don't know how, consider reading the auxiliary manual \href{file:TeX-programming-notes.pdf}{TeX-programming-notes.pdf} which comes with \PGFPlots. It is far from complete, but maybe it is a good starting point (at least for more literature).

\subsection{Acknowledgements}
I thank God for all hours of enjoyed programming. I thank Pascal Wolkotte and Nick Papior Andersen for their programming efforts and contributions as part of the development team. I thank Stefan Tibus, who contributed the |plot shell| feature. I thank Tom Cashman for the contribution of the |reverse legend| feature. Special thanks go to Stefan Pinnow whose tests of \PGFPlots\ lead to numerous quality improvements. Furthermore, I thank Dr.~Schweitzer for many fruitful discussions and Dr.~Meine for his ideas and suggestions. Thanks as well to the many international contributors who provided feature requests or identified bugs or simply manual improvements!

Last but not least, I thank Till Tantau and Mark Wibrow for their excellent graphics (and more) package \PGF\ and \Tikz\ which is the base of \PGFPlots.

%%%%%%%%%%%%%%%%%%%%%%%%%%%%%%%%%%%%%%%%%%%%%%%%%%%%%%%%%%%%%%%%%%%%%%%%%%%%%
%
% Package pgfplots.sty documentation. 
%
% Copyright 2007/2008 by Christian Feuersaenger.
%
% This program is free software: you can redistribute it and/or modify
% it under the terms of the GNU General Public License as published by
% the Free Software Foundation, either version 3 of the License, or
% (at your option) any later version.
% 
% This program is distributed in the hope that it will be useful,
% but WITHOUT ANY WARRANTY; without even the implied warranty of
% MERCHANTABILITY or FITNESS FOR A PARTICULAR PURPOSE.  See the
% GNU General Public License for more details.
% 
% You should have received a copy of the GNU General Public License
% along with this program.  If not, see <http://www.gnu.org/licenses/>.
%
%
%%%%%%%%%%%%%%%%%%%%%%%%%%%%%%%%%%%%%%%%%%%%%%%%%%%%%%%%%%%%%%%%%%%%%%%%%%%%%
\input{pgfplots.preamble.tex}
%\RequirePackage[german,english,francais]{babel}

\pgfplotsmanualexternalexpensivetrue

%\usetikzlibrary{external}
%\tikzexternalize[prefix=figures/]{pgfplots}

\title{%
	Manual for Package \PGFPlots\\
	{\small Version \pgfplotsversion}\\
	{\small\href{http://sourceforge.net/projects/pgfplots}{http://sourceforge.net/projects/pgfplots}}}

%\includeonly{pgfplots.libs}


\begin{document}

\def\plotcoords{%
\addplot coordinates {
(5,8.312e-02)    (17,2.547e-02)   (49,7.407e-03)
(129,2.102e-03)  (321,5.874e-04)  (769,1.623e-04)
(1793,4.442e-05) (4097,1.207e-05) (9217,3.261e-06)
};

\addplot coordinates{
(7,8.472e-02)    (31,3.044e-02)    (111,1.022e-02)
(351,3.303e-03)  (1023,1.039e-03)  (2815,3.196e-04)
(7423,9.658e-05) (18943,2.873e-05) (47103,8.437e-06)
};

\addplot coordinates{
(9,7.881e-02)     (49,3.243e-02)    (209,1.232e-02)
(769,4.454e-03)   (2561,1.551e-03)  (7937,5.236e-04)
(23297,1.723e-04) (65537,5.545e-05) (178177,1.751e-05)
};

\addplot coordinates{
(11,6.887e-02)    (71,3.177e-02)     (351,1.341e-02)
(1471,5.334e-03)  (5503,2.027e-03)   (18943,7.415e-04)
(61183,2.628e-04) (187903,9.063e-05) (553983,3.053e-05)
};

\addplot coordinates{
(13,5.755e-02)     (97,2.925e-02)     (545,1.351e-02)
(2561,5.842e-03)   (10625,2.397e-03)  (40193,9.414e-04)
(141569,3.564e-04) (471041,1.308e-04) 
(1496065,4.670e-05)
};
}%
\maketitle
\begin{abstract}%
\PGFPlots\ draws high--quality function plots in normal or logarithmic scaling with a user-friendly interface directly in \TeX. The user supplies axis labels, legend entries and the plot coordinates for one or more plots and \PGFPlots\ applies axis scaling, computes any logarithms and axis ticks and draws the plots, supporting line plots, scatter plots, piecewise constant plots, bar plots, area plots, mesh-- and surface plots and some more. It is based on Till Tantau's package \PGF/\Tikz.
\end{abstract}
\tableofcontents
\section{Introduction}
This package provides tools to generate plots and labeled axes easily. It draws normal plots, logplots and semi-logplots, in two and three dimensions. Axis ticks, labels, legends (in case of multiple plots) can be added with key-value options. It can cycle through a set of predefined line/marker/color specifications. In summary, its purpose is to simplify the generation of high-quality function and/or data plots, and solving the problems of
\begin{itemize}
	\item consistency of document and font type and font size,
	\item direct use of \TeX\ math mode in axis descriptions,
	\item consistency of data and figures (no third party tool necessary),
	\item inter document consistency using preamble configurations and styles.
\end{itemize}
Although not necessary, separate |.pdf| or |.eps| graphics can be generated using the |external| library developed as part of \Tikz.

\PGFPlots\ is build completely on \Tikz/\PGF. Knowledge of \Tikz\ will simplify the work with \PGFPlots, although it is not required.

\section[About PGFPlots: Preliminaries]{About {\normalfont\PGFPlots}: Preliminaries}
This section contains information about upgrades, the team, the installation (in case you need to do it manually) and troubleshooting. You may skip it completely except for the upgrade remarks.

However, note that this library requires at least \PGF\ version $2.00$. At the time of this writing, many \TeX-distributions still contain the older \PGF\ version $1.18$, so it may be necessary to install a recent \PGF\ prior to using \PGFPlots.

\subsection{Components}
\PGFPlots\ comes with two components:
\begin{enumerate}
	\item the plotting component (which you are currently reading) and
	\item the \PGFPlotstable\ component which simplifies number formatting and postprocessing of numerical tables. It comes as a separate package and has its own manual \href{file:pgfplotstable.pdf}{pgfplotstable.pdf}.
\end{enumerate}

\subsection{Upgrade remarks}
This release provides a lot of improvements which can be found in all detail in \texttt{ChangeLog} for interested readers. However, some attention is useful with respect to the following changes.

\subsubsection{New Optional Features}
\begin{enumerate}
	\item \PGFPlots\ 1.3 comes with user interface improvements. The technical distinction between ``behavior options'' and ``style options'' of older versions is no longer necessary (although still fully supported).

	\item \PGFPlots\ 1.3 has a new feature which allows to \emph{move axis labels tight to tick labels} automatically.

	Since this affects the spacing, it is not enabled be default.

	Use
\begin{codeexample}[code only]
\usepackage{pgfplots}
\pgfplotsset{compat=1.3}
\end{codeexample}
	in your preamble to benefit from the improved spacing\footnote{The |compat=1.3| setting is necessary for new features which move axis labels around like reversed axis. However, \PGFPlots\ will still work as it does in earlier versions, your documents will remain the same if you don't set it explicitly.}.
	Take a look at the next page for the precise definition of |compat|.

	\item \PGFPlots\ 1.3 now supports reversed axes. It is no longer necessary to use work--arounds with negative units.
\pgfkeys{/pdflinks/search key prefixes in/.add={/pgfplots/,}{}}

	Take a look at the |x dir=reverse| key.

	Existing work--arounds will still function properly. Use |\pgfplotsset{compat=1.3}| together with |x dir=reverse| to switch to the new version.
\end{enumerate}

\subsubsection{Old Features Which May Need Attention}
\begin{enumerate}
	\item The |scatter/classes| feature produces proper legends as of version 1.3. This may change the appearance of existing legends of plots with |scatter/classes|.

	\item Due to a small math bug in \PGF\ $2.00$, you \emph{can't} use the math expression `|-x^2|'. It is necessary to use `|0-x^2|' instead. The same holds for
		`|exp(-x^2)|'; use `|exp(0-x^2)|' instead.

		This will be fixed with \PGF\ $>2.00$.

	\item Starting with \PGFPlots\ $1.1$, |\tikzstyle| should \emph{no longer be used} to set \PGFPlots\ options.
	
	Although |\tikzstyle| is still supported for some older \PGFPlots\ options, you should replace any occurance of |\tikzstyle| with |\pgfplotsset{|\meta{style name}|/.style={|\meta{key-value-list}|}}| or the associated |/.append style| variant. See section~\ref{sec:styles} for more detail.
\end{enumerate}
I apologize for any inconvenience caused by these changes.

\begin{pgfplotskey}{compat=\mchoice{1.3,pre 1.3,default,newest} (initially default)}
	The preamble configuration 
\begin{codeexample}[code only]
\usepackage{pgfplots}
\pgfplotsset{compat=1.3}
\end{codeexample}
	allows to choose between backwards compatibility and most recent features.

	\PGFPlots\ version 1.3 comes with new features which may lead to different spacings compared to earlier versions:
	\begin{enumerate}
		\item Version 1.3 comes with the |xlabel near ticks| feature which places (in this case $x$) axis labels ``near'' the tick labels, taking the size of tick labels into account.
		
		Older versions used |xlabel absolute|, i.e.\ an absolute distance between the axis and axis labels (now matter how large or small tick labels are).

		The initial setting \emph{keeps backwards compatibility}.

		You are encouraged to use
\begin{codeexample}[code only]
\usepackage{pgfplots}
\pgfplotsset{compat=1.3}
\end{codeexample}
		in your preamble.
		
		\item Version 1.3 fixes a bug with |anchor=south west| which occurred mainly for reversed axes: the anchors where upside--down. This is fixed now.

		
		If you have written some work--around and want to keep it going, use

		|\pgfplotsset{compat/anchors=pre 1.3}|\pgfmanualpdflabel{/pgfplots/compat/anchors}{} 

		to disable the bug fix.
	\end{enumerate}

	Use |\pgfplotsset{compat=default}| to restore the factory settings.

	The setting |\pgfplotsset{compat=newest}| will always use features of the most recent version. This might result in small changes in the document's appearance.
\end{pgfplotskey}

\subsection{The Team}
\PGFPlots\ has been written mainly by Christian Feuers�nger with many improvements of Pascal Wolkotte and Nick Papior Andersen as a spare time project. We hope it is useful and provides valuable plots.

If you are interested in writing something but don't know how, consider reading the auxiliary manual \href{file:TeX-programming-notes.pdf}{TeX-programming-notes.pdf} which comes with \PGFPlots. It is far from complete, but maybe it is a good starting point (at least for more literature).

\subsection{Acknowledgements}
I thank God for all hours of enjoyed programming. I thank Pascal Wolkotte and Nick Papior Andersen for their programming efforts and contributions as part of the development team. I thank Stefan Tibus, who contributed the |plot shell| feature. I thank Tom Cashman for the contribution of the |reverse legend| feature. Special thanks go to Stefan Pinnow whose tests of \PGFPlots\ lead to numerous quality improvements. Furthermore, I thank Dr.~Schweitzer for many fruitful discussions and Dr.~Meine for his ideas and suggestions. Thanks as well to the many international contributors who provided feature requests or identified bugs or simply manual improvements!

Last but not least, I thank Till Tantau and Mark Wibrow for their excellent graphics (and more) package \PGF\ and \Tikz\ which is the base of \PGFPlots.

\include{pgfplots.install}%
\include{pgfplots.intro}%
\include{pgfplots.reference}%
\include{pgfplots.libs}%
\include{pgfplots.resources}%
\include{pgfplots.importexport}%
\include{pgfplots.basic.reference}%

\printindex

\bibliographystyle{abbrv} %gerapali} %gerabbrv} %gerunsrt.bst} %gerabbrv}% gerplain}
\nocite{pgfplotstable}
\nocite{programmingnotes}
\bibliography{pgfplots}
\end{document}
%
%%%%%%%%%%%%%%%%%%%%%%%%%%%%%%%%%%%%%%%%%%%%%%%%%%%%%%%%%%%%%%%%%%%%%%%%%%%%%
%
% Package pgfplots.sty documentation. 
%
% Copyright 2007/2008 by Christian Feuersaenger.
%
% This program is free software: you can redistribute it and/or modify
% it under the terms of the GNU General Public License as published by
% the Free Software Foundation, either version 3 of the License, or
% (at your option) any later version.
% 
% This program is distributed in the hope that it will be useful,
% but WITHOUT ANY WARRANTY; without even the implied warranty of
% MERCHANTABILITY or FITNESS FOR A PARTICULAR PURPOSE.  See the
% GNU General Public License for more details.
% 
% You should have received a copy of the GNU General Public License
% along with this program.  If not, see <http://www.gnu.org/licenses/>.
%
%
%%%%%%%%%%%%%%%%%%%%%%%%%%%%%%%%%%%%%%%%%%%%%%%%%%%%%%%%%%%%%%%%%%%%%%%%%%%%%
\input{pgfplots.preamble.tex}
%\RequirePackage[german,english,francais]{babel}

\pgfplotsmanualexternalexpensivetrue

%\usetikzlibrary{external}
%\tikzexternalize[prefix=figures/]{pgfplots}

\title{%
	Manual for Package \PGFPlots\\
	{\small Version \pgfplotsversion}\\
	{\small\href{http://sourceforge.net/projects/pgfplots}{http://sourceforge.net/projects/pgfplots}}}

%\includeonly{pgfplots.libs}


\begin{document}

\def\plotcoords{%
\addplot coordinates {
(5,8.312e-02)    (17,2.547e-02)   (49,7.407e-03)
(129,2.102e-03)  (321,5.874e-04)  (769,1.623e-04)
(1793,4.442e-05) (4097,1.207e-05) (9217,3.261e-06)
};

\addplot coordinates{
(7,8.472e-02)    (31,3.044e-02)    (111,1.022e-02)
(351,3.303e-03)  (1023,1.039e-03)  (2815,3.196e-04)
(7423,9.658e-05) (18943,2.873e-05) (47103,8.437e-06)
};

\addplot coordinates{
(9,7.881e-02)     (49,3.243e-02)    (209,1.232e-02)
(769,4.454e-03)   (2561,1.551e-03)  (7937,5.236e-04)
(23297,1.723e-04) (65537,5.545e-05) (178177,1.751e-05)
};

\addplot coordinates{
(11,6.887e-02)    (71,3.177e-02)     (351,1.341e-02)
(1471,5.334e-03)  (5503,2.027e-03)   (18943,7.415e-04)
(61183,2.628e-04) (187903,9.063e-05) (553983,3.053e-05)
};

\addplot coordinates{
(13,5.755e-02)     (97,2.925e-02)     (545,1.351e-02)
(2561,5.842e-03)   (10625,2.397e-03)  (40193,9.414e-04)
(141569,3.564e-04) (471041,1.308e-04) 
(1496065,4.670e-05)
};
}%
\maketitle
\begin{abstract}%
\PGFPlots\ draws high--quality function plots in normal or logarithmic scaling with a user-friendly interface directly in \TeX. The user supplies axis labels, legend entries and the plot coordinates for one or more plots and \PGFPlots\ applies axis scaling, computes any logarithms and axis ticks and draws the plots, supporting line plots, scatter plots, piecewise constant plots, bar plots, area plots, mesh-- and surface plots and some more. It is based on Till Tantau's package \PGF/\Tikz.
\end{abstract}
\tableofcontents
\section{Introduction}
This package provides tools to generate plots and labeled axes easily. It draws normal plots, logplots and semi-logplots, in two and three dimensions. Axis ticks, labels, legends (in case of multiple plots) can be added with key-value options. It can cycle through a set of predefined line/marker/color specifications. In summary, its purpose is to simplify the generation of high-quality function and/or data plots, and solving the problems of
\begin{itemize}
	\item consistency of document and font type and font size,
	\item direct use of \TeX\ math mode in axis descriptions,
	\item consistency of data and figures (no third party tool necessary),
	\item inter document consistency using preamble configurations and styles.
\end{itemize}
Although not necessary, separate |.pdf| or |.eps| graphics can be generated using the |external| library developed as part of \Tikz.

\PGFPlots\ is build completely on \Tikz/\PGF. Knowledge of \Tikz\ will simplify the work with \PGFPlots, although it is not required.

\section[About PGFPlots: Preliminaries]{About {\normalfont\PGFPlots}: Preliminaries}
This section contains information about upgrades, the team, the installation (in case you need to do it manually) and troubleshooting. You may skip it completely except for the upgrade remarks.

However, note that this library requires at least \PGF\ version $2.00$. At the time of this writing, many \TeX-distributions still contain the older \PGF\ version $1.18$, so it may be necessary to install a recent \PGF\ prior to using \PGFPlots.

\subsection{Components}
\PGFPlots\ comes with two components:
\begin{enumerate}
	\item the plotting component (which you are currently reading) and
	\item the \PGFPlotstable\ component which simplifies number formatting and postprocessing of numerical tables. It comes as a separate package and has its own manual \href{file:pgfplotstable.pdf}{pgfplotstable.pdf}.
\end{enumerate}

\subsection{Upgrade remarks}
This release provides a lot of improvements which can be found in all detail in \texttt{ChangeLog} for interested readers. However, some attention is useful with respect to the following changes.

\subsubsection{New Optional Features}
\begin{enumerate}
	\item \PGFPlots\ 1.3 comes with user interface improvements. The technical distinction between ``behavior options'' and ``style options'' of older versions is no longer necessary (although still fully supported).

	\item \PGFPlots\ 1.3 has a new feature which allows to \emph{move axis labels tight to tick labels} automatically.

	Since this affects the spacing, it is not enabled be default.

	Use
\begin{codeexample}[code only]
\usepackage{pgfplots}
\pgfplotsset{compat=1.3}
\end{codeexample}
	in your preamble to benefit from the improved spacing\footnote{The |compat=1.3| setting is necessary for new features which move axis labels around like reversed axis. However, \PGFPlots\ will still work as it does in earlier versions, your documents will remain the same if you don't set it explicitly.}.
	Take a look at the next page for the precise definition of |compat|.

	\item \PGFPlots\ 1.3 now supports reversed axes. It is no longer necessary to use work--arounds with negative units.
\pgfkeys{/pdflinks/search key prefixes in/.add={/pgfplots/,}{}}

	Take a look at the |x dir=reverse| key.

	Existing work--arounds will still function properly. Use |\pgfplotsset{compat=1.3}| together with |x dir=reverse| to switch to the new version.
\end{enumerate}

\subsubsection{Old Features Which May Need Attention}
\begin{enumerate}
	\item The |scatter/classes| feature produces proper legends as of version 1.3. This may change the appearance of existing legends of plots with |scatter/classes|.

	\item Due to a small math bug in \PGF\ $2.00$, you \emph{can't} use the math expression `|-x^2|'. It is necessary to use `|0-x^2|' instead. The same holds for
		`|exp(-x^2)|'; use `|exp(0-x^2)|' instead.

		This will be fixed with \PGF\ $>2.00$.

	\item Starting with \PGFPlots\ $1.1$, |\tikzstyle| should \emph{no longer be used} to set \PGFPlots\ options.
	
	Although |\tikzstyle| is still supported for some older \PGFPlots\ options, you should replace any occurance of |\tikzstyle| with |\pgfplotsset{|\meta{style name}|/.style={|\meta{key-value-list}|}}| or the associated |/.append style| variant. See section~\ref{sec:styles} for more detail.
\end{enumerate}
I apologize for any inconvenience caused by these changes.

\begin{pgfplotskey}{compat=\mchoice{1.3,pre 1.3,default,newest} (initially default)}
	The preamble configuration 
\begin{codeexample}[code only]
\usepackage{pgfplots}
\pgfplotsset{compat=1.3}
\end{codeexample}
	allows to choose between backwards compatibility and most recent features.

	\PGFPlots\ version 1.3 comes with new features which may lead to different spacings compared to earlier versions:
	\begin{enumerate}
		\item Version 1.3 comes with the |xlabel near ticks| feature which places (in this case $x$) axis labels ``near'' the tick labels, taking the size of tick labels into account.
		
		Older versions used |xlabel absolute|, i.e.\ an absolute distance between the axis and axis labels (now matter how large or small tick labels are).

		The initial setting \emph{keeps backwards compatibility}.

		You are encouraged to use
\begin{codeexample}[code only]
\usepackage{pgfplots}
\pgfplotsset{compat=1.3}
\end{codeexample}
		in your preamble.
		
		\item Version 1.3 fixes a bug with |anchor=south west| which occurred mainly for reversed axes: the anchors where upside--down. This is fixed now.

		
		If you have written some work--around and want to keep it going, use

		|\pgfplotsset{compat/anchors=pre 1.3}|\pgfmanualpdflabel{/pgfplots/compat/anchors}{} 

		to disable the bug fix.
	\end{enumerate}

	Use |\pgfplotsset{compat=default}| to restore the factory settings.

	The setting |\pgfplotsset{compat=newest}| will always use features of the most recent version. This might result in small changes in the document's appearance.
\end{pgfplotskey}

\subsection{The Team}
\PGFPlots\ has been written mainly by Christian Feuers�nger with many improvements of Pascal Wolkotte and Nick Papior Andersen as a spare time project. We hope it is useful and provides valuable plots.

If you are interested in writing something but don't know how, consider reading the auxiliary manual \href{file:TeX-programming-notes.pdf}{TeX-programming-notes.pdf} which comes with \PGFPlots. It is far from complete, but maybe it is a good starting point (at least for more literature).

\subsection{Acknowledgements}
I thank God for all hours of enjoyed programming. I thank Pascal Wolkotte and Nick Papior Andersen for their programming efforts and contributions as part of the development team. I thank Stefan Tibus, who contributed the |plot shell| feature. I thank Tom Cashman for the contribution of the |reverse legend| feature. Special thanks go to Stefan Pinnow whose tests of \PGFPlots\ lead to numerous quality improvements. Furthermore, I thank Dr.~Schweitzer for many fruitful discussions and Dr.~Meine for his ideas and suggestions. Thanks as well to the many international contributors who provided feature requests or identified bugs or simply manual improvements!

Last but not least, I thank Till Tantau and Mark Wibrow for their excellent graphics (and more) package \PGF\ and \Tikz\ which is the base of \PGFPlots.

\include{pgfplots.install}%
\include{pgfplots.intro}%
\include{pgfplots.reference}%
\include{pgfplots.libs}%
\include{pgfplots.resources}%
\include{pgfplots.importexport}%
\include{pgfplots.basic.reference}%

\printindex

\bibliographystyle{abbrv} %gerapali} %gerabbrv} %gerunsrt.bst} %gerabbrv}% gerplain}
\nocite{pgfplotstable}
\nocite{programmingnotes}
\bibliography{pgfplots}
\end{document}
%
%%%%%%%%%%%%%%%%%%%%%%%%%%%%%%%%%%%%%%%%%%%%%%%%%%%%%%%%%%%%%%%%%%%%%%%%%%%%%
%
% Package pgfplots.sty documentation. 
%
% Copyright 2007/2008 by Christian Feuersaenger.
%
% This program is free software: you can redistribute it and/or modify
% it under the terms of the GNU General Public License as published by
% the Free Software Foundation, either version 3 of the License, or
% (at your option) any later version.
% 
% This program is distributed in the hope that it will be useful,
% but WITHOUT ANY WARRANTY; without even the implied warranty of
% MERCHANTABILITY or FITNESS FOR A PARTICULAR PURPOSE.  See the
% GNU General Public License for more details.
% 
% You should have received a copy of the GNU General Public License
% along with this program.  If not, see <http://www.gnu.org/licenses/>.
%
%
%%%%%%%%%%%%%%%%%%%%%%%%%%%%%%%%%%%%%%%%%%%%%%%%%%%%%%%%%%%%%%%%%%%%%%%%%%%%%
\input{pgfplots.preamble.tex}
%\RequirePackage[german,english,francais]{babel}

\pgfplotsmanualexternalexpensivetrue

%\usetikzlibrary{external}
%\tikzexternalize[prefix=figures/]{pgfplots}

\title{%
	Manual for Package \PGFPlots\\
	{\small Version \pgfplotsversion}\\
	{\small\href{http://sourceforge.net/projects/pgfplots}{http://sourceforge.net/projects/pgfplots}}}

%\includeonly{pgfplots.libs}


\begin{document}

\def\plotcoords{%
\addplot coordinates {
(5,8.312e-02)    (17,2.547e-02)   (49,7.407e-03)
(129,2.102e-03)  (321,5.874e-04)  (769,1.623e-04)
(1793,4.442e-05) (4097,1.207e-05) (9217,3.261e-06)
};

\addplot coordinates{
(7,8.472e-02)    (31,3.044e-02)    (111,1.022e-02)
(351,3.303e-03)  (1023,1.039e-03)  (2815,3.196e-04)
(7423,9.658e-05) (18943,2.873e-05) (47103,8.437e-06)
};

\addplot coordinates{
(9,7.881e-02)     (49,3.243e-02)    (209,1.232e-02)
(769,4.454e-03)   (2561,1.551e-03)  (7937,5.236e-04)
(23297,1.723e-04) (65537,5.545e-05) (178177,1.751e-05)
};

\addplot coordinates{
(11,6.887e-02)    (71,3.177e-02)     (351,1.341e-02)
(1471,5.334e-03)  (5503,2.027e-03)   (18943,7.415e-04)
(61183,2.628e-04) (187903,9.063e-05) (553983,3.053e-05)
};

\addplot coordinates{
(13,5.755e-02)     (97,2.925e-02)     (545,1.351e-02)
(2561,5.842e-03)   (10625,2.397e-03)  (40193,9.414e-04)
(141569,3.564e-04) (471041,1.308e-04) 
(1496065,4.670e-05)
};
}%
\maketitle
\begin{abstract}%
\PGFPlots\ draws high--quality function plots in normal or logarithmic scaling with a user-friendly interface directly in \TeX. The user supplies axis labels, legend entries and the plot coordinates for one or more plots and \PGFPlots\ applies axis scaling, computes any logarithms and axis ticks and draws the plots, supporting line plots, scatter plots, piecewise constant plots, bar plots, area plots, mesh-- and surface plots and some more. It is based on Till Tantau's package \PGF/\Tikz.
\end{abstract}
\tableofcontents
\section{Introduction}
This package provides tools to generate plots and labeled axes easily. It draws normal plots, logplots and semi-logplots, in two and three dimensions. Axis ticks, labels, legends (in case of multiple plots) can be added with key-value options. It can cycle through a set of predefined line/marker/color specifications. In summary, its purpose is to simplify the generation of high-quality function and/or data plots, and solving the problems of
\begin{itemize}
	\item consistency of document and font type and font size,
	\item direct use of \TeX\ math mode in axis descriptions,
	\item consistency of data and figures (no third party tool necessary),
	\item inter document consistency using preamble configurations and styles.
\end{itemize}
Although not necessary, separate |.pdf| or |.eps| graphics can be generated using the |external| library developed as part of \Tikz.

\PGFPlots\ is build completely on \Tikz/\PGF. Knowledge of \Tikz\ will simplify the work with \PGFPlots, although it is not required.

\section[About PGFPlots: Preliminaries]{About {\normalfont\PGFPlots}: Preliminaries}
This section contains information about upgrades, the team, the installation (in case you need to do it manually) and troubleshooting. You may skip it completely except for the upgrade remarks.

However, note that this library requires at least \PGF\ version $2.00$. At the time of this writing, many \TeX-distributions still contain the older \PGF\ version $1.18$, so it may be necessary to install a recent \PGF\ prior to using \PGFPlots.

\subsection{Components}
\PGFPlots\ comes with two components:
\begin{enumerate}
	\item the plotting component (which you are currently reading) and
	\item the \PGFPlotstable\ component which simplifies number formatting and postprocessing of numerical tables. It comes as a separate package and has its own manual \href{file:pgfplotstable.pdf}{pgfplotstable.pdf}.
\end{enumerate}

\subsection{Upgrade remarks}
This release provides a lot of improvements which can be found in all detail in \texttt{ChangeLog} for interested readers. However, some attention is useful with respect to the following changes.

\subsubsection{New Optional Features}
\begin{enumerate}
	\item \PGFPlots\ 1.3 comes with user interface improvements. The technical distinction between ``behavior options'' and ``style options'' of older versions is no longer necessary (although still fully supported).

	\item \PGFPlots\ 1.3 has a new feature which allows to \emph{move axis labels tight to tick labels} automatically.

	Since this affects the spacing, it is not enabled be default.

	Use
\begin{codeexample}[code only]
\usepackage{pgfplots}
\pgfplotsset{compat=1.3}
\end{codeexample}
	in your preamble to benefit from the improved spacing\footnote{The |compat=1.3| setting is necessary for new features which move axis labels around like reversed axis. However, \PGFPlots\ will still work as it does in earlier versions, your documents will remain the same if you don't set it explicitly.}.
	Take a look at the next page for the precise definition of |compat|.

	\item \PGFPlots\ 1.3 now supports reversed axes. It is no longer necessary to use work--arounds with negative units.
\pgfkeys{/pdflinks/search key prefixes in/.add={/pgfplots/,}{}}

	Take a look at the |x dir=reverse| key.

	Existing work--arounds will still function properly. Use |\pgfplotsset{compat=1.3}| together with |x dir=reverse| to switch to the new version.
\end{enumerate}

\subsubsection{Old Features Which May Need Attention}
\begin{enumerate}
	\item The |scatter/classes| feature produces proper legends as of version 1.3. This may change the appearance of existing legends of plots with |scatter/classes|.

	\item Due to a small math bug in \PGF\ $2.00$, you \emph{can't} use the math expression `|-x^2|'. It is necessary to use `|0-x^2|' instead. The same holds for
		`|exp(-x^2)|'; use `|exp(0-x^2)|' instead.

		This will be fixed with \PGF\ $>2.00$.

	\item Starting with \PGFPlots\ $1.1$, |\tikzstyle| should \emph{no longer be used} to set \PGFPlots\ options.
	
	Although |\tikzstyle| is still supported for some older \PGFPlots\ options, you should replace any occurance of |\tikzstyle| with |\pgfplotsset{|\meta{style name}|/.style={|\meta{key-value-list}|}}| or the associated |/.append style| variant. See section~\ref{sec:styles} for more detail.
\end{enumerate}
I apologize for any inconvenience caused by these changes.

\begin{pgfplotskey}{compat=\mchoice{1.3,pre 1.3,default,newest} (initially default)}
	The preamble configuration 
\begin{codeexample}[code only]
\usepackage{pgfplots}
\pgfplotsset{compat=1.3}
\end{codeexample}
	allows to choose between backwards compatibility and most recent features.

	\PGFPlots\ version 1.3 comes with new features which may lead to different spacings compared to earlier versions:
	\begin{enumerate}
		\item Version 1.3 comes with the |xlabel near ticks| feature which places (in this case $x$) axis labels ``near'' the tick labels, taking the size of tick labels into account.
		
		Older versions used |xlabel absolute|, i.e.\ an absolute distance between the axis and axis labels (now matter how large or small tick labels are).

		The initial setting \emph{keeps backwards compatibility}.

		You are encouraged to use
\begin{codeexample}[code only]
\usepackage{pgfplots}
\pgfplotsset{compat=1.3}
\end{codeexample}
		in your preamble.
		
		\item Version 1.3 fixes a bug with |anchor=south west| which occurred mainly for reversed axes: the anchors where upside--down. This is fixed now.

		
		If you have written some work--around and want to keep it going, use

		|\pgfplotsset{compat/anchors=pre 1.3}|\pgfmanualpdflabel{/pgfplots/compat/anchors}{} 

		to disable the bug fix.
	\end{enumerate}

	Use |\pgfplotsset{compat=default}| to restore the factory settings.

	The setting |\pgfplotsset{compat=newest}| will always use features of the most recent version. This might result in small changes in the document's appearance.
\end{pgfplotskey}

\subsection{The Team}
\PGFPlots\ has been written mainly by Christian Feuers�nger with many improvements of Pascal Wolkotte and Nick Papior Andersen as a spare time project. We hope it is useful and provides valuable plots.

If you are interested in writing something but don't know how, consider reading the auxiliary manual \href{file:TeX-programming-notes.pdf}{TeX-programming-notes.pdf} which comes with \PGFPlots. It is far from complete, but maybe it is a good starting point (at least for more literature).

\subsection{Acknowledgements}
I thank God for all hours of enjoyed programming. I thank Pascal Wolkotte and Nick Papior Andersen for their programming efforts and contributions as part of the development team. I thank Stefan Tibus, who contributed the |plot shell| feature. I thank Tom Cashman for the contribution of the |reverse legend| feature. Special thanks go to Stefan Pinnow whose tests of \PGFPlots\ lead to numerous quality improvements. Furthermore, I thank Dr.~Schweitzer for many fruitful discussions and Dr.~Meine for his ideas and suggestions. Thanks as well to the many international contributors who provided feature requests or identified bugs or simply manual improvements!

Last but not least, I thank Till Tantau and Mark Wibrow for their excellent graphics (and more) package \PGF\ and \Tikz\ which is the base of \PGFPlots.

\include{pgfplots.install}%
\include{pgfplots.intro}%
\include{pgfplots.reference}%
\include{pgfplots.libs}%
\include{pgfplots.resources}%
\include{pgfplots.importexport}%
\include{pgfplots.basic.reference}%

\printindex

\bibliographystyle{abbrv} %gerapali} %gerabbrv} %gerunsrt.bst} %gerabbrv}% gerplain}
\nocite{pgfplotstable}
\nocite{programmingnotes}
\bibliography{pgfplots}
\end{document}
%
%%%%%%%%%%%%%%%%%%%%%%%%%%%%%%%%%%%%%%%%%%%%%%%%%%%%%%%%%%%%%%%%%%%%%%%%%%%%%
%
% Package pgfplots.sty documentation. 
%
% Copyright 2007/2008 by Christian Feuersaenger.
%
% This program is free software: you can redistribute it and/or modify
% it under the terms of the GNU General Public License as published by
% the Free Software Foundation, either version 3 of the License, or
% (at your option) any later version.
% 
% This program is distributed in the hope that it will be useful,
% but WITHOUT ANY WARRANTY; without even the implied warranty of
% MERCHANTABILITY or FITNESS FOR A PARTICULAR PURPOSE.  See the
% GNU General Public License for more details.
% 
% You should have received a copy of the GNU General Public License
% along with this program.  If not, see <http://www.gnu.org/licenses/>.
%
%
%%%%%%%%%%%%%%%%%%%%%%%%%%%%%%%%%%%%%%%%%%%%%%%%%%%%%%%%%%%%%%%%%%%%%%%%%%%%%
\input{pgfplots.preamble.tex}
%\RequirePackage[german,english,francais]{babel}

\pgfplotsmanualexternalexpensivetrue

%\usetikzlibrary{external}
%\tikzexternalize[prefix=figures/]{pgfplots}

\title{%
	Manual for Package \PGFPlots\\
	{\small Version \pgfplotsversion}\\
	{\small\href{http://sourceforge.net/projects/pgfplots}{http://sourceforge.net/projects/pgfplots}}}

%\includeonly{pgfplots.libs}


\begin{document}

\def\plotcoords{%
\addplot coordinates {
(5,8.312e-02)    (17,2.547e-02)   (49,7.407e-03)
(129,2.102e-03)  (321,5.874e-04)  (769,1.623e-04)
(1793,4.442e-05) (4097,1.207e-05) (9217,3.261e-06)
};

\addplot coordinates{
(7,8.472e-02)    (31,3.044e-02)    (111,1.022e-02)
(351,3.303e-03)  (1023,1.039e-03)  (2815,3.196e-04)
(7423,9.658e-05) (18943,2.873e-05) (47103,8.437e-06)
};

\addplot coordinates{
(9,7.881e-02)     (49,3.243e-02)    (209,1.232e-02)
(769,4.454e-03)   (2561,1.551e-03)  (7937,5.236e-04)
(23297,1.723e-04) (65537,5.545e-05) (178177,1.751e-05)
};

\addplot coordinates{
(11,6.887e-02)    (71,3.177e-02)     (351,1.341e-02)
(1471,5.334e-03)  (5503,2.027e-03)   (18943,7.415e-04)
(61183,2.628e-04) (187903,9.063e-05) (553983,3.053e-05)
};

\addplot coordinates{
(13,5.755e-02)     (97,2.925e-02)     (545,1.351e-02)
(2561,5.842e-03)   (10625,2.397e-03)  (40193,9.414e-04)
(141569,3.564e-04) (471041,1.308e-04) 
(1496065,4.670e-05)
};
}%
\maketitle
\begin{abstract}%
\PGFPlots\ draws high--quality function plots in normal or logarithmic scaling with a user-friendly interface directly in \TeX. The user supplies axis labels, legend entries and the plot coordinates for one or more plots and \PGFPlots\ applies axis scaling, computes any logarithms and axis ticks and draws the plots, supporting line plots, scatter plots, piecewise constant plots, bar plots, area plots, mesh-- and surface plots and some more. It is based on Till Tantau's package \PGF/\Tikz.
\end{abstract}
\tableofcontents
\section{Introduction}
This package provides tools to generate plots and labeled axes easily. It draws normal plots, logplots and semi-logplots, in two and three dimensions. Axis ticks, labels, legends (in case of multiple plots) can be added with key-value options. It can cycle through a set of predefined line/marker/color specifications. In summary, its purpose is to simplify the generation of high-quality function and/or data plots, and solving the problems of
\begin{itemize}
	\item consistency of document and font type and font size,
	\item direct use of \TeX\ math mode in axis descriptions,
	\item consistency of data and figures (no third party tool necessary),
	\item inter document consistency using preamble configurations and styles.
\end{itemize}
Although not necessary, separate |.pdf| or |.eps| graphics can be generated using the |external| library developed as part of \Tikz.

\PGFPlots\ is build completely on \Tikz/\PGF. Knowledge of \Tikz\ will simplify the work with \PGFPlots, although it is not required.

\section[About PGFPlots: Preliminaries]{About {\normalfont\PGFPlots}: Preliminaries}
This section contains information about upgrades, the team, the installation (in case you need to do it manually) and troubleshooting. You may skip it completely except for the upgrade remarks.

However, note that this library requires at least \PGF\ version $2.00$. At the time of this writing, many \TeX-distributions still contain the older \PGF\ version $1.18$, so it may be necessary to install a recent \PGF\ prior to using \PGFPlots.

\subsection{Components}
\PGFPlots\ comes with two components:
\begin{enumerate}
	\item the plotting component (which you are currently reading) and
	\item the \PGFPlotstable\ component which simplifies number formatting and postprocessing of numerical tables. It comes as a separate package and has its own manual \href{file:pgfplotstable.pdf}{pgfplotstable.pdf}.
\end{enumerate}

\subsection{Upgrade remarks}
This release provides a lot of improvements which can be found in all detail in \texttt{ChangeLog} for interested readers. However, some attention is useful with respect to the following changes.

\subsubsection{New Optional Features}
\begin{enumerate}
	\item \PGFPlots\ 1.3 comes with user interface improvements. The technical distinction between ``behavior options'' and ``style options'' of older versions is no longer necessary (although still fully supported).

	\item \PGFPlots\ 1.3 has a new feature which allows to \emph{move axis labels tight to tick labels} automatically.

	Since this affects the spacing, it is not enabled be default.

	Use
\begin{codeexample}[code only]
\usepackage{pgfplots}
\pgfplotsset{compat=1.3}
\end{codeexample}
	in your preamble to benefit from the improved spacing\footnote{The |compat=1.3| setting is necessary for new features which move axis labels around like reversed axis. However, \PGFPlots\ will still work as it does in earlier versions, your documents will remain the same if you don't set it explicitly.}.
	Take a look at the next page for the precise definition of |compat|.

	\item \PGFPlots\ 1.3 now supports reversed axes. It is no longer necessary to use work--arounds with negative units.
\pgfkeys{/pdflinks/search key prefixes in/.add={/pgfplots/,}{}}

	Take a look at the |x dir=reverse| key.

	Existing work--arounds will still function properly. Use |\pgfplotsset{compat=1.3}| together with |x dir=reverse| to switch to the new version.
\end{enumerate}

\subsubsection{Old Features Which May Need Attention}
\begin{enumerate}
	\item The |scatter/classes| feature produces proper legends as of version 1.3. This may change the appearance of existing legends of plots with |scatter/classes|.

	\item Due to a small math bug in \PGF\ $2.00$, you \emph{can't} use the math expression `|-x^2|'. It is necessary to use `|0-x^2|' instead. The same holds for
		`|exp(-x^2)|'; use `|exp(0-x^2)|' instead.

		This will be fixed with \PGF\ $>2.00$.

	\item Starting with \PGFPlots\ $1.1$, |\tikzstyle| should \emph{no longer be used} to set \PGFPlots\ options.
	
	Although |\tikzstyle| is still supported for some older \PGFPlots\ options, you should replace any occurance of |\tikzstyle| with |\pgfplotsset{|\meta{style name}|/.style={|\meta{key-value-list}|}}| or the associated |/.append style| variant. See section~\ref{sec:styles} for more detail.
\end{enumerate}
I apologize for any inconvenience caused by these changes.

\begin{pgfplotskey}{compat=\mchoice{1.3,pre 1.3,default,newest} (initially default)}
	The preamble configuration 
\begin{codeexample}[code only]
\usepackage{pgfplots}
\pgfplotsset{compat=1.3}
\end{codeexample}
	allows to choose between backwards compatibility and most recent features.

	\PGFPlots\ version 1.3 comes with new features which may lead to different spacings compared to earlier versions:
	\begin{enumerate}
		\item Version 1.3 comes with the |xlabel near ticks| feature which places (in this case $x$) axis labels ``near'' the tick labels, taking the size of tick labels into account.
		
		Older versions used |xlabel absolute|, i.e.\ an absolute distance between the axis and axis labels (now matter how large or small tick labels are).

		The initial setting \emph{keeps backwards compatibility}.

		You are encouraged to use
\begin{codeexample}[code only]
\usepackage{pgfplots}
\pgfplotsset{compat=1.3}
\end{codeexample}
		in your preamble.
		
		\item Version 1.3 fixes a bug with |anchor=south west| which occurred mainly for reversed axes: the anchors where upside--down. This is fixed now.

		
		If you have written some work--around and want to keep it going, use

		|\pgfplotsset{compat/anchors=pre 1.3}|\pgfmanualpdflabel{/pgfplots/compat/anchors}{} 

		to disable the bug fix.
	\end{enumerate}

	Use |\pgfplotsset{compat=default}| to restore the factory settings.

	The setting |\pgfplotsset{compat=newest}| will always use features of the most recent version. This might result in small changes in the document's appearance.
\end{pgfplotskey}

\subsection{The Team}
\PGFPlots\ has been written mainly by Christian Feuers�nger with many improvements of Pascal Wolkotte and Nick Papior Andersen as a spare time project. We hope it is useful and provides valuable plots.

If you are interested in writing something but don't know how, consider reading the auxiliary manual \href{file:TeX-programming-notes.pdf}{TeX-programming-notes.pdf} which comes with \PGFPlots. It is far from complete, but maybe it is a good starting point (at least for more literature).

\subsection{Acknowledgements}
I thank God for all hours of enjoyed programming. I thank Pascal Wolkotte and Nick Papior Andersen for their programming efforts and contributions as part of the development team. I thank Stefan Tibus, who contributed the |plot shell| feature. I thank Tom Cashman for the contribution of the |reverse legend| feature. Special thanks go to Stefan Pinnow whose tests of \PGFPlots\ lead to numerous quality improvements. Furthermore, I thank Dr.~Schweitzer for many fruitful discussions and Dr.~Meine for his ideas and suggestions. Thanks as well to the many international contributors who provided feature requests or identified bugs or simply manual improvements!

Last but not least, I thank Till Tantau and Mark Wibrow for their excellent graphics (and more) package \PGF\ and \Tikz\ which is the base of \PGFPlots.

\include{pgfplots.install}%
\include{pgfplots.intro}%
\include{pgfplots.reference}%
\include{pgfplots.libs}%
\include{pgfplots.resources}%
\include{pgfplots.importexport}%
\include{pgfplots.basic.reference}%

\printindex

\bibliographystyle{abbrv} %gerapali} %gerabbrv} %gerunsrt.bst} %gerabbrv}% gerplain}
\nocite{pgfplotstable}
\nocite{programmingnotes}
\bibliography{pgfplots}
\end{document}
%
%%%%%%%%%%%%%%%%%%%%%%%%%%%%%%%%%%%%%%%%%%%%%%%%%%%%%%%%%%%%%%%%%%%%%%%%%%%%%
%
% Package pgfplots.sty documentation. 
%
% Copyright 2007/2008 by Christian Feuersaenger.
%
% This program is free software: you can redistribute it and/or modify
% it under the terms of the GNU General Public License as published by
% the Free Software Foundation, either version 3 of the License, or
% (at your option) any later version.
% 
% This program is distributed in the hope that it will be useful,
% but WITHOUT ANY WARRANTY; without even the implied warranty of
% MERCHANTABILITY or FITNESS FOR A PARTICULAR PURPOSE.  See the
% GNU General Public License for more details.
% 
% You should have received a copy of the GNU General Public License
% along with this program.  If not, see <http://www.gnu.org/licenses/>.
%
%
%%%%%%%%%%%%%%%%%%%%%%%%%%%%%%%%%%%%%%%%%%%%%%%%%%%%%%%%%%%%%%%%%%%%%%%%%%%%%
\input{pgfplots.preamble.tex}
%\RequirePackage[german,english,francais]{babel}

\pgfplotsmanualexternalexpensivetrue

%\usetikzlibrary{external}
%\tikzexternalize[prefix=figures/]{pgfplots}

\title{%
	Manual for Package \PGFPlots\\
	{\small Version \pgfplotsversion}\\
	{\small\href{http://sourceforge.net/projects/pgfplots}{http://sourceforge.net/projects/pgfplots}}}

%\includeonly{pgfplots.libs}


\begin{document}

\def\plotcoords{%
\addplot coordinates {
(5,8.312e-02)    (17,2.547e-02)   (49,7.407e-03)
(129,2.102e-03)  (321,5.874e-04)  (769,1.623e-04)
(1793,4.442e-05) (4097,1.207e-05) (9217,3.261e-06)
};

\addplot coordinates{
(7,8.472e-02)    (31,3.044e-02)    (111,1.022e-02)
(351,3.303e-03)  (1023,1.039e-03)  (2815,3.196e-04)
(7423,9.658e-05) (18943,2.873e-05) (47103,8.437e-06)
};

\addplot coordinates{
(9,7.881e-02)     (49,3.243e-02)    (209,1.232e-02)
(769,4.454e-03)   (2561,1.551e-03)  (7937,5.236e-04)
(23297,1.723e-04) (65537,5.545e-05) (178177,1.751e-05)
};

\addplot coordinates{
(11,6.887e-02)    (71,3.177e-02)     (351,1.341e-02)
(1471,5.334e-03)  (5503,2.027e-03)   (18943,7.415e-04)
(61183,2.628e-04) (187903,9.063e-05) (553983,3.053e-05)
};

\addplot coordinates{
(13,5.755e-02)     (97,2.925e-02)     (545,1.351e-02)
(2561,5.842e-03)   (10625,2.397e-03)  (40193,9.414e-04)
(141569,3.564e-04) (471041,1.308e-04) 
(1496065,4.670e-05)
};
}%
\maketitle
\begin{abstract}%
\PGFPlots\ draws high--quality function plots in normal or logarithmic scaling with a user-friendly interface directly in \TeX. The user supplies axis labels, legend entries and the plot coordinates for one or more plots and \PGFPlots\ applies axis scaling, computes any logarithms and axis ticks and draws the plots, supporting line plots, scatter plots, piecewise constant plots, bar plots, area plots, mesh-- and surface plots and some more. It is based on Till Tantau's package \PGF/\Tikz.
\end{abstract}
\tableofcontents
\section{Introduction}
This package provides tools to generate plots and labeled axes easily. It draws normal plots, logplots and semi-logplots, in two and three dimensions. Axis ticks, labels, legends (in case of multiple plots) can be added with key-value options. It can cycle through a set of predefined line/marker/color specifications. In summary, its purpose is to simplify the generation of high-quality function and/or data plots, and solving the problems of
\begin{itemize}
	\item consistency of document and font type and font size,
	\item direct use of \TeX\ math mode in axis descriptions,
	\item consistency of data and figures (no third party tool necessary),
	\item inter document consistency using preamble configurations and styles.
\end{itemize}
Although not necessary, separate |.pdf| or |.eps| graphics can be generated using the |external| library developed as part of \Tikz.

\PGFPlots\ is build completely on \Tikz/\PGF. Knowledge of \Tikz\ will simplify the work with \PGFPlots, although it is not required.

\section[About PGFPlots: Preliminaries]{About {\normalfont\PGFPlots}: Preliminaries}
This section contains information about upgrades, the team, the installation (in case you need to do it manually) and troubleshooting. You may skip it completely except for the upgrade remarks.

However, note that this library requires at least \PGF\ version $2.00$. At the time of this writing, many \TeX-distributions still contain the older \PGF\ version $1.18$, so it may be necessary to install a recent \PGF\ prior to using \PGFPlots.

\subsection{Components}
\PGFPlots\ comes with two components:
\begin{enumerate}
	\item the plotting component (which you are currently reading) and
	\item the \PGFPlotstable\ component which simplifies number formatting and postprocessing of numerical tables. It comes as a separate package and has its own manual \href{file:pgfplotstable.pdf}{pgfplotstable.pdf}.
\end{enumerate}

\subsection{Upgrade remarks}
This release provides a lot of improvements which can be found in all detail in \texttt{ChangeLog} for interested readers. However, some attention is useful with respect to the following changes.

\subsubsection{New Optional Features}
\begin{enumerate}
	\item \PGFPlots\ 1.3 comes with user interface improvements. The technical distinction between ``behavior options'' and ``style options'' of older versions is no longer necessary (although still fully supported).

	\item \PGFPlots\ 1.3 has a new feature which allows to \emph{move axis labels tight to tick labels} automatically.

	Since this affects the spacing, it is not enabled be default.

	Use
\begin{codeexample}[code only]
\usepackage{pgfplots}
\pgfplotsset{compat=1.3}
\end{codeexample}
	in your preamble to benefit from the improved spacing\footnote{The |compat=1.3| setting is necessary for new features which move axis labels around like reversed axis. However, \PGFPlots\ will still work as it does in earlier versions, your documents will remain the same if you don't set it explicitly.}.
	Take a look at the next page for the precise definition of |compat|.

	\item \PGFPlots\ 1.3 now supports reversed axes. It is no longer necessary to use work--arounds with negative units.
\pgfkeys{/pdflinks/search key prefixes in/.add={/pgfplots/,}{}}

	Take a look at the |x dir=reverse| key.

	Existing work--arounds will still function properly. Use |\pgfplotsset{compat=1.3}| together with |x dir=reverse| to switch to the new version.
\end{enumerate}

\subsubsection{Old Features Which May Need Attention}
\begin{enumerate}
	\item The |scatter/classes| feature produces proper legends as of version 1.3. This may change the appearance of existing legends of plots with |scatter/classes|.

	\item Due to a small math bug in \PGF\ $2.00$, you \emph{can't} use the math expression `|-x^2|'. It is necessary to use `|0-x^2|' instead. The same holds for
		`|exp(-x^2)|'; use `|exp(0-x^2)|' instead.

		This will be fixed with \PGF\ $>2.00$.

	\item Starting with \PGFPlots\ $1.1$, |\tikzstyle| should \emph{no longer be used} to set \PGFPlots\ options.
	
	Although |\tikzstyle| is still supported for some older \PGFPlots\ options, you should replace any occurance of |\tikzstyle| with |\pgfplotsset{|\meta{style name}|/.style={|\meta{key-value-list}|}}| or the associated |/.append style| variant. See section~\ref{sec:styles} for more detail.
\end{enumerate}
I apologize for any inconvenience caused by these changes.

\begin{pgfplotskey}{compat=\mchoice{1.3,pre 1.3,default,newest} (initially default)}
	The preamble configuration 
\begin{codeexample}[code only]
\usepackage{pgfplots}
\pgfplotsset{compat=1.3}
\end{codeexample}
	allows to choose between backwards compatibility and most recent features.

	\PGFPlots\ version 1.3 comes with new features which may lead to different spacings compared to earlier versions:
	\begin{enumerate}
		\item Version 1.3 comes with the |xlabel near ticks| feature which places (in this case $x$) axis labels ``near'' the tick labels, taking the size of tick labels into account.
		
		Older versions used |xlabel absolute|, i.e.\ an absolute distance between the axis and axis labels (now matter how large or small tick labels are).

		The initial setting \emph{keeps backwards compatibility}.

		You are encouraged to use
\begin{codeexample}[code only]
\usepackage{pgfplots}
\pgfplotsset{compat=1.3}
\end{codeexample}
		in your preamble.
		
		\item Version 1.3 fixes a bug with |anchor=south west| which occurred mainly for reversed axes: the anchors where upside--down. This is fixed now.

		
		If you have written some work--around and want to keep it going, use

		|\pgfplotsset{compat/anchors=pre 1.3}|\pgfmanualpdflabel{/pgfplots/compat/anchors}{} 

		to disable the bug fix.
	\end{enumerate}

	Use |\pgfplotsset{compat=default}| to restore the factory settings.

	The setting |\pgfplotsset{compat=newest}| will always use features of the most recent version. This might result in small changes in the document's appearance.
\end{pgfplotskey}

\subsection{The Team}
\PGFPlots\ has been written mainly by Christian Feuers�nger with many improvements of Pascal Wolkotte and Nick Papior Andersen as a spare time project. We hope it is useful and provides valuable plots.

If you are interested in writing something but don't know how, consider reading the auxiliary manual \href{file:TeX-programming-notes.pdf}{TeX-programming-notes.pdf} which comes with \PGFPlots. It is far from complete, but maybe it is a good starting point (at least for more literature).

\subsection{Acknowledgements}
I thank God for all hours of enjoyed programming. I thank Pascal Wolkotte and Nick Papior Andersen for their programming efforts and contributions as part of the development team. I thank Stefan Tibus, who contributed the |plot shell| feature. I thank Tom Cashman for the contribution of the |reverse legend| feature. Special thanks go to Stefan Pinnow whose tests of \PGFPlots\ lead to numerous quality improvements. Furthermore, I thank Dr.~Schweitzer for many fruitful discussions and Dr.~Meine for his ideas and suggestions. Thanks as well to the many international contributors who provided feature requests or identified bugs or simply manual improvements!

Last but not least, I thank Till Tantau and Mark Wibrow for their excellent graphics (and more) package \PGF\ and \Tikz\ which is the base of \PGFPlots.

\include{pgfplots.install}%
\include{pgfplots.intro}%
\include{pgfplots.reference}%
\include{pgfplots.libs}%
\include{pgfplots.resources}%
\include{pgfplots.importexport}%
\include{pgfplots.basic.reference}%

\printindex

\bibliographystyle{abbrv} %gerapali} %gerabbrv} %gerunsrt.bst} %gerabbrv}% gerplain}
\nocite{pgfplotstable}
\nocite{programmingnotes}
\bibliography{pgfplots}
\end{document}
%
%%%%%%%%%%%%%%%%%%%%%%%%%%%%%%%%%%%%%%%%%%%%%%%%%%%%%%%%%%%%%%%%%%%%%%%%%%%%%
%
% Package pgfplots.sty documentation. 
%
% Copyright 2007/2008 by Christian Feuersaenger.
%
% This program is free software: you can redistribute it and/or modify
% it under the terms of the GNU General Public License as published by
% the Free Software Foundation, either version 3 of the License, or
% (at your option) any later version.
% 
% This program is distributed in the hope that it will be useful,
% but WITHOUT ANY WARRANTY; without even the implied warranty of
% MERCHANTABILITY or FITNESS FOR A PARTICULAR PURPOSE.  See the
% GNU General Public License for more details.
% 
% You should have received a copy of the GNU General Public License
% along with this program.  If not, see <http://www.gnu.org/licenses/>.
%
%
%%%%%%%%%%%%%%%%%%%%%%%%%%%%%%%%%%%%%%%%%%%%%%%%%%%%%%%%%%%%%%%%%%%%%%%%%%%%%
\input{pgfplots.preamble.tex}
%\RequirePackage[german,english,francais]{babel}

\pgfplotsmanualexternalexpensivetrue

%\usetikzlibrary{external}
%\tikzexternalize[prefix=figures/]{pgfplots}

\title{%
	Manual for Package \PGFPlots\\
	{\small Version \pgfplotsversion}\\
	{\small\href{http://sourceforge.net/projects/pgfplots}{http://sourceforge.net/projects/pgfplots}}}

%\includeonly{pgfplots.libs}


\begin{document}

\def\plotcoords{%
\addplot coordinates {
(5,8.312e-02)    (17,2.547e-02)   (49,7.407e-03)
(129,2.102e-03)  (321,5.874e-04)  (769,1.623e-04)
(1793,4.442e-05) (4097,1.207e-05) (9217,3.261e-06)
};

\addplot coordinates{
(7,8.472e-02)    (31,3.044e-02)    (111,1.022e-02)
(351,3.303e-03)  (1023,1.039e-03)  (2815,3.196e-04)
(7423,9.658e-05) (18943,2.873e-05) (47103,8.437e-06)
};

\addplot coordinates{
(9,7.881e-02)     (49,3.243e-02)    (209,1.232e-02)
(769,4.454e-03)   (2561,1.551e-03)  (7937,5.236e-04)
(23297,1.723e-04) (65537,5.545e-05) (178177,1.751e-05)
};

\addplot coordinates{
(11,6.887e-02)    (71,3.177e-02)     (351,1.341e-02)
(1471,5.334e-03)  (5503,2.027e-03)   (18943,7.415e-04)
(61183,2.628e-04) (187903,9.063e-05) (553983,3.053e-05)
};

\addplot coordinates{
(13,5.755e-02)     (97,2.925e-02)     (545,1.351e-02)
(2561,5.842e-03)   (10625,2.397e-03)  (40193,9.414e-04)
(141569,3.564e-04) (471041,1.308e-04) 
(1496065,4.670e-05)
};
}%
\maketitle
\begin{abstract}%
\PGFPlots\ draws high--quality function plots in normal or logarithmic scaling with a user-friendly interface directly in \TeX. The user supplies axis labels, legend entries and the plot coordinates for one or more plots and \PGFPlots\ applies axis scaling, computes any logarithms and axis ticks and draws the plots, supporting line plots, scatter plots, piecewise constant plots, bar plots, area plots, mesh-- and surface plots and some more. It is based on Till Tantau's package \PGF/\Tikz.
\end{abstract}
\tableofcontents
\section{Introduction}
This package provides tools to generate plots and labeled axes easily. It draws normal plots, logplots and semi-logplots, in two and three dimensions. Axis ticks, labels, legends (in case of multiple plots) can be added with key-value options. It can cycle through a set of predefined line/marker/color specifications. In summary, its purpose is to simplify the generation of high-quality function and/or data plots, and solving the problems of
\begin{itemize}
	\item consistency of document and font type and font size,
	\item direct use of \TeX\ math mode in axis descriptions,
	\item consistency of data and figures (no third party tool necessary),
	\item inter document consistency using preamble configurations and styles.
\end{itemize}
Although not necessary, separate |.pdf| or |.eps| graphics can be generated using the |external| library developed as part of \Tikz.

\PGFPlots\ is build completely on \Tikz/\PGF. Knowledge of \Tikz\ will simplify the work with \PGFPlots, although it is not required.

\section[About PGFPlots: Preliminaries]{About {\normalfont\PGFPlots}: Preliminaries}
This section contains information about upgrades, the team, the installation (in case you need to do it manually) and troubleshooting. You may skip it completely except for the upgrade remarks.

However, note that this library requires at least \PGF\ version $2.00$. At the time of this writing, many \TeX-distributions still contain the older \PGF\ version $1.18$, so it may be necessary to install a recent \PGF\ prior to using \PGFPlots.

\subsection{Components}
\PGFPlots\ comes with two components:
\begin{enumerate}
	\item the plotting component (which you are currently reading) and
	\item the \PGFPlotstable\ component which simplifies number formatting and postprocessing of numerical tables. It comes as a separate package and has its own manual \href{file:pgfplotstable.pdf}{pgfplotstable.pdf}.
\end{enumerate}

\subsection{Upgrade remarks}
This release provides a lot of improvements which can be found in all detail in \texttt{ChangeLog} for interested readers. However, some attention is useful with respect to the following changes.

\subsubsection{New Optional Features}
\begin{enumerate}
	\item \PGFPlots\ 1.3 comes with user interface improvements. The technical distinction between ``behavior options'' and ``style options'' of older versions is no longer necessary (although still fully supported).

	\item \PGFPlots\ 1.3 has a new feature which allows to \emph{move axis labels tight to tick labels} automatically.

	Since this affects the spacing, it is not enabled be default.

	Use
\begin{codeexample}[code only]
\usepackage{pgfplots}
\pgfplotsset{compat=1.3}
\end{codeexample}
	in your preamble to benefit from the improved spacing\footnote{The |compat=1.3| setting is necessary for new features which move axis labels around like reversed axis. However, \PGFPlots\ will still work as it does in earlier versions, your documents will remain the same if you don't set it explicitly.}.
	Take a look at the next page for the precise definition of |compat|.

	\item \PGFPlots\ 1.3 now supports reversed axes. It is no longer necessary to use work--arounds with negative units.
\pgfkeys{/pdflinks/search key prefixes in/.add={/pgfplots/,}{}}

	Take a look at the |x dir=reverse| key.

	Existing work--arounds will still function properly. Use |\pgfplotsset{compat=1.3}| together with |x dir=reverse| to switch to the new version.
\end{enumerate}

\subsubsection{Old Features Which May Need Attention}
\begin{enumerate}
	\item The |scatter/classes| feature produces proper legends as of version 1.3. This may change the appearance of existing legends of plots with |scatter/classes|.

	\item Due to a small math bug in \PGF\ $2.00$, you \emph{can't} use the math expression `|-x^2|'. It is necessary to use `|0-x^2|' instead. The same holds for
		`|exp(-x^2)|'; use `|exp(0-x^2)|' instead.

		This will be fixed with \PGF\ $>2.00$.

	\item Starting with \PGFPlots\ $1.1$, |\tikzstyle| should \emph{no longer be used} to set \PGFPlots\ options.
	
	Although |\tikzstyle| is still supported for some older \PGFPlots\ options, you should replace any occurance of |\tikzstyle| with |\pgfplotsset{|\meta{style name}|/.style={|\meta{key-value-list}|}}| or the associated |/.append style| variant. See section~\ref{sec:styles} for more detail.
\end{enumerate}
I apologize for any inconvenience caused by these changes.

\begin{pgfplotskey}{compat=\mchoice{1.3,pre 1.3,default,newest} (initially default)}
	The preamble configuration 
\begin{codeexample}[code only]
\usepackage{pgfplots}
\pgfplotsset{compat=1.3}
\end{codeexample}
	allows to choose between backwards compatibility and most recent features.

	\PGFPlots\ version 1.3 comes with new features which may lead to different spacings compared to earlier versions:
	\begin{enumerate}
		\item Version 1.3 comes with the |xlabel near ticks| feature which places (in this case $x$) axis labels ``near'' the tick labels, taking the size of tick labels into account.
		
		Older versions used |xlabel absolute|, i.e.\ an absolute distance between the axis and axis labels (now matter how large or small tick labels are).

		The initial setting \emph{keeps backwards compatibility}.

		You are encouraged to use
\begin{codeexample}[code only]
\usepackage{pgfplots}
\pgfplotsset{compat=1.3}
\end{codeexample}
		in your preamble.
		
		\item Version 1.3 fixes a bug with |anchor=south west| which occurred mainly for reversed axes: the anchors where upside--down. This is fixed now.

		
		If you have written some work--around and want to keep it going, use

		|\pgfplotsset{compat/anchors=pre 1.3}|\pgfmanualpdflabel{/pgfplots/compat/anchors}{} 

		to disable the bug fix.
	\end{enumerate}

	Use |\pgfplotsset{compat=default}| to restore the factory settings.

	The setting |\pgfplotsset{compat=newest}| will always use features of the most recent version. This might result in small changes in the document's appearance.
\end{pgfplotskey}

\subsection{The Team}
\PGFPlots\ has been written mainly by Christian Feuers�nger with many improvements of Pascal Wolkotte and Nick Papior Andersen as a spare time project. We hope it is useful and provides valuable plots.

If you are interested in writing something but don't know how, consider reading the auxiliary manual \href{file:TeX-programming-notes.pdf}{TeX-programming-notes.pdf} which comes with \PGFPlots. It is far from complete, but maybe it is a good starting point (at least for more literature).

\subsection{Acknowledgements}
I thank God for all hours of enjoyed programming. I thank Pascal Wolkotte and Nick Papior Andersen for their programming efforts and contributions as part of the development team. I thank Stefan Tibus, who contributed the |plot shell| feature. I thank Tom Cashman for the contribution of the |reverse legend| feature. Special thanks go to Stefan Pinnow whose tests of \PGFPlots\ lead to numerous quality improvements. Furthermore, I thank Dr.~Schweitzer for many fruitful discussions and Dr.~Meine for his ideas and suggestions. Thanks as well to the many international contributors who provided feature requests or identified bugs or simply manual improvements!

Last but not least, I thank Till Tantau and Mark Wibrow for their excellent graphics (and more) package \PGF\ and \Tikz\ which is the base of \PGFPlots.

\include{pgfplots.install}%
\include{pgfplots.intro}%
\include{pgfplots.reference}%
\include{pgfplots.libs}%
\include{pgfplots.resources}%
\include{pgfplots.importexport}%
\include{pgfplots.basic.reference}%

\printindex

\bibliographystyle{abbrv} %gerapali} %gerabbrv} %gerunsrt.bst} %gerabbrv}% gerplain}
\nocite{pgfplotstable}
\nocite{programmingnotes}
\bibliography{pgfplots}
\end{document}
%
\include{pgfplots.basic.reference}%

\printindex

\bibliographystyle{abbrv} %gerapali} %gerabbrv} %gerunsrt.bst} %gerabbrv}% gerplain}
\nocite{pgfplotstable}
\nocite{programmingnotes}
\bibliography{pgfplots}
\end{document}
%
%%%%%%%%%%%%%%%%%%%%%%%%%%%%%%%%%%%%%%%%%%%%%%%%%%%%%%%%%%%%%%%%%%%%%%%%%%%%%
%
% Package pgfplots.sty documentation. 
%
% Copyright 2007/2008 by Christian Feuersaenger.
%
% This program is free software: you can redistribute it and/or modify
% it under the terms of the GNU General Public License as published by
% the Free Software Foundation, either version 3 of the License, or
% (at your option) any later version.
% 
% This program is distributed in the hope that it will be useful,
% but WITHOUT ANY WARRANTY; without even the implied warranty of
% MERCHANTABILITY or FITNESS FOR A PARTICULAR PURPOSE.  See the
% GNU General Public License for more details.
% 
% You should have received a copy of the GNU General Public License
% along with this program.  If not, see <http://www.gnu.org/licenses/>.
%
%
%%%%%%%%%%%%%%%%%%%%%%%%%%%%%%%%%%%%%%%%%%%%%%%%%%%%%%%%%%%%%%%%%%%%%%%%%%%%%
%%%%%%%%%%%%%%%%%%%%%%%%%%%%%%%%%%%%%%%%%%%%%%%%%%%%%%%%%%%%%%%%%%%%%%%%%%%%%
%
% Package pgfplots.sty documentation. 
%
% Copyright 2007/2008 by Christian Feuersaenger.
%
% This program is free software: you can redistribute it and/or modify
% it under the terms of the GNU General Public License as published by
% the Free Software Foundation, either version 3 of the License, or
% (at your option) any later version.
% 
% This program is distributed in the hope that it will be useful,
% but WITHOUT ANY WARRANTY; without even the implied warranty of
% MERCHANTABILITY or FITNESS FOR A PARTICULAR PURPOSE.  See the
% GNU General Public License for more details.
% 
% You should have received a copy of the GNU General Public License
% along with this program.  If not, see <http://www.gnu.org/licenses/>.
%
%
%%%%%%%%%%%%%%%%%%%%%%%%%%%%%%%%%%%%%%%%%%%%%%%%%%%%%%%%%%%%%%%%%%%%%%%%%%%%%
\documentclass[a4paper]{ltxdoc}

\usepackage{makeidx}

% DON't let hyperref overload the format of index and glossary. 
% I want to do that on my own in the stylefiles for makeindex...
\makeatletter
\let\@old@wrindex=\@wrindex
\makeatother

\usepackage{ifpdf}
\ifpdf
	\usepackage[pdftex]{hyperref}
\else
	\usepackage[dvipdfm]{hyperref}
\fi
\hypersetup{pdfborder={0 0 0}}

\makeatletter
\let\@wrindex=\@old@wrindex
\makeatother

% Formatiere Seitennummern im Index:
\newcommand{\indexpageno}[1]{%
	{\bfseries\hyperpage{#1}}%
}


\newcommand{\R}{\mathbb{R}}
\newcommand{\N}{\mathbb{N}}
\newcommand{\Z}{\mathbb{Z}}

\long\def\COMMENTLOWLEVEL#1\ENDCOMMENT{}
\def\ENDCOMMENT{}

\usepackage{textcomp}
\usepackage{booktabs}

\usepackage{calc}
\usepackage[formats]{listings}
%\usepackage{courier} % don't use it - the '^' character can't be copy-pasted in courier

\usepackage{array}
\lstset{%
	basicstyle=\ttfamily,
	language=[LaTeX]tex, % Seems as if \lstset{language=tex} must be invoked BEFORE loading tikz!?
	tabsize=4,
	breaklines=true,
	breakindent=0pt
}

\ifpdf
	\pdfinfo {
		/Author	(Christian Feuersaenger)
	}

\else
	\def\pgfsysdriver{pgfsys-dvipdfm.def}
\fi
%\def\pgfsysdriver{pgfsys-pdftex.def}
\usepackage{pgfplots}

\ifpdf
	\usetikzlibrary{pgfplotsclickable}
\fi

%\usepackage{fp}
% ATTENTION:
% this requires pgf version NEWER than 2.00 :
%\usetikzlibrary{fixedpointarithmetic}

\usetikzlibrary{dateplot}

\usepackage[a4paper,left=2.25cm,right=2.25cm,top=2.5cm,bottom=2.5cm,nohead]{geometry}
\usepackage{amsmath,amssymb}
\usepackage{xxcolor}
\usepackage{pifont}
\usepackage[latin1]{inputenc}
\usepackage{amsmath}
\usepackage{eurosym}
\input{pgfplots-macros}

\pgfqkeys{/codeexample}{%
	every codeexample/.style={
		width=8cm,
		/pgfplots/every axis/.append style={legend style={fill=graphicbackground}}
	},
	tabsize=4,
}
\pgfplotsset{
	every axis/.append style={width=7cm},
	filter discard warning=false,
}

\usetikzlibrary{backgrounds,patterns}
% Global styles:
\tikzset{
  shape example/.style={
    color=black!30,
    draw,
    fill=yellow!30,
    line width=.5cm,
    inner xsep=2.5cm,
    inner ysep=0.5cm}
}

\newcommand{\FIXME}[1]{\textcolor{red}{(FIXME: #1)}}

% fuer endvironment 'sidewaysfigure' bspw
% \usepackage{rotating}

\newcommand\Tikz{Ti\textit kZ}
\newcommand\PGF{\textsc{pgf}}
\newcommand\PGFPlots{\textsc{pgfplots}}
\def\PGFPlotstable{\textsc{PgfplotsTable}}


\makeindex

% Fix overful hboxes automatically:
\tolerance=2000
\emergencystretch=10pt

\tikzset{prefix=gnuplot/pgfplots_} % prefix for 'plot function'

\author{%
	Christian Feuers\"anger\footnote{\url{http://wissrech.ins.uni-bonn.de/people/feuersaenger}}\\%
	Institut f\"ur Numerische Simulation\\
	Universit\"at Bonn, Germany}


%\RequirePackage[german,english,francais]{babel}

\pgfplotsmanualexternalexpensivetrue

%\usetikzlibrary{external}
%\tikzexternalize[prefix=figures/]{pgfplots}

\title{%
	Manual for Package \PGFPlots\\
	{\small Version \pgfplotsversion}\\
	{\small\href{http://sourceforge.net/projects/pgfplots}{http://sourceforge.net/projects/pgfplots}}}

%\includeonly{pgfplots.libs}


\begin{document}

\def\plotcoords{%
\addplot coordinates {
(5,8.312e-02)    (17,2.547e-02)   (49,7.407e-03)
(129,2.102e-03)  (321,5.874e-04)  (769,1.623e-04)
(1793,4.442e-05) (4097,1.207e-05) (9217,3.261e-06)
};

\addplot coordinates{
(7,8.472e-02)    (31,3.044e-02)    (111,1.022e-02)
(351,3.303e-03)  (1023,1.039e-03)  (2815,3.196e-04)
(7423,9.658e-05) (18943,2.873e-05) (47103,8.437e-06)
};

\addplot coordinates{
(9,7.881e-02)     (49,3.243e-02)    (209,1.232e-02)
(769,4.454e-03)   (2561,1.551e-03)  (7937,5.236e-04)
(23297,1.723e-04) (65537,5.545e-05) (178177,1.751e-05)
};

\addplot coordinates{
(11,6.887e-02)    (71,3.177e-02)     (351,1.341e-02)
(1471,5.334e-03)  (5503,2.027e-03)   (18943,7.415e-04)
(61183,2.628e-04) (187903,9.063e-05) (553983,3.053e-05)
};

\addplot coordinates{
(13,5.755e-02)     (97,2.925e-02)     (545,1.351e-02)
(2561,5.842e-03)   (10625,2.397e-03)  (40193,9.414e-04)
(141569,3.564e-04) (471041,1.308e-04) 
(1496065,4.670e-05)
};
}%
\maketitle
\begin{abstract}%
\PGFPlots\ draws high--quality function plots in normal or logarithmic scaling with a user-friendly interface directly in \TeX. The user supplies axis labels, legend entries and the plot coordinates for one or more plots and \PGFPlots\ applies axis scaling, computes any logarithms and axis ticks and draws the plots, supporting line plots, scatter plots, piecewise constant plots, bar plots, area plots, mesh-- and surface plots and some more. It is based on Till Tantau's package \PGF/\Tikz.
\end{abstract}
\tableofcontents
\section{Introduction}
This package provides tools to generate plots and labeled axes easily. It draws normal plots, logplots and semi-logplots, in two and three dimensions. Axis ticks, labels, legends (in case of multiple plots) can be added with key-value options. It can cycle through a set of predefined line/marker/color specifications. In summary, its purpose is to simplify the generation of high-quality function and/or data plots, and solving the problems of
\begin{itemize}
	\item consistency of document and font type and font size,
	\item direct use of \TeX\ math mode in axis descriptions,
	\item consistency of data and figures (no third party tool necessary),
	\item inter document consistency using preamble configurations and styles.
\end{itemize}
Although not necessary, separate |.pdf| or |.eps| graphics can be generated using the |external| library developed as part of \Tikz.

\PGFPlots\ is build completely on \Tikz/\PGF. Knowledge of \Tikz\ will simplify the work with \PGFPlots, although it is not required.

\section[About PGFPlots: Preliminaries]{About {\normalfont\PGFPlots}: Preliminaries}
This section contains information about upgrades, the team, the installation (in case you need to do it manually) and troubleshooting. You may skip it completely except for the upgrade remarks.

However, note that this library requires at least \PGF\ version $2.00$. At the time of this writing, many \TeX-distributions still contain the older \PGF\ version $1.18$, so it may be necessary to install a recent \PGF\ prior to using \PGFPlots.

\subsection{Components}
\PGFPlots\ comes with two components:
\begin{enumerate}
	\item the plotting component (which you are currently reading) and
	\item the \PGFPlotstable\ component which simplifies number formatting and postprocessing of numerical tables. It comes as a separate package and has its own manual \href{file:pgfplotstable.pdf}{pgfplotstable.pdf}.
\end{enumerate}

\subsection{Upgrade remarks}
This release provides a lot of improvements which can be found in all detail in \texttt{ChangeLog} for interested readers. However, some attention is useful with respect to the following changes.

\subsubsection{New Optional Features}
\begin{enumerate}
	\item \PGFPlots\ 1.3 comes with user interface improvements. The technical distinction between ``behavior options'' and ``style options'' of older versions is no longer necessary (although still fully supported).

	\item \PGFPlots\ 1.3 has a new feature which allows to \emph{move axis labels tight to tick labels} automatically.

	Since this affects the spacing, it is not enabled be default.

	Use
\begin{codeexample}[code only]
\usepackage{pgfplots}
\pgfplotsset{compat=1.3}
\end{codeexample}
	in your preamble to benefit from the improved spacing\footnote{The |compat=1.3| setting is necessary for new features which move axis labels around like reversed axis. However, \PGFPlots\ will still work as it does in earlier versions, your documents will remain the same if you don't set it explicitly.}.
	Take a look at the next page for the precise definition of |compat|.

	\item \PGFPlots\ 1.3 now supports reversed axes. It is no longer necessary to use work--arounds with negative units.
\pgfkeys{/pdflinks/search key prefixes in/.add={/pgfplots/,}{}}

	Take a look at the |x dir=reverse| key.

	Existing work--arounds will still function properly. Use |\pgfplotsset{compat=1.3}| together with |x dir=reverse| to switch to the new version.
\end{enumerate}

\subsubsection{Old Features Which May Need Attention}
\begin{enumerate}
	\item The |scatter/classes| feature produces proper legends as of version 1.3. This may change the appearance of existing legends of plots with |scatter/classes|.

	\item Due to a small math bug in \PGF\ $2.00$, you \emph{can't} use the math expression `|-x^2|'. It is necessary to use `|0-x^2|' instead. The same holds for
		`|exp(-x^2)|'; use `|exp(0-x^2)|' instead.

		This will be fixed with \PGF\ $>2.00$.

	\item Starting with \PGFPlots\ $1.1$, |\tikzstyle| should \emph{no longer be used} to set \PGFPlots\ options.
	
	Although |\tikzstyle| is still supported for some older \PGFPlots\ options, you should replace any occurance of |\tikzstyle| with |\pgfplotsset{|\meta{style name}|/.style={|\meta{key-value-list}|}}| or the associated |/.append style| variant. See section~\ref{sec:styles} for more detail.
\end{enumerate}
I apologize for any inconvenience caused by these changes.

\begin{pgfplotskey}{compat=\mchoice{1.3,pre 1.3,default,newest} (initially default)}
	The preamble configuration 
\begin{codeexample}[code only]
\usepackage{pgfplots}
\pgfplotsset{compat=1.3}
\end{codeexample}
	allows to choose between backwards compatibility and most recent features.

	\PGFPlots\ version 1.3 comes with new features which may lead to different spacings compared to earlier versions:
	\begin{enumerate}
		\item Version 1.3 comes with the |xlabel near ticks| feature which places (in this case $x$) axis labels ``near'' the tick labels, taking the size of tick labels into account.
		
		Older versions used |xlabel absolute|, i.e.\ an absolute distance between the axis and axis labels (now matter how large or small tick labels are).

		The initial setting \emph{keeps backwards compatibility}.

		You are encouraged to use
\begin{codeexample}[code only]
\usepackage{pgfplots}
\pgfplotsset{compat=1.3}
\end{codeexample}
		in your preamble.
		
		\item Version 1.3 fixes a bug with |anchor=south west| which occurred mainly for reversed axes: the anchors where upside--down. This is fixed now.

		
		If you have written some work--around and want to keep it going, use

		|\pgfplotsset{compat/anchors=pre 1.3}|\pgfmanualpdflabel{/pgfplots/compat/anchors}{} 

		to disable the bug fix.
	\end{enumerate}

	Use |\pgfplotsset{compat=default}| to restore the factory settings.

	The setting |\pgfplotsset{compat=newest}| will always use features of the most recent version. This might result in small changes in the document's appearance.
\end{pgfplotskey}

\subsection{The Team}
\PGFPlots\ has been written mainly by Christian Feuers�nger with many improvements of Pascal Wolkotte and Nick Papior Andersen as a spare time project. We hope it is useful and provides valuable plots.

If you are interested in writing something but don't know how, consider reading the auxiliary manual \href{file:TeX-programming-notes.pdf}{TeX-programming-notes.pdf} which comes with \PGFPlots. It is far from complete, but maybe it is a good starting point (at least for more literature).

\subsection{Acknowledgements}
I thank God for all hours of enjoyed programming. I thank Pascal Wolkotte and Nick Papior Andersen for their programming efforts and contributions as part of the development team. I thank Stefan Tibus, who contributed the |plot shell| feature. I thank Tom Cashman for the contribution of the |reverse legend| feature. Special thanks go to Stefan Pinnow whose tests of \PGFPlots\ lead to numerous quality improvements. Furthermore, I thank Dr.~Schweitzer for many fruitful discussions and Dr.~Meine for his ideas and suggestions. Thanks as well to the many international contributors who provided feature requests or identified bugs or simply manual improvements!

Last but not least, I thank Till Tantau and Mark Wibrow for their excellent graphics (and more) package \PGF\ and \Tikz\ which is the base of \PGFPlots.

%%%%%%%%%%%%%%%%%%%%%%%%%%%%%%%%%%%%%%%%%%%%%%%%%%%%%%%%%%%%%%%%%%%%%%%%%%%%%
%
% Package pgfplots.sty documentation. 
%
% Copyright 2007/2008 by Christian Feuersaenger.
%
% This program is free software: you can redistribute it and/or modify
% it under the terms of the GNU General Public License as published by
% the Free Software Foundation, either version 3 of the License, or
% (at your option) any later version.
% 
% This program is distributed in the hope that it will be useful,
% but WITHOUT ANY WARRANTY; without even the implied warranty of
% MERCHANTABILITY or FITNESS FOR A PARTICULAR PURPOSE.  See the
% GNU General Public License for more details.
% 
% You should have received a copy of the GNU General Public License
% along with this program.  If not, see <http://www.gnu.org/licenses/>.
%
%
%%%%%%%%%%%%%%%%%%%%%%%%%%%%%%%%%%%%%%%%%%%%%%%%%%%%%%%%%%%%%%%%%%%%%%%%%%%%%
\input{pgfplots.preamble.tex}
%\RequirePackage[german,english,francais]{babel}

\pgfplotsmanualexternalexpensivetrue

%\usetikzlibrary{external}
%\tikzexternalize[prefix=figures/]{pgfplots}

\title{%
	Manual for Package \PGFPlots\\
	{\small Version \pgfplotsversion}\\
	{\small\href{http://sourceforge.net/projects/pgfplots}{http://sourceforge.net/projects/pgfplots}}}

%\includeonly{pgfplots.libs}


\begin{document}

\def\plotcoords{%
\addplot coordinates {
(5,8.312e-02)    (17,2.547e-02)   (49,7.407e-03)
(129,2.102e-03)  (321,5.874e-04)  (769,1.623e-04)
(1793,4.442e-05) (4097,1.207e-05) (9217,3.261e-06)
};

\addplot coordinates{
(7,8.472e-02)    (31,3.044e-02)    (111,1.022e-02)
(351,3.303e-03)  (1023,1.039e-03)  (2815,3.196e-04)
(7423,9.658e-05) (18943,2.873e-05) (47103,8.437e-06)
};

\addplot coordinates{
(9,7.881e-02)     (49,3.243e-02)    (209,1.232e-02)
(769,4.454e-03)   (2561,1.551e-03)  (7937,5.236e-04)
(23297,1.723e-04) (65537,5.545e-05) (178177,1.751e-05)
};

\addplot coordinates{
(11,6.887e-02)    (71,3.177e-02)     (351,1.341e-02)
(1471,5.334e-03)  (5503,2.027e-03)   (18943,7.415e-04)
(61183,2.628e-04) (187903,9.063e-05) (553983,3.053e-05)
};

\addplot coordinates{
(13,5.755e-02)     (97,2.925e-02)     (545,1.351e-02)
(2561,5.842e-03)   (10625,2.397e-03)  (40193,9.414e-04)
(141569,3.564e-04) (471041,1.308e-04) 
(1496065,4.670e-05)
};
}%
\maketitle
\begin{abstract}%
\PGFPlots\ draws high--quality function plots in normal or logarithmic scaling with a user-friendly interface directly in \TeX. The user supplies axis labels, legend entries and the plot coordinates for one or more plots and \PGFPlots\ applies axis scaling, computes any logarithms and axis ticks and draws the plots, supporting line plots, scatter plots, piecewise constant plots, bar plots, area plots, mesh-- and surface plots and some more. It is based on Till Tantau's package \PGF/\Tikz.
\end{abstract}
\tableofcontents
\section{Introduction}
This package provides tools to generate plots and labeled axes easily. It draws normal plots, logplots and semi-logplots, in two and three dimensions. Axis ticks, labels, legends (in case of multiple plots) can be added with key-value options. It can cycle through a set of predefined line/marker/color specifications. In summary, its purpose is to simplify the generation of high-quality function and/or data plots, and solving the problems of
\begin{itemize}
	\item consistency of document and font type and font size,
	\item direct use of \TeX\ math mode in axis descriptions,
	\item consistency of data and figures (no third party tool necessary),
	\item inter document consistency using preamble configurations and styles.
\end{itemize}
Although not necessary, separate |.pdf| or |.eps| graphics can be generated using the |external| library developed as part of \Tikz.

\PGFPlots\ is build completely on \Tikz/\PGF. Knowledge of \Tikz\ will simplify the work with \PGFPlots, although it is not required.

\section[About PGFPlots: Preliminaries]{About {\normalfont\PGFPlots}: Preliminaries}
This section contains information about upgrades, the team, the installation (in case you need to do it manually) and troubleshooting. You may skip it completely except for the upgrade remarks.

However, note that this library requires at least \PGF\ version $2.00$. At the time of this writing, many \TeX-distributions still contain the older \PGF\ version $1.18$, so it may be necessary to install a recent \PGF\ prior to using \PGFPlots.

\subsection{Components}
\PGFPlots\ comes with two components:
\begin{enumerate}
	\item the plotting component (which you are currently reading) and
	\item the \PGFPlotstable\ component which simplifies number formatting and postprocessing of numerical tables. It comes as a separate package and has its own manual \href{file:pgfplotstable.pdf}{pgfplotstable.pdf}.
\end{enumerate}

\subsection{Upgrade remarks}
This release provides a lot of improvements which can be found in all detail in \texttt{ChangeLog} for interested readers. However, some attention is useful with respect to the following changes.

\subsubsection{New Optional Features}
\begin{enumerate}
	\item \PGFPlots\ 1.3 comes with user interface improvements. The technical distinction between ``behavior options'' and ``style options'' of older versions is no longer necessary (although still fully supported).

	\item \PGFPlots\ 1.3 has a new feature which allows to \emph{move axis labels tight to tick labels} automatically.

	Since this affects the spacing, it is not enabled be default.

	Use
\begin{codeexample}[code only]
\usepackage{pgfplots}
\pgfplotsset{compat=1.3}
\end{codeexample}
	in your preamble to benefit from the improved spacing\footnote{The |compat=1.3| setting is necessary for new features which move axis labels around like reversed axis. However, \PGFPlots\ will still work as it does in earlier versions, your documents will remain the same if you don't set it explicitly.}.
	Take a look at the next page for the precise definition of |compat|.

	\item \PGFPlots\ 1.3 now supports reversed axes. It is no longer necessary to use work--arounds with negative units.
\pgfkeys{/pdflinks/search key prefixes in/.add={/pgfplots/,}{}}

	Take a look at the |x dir=reverse| key.

	Existing work--arounds will still function properly. Use |\pgfplotsset{compat=1.3}| together with |x dir=reverse| to switch to the new version.
\end{enumerate}

\subsubsection{Old Features Which May Need Attention}
\begin{enumerate}
	\item The |scatter/classes| feature produces proper legends as of version 1.3. This may change the appearance of existing legends of plots with |scatter/classes|.

	\item Due to a small math bug in \PGF\ $2.00$, you \emph{can't} use the math expression `|-x^2|'. It is necessary to use `|0-x^2|' instead. The same holds for
		`|exp(-x^2)|'; use `|exp(0-x^2)|' instead.

		This will be fixed with \PGF\ $>2.00$.

	\item Starting with \PGFPlots\ $1.1$, |\tikzstyle| should \emph{no longer be used} to set \PGFPlots\ options.
	
	Although |\tikzstyle| is still supported for some older \PGFPlots\ options, you should replace any occurance of |\tikzstyle| with |\pgfplotsset{|\meta{style name}|/.style={|\meta{key-value-list}|}}| or the associated |/.append style| variant. See section~\ref{sec:styles} for more detail.
\end{enumerate}
I apologize for any inconvenience caused by these changes.

\begin{pgfplotskey}{compat=\mchoice{1.3,pre 1.3,default,newest} (initially default)}
	The preamble configuration 
\begin{codeexample}[code only]
\usepackage{pgfplots}
\pgfplotsset{compat=1.3}
\end{codeexample}
	allows to choose between backwards compatibility and most recent features.

	\PGFPlots\ version 1.3 comes with new features which may lead to different spacings compared to earlier versions:
	\begin{enumerate}
		\item Version 1.3 comes with the |xlabel near ticks| feature which places (in this case $x$) axis labels ``near'' the tick labels, taking the size of tick labels into account.
		
		Older versions used |xlabel absolute|, i.e.\ an absolute distance between the axis and axis labels (now matter how large or small tick labels are).

		The initial setting \emph{keeps backwards compatibility}.

		You are encouraged to use
\begin{codeexample}[code only]
\usepackage{pgfplots}
\pgfplotsset{compat=1.3}
\end{codeexample}
		in your preamble.
		
		\item Version 1.3 fixes a bug with |anchor=south west| which occurred mainly for reversed axes: the anchors where upside--down. This is fixed now.

		
		If you have written some work--around and want to keep it going, use

		|\pgfplotsset{compat/anchors=pre 1.3}|\pgfmanualpdflabel{/pgfplots/compat/anchors}{} 

		to disable the bug fix.
	\end{enumerate}

	Use |\pgfplotsset{compat=default}| to restore the factory settings.

	The setting |\pgfplotsset{compat=newest}| will always use features of the most recent version. This might result in small changes in the document's appearance.
\end{pgfplotskey}

\subsection{The Team}
\PGFPlots\ has been written mainly by Christian Feuers�nger with many improvements of Pascal Wolkotte and Nick Papior Andersen as a spare time project. We hope it is useful and provides valuable plots.

If you are interested in writing something but don't know how, consider reading the auxiliary manual \href{file:TeX-programming-notes.pdf}{TeX-programming-notes.pdf} which comes with \PGFPlots. It is far from complete, but maybe it is a good starting point (at least for more literature).

\subsection{Acknowledgements}
I thank God for all hours of enjoyed programming. I thank Pascal Wolkotte and Nick Papior Andersen for their programming efforts and contributions as part of the development team. I thank Stefan Tibus, who contributed the |plot shell| feature. I thank Tom Cashman for the contribution of the |reverse legend| feature. Special thanks go to Stefan Pinnow whose tests of \PGFPlots\ lead to numerous quality improvements. Furthermore, I thank Dr.~Schweitzer for many fruitful discussions and Dr.~Meine for his ideas and suggestions. Thanks as well to the many international contributors who provided feature requests or identified bugs or simply manual improvements!

Last but not least, I thank Till Tantau and Mark Wibrow for their excellent graphics (and more) package \PGF\ and \Tikz\ which is the base of \PGFPlots.

\include{pgfplots.install}%
\include{pgfplots.intro}%
\include{pgfplots.reference}%
\include{pgfplots.libs}%
\include{pgfplots.resources}%
\include{pgfplots.importexport}%
\include{pgfplots.basic.reference}%

\printindex

\bibliographystyle{abbrv} %gerapali} %gerabbrv} %gerunsrt.bst} %gerabbrv}% gerplain}
\nocite{pgfplotstable}
\nocite{programmingnotes}
\bibliography{pgfplots}
\end{document}
%
%%%%%%%%%%%%%%%%%%%%%%%%%%%%%%%%%%%%%%%%%%%%%%%%%%%%%%%%%%%%%%%%%%%%%%%%%%%%%
%
% Package pgfplots.sty documentation. 
%
% Copyright 2007/2008 by Christian Feuersaenger.
%
% This program is free software: you can redistribute it and/or modify
% it under the terms of the GNU General Public License as published by
% the Free Software Foundation, either version 3 of the License, or
% (at your option) any later version.
% 
% This program is distributed in the hope that it will be useful,
% but WITHOUT ANY WARRANTY; without even the implied warranty of
% MERCHANTABILITY or FITNESS FOR A PARTICULAR PURPOSE.  See the
% GNU General Public License for more details.
% 
% You should have received a copy of the GNU General Public License
% along with this program.  If not, see <http://www.gnu.org/licenses/>.
%
%
%%%%%%%%%%%%%%%%%%%%%%%%%%%%%%%%%%%%%%%%%%%%%%%%%%%%%%%%%%%%%%%%%%%%%%%%%%%%%
\input{pgfplots.preamble.tex}
%\RequirePackage[german,english,francais]{babel}

\pgfplotsmanualexternalexpensivetrue

%\usetikzlibrary{external}
%\tikzexternalize[prefix=figures/]{pgfplots}

\title{%
	Manual for Package \PGFPlots\\
	{\small Version \pgfplotsversion}\\
	{\small\href{http://sourceforge.net/projects/pgfplots}{http://sourceforge.net/projects/pgfplots}}}

%\includeonly{pgfplots.libs}


\begin{document}

\def\plotcoords{%
\addplot coordinates {
(5,8.312e-02)    (17,2.547e-02)   (49,7.407e-03)
(129,2.102e-03)  (321,5.874e-04)  (769,1.623e-04)
(1793,4.442e-05) (4097,1.207e-05) (9217,3.261e-06)
};

\addplot coordinates{
(7,8.472e-02)    (31,3.044e-02)    (111,1.022e-02)
(351,3.303e-03)  (1023,1.039e-03)  (2815,3.196e-04)
(7423,9.658e-05) (18943,2.873e-05) (47103,8.437e-06)
};

\addplot coordinates{
(9,7.881e-02)     (49,3.243e-02)    (209,1.232e-02)
(769,4.454e-03)   (2561,1.551e-03)  (7937,5.236e-04)
(23297,1.723e-04) (65537,5.545e-05) (178177,1.751e-05)
};

\addplot coordinates{
(11,6.887e-02)    (71,3.177e-02)     (351,1.341e-02)
(1471,5.334e-03)  (5503,2.027e-03)   (18943,7.415e-04)
(61183,2.628e-04) (187903,9.063e-05) (553983,3.053e-05)
};

\addplot coordinates{
(13,5.755e-02)     (97,2.925e-02)     (545,1.351e-02)
(2561,5.842e-03)   (10625,2.397e-03)  (40193,9.414e-04)
(141569,3.564e-04) (471041,1.308e-04) 
(1496065,4.670e-05)
};
}%
\maketitle
\begin{abstract}%
\PGFPlots\ draws high--quality function plots in normal or logarithmic scaling with a user-friendly interface directly in \TeX. The user supplies axis labels, legend entries and the plot coordinates for one or more plots and \PGFPlots\ applies axis scaling, computes any logarithms and axis ticks and draws the plots, supporting line plots, scatter plots, piecewise constant plots, bar plots, area plots, mesh-- and surface plots and some more. It is based on Till Tantau's package \PGF/\Tikz.
\end{abstract}
\tableofcontents
\section{Introduction}
This package provides tools to generate plots and labeled axes easily. It draws normal plots, logplots and semi-logplots, in two and three dimensions. Axis ticks, labels, legends (in case of multiple plots) can be added with key-value options. It can cycle through a set of predefined line/marker/color specifications. In summary, its purpose is to simplify the generation of high-quality function and/or data plots, and solving the problems of
\begin{itemize}
	\item consistency of document and font type and font size,
	\item direct use of \TeX\ math mode in axis descriptions,
	\item consistency of data and figures (no third party tool necessary),
	\item inter document consistency using preamble configurations and styles.
\end{itemize}
Although not necessary, separate |.pdf| or |.eps| graphics can be generated using the |external| library developed as part of \Tikz.

\PGFPlots\ is build completely on \Tikz/\PGF. Knowledge of \Tikz\ will simplify the work with \PGFPlots, although it is not required.

\section[About PGFPlots: Preliminaries]{About {\normalfont\PGFPlots}: Preliminaries}
This section contains information about upgrades, the team, the installation (in case you need to do it manually) and troubleshooting. You may skip it completely except for the upgrade remarks.

However, note that this library requires at least \PGF\ version $2.00$. At the time of this writing, many \TeX-distributions still contain the older \PGF\ version $1.18$, so it may be necessary to install a recent \PGF\ prior to using \PGFPlots.

\subsection{Components}
\PGFPlots\ comes with two components:
\begin{enumerate}
	\item the plotting component (which you are currently reading) and
	\item the \PGFPlotstable\ component which simplifies number formatting and postprocessing of numerical tables. It comes as a separate package and has its own manual \href{file:pgfplotstable.pdf}{pgfplotstable.pdf}.
\end{enumerate}

\subsection{Upgrade remarks}
This release provides a lot of improvements which can be found in all detail in \texttt{ChangeLog} for interested readers. However, some attention is useful with respect to the following changes.

\subsubsection{New Optional Features}
\begin{enumerate}
	\item \PGFPlots\ 1.3 comes with user interface improvements. The technical distinction between ``behavior options'' and ``style options'' of older versions is no longer necessary (although still fully supported).

	\item \PGFPlots\ 1.3 has a new feature which allows to \emph{move axis labels tight to tick labels} automatically.

	Since this affects the spacing, it is not enabled be default.

	Use
\begin{codeexample}[code only]
\usepackage{pgfplots}
\pgfplotsset{compat=1.3}
\end{codeexample}
	in your preamble to benefit from the improved spacing\footnote{The |compat=1.3| setting is necessary for new features which move axis labels around like reversed axis. However, \PGFPlots\ will still work as it does in earlier versions, your documents will remain the same if you don't set it explicitly.}.
	Take a look at the next page for the precise definition of |compat|.

	\item \PGFPlots\ 1.3 now supports reversed axes. It is no longer necessary to use work--arounds with negative units.
\pgfkeys{/pdflinks/search key prefixes in/.add={/pgfplots/,}{}}

	Take a look at the |x dir=reverse| key.

	Existing work--arounds will still function properly. Use |\pgfplotsset{compat=1.3}| together with |x dir=reverse| to switch to the new version.
\end{enumerate}

\subsubsection{Old Features Which May Need Attention}
\begin{enumerate}
	\item The |scatter/classes| feature produces proper legends as of version 1.3. This may change the appearance of existing legends of plots with |scatter/classes|.

	\item Due to a small math bug in \PGF\ $2.00$, you \emph{can't} use the math expression `|-x^2|'. It is necessary to use `|0-x^2|' instead. The same holds for
		`|exp(-x^2)|'; use `|exp(0-x^2)|' instead.

		This will be fixed with \PGF\ $>2.00$.

	\item Starting with \PGFPlots\ $1.1$, |\tikzstyle| should \emph{no longer be used} to set \PGFPlots\ options.
	
	Although |\tikzstyle| is still supported for some older \PGFPlots\ options, you should replace any occurance of |\tikzstyle| with |\pgfplotsset{|\meta{style name}|/.style={|\meta{key-value-list}|}}| or the associated |/.append style| variant. See section~\ref{sec:styles} for more detail.
\end{enumerate}
I apologize for any inconvenience caused by these changes.

\begin{pgfplotskey}{compat=\mchoice{1.3,pre 1.3,default,newest} (initially default)}
	The preamble configuration 
\begin{codeexample}[code only]
\usepackage{pgfplots}
\pgfplotsset{compat=1.3}
\end{codeexample}
	allows to choose between backwards compatibility and most recent features.

	\PGFPlots\ version 1.3 comes with new features which may lead to different spacings compared to earlier versions:
	\begin{enumerate}
		\item Version 1.3 comes with the |xlabel near ticks| feature which places (in this case $x$) axis labels ``near'' the tick labels, taking the size of tick labels into account.
		
		Older versions used |xlabel absolute|, i.e.\ an absolute distance between the axis and axis labels (now matter how large or small tick labels are).

		The initial setting \emph{keeps backwards compatibility}.

		You are encouraged to use
\begin{codeexample}[code only]
\usepackage{pgfplots}
\pgfplotsset{compat=1.3}
\end{codeexample}
		in your preamble.
		
		\item Version 1.3 fixes a bug with |anchor=south west| which occurred mainly for reversed axes: the anchors where upside--down. This is fixed now.

		
		If you have written some work--around and want to keep it going, use

		|\pgfplotsset{compat/anchors=pre 1.3}|\pgfmanualpdflabel{/pgfplots/compat/anchors}{} 

		to disable the bug fix.
	\end{enumerate}

	Use |\pgfplotsset{compat=default}| to restore the factory settings.

	The setting |\pgfplotsset{compat=newest}| will always use features of the most recent version. This might result in small changes in the document's appearance.
\end{pgfplotskey}

\subsection{The Team}
\PGFPlots\ has been written mainly by Christian Feuers�nger with many improvements of Pascal Wolkotte and Nick Papior Andersen as a spare time project. We hope it is useful and provides valuable plots.

If you are interested in writing something but don't know how, consider reading the auxiliary manual \href{file:TeX-programming-notes.pdf}{TeX-programming-notes.pdf} which comes with \PGFPlots. It is far from complete, but maybe it is a good starting point (at least for more literature).

\subsection{Acknowledgements}
I thank God for all hours of enjoyed programming. I thank Pascal Wolkotte and Nick Papior Andersen for their programming efforts and contributions as part of the development team. I thank Stefan Tibus, who contributed the |plot shell| feature. I thank Tom Cashman for the contribution of the |reverse legend| feature. Special thanks go to Stefan Pinnow whose tests of \PGFPlots\ lead to numerous quality improvements. Furthermore, I thank Dr.~Schweitzer for many fruitful discussions and Dr.~Meine for his ideas and suggestions. Thanks as well to the many international contributors who provided feature requests or identified bugs or simply manual improvements!

Last but not least, I thank Till Tantau and Mark Wibrow for their excellent graphics (and more) package \PGF\ and \Tikz\ which is the base of \PGFPlots.

\include{pgfplots.install}%
\include{pgfplots.intro}%
\include{pgfplots.reference}%
\include{pgfplots.libs}%
\include{pgfplots.resources}%
\include{pgfplots.importexport}%
\include{pgfplots.basic.reference}%

\printindex

\bibliographystyle{abbrv} %gerapali} %gerabbrv} %gerunsrt.bst} %gerabbrv}% gerplain}
\nocite{pgfplotstable}
\nocite{programmingnotes}
\bibliography{pgfplots}
\end{document}
%
%%%%%%%%%%%%%%%%%%%%%%%%%%%%%%%%%%%%%%%%%%%%%%%%%%%%%%%%%%%%%%%%%%%%%%%%%%%%%
%
% Package pgfplots.sty documentation. 
%
% Copyright 2007/2008 by Christian Feuersaenger.
%
% This program is free software: you can redistribute it and/or modify
% it under the terms of the GNU General Public License as published by
% the Free Software Foundation, either version 3 of the License, or
% (at your option) any later version.
% 
% This program is distributed in the hope that it will be useful,
% but WITHOUT ANY WARRANTY; without even the implied warranty of
% MERCHANTABILITY or FITNESS FOR A PARTICULAR PURPOSE.  See the
% GNU General Public License for more details.
% 
% You should have received a copy of the GNU General Public License
% along with this program.  If not, see <http://www.gnu.org/licenses/>.
%
%
%%%%%%%%%%%%%%%%%%%%%%%%%%%%%%%%%%%%%%%%%%%%%%%%%%%%%%%%%%%%%%%%%%%%%%%%%%%%%
\input{pgfplots.preamble.tex}
%\RequirePackage[german,english,francais]{babel}

\pgfplotsmanualexternalexpensivetrue

%\usetikzlibrary{external}
%\tikzexternalize[prefix=figures/]{pgfplots}

\title{%
	Manual for Package \PGFPlots\\
	{\small Version \pgfplotsversion}\\
	{\small\href{http://sourceforge.net/projects/pgfplots}{http://sourceforge.net/projects/pgfplots}}}

%\includeonly{pgfplots.libs}


\begin{document}

\def\plotcoords{%
\addplot coordinates {
(5,8.312e-02)    (17,2.547e-02)   (49,7.407e-03)
(129,2.102e-03)  (321,5.874e-04)  (769,1.623e-04)
(1793,4.442e-05) (4097,1.207e-05) (9217,3.261e-06)
};

\addplot coordinates{
(7,8.472e-02)    (31,3.044e-02)    (111,1.022e-02)
(351,3.303e-03)  (1023,1.039e-03)  (2815,3.196e-04)
(7423,9.658e-05) (18943,2.873e-05) (47103,8.437e-06)
};

\addplot coordinates{
(9,7.881e-02)     (49,3.243e-02)    (209,1.232e-02)
(769,4.454e-03)   (2561,1.551e-03)  (7937,5.236e-04)
(23297,1.723e-04) (65537,5.545e-05) (178177,1.751e-05)
};

\addplot coordinates{
(11,6.887e-02)    (71,3.177e-02)     (351,1.341e-02)
(1471,5.334e-03)  (5503,2.027e-03)   (18943,7.415e-04)
(61183,2.628e-04) (187903,9.063e-05) (553983,3.053e-05)
};

\addplot coordinates{
(13,5.755e-02)     (97,2.925e-02)     (545,1.351e-02)
(2561,5.842e-03)   (10625,2.397e-03)  (40193,9.414e-04)
(141569,3.564e-04) (471041,1.308e-04) 
(1496065,4.670e-05)
};
}%
\maketitle
\begin{abstract}%
\PGFPlots\ draws high--quality function plots in normal or logarithmic scaling with a user-friendly interface directly in \TeX. The user supplies axis labels, legend entries and the plot coordinates for one or more plots and \PGFPlots\ applies axis scaling, computes any logarithms and axis ticks and draws the plots, supporting line plots, scatter plots, piecewise constant plots, bar plots, area plots, mesh-- and surface plots and some more. It is based on Till Tantau's package \PGF/\Tikz.
\end{abstract}
\tableofcontents
\section{Introduction}
This package provides tools to generate plots and labeled axes easily. It draws normal plots, logplots and semi-logplots, in two and three dimensions. Axis ticks, labels, legends (in case of multiple plots) can be added with key-value options. It can cycle through a set of predefined line/marker/color specifications. In summary, its purpose is to simplify the generation of high-quality function and/or data plots, and solving the problems of
\begin{itemize}
	\item consistency of document and font type and font size,
	\item direct use of \TeX\ math mode in axis descriptions,
	\item consistency of data and figures (no third party tool necessary),
	\item inter document consistency using preamble configurations and styles.
\end{itemize}
Although not necessary, separate |.pdf| or |.eps| graphics can be generated using the |external| library developed as part of \Tikz.

\PGFPlots\ is build completely on \Tikz/\PGF. Knowledge of \Tikz\ will simplify the work with \PGFPlots, although it is not required.

\section[About PGFPlots: Preliminaries]{About {\normalfont\PGFPlots}: Preliminaries}
This section contains information about upgrades, the team, the installation (in case you need to do it manually) and troubleshooting. You may skip it completely except for the upgrade remarks.

However, note that this library requires at least \PGF\ version $2.00$. At the time of this writing, many \TeX-distributions still contain the older \PGF\ version $1.18$, so it may be necessary to install a recent \PGF\ prior to using \PGFPlots.

\subsection{Components}
\PGFPlots\ comes with two components:
\begin{enumerate}
	\item the plotting component (which you are currently reading) and
	\item the \PGFPlotstable\ component which simplifies number formatting and postprocessing of numerical tables. It comes as a separate package and has its own manual \href{file:pgfplotstable.pdf}{pgfplotstable.pdf}.
\end{enumerate}

\subsection{Upgrade remarks}
This release provides a lot of improvements which can be found in all detail in \texttt{ChangeLog} for interested readers. However, some attention is useful with respect to the following changes.

\subsubsection{New Optional Features}
\begin{enumerate}
	\item \PGFPlots\ 1.3 comes with user interface improvements. The technical distinction between ``behavior options'' and ``style options'' of older versions is no longer necessary (although still fully supported).

	\item \PGFPlots\ 1.3 has a new feature which allows to \emph{move axis labels tight to tick labels} automatically.

	Since this affects the spacing, it is not enabled be default.

	Use
\begin{codeexample}[code only]
\usepackage{pgfplots}
\pgfplotsset{compat=1.3}
\end{codeexample}
	in your preamble to benefit from the improved spacing\footnote{The |compat=1.3| setting is necessary for new features which move axis labels around like reversed axis. However, \PGFPlots\ will still work as it does in earlier versions, your documents will remain the same if you don't set it explicitly.}.
	Take a look at the next page for the precise definition of |compat|.

	\item \PGFPlots\ 1.3 now supports reversed axes. It is no longer necessary to use work--arounds with negative units.
\pgfkeys{/pdflinks/search key prefixes in/.add={/pgfplots/,}{}}

	Take a look at the |x dir=reverse| key.

	Existing work--arounds will still function properly. Use |\pgfplotsset{compat=1.3}| together with |x dir=reverse| to switch to the new version.
\end{enumerate}

\subsubsection{Old Features Which May Need Attention}
\begin{enumerate}
	\item The |scatter/classes| feature produces proper legends as of version 1.3. This may change the appearance of existing legends of plots with |scatter/classes|.

	\item Due to a small math bug in \PGF\ $2.00$, you \emph{can't} use the math expression `|-x^2|'. It is necessary to use `|0-x^2|' instead. The same holds for
		`|exp(-x^2)|'; use `|exp(0-x^2)|' instead.

		This will be fixed with \PGF\ $>2.00$.

	\item Starting with \PGFPlots\ $1.1$, |\tikzstyle| should \emph{no longer be used} to set \PGFPlots\ options.
	
	Although |\tikzstyle| is still supported for some older \PGFPlots\ options, you should replace any occurance of |\tikzstyle| with |\pgfplotsset{|\meta{style name}|/.style={|\meta{key-value-list}|}}| or the associated |/.append style| variant. See section~\ref{sec:styles} for more detail.
\end{enumerate}
I apologize for any inconvenience caused by these changes.

\begin{pgfplotskey}{compat=\mchoice{1.3,pre 1.3,default,newest} (initially default)}
	The preamble configuration 
\begin{codeexample}[code only]
\usepackage{pgfplots}
\pgfplotsset{compat=1.3}
\end{codeexample}
	allows to choose between backwards compatibility and most recent features.

	\PGFPlots\ version 1.3 comes with new features which may lead to different spacings compared to earlier versions:
	\begin{enumerate}
		\item Version 1.3 comes with the |xlabel near ticks| feature which places (in this case $x$) axis labels ``near'' the tick labels, taking the size of tick labels into account.
		
		Older versions used |xlabel absolute|, i.e.\ an absolute distance between the axis and axis labels (now matter how large or small tick labels are).

		The initial setting \emph{keeps backwards compatibility}.

		You are encouraged to use
\begin{codeexample}[code only]
\usepackage{pgfplots}
\pgfplotsset{compat=1.3}
\end{codeexample}
		in your preamble.
		
		\item Version 1.3 fixes a bug with |anchor=south west| which occurred mainly for reversed axes: the anchors where upside--down. This is fixed now.

		
		If you have written some work--around and want to keep it going, use

		|\pgfplotsset{compat/anchors=pre 1.3}|\pgfmanualpdflabel{/pgfplots/compat/anchors}{} 

		to disable the bug fix.
	\end{enumerate}

	Use |\pgfplotsset{compat=default}| to restore the factory settings.

	The setting |\pgfplotsset{compat=newest}| will always use features of the most recent version. This might result in small changes in the document's appearance.
\end{pgfplotskey}

\subsection{The Team}
\PGFPlots\ has been written mainly by Christian Feuers�nger with many improvements of Pascal Wolkotte and Nick Papior Andersen as a spare time project. We hope it is useful and provides valuable plots.

If you are interested in writing something but don't know how, consider reading the auxiliary manual \href{file:TeX-programming-notes.pdf}{TeX-programming-notes.pdf} which comes with \PGFPlots. It is far from complete, but maybe it is a good starting point (at least for more literature).

\subsection{Acknowledgements}
I thank God for all hours of enjoyed programming. I thank Pascal Wolkotte and Nick Papior Andersen for their programming efforts and contributions as part of the development team. I thank Stefan Tibus, who contributed the |plot shell| feature. I thank Tom Cashman for the contribution of the |reverse legend| feature. Special thanks go to Stefan Pinnow whose tests of \PGFPlots\ lead to numerous quality improvements. Furthermore, I thank Dr.~Schweitzer for many fruitful discussions and Dr.~Meine for his ideas and suggestions. Thanks as well to the many international contributors who provided feature requests or identified bugs or simply manual improvements!

Last but not least, I thank Till Tantau and Mark Wibrow for their excellent graphics (and more) package \PGF\ and \Tikz\ which is the base of \PGFPlots.

\include{pgfplots.install}%
\include{pgfplots.intro}%
\include{pgfplots.reference}%
\include{pgfplots.libs}%
\include{pgfplots.resources}%
\include{pgfplots.importexport}%
\include{pgfplots.basic.reference}%

\printindex

\bibliographystyle{abbrv} %gerapali} %gerabbrv} %gerunsrt.bst} %gerabbrv}% gerplain}
\nocite{pgfplotstable}
\nocite{programmingnotes}
\bibliography{pgfplots}
\end{document}
%
%%%%%%%%%%%%%%%%%%%%%%%%%%%%%%%%%%%%%%%%%%%%%%%%%%%%%%%%%%%%%%%%%%%%%%%%%%%%%
%
% Package pgfplots.sty documentation. 
%
% Copyright 2007/2008 by Christian Feuersaenger.
%
% This program is free software: you can redistribute it and/or modify
% it under the terms of the GNU General Public License as published by
% the Free Software Foundation, either version 3 of the License, or
% (at your option) any later version.
% 
% This program is distributed in the hope that it will be useful,
% but WITHOUT ANY WARRANTY; without even the implied warranty of
% MERCHANTABILITY or FITNESS FOR A PARTICULAR PURPOSE.  See the
% GNU General Public License for more details.
% 
% You should have received a copy of the GNU General Public License
% along with this program.  If not, see <http://www.gnu.org/licenses/>.
%
%
%%%%%%%%%%%%%%%%%%%%%%%%%%%%%%%%%%%%%%%%%%%%%%%%%%%%%%%%%%%%%%%%%%%%%%%%%%%%%
\input{pgfplots.preamble.tex}
%\RequirePackage[german,english,francais]{babel}

\pgfplotsmanualexternalexpensivetrue

%\usetikzlibrary{external}
%\tikzexternalize[prefix=figures/]{pgfplots}

\title{%
	Manual for Package \PGFPlots\\
	{\small Version \pgfplotsversion}\\
	{\small\href{http://sourceforge.net/projects/pgfplots}{http://sourceforge.net/projects/pgfplots}}}

%\includeonly{pgfplots.libs}


\begin{document}

\def\plotcoords{%
\addplot coordinates {
(5,8.312e-02)    (17,2.547e-02)   (49,7.407e-03)
(129,2.102e-03)  (321,5.874e-04)  (769,1.623e-04)
(1793,4.442e-05) (4097,1.207e-05) (9217,3.261e-06)
};

\addplot coordinates{
(7,8.472e-02)    (31,3.044e-02)    (111,1.022e-02)
(351,3.303e-03)  (1023,1.039e-03)  (2815,3.196e-04)
(7423,9.658e-05) (18943,2.873e-05) (47103,8.437e-06)
};

\addplot coordinates{
(9,7.881e-02)     (49,3.243e-02)    (209,1.232e-02)
(769,4.454e-03)   (2561,1.551e-03)  (7937,5.236e-04)
(23297,1.723e-04) (65537,5.545e-05) (178177,1.751e-05)
};

\addplot coordinates{
(11,6.887e-02)    (71,3.177e-02)     (351,1.341e-02)
(1471,5.334e-03)  (5503,2.027e-03)   (18943,7.415e-04)
(61183,2.628e-04) (187903,9.063e-05) (553983,3.053e-05)
};

\addplot coordinates{
(13,5.755e-02)     (97,2.925e-02)     (545,1.351e-02)
(2561,5.842e-03)   (10625,2.397e-03)  (40193,9.414e-04)
(141569,3.564e-04) (471041,1.308e-04) 
(1496065,4.670e-05)
};
}%
\maketitle
\begin{abstract}%
\PGFPlots\ draws high--quality function plots in normal or logarithmic scaling with a user-friendly interface directly in \TeX. The user supplies axis labels, legend entries and the plot coordinates for one or more plots and \PGFPlots\ applies axis scaling, computes any logarithms and axis ticks and draws the plots, supporting line plots, scatter plots, piecewise constant plots, bar plots, area plots, mesh-- and surface plots and some more. It is based on Till Tantau's package \PGF/\Tikz.
\end{abstract}
\tableofcontents
\section{Introduction}
This package provides tools to generate plots and labeled axes easily. It draws normal plots, logplots and semi-logplots, in two and three dimensions. Axis ticks, labels, legends (in case of multiple plots) can be added with key-value options. It can cycle through a set of predefined line/marker/color specifications. In summary, its purpose is to simplify the generation of high-quality function and/or data plots, and solving the problems of
\begin{itemize}
	\item consistency of document and font type and font size,
	\item direct use of \TeX\ math mode in axis descriptions,
	\item consistency of data and figures (no third party tool necessary),
	\item inter document consistency using preamble configurations and styles.
\end{itemize}
Although not necessary, separate |.pdf| or |.eps| graphics can be generated using the |external| library developed as part of \Tikz.

\PGFPlots\ is build completely on \Tikz/\PGF. Knowledge of \Tikz\ will simplify the work with \PGFPlots, although it is not required.

\section[About PGFPlots: Preliminaries]{About {\normalfont\PGFPlots}: Preliminaries}
This section contains information about upgrades, the team, the installation (in case you need to do it manually) and troubleshooting. You may skip it completely except for the upgrade remarks.

However, note that this library requires at least \PGF\ version $2.00$. At the time of this writing, many \TeX-distributions still contain the older \PGF\ version $1.18$, so it may be necessary to install a recent \PGF\ prior to using \PGFPlots.

\subsection{Components}
\PGFPlots\ comes with two components:
\begin{enumerate}
	\item the plotting component (which you are currently reading) and
	\item the \PGFPlotstable\ component which simplifies number formatting and postprocessing of numerical tables. It comes as a separate package and has its own manual \href{file:pgfplotstable.pdf}{pgfplotstable.pdf}.
\end{enumerate}

\subsection{Upgrade remarks}
This release provides a lot of improvements which can be found in all detail in \texttt{ChangeLog} for interested readers. However, some attention is useful with respect to the following changes.

\subsubsection{New Optional Features}
\begin{enumerate}
	\item \PGFPlots\ 1.3 comes with user interface improvements. The technical distinction between ``behavior options'' and ``style options'' of older versions is no longer necessary (although still fully supported).

	\item \PGFPlots\ 1.3 has a new feature which allows to \emph{move axis labels tight to tick labels} automatically.

	Since this affects the spacing, it is not enabled be default.

	Use
\begin{codeexample}[code only]
\usepackage{pgfplots}
\pgfplotsset{compat=1.3}
\end{codeexample}
	in your preamble to benefit from the improved spacing\footnote{The |compat=1.3| setting is necessary for new features which move axis labels around like reversed axis. However, \PGFPlots\ will still work as it does in earlier versions, your documents will remain the same if you don't set it explicitly.}.
	Take a look at the next page for the precise definition of |compat|.

	\item \PGFPlots\ 1.3 now supports reversed axes. It is no longer necessary to use work--arounds with negative units.
\pgfkeys{/pdflinks/search key prefixes in/.add={/pgfplots/,}{}}

	Take a look at the |x dir=reverse| key.

	Existing work--arounds will still function properly. Use |\pgfplotsset{compat=1.3}| together with |x dir=reverse| to switch to the new version.
\end{enumerate}

\subsubsection{Old Features Which May Need Attention}
\begin{enumerate}
	\item The |scatter/classes| feature produces proper legends as of version 1.3. This may change the appearance of existing legends of plots with |scatter/classes|.

	\item Due to a small math bug in \PGF\ $2.00$, you \emph{can't} use the math expression `|-x^2|'. It is necessary to use `|0-x^2|' instead. The same holds for
		`|exp(-x^2)|'; use `|exp(0-x^2)|' instead.

		This will be fixed with \PGF\ $>2.00$.

	\item Starting with \PGFPlots\ $1.1$, |\tikzstyle| should \emph{no longer be used} to set \PGFPlots\ options.
	
	Although |\tikzstyle| is still supported for some older \PGFPlots\ options, you should replace any occurance of |\tikzstyle| with |\pgfplotsset{|\meta{style name}|/.style={|\meta{key-value-list}|}}| or the associated |/.append style| variant. See section~\ref{sec:styles} for more detail.
\end{enumerate}
I apologize for any inconvenience caused by these changes.

\begin{pgfplotskey}{compat=\mchoice{1.3,pre 1.3,default,newest} (initially default)}
	The preamble configuration 
\begin{codeexample}[code only]
\usepackage{pgfplots}
\pgfplotsset{compat=1.3}
\end{codeexample}
	allows to choose between backwards compatibility and most recent features.

	\PGFPlots\ version 1.3 comes with new features which may lead to different spacings compared to earlier versions:
	\begin{enumerate}
		\item Version 1.3 comes with the |xlabel near ticks| feature which places (in this case $x$) axis labels ``near'' the tick labels, taking the size of tick labels into account.
		
		Older versions used |xlabel absolute|, i.e.\ an absolute distance between the axis and axis labels (now matter how large or small tick labels are).

		The initial setting \emph{keeps backwards compatibility}.

		You are encouraged to use
\begin{codeexample}[code only]
\usepackage{pgfplots}
\pgfplotsset{compat=1.3}
\end{codeexample}
		in your preamble.
		
		\item Version 1.3 fixes a bug with |anchor=south west| which occurred mainly for reversed axes: the anchors where upside--down. This is fixed now.

		
		If you have written some work--around and want to keep it going, use

		|\pgfplotsset{compat/anchors=pre 1.3}|\pgfmanualpdflabel{/pgfplots/compat/anchors}{} 

		to disable the bug fix.
	\end{enumerate}

	Use |\pgfplotsset{compat=default}| to restore the factory settings.

	The setting |\pgfplotsset{compat=newest}| will always use features of the most recent version. This might result in small changes in the document's appearance.
\end{pgfplotskey}

\subsection{The Team}
\PGFPlots\ has been written mainly by Christian Feuers�nger with many improvements of Pascal Wolkotte and Nick Papior Andersen as a spare time project. We hope it is useful and provides valuable plots.

If you are interested in writing something but don't know how, consider reading the auxiliary manual \href{file:TeX-programming-notes.pdf}{TeX-programming-notes.pdf} which comes with \PGFPlots. It is far from complete, but maybe it is a good starting point (at least for more literature).

\subsection{Acknowledgements}
I thank God for all hours of enjoyed programming. I thank Pascal Wolkotte and Nick Papior Andersen for their programming efforts and contributions as part of the development team. I thank Stefan Tibus, who contributed the |plot shell| feature. I thank Tom Cashman for the contribution of the |reverse legend| feature. Special thanks go to Stefan Pinnow whose tests of \PGFPlots\ lead to numerous quality improvements. Furthermore, I thank Dr.~Schweitzer for many fruitful discussions and Dr.~Meine for his ideas and suggestions. Thanks as well to the many international contributors who provided feature requests or identified bugs or simply manual improvements!

Last but not least, I thank Till Tantau and Mark Wibrow for their excellent graphics (and more) package \PGF\ and \Tikz\ which is the base of \PGFPlots.

\include{pgfplots.install}%
\include{pgfplots.intro}%
\include{pgfplots.reference}%
\include{pgfplots.libs}%
\include{pgfplots.resources}%
\include{pgfplots.importexport}%
\include{pgfplots.basic.reference}%

\printindex

\bibliographystyle{abbrv} %gerapali} %gerabbrv} %gerunsrt.bst} %gerabbrv}% gerplain}
\nocite{pgfplotstable}
\nocite{programmingnotes}
\bibliography{pgfplots}
\end{document}
%
%%%%%%%%%%%%%%%%%%%%%%%%%%%%%%%%%%%%%%%%%%%%%%%%%%%%%%%%%%%%%%%%%%%%%%%%%%%%%
%
% Package pgfplots.sty documentation. 
%
% Copyright 2007/2008 by Christian Feuersaenger.
%
% This program is free software: you can redistribute it and/or modify
% it under the terms of the GNU General Public License as published by
% the Free Software Foundation, either version 3 of the License, or
% (at your option) any later version.
% 
% This program is distributed in the hope that it will be useful,
% but WITHOUT ANY WARRANTY; without even the implied warranty of
% MERCHANTABILITY or FITNESS FOR A PARTICULAR PURPOSE.  See the
% GNU General Public License for more details.
% 
% You should have received a copy of the GNU General Public License
% along with this program.  If not, see <http://www.gnu.org/licenses/>.
%
%
%%%%%%%%%%%%%%%%%%%%%%%%%%%%%%%%%%%%%%%%%%%%%%%%%%%%%%%%%%%%%%%%%%%%%%%%%%%%%
\input{pgfplots.preamble.tex}
%\RequirePackage[german,english,francais]{babel}

\pgfplotsmanualexternalexpensivetrue

%\usetikzlibrary{external}
%\tikzexternalize[prefix=figures/]{pgfplots}

\title{%
	Manual for Package \PGFPlots\\
	{\small Version \pgfplotsversion}\\
	{\small\href{http://sourceforge.net/projects/pgfplots}{http://sourceforge.net/projects/pgfplots}}}

%\includeonly{pgfplots.libs}


\begin{document}

\def\plotcoords{%
\addplot coordinates {
(5,8.312e-02)    (17,2.547e-02)   (49,7.407e-03)
(129,2.102e-03)  (321,5.874e-04)  (769,1.623e-04)
(1793,4.442e-05) (4097,1.207e-05) (9217,3.261e-06)
};

\addplot coordinates{
(7,8.472e-02)    (31,3.044e-02)    (111,1.022e-02)
(351,3.303e-03)  (1023,1.039e-03)  (2815,3.196e-04)
(7423,9.658e-05) (18943,2.873e-05) (47103,8.437e-06)
};

\addplot coordinates{
(9,7.881e-02)     (49,3.243e-02)    (209,1.232e-02)
(769,4.454e-03)   (2561,1.551e-03)  (7937,5.236e-04)
(23297,1.723e-04) (65537,5.545e-05) (178177,1.751e-05)
};

\addplot coordinates{
(11,6.887e-02)    (71,3.177e-02)     (351,1.341e-02)
(1471,5.334e-03)  (5503,2.027e-03)   (18943,7.415e-04)
(61183,2.628e-04) (187903,9.063e-05) (553983,3.053e-05)
};

\addplot coordinates{
(13,5.755e-02)     (97,2.925e-02)     (545,1.351e-02)
(2561,5.842e-03)   (10625,2.397e-03)  (40193,9.414e-04)
(141569,3.564e-04) (471041,1.308e-04) 
(1496065,4.670e-05)
};
}%
\maketitle
\begin{abstract}%
\PGFPlots\ draws high--quality function plots in normal or logarithmic scaling with a user-friendly interface directly in \TeX. The user supplies axis labels, legend entries and the plot coordinates for one or more plots and \PGFPlots\ applies axis scaling, computes any logarithms and axis ticks and draws the plots, supporting line plots, scatter plots, piecewise constant plots, bar plots, area plots, mesh-- and surface plots and some more. It is based on Till Tantau's package \PGF/\Tikz.
\end{abstract}
\tableofcontents
\section{Introduction}
This package provides tools to generate plots and labeled axes easily. It draws normal plots, logplots and semi-logplots, in two and three dimensions. Axis ticks, labels, legends (in case of multiple plots) can be added with key-value options. It can cycle through a set of predefined line/marker/color specifications. In summary, its purpose is to simplify the generation of high-quality function and/or data plots, and solving the problems of
\begin{itemize}
	\item consistency of document and font type and font size,
	\item direct use of \TeX\ math mode in axis descriptions,
	\item consistency of data and figures (no third party tool necessary),
	\item inter document consistency using preamble configurations and styles.
\end{itemize}
Although not necessary, separate |.pdf| or |.eps| graphics can be generated using the |external| library developed as part of \Tikz.

\PGFPlots\ is build completely on \Tikz/\PGF. Knowledge of \Tikz\ will simplify the work with \PGFPlots, although it is not required.

\section[About PGFPlots: Preliminaries]{About {\normalfont\PGFPlots}: Preliminaries}
This section contains information about upgrades, the team, the installation (in case you need to do it manually) and troubleshooting. You may skip it completely except for the upgrade remarks.

However, note that this library requires at least \PGF\ version $2.00$. At the time of this writing, many \TeX-distributions still contain the older \PGF\ version $1.18$, so it may be necessary to install a recent \PGF\ prior to using \PGFPlots.

\subsection{Components}
\PGFPlots\ comes with two components:
\begin{enumerate}
	\item the plotting component (which you are currently reading) and
	\item the \PGFPlotstable\ component which simplifies number formatting and postprocessing of numerical tables. It comes as a separate package and has its own manual \href{file:pgfplotstable.pdf}{pgfplotstable.pdf}.
\end{enumerate}

\subsection{Upgrade remarks}
This release provides a lot of improvements which can be found in all detail in \texttt{ChangeLog} for interested readers. However, some attention is useful with respect to the following changes.

\subsubsection{New Optional Features}
\begin{enumerate}
	\item \PGFPlots\ 1.3 comes with user interface improvements. The technical distinction between ``behavior options'' and ``style options'' of older versions is no longer necessary (although still fully supported).

	\item \PGFPlots\ 1.3 has a new feature which allows to \emph{move axis labels tight to tick labels} automatically.

	Since this affects the spacing, it is not enabled be default.

	Use
\begin{codeexample}[code only]
\usepackage{pgfplots}
\pgfplotsset{compat=1.3}
\end{codeexample}
	in your preamble to benefit from the improved spacing\footnote{The |compat=1.3| setting is necessary for new features which move axis labels around like reversed axis. However, \PGFPlots\ will still work as it does in earlier versions, your documents will remain the same if you don't set it explicitly.}.
	Take a look at the next page for the precise definition of |compat|.

	\item \PGFPlots\ 1.3 now supports reversed axes. It is no longer necessary to use work--arounds with negative units.
\pgfkeys{/pdflinks/search key prefixes in/.add={/pgfplots/,}{}}

	Take a look at the |x dir=reverse| key.

	Existing work--arounds will still function properly. Use |\pgfplotsset{compat=1.3}| together with |x dir=reverse| to switch to the new version.
\end{enumerate}

\subsubsection{Old Features Which May Need Attention}
\begin{enumerate}
	\item The |scatter/classes| feature produces proper legends as of version 1.3. This may change the appearance of existing legends of plots with |scatter/classes|.

	\item Due to a small math bug in \PGF\ $2.00$, you \emph{can't} use the math expression `|-x^2|'. It is necessary to use `|0-x^2|' instead. The same holds for
		`|exp(-x^2)|'; use `|exp(0-x^2)|' instead.

		This will be fixed with \PGF\ $>2.00$.

	\item Starting with \PGFPlots\ $1.1$, |\tikzstyle| should \emph{no longer be used} to set \PGFPlots\ options.
	
	Although |\tikzstyle| is still supported for some older \PGFPlots\ options, you should replace any occurance of |\tikzstyle| with |\pgfplotsset{|\meta{style name}|/.style={|\meta{key-value-list}|}}| or the associated |/.append style| variant. See section~\ref{sec:styles} for more detail.
\end{enumerate}
I apologize for any inconvenience caused by these changes.

\begin{pgfplotskey}{compat=\mchoice{1.3,pre 1.3,default,newest} (initially default)}
	The preamble configuration 
\begin{codeexample}[code only]
\usepackage{pgfplots}
\pgfplotsset{compat=1.3}
\end{codeexample}
	allows to choose between backwards compatibility and most recent features.

	\PGFPlots\ version 1.3 comes with new features which may lead to different spacings compared to earlier versions:
	\begin{enumerate}
		\item Version 1.3 comes with the |xlabel near ticks| feature which places (in this case $x$) axis labels ``near'' the tick labels, taking the size of tick labels into account.
		
		Older versions used |xlabel absolute|, i.e.\ an absolute distance between the axis and axis labels (now matter how large or small tick labels are).

		The initial setting \emph{keeps backwards compatibility}.

		You are encouraged to use
\begin{codeexample}[code only]
\usepackage{pgfplots}
\pgfplotsset{compat=1.3}
\end{codeexample}
		in your preamble.
		
		\item Version 1.3 fixes a bug with |anchor=south west| which occurred mainly for reversed axes: the anchors where upside--down. This is fixed now.

		
		If you have written some work--around and want to keep it going, use

		|\pgfplotsset{compat/anchors=pre 1.3}|\pgfmanualpdflabel{/pgfplots/compat/anchors}{} 

		to disable the bug fix.
	\end{enumerate}

	Use |\pgfplotsset{compat=default}| to restore the factory settings.

	The setting |\pgfplotsset{compat=newest}| will always use features of the most recent version. This might result in small changes in the document's appearance.
\end{pgfplotskey}

\subsection{The Team}
\PGFPlots\ has been written mainly by Christian Feuers�nger with many improvements of Pascal Wolkotte and Nick Papior Andersen as a spare time project. We hope it is useful and provides valuable plots.

If you are interested in writing something but don't know how, consider reading the auxiliary manual \href{file:TeX-programming-notes.pdf}{TeX-programming-notes.pdf} which comes with \PGFPlots. It is far from complete, but maybe it is a good starting point (at least for more literature).

\subsection{Acknowledgements}
I thank God for all hours of enjoyed programming. I thank Pascal Wolkotte and Nick Papior Andersen for their programming efforts and contributions as part of the development team. I thank Stefan Tibus, who contributed the |plot shell| feature. I thank Tom Cashman for the contribution of the |reverse legend| feature. Special thanks go to Stefan Pinnow whose tests of \PGFPlots\ lead to numerous quality improvements. Furthermore, I thank Dr.~Schweitzer for many fruitful discussions and Dr.~Meine for his ideas and suggestions. Thanks as well to the many international contributors who provided feature requests or identified bugs or simply manual improvements!

Last but not least, I thank Till Tantau and Mark Wibrow for their excellent graphics (and more) package \PGF\ and \Tikz\ which is the base of \PGFPlots.

\include{pgfplots.install}%
\include{pgfplots.intro}%
\include{pgfplots.reference}%
\include{pgfplots.libs}%
\include{pgfplots.resources}%
\include{pgfplots.importexport}%
\include{pgfplots.basic.reference}%

\printindex

\bibliographystyle{abbrv} %gerapali} %gerabbrv} %gerunsrt.bst} %gerabbrv}% gerplain}
\nocite{pgfplotstable}
\nocite{programmingnotes}
\bibliography{pgfplots}
\end{document}
%
%%%%%%%%%%%%%%%%%%%%%%%%%%%%%%%%%%%%%%%%%%%%%%%%%%%%%%%%%%%%%%%%%%%%%%%%%%%%%
%
% Package pgfplots.sty documentation. 
%
% Copyright 2007/2008 by Christian Feuersaenger.
%
% This program is free software: you can redistribute it and/or modify
% it under the terms of the GNU General Public License as published by
% the Free Software Foundation, either version 3 of the License, or
% (at your option) any later version.
% 
% This program is distributed in the hope that it will be useful,
% but WITHOUT ANY WARRANTY; without even the implied warranty of
% MERCHANTABILITY or FITNESS FOR A PARTICULAR PURPOSE.  See the
% GNU General Public License for more details.
% 
% You should have received a copy of the GNU General Public License
% along with this program.  If not, see <http://www.gnu.org/licenses/>.
%
%
%%%%%%%%%%%%%%%%%%%%%%%%%%%%%%%%%%%%%%%%%%%%%%%%%%%%%%%%%%%%%%%%%%%%%%%%%%%%%
\input{pgfplots.preamble.tex}
%\RequirePackage[german,english,francais]{babel}

\pgfplotsmanualexternalexpensivetrue

%\usetikzlibrary{external}
%\tikzexternalize[prefix=figures/]{pgfplots}

\title{%
	Manual for Package \PGFPlots\\
	{\small Version \pgfplotsversion}\\
	{\small\href{http://sourceforge.net/projects/pgfplots}{http://sourceforge.net/projects/pgfplots}}}

%\includeonly{pgfplots.libs}


\begin{document}

\def\plotcoords{%
\addplot coordinates {
(5,8.312e-02)    (17,2.547e-02)   (49,7.407e-03)
(129,2.102e-03)  (321,5.874e-04)  (769,1.623e-04)
(1793,4.442e-05) (4097,1.207e-05) (9217,3.261e-06)
};

\addplot coordinates{
(7,8.472e-02)    (31,3.044e-02)    (111,1.022e-02)
(351,3.303e-03)  (1023,1.039e-03)  (2815,3.196e-04)
(7423,9.658e-05) (18943,2.873e-05) (47103,8.437e-06)
};

\addplot coordinates{
(9,7.881e-02)     (49,3.243e-02)    (209,1.232e-02)
(769,4.454e-03)   (2561,1.551e-03)  (7937,5.236e-04)
(23297,1.723e-04) (65537,5.545e-05) (178177,1.751e-05)
};

\addplot coordinates{
(11,6.887e-02)    (71,3.177e-02)     (351,1.341e-02)
(1471,5.334e-03)  (5503,2.027e-03)   (18943,7.415e-04)
(61183,2.628e-04) (187903,9.063e-05) (553983,3.053e-05)
};

\addplot coordinates{
(13,5.755e-02)     (97,2.925e-02)     (545,1.351e-02)
(2561,5.842e-03)   (10625,2.397e-03)  (40193,9.414e-04)
(141569,3.564e-04) (471041,1.308e-04) 
(1496065,4.670e-05)
};
}%
\maketitle
\begin{abstract}%
\PGFPlots\ draws high--quality function plots in normal or logarithmic scaling with a user-friendly interface directly in \TeX. The user supplies axis labels, legend entries and the plot coordinates for one or more plots and \PGFPlots\ applies axis scaling, computes any logarithms and axis ticks and draws the plots, supporting line plots, scatter plots, piecewise constant plots, bar plots, area plots, mesh-- and surface plots and some more. It is based on Till Tantau's package \PGF/\Tikz.
\end{abstract}
\tableofcontents
\section{Introduction}
This package provides tools to generate plots and labeled axes easily. It draws normal plots, logplots and semi-logplots, in two and three dimensions. Axis ticks, labels, legends (in case of multiple plots) can be added with key-value options. It can cycle through a set of predefined line/marker/color specifications. In summary, its purpose is to simplify the generation of high-quality function and/or data plots, and solving the problems of
\begin{itemize}
	\item consistency of document and font type and font size,
	\item direct use of \TeX\ math mode in axis descriptions,
	\item consistency of data and figures (no third party tool necessary),
	\item inter document consistency using preamble configurations and styles.
\end{itemize}
Although not necessary, separate |.pdf| or |.eps| graphics can be generated using the |external| library developed as part of \Tikz.

\PGFPlots\ is build completely on \Tikz/\PGF. Knowledge of \Tikz\ will simplify the work with \PGFPlots, although it is not required.

\section[About PGFPlots: Preliminaries]{About {\normalfont\PGFPlots}: Preliminaries}
This section contains information about upgrades, the team, the installation (in case you need to do it manually) and troubleshooting. You may skip it completely except for the upgrade remarks.

However, note that this library requires at least \PGF\ version $2.00$. At the time of this writing, many \TeX-distributions still contain the older \PGF\ version $1.18$, so it may be necessary to install a recent \PGF\ prior to using \PGFPlots.

\subsection{Components}
\PGFPlots\ comes with two components:
\begin{enumerate}
	\item the plotting component (which you are currently reading) and
	\item the \PGFPlotstable\ component which simplifies number formatting and postprocessing of numerical tables. It comes as a separate package and has its own manual \href{file:pgfplotstable.pdf}{pgfplotstable.pdf}.
\end{enumerate}

\subsection{Upgrade remarks}
This release provides a lot of improvements which can be found in all detail in \texttt{ChangeLog} for interested readers. However, some attention is useful with respect to the following changes.

\subsubsection{New Optional Features}
\begin{enumerate}
	\item \PGFPlots\ 1.3 comes with user interface improvements. The technical distinction between ``behavior options'' and ``style options'' of older versions is no longer necessary (although still fully supported).

	\item \PGFPlots\ 1.3 has a new feature which allows to \emph{move axis labels tight to tick labels} automatically.

	Since this affects the spacing, it is not enabled be default.

	Use
\begin{codeexample}[code only]
\usepackage{pgfplots}
\pgfplotsset{compat=1.3}
\end{codeexample}
	in your preamble to benefit from the improved spacing\footnote{The |compat=1.3| setting is necessary for new features which move axis labels around like reversed axis. However, \PGFPlots\ will still work as it does in earlier versions, your documents will remain the same if you don't set it explicitly.}.
	Take a look at the next page for the precise definition of |compat|.

	\item \PGFPlots\ 1.3 now supports reversed axes. It is no longer necessary to use work--arounds with negative units.
\pgfkeys{/pdflinks/search key prefixes in/.add={/pgfplots/,}{}}

	Take a look at the |x dir=reverse| key.

	Existing work--arounds will still function properly. Use |\pgfplotsset{compat=1.3}| together with |x dir=reverse| to switch to the new version.
\end{enumerate}

\subsubsection{Old Features Which May Need Attention}
\begin{enumerate}
	\item The |scatter/classes| feature produces proper legends as of version 1.3. This may change the appearance of existing legends of plots with |scatter/classes|.

	\item Due to a small math bug in \PGF\ $2.00$, you \emph{can't} use the math expression `|-x^2|'. It is necessary to use `|0-x^2|' instead. The same holds for
		`|exp(-x^2)|'; use `|exp(0-x^2)|' instead.

		This will be fixed with \PGF\ $>2.00$.

	\item Starting with \PGFPlots\ $1.1$, |\tikzstyle| should \emph{no longer be used} to set \PGFPlots\ options.
	
	Although |\tikzstyle| is still supported for some older \PGFPlots\ options, you should replace any occurance of |\tikzstyle| with |\pgfplotsset{|\meta{style name}|/.style={|\meta{key-value-list}|}}| or the associated |/.append style| variant. See section~\ref{sec:styles} for more detail.
\end{enumerate}
I apologize for any inconvenience caused by these changes.

\begin{pgfplotskey}{compat=\mchoice{1.3,pre 1.3,default,newest} (initially default)}
	The preamble configuration 
\begin{codeexample}[code only]
\usepackage{pgfplots}
\pgfplotsset{compat=1.3}
\end{codeexample}
	allows to choose between backwards compatibility and most recent features.

	\PGFPlots\ version 1.3 comes with new features which may lead to different spacings compared to earlier versions:
	\begin{enumerate}
		\item Version 1.3 comes with the |xlabel near ticks| feature which places (in this case $x$) axis labels ``near'' the tick labels, taking the size of tick labels into account.
		
		Older versions used |xlabel absolute|, i.e.\ an absolute distance between the axis and axis labels (now matter how large or small tick labels are).

		The initial setting \emph{keeps backwards compatibility}.

		You are encouraged to use
\begin{codeexample}[code only]
\usepackage{pgfplots}
\pgfplotsset{compat=1.3}
\end{codeexample}
		in your preamble.
		
		\item Version 1.3 fixes a bug with |anchor=south west| which occurred mainly for reversed axes: the anchors where upside--down. This is fixed now.

		
		If you have written some work--around and want to keep it going, use

		|\pgfplotsset{compat/anchors=pre 1.3}|\pgfmanualpdflabel{/pgfplots/compat/anchors}{} 

		to disable the bug fix.
	\end{enumerate}

	Use |\pgfplotsset{compat=default}| to restore the factory settings.

	The setting |\pgfplotsset{compat=newest}| will always use features of the most recent version. This might result in small changes in the document's appearance.
\end{pgfplotskey}

\subsection{The Team}
\PGFPlots\ has been written mainly by Christian Feuers�nger with many improvements of Pascal Wolkotte and Nick Papior Andersen as a spare time project. We hope it is useful and provides valuable plots.

If you are interested in writing something but don't know how, consider reading the auxiliary manual \href{file:TeX-programming-notes.pdf}{TeX-programming-notes.pdf} which comes with \PGFPlots. It is far from complete, but maybe it is a good starting point (at least for more literature).

\subsection{Acknowledgements}
I thank God for all hours of enjoyed programming. I thank Pascal Wolkotte and Nick Papior Andersen for their programming efforts and contributions as part of the development team. I thank Stefan Tibus, who contributed the |plot shell| feature. I thank Tom Cashman for the contribution of the |reverse legend| feature. Special thanks go to Stefan Pinnow whose tests of \PGFPlots\ lead to numerous quality improvements. Furthermore, I thank Dr.~Schweitzer for many fruitful discussions and Dr.~Meine for his ideas and suggestions. Thanks as well to the many international contributors who provided feature requests or identified bugs or simply manual improvements!

Last but not least, I thank Till Tantau and Mark Wibrow for their excellent graphics (and more) package \PGF\ and \Tikz\ which is the base of \PGFPlots.

\include{pgfplots.install}%
\include{pgfplots.intro}%
\include{pgfplots.reference}%
\include{pgfplots.libs}%
\include{pgfplots.resources}%
\include{pgfplots.importexport}%
\include{pgfplots.basic.reference}%

\printindex

\bibliographystyle{abbrv} %gerapali} %gerabbrv} %gerunsrt.bst} %gerabbrv}% gerplain}
\nocite{pgfplotstable}
\nocite{programmingnotes}
\bibliography{pgfplots}
\end{document}
%
\include{pgfplots.basic.reference}%

\printindex

\bibliographystyle{abbrv} %gerapali} %gerabbrv} %gerunsrt.bst} %gerabbrv}% gerplain}
\nocite{pgfplotstable}
\nocite{programmingnotes}
\bibliography{pgfplots}
\end{document}
%
%%%%%%%%%%%%%%%%%%%%%%%%%%%%%%%%%%%%%%%%%%%%%%%%%%%%%%%%%%%%%%%%%%%%%%%%%%%%%
%
% Package pgfplots.sty documentation. 
%
% Copyright 2007/2008 by Christian Feuersaenger.
%
% This program is free software: you can redistribute it and/or modify
% it under the terms of the GNU General Public License as published by
% the Free Software Foundation, either version 3 of the License, or
% (at your option) any later version.
% 
% This program is distributed in the hope that it will be useful,
% but WITHOUT ANY WARRANTY; without even the implied warranty of
% MERCHANTABILITY or FITNESS FOR A PARTICULAR PURPOSE.  See the
% GNU General Public License for more details.
% 
% You should have received a copy of the GNU General Public License
% along with this program.  If not, see <http://www.gnu.org/licenses/>.
%
%
%%%%%%%%%%%%%%%%%%%%%%%%%%%%%%%%%%%%%%%%%%%%%%%%%%%%%%%%%%%%%%%%%%%%%%%%%%%%%
%%%%%%%%%%%%%%%%%%%%%%%%%%%%%%%%%%%%%%%%%%%%%%%%%%%%%%%%%%%%%%%%%%%%%%%%%%%%%
%
% Package pgfplots.sty documentation. 
%
% Copyright 2007/2008 by Christian Feuersaenger.
%
% This program is free software: you can redistribute it and/or modify
% it under the terms of the GNU General Public License as published by
% the Free Software Foundation, either version 3 of the License, or
% (at your option) any later version.
% 
% This program is distributed in the hope that it will be useful,
% but WITHOUT ANY WARRANTY; without even the implied warranty of
% MERCHANTABILITY or FITNESS FOR A PARTICULAR PURPOSE.  See the
% GNU General Public License for more details.
% 
% You should have received a copy of the GNU General Public License
% along with this program.  If not, see <http://www.gnu.org/licenses/>.
%
%
%%%%%%%%%%%%%%%%%%%%%%%%%%%%%%%%%%%%%%%%%%%%%%%%%%%%%%%%%%%%%%%%%%%%%%%%%%%%%
\documentclass[a4paper]{ltxdoc}

\usepackage{makeidx}

% DON't let hyperref overload the format of index and glossary. 
% I want to do that on my own in the stylefiles for makeindex...
\makeatletter
\let\@old@wrindex=\@wrindex
\makeatother

\usepackage{ifpdf}
\ifpdf
	\usepackage[pdftex]{hyperref}
\else
	\usepackage[dvipdfm]{hyperref}
\fi
\hypersetup{pdfborder={0 0 0}}

\makeatletter
\let\@wrindex=\@old@wrindex
\makeatother

% Formatiere Seitennummern im Index:
\newcommand{\indexpageno}[1]{%
	{\bfseries\hyperpage{#1}}%
}


\newcommand{\R}{\mathbb{R}}
\newcommand{\N}{\mathbb{N}}
\newcommand{\Z}{\mathbb{Z}}

\long\def\COMMENTLOWLEVEL#1\ENDCOMMENT{}
\def\ENDCOMMENT{}

\usepackage{textcomp}
\usepackage{booktabs}

\usepackage{calc}
\usepackage[formats]{listings}
%\usepackage{courier} % don't use it - the '^' character can't be copy-pasted in courier

\usepackage{array}
\lstset{%
	basicstyle=\ttfamily,
	language=[LaTeX]tex, % Seems as if \lstset{language=tex} must be invoked BEFORE loading tikz!?
	tabsize=4,
	breaklines=true,
	breakindent=0pt
}

\ifpdf
	\pdfinfo {
		/Author	(Christian Feuersaenger)
	}

\else
	\def\pgfsysdriver{pgfsys-dvipdfm.def}
\fi
%\def\pgfsysdriver{pgfsys-pdftex.def}
\usepackage{pgfplots}

\ifpdf
	\usetikzlibrary{pgfplotsclickable}
\fi

%\usepackage{fp}
% ATTENTION:
% this requires pgf version NEWER than 2.00 :
%\usetikzlibrary{fixedpointarithmetic}

\usetikzlibrary{dateplot}

\usepackage[a4paper,left=2.25cm,right=2.25cm,top=2.5cm,bottom=2.5cm,nohead]{geometry}
\usepackage{amsmath,amssymb}
\usepackage{xxcolor}
\usepackage{pifont}
\usepackage[latin1]{inputenc}
\usepackage{amsmath}
\usepackage{eurosym}
\input{pgfplots-macros}

\pgfqkeys{/codeexample}{%
	every codeexample/.style={
		width=8cm,
		/pgfplots/every axis/.append style={legend style={fill=graphicbackground}}
	},
	tabsize=4,
}
\pgfplotsset{
	every axis/.append style={width=7cm},
	filter discard warning=false,
}

\usetikzlibrary{backgrounds,patterns}
% Global styles:
\tikzset{
  shape example/.style={
    color=black!30,
    draw,
    fill=yellow!30,
    line width=.5cm,
    inner xsep=2.5cm,
    inner ysep=0.5cm}
}

\newcommand{\FIXME}[1]{\textcolor{red}{(FIXME: #1)}}

% fuer endvironment 'sidewaysfigure' bspw
% \usepackage{rotating}

\newcommand\Tikz{Ti\textit kZ}
\newcommand\PGF{\textsc{pgf}}
\newcommand\PGFPlots{\textsc{pgfplots}}
\def\PGFPlotstable{\textsc{PgfplotsTable}}


\makeindex

% Fix overful hboxes automatically:
\tolerance=2000
\emergencystretch=10pt

\tikzset{prefix=gnuplot/pgfplots_} % prefix for 'plot function'

\author{%
	Christian Feuers\"anger\footnote{\url{http://wissrech.ins.uni-bonn.de/people/feuersaenger}}\\%
	Institut f\"ur Numerische Simulation\\
	Universit\"at Bonn, Germany}


%\RequirePackage[german,english,francais]{babel}

\pgfplotsmanualexternalexpensivetrue

%\usetikzlibrary{external}
%\tikzexternalize[prefix=figures/]{pgfplots}

\title{%
	Manual for Package \PGFPlots\\
	{\small Version \pgfplotsversion}\\
	{\small\href{http://sourceforge.net/projects/pgfplots}{http://sourceforge.net/projects/pgfplots}}}

%\includeonly{pgfplots.libs}


\begin{document}

\def\plotcoords{%
\addplot coordinates {
(5,8.312e-02)    (17,2.547e-02)   (49,7.407e-03)
(129,2.102e-03)  (321,5.874e-04)  (769,1.623e-04)
(1793,4.442e-05) (4097,1.207e-05) (9217,3.261e-06)
};

\addplot coordinates{
(7,8.472e-02)    (31,3.044e-02)    (111,1.022e-02)
(351,3.303e-03)  (1023,1.039e-03)  (2815,3.196e-04)
(7423,9.658e-05) (18943,2.873e-05) (47103,8.437e-06)
};

\addplot coordinates{
(9,7.881e-02)     (49,3.243e-02)    (209,1.232e-02)
(769,4.454e-03)   (2561,1.551e-03)  (7937,5.236e-04)
(23297,1.723e-04) (65537,5.545e-05) (178177,1.751e-05)
};

\addplot coordinates{
(11,6.887e-02)    (71,3.177e-02)     (351,1.341e-02)
(1471,5.334e-03)  (5503,2.027e-03)   (18943,7.415e-04)
(61183,2.628e-04) (187903,9.063e-05) (553983,3.053e-05)
};

\addplot coordinates{
(13,5.755e-02)     (97,2.925e-02)     (545,1.351e-02)
(2561,5.842e-03)   (10625,2.397e-03)  (40193,9.414e-04)
(141569,3.564e-04) (471041,1.308e-04) 
(1496065,4.670e-05)
};
}%
\maketitle
\begin{abstract}%
\PGFPlots\ draws high--quality function plots in normal or logarithmic scaling with a user-friendly interface directly in \TeX. The user supplies axis labels, legend entries and the plot coordinates for one or more plots and \PGFPlots\ applies axis scaling, computes any logarithms and axis ticks and draws the plots, supporting line plots, scatter plots, piecewise constant plots, bar plots, area plots, mesh-- and surface plots and some more. It is based on Till Tantau's package \PGF/\Tikz.
\end{abstract}
\tableofcontents
\section{Introduction}
This package provides tools to generate plots and labeled axes easily. It draws normal plots, logplots and semi-logplots, in two and three dimensions. Axis ticks, labels, legends (in case of multiple plots) can be added with key-value options. It can cycle through a set of predefined line/marker/color specifications. In summary, its purpose is to simplify the generation of high-quality function and/or data plots, and solving the problems of
\begin{itemize}
	\item consistency of document and font type and font size,
	\item direct use of \TeX\ math mode in axis descriptions,
	\item consistency of data and figures (no third party tool necessary),
	\item inter document consistency using preamble configurations and styles.
\end{itemize}
Although not necessary, separate |.pdf| or |.eps| graphics can be generated using the |external| library developed as part of \Tikz.

\PGFPlots\ is build completely on \Tikz/\PGF. Knowledge of \Tikz\ will simplify the work with \PGFPlots, although it is not required.

\section[About PGFPlots: Preliminaries]{About {\normalfont\PGFPlots}: Preliminaries}
This section contains information about upgrades, the team, the installation (in case you need to do it manually) and troubleshooting. You may skip it completely except for the upgrade remarks.

However, note that this library requires at least \PGF\ version $2.00$. At the time of this writing, many \TeX-distributions still contain the older \PGF\ version $1.18$, so it may be necessary to install a recent \PGF\ prior to using \PGFPlots.

\subsection{Components}
\PGFPlots\ comes with two components:
\begin{enumerate}
	\item the plotting component (which you are currently reading) and
	\item the \PGFPlotstable\ component which simplifies number formatting and postprocessing of numerical tables. It comes as a separate package and has its own manual \href{file:pgfplotstable.pdf}{pgfplotstable.pdf}.
\end{enumerate}

\subsection{Upgrade remarks}
This release provides a lot of improvements which can be found in all detail in \texttt{ChangeLog} for interested readers. However, some attention is useful with respect to the following changes.

\subsubsection{New Optional Features}
\begin{enumerate}
	\item \PGFPlots\ 1.3 comes with user interface improvements. The technical distinction between ``behavior options'' and ``style options'' of older versions is no longer necessary (although still fully supported).

	\item \PGFPlots\ 1.3 has a new feature which allows to \emph{move axis labels tight to tick labels} automatically.

	Since this affects the spacing, it is not enabled be default.

	Use
\begin{codeexample}[code only]
\usepackage{pgfplots}
\pgfplotsset{compat=1.3}
\end{codeexample}
	in your preamble to benefit from the improved spacing\footnote{The |compat=1.3| setting is necessary for new features which move axis labels around like reversed axis. However, \PGFPlots\ will still work as it does in earlier versions, your documents will remain the same if you don't set it explicitly.}.
	Take a look at the next page for the precise definition of |compat|.

	\item \PGFPlots\ 1.3 now supports reversed axes. It is no longer necessary to use work--arounds with negative units.
\pgfkeys{/pdflinks/search key prefixes in/.add={/pgfplots/,}{}}

	Take a look at the |x dir=reverse| key.

	Existing work--arounds will still function properly. Use |\pgfplotsset{compat=1.3}| together with |x dir=reverse| to switch to the new version.
\end{enumerate}

\subsubsection{Old Features Which May Need Attention}
\begin{enumerate}
	\item The |scatter/classes| feature produces proper legends as of version 1.3. This may change the appearance of existing legends of plots with |scatter/classes|.

	\item Due to a small math bug in \PGF\ $2.00$, you \emph{can't} use the math expression `|-x^2|'. It is necessary to use `|0-x^2|' instead. The same holds for
		`|exp(-x^2)|'; use `|exp(0-x^2)|' instead.

		This will be fixed with \PGF\ $>2.00$.

	\item Starting with \PGFPlots\ $1.1$, |\tikzstyle| should \emph{no longer be used} to set \PGFPlots\ options.
	
	Although |\tikzstyle| is still supported for some older \PGFPlots\ options, you should replace any occurance of |\tikzstyle| with |\pgfplotsset{|\meta{style name}|/.style={|\meta{key-value-list}|}}| or the associated |/.append style| variant. See section~\ref{sec:styles} for more detail.
\end{enumerate}
I apologize for any inconvenience caused by these changes.

\begin{pgfplotskey}{compat=\mchoice{1.3,pre 1.3,default,newest} (initially default)}
	The preamble configuration 
\begin{codeexample}[code only]
\usepackage{pgfplots}
\pgfplotsset{compat=1.3}
\end{codeexample}
	allows to choose between backwards compatibility and most recent features.

	\PGFPlots\ version 1.3 comes with new features which may lead to different spacings compared to earlier versions:
	\begin{enumerate}
		\item Version 1.3 comes with the |xlabel near ticks| feature which places (in this case $x$) axis labels ``near'' the tick labels, taking the size of tick labels into account.
		
		Older versions used |xlabel absolute|, i.e.\ an absolute distance between the axis and axis labels (now matter how large or small tick labels are).

		The initial setting \emph{keeps backwards compatibility}.

		You are encouraged to use
\begin{codeexample}[code only]
\usepackage{pgfplots}
\pgfplotsset{compat=1.3}
\end{codeexample}
		in your preamble.
		
		\item Version 1.3 fixes a bug with |anchor=south west| which occurred mainly for reversed axes: the anchors where upside--down. This is fixed now.

		
		If you have written some work--around and want to keep it going, use

		|\pgfplotsset{compat/anchors=pre 1.3}|\pgfmanualpdflabel{/pgfplots/compat/anchors}{} 

		to disable the bug fix.
	\end{enumerate}

	Use |\pgfplotsset{compat=default}| to restore the factory settings.

	The setting |\pgfplotsset{compat=newest}| will always use features of the most recent version. This might result in small changes in the document's appearance.
\end{pgfplotskey}

\subsection{The Team}
\PGFPlots\ has been written mainly by Christian Feuers�nger with many improvements of Pascal Wolkotte and Nick Papior Andersen as a spare time project. We hope it is useful and provides valuable plots.

If you are interested in writing something but don't know how, consider reading the auxiliary manual \href{file:TeX-programming-notes.pdf}{TeX-programming-notes.pdf} which comes with \PGFPlots. It is far from complete, but maybe it is a good starting point (at least for more literature).

\subsection{Acknowledgements}
I thank God for all hours of enjoyed programming. I thank Pascal Wolkotte and Nick Papior Andersen for their programming efforts and contributions as part of the development team. I thank Stefan Tibus, who contributed the |plot shell| feature. I thank Tom Cashman for the contribution of the |reverse legend| feature. Special thanks go to Stefan Pinnow whose tests of \PGFPlots\ lead to numerous quality improvements. Furthermore, I thank Dr.~Schweitzer for many fruitful discussions and Dr.~Meine for his ideas and suggestions. Thanks as well to the many international contributors who provided feature requests or identified bugs or simply manual improvements!

Last but not least, I thank Till Tantau and Mark Wibrow for their excellent graphics (and more) package \PGF\ and \Tikz\ which is the base of \PGFPlots.

%%%%%%%%%%%%%%%%%%%%%%%%%%%%%%%%%%%%%%%%%%%%%%%%%%%%%%%%%%%%%%%%%%%%%%%%%%%%%
%
% Package pgfplots.sty documentation. 
%
% Copyright 2007/2008 by Christian Feuersaenger.
%
% This program is free software: you can redistribute it and/or modify
% it under the terms of the GNU General Public License as published by
% the Free Software Foundation, either version 3 of the License, or
% (at your option) any later version.
% 
% This program is distributed in the hope that it will be useful,
% but WITHOUT ANY WARRANTY; without even the implied warranty of
% MERCHANTABILITY or FITNESS FOR A PARTICULAR PURPOSE.  See the
% GNU General Public License for more details.
% 
% You should have received a copy of the GNU General Public License
% along with this program.  If not, see <http://www.gnu.org/licenses/>.
%
%
%%%%%%%%%%%%%%%%%%%%%%%%%%%%%%%%%%%%%%%%%%%%%%%%%%%%%%%%%%%%%%%%%%%%%%%%%%%%%
\input{pgfplots.preamble.tex}
%\RequirePackage[german,english,francais]{babel}

\pgfplotsmanualexternalexpensivetrue

%\usetikzlibrary{external}
%\tikzexternalize[prefix=figures/]{pgfplots}

\title{%
	Manual for Package \PGFPlots\\
	{\small Version \pgfplotsversion}\\
	{\small\href{http://sourceforge.net/projects/pgfplots}{http://sourceforge.net/projects/pgfplots}}}

%\includeonly{pgfplots.libs}


\begin{document}

\def\plotcoords{%
\addplot coordinates {
(5,8.312e-02)    (17,2.547e-02)   (49,7.407e-03)
(129,2.102e-03)  (321,5.874e-04)  (769,1.623e-04)
(1793,4.442e-05) (4097,1.207e-05) (9217,3.261e-06)
};

\addplot coordinates{
(7,8.472e-02)    (31,3.044e-02)    (111,1.022e-02)
(351,3.303e-03)  (1023,1.039e-03)  (2815,3.196e-04)
(7423,9.658e-05) (18943,2.873e-05) (47103,8.437e-06)
};

\addplot coordinates{
(9,7.881e-02)     (49,3.243e-02)    (209,1.232e-02)
(769,4.454e-03)   (2561,1.551e-03)  (7937,5.236e-04)
(23297,1.723e-04) (65537,5.545e-05) (178177,1.751e-05)
};

\addplot coordinates{
(11,6.887e-02)    (71,3.177e-02)     (351,1.341e-02)
(1471,5.334e-03)  (5503,2.027e-03)   (18943,7.415e-04)
(61183,2.628e-04) (187903,9.063e-05) (553983,3.053e-05)
};

\addplot coordinates{
(13,5.755e-02)     (97,2.925e-02)     (545,1.351e-02)
(2561,5.842e-03)   (10625,2.397e-03)  (40193,9.414e-04)
(141569,3.564e-04) (471041,1.308e-04) 
(1496065,4.670e-05)
};
}%
\maketitle
\begin{abstract}%
\PGFPlots\ draws high--quality function plots in normal or logarithmic scaling with a user-friendly interface directly in \TeX. The user supplies axis labels, legend entries and the plot coordinates for one or more plots and \PGFPlots\ applies axis scaling, computes any logarithms and axis ticks and draws the plots, supporting line plots, scatter plots, piecewise constant plots, bar plots, area plots, mesh-- and surface plots and some more. It is based on Till Tantau's package \PGF/\Tikz.
\end{abstract}
\tableofcontents
\section{Introduction}
This package provides tools to generate plots and labeled axes easily. It draws normal plots, logplots and semi-logplots, in two and three dimensions. Axis ticks, labels, legends (in case of multiple plots) can be added with key-value options. It can cycle through a set of predefined line/marker/color specifications. In summary, its purpose is to simplify the generation of high-quality function and/or data plots, and solving the problems of
\begin{itemize}
	\item consistency of document and font type and font size,
	\item direct use of \TeX\ math mode in axis descriptions,
	\item consistency of data and figures (no third party tool necessary),
	\item inter document consistency using preamble configurations and styles.
\end{itemize}
Although not necessary, separate |.pdf| or |.eps| graphics can be generated using the |external| library developed as part of \Tikz.

\PGFPlots\ is build completely on \Tikz/\PGF. Knowledge of \Tikz\ will simplify the work with \PGFPlots, although it is not required.

\section[About PGFPlots: Preliminaries]{About {\normalfont\PGFPlots}: Preliminaries}
This section contains information about upgrades, the team, the installation (in case you need to do it manually) and troubleshooting. You may skip it completely except for the upgrade remarks.

However, note that this library requires at least \PGF\ version $2.00$. At the time of this writing, many \TeX-distributions still contain the older \PGF\ version $1.18$, so it may be necessary to install a recent \PGF\ prior to using \PGFPlots.

\subsection{Components}
\PGFPlots\ comes with two components:
\begin{enumerate}
	\item the plotting component (which you are currently reading) and
	\item the \PGFPlotstable\ component which simplifies number formatting and postprocessing of numerical tables. It comes as a separate package and has its own manual \href{file:pgfplotstable.pdf}{pgfplotstable.pdf}.
\end{enumerate}

\subsection{Upgrade remarks}
This release provides a lot of improvements which can be found in all detail in \texttt{ChangeLog} for interested readers. However, some attention is useful with respect to the following changes.

\subsubsection{New Optional Features}
\begin{enumerate}
	\item \PGFPlots\ 1.3 comes with user interface improvements. The technical distinction between ``behavior options'' and ``style options'' of older versions is no longer necessary (although still fully supported).

	\item \PGFPlots\ 1.3 has a new feature which allows to \emph{move axis labels tight to tick labels} automatically.

	Since this affects the spacing, it is not enabled be default.

	Use
\begin{codeexample}[code only]
\usepackage{pgfplots}
\pgfplotsset{compat=1.3}
\end{codeexample}
	in your preamble to benefit from the improved spacing\footnote{The |compat=1.3| setting is necessary for new features which move axis labels around like reversed axis. However, \PGFPlots\ will still work as it does in earlier versions, your documents will remain the same if you don't set it explicitly.}.
	Take a look at the next page for the precise definition of |compat|.

	\item \PGFPlots\ 1.3 now supports reversed axes. It is no longer necessary to use work--arounds with negative units.
\pgfkeys{/pdflinks/search key prefixes in/.add={/pgfplots/,}{}}

	Take a look at the |x dir=reverse| key.

	Existing work--arounds will still function properly. Use |\pgfplotsset{compat=1.3}| together with |x dir=reverse| to switch to the new version.
\end{enumerate}

\subsubsection{Old Features Which May Need Attention}
\begin{enumerate}
	\item The |scatter/classes| feature produces proper legends as of version 1.3. This may change the appearance of existing legends of plots with |scatter/classes|.

	\item Due to a small math bug in \PGF\ $2.00$, you \emph{can't} use the math expression `|-x^2|'. It is necessary to use `|0-x^2|' instead. The same holds for
		`|exp(-x^2)|'; use `|exp(0-x^2)|' instead.

		This will be fixed with \PGF\ $>2.00$.

	\item Starting with \PGFPlots\ $1.1$, |\tikzstyle| should \emph{no longer be used} to set \PGFPlots\ options.
	
	Although |\tikzstyle| is still supported for some older \PGFPlots\ options, you should replace any occurance of |\tikzstyle| with |\pgfplotsset{|\meta{style name}|/.style={|\meta{key-value-list}|}}| or the associated |/.append style| variant. See section~\ref{sec:styles} for more detail.
\end{enumerate}
I apologize for any inconvenience caused by these changes.

\begin{pgfplotskey}{compat=\mchoice{1.3,pre 1.3,default,newest} (initially default)}
	The preamble configuration 
\begin{codeexample}[code only]
\usepackage{pgfplots}
\pgfplotsset{compat=1.3}
\end{codeexample}
	allows to choose between backwards compatibility and most recent features.

	\PGFPlots\ version 1.3 comes with new features which may lead to different spacings compared to earlier versions:
	\begin{enumerate}
		\item Version 1.3 comes with the |xlabel near ticks| feature which places (in this case $x$) axis labels ``near'' the tick labels, taking the size of tick labels into account.
		
		Older versions used |xlabel absolute|, i.e.\ an absolute distance between the axis and axis labels (now matter how large or small tick labels are).

		The initial setting \emph{keeps backwards compatibility}.

		You are encouraged to use
\begin{codeexample}[code only]
\usepackage{pgfplots}
\pgfplotsset{compat=1.3}
\end{codeexample}
		in your preamble.
		
		\item Version 1.3 fixes a bug with |anchor=south west| which occurred mainly for reversed axes: the anchors where upside--down. This is fixed now.

		
		If you have written some work--around and want to keep it going, use

		|\pgfplotsset{compat/anchors=pre 1.3}|\pgfmanualpdflabel{/pgfplots/compat/anchors}{} 

		to disable the bug fix.
	\end{enumerate}

	Use |\pgfplotsset{compat=default}| to restore the factory settings.

	The setting |\pgfplotsset{compat=newest}| will always use features of the most recent version. This might result in small changes in the document's appearance.
\end{pgfplotskey}

\subsection{The Team}
\PGFPlots\ has been written mainly by Christian Feuers�nger with many improvements of Pascal Wolkotte and Nick Papior Andersen as a spare time project. We hope it is useful and provides valuable plots.

If you are interested in writing something but don't know how, consider reading the auxiliary manual \href{file:TeX-programming-notes.pdf}{TeX-programming-notes.pdf} which comes with \PGFPlots. It is far from complete, but maybe it is a good starting point (at least for more literature).

\subsection{Acknowledgements}
I thank God for all hours of enjoyed programming. I thank Pascal Wolkotte and Nick Papior Andersen for their programming efforts and contributions as part of the development team. I thank Stefan Tibus, who contributed the |plot shell| feature. I thank Tom Cashman for the contribution of the |reverse legend| feature. Special thanks go to Stefan Pinnow whose tests of \PGFPlots\ lead to numerous quality improvements. Furthermore, I thank Dr.~Schweitzer for many fruitful discussions and Dr.~Meine for his ideas and suggestions. Thanks as well to the many international contributors who provided feature requests or identified bugs or simply manual improvements!

Last but not least, I thank Till Tantau and Mark Wibrow for their excellent graphics (and more) package \PGF\ and \Tikz\ which is the base of \PGFPlots.

\include{pgfplots.install}%
\include{pgfplots.intro}%
\include{pgfplots.reference}%
\include{pgfplots.libs}%
\include{pgfplots.resources}%
\include{pgfplots.importexport}%
\include{pgfplots.basic.reference}%

\printindex

\bibliographystyle{abbrv} %gerapali} %gerabbrv} %gerunsrt.bst} %gerabbrv}% gerplain}
\nocite{pgfplotstable}
\nocite{programmingnotes}
\bibliography{pgfplots}
\end{document}
%
%%%%%%%%%%%%%%%%%%%%%%%%%%%%%%%%%%%%%%%%%%%%%%%%%%%%%%%%%%%%%%%%%%%%%%%%%%%%%
%
% Package pgfplots.sty documentation. 
%
% Copyright 2007/2008 by Christian Feuersaenger.
%
% This program is free software: you can redistribute it and/or modify
% it under the terms of the GNU General Public License as published by
% the Free Software Foundation, either version 3 of the License, or
% (at your option) any later version.
% 
% This program is distributed in the hope that it will be useful,
% but WITHOUT ANY WARRANTY; without even the implied warranty of
% MERCHANTABILITY or FITNESS FOR A PARTICULAR PURPOSE.  See the
% GNU General Public License for more details.
% 
% You should have received a copy of the GNU General Public License
% along with this program.  If not, see <http://www.gnu.org/licenses/>.
%
%
%%%%%%%%%%%%%%%%%%%%%%%%%%%%%%%%%%%%%%%%%%%%%%%%%%%%%%%%%%%%%%%%%%%%%%%%%%%%%
\input{pgfplots.preamble.tex}
%\RequirePackage[german,english,francais]{babel}

\pgfplotsmanualexternalexpensivetrue

%\usetikzlibrary{external}
%\tikzexternalize[prefix=figures/]{pgfplots}

\title{%
	Manual for Package \PGFPlots\\
	{\small Version \pgfplotsversion}\\
	{\small\href{http://sourceforge.net/projects/pgfplots}{http://sourceforge.net/projects/pgfplots}}}

%\includeonly{pgfplots.libs}


\begin{document}

\def\plotcoords{%
\addplot coordinates {
(5,8.312e-02)    (17,2.547e-02)   (49,7.407e-03)
(129,2.102e-03)  (321,5.874e-04)  (769,1.623e-04)
(1793,4.442e-05) (4097,1.207e-05) (9217,3.261e-06)
};

\addplot coordinates{
(7,8.472e-02)    (31,3.044e-02)    (111,1.022e-02)
(351,3.303e-03)  (1023,1.039e-03)  (2815,3.196e-04)
(7423,9.658e-05) (18943,2.873e-05) (47103,8.437e-06)
};

\addplot coordinates{
(9,7.881e-02)     (49,3.243e-02)    (209,1.232e-02)
(769,4.454e-03)   (2561,1.551e-03)  (7937,5.236e-04)
(23297,1.723e-04) (65537,5.545e-05) (178177,1.751e-05)
};

\addplot coordinates{
(11,6.887e-02)    (71,3.177e-02)     (351,1.341e-02)
(1471,5.334e-03)  (5503,2.027e-03)   (18943,7.415e-04)
(61183,2.628e-04) (187903,9.063e-05) (553983,3.053e-05)
};

\addplot coordinates{
(13,5.755e-02)     (97,2.925e-02)     (545,1.351e-02)
(2561,5.842e-03)   (10625,2.397e-03)  (40193,9.414e-04)
(141569,3.564e-04) (471041,1.308e-04) 
(1496065,4.670e-05)
};
}%
\maketitle
\begin{abstract}%
\PGFPlots\ draws high--quality function plots in normal or logarithmic scaling with a user-friendly interface directly in \TeX. The user supplies axis labels, legend entries and the plot coordinates for one or more plots and \PGFPlots\ applies axis scaling, computes any logarithms and axis ticks and draws the plots, supporting line plots, scatter plots, piecewise constant plots, bar plots, area plots, mesh-- and surface plots and some more. It is based on Till Tantau's package \PGF/\Tikz.
\end{abstract}
\tableofcontents
\section{Introduction}
This package provides tools to generate plots and labeled axes easily. It draws normal plots, logplots and semi-logplots, in two and three dimensions. Axis ticks, labels, legends (in case of multiple plots) can be added with key-value options. It can cycle through a set of predefined line/marker/color specifications. In summary, its purpose is to simplify the generation of high-quality function and/or data plots, and solving the problems of
\begin{itemize}
	\item consistency of document and font type and font size,
	\item direct use of \TeX\ math mode in axis descriptions,
	\item consistency of data and figures (no third party tool necessary),
	\item inter document consistency using preamble configurations and styles.
\end{itemize}
Although not necessary, separate |.pdf| or |.eps| graphics can be generated using the |external| library developed as part of \Tikz.

\PGFPlots\ is build completely on \Tikz/\PGF. Knowledge of \Tikz\ will simplify the work with \PGFPlots, although it is not required.

\section[About PGFPlots: Preliminaries]{About {\normalfont\PGFPlots}: Preliminaries}
This section contains information about upgrades, the team, the installation (in case you need to do it manually) and troubleshooting. You may skip it completely except for the upgrade remarks.

However, note that this library requires at least \PGF\ version $2.00$. At the time of this writing, many \TeX-distributions still contain the older \PGF\ version $1.18$, so it may be necessary to install a recent \PGF\ prior to using \PGFPlots.

\subsection{Components}
\PGFPlots\ comes with two components:
\begin{enumerate}
	\item the plotting component (which you are currently reading) and
	\item the \PGFPlotstable\ component which simplifies number formatting and postprocessing of numerical tables. It comes as a separate package and has its own manual \href{file:pgfplotstable.pdf}{pgfplotstable.pdf}.
\end{enumerate}

\subsection{Upgrade remarks}
This release provides a lot of improvements which can be found in all detail in \texttt{ChangeLog} for interested readers. However, some attention is useful with respect to the following changes.

\subsubsection{New Optional Features}
\begin{enumerate}
	\item \PGFPlots\ 1.3 comes with user interface improvements. The technical distinction between ``behavior options'' and ``style options'' of older versions is no longer necessary (although still fully supported).

	\item \PGFPlots\ 1.3 has a new feature which allows to \emph{move axis labels tight to tick labels} automatically.

	Since this affects the spacing, it is not enabled be default.

	Use
\begin{codeexample}[code only]
\usepackage{pgfplots}
\pgfplotsset{compat=1.3}
\end{codeexample}
	in your preamble to benefit from the improved spacing\footnote{The |compat=1.3| setting is necessary for new features which move axis labels around like reversed axis. However, \PGFPlots\ will still work as it does in earlier versions, your documents will remain the same if you don't set it explicitly.}.
	Take a look at the next page for the precise definition of |compat|.

	\item \PGFPlots\ 1.3 now supports reversed axes. It is no longer necessary to use work--arounds with negative units.
\pgfkeys{/pdflinks/search key prefixes in/.add={/pgfplots/,}{}}

	Take a look at the |x dir=reverse| key.

	Existing work--arounds will still function properly. Use |\pgfplotsset{compat=1.3}| together with |x dir=reverse| to switch to the new version.
\end{enumerate}

\subsubsection{Old Features Which May Need Attention}
\begin{enumerate}
	\item The |scatter/classes| feature produces proper legends as of version 1.3. This may change the appearance of existing legends of plots with |scatter/classes|.

	\item Due to a small math bug in \PGF\ $2.00$, you \emph{can't} use the math expression `|-x^2|'. It is necessary to use `|0-x^2|' instead. The same holds for
		`|exp(-x^2)|'; use `|exp(0-x^2)|' instead.

		This will be fixed with \PGF\ $>2.00$.

	\item Starting with \PGFPlots\ $1.1$, |\tikzstyle| should \emph{no longer be used} to set \PGFPlots\ options.
	
	Although |\tikzstyle| is still supported for some older \PGFPlots\ options, you should replace any occurance of |\tikzstyle| with |\pgfplotsset{|\meta{style name}|/.style={|\meta{key-value-list}|}}| or the associated |/.append style| variant. See section~\ref{sec:styles} for more detail.
\end{enumerate}
I apologize for any inconvenience caused by these changes.

\begin{pgfplotskey}{compat=\mchoice{1.3,pre 1.3,default,newest} (initially default)}
	The preamble configuration 
\begin{codeexample}[code only]
\usepackage{pgfplots}
\pgfplotsset{compat=1.3}
\end{codeexample}
	allows to choose between backwards compatibility and most recent features.

	\PGFPlots\ version 1.3 comes with new features which may lead to different spacings compared to earlier versions:
	\begin{enumerate}
		\item Version 1.3 comes with the |xlabel near ticks| feature which places (in this case $x$) axis labels ``near'' the tick labels, taking the size of tick labels into account.
		
		Older versions used |xlabel absolute|, i.e.\ an absolute distance between the axis and axis labels (now matter how large or small tick labels are).

		The initial setting \emph{keeps backwards compatibility}.

		You are encouraged to use
\begin{codeexample}[code only]
\usepackage{pgfplots}
\pgfplotsset{compat=1.3}
\end{codeexample}
		in your preamble.
		
		\item Version 1.3 fixes a bug with |anchor=south west| which occurred mainly for reversed axes: the anchors where upside--down. This is fixed now.

		
		If you have written some work--around and want to keep it going, use

		|\pgfplotsset{compat/anchors=pre 1.3}|\pgfmanualpdflabel{/pgfplots/compat/anchors}{} 

		to disable the bug fix.
	\end{enumerate}

	Use |\pgfplotsset{compat=default}| to restore the factory settings.

	The setting |\pgfplotsset{compat=newest}| will always use features of the most recent version. This might result in small changes in the document's appearance.
\end{pgfplotskey}

\subsection{The Team}
\PGFPlots\ has been written mainly by Christian Feuers�nger with many improvements of Pascal Wolkotte and Nick Papior Andersen as a spare time project. We hope it is useful and provides valuable plots.

If you are interested in writing something but don't know how, consider reading the auxiliary manual \href{file:TeX-programming-notes.pdf}{TeX-programming-notes.pdf} which comes with \PGFPlots. It is far from complete, but maybe it is a good starting point (at least for more literature).

\subsection{Acknowledgements}
I thank God for all hours of enjoyed programming. I thank Pascal Wolkotte and Nick Papior Andersen for their programming efforts and contributions as part of the development team. I thank Stefan Tibus, who contributed the |plot shell| feature. I thank Tom Cashman for the contribution of the |reverse legend| feature. Special thanks go to Stefan Pinnow whose tests of \PGFPlots\ lead to numerous quality improvements. Furthermore, I thank Dr.~Schweitzer for many fruitful discussions and Dr.~Meine for his ideas and suggestions. Thanks as well to the many international contributors who provided feature requests or identified bugs or simply manual improvements!

Last but not least, I thank Till Tantau and Mark Wibrow for their excellent graphics (and more) package \PGF\ and \Tikz\ which is the base of \PGFPlots.

\include{pgfplots.install}%
\include{pgfplots.intro}%
\include{pgfplots.reference}%
\include{pgfplots.libs}%
\include{pgfplots.resources}%
\include{pgfplots.importexport}%
\include{pgfplots.basic.reference}%

\printindex

\bibliographystyle{abbrv} %gerapali} %gerabbrv} %gerunsrt.bst} %gerabbrv}% gerplain}
\nocite{pgfplotstable}
\nocite{programmingnotes}
\bibliography{pgfplots}
\end{document}
%
%%%%%%%%%%%%%%%%%%%%%%%%%%%%%%%%%%%%%%%%%%%%%%%%%%%%%%%%%%%%%%%%%%%%%%%%%%%%%
%
% Package pgfplots.sty documentation. 
%
% Copyright 2007/2008 by Christian Feuersaenger.
%
% This program is free software: you can redistribute it and/or modify
% it under the terms of the GNU General Public License as published by
% the Free Software Foundation, either version 3 of the License, or
% (at your option) any later version.
% 
% This program is distributed in the hope that it will be useful,
% but WITHOUT ANY WARRANTY; without even the implied warranty of
% MERCHANTABILITY or FITNESS FOR A PARTICULAR PURPOSE.  See the
% GNU General Public License for more details.
% 
% You should have received a copy of the GNU General Public License
% along with this program.  If not, see <http://www.gnu.org/licenses/>.
%
%
%%%%%%%%%%%%%%%%%%%%%%%%%%%%%%%%%%%%%%%%%%%%%%%%%%%%%%%%%%%%%%%%%%%%%%%%%%%%%
\input{pgfplots.preamble.tex}
%\RequirePackage[german,english,francais]{babel}

\pgfplotsmanualexternalexpensivetrue

%\usetikzlibrary{external}
%\tikzexternalize[prefix=figures/]{pgfplots}

\title{%
	Manual for Package \PGFPlots\\
	{\small Version \pgfplotsversion}\\
	{\small\href{http://sourceforge.net/projects/pgfplots}{http://sourceforge.net/projects/pgfplots}}}

%\includeonly{pgfplots.libs}


\begin{document}

\def\plotcoords{%
\addplot coordinates {
(5,8.312e-02)    (17,2.547e-02)   (49,7.407e-03)
(129,2.102e-03)  (321,5.874e-04)  (769,1.623e-04)
(1793,4.442e-05) (4097,1.207e-05) (9217,3.261e-06)
};

\addplot coordinates{
(7,8.472e-02)    (31,3.044e-02)    (111,1.022e-02)
(351,3.303e-03)  (1023,1.039e-03)  (2815,3.196e-04)
(7423,9.658e-05) (18943,2.873e-05) (47103,8.437e-06)
};

\addplot coordinates{
(9,7.881e-02)     (49,3.243e-02)    (209,1.232e-02)
(769,4.454e-03)   (2561,1.551e-03)  (7937,5.236e-04)
(23297,1.723e-04) (65537,5.545e-05) (178177,1.751e-05)
};

\addplot coordinates{
(11,6.887e-02)    (71,3.177e-02)     (351,1.341e-02)
(1471,5.334e-03)  (5503,2.027e-03)   (18943,7.415e-04)
(61183,2.628e-04) (187903,9.063e-05) (553983,3.053e-05)
};

\addplot coordinates{
(13,5.755e-02)     (97,2.925e-02)     (545,1.351e-02)
(2561,5.842e-03)   (10625,2.397e-03)  (40193,9.414e-04)
(141569,3.564e-04) (471041,1.308e-04) 
(1496065,4.670e-05)
};
}%
\maketitle
\begin{abstract}%
\PGFPlots\ draws high--quality function plots in normal or logarithmic scaling with a user-friendly interface directly in \TeX. The user supplies axis labels, legend entries and the plot coordinates for one or more plots and \PGFPlots\ applies axis scaling, computes any logarithms and axis ticks and draws the plots, supporting line plots, scatter plots, piecewise constant plots, bar plots, area plots, mesh-- and surface plots and some more. It is based on Till Tantau's package \PGF/\Tikz.
\end{abstract}
\tableofcontents
\section{Introduction}
This package provides tools to generate plots and labeled axes easily. It draws normal plots, logplots and semi-logplots, in two and three dimensions. Axis ticks, labels, legends (in case of multiple plots) can be added with key-value options. It can cycle through a set of predefined line/marker/color specifications. In summary, its purpose is to simplify the generation of high-quality function and/or data plots, and solving the problems of
\begin{itemize}
	\item consistency of document and font type and font size,
	\item direct use of \TeX\ math mode in axis descriptions,
	\item consistency of data and figures (no third party tool necessary),
	\item inter document consistency using preamble configurations and styles.
\end{itemize}
Although not necessary, separate |.pdf| or |.eps| graphics can be generated using the |external| library developed as part of \Tikz.

\PGFPlots\ is build completely on \Tikz/\PGF. Knowledge of \Tikz\ will simplify the work with \PGFPlots, although it is not required.

\section[About PGFPlots: Preliminaries]{About {\normalfont\PGFPlots}: Preliminaries}
This section contains information about upgrades, the team, the installation (in case you need to do it manually) and troubleshooting. You may skip it completely except for the upgrade remarks.

However, note that this library requires at least \PGF\ version $2.00$. At the time of this writing, many \TeX-distributions still contain the older \PGF\ version $1.18$, so it may be necessary to install a recent \PGF\ prior to using \PGFPlots.

\subsection{Components}
\PGFPlots\ comes with two components:
\begin{enumerate}
	\item the plotting component (which you are currently reading) and
	\item the \PGFPlotstable\ component which simplifies number formatting and postprocessing of numerical tables. It comes as a separate package and has its own manual \href{file:pgfplotstable.pdf}{pgfplotstable.pdf}.
\end{enumerate}

\subsection{Upgrade remarks}
This release provides a lot of improvements which can be found in all detail in \texttt{ChangeLog} for interested readers. However, some attention is useful with respect to the following changes.

\subsubsection{New Optional Features}
\begin{enumerate}
	\item \PGFPlots\ 1.3 comes with user interface improvements. The technical distinction between ``behavior options'' and ``style options'' of older versions is no longer necessary (although still fully supported).

	\item \PGFPlots\ 1.3 has a new feature which allows to \emph{move axis labels tight to tick labels} automatically.

	Since this affects the spacing, it is not enabled be default.

	Use
\begin{codeexample}[code only]
\usepackage{pgfplots}
\pgfplotsset{compat=1.3}
\end{codeexample}
	in your preamble to benefit from the improved spacing\footnote{The |compat=1.3| setting is necessary for new features which move axis labels around like reversed axis. However, \PGFPlots\ will still work as it does in earlier versions, your documents will remain the same if you don't set it explicitly.}.
	Take a look at the next page for the precise definition of |compat|.

	\item \PGFPlots\ 1.3 now supports reversed axes. It is no longer necessary to use work--arounds with negative units.
\pgfkeys{/pdflinks/search key prefixes in/.add={/pgfplots/,}{}}

	Take a look at the |x dir=reverse| key.

	Existing work--arounds will still function properly. Use |\pgfplotsset{compat=1.3}| together with |x dir=reverse| to switch to the new version.
\end{enumerate}

\subsubsection{Old Features Which May Need Attention}
\begin{enumerate}
	\item The |scatter/classes| feature produces proper legends as of version 1.3. This may change the appearance of existing legends of plots with |scatter/classes|.

	\item Due to a small math bug in \PGF\ $2.00$, you \emph{can't} use the math expression `|-x^2|'. It is necessary to use `|0-x^2|' instead. The same holds for
		`|exp(-x^2)|'; use `|exp(0-x^2)|' instead.

		This will be fixed with \PGF\ $>2.00$.

	\item Starting with \PGFPlots\ $1.1$, |\tikzstyle| should \emph{no longer be used} to set \PGFPlots\ options.
	
	Although |\tikzstyle| is still supported for some older \PGFPlots\ options, you should replace any occurance of |\tikzstyle| with |\pgfplotsset{|\meta{style name}|/.style={|\meta{key-value-list}|}}| or the associated |/.append style| variant. See section~\ref{sec:styles} for more detail.
\end{enumerate}
I apologize for any inconvenience caused by these changes.

\begin{pgfplotskey}{compat=\mchoice{1.3,pre 1.3,default,newest} (initially default)}
	The preamble configuration 
\begin{codeexample}[code only]
\usepackage{pgfplots}
\pgfplotsset{compat=1.3}
\end{codeexample}
	allows to choose between backwards compatibility and most recent features.

	\PGFPlots\ version 1.3 comes with new features which may lead to different spacings compared to earlier versions:
	\begin{enumerate}
		\item Version 1.3 comes with the |xlabel near ticks| feature which places (in this case $x$) axis labels ``near'' the tick labels, taking the size of tick labels into account.
		
		Older versions used |xlabel absolute|, i.e.\ an absolute distance between the axis and axis labels (now matter how large or small tick labels are).

		The initial setting \emph{keeps backwards compatibility}.

		You are encouraged to use
\begin{codeexample}[code only]
\usepackage{pgfplots}
\pgfplotsset{compat=1.3}
\end{codeexample}
		in your preamble.
		
		\item Version 1.3 fixes a bug with |anchor=south west| which occurred mainly for reversed axes: the anchors where upside--down. This is fixed now.

		
		If you have written some work--around and want to keep it going, use

		|\pgfplotsset{compat/anchors=pre 1.3}|\pgfmanualpdflabel{/pgfplots/compat/anchors}{} 

		to disable the bug fix.
	\end{enumerate}

	Use |\pgfplotsset{compat=default}| to restore the factory settings.

	The setting |\pgfplotsset{compat=newest}| will always use features of the most recent version. This might result in small changes in the document's appearance.
\end{pgfplotskey}

\subsection{The Team}
\PGFPlots\ has been written mainly by Christian Feuers�nger with many improvements of Pascal Wolkotte and Nick Papior Andersen as a spare time project. We hope it is useful and provides valuable plots.

If you are interested in writing something but don't know how, consider reading the auxiliary manual \href{file:TeX-programming-notes.pdf}{TeX-programming-notes.pdf} which comes with \PGFPlots. It is far from complete, but maybe it is a good starting point (at least for more literature).

\subsection{Acknowledgements}
I thank God for all hours of enjoyed programming. I thank Pascal Wolkotte and Nick Papior Andersen for their programming efforts and contributions as part of the development team. I thank Stefan Tibus, who contributed the |plot shell| feature. I thank Tom Cashman for the contribution of the |reverse legend| feature. Special thanks go to Stefan Pinnow whose tests of \PGFPlots\ lead to numerous quality improvements. Furthermore, I thank Dr.~Schweitzer for many fruitful discussions and Dr.~Meine for his ideas and suggestions. Thanks as well to the many international contributors who provided feature requests or identified bugs or simply manual improvements!

Last but not least, I thank Till Tantau and Mark Wibrow for their excellent graphics (and more) package \PGF\ and \Tikz\ which is the base of \PGFPlots.

\include{pgfplots.install}%
\include{pgfplots.intro}%
\include{pgfplots.reference}%
\include{pgfplots.libs}%
\include{pgfplots.resources}%
\include{pgfplots.importexport}%
\include{pgfplots.basic.reference}%

\printindex

\bibliographystyle{abbrv} %gerapali} %gerabbrv} %gerunsrt.bst} %gerabbrv}% gerplain}
\nocite{pgfplotstable}
\nocite{programmingnotes}
\bibliography{pgfplots}
\end{document}
%
%%%%%%%%%%%%%%%%%%%%%%%%%%%%%%%%%%%%%%%%%%%%%%%%%%%%%%%%%%%%%%%%%%%%%%%%%%%%%
%
% Package pgfplots.sty documentation. 
%
% Copyright 2007/2008 by Christian Feuersaenger.
%
% This program is free software: you can redistribute it and/or modify
% it under the terms of the GNU General Public License as published by
% the Free Software Foundation, either version 3 of the License, or
% (at your option) any later version.
% 
% This program is distributed in the hope that it will be useful,
% but WITHOUT ANY WARRANTY; without even the implied warranty of
% MERCHANTABILITY or FITNESS FOR A PARTICULAR PURPOSE.  See the
% GNU General Public License for more details.
% 
% You should have received a copy of the GNU General Public License
% along with this program.  If not, see <http://www.gnu.org/licenses/>.
%
%
%%%%%%%%%%%%%%%%%%%%%%%%%%%%%%%%%%%%%%%%%%%%%%%%%%%%%%%%%%%%%%%%%%%%%%%%%%%%%
\input{pgfplots.preamble.tex}
%\RequirePackage[german,english,francais]{babel}

\pgfplotsmanualexternalexpensivetrue

%\usetikzlibrary{external}
%\tikzexternalize[prefix=figures/]{pgfplots}

\title{%
	Manual for Package \PGFPlots\\
	{\small Version \pgfplotsversion}\\
	{\small\href{http://sourceforge.net/projects/pgfplots}{http://sourceforge.net/projects/pgfplots}}}

%\includeonly{pgfplots.libs}


\begin{document}

\def\plotcoords{%
\addplot coordinates {
(5,8.312e-02)    (17,2.547e-02)   (49,7.407e-03)
(129,2.102e-03)  (321,5.874e-04)  (769,1.623e-04)
(1793,4.442e-05) (4097,1.207e-05) (9217,3.261e-06)
};

\addplot coordinates{
(7,8.472e-02)    (31,3.044e-02)    (111,1.022e-02)
(351,3.303e-03)  (1023,1.039e-03)  (2815,3.196e-04)
(7423,9.658e-05) (18943,2.873e-05) (47103,8.437e-06)
};

\addplot coordinates{
(9,7.881e-02)     (49,3.243e-02)    (209,1.232e-02)
(769,4.454e-03)   (2561,1.551e-03)  (7937,5.236e-04)
(23297,1.723e-04) (65537,5.545e-05) (178177,1.751e-05)
};

\addplot coordinates{
(11,6.887e-02)    (71,3.177e-02)     (351,1.341e-02)
(1471,5.334e-03)  (5503,2.027e-03)   (18943,7.415e-04)
(61183,2.628e-04) (187903,9.063e-05) (553983,3.053e-05)
};

\addplot coordinates{
(13,5.755e-02)     (97,2.925e-02)     (545,1.351e-02)
(2561,5.842e-03)   (10625,2.397e-03)  (40193,9.414e-04)
(141569,3.564e-04) (471041,1.308e-04) 
(1496065,4.670e-05)
};
}%
\maketitle
\begin{abstract}%
\PGFPlots\ draws high--quality function plots in normal or logarithmic scaling with a user-friendly interface directly in \TeX. The user supplies axis labels, legend entries and the plot coordinates for one or more plots and \PGFPlots\ applies axis scaling, computes any logarithms and axis ticks and draws the plots, supporting line plots, scatter plots, piecewise constant plots, bar plots, area plots, mesh-- and surface plots and some more. It is based on Till Tantau's package \PGF/\Tikz.
\end{abstract}
\tableofcontents
\section{Introduction}
This package provides tools to generate plots and labeled axes easily. It draws normal plots, logplots and semi-logplots, in two and three dimensions. Axis ticks, labels, legends (in case of multiple plots) can be added with key-value options. It can cycle through a set of predefined line/marker/color specifications. In summary, its purpose is to simplify the generation of high-quality function and/or data plots, and solving the problems of
\begin{itemize}
	\item consistency of document and font type and font size,
	\item direct use of \TeX\ math mode in axis descriptions,
	\item consistency of data and figures (no third party tool necessary),
	\item inter document consistency using preamble configurations and styles.
\end{itemize}
Although not necessary, separate |.pdf| or |.eps| graphics can be generated using the |external| library developed as part of \Tikz.

\PGFPlots\ is build completely on \Tikz/\PGF. Knowledge of \Tikz\ will simplify the work with \PGFPlots, although it is not required.

\section[About PGFPlots: Preliminaries]{About {\normalfont\PGFPlots}: Preliminaries}
This section contains information about upgrades, the team, the installation (in case you need to do it manually) and troubleshooting. You may skip it completely except for the upgrade remarks.

However, note that this library requires at least \PGF\ version $2.00$. At the time of this writing, many \TeX-distributions still contain the older \PGF\ version $1.18$, so it may be necessary to install a recent \PGF\ prior to using \PGFPlots.

\subsection{Components}
\PGFPlots\ comes with two components:
\begin{enumerate}
	\item the plotting component (which you are currently reading) and
	\item the \PGFPlotstable\ component which simplifies number formatting and postprocessing of numerical tables. It comes as a separate package and has its own manual \href{file:pgfplotstable.pdf}{pgfplotstable.pdf}.
\end{enumerate}

\subsection{Upgrade remarks}
This release provides a lot of improvements which can be found in all detail in \texttt{ChangeLog} for interested readers. However, some attention is useful with respect to the following changes.

\subsubsection{New Optional Features}
\begin{enumerate}
	\item \PGFPlots\ 1.3 comes with user interface improvements. The technical distinction between ``behavior options'' and ``style options'' of older versions is no longer necessary (although still fully supported).

	\item \PGFPlots\ 1.3 has a new feature which allows to \emph{move axis labels tight to tick labels} automatically.

	Since this affects the spacing, it is not enabled be default.

	Use
\begin{codeexample}[code only]
\usepackage{pgfplots}
\pgfplotsset{compat=1.3}
\end{codeexample}
	in your preamble to benefit from the improved spacing\footnote{The |compat=1.3| setting is necessary for new features which move axis labels around like reversed axis. However, \PGFPlots\ will still work as it does in earlier versions, your documents will remain the same if you don't set it explicitly.}.
	Take a look at the next page for the precise definition of |compat|.

	\item \PGFPlots\ 1.3 now supports reversed axes. It is no longer necessary to use work--arounds with negative units.
\pgfkeys{/pdflinks/search key prefixes in/.add={/pgfplots/,}{}}

	Take a look at the |x dir=reverse| key.

	Existing work--arounds will still function properly. Use |\pgfplotsset{compat=1.3}| together with |x dir=reverse| to switch to the new version.
\end{enumerate}

\subsubsection{Old Features Which May Need Attention}
\begin{enumerate}
	\item The |scatter/classes| feature produces proper legends as of version 1.3. This may change the appearance of existing legends of plots with |scatter/classes|.

	\item Due to a small math bug in \PGF\ $2.00$, you \emph{can't} use the math expression `|-x^2|'. It is necessary to use `|0-x^2|' instead. The same holds for
		`|exp(-x^2)|'; use `|exp(0-x^2)|' instead.

		This will be fixed with \PGF\ $>2.00$.

	\item Starting with \PGFPlots\ $1.1$, |\tikzstyle| should \emph{no longer be used} to set \PGFPlots\ options.
	
	Although |\tikzstyle| is still supported for some older \PGFPlots\ options, you should replace any occurance of |\tikzstyle| with |\pgfplotsset{|\meta{style name}|/.style={|\meta{key-value-list}|}}| or the associated |/.append style| variant. See section~\ref{sec:styles} for more detail.
\end{enumerate}
I apologize for any inconvenience caused by these changes.

\begin{pgfplotskey}{compat=\mchoice{1.3,pre 1.3,default,newest} (initially default)}
	The preamble configuration 
\begin{codeexample}[code only]
\usepackage{pgfplots}
\pgfplotsset{compat=1.3}
\end{codeexample}
	allows to choose between backwards compatibility and most recent features.

	\PGFPlots\ version 1.3 comes with new features which may lead to different spacings compared to earlier versions:
	\begin{enumerate}
		\item Version 1.3 comes with the |xlabel near ticks| feature which places (in this case $x$) axis labels ``near'' the tick labels, taking the size of tick labels into account.
		
		Older versions used |xlabel absolute|, i.e.\ an absolute distance between the axis and axis labels (now matter how large or small tick labels are).

		The initial setting \emph{keeps backwards compatibility}.

		You are encouraged to use
\begin{codeexample}[code only]
\usepackage{pgfplots}
\pgfplotsset{compat=1.3}
\end{codeexample}
		in your preamble.
		
		\item Version 1.3 fixes a bug with |anchor=south west| which occurred mainly for reversed axes: the anchors where upside--down. This is fixed now.

		
		If you have written some work--around and want to keep it going, use

		|\pgfplotsset{compat/anchors=pre 1.3}|\pgfmanualpdflabel{/pgfplots/compat/anchors}{} 

		to disable the bug fix.
	\end{enumerate}

	Use |\pgfplotsset{compat=default}| to restore the factory settings.

	The setting |\pgfplotsset{compat=newest}| will always use features of the most recent version. This might result in small changes in the document's appearance.
\end{pgfplotskey}

\subsection{The Team}
\PGFPlots\ has been written mainly by Christian Feuers�nger with many improvements of Pascal Wolkotte and Nick Papior Andersen as a spare time project. We hope it is useful and provides valuable plots.

If you are interested in writing something but don't know how, consider reading the auxiliary manual \href{file:TeX-programming-notes.pdf}{TeX-programming-notes.pdf} which comes with \PGFPlots. It is far from complete, but maybe it is a good starting point (at least for more literature).

\subsection{Acknowledgements}
I thank God for all hours of enjoyed programming. I thank Pascal Wolkotte and Nick Papior Andersen for their programming efforts and contributions as part of the development team. I thank Stefan Tibus, who contributed the |plot shell| feature. I thank Tom Cashman for the contribution of the |reverse legend| feature. Special thanks go to Stefan Pinnow whose tests of \PGFPlots\ lead to numerous quality improvements. Furthermore, I thank Dr.~Schweitzer for many fruitful discussions and Dr.~Meine for his ideas and suggestions. Thanks as well to the many international contributors who provided feature requests or identified bugs or simply manual improvements!

Last but not least, I thank Till Tantau and Mark Wibrow for their excellent graphics (and more) package \PGF\ and \Tikz\ which is the base of \PGFPlots.

\include{pgfplots.install}%
\include{pgfplots.intro}%
\include{pgfplots.reference}%
\include{pgfplots.libs}%
\include{pgfplots.resources}%
\include{pgfplots.importexport}%
\include{pgfplots.basic.reference}%

\printindex

\bibliographystyle{abbrv} %gerapali} %gerabbrv} %gerunsrt.bst} %gerabbrv}% gerplain}
\nocite{pgfplotstable}
\nocite{programmingnotes}
\bibliography{pgfplots}
\end{document}
%
%%%%%%%%%%%%%%%%%%%%%%%%%%%%%%%%%%%%%%%%%%%%%%%%%%%%%%%%%%%%%%%%%%%%%%%%%%%%%
%
% Package pgfplots.sty documentation. 
%
% Copyright 2007/2008 by Christian Feuersaenger.
%
% This program is free software: you can redistribute it and/or modify
% it under the terms of the GNU General Public License as published by
% the Free Software Foundation, either version 3 of the License, or
% (at your option) any later version.
% 
% This program is distributed in the hope that it will be useful,
% but WITHOUT ANY WARRANTY; without even the implied warranty of
% MERCHANTABILITY or FITNESS FOR A PARTICULAR PURPOSE.  See the
% GNU General Public License for more details.
% 
% You should have received a copy of the GNU General Public License
% along with this program.  If not, see <http://www.gnu.org/licenses/>.
%
%
%%%%%%%%%%%%%%%%%%%%%%%%%%%%%%%%%%%%%%%%%%%%%%%%%%%%%%%%%%%%%%%%%%%%%%%%%%%%%
\input{pgfplots.preamble.tex}
%\RequirePackage[german,english,francais]{babel}

\pgfplotsmanualexternalexpensivetrue

%\usetikzlibrary{external}
%\tikzexternalize[prefix=figures/]{pgfplots}

\title{%
	Manual for Package \PGFPlots\\
	{\small Version \pgfplotsversion}\\
	{\small\href{http://sourceforge.net/projects/pgfplots}{http://sourceforge.net/projects/pgfplots}}}

%\includeonly{pgfplots.libs}


\begin{document}

\def\plotcoords{%
\addplot coordinates {
(5,8.312e-02)    (17,2.547e-02)   (49,7.407e-03)
(129,2.102e-03)  (321,5.874e-04)  (769,1.623e-04)
(1793,4.442e-05) (4097,1.207e-05) (9217,3.261e-06)
};

\addplot coordinates{
(7,8.472e-02)    (31,3.044e-02)    (111,1.022e-02)
(351,3.303e-03)  (1023,1.039e-03)  (2815,3.196e-04)
(7423,9.658e-05) (18943,2.873e-05) (47103,8.437e-06)
};

\addplot coordinates{
(9,7.881e-02)     (49,3.243e-02)    (209,1.232e-02)
(769,4.454e-03)   (2561,1.551e-03)  (7937,5.236e-04)
(23297,1.723e-04) (65537,5.545e-05) (178177,1.751e-05)
};

\addplot coordinates{
(11,6.887e-02)    (71,3.177e-02)     (351,1.341e-02)
(1471,5.334e-03)  (5503,2.027e-03)   (18943,7.415e-04)
(61183,2.628e-04) (187903,9.063e-05) (553983,3.053e-05)
};

\addplot coordinates{
(13,5.755e-02)     (97,2.925e-02)     (545,1.351e-02)
(2561,5.842e-03)   (10625,2.397e-03)  (40193,9.414e-04)
(141569,3.564e-04) (471041,1.308e-04) 
(1496065,4.670e-05)
};
}%
\maketitle
\begin{abstract}%
\PGFPlots\ draws high--quality function plots in normal or logarithmic scaling with a user-friendly interface directly in \TeX. The user supplies axis labels, legend entries and the plot coordinates for one or more plots and \PGFPlots\ applies axis scaling, computes any logarithms and axis ticks and draws the plots, supporting line plots, scatter plots, piecewise constant plots, bar plots, area plots, mesh-- and surface plots and some more. It is based on Till Tantau's package \PGF/\Tikz.
\end{abstract}
\tableofcontents
\section{Introduction}
This package provides tools to generate plots and labeled axes easily. It draws normal plots, logplots and semi-logplots, in two and three dimensions. Axis ticks, labels, legends (in case of multiple plots) can be added with key-value options. It can cycle through a set of predefined line/marker/color specifications. In summary, its purpose is to simplify the generation of high-quality function and/or data plots, and solving the problems of
\begin{itemize}
	\item consistency of document and font type and font size,
	\item direct use of \TeX\ math mode in axis descriptions,
	\item consistency of data and figures (no third party tool necessary),
	\item inter document consistency using preamble configurations and styles.
\end{itemize}
Although not necessary, separate |.pdf| or |.eps| graphics can be generated using the |external| library developed as part of \Tikz.

\PGFPlots\ is build completely on \Tikz/\PGF. Knowledge of \Tikz\ will simplify the work with \PGFPlots, although it is not required.

\section[About PGFPlots: Preliminaries]{About {\normalfont\PGFPlots}: Preliminaries}
This section contains information about upgrades, the team, the installation (in case you need to do it manually) and troubleshooting. You may skip it completely except for the upgrade remarks.

However, note that this library requires at least \PGF\ version $2.00$. At the time of this writing, many \TeX-distributions still contain the older \PGF\ version $1.18$, so it may be necessary to install a recent \PGF\ prior to using \PGFPlots.

\subsection{Components}
\PGFPlots\ comes with two components:
\begin{enumerate}
	\item the plotting component (which you are currently reading) and
	\item the \PGFPlotstable\ component which simplifies number formatting and postprocessing of numerical tables. It comes as a separate package and has its own manual \href{file:pgfplotstable.pdf}{pgfplotstable.pdf}.
\end{enumerate}

\subsection{Upgrade remarks}
This release provides a lot of improvements which can be found in all detail in \texttt{ChangeLog} for interested readers. However, some attention is useful with respect to the following changes.

\subsubsection{New Optional Features}
\begin{enumerate}
	\item \PGFPlots\ 1.3 comes with user interface improvements. The technical distinction between ``behavior options'' and ``style options'' of older versions is no longer necessary (although still fully supported).

	\item \PGFPlots\ 1.3 has a new feature which allows to \emph{move axis labels tight to tick labels} automatically.

	Since this affects the spacing, it is not enabled be default.

	Use
\begin{codeexample}[code only]
\usepackage{pgfplots}
\pgfplotsset{compat=1.3}
\end{codeexample}
	in your preamble to benefit from the improved spacing\footnote{The |compat=1.3| setting is necessary for new features which move axis labels around like reversed axis. However, \PGFPlots\ will still work as it does in earlier versions, your documents will remain the same if you don't set it explicitly.}.
	Take a look at the next page for the precise definition of |compat|.

	\item \PGFPlots\ 1.3 now supports reversed axes. It is no longer necessary to use work--arounds with negative units.
\pgfkeys{/pdflinks/search key prefixes in/.add={/pgfplots/,}{}}

	Take a look at the |x dir=reverse| key.

	Existing work--arounds will still function properly. Use |\pgfplotsset{compat=1.3}| together with |x dir=reverse| to switch to the new version.
\end{enumerate}

\subsubsection{Old Features Which May Need Attention}
\begin{enumerate}
	\item The |scatter/classes| feature produces proper legends as of version 1.3. This may change the appearance of existing legends of plots with |scatter/classes|.

	\item Due to a small math bug in \PGF\ $2.00$, you \emph{can't} use the math expression `|-x^2|'. It is necessary to use `|0-x^2|' instead. The same holds for
		`|exp(-x^2)|'; use `|exp(0-x^2)|' instead.

		This will be fixed with \PGF\ $>2.00$.

	\item Starting with \PGFPlots\ $1.1$, |\tikzstyle| should \emph{no longer be used} to set \PGFPlots\ options.
	
	Although |\tikzstyle| is still supported for some older \PGFPlots\ options, you should replace any occurance of |\tikzstyle| with |\pgfplotsset{|\meta{style name}|/.style={|\meta{key-value-list}|}}| or the associated |/.append style| variant. See section~\ref{sec:styles} for more detail.
\end{enumerate}
I apologize for any inconvenience caused by these changes.

\begin{pgfplotskey}{compat=\mchoice{1.3,pre 1.3,default,newest} (initially default)}
	The preamble configuration 
\begin{codeexample}[code only]
\usepackage{pgfplots}
\pgfplotsset{compat=1.3}
\end{codeexample}
	allows to choose between backwards compatibility and most recent features.

	\PGFPlots\ version 1.3 comes with new features which may lead to different spacings compared to earlier versions:
	\begin{enumerate}
		\item Version 1.3 comes with the |xlabel near ticks| feature which places (in this case $x$) axis labels ``near'' the tick labels, taking the size of tick labels into account.
		
		Older versions used |xlabel absolute|, i.e.\ an absolute distance between the axis and axis labels (now matter how large or small tick labels are).

		The initial setting \emph{keeps backwards compatibility}.

		You are encouraged to use
\begin{codeexample}[code only]
\usepackage{pgfplots}
\pgfplotsset{compat=1.3}
\end{codeexample}
		in your preamble.
		
		\item Version 1.3 fixes a bug with |anchor=south west| which occurred mainly for reversed axes: the anchors where upside--down. This is fixed now.

		
		If you have written some work--around and want to keep it going, use

		|\pgfplotsset{compat/anchors=pre 1.3}|\pgfmanualpdflabel{/pgfplots/compat/anchors}{} 

		to disable the bug fix.
	\end{enumerate}

	Use |\pgfplotsset{compat=default}| to restore the factory settings.

	The setting |\pgfplotsset{compat=newest}| will always use features of the most recent version. This might result in small changes in the document's appearance.
\end{pgfplotskey}

\subsection{The Team}
\PGFPlots\ has been written mainly by Christian Feuers�nger with many improvements of Pascal Wolkotte and Nick Papior Andersen as a spare time project. We hope it is useful and provides valuable plots.

If you are interested in writing something but don't know how, consider reading the auxiliary manual \href{file:TeX-programming-notes.pdf}{TeX-programming-notes.pdf} which comes with \PGFPlots. It is far from complete, but maybe it is a good starting point (at least for more literature).

\subsection{Acknowledgements}
I thank God for all hours of enjoyed programming. I thank Pascal Wolkotte and Nick Papior Andersen for their programming efforts and contributions as part of the development team. I thank Stefan Tibus, who contributed the |plot shell| feature. I thank Tom Cashman for the contribution of the |reverse legend| feature. Special thanks go to Stefan Pinnow whose tests of \PGFPlots\ lead to numerous quality improvements. Furthermore, I thank Dr.~Schweitzer for many fruitful discussions and Dr.~Meine for his ideas and suggestions. Thanks as well to the many international contributors who provided feature requests or identified bugs or simply manual improvements!

Last but not least, I thank Till Tantau and Mark Wibrow for their excellent graphics (and more) package \PGF\ and \Tikz\ which is the base of \PGFPlots.

\include{pgfplots.install}%
\include{pgfplots.intro}%
\include{pgfplots.reference}%
\include{pgfplots.libs}%
\include{pgfplots.resources}%
\include{pgfplots.importexport}%
\include{pgfplots.basic.reference}%

\printindex

\bibliographystyle{abbrv} %gerapali} %gerabbrv} %gerunsrt.bst} %gerabbrv}% gerplain}
\nocite{pgfplotstable}
\nocite{programmingnotes}
\bibliography{pgfplots}
\end{document}
%
%%%%%%%%%%%%%%%%%%%%%%%%%%%%%%%%%%%%%%%%%%%%%%%%%%%%%%%%%%%%%%%%%%%%%%%%%%%%%
%
% Package pgfplots.sty documentation. 
%
% Copyright 2007/2008 by Christian Feuersaenger.
%
% This program is free software: you can redistribute it and/or modify
% it under the terms of the GNU General Public License as published by
% the Free Software Foundation, either version 3 of the License, or
% (at your option) any later version.
% 
% This program is distributed in the hope that it will be useful,
% but WITHOUT ANY WARRANTY; without even the implied warranty of
% MERCHANTABILITY or FITNESS FOR A PARTICULAR PURPOSE.  See the
% GNU General Public License for more details.
% 
% You should have received a copy of the GNU General Public License
% along with this program.  If not, see <http://www.gnu.org/licenses/>.
%
%
%%%%%%%%%%%%%%%%%%%%%%%%%%%%%%%%%%%%%%%%%%%%%%%%%%%%%%%%%%%%%%%%%%%%%%%%%%%%%
\input{pgfplots.preamble.tex}
%\RequirePackage[german,english,francais]{babel}

\pgfplotsmanualexternalexpensivetrue

%\usetikzlibrary{external}
%\tikzexternalize[prefix=figures/]{pgfplots}

\title{%
	Manual for Package \PGFPlots\\
	{\small Version \pgfplotsversion}\\
	{\small\href{http://sourceforge.net/projects/pgfplots}{http://sourceforge.net/projects/pgfplots}}}

%\includeonly{pgfplots.libs}


\begin{document}

\def\plotcoords{%
\addplot coordinates {
(5,8.312e-02)    (17,2.547e-02)   (49,7.407e-03)
(129,2.102e-03)  (321,5.874e-04)  (769,1.623e-04)
(1793,4.442e-05) (4097,1.207e-05) (9217,3.261e-06)
};

\addplot coordinates{
(7,8.472e-02)    (31,3.044e-02)    (111,1.022e-02)
(351,3.303e-03)  (1023,1.039e-03)  (2815,3.196e-04)
(7423,9.658e-05) (18943,2.873e-05) (47103,8.437e-06)
};

\addplot coordinates{
(9,7.881e-02)     (49,3.243e-02)    (209,1.232e-02)
(769,4.454e-03)   (2561,1.551e-03)  (7937,5.236e-04)
(23297,1.723e-04) (65537,5.545e-05) (178177,1.751e-05)
};

\addplot coordinates{
(11,6.887e-02)    (71,3.177e-02)     (351,1.341e-02)
(1471,5.334e-03)  (5503,2.027e-03)   (18943,7.415e-04)
(61183,2.628e-04) (187903,9.063e-05) (553983,3.053e-05)
};

\addplot coordinates{
(13,5.755e-02)     (97,2.925e-02)     (545,1.351e-02)
(2561,5.842e-03)   (10625,2.397e-03)  (40193,9.414e-04)
(141569,3.564e-04) (471041,1.308e-04) 
(1496065,4.670e-05)
};
}%
\maketitle
\begin{abstract}%
\PGFPlots\ draws high--quality function plots in normal or logarithmic scaling with a user-friendly interface directly in \TeX. The user supplies axis labels, legend entries and the plot coordinates for one or more plots and \PGFPlots\ applies axis scaling, computes any logarithms and axis ticks and draws the plots, supporting line plots, scatter plots, piecewise constant plots, bar plots, area plots, mesh-- and surface plots and some more. It is based on Till Tantau's package \PGF/\Tikz.
\end{abstract}
\tableofcontents
\section{Introduction}
This package provides tools to generate plots and labeled axes easily. It draws normal plots, logplots and semi-logplots, in two and three dimensions. Axis ticks, labels, legends (in case of multiple plots) can be added with key-value options. It can cycle through a set of predefined line/marker/color specifications. In summary, its purpose is to simplify the generation of high-quality function and/or data plots, and solving the problems of
\begin{itemize}
	\item consistency of document and font type and font size,
	\item direct use of \TeX\ math mode in axis descriptions,
	\item consistency of data and figures (no third party tool necessary),
	\item inter document consistency using preamble configurations and styles.
\end{itemize}
Although not necessary, separate |.pdf| or |.eps| graphics can be generated using the |external| library developed as part of \Tikz.

\PGFPlots\ is build completely on \Tikz/\PGF. Knowledge of \Tikz\ will simplify the work with \PGFPlots, although it is not required.

\section[About PGFPlots: Preliminaries]{About {\normalfont\PGFPlots}: Preliminaries}
This section contains information about upgrades, the team, the installation (in case you need to do it manually) and troubleshooting. You may skip it completely except for the upgrade remarks.

However, note that this library requires at least \PGF\ version $2.00$. At the time of this writing, many \TeX-distributions still contain the older \PGF\ version $1.18$, so it may be necessary to install a recent \PGF\ prior to using \PGFPlots.

\subsection{Components}
\PGFPlots\ comes with two components:
\begin{enumerate}
	\item the plotting component (which you are currently reading) and
	\item the \PGFPlotstable\ component which simplifies number formatting and postprocessing of numerical tables. It comes as a separate package and has its own manual \href{file:pgfplotstable.pdf}{pgfplotstable.pdf}.
\end{enumerate}

\subsection{Upgrade remarks}
This release provides a lot of improvements which can be found in all detail in \texttt{ChangeLog} for interested readers. However, some attention is useful with respect to the following changes.

\subsubsection{New Optional Features}
\begin{enumerate}
	\item \PGFPlots\ 1.3 comes with user interface improvements. The technical distinction between ``behavior options'' and ``style options'' of older versions is no longer necessary (although still fully supported).

	\item \PGFPlots\ 1.3 has a new feature which allows to \emph{move axis labels tight to tick labels} automatically.

	Since this affects the spacing, it is not enabled be default.

	Use
\begin{codeexample}[code only]
\usepackage{pgfplots}
\pgfplotsset{compat=1.3}
\end{codeexample}
	in your preamble to benefit from the improved spacing\footnote{The |compat=1.3| setting is necessary for new features which move axis labels around like reversed axis. However, \PGFPlots\ will still work as it does in earlier versions, your documents will remain the same if you don't set it explicitly.}.
	Take a look at the next page for the precise definition of |compat|.

	\item \PGFPlots\ 1.3 now supports reversed axes. It is no longer necessary to use work--arounds with negative units.
\pgfkeys{/pdflinks/search key prefixes in/.add={/pgfplots/,}{}}

	Take a look at the |x dir=reverse| key.

	Existing work--arounds will still function properly. Use |\pgfplotsset{compat=1.3}| together with |x dir=reverse| to switch to the new version.
\end{enumerate}

\subsubsection{Old Features Which May Need Attention}
\begin{enumerate}
	\item The |scatter/classes| feature produces proper legends as of version 1.3. This may change the appearance of existing legends of plots with |scatter/classes|.

	\item Due to a small math bug in \PGF\ $2.00$, you \emph{can't} use the math expression `|-x^2|'. It is necessary to use `|0-x^2|' instead. The same holds for
		`|exp(-x^2)|'; use `|exp(0-x^2)|' instead.

		This will be fixed with \PGF\ $>2.00$.

	\item Starting with \PGFPlots\ $1.1$, |\tikzstyle| should \emph{no longer be used} to set \PGFPlots\ options.
	
	Although |\tikzstyle| is still supported for some older \PGFPlots\ options, you should replace any occurance of |\tikzstyle| with |\pgfplotsset{|\meta{style name}|/.style={|\meta{key-value-list}|}}| or the associated |/.append style| variant. See section~\ref{sec:styles} for more detail.
\end{enumerate}
I apologize for any inconvenience caused by these changes.

\begin{pgfplotskey}{compat=\mchoice{1.3,pre 1.3,default,newest} (initially default)}
	The preamble configuration 
\begin{codeexample}[code only]
\usepackage{pgfplots}
\pgfplotsset{compat=1.3}
\end{codeexample}
	allows to choose between backwards compatibility and most recent features.

	\PGFPlots\ version 1.3 comes with new features which may lead to different spacings compared to earlier versions:
	\begin{enumerate}
		\item Version 1.3 comes with the |xlabel near ticks| feature which places (in this case $x$) axis labels ``near'' the tick labels, taking the size of tick labels into account.
		
		Older versions used |xlabel absolute|, i.e.\ an absolute distance between the axis and axis labels (now matter how large or small tick labels are).

		The initial setting \emph{keeps backwards compatibility}.

		You are encouraged to use
\begin{codeexample}[code only]
\usepackage{pgfplots}
\pgfplotsset{compat=1.3}
\end{codeexample}
		in your preamble.
		
		\item Version 1.3 fixes a bug with |anchor=south west| which occurred mainly for reversed axes: the anchors where upside--down. This is fixed now.

		
		If you have written some work--around and want to keep it going, use

		|\pgfplotsset{compat/anchors=pre 1.3}|\pgfmanualpdflabel{/pgfplots/compat/anchors}{} 

		to disable the bug fix.
	\end{enumerate}

	Use |\pgfplotsset{compat=default}| to restore the factory settings.

	The setting |\pgfplotsset{compat=newest}| will always use features of the most recent version. This might result in small changes in the document's appearance.
\end{pgfplotskey}

\subsection{The Team}
\PGFPlots\ has been written mainly by Christian Feuers�nger with many improvements of Pascal Wolkotte and Nick Papior Andersen as a spare time project. We hope it is useful and provides valuable plots.

If you are interested in writing something but don't know how, consider reading the auxiliary manual \href{file:TeX-programming-notes.pdf}{TeX-programming-notes.pdf} which comes with \PGFPlots. It is far from complete, but maybe it is a good starting point (at least for more literature).

\subsection{Acknowledgements}
I thank God for all hours of enjoyed programming. I thank Pascal Wolkotte and Nick Papior Andersen for their programming efforts and contributions as part of the development team. I thank Stefan Tibus, who contributed the |plot shell| feature. I thank Tom Cashman for the contribution of the |reverse legend| feature. Special thanks go to Stefan Pinnow whose tests of \PGFPlots\ lead to numerous quality improvements. Furthermore, I thank Dr.~Schweitzer for many fruitful discussions and Dr.~Meine for his ideas and suggestions. Thanks as well to the many international contributors who provided feature requests or identified bugs or simply manual improvements!

Last but not least, I thank Till Tantau and Mark Wibrow for their excellent graphics (and more) package \PGF\ and \Tikz\ which is the base of \PGFPlots.

\include{pgfplots.install}%
\include{pgfplots.intro}%
\include{pgfplots.reference}%
\include{pgfplots.libs}%
\include{pgfplots.resources}%
\include{pgfplots.importexport}%
\include{pgfplots.basic.reference}%

\printindex

\bibliographystyle{abbrv} %gerapali} %gerabbrv} %gerunsrt.bst} %gerabbrv}% gerplain}
\nocite{pgfplotstable}
\nocite{programmingnotes}
\bibliography{pgfplots}
\end{document}
%
\include{pgfplots.basic.reference}%

\printindex

\bibliographystyle{abbrv} %gerapali} %gerabbrv} %gerunsrt.bst} %gerabbrv}% gerplain}
\nocite{pgfplotstable}
\nocite{programmingnotes}
\bibliography{pgfplots}
\end{document}
%
%%%%%%%%%%%%%%%%%%%%%%%%%%%%%%%%%%%%%%%%%%%%%%%%%%%%%%%%%%%%%%%%%%%%%%%%%%%%%
%
% Package pgfplots.sty documentation. 
%
% Copyright 2007/2008 by Christian Feuersaenger.
%
% This program is free software: you can redistribute it and/or modify
% it under the terms of the GNU General Public License as published by
% the Free Software Foundation, either version 3 of the License, or
% (at your option) any later version.
% 
% This program is distributed in the hope that it will be useful,
% but WITHOUT ANY WARRANTY; without even the implied warranty of
% MERCHANTABILITY or FITNESS FOR A PARTICULAR PURPOSE.  See the
% GNU General Public License for more details.
% 
% You should have received a copy of the GNU General Public License
% along with this program.  If not, see <http://www.gnu.org/licenses/>.
%
%
%%%%%%%%%%%%%%%%%%%%%%%%%%%%%%%%%%%%%%%%%%%%%%%%%%%%%%%%%%%%%%%%%%%%%%%%%%%%%
%%%%%%%%%%%%%%%%%%%%%%%%%%%%%%%%%%%%%%%%%%%%%%%%%%%%%%%%%%%%%%%%%%%%%%%%%%%%%
%
% Package pgfplots.sty documentation. 
%
% Copyright 2007/2008 by Christian Feuersaenger.
%
% This program is free software: you can redistribute it and/or modify
% it under the terms of the GNU General Public License as published by
% the Free Software Foundation, either version 3 of the License, or
% (at your option) any later version.
% 
% This program is distributed in the hope that it will be useful,
% but WITHOUT ANY WARRANTY; without even the implied warranty of
% MERCHANTABILITY or FITNESS FOR A PARTICULAR PURPOSE.  See the
% GNU General Public License for more details.
% 
% You should have received a copy of the GNU General Public License
% along with this program.  If not, see <http://www.gnu.org/licenses/>.
%
%
%%%%%%%%%%%%%%%%%%%%%%%%%%%%%%%%%%%%%%%%%%%%%%%%%%%%%%%%%%%%%%%%%%%%%%%%%%%%%
\documentclass[a4paper]{ltxdoc}

\usepackage{makeidx}

% DON't let hyperref overload the format of index and glossary. 
% I want to do that on my own in the stylefiles for makeindex...
\makeatletter
\let\@old@wrindex=\@wrindex
\makeatother

\usepackage{ifpdf}
\ifpdf
	\usepackage[pdftex]{hyperref}
\else
	\usepackage[dvipdfm]{hyperref}
\fi
\hypersetup{pdfborder={0 0 0}}

\makeatletter
\let\@wrindex=\@old@wrindex
\makeatother

% Formatiere Seitennummern im Index:
\newcommand{\indexpageno}[1]{%
	{\bfseries\hyperpage{#1}}%
}


\newcommand{\R}{\mathbb{R}}
\newcommand{\N}{\mathbb{N}}
\newcommand{\Z}{\mathbb{Z}}

\long\def\COMMENTLOWLEVEL#1\ENDCOMMENT{}
\def\ENDCOMMENT{}

\usepackage{textcomp}
\usepackage{booktabs}

\usepackage{calc}
\usepackage[formats]{listings}
%\usepackage{courier} % don't use it - the '^' character can't be copy-pasted in courier

\usepackage{array}
\lstset{%
	basicstyle=\ttfamily,
	language=[LaTeX]tex, % Seems as if \lstset{language=tex} must be invoked BEFORE loading tikz!?
	tabsize=4,
	breaklines=true,
	breakindent=0pt
}

\ifpdf
	\pdfinfo {
		/Author	(Christian Feuersaenger)
	}

\else
	\def\pgfsysdriver{pgfsys-dvipdfm.def}
\fi
%\def\pgfsysdriver{pgfsys-pdftex.def}
\usepackage{pgfplots}

\ifpdf
	\usetikzlibrary{pgfplotsclickable}
\fi

%\usepackage{fp}
% ATTENTION:
% this requires pgf version NEWER than 2.00 :
%\usetikzlibrary{fixedpointarithmetic}

\usetikzlibrary{dateplot}

\usepackage[a4paper,left=2.25cm,right=2.25cm,top=2.5cm,bottom=2.5cm,nohead]{geometry}
\usepackage{amsmath,amssymb}
\usepackage{xxcolor}
\usepackage{pifont}
\usepackage[latin1]{inputenc}
\usepackage{amsmath}
\usepackage{eurosym}
\input{pgfplots-macros}

\pgfqkeys{/codeexample}{%
	every codeexample/.style={
		width=8cm,
		/pgfplots/every axis/.append style={legend style={fill=graphicbackground}}
	},
	tabsize=4,
}
\pgfplotsset{
	every axis/.append style={width=7cm},
	filter discard warning=false,
}

\usetikzlibrary{backgrounds,patterns}
% Global styles:
\tikzset{
  shape example/.style={
    color=black!30,
    draw,
    fill=yellow!30,
    line width=.5cm,
    inner xsep=2.5cm,
    inner ysep=0.5cm}
}

\newcommand{\FIXME}[1]{\textcolor{red}{(FIXME: #1)}}

% fuer endvironment 'sidewaysfigure' bspw
% \usepackage{rotating}

\newcommand\Tikz{Ti\textit kZ}
\newcommand\PGF{\textsc{pgf}}
\newcommand\PGFPlots{\textsc{pgfplots}}
\def\PGFPlotstable{\textsc{PgfplotsTable}}


\makeindex

% Fix overful hboxes automatically:
\tolerance=2000
\emergencystretch=10pt

\tikzset{prefix=gnuplot/pgfplots_} % prefix for 'plot function'

\author{%
	Christian Feuers\"anger\footnote{\url{http://wissrech.ins.uni-bonn.de/people/feuersaenger}}\\%
	Institut f\"ur Numerische Simulation\\
	Universit\"at Bonn, Germany}


%\RequirePackage[german,english,francais]{babel}

\pgfplotsmanualexternalexpensivetrue

%\usetikzlibrary{external}
%\tikzexternalize[prefix=figures/]{pgfplots}

\title{%
	Manual for Package \PGFPlots\\
	{\small Version \pgfplotsversion}\\
	{\small\href{http://sourceforge.net/projects/pgfplots}{http://sourceforge.net/projects/pgfplots}}}

%\includeonly{pgfplots.libs}


\begin{document}

\def\plotcoords{%
\addplot coordinates {
(5,8.312e-02)    (17,2.547e-02)   (49,7.407e-03)
(129,2.102e-03)  (321,5.874e-04)  (769,1.623e-04)
(1793,4.442e-05) (4097,1.207e-05) (9217,3.261e-06)
};

\addplot coordinates{
(7,8.472e-02)    (31,3.044e-02)    (111,1.022e-02)
(351,3.303e-03)  (1023,1.039e-03)  (2815,3.196e-04)
(7423,9.658e-05) (18943,2.873e-05) (47103,8.437e-06)
};

\addplot coordinates{
(9,7.881e-02)     (49,3.243e-02)    (209,1.232e-02)
(769,4.454e-03)   (2561,1.551e-03)  (7937,5.236e-04)
(23297,1.723e-04) (65537,5.545e-05) (178177,1.751e-05)
};

\addplot coordinates{
(11,6.887e-02)    (71,3.177e-02)     (351,1.341e-02)
(1471,5.334e-03)  (5503,2.027e-03)   (18943,7.415e-04)
(61183,2.628e-04) (187903,9.063e-05) (553983,3.053e-05)
};

\addplot coordinates{
(13,5.755e-02)     (97,2.925e-02)     (545,1.351e-02)
(2561,5.842e-03)   (10625,2.397e-03)  (40193,9.414e-04)
(141569,3.564e-04) (471041,1.308e-04) 
(1496065,4.670e-05)
};
}%
\maketitle
\begin{abstract}%
\PGFPlots\ draws high--quality function plots in normal or logarithmic scaling with a user-friendly interface directly in \TeX. The user supplies axis labels, legend entries and the plot coordinates for one or more plots and \PGFPlots\ applies axis scaling, computes any logarithms and axis ticks and draws the plots, supporting line plots, scatter plots, piecewise constant plots, bar plots, area plots, mesh-- and surface plots and some more. It is based on Till Tantau's package \PGF/\Tikz.
\end{abstract}
\tableofcontents
\section{Introduction}
This package provides tools to generate plots and labeled axes easily. It draws normal plots, logplots and semi-logplots, in two and three dimensions. Axis ticks, labels, legends (in case of multiple plots) can be added with key-value options. It can cycle through a set of predefined line/marker/color specifications. In summary, its purpose is to simplify the generation of high-quality function and/or data plots, and solving the problems of
\begin{itemize}
	\item consistency of document and font type and font size,
	\item direct use of \TeX\ math mode in axis descriptions,
	\item consistency of data and figures (no third party tool necessary),
	\item inter document consistency using preamble configurations and styles.
\end{itemize}
Although not necessary, separate |.pdf| or |.eps| graphics can be generated using the |external| library developed as part of \Tikz.

\PGFPlots\ is build completely on \Tikz/\PGF. Knowledge of \Tikz\ will simplify the work with \PGFPlots, although it is not required.

\section[About PGFPlots: Preliminaries]{About {\normalfont\PGFPlots}: Preliminaries}
This section contains information about upgrades, the team, the installation (in case you need to do it manually) and troubleshooting. You may skip it completely except for the upgrade remarks.

However, note that this library requires at least \PGF\ version $2.00$. At the time of this writing, many \TeX-distributions still contain the older \PGF\ version $1.18$, so it may be necessary to install a recent \PGF\ prior to using \PGFPlots.

\subsection{Components}
\PGFPlots\ comes with two components:
\begin{enumerate}
	\item the plotting component (which you are currently reading) and
	\item the \PGFPlotstable\ component which simplifies number formatting and postprocessing of numerical tables. It comes as a separate package and has its own manual \href{file:pgfplotstable.pdf}{pgfplotstable.pdf}.
\end{enumerate}

\subsection{Upgrade remarks}
This release provides a lot of improvements which can be found in all detail in \texttt{ChangeLog} for interested readers. However, some attention is useful with respect to the following changes.

\subsubsection{New Optional Features}
\begin{enumerate}
	\item \PGFPlots\ 1.3 comes with user interface improvements. The technical distinction between ``behavior options'' and ``style options'' of older versions is no longer necessary (although still fully supported).

	\item \PGFPlots\ 1.3 has a new feature which allows to \emph{move axis labels tight to tick labels} automatically.

	Since this affects the spacing, it is not enabled be default.

	Use
\begin{codeexample}[code only]
\usepackage{pgfplots}
\pgfplotsset{compat=1.3}
\end{codeexample}
	in your preamble to benefit from the improved spacing\footnote{The |compat=1.3| setting is necessary for new features which move axis labels around like reversed axis. However, \PGFPlots\ will still work as it does in earlier versions, your documents will remain the same if you don't set it explicitly.}.
	Take a look at the next page for the precise definition of |compat|.

	\item \PGFPlots\ 1.3 now supports reversed axes. It is no longer necessary to use work--arounds with negative units.
\pgfkeys{/pdflinks/search key prefixes in/.add={/pgfplots/,}{}}

	Take a look at the |x dir=reverse| key.

	Existing work--arounds will still function properly. Use |\pgfplotsset{compat=1.3}| together with |x dir=reverse| to switch to the new version.
\end{enumerate}

\subsubsection{Old Features Which May Need Attention}
\begin{enumerate}
	\item The |scatter/classes| feature produces proper legends as of version 1.3. This may change the appearance of existing legends of plots with |scatter/classes|.

	\item Due to a small math bug in \PGF\ $2.00$, you \emph{can't} use the math expression `|-x^2|'. It is necessary to use `|0-x^2|' instead. The same holds for
		`|exp(-x^2)|'; use `|exp(0-x^2)|' instead.

		This will be fixed with \PGF\ $>2.00$.

	\item Starting with \PGFPlots\ $1.1$, |\tikzstyle| should \emph{no longer be used} to set \PGFPlots\ options.
	
	Although |\tikzstyle| is still supported for some older \PGFPlots\ options, you should replace any occurance of |\tikzstyle| with |\pgfplotsset{|\meta{style name}|/.style={|\meta{key-value-list}|}}| or the associated |/.append style| variant. See section~\ref{sec:styles} for more detail.
\end{enumerate}
I apologize for any inconvenience caused by these changes.

\begin{pgfplotskey}{compat=\mchoice{1.3,pre 1.3,default,newest} (initially default)}
	The preamble configuration 
\begin{codeexample}[code only]
\usepackage{pgfplots}
\pgfplotsset{compat=1.3}
\end{codeexample}
	allows to choose between backwards compatibility and most recent features.

	\PGFPlots\ version 1.3 comes with new features which may lead to different spacings compared to earlier versions:
	\begin{enumerate}
		\item Version 1.3 comes with the |xlabel near ticks| feature which places (in this case $x$) axis labels ``near'' the tick labels, taking the size of tick labels into account.
		
		Older versions used |xlabel absolute|, i.e.\ an absolute distance between the axis and axis labels (now matter how large or small tick labels are).

		The initial setting \emph{keeps backwards compatibility}.

		You are encouraged to use
\begin{codeexample}[code only]
\usepackage{pgfplots}
\pgfplotsset{compat=1.3}
\end{codeexample}
		in your preamble.
		
		\item Version 1.3 fixes a bug with |anchor=south west| which occurred mainly for reversed axes: the anchors where upside--down. This is fixed now.

		
		If you have written some work--around and want to keep it going, use

		|\pgfplotsset{compat/anchors=pre 1.3}|\pgfmanualpdflabel{/pgfplots/compat/anchors}{} 

		to disable the bug fix.
	\end{enumerate}

	Use |\pgfplotsset{compat=default}| to restore the factory settings.

	The setting |\pgfplotsset{compat=newest}| will always use features of the most recent version. This might result in small changes in the document's appearance.
\end{pgfplotskey}

\subsection{The Team}
\PGFPlots\ has been written mainly by Christian Feuers�nger with many improvements of Pascal Wolkotte and Nick Papior Andersen as a spare time project. We hope it is useful and provides valuable plots.

If you are interested in writing something but don't know how, consider reading the auxiliary manual \href{file:TeX-programming-notes.pdf}{TeX-programming-notes.pdf} which comes with \PGFPlots. It is far from complete, but maybe it is a good starting point (at least for more literature).

\subsection{Acknowledgements}
I thank God for all hours of enjoyed programming. I thank Pascal Wolkotte and Nick Papior Andersen for their programming efforts and contributions as part of the development team. I thank Stefan Tibus, who contributed the |plot shell| feature. I thank Tom Cashman for the contribution of the |reverse legend| feature. Special thanks go to Stefan Pinnow whose tests of \PGFPlots\ lead to numerous quality improvements. Furthermore, I thank Dr.~Schweitzer for many fruitful discussions and Dr.~Meine for his ideas and suggestions. Thanks as well to the many international contributors who provided feature requests or identified bugs or simply manual improvements!

Last but not least, I thank Till Tantau and Mark Wibrow for their excellent graphics (and more) package \PGF\ and \Tikz\ which is the base of \PGFPlots.

%%%%%%%%%%%%%%%%%%%%%%%%%%%%%%%%%%%%%%%%%%%%%%%%%%%%%%%%%%%%%%%%%%%%%%%%%%%%%
%
% Package pgfplots.sty documentation. 
%
% Copyright 2007/2008 by Christian Feuersaenger.
%
% This program is free software: you can redistribute it and/or modify
% it under the terms of the GNU General Public License as published by
% the Free Software Foundation, either version 3 of the License, or
% (at your option) any later version.
% 
% This program is distributed in the hope that it will be useful,
% but WITHOUT ANY WARRANTY; without even the implied warranty of
% MERCHANTABILITY or FITNESS FOR A PARTICULAR PURPOSE.  See the
% GNU General Public License for more details.
% 
% You should have received a copy of the GNU General Public License
% along with this program.  If not, see <http://www.gnu.org/licenses/>.
%
%
%%%%%%%%%%%%%%%%%%%%%%%%%%%%%%%%%%%%%%%%%%%%%%%%%%%%%%%%%%%%%%%%%%%%%%%%%%%%%
\input{pgfplots.preamble.tex}
%\RequirePackage[german,english,francais]{babel}

\pgfplotsmanualexternalexpensivetrue

%\usetikzlibrary{external}
%\tikzexternalize[prefix=figures/]{pgfplots}

\title{%
	Manual for Package \PGFPlots\\
	{\small Version \pgfplotsversion}\\
	{\small\href{http://sourceforge.net/projects/pgfplots}{http://sourceforge.net/projects/pgfplots}}}

%\includeonly{pgfplots.libs}


\begin{document}

\def\plotcoords{%
\addplot coordinates {
(5,8.312e-02)    (17,2.547e-02)   (49,7.407e-03)
(129,2.102e-03)  (321,5.874e-04)  (769,1.623e-04)
(1793,4.442e-05) (4097,1.207e-05) (9217,3.261e-06)
};

\addplot coordinates{
(7,8.472e-02)    (31,3.044e-02)    (111,1.022e-02)
(351,3.303e-03)  (1023,1.039e-03)  (2815,3.196e-04)
(7423,9.658e-05) (18943,2.873e-05) (47103,8.437e-06)
};

\addplot coordinates{
(9,7.881e-02)     (49,3.243e-02)    (209,1.232e-02)
(769,4.454e-03)   (2561,1.551e-03)  (7937,5.236e-04)
(23297,1.723e-04) (65537,5.545e-05) (178177,1.751e-05)
};

\addplot coordinates{
(11,6.887e-02)    (71,3.177e-02)     (351,1.341e-02)
(1471,5.334e-03)  (5503,2.027e-03)   (18943,7.415e-04)
(61183,2.628e-04) (187903,9.063e-05) (553983,3.053e-05)
};

\addplot coordinates{
(13,5.755e-02)     (97,2.925e-02)     (545,1.351e-02)
(2561,5.842e-03)   (10625,2.397e-03)  (40193,9.414e-04)
(141569,3.564e-04) (471041,1.308e-04) 
(1496065,4.670e-05)
};
}%
\maketitle
\begin{abstract}%
\PGFPlots\ draws high--quality function plots in normal or logarithmic scaling with a user-friendly interface directly in \TeX. The user supplies axis labels, legend entries and the plot coordinates for one or more plots and \PGFPlots\ applies axis scaling, computes any logarithms and axis ticks and draws the plots, supporting line plots, scatter plots, piecewise constant plots, bar plots, area plots, mesh-- and surface plots and some more. It is based on Till Tantau's package \PGF/\Tikz.
\end{abstract}
\tableofcontents
\section{Introduction}
This package provides tools to generate plots and labeled axes easily. It draws normal plots, logplots and semi-logplots, in two and three dimensions. Axis ticks, labels, legends (in case of multiple plots) can be added with key-value options. It can cycle through a set of predefined line/marker/color specifications. In summary, its purpose is to simplify the generation of high-quality function and/or data plots, and solving the problems of
\begin{itemize}
	\item consistency of document and font type and font size,
	\item direct use of \TeX\ math mode in axis descriptions,
	\item consistency of data and figures (no third party tool necessary),
	\item inter document consistency using preamble configurations and styles.
\end{itemize}
Although not necessary, separate |.pdf| or |.eps| graphics can be generated using the |external| library developed as part of \Tikz.

\PGFPlots\ is build completely on \Tikz/\PGF. Knowledge of \Tikz\ will simplify the work with \PGFPlots, although it is not required.

\section[About PGFPlots: Preliminaries]{About {\normalfont\PGFPlots}: Preliminaries}
This section contains information about upgrades, the team, the installation (in case you need to do it manually) and troubleshooting. You may skip it completely except for the upgrade remarks.

However, note that this library requires at least \PGF\ version $2.00$. At the time of this writing, many \TeX-distributions still contain the older \PGF\ version $1.18$, so it may be necessary to install a recent \PGF\ prior to using \PGFPlots.

\subsection{Components}
\PGFPlots\ comes with two components:
\begin{enumerate}
	\item the plotting component (which you are currently reading) and
	\item the \PGFPlotstable\ component which simplifies number formatting and postprocessing of numerical tables. It comes as a separate package and has its own manual \href{file:pgfplotstable.pdf}{pgfplotstable.pdf}.
\end{enumerate}

\subsection{Upgrade remarks}
This release provides a lot of improvements which can be found in all detail in \texttt{ChangeLog} for interested readers. However, some attention is useful with respect to the following changes.

\subsubsection{New Optional Features}
\begin{enumerate}
	\item \PGFPlots\ 1.3 comes with user interface improvements. The technical distinction between ``behavior options'' and ``style options'' of older versions is no longer necessary (although still fully supported).

	\item \PGFPlots\ 1.3 has a new feature which allows to \emph{move axis labels tight to tick labels} automatically.

	Since this affects the spacing, it is not enabled be default.

	Use
\begin{codeexample}[code only]
\usepackage{pgfplots}
\pgfplotsset{compat=1.3}
\end{codeexample}
	in your preamble to benefit from the improved spacing\footnote{The |compat=1.3| setting is necessary for new features which move axis labels around like reversed axis. However, \PGFPlots\ will still work as it does in earlier versions, your documents will remain the same if you don't set it explicitly.}.
	Take a look at the next page for the precise definition of |compat|.

	\item \PGFPlots\ 1.3 now supports reversed axes. It is no longer necessary to use work--arounds with negative units.
\pgfkeys{/pdflinks/search key prefixes in/.add={/pgfplots/,}{}}

	Take a look at the |x dir=reverse| key.

	Existing work--arounds will still function properly. Use |\pgfplotsset{compat=1.3}| together with |x dir=reverse| to switch to the new version.
\end{enumerate}

\subsubsection{Old Features Which May Need Attention}
\begin{enumerate}
	\item The |scatter/classes| feature produces proper legends as of version 1.3. This may change the appearance of existing legends of plots with |scatter/classes|.

	\item Due to a small math bug in \PGF\ $2.00$, you \emph{can't} use the math expression `|-x^2|'. It is necessary to use `|0-x^2|' instead. The same holds for
		`|exp(-x^2)|'; use `|exp(0-x^2)|' instead.

		This will be fixed with \PGF\ $>2.00$.

	\item Starting with \PGFPlots\ $1.1$, |\tikzstyle| should \emph{no longer be used} to set \PGFPlots\ options.
	
	Although |\tikzstyle| is still supported for some older \PGFPlots\ options, you should replace any occurance of |\tikzstyle| with |\pgfplotsset{|\meta{style name}|/.style={|\meta{key-value-list}|}}| or the associated |/.append style| variant. See section~\ref{sec:styles} for more detail.
\end{enumerate}
I apologize for any inconvenience caused by these changes.

\begin{pgfplotskey}{compat=\mchoice{1.3,pre 1.3,default,newest} (initially default)}
	The preamble configuration 
\begin{codeexample}[code only]
\usepackage{pgfplots}
\pgfplotsset{compat=1.3}
\end{codeexample}
	allows to choose between backwards compatibility and most recent features.

	\PGFPlots\ version 1.3 comes with new features which may lead to different spacings compared to earlier versions:
	\begin{enumerate}
		\item Version 1.3 comes with the |xlabel near ticks| feature which places (in this case $x$) axis labels ``near'' the tick labels, taking the size of tick labels into account.
		
		Older versions used |xlabel absolute|, i.e.\ an absolute distance between the axis and axis labels (now matter how large or small tick labels are).

		The initial setting \emph{keeps backwards compatibility}.

		You are encouraged to use
\begin{codeexample}[code only]
\usepackage{pgfplots}
\pgfplotsset{compat=1.3}
\end{codeexample}
		in your preamble.
		
		\item Version 1.3 fixes a bug with |anchor=south west| which occurred mainly for reversed axes: the anchors where upside--down. This is fixed now.

		
		If you have written some work--around and want to keep it going, use

		|\pgfplotsset{compat/anchors=pre 1.3}|\pgfmanualpdflabel{/pgfplots/compat/anchors}{} 

		to disable the bug fix.
	\end{enumerate}

	Use |\pgfplotsset{compat=default}| to restore the factory settings.

	The setting |\pgfplotsset{compat=newest}| will always use features of the most recent version. This might result in small changes in the document's appearance.
\end{pgfplotskey}

\subsection{The Team}
\PGFPlots\ has been written mainly by Christian Feuers�nger with many improvements of Pascal Wolkotte and Nick Papior Andersen as a spare time project. We hope it is useful and provides valuable plots.

If you are interested in writing something but don't know how, consider reading the auxiliary manual \href{file:TeX-programming-notes.pdf}{TeX-programming-notes.pdf} which comes with \PGFPlots. It is far from complete, but maybe it is a good starting point (at least for more literature).

\subsection{Acknowledgements}
I thank God for all hours of enjoyed programming. I thank Pascal Wolkotte and Nick Papior Andersen for their programming efforts and contributions as part of the development team. I thank Stefan Tibus, who contributed the |plot shell| feature. I thank Tom Cashman for the contribution of the |reverse legend| feature. Special thanks go to Stefan Pinnow whose tests of \PGFPlots\ lead to numerous quality improvements. Furthermore, I thank Dr.~Schweitzer for many fruitful discussions and Dr.~Meine for his ideas and suggestions. Thanks as well to the many international contributors who provided feature requests or identified bugs or simply manual improvements!

Last but not least, I thank Till Tantau and Mark Wibrow for their excellent graphics (and more) package \PGF\ and \Tikz\ which is the base of \PGFPlots.

\include{pgfplots.install}%
\include{pgfplots.intro}%
\include{pgfplots.reference}%
\include{pgfplots.libs}%
\include{pgfplots.resources}%
\include{pgfplots.importexport}%
\include{pgfplots.basic.reference}%

\printindex

\bibliographystyle{abbrv} %gerapali} %gerabbrv} %gerunsrt.bst} %gerabbrv}% gerplain}
\nocite{pgfplotstable}
\nocite{programmingnotes}
\bibliography{pgfplots}
\end{document}
%
%%%%%%%%%%%%%%%%%%%%%%%%%%%%%%%%%%%%%%%%%%%%%%%%%%%%%%%%%%%%%%%%%%%%%%%%%%%%%
%
% Package pgfplots.sty documentation. 
%
% Copyright 2007/2008 by Christian Feuersaenger.
%
% This program is free software: you can redistribute it and/or modify
% it under the terms of the GNU General Public License as published by
% the Free Software Foundation, either version 3 of the License, or
% (at your option) any later version.
% 
% This program is distributed in the hope that it will be useful,
% but WITHOUT ANY WARRANTY; without even the implied warranty of
% MERCHANTABILITY or FITNESS FOR A PARTICULAR PURPOSE.  See the
% GNU General Public License for more details.
% 
% You should have received a copy of the GNU General Public License
% along with this program.  If not, see <http://www.gnu.org/licenses/>.
%
%
%%%%%%%%%%%%%%%%%%%%%%%%%%%%%%%%%%%%%%%%%%%%%%%%%%%%%%%%%%%%%%%%%%%%%%%%%%%%%
\input{pgfplots.preamble.tex}
%\RequirePackage[german,english,francais]{babel}

\pgfplotsmanualexternalexpensivetrue

%\usetikzlibrary{external}
%\tikzexternalize[prefix=figures/]{pgfplots}

\title{%
	Manual for Package \PGFPlots\\
	{\small Version \pgfplotsversion}\\
	{\small\href{http://sourceforge.net/projects/pgfplots}{http://sourceforge.net/projects/pgfplots}}}

%\includeonly{pgfplots.libs}


\begin{document}

\def\plotcoords{%
\addplot coordinates {
(5,8.312e-02)    (17,2.547e-02)   (49,7.407e-03)
(129,2.102e-03)  (321,5.874e-04)  (769,1.623e-04)
(1793,4.442e-05) (4097,1.207e-05) (9217,3.261e-06)
};

\addplot coordinates{
(7,8.472e-02)    (31,3.044e-02)    (111,1.022e-02)
(351,3.303e-03)  (1023,1.039e-03)  (2815,3.196e-04)
(7423,9.658e-05) (18943,2.873e-05) (47103,8.437e-06)
};

\addplot coordinates{
(9,7.881e-02)     (49,3.243e-02)    (209,1.232e-02)
(769,4.454e-03)   (2561,1.551e-03)  (7937,5.236e-04)
(23297,1.723e-04) (65537,5.545e-05) (178177,1.751e-05)
};

\addplot coordinates{
(11,6.887e-02)    (71,3.177e-02)     (351,1.341e-02)
(1471,5.334e-03)  (5503,2.027e-03)   (18943,7.415e-04)
(61183,2.628e-04) (187903,9.063e-05) (553983,3.053e-05)
};

\addplot coordinates{
(13,5.755e-02)     (97,2.925e-02)     (545,1.351e-02)
(2561,5.842e-03)   (10625,2.397e-03)  (40193,9.414e-04)
(141569,3.564e-04) (471041,1.308e-04) 
(1496065,4.670e-05)
};
}%
\maketitle
\begin{abstract}%
\PGFPlots\ draws high--quality function plots in normal or logarithmic scaling with a user-friendly interface directly in \TeX. The user supplies axis labels, legend entries and the plot coordinates for one or more plots and \PGFPlots\ applies axis scaling, computes any logarithms and axis ticks and draws the plots, supporting line plots, scatter plots, piecewise constant plots, bar plots, area plots, mesh-- and surface plots and some more. It is based on Till Tantau's package \PGF/\Tikz.
\end{abstract}
\tableofcontents
\section{Introduction}
This package provides tools to generate plots and labeled axes easily. It draws normal plots, logplots and semi-logplots, in two and three dimensions. Axis ticks, labels, legends (in case of multiple plots) can be added with key-value options. It can cycle through a set of predefined line/marker/color specifications. In summary, its purpose is to simplify the generation of high-quality function and/or data plots, and solving the problems of
\begin{itemize}
	\item consistency of document and font type and font size,
	\item direct use of \TeX\ math mode in axis descriptions,
	\item consistency of data and figures (no third party tool necessary),
	\item inter document consistency using preamble configurations and styles.
\end{itemize}
Although not necessary, separate |.pdf| or |.eps| graphics can be generated using the |external| library developed as part of \Tikz.

\PGFPlots\ is build completely on \Tikz/\PGF. Knowledge of \Tikz\ will simplify the work with \PGFPlots, although it is not required.

\section[About PGFPlots: Preliminaries]{About {\normalfont\PGFPlots}: Preliminaries}
This section contains information about upgrades, the team, the installation (in case you need to do it manually) and troubleshooting. You may skip it completely except for the upgrade remarks.

However, note that this library requires at least \PGF\ version $2.00$. At the time of this writing, many \TeX-distributions still contain the older \PGF\ version $1.18$, so it may be necessary to install a recent \PGF\ prior to using \PGFPlots.

\subsection{Components}
\PGFPlots\ comes with two components:
\begin{enumerate}
	\item the plotting component (which you are currently reading) and
	\item the \PGFPlotstable\ component which simplifies number formatting and postprocessing of numerical tables. It comes as a separate package and has its own manual \href{file:pgfplotstable.pdf}{pgfplotstable.pdf}.
\end{enumerate}

\subsection{Upgrade remarks}
This release provides a lot of improvements which can be found in all detail in \texttt{ChangeLog} for interested readers. However, some attention is useful with respect to the following changes.

\subsubsection{New Optional Features}
\begin{enumerate}
	\item \PGFPlots\ 1.3 comes with user interface improvements. The technical distinction between ``behavior options'' and ``style options'' of older versions is no longer necessary (although still fully supported).

	\item \PGFPlots\ 1.3 has a new feature which allows to \emph{move axis labels tight to tick labels} automatically.

	Since this affects the spacing, it is not enabled be default.

	Use
\begin{codeexample}[code only]
\usepackage{pgfplots}
\pgfplotsset{compat=1.3}
\end{codeexample}
	in your preamble to benefit from the improved spacing\footnote{The |compat=1.3| setting is necessary for new features which move axis labels around like reversed axis. However, \PGFPlots\ will still work as it does in earlier versions, your documents will remain the same if you don't set it explicitly.}.
	Take a look at the next page for the precise definition of |compat|.

	\item \PGFPlots\ 1.3 now supports reversed axes. It is no longer necessary to use work--arounds with negative units.
\pgfkeys{/pdflinks/search key prefixes in/.add={/pgfplots/,}{}}

	Take a look at the |x dir=reverse| key.

	Existing work--arounds will still function properly. Use |\pgfplotsset{compat=1.3}| together with |x dir=reverse| to switch to the new version.
\end{enumerate}

\subsubsection{Old Features Which May Need Attention}
\begin{enumerate}
	\item The |scatter/classes| feature produces proper legends as of version 1.3. This may change the appearance of existing legends of plots with |scatter/classes|.

	\item Due to a small math bug in \PGF\ $2.00$, you \emph{can't} use the math expression `|-x^2|'. It is necessary to use `|0-x^2|' instead. The same holds for
		`|exp(-x^2)|'; use `|exp(0-x^2)|' instead.

		This will be fixed with \PGF\ $>2.00$.

	\item Starting with \PGFPlots\ $1.1$, |\tikzstyle| should \emph{no longer be used} to set \PGFPlots\ options.
	
	Although |\tikzstyle| is still supported for some older \PGFPlots\ options, you should replace any occurance of |\tikzstyle| with |\pgfplotsset{|\meta{style name}|/.style={|\meta{key-value-list}|}}| or the associated |/.append style| variant. See section~\ref{sec:styles} for more detail.
\end{enumerate}
I apologize for any inconvenience caused by these changes.

\begin{pgfplotskey}{compat=\mchoice{1.3,pre 1.3,default,newest} (initially default)}
	The preamble configuration 
\begin{codeexample}[code only]
\usepackage{pgfplots}
\pgfplotsset{compat=1.3}
\end{codeexample}
	allows to choose between backwards compatibility and most recent features.

	\PGFPlots\ version 1.3 comes with new features which may lead to different spacings compared to earlier versions:
	\begin{enumerate}
		\item Version 1.3 comes with the |xlabel near ticks| feature which places (in this case $x$) axis labels ``near'' the tick labels, taking the size of tick labels into account.
		
		Older versions used |xlabel absolute|, i.e.\ an absolute distance between the axis and axis labels (now matter how large or small tick labels are).

		The initial setting \emph{keeps backwards compatibility}.

		You are encouraged to use
\begin{codeexample}[code only]
\usepackage{pgfplots}
\pgfplotsset{compat=1.3}
\end{codeexample}
		in your preamble.
		
		\item Version 1.3 fixes a bug with |anchor=south west| which occurred mainly for reversed axes: the anchors where upside--down. This is fixed now.

		
		If you have written some work--around and want to keep it going, use

		|\pgfplotsset{compat/anchors=pre 1.3}|\pgfmanualpdflabel{/pgfplots/compat/anchors}{} 

		to disable the bug fix.
	\end{enumerate}

	Use |\pgfplotsset{compat=default}| to restore the factory settings.

	The setting |\pgfplotsset{compat=newest}| will always use features of the most recent version. This might result in small changes in the document's appearance.
\end{pgfplotskey}

\subsection{The Team}
\PGFPlots\ has been written mainly by Christian Feuers�nger with many improvements of Pascal Wolkotte and Nick Papior Andersen as a spare time project. We hope it is useful and provides valuable plots.

If you are interested in writing something but don't know how, consider reading the auxiliary manual \href{file:TeX-programming-notes.pdf}{TeX-programming-notes.pdf} which comes with \PGFPlots. It is far from complete, but maybe it is a good starting point (at least for more literature).

\subsection{Acknowledgements}
I thank God for all hours of enjoyed programming. I thank Pascal Wolkotte and Nick Papior Andersen for their programming efforts and contributions as part of the development team. I thank Stefan Tibus, who contributed the |plot shell| feature. I thank Tom Cashman for the contribution of the |reverse legend| feature. Special thanks go to Stefan Pinnow whose tests of \PGFPlots\ lead to numerous quality improvements. Furthermore, I thank Dr.~Schweitzer for many fruitful discussions and Dr.~Meine for his ideas and suggestions. Thanks as well to the many international contributors who provided feature requests or identified bugs or simply manual improvements!

Last but not least, I thank Till Tantau and Mark Wibrow for their excellent graphics (and more) package \PGF\ and \Tikz\ which is the base of \PGFPlots.

\include{pgfplots.install}%
\include{pgfplots.intro}%
\include{pgfplots.reference}%
\include{pgfplots.libs}%
\include{pgfplots.resources}%
\include{pgfplots.importexport}%
\include{pgfplots.basic.reference}%

\printindex

\bibliographystyle{abbrv} %gerapali} %gerabbrv} %gerunsrt.bst} %gerabbrv}% gerplain}
\nocite{pgfplotstable}
\nocite{programmingnotes}
\bibliography{pgfplots}
\end{document}
%
%%%%%%%%%%%%%%%%%%%%%%%%%%%%%%%%%%%%%%%%%%%%%%%%%%%%%%%%%%%%%%%%%%%%%%%%%%%%%
%
% Package pgfplots.sty documentation. 
%
% Copyright 2007/2008 by Christian Feuersaenger.
%
% This program is free software: you can redistribute it and/or modify
% it under the terms of the GNU General Public License as published by
% the Free Software Foundation, either version 3 of the License, or
% (at your option) any later version.
% 
% This program is distributed in the hope that it will be useful,
% but WITHOUT ANY WARRANTY; without even the implied warranty of
% MERCHANTABILITY or FITNESS FOR A PARTICULAR PURPOSE.  See the
% GNU General Public License for more details.
% 
% You should have received a copy of the GNU General Public License
% along with this program.  If not, see <http://www.gnu.org/licenses/>.
%
%
%%%%%%%%%%%%%%%%%%%%%%%%%%%%%%%%%%%%%%%%%%%%%%%%%%%%%%%%%%%%%%%%%%%%%%%%%%%%%
\input{pgfplots.preamble.tex}
%\RequirePackage[german,english,francais]{babel}

\pgfplotsmanualexternalexpensivetrue

%\usetikzlibrary{external}
%\tikzexternalize[prefix=figures/]{pgfplots}

\title{%
	Manual for Package \PGFPlots\\
	{\small Version \pgfplotsversion}\\
	{\small\href{http://sourceforge.net/projects/pgfplots}{http://sourceforge.net/projects/pgfplots}}}

%\includeonly{pgfplots.libs}


\begin{document}

\def\plotcoords{%
\addplot coordinates {
(5,8.312e-02)    (17,2.547e-02)   (49,7.407e-03)
(129,2.102e-03)  (321,5.874e-04)  (769,1.623e-04)
(1793,4.442e-05) (4097,1.207e-05) (9217,3.261e-06)
};

\addplot coordinates{
(7,8.472e-02)    (31,3.044e-02)    (111,1.022e-02)
(351,3.303e-03)  (1023,1.039e-03)  (2815,3.196e-04)
(7423,9.658e-05) (18943,2.873e-05) (47103,8.437e-06)
};

\addplot coordinates{
(9,7.881e-02)     (49,3.243e-02)    (209,1.232e-02)
(769,4.454e-03)   (2561,1.551e-03)  (7937,5.236e-04)
(23297,1.723e-04) (65537,5.545e-05) (178177,1.751e-05)
};

\addplot coordinates{
(11,6.887e-02)    (71,3.177e-02)     (351,1.341e-02)
(1471,5.334e-03)  (5503,2.027e-03)   (18943,7.415e-04)
(61183,2.628e-04) (187903,9.063e-05) (553983,3.053e-05)
};

\addplot coordinates{
(13,5.755e-02)     (97,2.925e-02)     (545,1.351e-02)
(2561,5.842e-03)   (10625,2.397e-03)  (40193,9.414e-04)
(141569,3.564e-04) (471041,1.308e-04) 
(1496065,4.670e-05)
};
}%
\maketitle
\begin{abstract}%
\PGFPlots\ draws high--quality function plots in normal or logarithmic scaling with a user-friendly interface directly in \TeX. The user supplies axis labels, legend entries and the plot coordinates for one or more plots and \PGFPlots\ applies axis scaling, computes any logarithms and axis ticks and draws the plots, supporting line plots, scatter plots, piecewise constant plots, bar plots, area plots, mesh-- and surface plots and some more. It is based on Till Tantau's package \PGF/\Tikz.
\end{abstract}
\tableofcontents
\section{Introduction}
This package provides tools to generate plots and labeled axes easily. It draws normal plots, logplots and semi-logplots, in two and three dimensions. Axis ticks, labels, legends (in case of multiple plots) can be added with key-value options. It can cycle through a set of predefined line/marker/color specifications. In summary, its purpose is to simplify the generation of high-quality function and/or data plots, and solving the problems of
\begin{itemize}
	\item consistency of document and font type and font size,
	\item direct use of \TeX\ math mode in axis descriptions,
	\item consistency of data and figures (no third party tool necessary),
	\item inter document consistency using preamble configurations and styles.
\end{itemize}
Although not necessary, separate |.pdf| or |.eps| graphics can be generated using the |external| library developed as part of \Tikz.

\PGFPlots\ is build completely on \Tikz/\PGF. Knowledge of \Tikz\ will simplify the work with \PGFPlots, although it is not required.

\section[About PGFPlots: Preliminaries]{About {\normalfont\PGFPlots}: Preliminaries}
This section contains information about upgrades, the team, the installation (in case you need to do it manually) and troubleshooting. You may skip it completely except for the upgrade remarks.

However, note that this library requires at least \PGF\ version $2.00$. At the time of this writing, many \TeX-distributions still contain the older \PGF\ version $1.18$, so it may be necessary to install a recent \PGF\ prior to using \PGFPlots.

\subsection{Components}
\PGFPlots\ comes with two components:
\begin{enumerate}
	\item the plotting component (which you are currently reading) and
	\item the \PGFPlotstable\ component which simplifies number formatting and postprocessing of numerical tables. It comes as a separate package and has its own manual \href{file:pgfplotstable.pdf}{pgfplotstable.pdf}.
\end{enumerate}

\subsection{Upgrade remarks}
This release provides a lot of improvements which can be found in all detail in \texttt{ChangeLog} for interested readers. However, some attention is useful with respect to the following changes.

\subsubsection{New Optional Features}
\begin{enumerate}
	\item \PGFPlots\ 1.3 comes with user interface improvements. The technical distinction between ``behavior options'' and ``style options'' of older versions is no longer necessary (although still fully supported).

	\item \PGFPlots\ 1.3 has a new feature which allows to \emph{move axis labels tight to tick labels} automatically.

	Since this affects the spacing, it is not enabled be default.

	Use
\begin{codeexample}[code only]
\usepackage{pgfplots}
\pgfplotsset{compat=1.3}
\end{codeexample}
	in your preamble to benefit from the improved spacing\footnote{The |compat=1.3| setting is necessary for new features which move axis labels around like reversed axis. However, \PGFPlots\ will still work as it does in earlier versions, your documents will remain the same if you don't set it explicitly.}.
	Take a look at the next page for the precise definition of |compat|.

	\item \PGFPlots\ 1.3 now supports reversed axes. It is no longer necessary to use work--arounds with negative units.
\pgfkeys{/pdflinks/search key prefixes in/.add={/pgfplots/,}{}}

	Take a look at the |x dir=reverse| key.

	Existing work--arounds will still function properly. Use |\pgfplotsset{compat=1.3}| together with |x dir=reverse| to switch to the new version.
\end{enumerate}

\subsubsection{Old Features Which May Need Attention}
\begin{enumerate}
	\item The |scatter/classes| feature produces proper legends as of version 1.3. This may change the appearance of existing legends of plots with |scatter/classes|.

	\item Due to a small math bug in \PGF\ $2.00$, you \emph{can't} use the math expression `|-x^2|'. It is necessary to use `|0-x^2|' instead. The same holds for
		`|exp(-x^2)|'; use `|exp(0-x^2)|' instead.

		This will be fixed with \PGF\ $>2.00$.

	\item Starting with \PGFPlots\ $1.1$, |\tikzstyle| should \emph{no longer be used} to set \PGFPlots\ options.
	
	Although |\tikzstyle| is still supported for some older \PGFPlots\ options, you should replace any occurance of |\tikzstyle| with |\pgfplotsset{|\meta{style name}|/.style={|\meta{key-value-list}|}}| or the associated |/.append style| variant. See section~\ref{sec:styles} for more detail.
\end{enumerate}
I apologize for any inconvenience caused by these changes.

\begin{pgfplotskey}{compat=\mchoice{1.3,pre 1.3,default,newest} (initially default)}
	The preamble configuration 
\begin{codeexample}[code only]
\usepackage{pgfplots}
\pgfplotsset{compat=1.3}
\end{codeexample}
	allows to choose between backwards compatibility and most recent features.

	\PGFPlots\ version 1.3 comes with new features which may lead to different spacings compared to earlier versions:
	\begin{enumerate}
		\item Version 1.3 comes with the |xlabel near ticks| feature which places (in this case $x$) axis labels ``near'' the tick labels, taking the size of tick labels into account.
		
		Older versions used |xlabel absolute|, i.e.\ an absolute distance between the axis and axis labels (now matter how large or small tick labels are).

		The initial setting \emph{keeps backwards compatibility}.

		You are encouraged to use
\begin{codeexample}[code only]
\usepackage{pgfplots}
\pgfplotsset{compat=1.3}
\end{codeexample}
		in your preamble.
		
		\item Version 1.3 fixes a bug with |anchor=south west| which occurred mainly for reversed axes: the anchors where upside--down. This is fixed now.

		
		If you have written some work--around and want to keep it going, use

		|\pgfplotsset{compat/anchors=pre 1.3}|\pgfmanualpdflabel{/pgfplots/compat/anchors}{} 

		to disable the bug fix.
	\end{enumerate}

	Use |\pgfplotsset{compat=default}| to restore the factory settings.

	The setting |\pgfplotsset{compat=newest}| will always use features of the most recent version. This might result in small changes in the document's appearance.
\end{pgfplotskey}

\subsection{The Team}
\PGFPlots\ has been written mainly by Christian Feuers�nger with many improvements of Pascal Wolkotte and Nick Papior Andersen as a spare time project. We hope it is useful and provides valuable plots.

If you are interested in writing something but don't know how, consider reading the auxiliary manual \href{file:TeX-programming-notes.pdf}{TeX-programming-notes.pdf} which comes with \PGFPlots. It is far from complete, but maybe it is a good starting point (at least for more literature).

\subsection{Acknowledgements}
I thank God for all hours of enjoyed programming. I thank Pascal Wolkotte and Nick Papior Andersen for their programming efforts and contributions as part of the development team. I thank Stefan Tibus, who contributed the |plot shell| feature. I thank Tom Cashman for the contribution of the |reverse legend| feature. Special thanks go to Stefan Pinnow whose tests of \PGFPlots\ lead to numerous quality improvements. Furthermore, I thank Dr.~Schweitzer for many fruitful discussions and Dr.~Meine for his ideas and suggestions. Thanks as well to the many international contributors who provided feature requests or identified bugs or simply manual improvements!

Last but not least, I thank Till Tantau and Mark Wibrow for their excellent graphics (and more) package \PGF\ and \Tikz\ which is the base of \PGFPlots.

\include{pgfplots.install}%
\include{pgfplots.intro}%
\include{pgfplots.reference}%
\include{pgfplots.libs}%
\include{pgfplots.resources}%
\include{pgfplots.importexport}%
\include{pgfplots.basic.reference}%

\printindex

\bibliographystyle{abbrv} %gerapali} %gerabbrv} %gerunsrt.bst} %gerabbrv}% gerplain}
\nocite{pgfplotstable}
\nocite{programmingnotes}
\bibliography{pgfplots}
\end{document}
%
%%%%%%%%%%%%%%%%%%%%%%%%%%%%%%%%%%%%%%%%%%%%%%%%%%%%%%%%%%%%%%%%%%%%%%%%%%%%%
%
% Package pgfplots.sty documentation. 
%
% Copyright 2007/2008 by Christian Feuersaenger.
%
% This program is free software: you can redistribute it and/or modify
% it under the terms of the GNU General Public License as published by
% the Free Software Foundation, either version 3 of the License, or
% (at your option) any later version.
% 
% This program is distributed in the hope that it will be useful,
% but WITHOUT ANY WARRANTY; without even the implied warranty of
% MERCHANTABILITY or FITNESS FOR A PARTICULAR PURPOSE.  See the
% GNU General Public License for more details.
% 
% You should have received a copy of the GNU General Public License
% along with this program.  If not, see <http://www.gnu.org/licenses/>.
%
%
%%%%%%%%%%%%%%%%%%%%%%%%%%%%%%%%%%%%%%%%%%%%%%%%%%%%%%%%%%%%%%%%%%%%%%%%%%%%%
\input{pgfplots.preamble.tex}
%\RequirePackage[german,english,francais]{babel}

\pgfplotsmanualexternalexpensivetrue

%\usetikzlibrary{external}
%\tikzexternalize[prefix=figures/]{pgfplots}

\title{%
	Manual for Package \PGFPlots\\
	{\small Version \pgfplotsversion}\\
	{\small\href{http://sourceforge.net/projects/pgfplots}{http://sourceforge.net/projects/pgfplots}}}

%\includeonly{pgfplots.libs}


\begin{document}

\def\plotcoords{%
\addplot coordinates {
(5,8.312e-02)    (17,2.547e-02)   (49,7.407e-03)
(129,2.102e-03)  (321,5.874e-04)  (769,1.623e-04)
(1793,4.442e-05) (4097,1.207e-05) (9217,3.261e-06)
};

\addplot coordinates{
(7,8.472e-02)    (31,3.044e-02)    (111,1.022e-02)
(351,3.303e-03)  (1023,1.039e-03)  (2815,3.196e-04)
(7423,9.658e-05) (18943,2.873e-05) (47103,8.437e-06)
};

\addplot coordinates{
(9,7.881e-02)     (49,3.243e-02)    (209,1.232e-02)
(769,4.454e-03)   (2561,1.551e-03)  (7937,5.236e-04)
(23297,1.723e-04) (65537,5.545e-05) (178177,1.751e-05)
};

\addplot coordinates{
(11,6.887e-02)    (71,3.177e-02)     (351,1.341e-02)
(1471,5.334e-03)  (5503,2.027e-03)   (18943,7.415e-04)
(61183,2.628e-04) (187903,9.063e-05) (553983,3.053e-05)
};

\addplot coordinates{
(13,5.755e-02)     (97,2.925e-02)     (545,1.351e-02)
(2561,5.842e-03)   (10625,2.397e-03)  (40193,9.414e-04)
(141569,3.564e-04) (471041,1.308e-04) 
(1496065,4.670e-05)
};
}%
\maketitle
\begin{abstract}%
\PGFPlots\ draws high--quality function plots in normal or logarithmic scaling with a user-friendly interface directly in \TeX. The user supplies axis labels, legend entries and the plot coordinates for one or more plots and \PGFPlots\ applies axis scaling, computes any logarithms and axis ticks and draws the plots, supporting line plots, scatter plots, piecewise constant plots, bar plots, area plots, mesh-- and surface plots and some more. It is based on Till Tantau's package \PGF/\Tikz.
\end{abstract}
\tableofcontents
\section{Introduction}
This package provides tools to generate plots and labeled axes easily. It draws normal plots, logplots and semi-logplots, in two and three dimensions. Axis ticks, labels, legends (in case of multiple plots) can be added with key-value options. It can cycle through a set of predefined line/marker/color specifications. In summary, its purpose is to simplify the generation of high-quality function and/or data plots, and solving the problems of
\begin{itemize}
	\item consistency of document and font type and font size,
	\item direct use of \TeX\ math mode in axis descriptions,
	\item consistency of data and figures (no third party tool necessary),
	\item inter document consistency using preamble configurations and styles.
\end{itemize}
Although not necessary, separate |.pdf| or |.eps| graphics can be generated using the |external| library developed as part of \Tikz.

\PGFPlots\ is build completely on \Tikz/\PGF. Knowledge of \Tikz\ will simplify the work with \PGFPlots, although it is not required.

\section[About PGFPlots: Preliminaries]{About {\normalfont\PGFPlots}: Preliminaries}
This section contains information about upgrades, the team, the installation (in case you need to do it manually) and troubleshooting. You may skip it completely except for the upgrade remarks.

However, note that this library requires at least \PGF\ version $2.00$. At the time of this writing, many \TeX-distributions still contain the older \PGF\ version $1.18$, so it may be necessary to install a recent \PGF\ prior to using \PGFPlots.

\subsection{Components}
\PGFPlots\ comes with two components:
\begin{enumerate}
	\item the plotting component (which you are currently reading) and
	\item the \PGFPlotstable\ component which simplifies number formatting and postprocessing of numerical tables. It comes as a separate package and has its own manual \href{file:pgfplotstable.pdf}{pgfplotstable.pdf}.
\end{enumerate}

\subsection{Upgrade remarks}
This release provides a lot of improvements which can be found in all detail in \texttt{ChangeLog} for interested readers. However, some attention is useful with respect to the following changes.

\subsubsection{New Optional Features}
\begin{enumerate}
	\item \PGFPlots\ 1.3 comes with user interface improvements. The technical distinction between ``behavior options'' and ``style options'' of older versions is no longer necessary (although still fully supported).

	\item \PGFPlots\ 1.3 has a new feature which allows to \emph{move axis labels tight to tick labels} automatically.

	Since this affects the spacing, it is not enabled be default.

	Use
\begin{codeexample}[code only]
\usepackage{pgfplots}
\pgfplotsset{compat=1.3}
\end{codeexample}
	in your preamble to benefit from the improved spacing\footnote{The |compat=1.3| setting is necessary for new features which move axis labels around like reversed axis. However, \PGFPlots\ will still work as it does in earlier versions, your documents will remain the same if you don't set it explicitly.}.
	Take a look at the next page for the precise definition of |compat|.

	\item \PGFPlots\ 1.3 now supports reversed axes. It is no longer necessary to use work--arounds with negative units.
\pgfkeys{/pdflinks/search key prefixes in/.add={/pgfplots/,}{}}

	Take a look at the |x dir=reverse| key.

	Existing work--arounds will still function properly. Use |\pgfplotsset{compat=1.3}| together with |x dir=reverse| to switch to the new version.
\end{enumerate}

\subsubsection{Old Features Which May Need Attention}
\begin{enumerate}
	\item The |scatter/classes| feature produces proper legends as of version 1.3. This may change the appearance of existing legends of plots with |scatter/classes|.

	\item Due to a small math bug in \PGF\ $2.00$, you \emph{can't} use the math expression `|-x^2|'. It is necessary to use `|0-x^2|' instead. The same holds for
		`|exp(-x^2)|'; use `|exp(0-x^2)|' instead.

		This will be fixed with \PGF\ $>2.00$.

	\item Starting with \PGFPlots\ $1.1$, |\tikzstyle| should \emph{no longer be used} to set \PGFPlots\ options.
	
	Although |\tikzstyle| is still supported for some older \PGFPlots\ options, you should replace any occurance of |\tikzstyle| with |\pgfplotsset{|\meta{style name}|/.style={|\meta{key-value-list}|}}| or the associated |/.append style| variant. See section~\ref{sec:styles} for more detail.
\end{enumerate}
I apologize for any inconvenience caused by these changes.

\begin{pgfplotskey}{compat=\mchoice{1.3,pre 1.3,default,newest} (initially default)}
	The preamble configuration 
\begin{codeexample}[code only]
\usepackage{pgfplots}
\pgfplotsset{compat=1.3}
\end{codeexample}
	allows to choose between backwards compatibility and most recent features.

	\PGFPlots\ version 1.3 comes with new features which may lead to different spacings compared to earlier versions:
	\begin{enumerate}
		\item Version 1.3 comes with the |xlabel near ticks| feature which places (in this case $x$) axis labels ``near'' the tick labels, taking the size of tick labels into account.
		
		Older versions used |xlabel absolute|, i.e.\ an absolute distance between the axis and axis labels (now matter how large or small tick labels are).

		The initial setting \emph{keeps backwards compatibility}.

		You are encouraged to use
\begin{codeexample}[code only]
\usepackage{pgfplots}
\pgfplotsset{compat=1.3}
\end{codeexample}
		in your preamble.
		
		\item Version 1.3 fixes a bug with |anchor=south west| which occurred mainly for reversed axes: the anchors where upside--down. This is fixed now.

		
		If you have written some work--around and want to keep it going, use

		|\pgfplotsset{compat/anchors=pre 1.3}|\pgfmanualpdflabel{/pgfplots/compat/anchors}{} 

		to disable the bug fix.
	\end{enumerate}

	Use |\pgfplotsset{compat=default}| to restore the factory settings.

	The setting |\pgfplotsset{compat=newest}| will always use features of the most recent version. This might result in small changes in the document's appearance.
\end{pgfplotskey}

\subsection{The Team}
\PGFPlots\ has been written mainly by Christian Feuers�nger with many improvements of Pascal Wolkotte and Nick Papior Andersen as a spare time project. We hope it is useful and provides valuable plots.

If you are interested in writing something but don't know how, consider reading the auxiliary manual \href{file:TeX-programming-notes.pdf}{TeX-programming-notes.pdf} which comes with \PGFPlots. It is far from complete, but maybe it is a good starting point (at least for more literature).

\subsection{Acknowledgements}
I thank God for all hours of enjoyed programming. I thank Pascal Wolkotte and Nick Papior Andersen for their programming efforts and contributions as part of the development team. I thank Stefan Tibus, who contributed the |plot shell| feature. I thank Tom Cashman for the contribution of the |reverse legend| feature. Special thanks go to Stefan Pinnow whose tests of \PGFPlots\ lead to numerous quality improvements. Furthermore, I thank Dr.~Schweitzer for many fruitful discussions and Dr.~Meine for his ideas and suggestions. Thanks as well to the many international contributors who provided feature requests or identified bugs or simply manual improvements!

Last but not least, I thank Till Tantau and Mark Wibrow for their excellent graphics (and more) package \PGF\ and \Tikz\ which is the base of \PGFPlots.

\include{pgfplots.install}%
\include{pgfplots.intro}%
\include{pgfplots.reference}%
\include{pgfplots.libs}%
\include{pgfplots.resources}%
\include{pgfplots.importexport}%
\include{pgfplots.basic.reference}%

\printindex

\bibliographystyle{abbrv} %gerapali} %gerabbrv} %gerunsrt.bst} %gerabbrv}% gerplain}
\nocite{pgfplotstable}
\nocite{programmingnotes}
\bibliography{pgfplots}
\end{document}
%
%%%%%%%%%%%%%%%%%%%%%%%%%%%%%%%%%%%%%%%%%%%%%%%%%%%%%%%%%%%%%%%%%%%%%%%%%%%%%
%
% Package pgfplots.sty documentation. 
%
% Copyright 2007/2008 by Christian Feuersaenger.
%
% This program is free software: you can redistribute it and/or modify
% it under the terms of the GNU General Public License as published by
% the Free Software Foundation, either version 3 of the License, or
% (at your option) any later version.
% 
% This program is distributed in the hope that it will be useful,
% but WITHOUT ANY WARRANTY; without even the implied warranty of
% MERCHANTABILITY or FITNESS FOR A PARTICULAR PURPOSE.  See the
% GNU General Public License for more details.
% 
% You should have received a copy of the GNU General Public License
% along with this program.  If not, see <http://www.gnu.org/licenses/>.
%
%
%%%%%%%%%%%%%%%%%%%%%%%%%%%%%%%%%%%%%%%%%%%%%%%%%%%%%%%%%%%%%%%%%%%%%%%%%%%%%
\input{pgfplots.preamble.tex}
%\RequirePackage[german,english,francais]{babel}

\pgfplotsmanualexternalexpensivetrue

%\usetikzlibrary{external}
%\tikzexternalize[prefix=figures/]{pgfplots}

\title{%
	Manual for Package \PGFPlots\\
	{\small Version \pgfplotsversion}\\
	{\small\href{http://sourceforge.net/projects/pgfplots}{http://sourceforge.net/projects/pgfplots}}}

%\includeonly{pgfplots.libs}


\begin{document}

\def\plotcoords{%
\addplot coordinates {
(5,8.312e-02)    (17,2.547e-02)   (49,7.407e-03)
(129,2.102e-03)  (321,5.874e-04)  (769,1.623e-04)
(1793,4.442e-05) (4097,1.207e-05) (9217,3.261e-06)
};

\addplot coordinates{
(7,8.472e-02)    (31,3.044e-02)    (111,1.022e-02)
(351,3.303e-03)  (1023,1.039e-03)  (2815,3.196e-04)
(7423,9.658e-05) (18943,2.873e-05) (47103,8.437e-06)
};

\addplot coordinates{
(9,7.881e-02)     (49,3.243e-02)    (209,1.232e-02)
(769,4.454e-03)   (2561,1.551e-03)  (7937,5.236e-04)
(23297,1.723e-04) (65537,5.545e-05) (178177,1.751e-05)
};

\addplot coordinates{
(11,6.887e-02)    (71,3.177e-02)     (351,1.341e-02)
(1471,5.334e-03)  (5503,2.027e-03)   (18943,7.415e-04)
(61183,2.628e-04) (187903,9.063e-05) (553983,3.053e-05)
};

\addplot coordinates{
(13,5.755e-02)     (97,2.925e-02)     (545,1.351e-02)
(2561,5.842e-03)   (10625,2.397e-03)  (40193,9.414e-04)
(141569,3.564e-04) (471041,1.308e-04) 
(1496065,4.670e-05)
};
}%
\maketitle
\begin{abstract}%
\PGFPlots\ draws high--quality function plots in normal or logarithmic scaling with a user-friendly interface directly in \TeX. The user supplies axis labels, legend entries and the plot coordinates for one or more plots and \PGFPlots\ applies axis scaling, computes any logarithms and axis ticks and draws the plots, supporting line plots, scatter plots, piecewise constant plots, bar plots, area plots, mesh-- and surface plots and some more. It is based on Till Tantau's package \PGF/\Tikz.
\end{abstract}
\tableofcontents
\section{Introduction}
This package provides tools to generate plots and labeled axes easily. It draws normal plots, logplots and semi-logplots, in two and three dimensions. Axis ticks, labels, legends (in case of multiple plots) can be added with key-value options. It can cycle through a set of predefined line/marker/color specifications. In summary, its purpose is to simplify the generation of high-quality function and/or data plots, and solving the problems of
\begin{itemize}
	\item consistency of document and font type and font size,
	\item direct use of \TeX\ math mode in axis descriptions,
	\item consistency of data and figures (no third party tool necessary),
	\item inter document consistency using preamble configurations and styles.
\end{itemize}
Although not necessary, separate |.pdf| or |.eps| graphics can be generated using the |external| library developed as part of \Tikz.

\PGFPlots\ is build completely on \Tikz/\PGF. Knowledge of \Tikz\ will simplify the work with \PGFPlots, although it is not required.

\section[About PGFPlots: Preliminaries]{About {\normalfont\PGFPlots}: Preliminaries}
This section contains information about upgrades, the team, the installation (in case you need to do it manually) and troubleshooting. You may skip it completely except for the upgrade remarks.

However, note that this library requires at least \PGF\ version $2.00$. At the time of this writing, many \TeX-distributions still contain the older \PGF\ version $1.18$, so it may be necessary to install a recent \PGF\ prior to using \PGFPlots.

\subsection{Components}
\PGFPlots\ comes with two components:
\begin{enumerate}
	\item the plotting component (which you are currently reading) and
	\item the \PGFPlotstable\ component which simplifies number formatting and postprocessing of numerical tables. It comes as a separate package and has its own manual \href{file:pgfplotstable.pdf}{pgfplotstable.pdf}.
\end{enumerate}

\subsection{Upgrade remarks}
This release provides a lot of improvements which can be found in all detail in \texttt{ChangeLog} for interested readers. However, some attention is useful with respect to the following changes.

\subsubsection{New Optional Features}
\begin{enumerate}
	\item \PGFPlots\ 1.3 comes with user interface improvements. The technical distinction between ``behavior options'' and ``style options'' of older versions is no longer necessary (although still fully supported).

	\item \PGFPlots\ 1.3 has a new feature which allows to \emph{move axis labels tight to tick labels} automatically.

	Since this affects the spacing, it is not enabled be default.

	Use
\begin{codeexample}[code only]
\usepackage{pgfplots}
\pgfplotsset{compat=1.3}
\end{codeexample}
	in your preamble to benefit from the improved spacing\footnote{The |compat=1.3| setting is necessary for new features which move axis labels around like reversed axis. However, \PGFPlots\ will still work as it does in earlier versions, your documents will remain the same if you don't set it explicitly.}.
	Take a look at the next page for the precise definition of |compat|.

	\item \PGFPlots\ 1.3 now supports reversed axes. It is no longer necessary to use work--arounds with negative units.
\pgfkeys{/pdflinks/search key prefixes in/.add={/pgfplots/,}{}}

	Take a look at the |x dir=reverse| key.

	Existing work--arounds will still function properly. Use |\pgfplotsset{compat=1.3}| together with |x dir=reverse| to switch to the new version.
\end{enumerate}

\subsubsection{Old Features Which May Need Attention}
\begin{enumerate}
	\item The |scatter/classes| feature produces proper legends as of version 1.3. This may change the appearance of existing legends of plots with |scatter/classes|.

	\item Due to a small math bug in \PGF\ $2.00$, you \emph{can't} use the math expression `|-x^2|'. It is necessary to use `|0-x^2|' instead. The same holds for
		`|exp(-x^2)|'; use `|exp(0-x^2)|' instead.

		This will be fixed with \PGF\ $>2.00$.

	\item Starting with \PGFPlots\ $1.1$, |\tikzstyle| should \emph{no longer be used} to set \PGFPlots\ options.
	
	Although |\tikzstyle| is still supported for some older \PGFPlots\ options, you should replace any occurance of |\tikzstyle| with |\pgfplotsset{|\meta{style name}|/.style={|\meta{key-value-list}|}}| or the associated |/.append style| variant. See section~\ref{sec:styles} for more detail.
\end{enumerate}
I apologize for any inconvenience caused by these changes.

\begin{pgfplotskey}{compat=\mchoice{1.3,pre 1.3,default,newest} (initially default)}
	The preamble configuration 
\begin{codeexample}[code only]
\usepackage{pgfplots}
\pgfplotsset{compat=1.3}
\end{codeexample}
	allows to choose between backwards compatibility and most recent features.

	\PGFPlots\ version 1.3 comes with new features which may lead to different spacings compared to earlier versions:
	\begin{enumerate}
		\item Version 1.3 comes with the |xlabel near ticks| feature which places (in this case $x$) axis labels ``near'' the tick labels, taking the size of tick labels into account.
		
		Older versions used |xlabel absolute|, i.e.\ an absolute distance between the axis and axis labels (now matter how large or small tick labels are).

		The initial setting \emph{keeps backwards compatibility}.

		You are encouraged to use
\begin{codeexample}[code only]
\usepackage{pgfplots}
\pgfplotsset{compat=1.3}
\end{codeexample}
		in your preamble.
		
		\item Version 1.3 fixes a bug with |anchor=south west| which occurred mainly for reversed axes: the anchors where upside--down. This is fixed now.

		
		If you have written some work--around and want to keep it going, use

		|\pgfplotsset{compat/anchors=pre 1.3}|\pgfmanualpdflabel{/pgfplots/compat/anchors}{} 

		to disable the bug fix.
	\end{enumerate}

	Use |\pgfplotsset{compat=default}| to restore the factory settings.

	The setting |\pgfplotsset{compat=newest}| will always use features of the most recent version. This might result in small changes in the document's appearance.
\end{pgfplotskey}

\subsection{The Team}
\PGFPlots\ has been written mainly by Christian Feuers�nger with many improvements of Pascal Wolkotte and Nick Papior Andersen as a spare time project. We hope it is useful and provides valuable plots.

If you are interested in writing something but don't know how, consider reading the auxiliary manual \href{file:TeX-programming-notes.pdf}{TeX-programming-notes.pdf} which comes with \PGFPlots. It is far from complete, but maybe it is a good starting point (at least for more literature).

\subsection{Acknowledgements}
I thank God for all hours of enjoyed programming. I thank Pascal Wolkotte and Nick Papior Andersen for their programming efforts and contributions as part of the development team. I thank Stefan Tibus, who contributed the |plot shell| feature. I thank Tom Cashman for the contribution of the |reverse legend| feature. Special thanks go to Stefan Pinnow whose tests of \PGFPlots\ lead to numerous quality improvements. Furthermore, I thank Dr.~Schweitzer for many fruitful discussions and Dr.~Meine for his ideas and suggestions. Thanks as well to the many international contributors who provided feature requests or identified bugs or simply manual improvements!

Last but not least, I thank Till Tantau and Mark Wibrow for their excellent graphics (and more) package \PGF\ and \Tikz\ which is the base of \PGFPlots.

\include{pgfplots.install}%
\include{pgfplots.intro}%
\include{pgfplots.reference}%
\include{pgfplots.libs}%
\include{pgfplots.resources}%
\include{pgfplots.importexport}%
\include{pgfplots.basic.reference}%

\printindex

\bibliographystyle{abbrv} %gerapali} %gerabbrv} %gerunsrt.bst} %gerabbrv}% gerplain}
\nocite{pgfplotstable}
\nocite{programmingnotes}
\bibliography{pgfplots}
\end{document}
%
%%%%%%%%%%%%%%%%%%%%%%%%%%%%%%%%%%%%%%%%%%%%%%%%%%%%%%%%%%%%%%%%%%%%%%%%%%%%%
%
% Package pgfplots.sty documentation. 
%
% Copyright 2007/2008 by Christian Feuersaenger.
%
% This program is free software: you can redistribute it and/or modify
% it under the terms of the GNU General Public License as published by
% the Free Software Foundation, either version 3 of the License, or
% (at your option) any later version.
% 
% This program is distributed in the hope that it will be useful,
% but WITHOUT ANY WARRANTY; without even the implied warranty of
% MERCHANTABILITY or FITNESS FOR A PARTICULAR PURPOSE.  See the
% GNU General Public License for more details.
% 
% You should have received a copy of the GNU General Public License
% along with this program.  If not, see <http://www.gnu.org/licenses/>.
%
%
%%%%%%%%%%%%%%%%%%%%%%%%%%%%%%%%%%%%%%%%%%%%%%%%%%%%%%%%%%%%%%%%%%%%%%%%%%%%%
\input{pgfplots.preamble.tex}
%\RequirePackage[german,english,francais]{babel}

\pgfplotsmanualexternalexpensivetrue

%\usetikzlibrary{external}
%\tikzexternalize[prefix=figures/]{pgfplots}

\title{%
	Manual for Package \PGFPlots\\
	{\small Version \pgfplotsversion}\\
	{\small\href{http://sourceforge.net/projects/pgfplots}{http://sourceforge.net/projects/pgfplots}}}

%\includeonly{pgfplots.libs}


\begin{document}

\def\plotcoords{%
\addplot coordinates {
(5,8.312e-02)    (17,2.547e-02)   (49,7.407e-03)
(129,2.102e-03)  (321,5.874e-04)  (769,1.623e-04)
(1793,4.442e-05) (4097,1.207e-05) (9217,3.261e-06)
};

\addplot coordinates{
(7,8.472e-02)    (31,3.044e-02)    (111,1.022e-02)
(351,3.303e-03)  (1023,1.039e-03)  (2815,3.196e-04)
(7423,9.658e-05) (18943,2.873e-05) (47103,8.437e-06)
};

\addplot coordinates{
(9,7.881e-02)     (49,3.243e-02)    (209,1.232e-02)
(769,4.454e-03)   (2561,1.551e-03)  (7937,5.236e-04)
(23297,1.723e-04) (65537,5.545e-05) (178177,1.751e-05)
};

\addplot coordinates{
(11,6.887e-02)    (71,3.177e-02)     (351,1.341e-02)
(1471,5.334e-03)  (5503,2.027e-03)   (18943,7.415e-04)
(61183,2.628e-04) (187903,9.063e-05) (553983,3.053e-05)
};

\addplot coordinates{
(13,5.755e-02)     (97,2.925e-02)     (545,1.351e-02)
(2561,5.842e-03)   (10625,2.397e-03)  (40193,9.414e-04)
(141569,3.564e-04) (471041,1.308e-04) 
(1496065,4.670e-05)
};
}%
\maketitle
\begin{abstract}%
\PGFPlots\ draws high--quality function plots in normal or logarithmic scaling with a user-friendly interface directly in \TeX. The user supplies axis labels, legend entries and the plot coordinates for one or more plots and \PGFPlots\ applies axis scaling, computes any logarithms and axis ticks and draws the plots, supporting line plots, scatter plots, piecewise constant plots, bar plots, area plots, mesh-- and surface plots and some more. It is based on Till Tantau's package \PGF/\Tikz.
\end{abstract}
\tableofcontents
\section{Introduction}
This package provides tools to generate plots and labeled axes easily. It draws normal plots, logplots and semi-logplots, in two and three dimensions. Axis ticks, labels, legends (in case of multiple plots) can be added with key-value options. It can cycle through a set of predefined line/marker/color specifications. In summary, its purpose is to simplify the generation of high-quality function and/or data plots, and solving the problems of
\begin{itemize}
	\item consistency of document and font type and font size,
	\item direct use of \TeX\ math mode in axis descriptions,
	\item consistency of data and figures (no third party tool necessary),
	\item inter document consistency using preamble configurations and styles.
\end{itemize}
Although not necessary, separate |.pdf| or |.eps| graphics can be generated using the |external| library developed as part of \Tikz.

\PGFPlots\ is build completely on \Tikz/\PGF. Knowledge of \Tikz\ will simplify the work with \PGFPlots, although it is not required.

\section[About PGFPlots: Preliminaries]{About {\normalfont\PGFPlots}: Preliminaries}
This section contains information about upgrades, the team, the installation (in case you need to do it manually) and troubleshooting. You may skip it completely except for the upgrade remarks.

However, note that this library requires at least \PGF\ version $2.00$. At the time of this writing, many \TeX-distributions still contain the older \PGF\ version $1.18$, so it may be necessary to install a recent \PGF\ prior to using \PGFPlots.

\subsection{Components}
\PGFPlots\ comes with two components:
\begin{enumerate}
	\item the plotting component (which you are currently reading) and
	\item the \PGFPlotstable\ component which simplifies number formatting and postprocessing of numerical tables. It comes as a separate package and has its own manual \href{file:pgfplotstable.pdf}{pgfplotstable.pdf}.
\end{enumerate}

\subsection{Upgrade remarks}
This release provides a lot of improvements which can be found in all detail in \texttt{ChangeLog} for interested readers. However, some attention is useful with respect to the following changes.

\subsubsection{New Optional Features}
\begin{enumerate}
	\item \PGFPlots\ 1.3 comes with user interface improvements. The technical distinction between ``behavior options'' and ``style options'' of older versions is no longer necessary (although still fully supported).

	\item \PGFPlots\ 1.3 has a new feature which allows to \emph{move axis labels tight to tick labels} automatically.

	Since this affects the spacing, it is not enabled be default.

	Use
\begin{codeexample}[code only]
\usepackage{pgfplots}
\pgfplotsset{compat=1.3}
\end{codeexample}
	in your preamble to benefit from the improved spacing\footnote{The |compat=1.3| setting is necessary for new features which move axis labels around like reversed axis. However, \PGFPlots\ will still work as it does in earlier versions, your documents will remain the same if you don't set it explicitly.}.
	Take a look at the next page for the precise definition of |compat|.

	\item \PGFPlots\ 1.3 now supports reversed axes. It is no longer necessary to use work--arounds with negative units.
\pgfkeys{/pdflinks/search key prefixes in/.add={/pgfplots/,}{}}

	Take a look at the |x dir=reverse| key.

	Existing work--arounds will still function properly. Use |\pgfplotsset{compat=1.3}| together with |x dir=reverse| to switch to the new version.
\end{enumerate}

\subsubsection{Old Features Which May Need Attention}
\begin{enumerate}
	\item The |scatter/classes| feature produces proper legends as of version 1.3. This may change the appearance of existing legends of plots with |scatter/classes|.

	\item Due to a small math bug in \PGF\ $2.00$, you \emph{can't} use the math expression `|-x^2|'. It is necessary to use `|0-x^2|' instead. The same holds for
		`|exp(-x^2)|'; use `|exp(0-x^2)|' instead.

		This will be fixed with \PGF\ $>2.00$.

	\item Starting with \PGFPlots\ $1.1$, |\tikzstyle| should \emph{no longer be used} to set \PGFPlots\ options.
	
	Although |\tikzstyle| is still supported for some older \PGFPlots\ options, you should replace any occurance of |\tikzstyle| with |\pgfplotsset{|\meta{style name}|/.style={|\meta{key-value-list}|}}| or the associated |/.append style| variant. See section~\ref{sec:styles} for more detail.
\end{enumerate}
I apologize for any inconvenience caused by these changes.

\begin{pgfplotskey}{compat=\mchoice{1.3,pre 1.3,default,newest} (initially default)}
	The preamble configuration 
\begin{codeexample}[code only]
\usepackage{pgfplots}
\pgfplotsset{compat=1.3}
\end{codeexample}
	allows to choose between backwards compatibility and most recent features.

	\PGFPlots\ version 1.3 comes with new features which may lead to different spacings compared to earlier versions:
	\begin{enumerate}
		\item Version 1.3 comes with the |xlabel near ticks| feature which places (in this case $x$) axis labels ``near'' the tick labels, taking the size of tick labels into account.
		
		Older versions used |xlabel absolute|, i.e.\ an absolute distance between the axis and axis labels (now matter how large or small tick labels are).

		The initial setting \emph{keeps backwards compatibility}.

		You are encouraged to use
\begin{codeexample}[code only]
\usepackage{pgfplots}
\pgfplotsset{compat=1.3}
\end{codeexample}
		in your preamble.
		
		\item Version 1.3 fixes a bug with |anchor=south west| which occurred mainly for reversed axes: the anchors where upside--down. This is fixed now.

		
		If you have written some work--around and want to keep it going, use

		|\pgfplotsset{compat/anchors=pre 1.3}|\pgfmanualpdflabel{/pgfplots/compat/anchors}{} 

		to disable the bug fix.
	\end{enumerate}

	Use |\pgfplotsset{compat=default}| to restore the factory settings.

	The setting |\pgfplotsset{compat=newest}| will always use features of the most recent version. This might result in small changes in the document's appearance.
\end{pgfplotskey}

\subsection{The Team}
\PGFPlots\ has been written mainly by Christian Feuers�nger with many improvements of Pascal Wolkotte and Nick Papior Andersen as a spare time project. We hope it is useful and provides valuable plots.

If you are interested in writing something but don't know how, consider reading the auxiliary manual \href{file:TeX-programming-notes.pdf}{TeX-programming-notes.pdf} which comes with \PGFPlots. It is far from complete, but maybe it is a good starting point (at least for more literature).

\subsection{Acknowledgements}
I thank God for all hours of enjoyed programming. I thank Pascal Wolkotte and Nick Papior Andersen for their programming efforts and contributions as part of the development team. I thank Stefan Tibus, who contributed the |plot shell| feature. I thank Tom Cashman for the contribution of the |reverse legend| feature. Special thanks go to Stefan Pinnow whose tests of \PGFPlots\ lead to numerous quality improvements. Furthermore, I thank Dr.~Schweitzer for many fruitful discussions and Dr.~Meine for his ideas and suggestions. Thanks as well to the many international contributors who provided feature requests or identified bugs or simply manual improvements!

Last but not least, I thank Till Tantau and Mark Wibrow for their excellent graphics (and more) package \PGF\ and \Tikz\ which is the base of \PGFPlots.

\include{pgfplots.install}%
\include{pgfplots.intro}%
\include{pgfplots.reference}%
\include{pgfplots.libs}%
\include{pgfplots.resources}%
\include{pgfplots.importexport}%
\include{pgfplots.basic.reference}%

\printindex

\bibliographystyle{abbrv} %gerapali} %gerabbrv} %gerunsrt.bst} %gerabbrv}% gerplain}
\nocite{pgfplotstable}
\nocite{programmingnotes}
\bibliography{pgfplots}
\end{document}
%
\include{pgfplots.basic.reference}%

\printindex

\bibliographystyle{abbrv} %gerapali} %gerabbrv} %gerunsrt.bst} %gerabbrv}% gerplain}
\nocite{pgfplotstable}
\nocite{programmingnotes}
\bibliography{pgfplots}
\end{document}
%
%%%%%%%%%%%%%%%%%%%%%%%%%%%%%%%%%%%%%%%%%%%%%%%%%%%%%%%%%%%%%%%%%%%%%%%%%%%%%
%
% Package pgfplots.sty documentation. 
%
% Copyright 2007/2008 by Christian Feuersaenger.
%
% This program is free software: you can redistribute it and/or modify
% it under the terms of the GNU General Public License as published by
% the Free Software Foundation, either version 3 of the License, or
% (at your option) any later version.
% 
% This program is distributed in the hope that it will be useful,
% but WITHOUT ANY WARRANTY; without even the implied warranty of
% MERCHANTABILITY or FITNESS FOR A PARTICULAR PURPOSE.  See the
% GNU General Public License for more details.
% 
% You should have received a copy of the GNU General Public License
% along with this program.  If not, see <http://www.gnu.org/licenses/>.
%
%
%%%%%%%%%%%%%%%%%%%%%%%%%%%%%%%%%%%%%%%%%%%%%%%%%%%%%%%%%%%%%%%%%%%%%%%%%%%%%
%%%%%%%%%%%%%%%%%%%%%%%%%%%%%%%%%%%%%%%%%%%%%%%%%%%%%%%%%%%%%%%%%%%%%%%%%%%%%
%
% Package pgfplots.sty documentation. 
%
% Copyright 2007/2008 by Christian Feuersaenger.
%
% This program is free software: you can redistribute it and/or modify
% it under the terms of the GNU General Public License as published by
% the Free Software Foundation, either version 3 of the License, or
% (at your option) any later version.
% 
% This program is distributed in the hope that it will be useful,
% but WITHOUT ANY WARRANTY; without even the implied warranty of
% MERCHANTABILITY or FITNESS FOR A PARTICULAR PURPOSE.  See the
% GNU General Public License for more details.
% 
% You should have received a copy of the GNU General Public License
% along with this program.  If not, see <http://www.gnu.org/licenses/>.
%
%
%%%%%%%%%%%%%%%%%%%%%%%%%%%%%%%%%%%%%%%%%%%%%%%%%%%%%%%%%%%%%%%%%%%%%%%%%%%%%
\documentclass[a4paper]{ltxdoc}

\usepackage{makeidx}

% DON't let hyperref overload the format of index and glossary. 
% I want to do that on my own in the stylefiles for makeindex...
\makeatletter
\let\@old@wrindex=\@wrindex
\makeatother

\usepackage{ifpdf}
\ifpdf
	\usepackage[pdftex]{hyperref}
\else
	\usepackage[dvipdfm]{hyperref}
\fi
\hypersetup{pdfborder={0 0 0}}

\makeatletter
\let\@wrindex=\@old@wrindex
\makeatother

% Formatiere Seitennummern im Index:
\newcommand{\indexpageno}[1]{%
	{\bfseries\hyperpage{#1}}%
}


\newcommand{\R}{\mathbb{R}}
\newcommand{\N}{\mathbb{N}}
\newcommand{\Z}{\mathbb{Z}}

\long\def\COMMENTLOWLEVEL#1\ENDCOMMENT{}
\def\ENDCOMMENT{}

\usepackage{textcomp}
\usepackage{booktabs}

\usepackage{calc}
\usepackage[formats]{listings}
%\usepackage{courier} % don't use it - the '^' character can't be copy-pasted in courier

\usepackage{array}
\lstset{%
	basicstyle=\ttfamily,
	language=[LaTeX]tex, % Seems as if \lstset{language=tex} must be invoked BEFORE loading tikz!?
	tabsize=4,
	breaklines=true,
	breakindent=0pt
}

\ifpdf
	\pdfinfo {
		/Author	(Christian Feuersaenger)
	}

\else
	\def\pgfsysdriver{pgfsys-dvipdfm.def}
\fi
%\def\pgfsysdriver{pgfsys-pdftex.def}
\usepackage{pgfplots}

\ifpdf
	\usetikzlibrary{pgfplotsclickable}
\fi

%\usepackage{fp}
% ATTENTION:
% this requires pgf version NEWER than 2.00 :
%\usetikzlibrary{fixedpointarithmetic}

\usetikzlibrary{dateplot}

\usepackage[a4paper,left=2.25cm,right=2.25cm,top=2.5cm,bottom=2.5cm,nohead]{geometry}
\usepackage{amsmath,amssymb}
\usepackage{xxcolor}
\usepackage{pifont}
\usepackage[latin1]{inputenc}
\usepackage{amsmath}
\usepackage{eurosym}
\input{pgfplots-macros}

\pgfqkeys{/codeexample}{%
	every codeexample/.style={
		width=8cm,
		/pgfplots/every axis/.append style={legend style={fill=graphicbackground}}
	},
	tabsize=4,
}
\pgfplotsset{
	every axis/.append style={width=7cm},
	filter discard warning=false,
}

\usetikzlibrary{backgrounds,patterns}
% Global styles:
\tikzset{
  shape example/.style={
    color=black!30,
    draw,
    fill=yellow!30,
    line width=.5cm,
    inner xsep=2.5cm,
    inner ysep=0.5cm}
}

\newcommand{\FIXME}[1]{\textcolor{red}{(FIXME: #1)}}

% fuer endvironment 'sidewaysfigure' bspw
% \usepackage{rotating}

\newcommand\Tikz{Ti\textit kZ}
\newcommand\PGF{\textsc{pgf}}
\newcommand\PGFPlots{\textsc{pgfplots}}
\def\PGFPlotstable{\textsc{PgfplotsTable}}


\makeindex

% Fix overful hboxes automatically:
\tolerance=2000
\emergencystretch=10pt

\tikzset{prefix=gnuplot/pgfplots_} % prefix for 'plot function'

\author{%
	Christian Feuers\"anger\footnote{\url{http://wissrech.ins.uni-bonn.de/people/feuersaenger}}\\%
	Institut f\"ur Numerische Simulation\\
	Universit\"at Bonn, Germany}


%\RequirePackage[german,english,francais]{babel}

\pgfplotsmanualexternalexpensivetrue

%\usetikzlibrary{external}
%\tikzexternalize[prefix=figures/]{pgfplots}

\title{%
	Manual for Package \PGFPlots\\
	{\small Version \pgfplotsversion}\\
	{\small\href{http://sourceforge.net/projects/pgfplots}{http://sourceforge.net/projects/pgfplots}}}

%\includeonly{pgfplots.libs}


\begin{document}

\def\plotcoords{%
\addplot coordinates {
(5,8.312e-02)    (17,2.547e-02)   (49,7.407e-03)
(129,2.102e-03)  (321,5.874e-04)  (769,1.623e-04)
(1793,4.442e-05) (4097,1.207e-05) (9217,3.261e-06)
};

\addplot coordinates{
(7,8.472e-02)    (31,3.044e-02)    (111,1.022e-02)
(351,3.303e-03)  (1023,1.039e-03)  (2815,3.196e-04)
(7423,9.658e-05) (18943,2.873e-05) (47103,8.437e-06)
};

\addplot coordinates{
(9,7.881e-02)     (49,3.243e-02)    (209,1.232e-02)
(769,4.454e-03)   (2561,1.551e-03)  (7937,5.236e-04)
(23297,1.723e-04) (65537,5.545e-05) (178177,1.751e-05)
};

\addplot coordinates{
(11,6.887e-02)    (71,3.177e-02)     (351,1.341e-02)
(1471,5.334e-03)  (5503,2.027e-03)   (18943,7.415e-04)
(61183,2.628e-04) (187903,9.063e-05) (553983,3.053e-05)
};

\addplot coordinates{
(13,5.755e-02)     (97,2.925e-02)     (545,1.351e-02)
(2561,5.842e-03)   (10625,2.397e-03)  (40193,9.414e-04)
(141569,3.564e-04) (471041,1.308e-04) 
(1496065,4.670e-05)
};
}%
\maketitle
\begin{abstract}%
\PGFPlots\ draws high--quality function plots in normal or logarithmic scaling with a user-friendly interface directly in \TeX. The user supplies axis labels, legend entries and the plot coordinates for one or more plots and \PGFPlots\ applies axis scaling, computes any logarithms and axis ticks and draws the plots, supporting line plots, scatter plots, piecewise constant plots, bar plots, area plots, mesh-- and surface plots and some more. It is based on Till Tantau's package \PGF/\Tikz.
\end{abstract}
\tableofcontents
\section{Introduction}
This package provides tools to generate plots and labeled axes easily. It draws normal plots, logplots and semi-logplots, in two and three dimensions. Axis ticks, labels, legends (in case of multiple plots) can be added with key-value options. It can cycle through a set of predefined line/marker/color specifications. In summary, its purpose is to simplify the generation of high-quality function and/or data plots, and solving the problems of
\begin{itemize}
	\item consistency of document and font type and font size,
	\item direct use of \TeX\ math mode in axis descriptions,
	\item consistency of data and figures (no third party tool necessary),
	\item inter document consistency using preamble configurations and styles.
\end{itemize}
Although not necessary, separate |.pdf| or |.eps| graphics can be generated using the |external| library developed as part of \Tikz.

\PGFPlots\ is build completely on \Tikz/\PGF. Knowledge of \Tikz\ will simplify the work with \PGFPlots, although it is not required.

\section[About PGFPlots: Preliminaries]{About {\normalfont\PGFPlots}: Preliminaries}
This section contains information about upgrades, the team, the installation (in case you need to do it manually) and troubleshooting. You may skip it completely except for the upgrade remarks.

However, note that this library requires at least \PGF\ version $2.00$. At the time of this writing, many \TeX-distributions still contain the older \PGF\ version $1.18$, so it may be necessary to install a recent \PGF\ prior to using \PGFPlots.

\subsection{Components}
\PGFPlots\ comes with two components:
\begin{enumerate}
	\item the plotting component (which you are currently reading) and
	\item the \PGFPlotstable\ component which simplifies number formatting and postprocessing of numerical tables. It comes as a separate package and has its own manual \href{file:pgfplotstable.pdf}{pgfplotstable.pdf}.
\end{enumerate}

\subsection{Upgrade remarks}
This release provides a lot of improvements which can be found in all detail in \texttt{ChangeLog} for interested readers. However, some attention is useful with respect to the following changes.

\subsubsection{New Optional Features}
\begin{enumerate}
	\item \PGFPlots\ 1.3 comes with user interface improvements. The technical distinction between ``behavior options'' and ``style options'' of older versions is no longer necessary (although still fully supported).

	\item \PGFPlots\ 1.3 has a new feature which allows to \emph{move axis labels tight to tick labels} automatically.

	Since this affects the spacing, it is not enabled be default.

	Use
\begin{codeexample}[code only]
\usepackage{pgfplots}
\pgfplotsset{compat=1.3}
\end{codeexample}
	in your preamble to benefit from the improved spacing\footnote{The |compat=1.3| setting is necessary for new features which move axis labels around like reversed axis. However, \PGFPlots\ will still work as it does in earlier versions, your documents will remain the same if you don't set it explicitly.}.
	Take a look at the next page for the precise definition of |compat|.

	\item \PGFPlots\ 1.3 now supports reversed axes. It is no longer necessary to use work--arounds with negative units.
\pgfkeys{/pdflinks/search key prefixes in/.add={/pgfplots/,}{}}

	Take a look at the |x dir=reverse| key.

	Existing work--arounds will still function properly. Use |\pgfplotsset{compat=1.3}| together with |x dir=reverse| to switch to the new version.
\end{enumerate}

\subsubsection{Old Features Which May Need Attention}
\begin{enumerate}
	\item The |scatter/classes| feature produces proper legends as of version 1.3. This may change the appearance of existing legends of plots with |scatter/classes|.

	\item Due to a small math bug in \PGF\ $2.00$, you \emph{can't} use the math expression `|-x^2|'. It is necessary to use `|0-x^2|' instead. The same holds for
		`|exp(-x^2)|'; use `|exp(0-x^2)|' instead.

		This will be fixed with \PGF\ $>2.00$.

	\item Starting with \PGFPlots\ $1.1$, |\tikzstyle| should \emph{no longer be used} to set \PGFPlots\ options.
	
	Although |\tikzstyle| is still supported for some older \PGFPlots\ options, you should replace any occurance of |\tikzstyle| with |\pgfplotsset{|\meta{style name}|/.style={|\meta{key-value-list}|}}| or the associated |/.append style| variant. See section~\ref{sec:styles} for more detail.
\end{enumerate}
I apologize for any inconvenience caused by these changes.

\begin{pgfplotskey}{compat=\mchoice{1.3,pre 1.3,default,newest} (initially default)}
	The preamble configuration 
\begin{codeexample}[code only]
\usepackage{pgfplots}
\pgfplotsset{compat=1.3}
\end{codeexample}
	allows to choose between backwards compatibility and most recent features.

	\PGFPlots\ version 1.3 comes with new features which may lead to different spacings compared to earlier versions:
	\begin{enumerate}
		\item Version 1.3 comes with the |xlabel near ticks| feature which places (in this case $x$) axis labels ``near'' the tick labels, taking the size of tick labels into account.
		
		Older versions used |xlabel absolute|, i.e.\ an absolute distance between the axis and axis labels (now matter how large or small tick labels are).

		The initial setting \emph{keeps backwards compatibility}.

		You are encouraged to use
\begin{codeexample}[code only]
\usepackage{pgfplots}
\pgfplotsset{compat=1.3}
\end{codeexample}
		in your preamble.
		
		\item Version 1.3 fixes a bug with |anchor=south west| which occurred mainly for reversed axes: the anchors where upside--down. This is fixed now.

		
		If you have written some work--around and want to keep it going, use

		|\pgfplotsset{compat/anchors=pre 1.3}|\pgfmanualpdflabel{/pgfplots/compat/anchors}{} 

		to disable the bug fix.
	\end{enumerate}

	Use |\pgfplotsset{compat=default}| to restore the factory settings.

	The setting |\pgfplotsset{compat=newest}| will always use features of the most recent version. This might result in small changes in the document's appearance.
\end{pgfplotskey}

\subsection{The Team}
\PGFPlots\ has been written mainly by Christian Feuers�nger with many improvements of Pascal Wolkotte and Nick Papior Andersen as a spare time project. We hope it is useful and provides valuable plots.

If you are interested in writing something but don't know how, consider reading the auxiliary manual \href{file:TeX-programming-notes.pdf}{TeX-programming-notes.pdf} which comes with \PGFPlots. It is far from complete, but maybe it is a good starting point (at least for more literature).

\subsection{Acknowledgements}
I thank God for all hours of enjoyed programming. I thank Pascal Wolkotte and Nick Papior Andersen for their programming efforts and contributions as part of the development team. I thank Stefan Tibus, who contributed the |plot shell| feature. I thank Tom Cashman for the contribution of the |reverse legend| feature. Special thanks go to Stefan Pinnow whose tests of \PGFPlots\ lead to numerous quality improvements. Furthermore, I thank Dr.~Schweitzer for many fruitful discussions and Dr.~Meine for his ideas and suggestions. Thanks as well to the many international contributors who provided feature requests or identified bugs or simply manual improvements!

Last but not least, I thank Till Tantau and Mark Wibrow for their excellent graphics (and more) package \PGF\ and \Tikz\ which is the base of \PGFPlots.

%%%%%%%%%%%%%%%%%%%%%%%%%%%%%%%%%%%%%%%%%%%%%%%%%%%%%%%%%%%%%%%%%%%%%%%%%%%%%
%
% Package pgfplots.sty documentation. 
%
% Copyright 2007/2008 by Christian Feuersaenger.
%
% This program is free software: you can redistribute it and/or modify
% it under the terms of the GNU General Public License as published by
% the Free Software Foundation, either version 3 of the License, or
% (at your option) any later version.
% 
% This program is distributed in the hope that it will be useful,
% but WITHOUT ANY WARRANTY; without even the implied warranty of
% MERCHANTABILITY or FITNESS FOR A PARTICULAR PURPOSE.  See the
% GNU General Public License for more details.
% 
% You should have received a copy of the GNU General Public License
% along with this program.  If not, see <http://www.gnu.org/licenses/>.
%
%
%%%%%%%%%%%%%%%%%%%%%%%%%%%%%%%%%%%%%%%%%%%%%%%%%%%%%%%%%%%%%%%%%%%%%%%%%%%%%
\input{pgfplots.preamble.tex}
%\RequirePackage[german,english,francais]{babel}

\pgfplotsmanualexternalexpensivetrue

%\usetikzlibrary{external}
%\tikzexternalize[prefix=figures/]{pgfplots}

\title{%
	Manual for Package \PGFPlots\\
	{\small Version \pgfplotsversion}\\
	{\small\href{http://sourceforge.net/projects/pgfplots}{http://sourceforge.net/projects/pgfplots}}}

%\includeonly{pgfplots.libs}


\begin{document}

\def\plotcoords{%
\addplot coordinates {
(5,8.312e-02)    (17,2.547e-02)   (49,7.407e-03)
(129,2.102e-03)  (321,5.874e-04)  (769,1.623e-04)
(1793,4.442e-05) (4097,1.207e-05) (9217,3.261e-06)
};

\addplot coordinates{
(7,8.472e-02)    (31,3.044e-02)    (111,1.022e-02)
(351,3.303e-03)  (1023,1.039e-03)  (2815,3.196e-04)
(7423,9.658e-05) (18943,2.873e-05) (47103,8.437e-06)
};

\addplot coordinates{
(9,7.881e-02)     (49,3.243e-02)    (209,1.232e-02)
(769,4.454e-03)   (2561,1.551e-03)  (7937,5.236e-04)
(23297,1.723e-04) (65537,5.545e-05) (178177,1.751e-05)
};

\addplot coordinates{
(11,6.887e-02)    (71,3.177e-02)     (351,1.341e-02)
(1471,5.334e-03)  (5503,2.027e-03)   (18943,7.415e-04)
(61183,2.628e-04) (187903,9.063e-05) (553983,3.053e-05)
};

\addplot coordinates{
(13,5.755e-02)     (97,2.925e-02)     (545,1.351e-02)
(2561,5.842e-03)   (10625,2.397e-03)  (40193,9.414e-04)
(141569,3.564e-04) (471041,1.308e-04) 
(1496065,4.670e-05)
};
}%
\maketitle
\begin{abstract}%
\PGFPlots\ draws high--quality function plots in normal or logarithmic scaling with a user-friendly interface directly in \TeX. The user supplies axis labels, legend entries and the plot coordinates for one or more plots and \PGFPlots\ applies axis scaling, computes any logarithms and axis ticks and draws the plots, supporting line plots, scatter plots, piecewise constant plots, bar plots, area plots, mesh-- and surface plots and some more. It is based on Till Tantau's package \PGF/\Tikz.
\end{abstract}
\tableofcontents
\section{Introduction}
This package provides tools to generate plots and labeled axes easily. It draws normal plots, logplots and semi-logplots, in two and three dimensions. Axis ticks, labels, legends (in case of multiple plots) can be added with key-value options. It can cycle through a set of predefined line/marker/color specifications. In summary, its purpose is to simplify the generation of high-quality function and/or data plots, and solving the problems of
\begin{itemize}
	\item consistency of document and font type and font size,
	\item direct use of \TeX\ math mode in axis descriptions,
	\item consistency of data and figures (no third party tool necessary),
	\item inter document consistency using preamble configurations and styles.
\end{itemize}
Although not necessary, separate |.pdf| or |.eps| graphics can be generated using the |external| library developed as part of \Tikz.

\PGFPlots\ is build completely on \Tikz/\PGF. Knowledge of \Tikz\ will simplify the work with \PGFPlots, although it is not required.

\section[About PGFPlots: Preliminaries]{About {\normalfont\PGFPlots}: Preliminaries}
This section contains information about upgrades, the team, the installation (in case you need to do it manually) and troubleshooting. You may skip it completely except for the upgrade remarks.

However, note that this library requires at least \PGF\ version $2.00$. At the time of this writing, many \TeX-distributions still contain the older \PGF\ version $1.18$, so it may be necessary to install a recent \PGF\ prior to using \PGFPlots.

\subsection{Components}
\PGFPlots\ comes with two components:
\begin{enumerate}
	\item the plotting component (which you are currently reading) and
	\item the \PGFPlotstable\ component which simplifies number formatting and postprocessing of numerical tables. It comes as a separate package and has its own manual \href{file:pgfplotstable.pdf}{pgfplotstable.pdf}.
\end{enumerate}

\subsection{Upgrade remarks}
This release provides a lot of improvements which can be found in all detail in \texttt{ChangeLog} for interested readers. However, some attention is useful with respect to the following changes.

\subsubsection{New Optional Features}
\begin{enumerate}
	\item \PGFPlots\ 1.3 comes with user interface improvements. The technical distinction between ``behavior options'' and ``style options'' of older versions is no longer necessary (although still fully supported).

	\item \PGFPlots\ 1.3 has a new feature which allows to \emph{move axis labels tight to tick labels} automatically.

	Since this affects the spacing, it is not enabled be default.

	Use
\begin{codeexample}[code only]
\usepackage{pgfplots}
\pgfplotsset{compat=1.3}
\end{codeexample}
	in your preamble to benefit from the improved spacing\footnote{The |compat=1.3| setting is necessary for new features which move axis labels around like reversed axis. However, \PGFPlots\ will still work as it does in earlier versions, your documents will remain the same if you don't set it explicitly.}.
	Take a look at the next page for the precise definition of |compat|.

	\item \PGFPlots\ 1.3 now supports reversed axes. It is no longer necessary to use work--arounds with negative units.
\pgfkeys{/pdflinks/search key prefixes in/.add={/pgfplots/,}{}}

	Take a look at the |x dir=reverse| key.

	Existing work--arounds will still function properly. Use |\pgfplotsset{compat=1.3}| together with |x dir=reverse| to switch to the new version.
\end{enumerate}

\subsubsection{Old Features Which May Need Attention}
\begin{enumerate}
	\item The |scatter/classes| feature produces proper legends as of version 1.3. This may change the appearance of existing legends of plots with |scatter/classes|.

	\item Due to a small math bug in \PGF\ $2.00$, you \emph{can't} use the math expression `|-x^2|'. It is necessary to use `|0-x^2|' instead. The same holds for
		`|exp(-x^2)|'; use `|exp(0-x^2)|' instead.

		This will be fixed with \PGF\ $>2.00$.

	\item Starting with \PGFPlots\ $1.1$, |\tikzstyle| should \emph{no longer be used} to set \PGFPlots\ options.
	
	Although |\tikzstyle| is still supported for some older \PGFPlots\ options, you should replace any occurance of |\tikzstyle| with |\pgfplotsset{|\meta{style name}|/.style={|\meta{key-value-list}|}}| or the associated |/.append style| variant. See section~\ref{sec:styles} for more detail.
\end{enumerate}
I apologize for any inconvenience caused by these changes.

\begin{pgfplotskey}{compat=\mchoice{1.3,pre 1.3,default,newest} (initially default)}
	The preamble configuration 
\begin{codeexample}[code only]
\usepackage{pgfplots}
\pgfplotsset{compat=1.3}
\end{codeexample}
	allows to choose between backwards compatibility and most recent features.

	\PGFPlots\ version 1.3 comes with new features which may lead to different spacings compared to earlier versions:
	\begin{enumerate}
		\item Version 1.3 comes with the |xlabel near ticks| feature which places (in this case $x$) axis labels ``near'' the tick labels, taking the size of tick labels into account.
		
		Older versions used |xlabel absolute|, i.e.\ an absolute distance between the axis and axis labels (now matter how large or small tick labels are).

		The initial setting \emph{keeps backwards compatibility}.

		You are encouraged to use
\begin{codeexample}[code only]
\usepackage{pgfplots}
\pgfplotsset{compat=1.3}
\end{codeexample}
		in your preamble.
		
		\item Version 1.3 fixes a bug with |anchor=south west| which occurred mainly for reversed axes: the anchors where upside--down. This is fixed now.

		
		If you have written some work--around and want to keep it going, use

		|\pgfplotsset{compat/anchors=pre 1.3}|\pgfmanualpdflabel{/pgfplots/compat/anchors}{} 

		to disable the bug fix.
	\end{enumerate}

	Use |\pgfplotsset{compat=default}| to restore the factory settings.

	The setting |\pgfplotsset{compat=newest}| will always use features of the most recent version. This might result in small changes in the document's appearance.
\end{pgfplotskey}

\subsection{The Team}
\PGFPlots\ has been written mainly by Christian Feuers�nger with many improvements of Pascal Wolkotte and Nick Papior Andersen as a spare time project. We hope it is useful and provides valuable plots.

If you are interested in writing something but don't know how, consider reading the auxiliary manual \href{file:TeX-programming-notes.pdf}{TeX-programming-notes.pdf} which comes with \PGFPlots. It is far from complete, but maybe it is a good starting point (at least for more literature).

\subsection{Acknowledgements}
I thank God for all hours of enjoyed programming. I thank Pascal Wolkotte and Nick Papior Andersen for their programming efforts and contributions as part of the development team. I thank Stefan Tibus, who contributed the |plot shell| feature. I thank Tom Cashman for the contribution of the |reverse legend| feature. Special thanks go to Stefan Pinnow whose tests of \PGFPlots\ lead to numerous quality improvements. Furthermore, I thank Dr.~Schweitzer for many fruitful discussions and Dr.~Meine for his ideas and suggestions. Thanks as well to the many international contributors who provided feature requests or identified bugs or simply manual improvements!

Last but not least, I thank Till Tantau and Mark Wibrow for their excellent graphics (and more) package \PGF\ and \Tikz\ which is the base of \PGFPlots.

\include{pgfplots.install}%
\include{pgfplots.intro}%
\include{pgfplots.reference}%
\include{pgfplots.libs}%
\include{pgfplots.resources}%
\include{pgfplots.importexport}%
\include{pgfplots.basic.reference}%

\printindex

\bibliographystyle{abbrv} %gerapali} %gerabbrv} %gerunsrt.bst} %gerabbrv}% gerplain}
\nocite{pgfplotstable}
\nocite{programmingnotes}
\bibliography{pgfplots}
\end{document}
%
%%%%%%%%%%%%%%%%%%%%%%%%%%%%%%%%%%%%%%%%%%%%%%%%%%%%%%%%%%%%%%%%%%%%%%%%%%%%%
%
% Package pgfplots.sty documentation. 
%
% Copyright 2007/2008 by Christian Feuersaenger.
%
% This program is free software: you can redistribute it and/or modify
% it under the terms of the GNU General Public License as published by
% the Free Software Foundation, either version 3 of the License, or
% (at your option) any later version.
% 
% This program is distributed in the hope that it will be useful,
% but WITHOUT ANY WARRANTY; without even the implied warranty of
% MERCHANTABILITY or FITNESS FOR A PARTICULAR PURPOSE.  See the
% GNU General Public License for more details.
% 
% You should have received a copy of the GNU General Public License
% along with this program.  If not, see <http://www.gnu.org/licenses/>.
%
%
%%%%%%%%%%%%%%%%%%%%%%%%%%%%%%%%%%%%%%%%%%%%%%%%%%%%%%%%%%%%%%%%%%%%%%%%%%%%%
\input{pgfplots.preamble.tex}
%\RequirePackage[german,english,francais]{babel}

\pgfplotsmanualexternalexpensivetrue

%\usetikzlibrary{external}
%\tikzexternalize[prefix=figures/]{pgfplots}

\title{%
	Manual for Package \PGFPlots\\
	{\small Version \pgfplotsversion}\\
	{\small\href{http://sourceforge.net/projects/pgfplots}{http://sourceforge.net/projects/pgfplots}}}

%\includeonly{pgfplots.libs}


\begin{document}

\def\plotcoords{%
\addplot coordinates {
(5,8.312e-02)    (17,2.547e-02)   (49,7.407e-03)
(129,2.102e-03)  (321,5.874e-04)  (769,1.623e-04)
(1793,4.442e-05) (4097,1.207e-05) (9217,3.261e-06)
};

\addplot coordinates{
(7,8.472e-02)    (31,3.044e-02)    (111,1.022e-02)
(351,3.303e-03)  (1023,1.039e-03)  (2815,3.196e-04)
(7423,9.658e-05) (18943,2.873e-05) (47103,8.437e-06)
};

\addplot coordinates{
(9,7.881e-02)     (49,3.243e-02)    (209,1.232e-02)
(769,4.454e-03)   (2561,1.551e-03)  (7937,5.236e-04)
(23297,1.723e-04) (65537,5.545e-05) (178177,1.751e-05)
};

\addplot coordinates{
(11,6.887e-02)    (71,3.177e-02)     (351,1.341e-02)
(1471,5.334e-03)  (5503,2.027e-03)   (18943,7.415e-04)
(61183,2.628e-04) (187903,9.063e-05) (553983,3.053e-05)
};

\addplot coordinates{
(13,5.755e-02)     (97,2.925e-02)     (545,1.351e-02)
(2561,5.842e-03)   (10625,2.397e-03)  (40193,9.414e-04)
(141569,3.564e-04) (471041,1.308e-04) 
(1496065,4.670e-05)
};
}%
\maketitle
\begin{abstract}%
\PGFPlots\ draws high--quality function plots in normal or logarithmic scaling with a user-friendly interface directly in \TeX. The user supplies axis labels, legend entries and the plot coordinates for one or more plots and \PGFPlots\ applies axis scaling, computes any logarithms and axis ticks and draws the plots, supporting line plots, scatter plots, piecewise constant plots, bar plots, area plots, mesh-- and surface plots and some more. It is based on Till Tantau's package \PGF/\Tikz.
\end{abstract}
\tableofcontents
\section{Introduction}
This package provides tools to generate plots and labeled axes easily. It draws normal plots, logplots and semi-logplots, in two and three dimensions. Axis ticks, labels, legends (in case of multiple plots) can be added with key-value options. It can cycle through a set of predefined line/marker/color specifications. In summary, its purpose is to simplify the generation of high-quality function and/or data plots, and solving the problems of
\begin{itemize}
	\item consistency of document and font type and font size,
	\item direct use of \TeX\ math mode in axis descriptions,
	\item consistency of data and figures (no third party tool necessary),
	\item inter document consistency using preamble configurations and styles.
\end{itemize}
Although not necessary, separate |.pdf| or |.eps| graphics can be generated using the |external| library developed as part of \Tikz.

\PGFPlots\ is build completely on \Tikz/\PGF. Knowledge of \Tikz\ will simplify the work with \PGFPlots, although it is not required.

\section[About PGFPlots: Preliminaries]{About {\normalfont\PGFPlots}: Preliminaries}
This section contains information about upgrades, the team, the installation (in case you need to do it manually) and troubleshooting. You may skip it completely except for the upgrade remarks.

However, note that this library requires at least \PGF\ version $2.00$. At the time of this writing, many \TeX-distributions still contain the older \PGF\ version $1.18$, so it may be necessary to install a recent \PGF\ prior to using \PGFPlots.

\subsection{Components}
\PGFPlots\ comes with two components:
\begin{enumerate}
	\item the plotting component (which you are currently reading) and
	\item the \PGFPlotstable\ component which simplifies number formatting and postprocessing of numerical tables. It comes as a separate package and has its own manual \href{file:pgfplotstable.pdf}{pgfplotstable.pdf}.
\end{enumerate}

\subsection{Upgrade remarks}
This release provides a lot of improvements which can be found in all detail in \texttt{ChangeLog} for interested readers. However, some attention is useful with respect to the following changes.

\subsubsection{New Optional Features}
\begin{enumerate}
	\item \PGFPlots\ 1.3 comes with user interface improvements. The technical distinction between ``behavior options'' and ``style options'' of older versions is no longer necessary (although still fully supported).

	\item \PGFPlots\ 1.3 has a new feature which allows to \emph{move axis labels tight to tick labels} automatically.

	Since this affects the spacing, it is not enabled be default.

	Use
\begin{codeexample}[code only]
\usepackage{pgfplots}
\pgfplotsset{compat=1.3}
\end{codeexample}
	in your preamble to benefit from the improved spacing\footnote{The |compat=1.3| setting is necessary for new features which move axis labels around like reversed axis. However, \PGFPlots\ will still work as it does in earlier versions, your documents will remain the same if you don't set it explicitly.}.
	Take a look at the next page for the precise definition of |compat|.

	\item \PGFPlots\ 1.3 now supports reversed axes. It is no longer necessary to use work--arounds with negative units.
\pgfkeys{/pdflinks/search key prefixes in/.add={/pgfplots/,}{}}

	Take a look at the |x dir=reverse| key.

	Existing work--arounds will still function properly. Use |\pgfplotsset{compat=1.3}| together with |x dir=reverse| to switch to the new version.
\end{enumerate}

\subsubsection{Old Features Which May Need Attention}
\begin{enumerate}
	\item The |scatter/classes| feature produces proper legends as of version 1.3. This may change the appearance of existing legends of plots with |scatter/classes|.

	\item Due to a small math bug in \PGF\ $2.00$, you \emph{can't} use the math expression `|-x^2|'. It is necessary to use `|0-x^2|' instead. The same holds for
		`|exp(-x^2)|'; use `|exp(0-x^2)|' instead.

		This will be fixed with \PGF\ $>2.00$.

	\item Starting with \PGFPlots\ $1.1$, |\tikzstyle| should \emph{no longer be used} to set \PGFPlots\ options.
	
	Although |\tikzstyle| is still supported for some older \PGFPlots\ options, you should replace any occurance of |\tikzstyle| with |\pgfplotsset{|\meta{style name}|/.style={|\meta{key-value-list}|}}| or the associated |/.append style| variant. See section~\ref{sec:styles} for more detail.
\end{enumerate}
I apologize for any inconvenience caused by these changes.

\begin{pgfplotskey}{compat=\mchoice{1.3,pre 1.3,default,newest} (initially default)}
	The preamble configuration 
\begin{codeexample}[code only]
\usepackage{pgfplots}
\pgfplotsset{compat=1.3}
\end{codeexample}
	allows to choose between backwards compatibility and most recent features.

	\PGFPlots\ version 1.3 comes with new features which may lead to different spacings compared to earlier versions:
	\begin{enumerate}
		\item Version 1.3 comes with the |xlabel near ticks| feature which places (in this case $x$) axis labels ``near'' the tick labels, taking the size of tick labels into account.
		
		Older versions used |xlabel absolute|, i.e.\ an absolute distance between the axis and axis labels (now matter how large or small tick labels are).

		The initial setting \emph{keeps backwards compatibility}.

		You are encouraged to use
\begin{codeexample}[code only]
\usepackage{pgfplots}
\pgfplotsset{compat=1.3}
\end{codeexample}
		in your preamble.
		
		\item Version 1.3 fixes a bug with |anchor=south west| which occurred mainly for reversed axes: the anchors where upside--down. This is fixed now.

		
		If you have written some work--around and want to keep it going, use

		|\pgfplotsset{compat/anchors=pre 1.3}|\pgfmanualpdflabel{/pgfplots/compat/anchors}{} 

		to disable the bug fix.
	\end{enumerate}

	Use |\pgfplotsset{compat=default}| to restore the factory settings.

	The setting |\pgfplotsset{compat=newest}| will always use features of the most recent version. This might result in small changes in the document's appearance.
\end{pgfplotskey}

\subsection{The Team}
\PGFPlots\ has been written mainly by Christian Feuers�nger with many improvements of Pascal Wolkotte and Nick Papior Andersen as a spare time project. We hope it is useful and provides valuable plots.

If you are interested in writing something but don't know how, consider reading the auxiliary manual \href{file:TeX-programming-notes.pdf}{TeX-programming-notes.pdf} which comes with \PGFPlots. It is far from complete, but maybe it is a good starting point (at least for more literature).

\subsection{Acknowledgements}
I thank God for all hours of enjoyed programming. I thank Pascal Wolkotte and Nick Papior Andersen for their programming efforts and contributions as part of the development team. I thank Stefan Tibus, who contributed the |plot shell| feature. I thank Tom Cashman for the contribution of the |reverse legend| feature. Special thanks go to Stefan Pinnow whose tests of \PGFPlots\ lead to numerous quality improvements. Furthermore, I thank Dr.~Schweitzer for many fruitful discussions and Dr.~Meine for his ideas and suggestions. Thanks as well to the many international contributors who provided feature requests or identified bugs or simply manual improvements!

Last but not least, I thank Till Tantau and Mark Wibrow for their excellent graphics (and more) package \PGF\ and \Tikz\ which is the base of \PGFPlots.

\include{pgfplots.install}%
\include{pgfplots.intro}%
\include{pgfplots.reference}%
\include{pgfplots.libs}%
\include{pgfplots.resources}%
\include{pgfplots.importexport}%
\include{pgfplots.basic.reference}%

\printindex

\bibliographystyle{abbrv} %gerapali} %gerabbrv} %gerunsrt.bst} %gerabbrv}% gerplain}
\nocite{pgfplotstable}
\nocite{programmingnotes}
\bibliography{pgfplots}
\end{document}
%
%%%%%%%%%%%%%%%%%%%%%%%%%%%%%%%%%%%%%%%%%%%%%%%%%%%%%%%%%%%%%%%%%%%%%%%%%%%%%
%
% Package pgfplots.sty documentation. 
%
% Copyright 2007/2008 by Christian Feuersaenger.
%
% This program is free software: you can redistribute it and/or modify
% it under the terms of the GNU General Public License as published by
% the Free Software Foundation, either version 3 of the License, or
% (at your option) any later version.
% 
% This program is distributed in the hope that it will be useful,
% but WITHOUT ANY WARRANTY; without even the implied warranty of
% MERCHANTABILITY or FITNESS FOR A PARTICULAR PURPOSE.  See the
% GNU General Public License for more details.
% 
% You should have received a copy of the GNU General Public License
% along with this program.  If not, see <http://www.gnu.org/licenses/>.
%
%
%%%%%%%%%%%%%%%%%%%%%%%%%%%%%%%%%%%%%%%%%%%%%%%%%%%%%%%%%%%%%%%%%%%%%%%%%%%%%
\input{pgfplots.preamble.tex}
%\RequirePackage[german,english,francais]{babel}

\pgfplotsmanualexternalexpensivetrue

%\usetikzlibrary{external}
%\tikzexternalize[prefix=figures/]{pgfplots}

\title{%
	Manual for Package \PGFPlots\\
	{\small Version \pgfplotsversion}\\
	{\small\href{http://sourceforge.net/projects/pgfplots}{http://sourceforge.net/projects/pgfplots}}}

%\includeonly{pgfplots.libs}


\begin{document}

\def\plotcoords{%
\addplot coordinates {
(5,8.312e-02)    (17,2.547e-02)   (49,7.407e-03)
(129,2.102e-03)  (321,5.874e-04)  (769,1.623e-04)
(1793,4.442e-05) (4097,1.207e-05) (9217,3.261e-06)
};

\addplot coordinates{
(7,8.472e-02)    (31,3.044e-02)    (111,1.022e-02)
(351,3.303e-03)  (1023,1.039e-03)  (2815,3.196e-04)
(7423,9.658e-05) (18943,2.873e-05) (47103,8.437e-06)
};

\addplot coordinates{
(9,7.881e-02)     (49,3.243e-02)    (209,1.232e-02)
(769,4.454e-03)   (2561,1.551e-03)  (7937,5.236e-04)
(23297,1.723e-04) (65537,5.545e-05) (178177,1.751e-05)
};

\addplot coordinates{
(11,6.887e-02)    (71,3.177e-02)     (351,1.341e-02)
(1471,5.334e-03)  (5503,2.027e-03)   (18943,7.415e-04)
(61183,2.628e-04) (187903,9.063e-05) (553983,3.053e-05)
};

\addplot coordinates{
(13,5.755e-02)     (97,2.925e-02)     (545,1.351e-02)
(2561,5.842e-03)   (10625,2.397e-03)  (40193,9.414e-04)
(141569,3.564e-04) (471041,1.308e-04) 
(1496065,4.670e-05)
};
}%
\maketitle
\begin{abstract}%
\PGFPlots\ draws high--quality function plots in normal or logarithmic scaling with a user-friendly interface directly in \TeX. The user supplies axis labels, legend entries and the plot coordinates for one or more plots and \PGFPlots\ applies axis scaling, computes any logarithms and axis ticks and draws the plots, supporting line plots, scatter plots, piecewise constant plots, bar plots, area plots, mesh-- and surface plots and some more. It is based on Till Tantau's package \PGF/\Tikz.
\end{abstract}
\tableofcontents
\section{Introduction}
This package provides tools to generate plots and labeled axes easily. It draws normal plots, logplots and semi-logplots, in two and three dimensions. Axis ticks, labels, legends (in case of multiple plots) can be added with key-value options. It can cycle through a set of predefined line/marker/color specifications. In summary, its purpose is to simplify the generation of high-quality function and/or data plots, and solving the problems of
\begin{itemize}
	\item consistency of document and font type and font size,
	\item direct use of \TeX\ math mode in axis descriptions,
	\item consistency of data and figures (no third party tool necessary),
	\item inter document consistency using preamble configurations and styles.
\end{itemize}
Although not necessary, separate |.pdf| or |.eps| graphics can be generated using the |external| library developed as part of \Tikz.

\PGFPlots\ is build completely on \Tikz/\PGF. Knowledge of \Tikz\ will simplify the work with \PGFPlots, although it is not required.

\section[About PGFPlots: Preliminaries]{About {\normalfont\PGFPlots}: Preliminaries}
This section contains information about upgrades, the team, the installation (in case you need to do it manually) and troubleshooting. You may skip it completely except for the upgrade remarks.

However, note that this library requires at least \PGF\ version $2.00$. At the time of this writing, many \TeX-distributions still contain the older \PGF\ version $1.18$, so it may be necessary to install a recent \PGF\ prior to using \PGFPlots.

\subsection{Components}
\PGFPlots\ comes with two components:
\begin{enumerate}
	\item the plotting component (which you are currently reading) and
	\item the \PGFPlotstable\ component which simplifies number formatting and postprocessing of numerical tables. It comes as a separate package and has its own manual \href{file:pgfplotstable.pdf}{pgfplotstable.pdf}.
\end{enumerate}

\subsection{Upgrade remarks}
This release provides a lot of improvements which can be found in all detail in \texttt{ChangeLog} for interested readers. However, some attention is useful with respect to the following changes.

\subsubsection{New Optional Features}
\begin{enumerate}
	\item \PGFPlots\ 1.3 comes with user interface improvements. The technical distinction between ``behavior options'' and ``style options'' of older versions is no longer necessary (although still fully supported).

	\item \PGFPlots\ 1.3 has a new feature which allows to \emph{move axis labels tight to tick labels} automatically.

	Since this affects the spacing, it is not enabled be default.

	Use
\begin{codeexample}[code only]
\usepackage{pgfplots}
\pgfplotsset{compat=1.3}
\end{codeexample}
	in your preamble to benefit from the improved spacing\footnote{The |compat=1.3| setting is necessary for new features which move axis labels around like reversed axis. However, \PGFPlots\ will still work as it does in earlier versions, your documents will remain the same if you don't set it explicitly.}.
	Take a look at the next page for the precise definition of |compat|.

	\item \PGFPlots\ 1.3 now supports reversed axes. It is no longer necessary to use work--arounds with negative units.
\pgfkeys{/pdflinks/search key prefixes in/.add={/pgfplots/,}{}}

	Take a look at the |x dir=reverse| key.

	Existing work--arounds will still function properly. Use |\pgfplotsset{compat=1.3}| together with |x dir=reverse| to switch to the new version.
\end{enumerate}

\subsubsection{Old Features Which May Need Attention}
\begin{enumerate}
	\item The |scatter/classes| feature produces proper legends as of version 1.3. This may change the appearance of existing legends of plots with |scatter/classes|.

	\item Due to a small math bug in \PGF\ $2.00$, you \emph{can't} use the math expression `|-x^2|'. It is necessary to use `|0-x^2|' instead. The same holds for
		`|exp(-x^2)|'; use `|exp(0-x^2)|' instead.

		This will be fixed with \PGF\ $>2.00$.

	\item Starting with \PGFPlots\ $1.1$, |\tikzstyle| should \emph{no longer be used} to set \PGFPlots\ options.
	
	Although |\tikzstyle| is still supported for some older \PGFPlots\ options, you should replace any occurance of |\tikzstyle| with |\pgfplotsset{|\meta{style name}|/.style={|\meta{key-value-list}|}}| or the associated |/.append style| variant. See section~\ref{sec:styles} for more detail.
\end{enumerate}
I apologize for any inconvenience caused by these changes.

\begin{pgfplotskey}{compat=\mchoice{1.3,pre 1.3,default,newest} (initially default)}
	The preamble configuration 
\begin{codeexample}[code only]
\usepackage{pgfplots}
\pgfplotsset{compat=1.3}
\end{codeexample}
	allows to choose between backwards compatibility and most recent features.

	\PGFPlots\ version 1.3 comes with new features which may lead to different spacings compared to earlier versions:
	\begin{enumerate}
		\item Version 1.3 comes with the |xlabel near ticks| feature which places (in this case $x$) axis labels ``near'' the tick labels, taking the size of tick labels into account.
		
		Older versions used |xlabel absolute|, i.e.\ an absolute distance between the axis and axis labels (now matter how large or small tick labels are).

		The initial setting \emph{keeps backwards compatibility}.

		You are encouraged to use
\begin{codeexample}[code only]
\usepackage{pgfplots}
\pgfplotsset{compat=1.3}
\end{codeexample}
		in your preamble.
		
		\item Version 1.3 fixes a bug with |anchor=south west| which occurred mainly for reversed axes: the anchors where upside--down. This is fixed now.

		
		If you have written some work--around and want to keep it going, use

		|\pgfplotsset{compat/anchors=pre 1.3}|\pgfmanualpdflabel{/pgfplots/compat/anchors}{} 

		to disable the bug fix.
	\end{enumerate}

	Use |\pgfplotsset{compat=default}| to restore the factory settings.

	The setting |\pgfplotsset{compat=newest}| will always use features of the most recent version. This might result in small changes in the document's appearance.
\end{pgfplotskey}

\subsection{The Team}
\PGFPlots\ has been written mainly by Christian Feuers�nger with many improvements of Pascal Wolkotte and Nick Papior Andersen as a spare time project. We hope it is useful and provides valuable plots.

If you are interested in writing something but don't know how, consider reading the auxiliary manual \href{file:TeX-programming-notes.pdf}{TeX-programming-notes.pdf} which comes with \PGFPlots. It is far from complete, but maybe it is a good starting point (at least for more literature).

\subsection{Acknowledgements}
I thank God for all hours of enjoyed programming. I thank Pascal Wolkotte and Nick Papior Andersen for their programming efforts and contributions as part of the development team. I thank Stefan Tibus, who contributed the |plot shell| feature. I thank Tom Cashman for the contribution of the |reverse legend| feature. Special thanks go to Stefan Pinnow whose tests of \PGFPlots\ lead to numerous quality improvements. Furthermore, I thank Dr.~Schweitzer for many fruitful discussions and Dr.~Meine for his ideas and suggestions. Thanks as well to the many international contributors who provided feature requests or identified bugs or simply manual improvements!

Last but not least, I thank Till Tantau and Mark Wibrow for their excellent graphics (and more) package \PGF\ and \Tikz\ which is the base of \PGFPlots.

\include{pgfplots.install}%
\include{pgfplots.intro}%
\include{pgfplots.reference}%
\include{pgfplots.libs}%
\include{pgfplots.resources}%
\include{pgfplots.importexport}%
\include{pgfplots.basic.reference}%

\printindex

\bibliographystyle{abbrv} %gerapali} %gerabbrv} %gerunsrt.bst} %gerabbrv}% gerplain}
\nocite{pgfplotstable}
\nocite{programmingnotes}
\bibliography{pgfplots}
\end{document}
%
%%%%%%%%%%%%%%%%%%%%%%%%%%%%%%%%%%%%%%%%%%%%%%%%%%%%%%%%%%%%%%%%%%%%%%%%%%%%%
%
% Package pgfplots.sty documentation. 
%
% Copyright 2007/2008 by Christian Feuersaenger.
%
% This program is free software: you can redistribute it and/or modify
% it under the terms of the GNU General Public License as published by
% the Free Software Foundation, either version 3 of the License, or
% (at your option) any later version.
% 
% This program is distributed in the hope that it will be useful,
% but WITHOUT ANY WARRANTY; without even the implied warranty of
% MERCHANTABILITY or FITNESS FOR A PARTICULAR PURPOSE.  See the
% GNU General Public License for more details.
% 
% You should have received a copy of the GNU General Public License
% along with this program.  If not, see <http://www.gnu.org/licenses/>.
%
%
%%%%%%%%%%%%%%%%%%%%%%%%%%%%%%%%%%%%%%%%%%%%%%%%%%%%%%%%%%%%%%%%%%%%%%%%%%%%%
\input{pgfplots.preamble.tex}
%\RequirePackage[german,english,francais]{babel}

\pgfplotsmanualexternalexpensivetrue

%\usetikzlibrary{external}
%\tikzexternalize[prefix=figures/]{pgfplots}

\title{%
	Manual for Package \PGFPlots\\
	{\small Version \pgfplotsversion}\\
	{\small\href{http://sourceforge.net/projects/pgfplots}{http://sourceforge.net/projects/pgfplots}}}

%\includeonly{pgfplots.libs}


\begin{document}

\def\plotcoords{%
\addplot coordinates {
(5,8.312e-02)    (17,2.547e-02)   (49,7.407e-03)
(129,2.102e-03)  (321,5.874e-04)  (769,1.623e-04)
(1793,4.442e-05) (4097,1.207e-05) (9217,3.261e-06)
};

\addplot coordinates{
(7,8.472e-02)    (31,3.044e-02)    (111,1.022e-02)
(351,3.303e-03)  (1023,1.039e-03)  (2815,3.196e-04)
(7423,9.658e-05) (18943,2.873e-05) (47103,8.437e-06)
};

\addplot coordinates{
(9,7.881e-02)     (49,3.243e-02)    (209,1.232e-02)
(769,4.454e-03)   (2561,1.551e-03)  (7937,5.236e-04)
(23297,1.723e-04) (65537,5.545e-05) (178177,1.751e-05)
};

\addplot coordinates{
(11,6.887e-02)    (71,3.177e-02)     (351,1.341e-02)
(1471,5.334e-03)  (5503,2.027e-03)   (18943,7.415e-04)
(61183,2.628e-04) (187903,9.063e-05) (553983,3.053e-05)
};

\addplot coordinates{
(13,5.755e-02)     (97,2.925e-02)     (545,1.351e-02)
(2561,5.842e-03)   (10625,2.397e-03)  (40193,9.414e-04)
(141569,3.564e-04) (471041,1.308e-04) 
(1496065,4.670e-05)
};
}%
\maketitle
\begin{abstract}%
\PGFPlots\ draws high--quality function plots in normal or logarithmic scaling with a user-friendly interface directly in \TeX. The user supplies axis labels, legend entries and the plot coordinates for one or more plots and \PGFPlots\ applies axis scaling, computes any logarithms and axis ticks and draws the plots, supporting line plots, scatter plots, piecewise constant plots, bar plots, area plots, mesh-- and surface plots and some more. It is based on Till Tantau's package \PGF/\Tikz.
\end{abstract}
\tableofcontents
\section{Introduction}
This package provides tools to generate plots and labeled axes easily. It draws normal plots, logplots and semi-logplots, in two and three dimensions. Axis ticks, labels, legends (in case of multiple plots) can be added with key-value options. It can cycle through a set of predefined line/marker/color specifications. In summary, its purpose is to simplify the generation of high-quality function and/or data plots, and solving the problems of
\begin{itemize}
	\item consistency of document and font type and font size,
	\item direct use of \TeX\ math mode in axis descriptions,
	\item consistency of data and figures (no third party tool necessary),
	\item inter document consistency using preamble configurations and styles.
\end{itemize}
Although not necessary, separate |.pdf| or |.eps| graphics can be generated using the |external| library developed as part of \Tikz.

\PGFPlots\ is build completely on \Tikz/\PGF. Knowledge of \Tikz\ will simplify the work with \PGFPlots, although it is not required.

\section[About PGFPlots: Preliminaries]{About {\normalfont\PGFPlots}: Preliminaries}
This section contains information about upgrades, the team, the installation (in case you need to do it manually) and troubleshooting. You may skip it completely except for the upgrade remarks.

However, note that this library requires at least \PGF\ version $2.00$. At the time of this writing, many \TeX-distributions still contain the older \PGF\ version $1.18$, so it may be necessary to install a recent \PGF\ prior to using \PGFPlots.

\subsection{Components}
\PGFPlots\ comes with two components:
\begin{enumerate}
	\item the plotting component (which you are currently reading) and
	\item the \PGFPlotstable\ component which simplifies number formatting and postprocessing of numerical tables. It comes as a separate package and has its own manual \href{file:pgfplotstable.pdf}{pgfplotstable.pdf}.
\end{enumerate}

\subsection{Upgrade remarks}
This release provides a lot of improvements which can be found in all detail in \texttt{ChangeLog} for interested readers. However, some attention is useful with respect to the following changes.

\subsubsection{New Optional Features}
\begin{enumerate}
	\item \PGFPlots\ 1.3 comes with user interface improvements. The technical distinction between ``behavior options'' and ``style options'' of older versions is no longer necessary (although still fully supported).

	\item \PGFPlots\ 1.3 has a new feature which allows to \emph{move axis labels tight to tick labels} automatically.

	Since this affects the spacing, it is not enabled be default.

	Use
\begin{codeexample}[code only]
\usepackage{pgfplots}
\pgfplotsset{compat=1.3}
\end{codeexample}
	in your preamble to benefit from the improved spacing\footnote{The |compat=1.3| setting is necessary for new features which move axis labels around like reversed axis. However, \PGFPlots\ will still work as it does in earlier versions, your documents will remain the same if you don't set it explicitly.}.
	Take a look at the next page for the precise definition of |compat|.

	\item \PGFPlots\ 1.3 now supports reversed axes. It is no longer necessary to use work--arounds with negative units.
\pgfkeys{/pdflinks/search key prefixes in/.add={/pgfplots/,}{}}

	Take a look at the |x dir=reverse| key.

	Existing work--arounds will still function properly. Use |\pgfplotsset{compat=1.3}| together with |x dir=reverse| to switch to the new version.
\end{enumerate}

\subsubsection{Old Features Which May Need Attention}
\begin{enumerate}
	\item The |scatter/classes| feature produces proper legends as of version 1.3. This may change the appearance of existing legends of plots with |scatter/classes|.

	\item Due to a small math bug in \PGF\ $2.00$, you \emph{can't} use the math expression `|-x^2|'. It is necessary to use `|0-x^2|' instead. The same holds for
		`|exp(-x^2)|'; use `|exp(0-x^2)|' instead.

		This will be fixed with \PGF\ $>2.00$.

	\item Starting with \PGFPlots\ $1.1$, |\tikzstyle| should \emph{no longer be used} to set \PGFPlots\ options.
	
	Although |\tikzstyle| is still supported for some older \PGFPlots\ options, you should replace any occurance of |\tikzstyle| with |\pgfplotsset{|\meta{style name}|/.style={|\meta{key-value-list}|}}| or the associated |/.append style| variant. See section~\ref{sec:styles} for more detail.
\end{enumerate}
I apologize for any inconvenience caused by these changes.

\begin{pgfplotskey}{compat=\mchoice{1.3,pre 1.3,default,newest} (initially default)}
	The preamble configuration 
\begin{codeexample}[code only]
\usepackage{pgfplots}
\pgfplotsset{compat=1.3}
\end{codeexample}
	allows to choose between backwards compatibility and most recent features.

	\PGFPlots\ version 1.3 comes with new features which may lead to different spacings compared to earlier versions:
	\begin{enumerate}
		\item Version 1.3 comes with the |xlabel near ticks| feature which places (in this case $x$) axis labels ``near'' the tick labels, taking the size of tick labels into account.
		
		Older versions used |xlabel absolute|, i.e.\ an absolute distance between the axis and axis labels (now matter how large or small tick labels are).

		The initial setting \emph{keeps backwards compatibility}.

		You are encouraged to use
\begin{codeexample}[code only]
\usepackage{pgfplots}
\pgfplotsset{compat=1.3}
\end{codeexample}
		in your preamble.
		
		\item Version 1.3 fixes a bug with |anchor=south west| which occurred mainly for reversed axes: the anchors where upside--down. This is fixed now.

		
		If you have written some work--around and want to keep it going, use

		|\pgfplotsset{compat/anchors=pre 1.3}|\pgfmanualpdflabel{/pgfplots/compat/anchors}{} 

		to disable the bug fix.
	\end{enumerate}

	Use |\pgfplotsset{compat=default}| to restore the factory settings.

	The setting |\pgfplotsset{compat=newest}| will always use features of the most recent version. This might result in small changes in the document's appearance.
\end{pgfplotskey}

\subsection{The Team}
\PGFPlots\ has been written mainly by Christian Feuers�nger with many improvements of Pascal Wolkotte and Nick Papior Andersen as a spare time project. We hope it is useful and provides valuable plots.

If you are interested in writing something but don't know how, consider reading the auxiliary manual \href{file:TeX-programming-notes.pdf}{TeX-programming-notes.pdf} which comes with \PGFPlots. It is far from complete, but maybe it is a good starting point (at least for more literature).

\subsection{Acknowledgements}
I thank God for all hours of enjoyed programming. I thank Pascal Wolkotte and Nick Papior Andersen for their programming efforts and contributions as part of the development team. I thank Stefan Tibus, who contributed the |plot shell| feature. I thank Tom Cashman for the contribution of the |reverse legend| feature. Special thanks go to Stefan Pinnow whose tests of \PGFPlots\ lead to numerous quality improvements. Furthermore, I thank Dr.~Schweitzer for many fruitful discussions and Dr.~Meine for his ideas and suggestions. Thanks as well to the many international contributors who provided feature requests or identified bugs or simply manual improvements!

Last but not least, I thank Till Tantau and Mark Wibrow for their excellent graphics (and more) package \PGF\ and \Tikz\ which is the base of \PGFPlots.

\include{pgfplots.install}%
\include{pgfplots.intro}%
\include{pgfplots.reference}%
\include{pgfplots.libs}%
\include{pgfplots.resources}%
\include{pgfplots.importexport}%
\include{pgfplots.basic.reference}%

\printindex

\bibliographystyle{abbrv} %gerapali} %gerabbrv} %gerunsrt.bst} %gerabbrv}% gerplain}
\nocite{pgfplotstable}
\nocite{programmingnotes}
\bibliography{pgfplots}
\end{document}
%
%%%%%%%%%%%%%%%%%%%%%%%%%%%%%%%%%%%%%%%%%%%%%%%%%%%%%%%%%%%%%%%%%%%%%%%%%%%%%
%
% Package pgfplots.sty documentation. 
%
% Copyright 2007/2008 by Christian Feuersaenger.
%
% This program is free software: you can redistribute it and/or modify
% it under the terms of the GNU General Public License as published by
% the Free Software Foundation, either version 3 of the License, or
% (at your option) any later version.
% 
% This program is distributed in the hope that it will be useful,
% but WITHOUT ANY WARRANTY; without even the implied warranty of
% MERCHANTABILITY or FITNESS FOR A PARTICULAR PURPOSE.  See the
% GNU General Public License for more details.
% 
% You should have received a copy of the GNU General Public License
% along with this program.  If not, see <http://www.gnu.org/licenses/>.
%
%
%%%%%%%%%%%%%%%%%%%%%%%%%%%%%%%%%%%%%%%%%%%%%%%%%%%%%%%%%%%%%%%%%%%%%%%%%%%%%
\input{pgfplots.preamble.tex}
%\RequirePackage[german,english,francais]{babel}

\pgfplotsmanualexternalexpensivetrue

%\usetikzlibrary{external}
%\tikzexternalize[prefix=figures/]{pgfplots}

\title{%
	Manual for Package \PGFPlots\\
	{\small Version \pgfplotsversion}\\
	{\small\href{http://sourceforge.net/projects/pgfplots}{http://sourceforge.net/projects/pgfplots}}}

%\includeonly{pgfplots.libs}


\begin{document}

\def\plotcoords{%
\addplot coordinates {
(5,8.312e-02)    (17,2.547e-02)   (49,7.407e-03)
(129,2.102e-03)  (321,5.874e-04)  (769,1.623e-04)
(1793,4.442e-05) (4097,1.207e-05) (9217,3.261e-06)
};

\addplot coordinates{
(7,8.472e-02)    (31,3.044e-02)    (111,1.022e-02)
(351,3.303e-03)  (1023,1.039e-03)  (2815,3.196e-04)
(7423,9.658e-05) (18943,2.873e-05) (47103,8.437e-06)
};

\addplot coordinates{
(9,7.881e-02)     (49,3.243e-02)    (209,1.232e-02)
(769,4.454e-03)   (2561,1.551e-03)  (7937,5.236e-04)
(23297,1.723e-04) (65537,5.545e-05) (178177,1.751e-05)
};

\addplot coordinates{
(11,6.887e-02)    (71,3.177e-02)     (351,1.341e-02)
(1471,5.334e-03)  (5503,2.027e-03)   (18943,7.415e-04)
(61183,2.628e-04) (187903,9.063e-05) (553983,3.053e-05)
};

\addplot coordinates{
(13,5.755e-02)     (97,2.925e-02)     (545,1.351e-02)
(2561,5.842e-03)   (10625,2.397e-03)  (40193,9.414e-04)
(141569,3.564e-04) (471041,1.308e-04) 
(1496065,4.670e-05)
};
}%
\maketitle
\begin{abstract}%
\PGFPlots\ draws high--quality function plots in normal or logarithmic scaling with a user-friendly interface directly in \TeX. The user supplies axis labels, legend entries and the plot coordinates for one or more plots and \PGFPlots\ applies axis scaling, computes any logarithms and axis ticks and draws the plots, supporting line plots, scatter plots, piecewise constant plots, bar plots, area plots, mesh-- and surface plots and some more. It is based on Till Tantau's package \PGF/\Tikz.
\end{abstract}
\tableofcontents
\section{Introduction}
This package provides tools to generate plots and labeled axes easily. It draws normal plots, logplots and semi-logplots, in two and three dimensions. Axis ticks, labels, legends (in case of multiple plots) can be added with key-value options. It can cycle through a set of predefined line/marker/color specifications. In summary, its purpose is to simplify the generation of high-quality function and/or data plots, and solving the problems of
\begin{itemize}
	\item consistency of document and font type and font size,
	\item direct use of \TeX\ math mode in axis descriptions,
	\item consistency of data and figures (no third party tool necessary),
	\item inter document consistency using preamble configurations and styles.
\end{itemize}
Although not necessary, separate |.pdf| or |.eps| graphics can be generated using the |external| library developed as part of \Tikz.

\PGFPlots\ is build completely on \Tikz/\PGF. Knowledge of \Tikz\ will simplify the work with \PGFPlots, although it is not required.

\section[About PGFPlots: Preliminaries]{About {\normalfont\PGFPlots}: Preliminaries}
This section contains information about upgrades, the team, the installation (in case you need to do it manually) and troubleshooting. You may skip it completely except for the upgrade remarks.

However, note that this library requires at least \PGF\ version $2.00$. At the time of this writing, many \TeX-distributions still contain the older \PGF\ version $1.18$, so it may be necessary to install a recent \PGF\ prior to using \PGFPlots.

\subsection{Components}
\PGFPlots\ comes with two components:
\begin{enumerate}
	\item the plotting component (which you are currently reading) and
	\item the \PGFPlotstable\ component which simplifies number formatting and postprocessing of numerical tables. It comes as a separate package and has its own manual \href{file:pgfplotstable.pdf}{pgfplotstable.pdf}.
\end{enumerate}

\subsection{Upgrade remarks}
This release provides a lot of improvements which can be found in all detail in \texttt{ChangeLog} for interested readers. However, some attention is useful with respect to the following changes.

\subsubsection{New Optional Features}
\begin{enumerate}
	\item \PGFPlots\ 1.3 comes with user interface improvements. The technical distinction between ``behavior options'' and ``style options'' of older versions is no longer necessary (although still fully supported).

	\item \PGFPlots\ 1.3 has a new feature which allows to \emph{move axis labels tight to tick labels} automatically.

	Since this affects the spacing, it is not enabled be default.

	Use
\begin{codeexample}[code only]
\usepackage{pgfplots}
\pgfplotsset{compat=1.3}
\end{codeexample}
	in your preamble to benefit from the improved spacing\footnote{The |compat=1.3| setting is necessary for new features which move axis labels around like reversed axis. However, \PGFPlots\ will still work as it does in earlier versions, your documents will remain the same if you don't set it explicitly.}.
	Take a look at the next page for the precise definition of |compat|.

	\item \PGFPlots\ 1.3 now supports reversed axes. It is no longer necessary to use work--arounds with negative units.
\pgfkeys{/pdflinks/search key prefixes in/.add={/pgfplots/,}{}}

	Take a look at the |x dir=reverse| key.

	Existing work--arounds will still function properly. Use |\pgfplotsset{compat=1.3}| together with |x dir=reverse| to switch to the new version.
\end{enumerate}

\subsubsection{Old Features Which May Need Attention}
\begin{enumerate}
	\item The |scatter/classes| feature produces proper legends as of version 1.3. This may change the appearance of existing legends of plots with |scatter/classes|.

	\item Due to a small math bug in \PGF\ $2.00$, you \emph{can't} use the math expression `|-x^2|'. It is necessary to use `|0-x^2|' instead. The same holds for
		`|exp(-x^2)|'; use `|exp(0-x^2)|' instead.

		This will be fixed with \PGF\ $>2.00$.

	\item Starting with \PGFPlots\ $1.1$, |\tikzstyle| should \emph{no longer be used} to set \PGFPlots\ options.
	
	Although |\tikzstyle| is still supported for some older \PGFPlots\ options, you should replace any occurance of |\tikzstyle| with |\pgfplotsset{|\meta{style name}|/.style={|\meta{key-value-list}|}}| or the associated |/.append style| variant. See section~\ref{sec:styles} for more detail.
\end{enumerate}
I apologize for any inconvenience caused by these changes.

\begin{pgfplotskey}{compat=\mchoice{1.3,pre 1.3,default,newest} (initially default)}
	The preamble configuration 
\begin{codeexample}[code only]
\usepackage{pgfplots}
\pgfplotsset{compat=1.3}
\end{codeexample}
	allows to choose between backwards compatibility and most recent features.

	\PGFPlots\ version 1.3 comes with new features which may lead to different spacings compared to earlier versions:
	\begin{enumerate}
		\item Version 1.3 comes with the |xlabel near ticks| feature which places (in this case $x$) axis labels ``near'' the tick labels, taking the size of tick labels into account.
		
		Older versions used |xlabel absolute|, i.e.\ an absolute distance between the axis and axis labels (now matter how large or small tick labels are).

		The initial setting \emph{keeps backwards compatibility}.

		You are encouraged to use
\begin{codeexample}[code only]
\usepackage{pgfplots}
\pgfplotsset{compat=1.3}
\end{codeexample}
		in your preamble.
		
		\item Version 1.3 fixes a bug with |anchor=south west| which occurred mainly for reversed axes: the anchors where upside--down. This is fixed now.

		
		If you have written some work--around and want to keep it going, use

		|\pgfplotsset{compat/anchors=pre 1.3}|\pgfmanualpdflabel{/pgfplots/compat/anchors}{} 

		to disable the bug fix.
	\end{enumerate}

	Use |\pgfplotsset{compat=default}| to restore the factory settings.

	The setting |\pgfplotsset{compat=newest}| will always use features of the most recent version. This might result in small changes in the document's appearance.
\end{pgfplotskey}

\subsection{The Team}
\PGFPlots\ has been written mainly by Christian Feuers�nger with many improvements of Pascal Wolkotte and Nick Papior Andersen as a spare time project. We hope it is useful and provides valuable plots.

If you are interested in writing something but don't know how, consider reading the auxiliary manual \href{file:TeX-programming-notes.pdf}{TeX-programming-notes.pdf} which comes with \PGFPlots. It is far from complete, but maybe it is a good starting point (at least for more literature).

\subsection{Acknowledgements}
I thank God for all hours of enjoyed programming. I thank Pascal Wolkotte and Nick Papior Andersen for their programming efforts and contributions as part of the development team. I thank Stefan Tibus, who contributed the |plot shell| feature. I thank Tom Cashman for the contribution of the |reverse legend| feature. Special thanks go to Stefan Pinnow whose tests of \PGFPlots\ lead to numerous quality improvements. Furthermore, I thank Dr.~Schweitzer for many fruitful discussions and Dr.~Meine for his ideas and suggestions. Thanks as well to the many international contributors who provided feature requests or identified bugs or simply manual improvements!

Last but not least, I thank Till Tantau and Mark Wibrow for their excellent graphics (and more) package \PGF\ and \Tikz\ which is the base of \PGFPlots.

\include{pgfplots.install}%
\include{pgfplots.intro}%
\include{pgfplots.reference}%
\include{pgfplots.libs}%
\include{pgfplots.resources}%
\include{pgfplots.importexport}%
\include{pgfplots.basic.reference}%

\printindex

\bibliographystyle{abbrv} %gerapali} %gerabbrv} %gerunsrt.bst} %gerabbrv}% gerplain}
\nocite{pgfplotstable}
\nocite{programmingnotes}
\bibliography{pgfplots}
\end{document}
%
%%%%%%%%%%%%%%%%%%%%%%%%%%%%%%%%%%%%%%%%%%%%%%%%%%%%%%%%%%%%%%%%%%%%%%%%%%%%%
%
% Package pgfplots.sty documentation. 
%
% Copyright 2007/2008 by Christian Feuersaenger.
%
% This program is free software: you can redistribute it and/or modify
% it under the terms of the GNU General Public License as published by
% the Free Software Foundation, either version 3 of the License, or
% (at your option) any later version.
% 
% This program is distributed in the hope that it will be useful,
% but WITHOUT ANY WARRANTY; without even the implied warranty of
% MERCHANTABILITY or FITNESS FOR A PARTICULAR PURPOSE.  See the
% GNU General Public License for more details.
% 
% You should have received a copy of the GNU General Public License
% along with this program.  If not, see <http://www.gnu.org/licenses/>.
%
%
%%%%%%%%%%%%%%%%%%%%%%%%%%%%%%%%%%%%%%%%%%%%%%%%%%%%%%%%%%%%%%%%%%%%%%%%%%%%%
\input{pgfplots.preamble.tex}
%\RequirePackage[german,english,francais]{babel}

\pgfplotsmanualexternalexpensivetrue

%\usetikzlibrary{external}
%\tikzexternalize[prefix=figures/]{pgfplots}

\title{%
	Manual for Package \PGFPlots\\
	{\small Version \pgfplotsversion}\\
	{\small\href{http://sourceforge.net/projects/pgfplots}{http://sourceforge.net/projects/pgfplots}}}

%\includeonly{pgfplots.libs}


\begin{document}

\def\plotcoords{%
\addplot coordinates {
(5,8.312e-02)    (17,2.547e-02)   (49,7.407e-03)
(129,2.102e-03)  (321,5.874e-04)  (769,1.623e-04)
(1793,4.442e-05) (4097,1.207e-05) (9217,3.261e-06)
};

\addplot coordinates{
(7,8.472e-02)    (31,3.044e-02)    (111,1.022e-02)
(351,3.303e-03)  (1023,1.039e-03)  (2815,3.196e-04)
(7423,9.658e-05) (18943,2.873e-05) (47103,8.437e-06)
};

\addplot coordinates{
(9,7.881e-02)     (49,3.243e-02)    (209,1.232e-02)
(769,4.454e-03)   (2561,1.551e-03)  (7937,5.236e-04)
(23297,1.723e-04) (65537,5.545e-05) (178177,1.751e-05)
};

\addplot coordinates{
(11,6.887e-02)    (71,3.177e-02)     (351,1.341e-02)
(1471,5.334e-03)  (5503,2.027e-03)   (18943,7.415e-04)
(61183,2.628e-04) (187903,9.063e-05) (553983,3.053e-05)
};

\addplot coordinates{
(13,5.755e-02)     (97,2.925e-02)     (545,1.351e-02)
(2561,5.842e-03)   (10625,2.397e-03)  (40193,9.414e-04)
(141569,3.564e-04) (471041,1.308e-04) 
(1496065,4.670e-05)
};
}%
\maketitle
\begin{abstract}%
\PGFPlots\ draws high--quality function plots in normal or logarithmic scaling with a user-friendly interface directly in \TeX. The user supplies axis labels, legend entries and the plot coordinates for one or more plots and \PGFPlots\ applies axis scaling, computes any logarithms and axis ticks and draws the plots, supporting line plots, scatter plots, piecewise constant plots, bar plots, area plots, mesh-- and surface plots and some more. It is based on Till Tantau's package \PGF/\Tikz.
\end{abstract}
\tableofcontents
\section{Introduction}
This package provides tools to generate plots and labeled axes easily. It draws normal plots, logplots and semi-logplots, in two and three dimensions. Axis ticks, labels, legends (in case of multiple plots) can be added with key-value options. It can cycle through a set of predefined line/marker/color specifications. In summary, its purpose is to simplify the generation of high-quality function and/or data plots, and solving the problems of
\begin{itemize}
	\item consistency of document and font type and font size,
	\item direct use of \TeX\ math mode in axis descriptions,
	\item consistency of data and figures (no third party tool necessary),
	\item inter document consistency using preamble configurations and styles.
\end{itemize}
Although not necessary, separate |.pdf| or |.eps| graphics can be generated using the |external| library developed as part of \Tikz.

\PGFPlots\ is build completely on \Tikz/\PGF. Knowledge of \Tikz\ will simplify the work with \PGFPlots, although it is not required.

\section[About PGFPlots: Preliminaries]{About {\normalfont\PGFPlots}: Preliminaries}
This section contains information about upgrades, the team, the installation (in case you need to do it manually) and troubleshooting. You may skip it completely except for the upgrade remarks.

However, note that this library requires at least \PGF\ version $2.00$. At the time of this writing, many \TeX-distributions still contain the older \PGF\ version $1.18$, so it may be necessary to install a recent \PGF\ prior to using \PGFPlots.

\subsection{Components}
\PGFPlots\ comes with two components:
\begin{enumerate}
	\item the plotting component (which you are currently reading) and
	\item the \PGFPlotstable\ component which simplifies number formatting and postprocessing of numerical tables. It comes as a separate package and has its own manual \href{file:pgfplotstable.pdf}{pgfplotstable.pdf}.
\end{enumerate}

\subsection{Upgrade remarks}
This release provides a lot of improvements which can be found in all detail in \texttt{ChangeLog} for interested readers. However, some attention is useful with respect to the following changes.

\subsubsection{New Optional Features}
\begin{enumerate}
	\item \PGFPlots\ 1.3 comes with user interface improvements. The technical distinction between ``behavior options'' and ``style options'' of older versions is no longer necessary (although still fully supported).

	\item \PGFPlots\ 1.3 has a new feature which allows to \emph{move axis labels tight to tick labels} automatically.

	Since this affects the spacing, it is not enabled be default.

	Use
\begin{codeexample}[code only]
\usepackage{pgfplots}
\pgfplotsset{compat=1.3}
\end{codeexample}
	in your preamble to benefit from the improved spacing\footnote{The |compat=1.3| setting is necessary for new features which move axis labels around like reversed axis. However, \PGFPlots\ will still work as it does in earlier versions, your documents will remain the same if you don't set it explicitly.}.
	Take a look at the next page for the precise definition of |compat|.

	\item \PGFPlots\ 1.3 now supports reversed axes. It is no longer necessary to use work--arounds with negative units.
\pgfkeys{/pdflinks/search key prefixes in/.add={/pgfplots/,}{}}

	Take a look at the |x dir=reverse| key.

	Existing work--arounds will still function properly. Use |\pgfplotsset{compat=1.3}| together with |x dir=reverse| to switch to the new version.
\end{enumerate}

\subsubsection{Old Features Which May Need Attention}
\begin{enumerate}
	\item The |scatter/classes| feature produces proper legends as of version 1.3. This may change the appearance of existing legends of plots with |scatter/classes|.

	\item Due to a small math bug in \PGF\ $2.00$, you \emph{can't} use the math expression `|-x^2|'. It is necessary to use `|0-x^2|' instead. The same holds for
		`|exp(-x^2)|'; use `|exp(0-x^2)|' instead.

		This will be fixed with \PGF\ $>2.00$.

	\item Starting with \PGFPlots\ $1.1$, |\tikzstyle| should \emph{no longer be used} to set \PGFPlots\ options.
	
	Although |\tikzstyle| is still supported for some older \PGFPlots\ options, you should replace any occurance of |\tikzstyle| with |\pgfplotsset{|\meta{style name}|/.style={|\meta{key-value-list}|}}| or the associated |/.append style| variant. See section~\ref{sec:styles} for more detail.
\end{enumerate}
I apologize for any inconvenience caused by these changes.

\begin{pgfplotskey}{compat=\mchoice{1.3,pre 1.3,default,newest} (initially default)}
	The preamble configuration 
\begin{codeexample}[code only]
\usepackage{pgfplots}
\pgfplotsset{compat=1.3}
\end{codeexample}
	allows to choose between backwards compatibility and most recent features.

	\PGFPlots\ version 1.3 comes with new features which may lead to different spacings compared to earlier versions:
	\begin{enumerate}
		\item Version 1.3 comes with the |xlabel near ticks| feature which places (in this case $x$) axis labels ``near'' the tick labels, taking the size of tick labels into account.
		
		Older versions used |xlabel absolute|, i.e.\ an absolute distance between the axis and axis labels (now matter how large or small tick labels are).

		The initial setting \emph{keeps backwards compatibility}.

		You are encouraged to use
\begin{codeexample}[code only]
\usepackage{pgfplots}
\pgfplotsset{compat=1.3}
\end{codeexample}
		in your preamble.
		
		\item Version 1.3 fixes a bug with |anchor=south west| which occurred mainly for reversed axes: the anchors where upside--down. This is fixed now.

		
		If you have written some work--around and want to keep it going, use

		|\pgfplotsset{compat/anchors=pre 1.3}|\pgfmanualpdflabel{/pgfplots/compat/anchors}{} 

		to disable the bug fix.
	\end{enumerate}

	Use |\pgfplotsset{compat=default}| to restore the factory settings.

	The setting |\pgfplotsset{compat=newest}| will always use features of the most recent version. This might result in small changes in the document's appearance.
\end{pgfplotskey}

\subsection{The Team}
\PGFPlots\ has been written mainly by Christian Feuers�nger with many improvements of Pascal Wolkotte and Nick Papior Andersen as a spare time project. We hope it is useful and provides valuable plots.

If you are interested in writing something but don't know how, consider reading the auxiliary manual \href{file:TeX-programming-notes.pdf}{TeX-programming-notes.pdf} which comes with \PGFPlots. It is far from complete, but maybe it is a good starting point (at least for more literature).

\subsection{Acknowledgements}
I thank God for all hours of enjoyed programming. I thank Pascal Wolkotte and Nick Papior Andersen for their programming efforts and contributions as part of the development team. I thank Stefan Tibus, who contributed the |plot shell| feature. I thank Tom Cashman for the contribution of the |reverse legend| feature. Special thanks go to Stefan Pinnow whose tests of \PGFPlots\ lead to numerous quality improvements. Furthermore, I thank Dr.~Schweitzer for many fruitful discussions and Dr.~Meine for his ideas and suggestions. Thanks as well to the many international contributors who provided feature requests or identified bugs or simply manual improvements!

Last but not least, I thank Till Tantau and Mark Wibrow for their excellent graphics (and more) package \PGF\ and \Tikz\ which is the base of \PGFPlots.

\include{pgfplots.install}%
\include{pgfplots.intro}%
\include{pgfplots.reference}%
\include{pgfplots.libs}%
\include{pgfplots.resources}%
\include{pgfplots.importexport}%
\include{pgfplots.basic.reference}%

\printindex

\bibliographystyle{abbrv} %gerapali} %gerabbrv} %gerunsrt.bst} %gerabbrv}% gerplain}
\nocite{pgfplotstable}
\nocite{programmingnotes}
\bibliography{pgfplots}
\end{document}
%
\include{pgfplots.basic.reference}%

\printindex

\bibliographystyle{abbrv} %gerapali} %gerabbrv} %gerunsrt.bst} %gerabbrv}% gerplain}
\nocite{pgfplotstable}
\nocite{programmingnotes}
\bibliography{pgfplots}
\end{document}
%
\include{pgfplots.basic.reference}%

\printindex

\bibliographystyle{abbrv} %gerapali} %gerabbrv} %gerunsrt.bst} %gerabbrv}% gerplain}
\nocite{pgfplotstable}
\nocite{programmingnotes}
\bibliography{pgfplots}
\end{document}
%
%%%%%%%%%%%%%%%%%%%%%%%%%%%%%%%%%%%%%%%%%%%%%%%%%%%%%%%%%%%%%%%%%%%%%%%%%%%%%
%
% Package pgfplots.sty documentation. 
%
% Copyright 2007/2008 by Christian Feuersaenger.
%
% This program is free software: you can redistribute it and/or modify
% it under the terms of the GNU General Public License as published by
% the Free Software Foundation, either version 3 of the License, or
% (at your option) any later version.
% 
% This program is distributed in the hope that it will be useful,
% but WITHOUT ANY WARRANTY; without even the implied warranty of
% MERCHANTABILITY or FITNESS FOR A PARTICULAR PURPOSE.  See the
% GNU General Public License for more details.
% 
% You should have received a copy of the GNU General Public License
% along with this program.  If not, see <http://www.gnu.org/licenses/>.
%
%
%%%%%%%%%%%%%%%%%%%%%%%%%%%%%%%%%%%%%%%%%%%%%%%%%%%%%%%%%%%%%%%%%%%%%%%%%%%%%
%%%%%%%%%%%%%%%%%%%%%%%%%%%%%%%%%%%%%%%%%%%%%%%%%%%%%%%%%%%%%%%%%%%%%%%%%%%%%
%
% Package pgfplots.sty documentation. 
%
% Copyright 2007/2008 by Christian Feuersaenger.
%
% This program is free software: you can redistribute it and/or modify
% it under the terms of the GNU General Public License as published by
% the Free Software Foundation, either version 3 of the License, or
% (at your option) any later version.
% 
% This program is distributed in the hope that it will be useful,
% but WITHOUT ANY WARRANTY; without even the implied warranty of
% MERCHANTABILITY or FITNESS FOR A PARTICULAR PURPOSE.  See the
% GNU General Public License for more details.
% 
% You should have received a copy of the GNU General Public License
% along with this program.  If not, see <http://www.gnu.org/licenses/>.
%
%
%%%%%%%%%%%%%%%%%%%%%%%%%%%%%%%%%%%%%%%%%%%%%%%%%%%%%%%%%%%%%%%%%%%%%%%%%%%%%
\documentclass[a4paper]{ltxdoc}

\usepackage{makeidx}

% DON't let hyperref overload the format of index and glossary. 
% I want to do that on my own in the stylefiles for makeindex...
\makeatletter
\let\@old@wrindex=\@wrindex
\makeatother

\usepackage{ifpdf}
\ifpdf
	\usepackage[pdftex]{hyperref}
\else
	\usepackage[dvipdfm]{hyperref}
\fi
\hypersetup{pdfborder={0 0 0}}

\makeatletter
\let\@wrindex=\@old@wrindex
\makeatother

% Formatiere Seitennummern im Index:
\newcommand{\indexpageno}[1]{%
	{\bfseries\hyperpage{#1}}%
}


\newcommand{\R}{\mathbb{R}}
\newcommand{\N}{\mathbb{N}}
\newcommand{\Z}{\mathbb{Z}}

\long\def\COMMENTLOWLEVEL#1\ENDCOMMENT{}
\def\ENDCOMMENT{}

\usepackage{textcomp}
\usepackage{booktabs}

\usepackage{calc}
\usepackage[formats]{listings}
%\usepackage{courier} % don't use it - the '^' character can't be copy-pasted in courier

\usepackage{array}
\lstset{%
	basicstyle=\ttfamily,
	language=[LaTeX]tex, % Seems as if \lstset{language=tex} must be invoked BEFORE loading tikz!?
	tabsize=4,
	breaklines=true,
	breakindent=0pt
}

\ifpdf
	\pdfinfo {
		/Author	(Christian Feuersaenger)
	}

\else
	\def\pgfsysdriver{pgfsys-dvipdfm.def}
\fi
%\def\pgfsysdriver{pgfsys-pdftex.def}
\usepackage{pgfplots}

\ifpdf
	\usetikzlibrary{pgfplotsclickable}
\fi

%\usepackage{fp}
% ATTENTION:
% this requires pgf version NEWER than 2.00 :
%\usetikzlibrary{fixedpointarithmetic}

\usetikzlibrary{dateplot}

\usepackage[a4paper,left=2.25cm,right=2.25cm,top=2.5cm,bottom=2.5cm,nohead]{geometry}
\usepackage{amsmath,amssymb}
\usepackage{xxcolor}
\usepackage{pifont}
\usepackage[latin1]{inputenc}
\usepackage{amsmath}
\usepackage{eurosym}

\def\eps{\textsc{eps}}

\input pgfmanual-en-macros.tex

\def\pgfmanualbar{\char`\|}
\makeatletter
\newenvironment{addplotoperation}[3][]{
  \begin{pgfmanualentry}
  	{%
	\let\ltxdoc@marg=\marg
	\let\ltxdoc@oarg=\oarg
	\let\ltxdoc@parg=\parg
	\let\ltxdoc@meta=\meta
	\def\marg##1{{\normalfont\ltxdoc@marg{##1}}}%
	\def\oarg##1{{\normalfont\ltxdoc@oarg{##1}}}%
	\def\parg##1{{\normalfont\ltxdoc@parg{##1}}}%
	\def\meta##1{{\normalfont\ltxdoc@meta{##1}}}%
    \pgfmanualentryheadline{\textcolor{gray}{{\ttfamily\char`\\addplot\ }}%
      \declare{\texttt{#2}} \texttt{#3;}}%
	  \unskip
	 \nobreak
    \pgfmanualentryheadline{\textcolor{gray}{{\texttt{\char`\\addplot}\oarg{style options} \texttt{plot}\oarg{behavior options}}\ }%
      \declare{\texttt{#2}} \texttt{#3} \textcolor{gray}{\meta{trailing path commands}}\texttt{;}}%
    \def\pgfmanualtest{#1}%
    \ifx\pgfmanualtest\@empty%
      \index{#2@\protect\textcolor{gray}{\protect\texttt{plot}}\protect\texttt{ #2}}%
      \index{Plot operations!plot #2@\protect\texttt{plot #2}}%
    \fi%
	}%
    \pgfmanualbody
}
{
  \end{pgfmanualentry}
}

\newenvironment{codekey}[1]{%
  \begin{pgfmanualentry}
 	\pgfmanualentryheadline{{\ttfamily\declare{#1}\textcolor{gray}{/.code}=\marg{...}}\hfill}%
	\def\mykey{#1}%
	\def\mypath{}%
	\def\myname{}%
	\firsttimetrue%
	\decompose#1/\nil%
    \pgfmanualbody
}
{
  \end{pgfmanualentry}
}

\newenvironment{pgfplotscodekey}[1]{%
	\begin{codekey}{/pgfplots/#1}%
}
{
  \end{codekey}
}
\newenvironment{pgfplotscodetwokey}[1]{%
  \begin{pgfmanualentry}
 	\pgfmanualentryheadline{{\ttfamily\declare{/pgfplots/#1}\textcolor{gray}{/.code 2 args}=\marg{...}}\hfill}%
	\def\mykey{/pgfplots/#1}%
	\def\mypath{}%
	\def\myname{}%
	\firsttimetrue%
	\decompose/pgfplots/#1/\nil%
    \pgfmanualbody
}
{
  \end{pgfmanualentry}
}

\newenvironment{pgfplotsxycodekeylist}[1]{%
	\begingroup
	\let\oldpgfmanualentryheadline=\pgfmanualentryheadline
	\def\pgfmanualentryheadline##1{%
		\pgfmanualentryheadline@##1\pgfplots@EOI
	}%
	\def\pgfmanualentryheadline@##1\hfill##2\pgfplots@EOI{%
		\oldpgfmanualentryheadline{{\ttfamily\declare{##1}\textcolor{gray}{/.code}=\marg{...}}\hfill}%
	}
	\begin{pgfplotsxykeylist}{#1}%
}
{
	\end{pgfplotsxykeylist}
	\endgroup
}

\newenvironment{pgfplotskey}[1]{%
  \begin{key}{/pgfplots/#1}%
}
{
  \end{key}
}

\def\choicesep{$\vert$}%
\def\choicearg#1{\texttt{#1}}

\newif\iffirstchoice
\newcommand\mchoice[1]{%
	\begingroup
	\firstchoicetrue
	\foreach \mchoice@ in {#1} {%
		\iffirstchoice
			\global\firstchoicefalse
		\else
			\choicesep
		\fi
		\choicearg{\mchoice@}%
	}%
	\endgroup
}%

% \begin{xykey}{/path/\x label=value}
% \end{xykey}
%
% has same features with 'default', 'initially' etc as key environment
\newenvironment{xykey}[2][]{%
	\begin{pgfmanualentry}
    \def\extrakeytext{}
	\insertpathifneeded{#2}{#1}%
	\expandafter\pgfutil@in@\expandafter=\expandafter{\mykey}%
	\ifpgfutil@in@%
		\expandafter\xykey@eq\mykey\@nil
	\else
		\expandafter\xykey@noeq\mykey\@nil
	\fi
	\pgfmanualbody
}{%
	\end{pgfmanualentry}
}%

% \begin{xystylekey}{/path/\x label=value}
% \end{xystylekey}
%
% has same features with 'default', 'initially' etc as key environment
\newenvironment{xystylekey}[2][]{%
	\begin{pgfmanualentry}
    \def\extrakeytext{style, }
	\insertpathifneeded{#2}{#1}%
	\expandafter\pgfutil@in@\expandafter=\expandafter{\mykey}%
	\ifpgfutil@in@%
		\expandafter\xykey@eq\mykey\@nil
	\else
		\expandafter\xykey@noeq\mykey\@nil
	\fi
	\pgfmanualbody
}{%
	\end{pgfmanualentry}
}%

% \insertpathifneeded{a key}{/pgfplots} -> assign mykey={/pgfplots/a key}
% \insertpathifneeded{/tikz/a key}{/pgfplots} -> assign mykey={/tikz/a key}
%
% #1: the key
% #2: a default path (or empty)
\def\insertpathifneeded#1#2{%
	\def\insertpathifneeded@@{#2}%
	\ifx\insertpathifneeded@@\empty
		\def\mykey{#1}%
	\else
		\insertpathifneeded@#2\@nil
		\ifpgfutil@in@
			\def\mykey{#2/#1}%
		\else
			\def\mykey{#1}%
		\fi
	\fi
}%
\def\insertpathifneeded@#1#2\@nil{%
	\def\insertpathifneeded@@{#1}%
	\def\insertpathifneeded@@@{/}%
	\ifx\insertpathifneeded@@\insertpathifneeded@@@
		\pgfutil@in@true
	\else
		\pgfutil@in@false
	\fi
}%

% \begin{keylist}[default path]
% 	{/path/option 1=value,/path/option 2=value2}
% \end{keylist}
\newenvironment{keylist}[2][]{%
	\begin{pgfmanualentry}
    \def\extrakeytext{}%
	\foreach \xx in {#2} {%
		\expandafter\insertpathifneeded\expandafter{\xx}{#1}%
		\expandafter\extractkey\mykey\@nil%
	}%
	\pgfmanualbody
}{%
  \end{pgfmanualentry}
}%

\newenvironment{pgfplotskeylist}[1]{%
	\begin{keylist}[/pgfplots]{#1}%
}{%
	\end{keylist}%
}

% \begin{xykeylist}[default path]
% 	{/path/option \x1=value,/path/option \x2=value2,/path/option \x3=value}
% \end{xykeylist}
\newenvironment{xykeylist}[2][]{%
	\begin{pgfmanualentry}
    \def\extrakeytext{}
	\foreach \xx in {#2} {%
		\expandafter\insertpathifneeded\expandafter{\xx}{#1}%
		\expandafter\pgfutil@in@\expandafter=\expandafter{\mykey}%
		\ifpgfutil@in@%
			\expandafter\xykey@eq\mykey\@nil
		\else
			\expandafter\xykey@noeq\mykey\@nil
		\fi
	}%
	\pgfmanualbody
}{%
  \end{pgfmanualentry}
}%

\newif\ifxykeyfound

\def\xykey@eq#1=#2\@nil{%
	\def\x{x}%
	\xdef\mykey{#1}%
	\def\xykey@@{#1}%
	\ifx\xykey@@\mykey
		\xykeyfoundfalse
	\else
		\xykeyfoundtrue
	\fi
    \expandafter\extractkey\mykey=#2\@nil%
	\ifxykeyfound
		\def\x{y}%
		\xdef\mykey{#1}%
		\expandafter\extractkey\mykey=#2\@nil%
	\fi
}
\def\xykey@noeq#1\@nil{%
	\def\x{x}%
	\xdef\mykey{#1}%
	\def\xykey@@{#1}%
	\ifx\xykey@@\mykey
		\xykeyfoundfalse
	\else
		\xykeyfoundtrue
	\fi
    \expandafter\extractkey\mykey\@nil%
	\ifxykeyfound
		\def\x{y}%
		\xdef\mykey{#1}%
		\expandafter\extractkey\mykey\@nil%
	\fi
}

% \begin{pgfplotsxykey}{\x label=value}
% \end{pgfplotsxykey}
%
% It introduces the path /pgfplots/ automatically.
%
% has same features with 'default', 'initially' etc as key environment
\newenvironment{pgfplotsxykey}[1]{%
	\begin{xykey}[/pgfplots]{#1}%
}{%
	\end{xykey}%
}


\newenvironment{pgfplotsxykeylist}[1]{%
	\begin{xykeylist}[/pgfplots]{#1}%
}{%
	\end{xykeylist}%
}


\newenvironment{plottype}[1]{%
	\begin{key}{/tikz/#1}%
	\end{key}
  \begin{pgfmanualentry}
    \pgfmanualentryheadline{\textcolor{gray}{{\ttfamily\char`\\addplot+[\declare{#1}]}}}%
    \pgfmanualbody
}
{
  \end{pgfmanualentry}
}

\def\index@prologue{\section*{Index}\addcontentsline{toc}{section}{Index}
}

\newenvironment{pgfplotstablecolumnkey}{%
  \begin{pgfmanualentry}
 	\pgfmanualentryheadline{{\ttfamily\textcolor{gray}{/pgfplots/table/}\declare{columns/\meta{column name}}\textcolor{gray}{/.style}=\marg{key-value-list}}\hfill}%
	\pgfplotsmanualkeyindex{/pgfplots/table/columns}%
    \pgfmanualbody
}
{
  \end{pgfmanualentry}
}
\newenvironment{pgfplotstabledisplaycolumnkey}{%
  \begin{pgfmanualentry}
 	\pgfmanualentryheadline{{\ttfamily\textcolor{gray}{/pgfplots/table/}\declare{display columns/\meta{index}}\textcolor{gray}{/.style}=\marg{key-value-list}}\hfill}%
	\pgfplotsmanualkeyindex{/pgfplots/table/display columns}%
    \pgfmanualbody
}
{
  \end{pgfmanualentry}
}
\newenvironment{pgfplotstablealiaskey}{%
  \begin{pgfmanualentry}
 	\pgfmanualentryheadline{{\ttfamily\textcolor{gray}{/pgfplots/table/}\declare{alias/\meta{col name}}\textcolor{gray}{/.initial}=\marg{real col name}}\hfill}%
	\pgfplotsmanualkeyindex{/pgfplots/table/alias}%
    \pgfmanualbody
}
{
  \end{pgfmanualentry}
}


\def\pgfplotsmanualkeyindex#1{%
	\def\mypath{#1}%
	\def\myname{}%
	\firsttimetrue%
	\decompose#1/\nil%
}
\newenvironment{pgfplotstablecreateonusekey}{%
  \begin{pgfmanualentry}
 	\pgfmanualentryheadline{{\ttfamily\textcolor{gray}{/pgfplots/table/}\declare{create on use/\meta{col name}}\textcolor{gray}{/.style}=\marg{create options}}\hfill}%
	\def\mykey{/pgfplots/table/create on use}%
    \pgfmanualbody
	\pgfplotsmanualkeyindex{/pgfplots/table/create on use}%
}
{
  \end{pgfmanualentry}
}

\def\pgfplotsassertcmdkeyexists#1{%
	\pgfkeysifdefined{/pgfplots/#1/.@cmd}\relax{%
		\pgfplots@error{DOCUMENTATION ERROR: command key /pgfplots/#1 does not exist!}%
	}%
}%

{
\catcode`\ =12%
\gdef\makespaceexpandable{\def\ { }}}%

\def\pgfplotsassertXYcmdkeyexists#1{%
	{\makespaceexpandable\def\x{x}\edef\pgfplotsassertXYcmdkeyexists@tmp{#1}%
	\pgfkeysifdefined{/pgfplots/\pgfplotsassertXYcmdkeyexists@tmp/.@cmd}\relax{%
		\pgfplots@error{DOCUMENTATION ERROR: command key /pgfplots/#1 does not exist!}%
	}}%
	{\makespaceexpandable\def\x{y}\edef\pgfplotsassertXYcmdkeyexists@tmp{#1}%
	\pgfkeysifdefined{/pgfplots/\pgfplotsassertXYcmdkeyexists@tmp/.@cmd}\relax{%
		\pgfplots@error{DOCUMENTATION ERROR: command key /pgfplots/#1 does not exist!}%
	}}%
}%

\def\pgfplotsshortstylekey #1=#2\pgfeov{%
	\pgfplotsassertcmdkeyexists{#1}%
	\pgfplotsassertcmdkeyexists{#2}%
	\begin{pgfplotskey}{#1=\marg{key-value-list}}
		An abbreviation for \texttt{#2/.append style=}\marg{key-value-list}.
	\end{pgfplotskey}
}
\def\pgfplotsshortxystylekey #1=#2\pgfeov{%
	\pgfplotsassertXYcmdkeyexists{#1}%
	\pgfplotsassertXYcmdkeyexists{#2}%
	\begin{pgfplotsxykey}{#1=\marg{key-value-list}}
		An abbreviation for {\def\x{x}\texttt{#2/.append style=}}\marg{key-value-list} (or the respective style for $y$, {\def\x{y}\texttt{#2}}).
	\end{pgfplotsxykey}
}
\def\pgfplotsshortstylekeys #1,#2=#3\pgfeov{%
	\pgfplotsassertcmdkeyexists{#1}%
	\pgfplotsassertcmdkeyexists{#2}%
	\pgfplotsassertcmdkeyexists{#3}%
	\begin{pgfplotskeylist}{%
		#1=\marg{key-value-list},
		#2=\marg{key-value-list}}
		Different abbreviations for \texttt{#3/.append style=}\marg{key-value-list}.
	\end{pgfplotskeylist}
}
\def\pgfplotsshortxystylekeys #1,#2=#3\pgfeov{%
	\pgfplotsassertXYcmdkeyexists{#1}%
	\pgfplotsassertXYcmdkeyexists{#2}%
	\pgfplotsassertXYcmdkeyexists{#3}%
	\begin{pgfplotsxykeylist}{%
		#1=\marg{key-value-list},
		#2=\marg{key-value-list}}
		Different abbreviations for {\def\x{x}\texttt{#3/.append style=}}\marg{key-value-list} (or the respective style for $y$, {\def\x{y}\texttt{#3}}).
	\end{pgfplotsxykeylist}
}

%
% Creates and shows a colormap with specification '#1'.
\def\pgfplotsshowcolormapexample#1{%
	\pgfplotscreatecolormap{tempcolormap}{#1}%
	\pgfplotsshowcolormap{tempcolormap}%
}

% Shows the colormap named '#1'.
\def\pgfplotsshowcolormap#1{%
	\pgfplotscolormaptoshadingspec{#1}{8cm}\result
	\def\tempb{\pgfdeclarehorizontalshading{tempshading}{1cm}}%
	\expandafter\tempb\expandafter{\result}%
	\pgfuseshading{tempshading}%
}

\makeatother


\pgfqkeys{/codeexample}{%
	every codeexample/.style={
		width=8cm,
		/pgfplots/every axis/.append style={legend style={fill=graphicbackground}}
	},
	tabsize=4,
}
\pgfplotsset{
	every axis/.append style={width=7cm},
	filter discard warning=false,
}

\usetikzlibrary{backgrounds,patterns}
% Global styles:
\tikzset{
  shape example/.style={
    color=black!30,
    draw,
    fill=yellow!30,
    line width=.5cm,
    inner xsep=2.5cm,
    inner ysep=0.5cm}
}

\newcommand{\FIXME}[1]{\textcolor{red}{(FIXME: #1)}}

% fuer endvironment 'sidewaysfigure' bspw
% \usepackage{rotating}

\newcommand\Tikz{Ti\textit kZ}
\newcommand\PGF{\textsc{pgf}}
\newcommand\PGFPlots{\textsc{pgfplots}}
\def\PGFPlotstable{\textsc{PgfplotsTable}}


\makeindex

% Fix overful hboxes automatically:
\tolerance=2000
\emergencystretch=10pt

\tikzset{prefix=gnuplot/pgfplots_} % prefix for 'plot function'

\author{%
	Christian Feuers\"anger\footnote{\url{http://wissrech.ins.uni-bonn.de/people/feuersaenger}}\\%
	Institut f\"ur Numerische Simulation\\
	Universit\"at Bonn, Germany}


%\RequirePackage[german,english,francais]{babel}

\pgfplotsmanualexternalexpensivetrue

%\usetikzlibrary{external}
%\tikzexternalize[prefix=figures/]{pgfplots}

\title{%
	Manual for Package \PGFPlots\\
	{\small Version \pgfplotsversion}\\
	{\small\href{http://sourceforge.net/projects/pgfplots}{http://sourceforge.net/projects/pgfplots}}}

%\includeonly{pgfplots.libs}


\begin{document}

\def\plotcoords{%
\addplot coordinates {
(5,8.312e-02)    (17,2.547e-02)   (49,7.407e-03)
(129,2.102e-03)  (321,5.874e-04)  (769,1.623e-04)
(1793,4.442e-05) (4097,1.207e-05) (9217,3.261e-06)
};

\addplot coordinates{
(7,8.472e-02)    (31,3.044e-02)    (111,1.022e-02)
(351,3.303e-03)  (1023,1.039e-03)  (2815,3.196e-04)
(7423,9.658e-05) (18943,2.873e-05) (47103,8.437e-06)
};

\addplot coordinates{
(9,7.881e-02)     (49,3.243e-02)    (209,1.232e-02)
(769,4.454e-03)   (2561,1.551e-03)  (7937,5.236e-04)
(23297,1.723e-04) (65537,5.545e-05) (178177,1.751e-05)
};

\addplot coordinates{
(11,6.887e-02)    (71,3.177e-02)     (351,1.341e-02)
(1471,5.334e-03)  (5503,2.027e-03)   (18943,7.415e-04)
(61183,2.628e-04) (187903,9.063e-05) (553983,3.053e-05)
};

\addplot coordinates{
(13,5.755e-02)     (97,2.925e-02)     (545,1.351e-02)
(2561,5.842e-03)   (10625,2.397e-03)  (40193,9.414e-04)
(141569,3.564e-04) (471041,1.308e-04) 
(1496065,4.670e-05)
};
}%
\maketitle
\begin{abstract}%
\PGFPlots\ draws high--quality function plots in normal or logarithmic scaling with a user-friendly interface directly in \TeX. The user supplies axis labels, legend entries and the plot coordinates for one or more plots and \PGFPlots\ applies axis scaling, computes any logarithms and axis ticks and draws the plots, supporting line plots, scatter plots, piecewise constant plots, bar plots, area plots, mesh-- and surface plots and some more. It is based on Till Tantau's package \PGF/\Tikz.
\end{abstract}
\tableofcontents
\section{Introduction}
This package provides tools to generate plots and labeled axes easily. It draws normal plots, logplots and semi-logplots, in two and three dimensions. Axis ticks, labels, legends (in case of multiple plots) can be added with key-value options. It can cycle through a set of predefined line/marker/color specifications. In summary, its purpose is to simplify the generation of high-quality function and/or data plots, and solving the problems of
\begin{itemize}
	\item consistency of document and font type and font size,
	\item direct use of \TeX\ math mode in axis descriptions,
	\item consistency of data and figures (no third party tool necessary),
	\item inter document consistency using preamble configurations and styles.
\end{itemize}
Although not necessary, separate |.pdf| or |.eps| graphics can be generated using the |external| library developed as part of \Tikz.

\PGFPlots\ is build completely on \Tikz/\PGF. Knowledge of \Tikz\ will simplify the work with \PGFPlots, although it is not required.

\section[About PGFPlots: Preliminaries]{About {\normalfont\PGFPlots}: Preliminaries}
This section contains information about upgrades, the team, the installation (in case you need to do it manually) and troubleshooting. You may skip it completely except for the upgrade remarks.

However, note that this library requires at least \PGF\ version $2.00$. At the time of this writing, many \TeX-distributions still contain the older \PGF\ version $1.18$, so it may be necessary to install a recent \PGF\ prior to using \PGFPlots.

\subsection{Components}
\PGFPlots\ comes with two components:
\begin{enumerate}
	\item the plotting component (which you are currently reading) and
	\item the \PGFPlotstable\ component which simplifies number formatting and postprocessing of numerical tables. It comes as a separate package and has its own manual \href{file:pgfplotstable.pdf}{pgfplotstable.pdf}.
\end{enumerate}

\subsection{Upgrade remarks}
This release provides a lot of improvements which can be found in all detail in \texttt{ChangeLog} for interested readers. However, some attention is useful with respect to the following changes.

\subsubsection{New Optional Features}
\begin{enumerate}
	\item \PGFPlots\ 1.3 comes with user interface improvements. The technical distinction between ``behavior options'' and ``style options'' of older versions is no longer necessary (although still fully supported).

	\item \PGFPlots\ 1.3 has a new feature which allows to \emph{move axis labels tight to tick labels} automatically.

	Since this affects the spacing, it is not enabled be default.

	Use
\begin{codeexample}[code only]
\usepackage{pgfplots}
\pgfplotsset{compat=1.3}
\end{codeexample}
	in your preamble to benefit from the improved spacing\footnote{The |compat=1.3| setting is necessary for new features which move axis labels around like reversed axis. However, \PGFPlots\ will still work as it does in earlier versions, your documents will remain the same if you don't set it explicitly.}.
	Take a look at the next page for the precise definition of |compat|.

	\item \PGFPlots\ 1.3 now supports reversed axes. It is no longer necessary to use work--arounds with negative units.
\pgfkeys{/pdflinks/search key prefixes in/.add={/pgfplots/,}{}}

	Take a look at the |x dir=reverse| key.

	Existing work--arounds will still function properly. Use |\pgfplotsset{compat=1.3}| together with |x dir=reverse| to switch to the new version.
\end{enumerate}

\subsubsection{Old Features Which May Need Attention}
\begin{enumerate}
	\item The |scatter/classes| feature produces proper legends as of version 1.3. This may change the appearance of existing legends of plots with |scatter/classes|.

	\item Due to a small math bug in \PGF\ $2.00$, you \emph{can't} use the math expression `|-x^2|'. It is necessary to use `|0-x^2|' instead. The same holds for
		`|exp(-x^2)|'; use `|exp(0-x^2)|' instead.

		This will be fixed with \PGF\ $>2.00$.

	\item Starting with \PGFPlots\ $1.1$, |\tikzstyle| should \emph{no longer be used} to set \PGFPlots\ options.
	
	Although |\tikzstyle| is still supported for some older \PGFPlots\ options, you should replace any occurance of |\tikzstyle| with |\pgfplotsset{|\meta{style name}|/.style={|\meta{key-value-list}|}}| or the associated |/.append style| variant. See section~\ref{sec:styles} for more detail.
\end{enumerate}
I apologize for any inconvenience caused by these changes.

\begin{pgfplotskey}{compat=\mchoice{1.3,pre 1.3,default,newest} (initially default)}
	The preamble configuration 
\begin{codeexample}[code only]
\usepackage{pgfplots}
\pgfplotsset{compat=1.3}
\end{codeexample}
	allows to choose between backwards compatibility and most recent features.

	\PGFPlots\ version 1.3 comes with new features which may lead to different spacings compared to earlier versions:
	\begin{enumerate}
		\item Version 1.3 comes with the |xlabel near ticks| feature which places (in this case $x$) axis labels ``near'' the tick labels, taking the size of tick labels into account.
		
		Older versions used |xlabel absolute|, i.e.\ an absolute distance between the axis and axis labels (now matter how large or small tick labels are).

		The initial setting \emph{keeps backwards compatibility}.

		You are encouraged to use
\begin{codeexample}[code only]
\usepackage{pgfplots}
\pgfplotsset{compat=1.3}
\end{codeexample}
		in your preamble.
		
		\item Version 1.3 fixes a bug with |anchor=south west| which occurred mainly for reversed axes: the anchors where upside--down. This is fixed now.

		
		If you have written some work--around and want to keep it going, use

		|\pgfplotsset{compat/anchors=pre 1.3}|\pgfmanualpdflabel{/pgfplots/compat/anchors}{} 

		to disable the bug fix.
	\end{enumerate}

	Use |\pgfplotsset{compat=default}| to restore the factory settings.

	The setting |\pgfplotsset{compat=newest}| will always use features of the most recent version. This might result in small changes in the document's appearance.
\end{pgfplotskey}

\subsection{The Team}
\PGFPlots\ has been written mainly by Christian Feuers�nger with many improvements of Pascal Wolkotte and Nick Papior Andersen as a spare time project. We hope it is useful and provides valuable plots.

If you are interested in writing something but don't know how, consider reading the auxiliary manual \href{file:TeX-programming-notes.pdf}{TeX-programming-notes.pdf} which comes with \PGFPlots. It is far from complete, but maybe it is a good starting point (at least for more literature).

\subsection{Acknowledgements}
I thank God for all hours of enjoyed programming. I thank Pascal Wolkotte and Nick Papior Andersen for their programming efforts and contributions as part of the development team. I thank Stefan Tibus, who contributed the |plot shell| feature. I thank Tom Cashman for the contribution of the |reverse legend| feature. Special thanks go to Stefan Pinnow whose tests of \PGFPlots\ lead to numerous quality improvements. Furthermore, I thank Dr.~Schweitzer for many fruitful discussions and Dr.~Meine for his ideas and suggestions. Thanks as well to the many international contributors who provided feature requests or identified bugs or simply manual improvements!

Last but not least, I thank Till Tantau and Mark Wibrow for their excellent graphics (and more) package \PGF\ and \Tikz\ which is the base of \PGFPlots.

%%%%%%%%%%%%%%%%%%%%%%%%%%%%%%%%%%%%%%%%%%%%%%%%%%%%%%%%%%%%%%%%%%%%%%%%%%%%%
%
% Package pgfplots.sty documentation. 
%
% Copyright 2007/2008 by Christian Feuersaenger.
%
% This program is free software: you can redistribute it and/or modify
% it under the terms of the GNU General Public License as published by
% the Free Software Foundation, either version 3 of the License, or
% (at your option) any later version.
% 
% This program is distributed in the hope that it will be useful,
% but WITHOUT ANY WARRANTY; without even the implied warranty of
% MERCHANTABILITY or FITNESS FOR A PARTICULAR PURPOSE.  See the
% GNU General Public License for more details.
% 
% You should have received a copy of the GNU General Public License
% along with this program.  If not, see <http://www.gnu.org/licenses/>.
%
%
%%%%%%%%%%%%%%%%%%%%%%%%%%%%%%%%%%%%%%%%%%%%%%%%%%%%%%%%%%%%%%%%%%%%%%%%%%%%%
%%%%%%%%%%%%%%%%%%%%%%%%%%%%%%%%%%%%%%%%%%%%%%%%%%%%%%%%%%%%%%%%%%%%%%%%%%%%%
%
% Package pgfplots.sty documentation. 
%
% Copyright 2007/2008 by Christian Feuersaenger.
%
% This program is free software: you can redistribute it and/or modify
% it under the terms of the GNU General Public License as published by
% the Free Software Foundation, either version 3 of the License, or
% (at your option) any later version.
% 
% This program is distributed in the hope that it will be useful,
% but WITHOUT ANY WARRANTY; without even the implied warranty of
% MERCHANTABILITY or FITNESS FOR A PARTICULAR PURPOSE.  See the
% GNU General Public License for more details.
% 
% You should have received a copy of the GNU General Public License
% along with this program.  If not, see <http://www.gnu.org/licenses/>.
%
%
%%%%%%%%%%%%%%%%%%%%%%%%%%%%%%%%%%%%%%%%%%%%%%%%%%%%%%%%%%%%%%%%%%%%%%%%%%%%%
\documentclass[a4paper]{ltxdoc}

\usepackage{makeidx}

% DON't let hyperref overload the format of index and glossary. 
% I want to do that on my own in the stylefiles for makeindex...
\makeatletter
\let\@old@wrindex=\@wrindex
\makeatother

\usepackage{ifpdf}
\ifpdf
	\usepackage[pdftex]{hyperref}
\else
	\usepackage[dvipdfm]{hyperref}
\fi
\hypersetup{pdfborder={0 0 0}}

\makeatletter
\let\@wrindex=\@old@wrindex
\makeatother

% Formatiere Seitennummern im Index:
\newcommand{\indexpageno}[1]{%
	{\bfseries\hyperpage{#1}}%
}


\newcommand{\R}{\mathbb{R}}
\newcommand{\N}{\mathbb{N}}
\newcommand{\Z}{\mathbb{Z}}

\long\def\COMMENTLOWLEVEL#1\ENDCOMMENT{}
\def\ENDCOMMENT{}

\usepackage{textcomp}
\usepackage{booktabs}

\usepackage{calc}
\usepackage[formats]{listings}
%\usepackage{courier} % don't use it - the '^' character can't be copy-pasted in courier

\usepackage{array}
\lstset{%
	basicstyle=\ttfamily,
	language=[LaTeX]tex, % Seems as if \lstset{language=tex} must be invoked BEFORE loading tikz!?
	tabsize=4,
	breaklines=true,
	breakindent=0pt
}

\ifpdf
	\pdfinfo {
		/Author	(Christian Feuersaenger)
	}

\else
	\def\pgfsysdriver{pgfsys-dvipdfm.def}
\fi
%\def\pgfsysdriver{pgfsys-pdftex.def}
\usepackage{pgfplots}

\ifpdf
	\usetikzlibrary{pgfplotsclickable}
\fi

%\usepackage{fp}
% ATTENTION:
% this requires pgf version NEWER than 2.00 :
%\usetikzlibrary{fixedpointarithmetic}

\usetikzlibrary{dateplot}

\usepackage[a4paper,left=2.25cm,right=2.25cm,top=2.5cm,bottom=2.5cm,nohead]{geometry}
\usepackage{amsmath,amssymb}
\usepackage{xxcolor}
\usepackage{pifont}
\usepackage[latin1]{inputenc}
\usepackage{amsmath}
\usepackage{eurosym}
\input{pgfplots-macros}

\pgfqkeys{/codeexample}{%
	every codeexample/.style={
		width=8cm,
		/pgfplots/every axis/.append style={legend style={fill=graphicbackground}}
	},
	tabsize=4,
}
\pgfplotsset{
	every axis/.append style={width=7cm},
	filter discard warning=false,
}

\usetikzlibrary{backgrounds,patterns}
% Global styles:
\tikzset{
  shape example/.style={
    color=black!30,
    draw,
    fill=yellow!30,
    line width=.5cm,
    inner xsep=2.5cm,
    inner ysep=0.5cm}
}

\newcommand{\FIXME}[1]{\textcolor{red}{(FIXME: #1)}}

% fuer endvironment 'sidewaysfigure' bspw
% \usepackage{rotating}

\newcommand\Tikz{Ti\textit kZ}
\newcommand\PGF{\textsc{pgf}}
\newcommand\PGFPlots{\textsc{pgfplots}}
\def\PGFPlotstable{\textsc{PgfplotsTable}}


\makeindex

% Fix overful hboxes automatically:
\tolerance=2000
\emergencystretch=10pt

\tikzset{prefix=gnuplot/pgfplots_} % prefix for 'plot function'

\author{%
	Christian Feuers\"anger\footnote{\url{http://wissrech.ins.uni-bonn.de/people/feuersaenger}}\\%
	Institut f\"ur Numerische Simulation\\
	Universit\"at Bonn, Germany}


%\RequirePackage[german,english,francais]{babel}

\pgfplotsmanualexternalexpensivetrue

%\usetikzlibrary{external}
%\tikzexternalize[prefix=figures/]{pgfplots}

\title{%
	Manual for Package \PGFPlots\\
	{\small Version \pgfplotsversion}\\
	{\small\href{http://sourceforge.net/projects/pgfplots}{http://sourceforge.net/projects/pgfplots}}}

%\includeonly{pgfplots.libs}


\begin{document}

\def\plotcoords{%
\addplot coordinates {
(5,8.312e-02)    (17,2.547e-02)   (49,7.407e-03)
(129,2.102e-03)  (321,5.874e-04)  (769,1.623e-04)
(1793,4.442e-05) (4097,1.207e-05) (9217,3.261e-06)
};

\addplot coordinates{
(7,8.472e-02)    (31,3.044e-02)    (111,1.022e-02)
(351,3.303e-03)  (1023,1.039e-03)  (2815,3.196e-04)
(7423,9.658e-05) (18943,2.873e-05) (47103,8.437e-06)
};

\addplot coordinates{
(9,7.881e-02)     (49,3.243e-02)    (209,1.232e-02)
(769,4.454e-03)   (2561,1.551e-03)  (7937,5.236e-04)
(23297,1.723e-04) (65537,5.545e-05) (178177,1.751e-05)
};

\addplot coordinates{
(11,6.887e-02)    (71,3.177e-02)     (351,1.341e-02)
(1471,5.334e-03)  (5503,2.027e-03)   (18943,7.415e-04)
(61183,2.628e-04) (187903,9.063e-05) (553983,3.053e-05)
};

\addplot coordinates{
(13,5.755e-02)     (97,2.925e-02)     (545,1.351e-02)
(2561,5.842e-03)   (10625,2.397e-03)  (40193,9.414e-04)
(141569,3.564e-04) (471041,1.308e-04) 
(1496065,4.670e-05)
};
}%
\maketitle
\begin{abstract}%
\PGFPlots\ draws high--quality function plots in normal or logarithmic scaling with a user-friendly interface directly in \TeX. The user supplies axis labels, legend entries and the plot coordinates for one or more plots and \PGFPlots\ applies axis scaling, computes any logarithms and axis ticks and draws the plots, supporting line plots, scatter plots, piecewise constant plots, bar plots, area plots, mesh-- and surface plots and some more. It is based on Till Tantau's package \PGF/\Tikz.
\end{abstract}
\tableofcontents
\section{Introduction}
This package provides tools to generate plots and labeled axes easily. It draws normal plots, logplots and semi-logplots, in two and three dimensions. Axis ticks, labels, legends (in case of multiple plots) can be added with key-value options. It can cycle through a set of predefined line/marker/color specifications. In summary, its purpose is to simplify the generation of high-quality function and/or data plots, and solving the problems of
\begin{itemize}
	\item consistency of document and font type and font size,
	\item direct use of \TeX\ math mode in axis descriptions,
	\item consistency of data and figures (no third party tool necessary),
	\item inter document consistency using preamble configurations and styles.
\end{itemize}
Although not necessary, separate |.pdf| or |.eps| graphics can be generated using the |external| library developed as part of \Tikz.

\PGFPlots\ is build completely on \Tikz/\PGF. Knowledge of \Tikz\ will simplify the work with \PGFPlots, although it is not required.

\section[About PGFPlots: Preliminaries]{About {\normalfont\PGFPlots}: Preliminaries}
This section contains information about upgrades, the team, the installation (in case you need to do it manually) and troubleshooting. You may skip it completely except for the upgrade remarks.

However, note that this library requires at least \PGF\ version $2.00$. At the time of this writing, many \TeX-distributions still contain the older \PGF\ version $1.18$, so it may be necessary to install a recent \PGF\ prior to using \PGFPlots.

\subsection{Components}
\PGFPlots\ comes with two components:
\begin{enumerate}
	\item the plotting component (which you are currently reading) and
	\item the \PGFPlotstable\ component which simplifies number formatting and postprocessing of numerical tables. It comes as a separate package and has its own manual \href{file:pgfplotstable.pdf}{pgfplotstable.pdf}.
\end{enumerate}

\subsection{Upgrade remarks}
This release provides a lot of improvements which can be found in all detail in \texttt{ChangeLog} for interested readers. However, some attention is useful with respect to the following changes.

\subsubsection{New Optional Features}
\begin{enumerate}
	\item \PGFPlots\ 1.3 comes with user interface improvements. The technical distinction between ``behavior options'' and ``style options'' of older versions is no longer necessary (although still fully supported).

	\item \PGFPlots\ 1.3 has a new feature which allows to \emph{move axis labels tight to tick labels} automatically.

	Since this affects the spacing, it is not enabled be default.

	Use
\begin{codeexample}[code only]
\usepackage{pgfplots}
\pgfplotsset{compat=1.3}
\end{codeexample}
	in your preamble to benefit from the improved spacing\footnote{The |compat=1.3| setting is necessary for new features which move axis labels around like reversed axis. However, \PGFPlots\ will still work as it does in earlier versions, your documents will remain the same if you don't set it explicitly.}.
	Take a look at the next page for the precise definition of |compat|.

	\item \PGFPlots\ 1.3 now supports reversed axes. It is no longer necessary to use work--arounds with negative units.
\pgfkeys{/pdflinks/search key prefixes in/.add={/pgfplots/,}{}}

	Take a look at the |x dir=reverse| key.

	Existing work--arounds will still function properly. Use |\pgfplotsset{compat=1.3}| together with |x dir=reverse| to switch to the new version.
\end{enumerate}

\subsubsection{Old Features Which May Need Attention}
\begin{enumerate}
	\item The |scatter/classes| feature produces proper legends as of version 1.3. This may change the appearance of existing legends of plots with |scatter/classes|.

	\item Due to a small math bug in \PGF\ $2.00$, you \emph{can't} use the math expression `|-x^2|'. It is necessary to use `|0-x^2|' instead. The same holds for
		`|exp(-x^2)|'; use `|exp(0-x^2)|' instead.

		This will be fixed with \PGF\ $>2.00$.

	\item Starting with \PGFPlots\ $1.1$, |\tikzstyle| should \emph{no longer be used} to set \PGFPlots\ options.
	
	Although |\tikzstyle| is still supported for some older \PGFPlots\ options, you should replace any occurance of |\tikzstyle| with |\pgfplotsset{|\meta{style name}|/.style={|\meta{key-value-list}|}}| or the associated |/.append style| variant. See section~\ref{sec:styles} for more detail.
\end{enumerate}
I apologize for any inconvenience caused by these changes.

\begin{pgfplotskey}{compat=\mchoice{1.3,pre 1.3,default,newest} (initially default)}
	The preamble configuration 
\begin{codeexample}[code only]
\usepackage{pgfplots}
\pgfplotsset{compat=1.3}
\end{codeexample}
	allows to choose between backwards compatibility and most recent features.

	\PGFPlots\ version 1.3 comes with new features which may lead to different spacings compared to earlier versions:
	\begin{enumerate}
		\item Version 1.3 comes with the |xlabel near ticks| feature which places (in this case $x$) axis labels ``near'' the tick labels, taking the size of tick labels into account.
		
		Older versions used |xlabel absolute|, i.e.\ an absolute distance between the axis and axis labels (now matter how large or small tick labels are).

		The initial setting \emph{keeps backwards compatibility}.

		You are encouraged to use
\begin{codeexample}[code only]
\usepackage{pgfplots}
\pgfplotsset{compat=1.3}
\end{codeexample}
		in your preamble.
		
		\item Version 1.3 fixes a bug with |anchor=south west| which occurred mainly for reversed axes: the anchors where upside--down. This is fixed now.

		
		If you have written some work--around and want to keep it going, use

		|\pgfplotsset{compat/anchors=pre 1.3}|\pgfmanualpdflabel{/pgfplots/compat/anchors}{} 

		to disable the bug fix.
	\end{enumerate}

	Use |\pgfplotsset{compat=default}| to restore the factory settings.

	The setting |\pgfplotsset{compat=newest}| will always use features of the most recent version. This might result in small changes in the document's appearance.
\end{pgfplotskey}

\subsection{The Team}
\PGFPlots\ has been written mainly by Christian Feuers�nger with many improvements of Pascal Wolkotte and Nick Papior Andersen as a spare time project. We hope it is useful and provides valuable plots.

If you are interested in writing something but don't know how, consider reading the auxiliary manual \href{file:TeX-programming-notes.pdf}{TeX-programming-notes.pdf} which comes with \PGFPlots. It is far from complete, but maybe it is a good starting point (at least for more literature).

\subsection{Acknowledgements}
I thank God for all hours of enjoyed programming. I thank Pascal Wolkotte and Nick Papior Andersen for their programming efforts and contributions as part of the development team. I thank Stefan Tibus, who contributed the |plot shell| feature. I thank Tom Cashman for the contribution of the |reverse legend| feature. Special thanks go to Stefan Pinnow whose tests of \PGFPlots\ lead to numerous quality improvements. Furthermore, I thank Dr.~Schweitzer for many fruitful discussions and Dr.~Meine for his ideas and suggestions. Thanks as well to the many international contributors who provided feature requests or identified bugs or simply manual improvements!

Last but not least, I thank Till Tantau and Mark Wibrow for their excellent graphics (and more) package \PGF\ and \Tikz\ which is the base of \PGFPlots.

%%%%%%%%%%%%%%%%%%%%%%%%%%%%%%%%%%%%%%%%%%%%%%%%%%%%%%%%%%%%%%%%%%%%%%%%%%%%%
%
% Package pgfplots.sty documentation. 
%
% Copyright 2007/2008 by Christian Feuersaenger.
%
% This program is free software: you can redistribute it and/or modify
% it under the terms of the GNU General Public License as published by
% the Free Software Foundation, either version 3 of the License, or
% (at your option) any later version.
% 
% This program is distributed in the hope that it will be useful,
% but WITHOUT ANY WARRANTY; without even the implied warranty of
% MERCHANTABILITY or FITNESS FOR A PARTICULAR PURPOSE.  See the
% GNU General Public License for more details.
% 
% You should have received a copy of the GNU General Public License
% along with this program.  If not, see <http://www.gnu.org/licenses/>.
%
%
%%%%%%%%%%%%%%%%%%%%%%%%%%%%%%%%%%%%%%%%%%%%%%%%%%%%%%%%%%%%%%%%%%%%%%%%%%%%%
\input{pgfplots.preamble.tex}
%\RequirePackage[german,english,francais]{babel}

\pgfplotsmanualexternalexpensivetrue

%\usetikzlibrary{external}
%\tikzexternalize[prefix=figures/]{pgfplots}

\title{%
	Manual for Package \PGFPlots\\
	{\small Version \pgfplotsversion}\\
	{\small\href{http://sourceforge.net/projects/pgfplots}{http://sourceforge.net/projects/pgfplots}}}

%\includeonly{pgfplots.libs}


\begin{document}

\def\plotcoords{%
\addplot coordinates {
(5,8.312e-02)    (17,2.547e-02)   (49,7.407e-03)
(129,2.102e-03)  (321,5.874e-04)  (769,1.623e-04)
(1793,4.442e-05) (4097,1.207e-05) (9217,3.261e-06)
};

\addplot coordinates{
(7,8.472e-02)    (31,3.044e-02)    (111,1.022e-02)
(351,3.303e-03)  (1023,1.039e-03)  (2815,3.196e-04)
(7423,9.658e-05) (18943,2.873e-05) (47103,8.437e-06)
};

\addplot coordinates{
(9,7.881e-02)     (49,3.243e-02)    (209,1.232e-02)
(769,4.454e-03)   (2561,1.551e-03)  (7937,5.236e-04)
(23297,1.723e-04) (65537,5.545e-05) (178177,1.751e-05)
};

\addplot coordinates{
(11,6.887e-02)    (71,3.177e-02)     (351,1.341e-02)
(1471,5.334e-03)  (5503,2.027e-03)   (18943,7.415e-04)
(61183,2.628e-04) (187903,9.063e-05) (553983,3.053e-05)
};

\addplot coordinates{
(13,5.755e-02)     (97,2.925e-02)     (545,1.351e-02)
(2561,5.842e-03)   (10625,2.397e-03)  (40193,9.414e-04)
(141569,3.564e-04) (471041,1.308e-04) 
(1496065,4.670e-05)
};
}%
\maketitle
\begin{abstract}%
\PGFPlots\ draws high--quality function plots in normal or logarithmic scaling with a user-friendly interface directly in \TeX. The user supplies axis labels, legend entries and the plot coordinates for one or more plots and \PGFPlots\ applies axis scaling, computes any logarithms and axis ticks and draws the plots, supporting line plots, scatter plots, piecewise constant plots, bar plots, area plots, mesh-- and surface plots and some more. It is based on Till Tantau's package \PGF/\Tikz.
\end{abstract}
\tableofcontents
\section{Introduction}
This package provides tools to generate plots and labeled axes easily. It draws normal plots, logplots and semi-logplots, in two and three dimensions. Axis ticks, labels, legends (in case of multiple plots) can be added with key-value options. It can cycle through a set of predefined line/marker/color specifications. In summary, its purpose is to simplify the generation of high-quality function and/or data plots, and solving the problems of
\begin{itemize}
	\item consistency of document and font type and font size,
	\item direct use of \TeX\ math mode in axis descriptions,
	\item consistency of data and figures (no third party tool necessary),
	\item inter document consistency using preamble configurations and styles.
\end{itemize}
Although not necessary, separate |.pdf| or |.eps| graphics can be generated using the |external| library developed as part of \Tikz.

\PGFPlots\ is build completely on \Tikz/\PGF. Knowledge of \Tikz\ will simplify the work with \PGFPlots, although it is not required.

\section[About PGFPlots: Preliminaries]{About {\normalfont\PGFPlots}: Preliminaries}
This section contains information about upgrades, the team, the installation (in case you need to do it manually) and troubleshooting. You may skip it completely except for the upgrade remarks.

However, note that this library requires at least \PGF\ version $2.00$. At the time of this writing, many \TeX-distributions still contain the older \PGF\ version $1.18$, so it may be necessary to install a recent \PGF\ prior to using \PGFPlots.

\subsection{Components}
\PGFPlots\ comes with two components:
\begin{enumerate}
	\item the plotting component (which you are currently reading) and
	\item the \PGFPlotstable\ component which simplifies number formatting and postprocessing of numerical tables. It comes as a separate package and has its own manual \href{file:pgfplotstable.pdf}{pgfplotstable.pdf}.
\end{enumerate}

\subsection{Upgrade remarks}
This release provides a lot of improvements which can be found in all detail in \texttt{ChangeLog} for interested readers. However, some attention is useful with respect to the following changes.

\subsubsection{New Optional Features}
\begin{enumerate}
	\item \PGFPlots\ 1.3 comes with user interface improvements. The technical distinction between ``behavior options'' and ``style options'' of older versions is no longer necessary (although still fully supported).

	\item \PGFPlots\ 1.3 has a new feature which allows to \emph{move axis labels tight to tick labels} automatically.

	Since this affects the spacing, it is not enabled be default.

	Use
\begin{codeexample}[code only]
\usepackage{pgfplots}
\pgfplotsset{compat=1.3}
\end{codeexample}
	in your preamble to benefit from the improved spacing\footnote{The |compat=1.3| setting is necessary for new features which move axis labels around like reversed axis. However, \PGFPlots\ will still work as it does in earlier versions, your documents will remain the same if you don't set it explicitly.}.
	Take a look at the next page for the precise definition of |compat|.

	\item \PGFPlots\ 1.3 now supports reversed axes. It is no longer necessary to use work--arounds with negative units.
\pgfkeys{/pdflinks/search key prefixes in/.add={/pgfplots/,}{}}

	Take a look at the |x dir=reverse| key.

	Existing work--arounds will still function properly. Use |\pgfplotsset{compat=1.3}| together with |x dir=reverse| to switch to the new version.
\end{enumerate}

\subsubsection{Old Features Which May Need Attention}
\begin{enumerate}
	\item The |scatter/classes| feature produces proper legends as of version 1.3. This may change the appearance of existing legends of plots with |scatter/classes|.

	\item Due to a small math bug in \PGF\ $2.00$, you \emph{can't} use the math expression `|-x^2|'. It is necessary to use `|0-x^2|' instead. The same holds for
		`|exp(-x^2)|'; use `|exp(0-x^2)|' instead.

		This will be fixed with \PGF\ $>2.00$.

	\item Starting with \PGFPlots\ $1.1$, |\tikzstyle| should \emph{no longer be used} to set \PGFPlots\ options.
	
	Although |\tikzstyle| is still supported for some older \PGFPlots\ options, you should replace any occurance of |\tikzstyle| with |\pgfplotsset{|\meta{style name}|/.style={|\meta{key-value-list}|}}| or the associated |/.append style| variant. See section~\ref{sec:styles} for more detail.
\end{enumerate}
I apologize for any inconvenience caused by these changes.

\begin{pgfplotskey}{compat=\mchoice{1.3,pre 1.3,default,newest} (initially default)}
	The preamble configuration 
\begin{codeexample}[code only]
\usepackage{pgfplots}
\pgfplotsset{compat=1.3}
\end{codeexample}
	allows to choose between backwards compatibility and most recent features.

	\PGFPlots\ version 1.3 comes with new features which may lead to different spacings compared to earlier versions:
	\begin{enumerate}
		\item Version 1.3 comes with the |xlabel near ticks| feature which places (in this case $x$) axis labels ``near'' the tick labels, taking the size of tick labels into account.
		
		Older versions used |xlabel absolute|, i.e.\ an absolute distance between the axis and axis labels (now matter how large or small tick labels are).

		The initial setting \emph{keeps backwards compatibility}.

		You are encouraged to use
\begin{codeexample}[code only]
\usepackage{pgfplots}
\pgfplotsset{compat=1.3}
\end{codeexample}
		in your preamble.
		
		\item Version 1.3 fixes a bug with |anchor=south west| which occurred mainly for reversed axes: the anchors where upside--down. This is fixed now.

		
		If you have written some work--around and want to keep it going, use

		|\pgfplotsset{compat/anchors=pre 1.3}|\pgfmanualpdflabel{/pgfplots/compat/anchors}{} 

		to disable the bug fix.
	\end{enumerate}

	Use |\pgfplotsset{compat=default}| to restore the factory settings.

	The setting |\pgfplotsset{compat=newest}| will always use features of the most recent version. This might result in small changes in the document's appearance.
\end{pgfplotskey}

\subsection{The Team}
\PGFPlots\ has been written mainly by Christian Feuers�nger with many improvements of Pascal Wolkotte and Nick Papior Andersen as a spare time project. We hope it is useful and provides valuable plots.

If you are interested in writing something but don't know how, consider reading the auxiliary manual \href{file:TeX-programming-notes.pdf}{TeX-programming-notes.pdf} which comes with \PGFPlots. It is far from complete, but maybe it is a good starting point (at least for more literature).

\subsection{Acknowledgements}
I thank God for all hours of enjoyed programming. I thank Pascal Wolkotte and Nick Papior Andersen for their programming efforts and contributions as part of the development team. I thank Stefan Tibus, who contributed the |plot shell| feature. I thank Tom Cashman for the contribution of the |reverse legend| feature. Special thanks go to Stefan Pinnow whose tests of \PGFPlots\ lead to numerous quality improvements. Furthermore, I thank Dr.~Schweitzer for many fruitful discussions and Dr.~Meine for his ideas and suggestions. Thanks as well to the many international contributors who provided feature requests or identified bugs or simply manual improvements!

Last but not least, I thank Till Tantau and Mark Wibrow for their excellent graphics (and more) package \PGF\ and \Tikz\ which is the base of \PGFPlots.

\include{pgfplots.install}%
\include{pgfplots.intro}%
\include{pgfplots.reference}%
\include{pgfplots.libs}%
\include{pgfplots.resources}%
\include{pgfplots.importexport}%
\include{pgfplots.basic.reference}%

\printindex

\bibliographystyle{abbrv} %gerapali} %gerabbrv} %gerunsrt.bst} %gerabbrv}% gerplain}
\nocite{pgfplotstable}
\nocite{programmingnotes}
\bibliography{pgfplots}
\end{document}
%
%%%%%%%%%%%%%%%%%%%%%%%%%%%%%%%%%%%%%%%%%%%%%%%%%%%%%%%%%%%%%%%%%%%%%%%%%%%%%
%
% Package pgfplots.sty documentation. 
%
% Copyright 2007/2008 by Christian Feuersaenger.
%
% This program is free software: you can redistribute it and/or modify
% it under the terms of the GNU General Public License as published by
% the Free Software Foundation, either version 3 of the License, or
% (at your option) any later version.
% 
% This program is distributed in the hope that it will be useful,
% but WITHOUT ANY WARRANTY; without even the implied warranty of
% MERCHANTABILITY or FITNESS FOR A PARTICULAR PURPOSE.  See the
% GNU General Public License for more details.
% 
% You should have received a copy of the GNU General Public License
% along with this program.  If not, see <http://www.gnu.org/licenses/>.
%
%
%%%%%%%%%%%%%%%%%%%%%%%%%%%%%%%%%%%%%%%%%%%%%%%%%%%%%%%%%%%%%%%%%%%%%%%%%%%%%
\input{pgfplots.preamble.tex}
%\RequirePackage[german,english,francais]{babel}

\pgfplotsmanualexternalexpensivetrue

%\usetikzlibrary{external}
%\tikzexternalize[prefix=figures/]{pgfplots}

\title{%
	Manual for Package \PGFPlots\\
	{\small Version \pgfplotsversion}\\
	{\small\href{http://sourceforge.net/projects/pgfplots}{http://sourceforge.net/projects/pgfplots}}}

%\includeonly{pgfplots.libs}


\begin{document}

\def\plotcoords{%
\addplot coordinates {
(5,8.312e-02)    (17,2.547e-02)   (49,7.407e-03)
(129,2.102e-03)  (321,5.874e-04)  (769,1.623e-04)
(1793,4.442e-05) (4097,1.207e-05) (9217,3.261e-06)
};

\addplot coordinates{
(7,8.472e-02)    (31,3.044e-02)    (111,1.022e-02)
(351,3.303e-03)  (1023,1.039e-03)  (2815,3.196e-04)
(7423,9.658e-05) (18943,2.873e-05) (47103,8.437e-06)
};

\addplot coordinates{
(9,7.881e-02)     (49,3.243e-02)    (209,1.232e-02)
(769,4.454e-03)   (2561,1.551e-03)  (7937,5.236e-04)
(23297,1.723e-04) (65537,5.545e-05) (178177,1.751e-05)
};

\addplot coordinates{
(11,6.887e-02)    (71,3.177e-02)     (351,1.341e-02)
(1471,5.334e-03)  (5503,2.027e-03)   (18943,7.415e-04)
(61183,2.628e-04) (187903,9.063e-05) (553983,3.053e-05)
};

\addplot coordinates{
(13,5.755e-02)     (97,2.925e-02)     (545,1.351e-02)
(2561,5.842e-03)   (10625,2.397e-03)  (40193,9.414e-04)
(141569,3.564e-04) (471041,1.308e-04) 
(1496065,4.670e-05)
};
}%
\maketitle
\begin{abstract}%
\PGFPlots\ draws high--quality function plots in normal or logarithmic scaling with a user-friendly interface directly in \TeX. The user supplies axis labels, legend entries and the plot coordinates for one or more plots and \PGFPlots\ applies axis scaling, computes any logarithms and axis ticks and draws the plots, supporting line plots, scatter plots, piecewise constant plots, bar plots, area plots, mesh-- and surface plots and some more. It is based on Till Tantau's package \PGF/\Tikz.
\end{abstract}
\tableofcontents
\section{Introduction}
This package provides tools to generate plots and labeled axes easily. It draws normal plots, logplots and semi-logplots, in two and three dimensions. Axis ticks, labels, legends (in case of multiple plots) can be added with key-value options. It can cycle through a set of predefined line/marker/color specifications. In summary, its purpose is to simplify the generation of high-quality function and/or data plots, and solving the problems of
\begin{itemize}
	\item consistency of document and font type and font size,
	\item direct use of \TeX\ math mode in axis descriptions,
	\item consistency of data and figures (no third party tool necessary),
	\item inter document consistency using preamble configurations and styles.
\end{itemize}
Although not necessary, separate |.pdf| or |.eps| graphics can be generated using the |external| library developed as part of \Tikz.

\PGFPlots\ is build completely on \Tikz/\PGF. Knowledge of \Tikz\ will simplify the work with \PGFPlots, although it is not required.

\section[About PGFPlots: Preliminaries]{About {\normalfont\PGFPlots}: Preliminaries}
This section contains information about upgrades, the team, the installation (in case you need to do it manually) and troubleshooting. You may skip it completely except for the upgrade remarks.

However, note that this library requires at least \PGF\ version $2.00$. At the time of this writing, many \TeX-distributions still contain the older \PGF\ version $1.18$, so it may be necessary to install a recent \PGF\ prior to using \PGFPlots.

\subsection{Components}
\PGFPlots\ comes with two components:
\begin{enumerate}
	\item the plotting component (which you are currently reading) and
	\item the \PGFPlotstable\ component which simplifies number formatting and postprocessing of numerical tables. It comes as a separate package and has its own manual \href{file:pgfplotstable.pdf}{pgfplotstable.pdf}.
\end{enumerate}

\subsection{Upgrade remarks}
This release provides a lot of improvements which can be found in all detail in \texttt{ChangeLog} for interested readers. However, some attention is useful with respect to the following changes.

\subsubsection{New Optional Features}
\begin{enumerate}
	\item \PGFPlots\ 1.3 comes with user interface improvements. The technical distinction between ``behavior options'' and ``style options'' of older versions is no longer necessary (although still fully supported).

	\item \PGFPlots\ 1.3 has a new feature which allows to \emph{move axis labels tight to tick labels} automatically.

	Since this affects the spacing, it is not enabled be default.

	Use
\begin{codeexample}[code only]
\usepackage{pgfplots}
\pgfplotsset{compat=1.3}
\end{codeexample}
	in your preamble to benefit from the improved spacing\footnote{The |compat=1.3| setting is necessary for new features which move axis labels around like reversed axis. However, \PGFPlots\ will still work as it does in earlier versions, your documents will remain the same if you don't set it explicitly.}.
	Take a look at the next page for the precise definition of |compat|.

	\item \PGFPlots\ 1.3 now supports reversed axes. It is no longer necessary to use work--arounds with negative units.
\pgfkeys{/pdflinks/search key prefixes in/.add={/pgfplots/,}{}}

	Take a look at the |x dir=reverse| key.

	Existing work--arounds will still function properly. Use |\pgfplotsset{compat=1.3}| together with |x dir=reverse| to switch to the new version.
\end{enumerate}

\subsubsection{Old Features Which May Need Attention}
\begin{enumerate}
	\item The |scatter/classes| feature produces proper legends as of version 1.3. This may change the appearance of existing legends of plots with |scatter/classes|.

	\item Due to a small math bug in \PGF\ $2.00$, you \emph{can't} use the math expression `|-x^2|'. It is necessary to use `|0-x^2|' instead. The same holds for
		`|exp(-x^2)|'; use `|exp(0-x^2)|' instead.

		This will be fixed with \PGF\ $>2.00$.

	\item Starting with \PGFPlots\ $1.1$, |\tikzstyle| should \emph{no longer be used} to set \PGFPlots\ options.
	
	Although |\tikzstyle| is still supported for some older \PGFPlots\ options, you should replace any occurance of |\tikzstyle| with |\pgfplotsset{|\meta{style name}|/.style={|\meta{key-value-list}|}}| or the associated |/.append style| variant. See section~\ref{sec:styles} for more detail.
\end{enumerate}
I apologize for any inconvenience caused by these changes.

\begin{pgfplotskey}{compat=\mchoice{1.3,pre 1.3,default,newest} (initially default)}
	The preamble configuration 
\begin{codeexample}[code only]
\usepackage{pgfplots}
\pgfplotsset{compat=1.3}
\end{codeexample}
	allows to choose between backwards compatibility and most recent features.

	\PGFPlots\ version 1.3 comes with new features which may lead to different spacings compared to earlier versions:
	\begin{enumerate}
		\item Version 1.3 comes with the |xlabel near ticks| feature which places (in this case $x$) axis labels ``near'' the tick labels, taking the size of tick labels into account.
		
		Older versions used |xlabel absolute|, i.e.\ an absolute distance between the axis and axis labels (now matter how large or small tick labels are).

		The initial setting \emph{keeps backwards compatibility}.

		You are encouraged to use
\begin{codeexample}[code only]
\usepackage{pgfplots}
\pgfplotsset{compat=1.3}
\end{codeexample}
		in your preamble.
		
		\item Version 1.3 fixes a bug with |anchor=south west| which occurred mainly for reversed axes: the anchors where upside--down. This is fixed now.

		
		If you have written some work--around and want to keep it going, use

		|\pgfplotsset{compat/anchors=pre 1.3}|\pgfmanualpdflabel{/pgfplots/compat/anchors}{} 

		to disable the bug fix.
	\end{enumerate}

	Use |\pgfplotsset{compat=default}| to restore the factory settings.

	The setting |\pgfplotsset{compat=newest}| will always use features of the most recent version. This might result in small changes in the document's appearance.
\end{pgfplotskey}

\subsection{The Team}
\PGFPlots\ has been written mainly by Christian Feuers�nger with many improvements of Pascal Wolkotte and Nick Papior Andersen as a spare time project. We hope it is useful and provides valuable plots.

If you are interested in writing something but don't know how, consider reading the auxiliary manual \href{file:TeX-programming-notes.pdf}{TeX-programming-notes.pdf} which comes with \PGFPlots. It is far from complete, but maybe it is a good starting point (at least for more literature).

\subsection{Acknowledgements}
I thank God for all hours of enjoyed programming. I thank Pascal Wolkotte and Nick Papior Andersen for their programming efforts and contributions as part of the development team. I thank Stefan Tibus, who contributed the |plot shell| feature. I thank Tom Cashman for the contribution of the |reverse legend| feature. Special thanks go to Stefan Pinnow whose tests of \PGFPlots\ lead to numerous quality improvements. Furthermore, I thank Dr.~Schweitzer for many fruitful discussions and Dr.~Meine for his ideas and suggestions. Thanks as well to the many international contributors who provided feature requests or identified bugs or simply manual improvements!

Last but not least, I thank Till Tantau and Mark Wibrow for their excellent graphics (and more) package \PGF\ and \Tikz\ which is the base of \PGFPlots.

\include{pgfplots.install}%
\include{pgfplots.intro}%
\include{pgfplots.reference}%
\include{pgfplots.libs}%
\include{pgfplots.resources}%
\include{pgfplots.importexport}%
\include{pgfplots.basic.reference}%

\printindex

\bibliographystyle{abbrv} %gerapali} %gerabbrv} %gerunsrt.bst} %gerabbrv}% gerplain}
\nocite{pgfplotstable}
\nocite{programmingnotes}
\bibliography{pgfplots}
\end{document}
%
%%%%%%%%%%%%%%%%%%%%%%%%%%%%%%%%%%%%%%%%%%%%%%%%%%%%%%%%%%%%%%%%%%%%%%%%%%%%%
%
% Package pgfplots.sty documentation. 
%
% Copyright 2007/2008 by Christian Feuersaenger.
%
% This program is free software: you can redistribute it and/or modify
% it under the terms of the GNU General Public License as published by
% the Free Software Foundation, either version 3 of the License, or
% (at your option) any later version.
% 
% This program is distributed in the hope that it will be useful,
% but WITHOUT ANY WARRANTY; without even the implied warranty of
% MERCHANTABILITY or FITNESS FOR A PARTICULAR PURPOSE.  See the
% GNU General Public License for more details.
% 
% You should have received a copy of the GNU General Public License
% along with this program.  If not, see <http://www.gnu.org/licenses/>.
%
%
%%%%%%%%%%%%%%%%%%%%%%%%%%%%%%%%%%%%%%%%%%%%%%%%%%%%%%%%%%%%%%%%%%%%%%%%%%%%%
\input{pgfplots.preamble.tex}
%\RequirePackage[german,english,francais]{babel}

\pgfplotsmanualexternalexpensivetrue

%\usetikzlibrary{external}
%\tikzexternalize[prefix=figures/]{pgfplots}

\title{%
	Manual for Package \PGFPlots\\
	{\small Version \pgfplotsversion}\\
	{\small\href{http://sourceforge.net/projects/pgfplots}{http://sourceforge.net/projects/pgfplots}}}

%\includeonly{pgfplots.libs}


\begin{document}

\def\plotcoords{%
\addplot coordinates {
(5,8.312e-02)    (17,2.547e-02)   (49,7.407e-03)
(129,2.102e-03)  (321,5.874e-04)  (769,1.623e-04)
(1793,4.442e-05) (4097,1.207e-05) (9217,3.261e-06)
};

\addplot coordinates{
(7,8.472e-02)    (31,3.044e-02)    (111,1.022e-02)
(351,3.303e-03)  (1023,1.039e-03)  (2815,3.196e-04)
(7423,9.658e-05) (18943,2.873e-05) (47103,8.437e-06)
};

\addplot coordinates{
(9,7.881e-02)     (49,3.243e-02)    (209,1.232e-02)
(769,4.454e-03)   (2561,1.551e-03)  (7937,5.236e-04)
(23297,1.723e-04) (65537,5.545e-05) (178177,1.751e-05)
};

\addplot coordinates{
(11,6.887e-02)    (71,3.177e-02)     (351,1.341e-02)
(1471,5.334e-03)  (5503,2.027e-03)   (18943,7.415e-04)
(61183,2.628e-04) (187903,9.063e-05) (553983,3.053e-05)
};

\addplot coordinates{
(13,5.755e-02)     (97,2.925e-02)     (545,1.351e-02)
(2561,5.842e-03)   (10625,2.397e-03)  (40193,9.414e-04)
(141569,3.564e-04) (471041,1.308e-04) 
(1496065,4.670e-05)
};
}%
\maketitle
\begin{abstract}%
\PGFPlots\ draws high--quality function plots in normal or logarithmic scaling with a user-friendly interface directly in \TeX. The user supplies axis labels, legend entries and the plot coordinates for one or more plots and \PGFPlots\ applies axis scaling, computes any logarithms and axis ticks and draws the plots, supporting line plots, scatter plots, piecewise constant plots, bar plots, area plots, mesh-- and surface plots and some more. It is based on Till Tantau's package \PGF/\Tikz.
\end{abstract}
\tableofcontents
\section{Introduction}
This package provides tools to generate plots and labeled axes easily. It draws normal plots, logplots and semi-logplots, in two and three dimensions. Axis ticks, labels, legends (in case of multiple plots) can be added with key-value options. It can cycle through a set of predefined line/marker/color specifications. In summary, its purpose is to simplify the generation of high-quality function and/or data plots, and solving the problems of
\begin{itemize}
	\item consistency of document and font type and font size,
	\item direct use of \TeX\ math mode in axis descriptions,
	\item consistency of data and figures (no third party tool necessary),
	\item inter document consistency using preamble configurations and styles.
\end{itemize}
Although not necessary, separate |.pdf| or |.eps| graphics can be generated using the |external| library developed as part of \Tikz.

\PGFPlots\ is build completely on \Tikz/\PGF. Knowledge of \Tikz\ will simplify the work with \PGFPlots, although it is not required.

\section[About PGFPlots: Preliminaries]{About {\normalfont\PGFPlots}: Preliminaries}
This section contains information about upgrades, the team, the installation (in case you need to do it manually) and troubleshooting. You may skip it completely except for the upgrade remarks.

However, note that this library requires at least \PGF\ version $2.00$. At the time of this writing, many \TeX-distributions still contain the older \PGF\ version $1.18$, so it may be necessary to install a recent \PGF\ prior to using \PGFPlots.

\subsection{Components}
\PGFPlots\ comes with two components:
\begin{enumerate}
	\item the plotting component (which you are currently reading) and
	\item the \PGFPlotstable\ component which simplifies number formatting and postprocessing of numerical tables. It comes as a separate package and has its own manual \href{file:pgfplotstable.pdf}{pgfplotstable.pdf}.
\end{enumerate}

\subsection{Upgrade remarks}
This release provides a lot of improvements which can be found in all detail in \texttt{ChangeLog} for interested readers. However, some attention is useful with respect to the following changes.

\subsubsection{New Optional Features}
\begin{enumerate}
	\item \PGFPlots\ 1.3 comes with user interface improvements. The technical distinction between ``behavior options'' and ``style options'' of older versions is no longer necessary (although still fully supported).

	\item \PGFPlots\ 1.3 has a new feature which allows to \emph{move axis labels tight to tick labels} automatically.

	Since this affects the spacing, it is not enabled be default.

	Use
\begin{codeexample}[code only]
\usepackage{pgfplots}
\pgfplotsset{compat=1.3}
\end{codeexample}
	in your preamble to benefit from the improved spacing\footnote{The |compat=1.3| setting is necessary for new features which move axis labels around like reversed axis. However, \PGFPlots\ will still work as it does in earlier versions, your documents will remain the same if you don't set it explicitly.}.
	Take a look at the next page for the precise definition of |compat|.

	\item \PGFPlots\ 1.3 now supports reversed axes. It is no longer necessary to use work--arounds with negative units.
\pgfkeys{/pdflinks/search key prefixes in/.add={/pgfplots/,}{}}

	Take a look at the |x dir=reverse| key.

	Existing work--arounds will still function properly. Use |\pgfplotsset{compat=1.3}| together with |x dir=reverse| to switch to the new version.
\end{enumerate}

\subsubsection{Old Features Which May Need Attention}
\begin{enumerate}
	\item The |scatter/classes| feature produces proper legends as of version 1.3. This may change the appearance of existing legends of plots with |scatter/classes|.

	\item Due to a small math bug in \PGF\ $2.00$, you \emph{can't} use the math expression `|-x^2|'. It is necessary to use `|0-x^2|' instead. The same holds for
		`|exp(-x^2)|'; use `|exp(0-x^2)|' instead.

		This will be fixed with \PGF\ $>2.00$.

	\item Starting with \PGFPlots\ $1.1$, |\tikzstyle| should \emph{no longer be used} to set \PGFPlots\ options.
	
	Although |\tikzstyle| is still supported for some older \PGFPlots\ options, you should replace any occurance of |\tikzstyle| with |\pgfplotsset{|\meta{style name}|/.style={|\meta{key-value-list}|}}| or the associated |/.append style| variant. See section~\ref{sec:styles} for more detail.
\end{enumerate}
I apologize for any inconvenience caused by these changes.

\begin{pgfplotskey}{compat=\mchoice{1.3,pre 1.3,default,newest} (initially default)}
	The preamble configuration 
\begin{codeexample}[code only]
\usepackage{pgfplots}
\pgfplotsset{compat=1.3}
\end{codeexample}
	allows to choose between backwards compatibility and most recent features.

	\PGFPlots\ version 1.3 comes with new features which may lead to different spacings compared to earlier versions:
	\begin{enumerate}
		\item Version 1.3 comes with the |xlabel near ticks| feature which places (in this case $x$) axis labels ``near'' the tick labels, taking the size of tick labels into account.
		
		Older versions used |xlabel absolute|, i.e.\ an absolute distance between the axis and axis labels (now matter how large or small tick labels are).

		The initial setting \emph{keeps backwards compatibility}.

		You are encouraged to use
\begin{codeexample}[code only]
\usepackage{pgfplots}
\pgfplotsset{compat=1.3}
\end{codeexample}
		in your preamble.
		
		\item Version 1.3 fixes a bug with |anchor=south west| which occurred mainly for reversed axes: the anchors where upside--down. This is fixed now.

		
		If you have written some work--around and want to keep it going, use

		|\pgfplotsset{compat/anchors=pre 1.3}|\pgfmanualpdflabel{/pgfplots/compat/anchors}{} 

		to disable the bug fix.
	\end{enumerate}

	Use |\pgfplotsset{compat=default}| to restore the factory settings.

	The setting |\pgfplotsset{compat=newest}| will always use features of the most recent version. This might result in small changes in the document's appearance.
\end{pgfplotskey}

\subsection{The Team}
\PGFPlots\ has been written mainly by Christian Feuers�nger with many improvements of Pascal Wolkotte and Nick Papior Andersen as a spare time project. We hope it is useful and provides valuable plots.

If you are interested in writing something but don't know how, consider reading the auxiliary manual \href{file:TeX-programming-notes.pdf}{TeX-programming-notes.pdf} which comes with \PGFPlots. It is far from complete, but maybe it is a good starting point (at least for more literature).

\subsection{Acknowledgements}
I thank God for all hours of enjoyed programming. I thank Pascal Wolkotte and Nick Papior Andersen for their programming efforts and contributions as part of the development team. I thank Stefan Tibus, who contributed the |plot shell| feature. I thank Tom Cashman for the contribution of the |reverse legend| feature. Special thanks go to Stefan Pinnow whose tests of \PGFPlots\ lead to numerous quality improvements. Furthermore, I thank Dr.~Schweitzer for many fruitful discussions and Dr.~Meine for his ideas and suggestions. Thanks as well to the many international contributors who provided feature requests or identified bugs or simply manual improvements!

Last but not least, I thank Till Tantau and Mark Wibrow for their excellent graphics (and more) package \PGF\ and \Tikz\ which is the base of \PGFPlots.

\include{pgfplots.install}%
\include{pgfplots.intro}%
\include{pgfplots.reference}%
\include{pgfplots.libs}%
\include{pgfplots.resources}%
\include{pgfplots.importexport}%
\include{pgfplots.basic.reference}%

\printindex

\bibliographystyle{abbrv} %gerapali} %gerabbrv} %gerunsrt.bst} %gerabbrv}% gerplain}
\nocite{pgfplotstable}
\nocite{programmingnotes}
\bibliography{pgfplots}
\end{document}
%
%%%%%%%%%%%%%%%%%%%%%%%%%%%%%%%%%%%%%%%%%%%%%%%%%%%%%%%%%%%%%%%%%%%%%%%%%%%%%
%
% Package pgfplots.sty documentation. 
%
% Copyright 2007/2008 by Christian Feuersaenger.
%
% This program is free software: you can redistribute it and/or modify
% it under the terms of the GNU General Public License as published by
% the Free Software Foundation, either version 3 of the License, or
% (at your option) any later version.
% 
% This program is distributed in the hope that it will be useful,
% but WITHOUT ANY WARRANTY; without even the implied warranty of
% MERCHANTABILITY or FITNESS FOR A PARTICULAR PURPOSE.  See the
% GNU General Public License for more details.
% 
% You should have received a copy of the GNU General Public License
% along with this program.  If not, see <http://www.gnu.org/licenses/>.
%
%
%%%%%%%%%%%%%%%%%%%%%%%%%%%%%%%%%%%%%%%%%%%%%%%%%%%%%%%%%%%%%%%%%%%%%%%%%%%%%
\input{pgfplots.preamble.tex}
%\RequirePackage[german,english,francais]{babel}

\pgfplotsmanualexternalexpensivetrue

%\usetikzlibrary{external}
%\tikzexternalize[prefix=figures/]{pgfplots}

\title{%
	Manual for Package \PGFPlots\\
	{\small Version \pgfplotsversion}\\
	{\small\href{http://sourceforge.net/projects/pgfplots}{http://sourceforge.net/projects/pgfplots}}}

%\includeonly{pgfplots.libs}


\begin{document}

\def\plotcoords{%
\addplot coordinates {
(5,8.312e-02)    (17,2.547e-02)   (49,7.407e-03)
(129,2.102e-03)  (321,5.874e-04)  (769,1.623e-04)
(1793,4.442e-05) (4097,1.207e-05) (9217,3.261e-06)
};

\addplot coordinates{
(7,8.472e-02)    (31,3.044e-02)    (111,1.022e-02)
(351,3.303e-03)  (1023,1.039e-03)  (2815,3.196e-04)
(7423,9.658e-05) (18943,2.873e-05) (47103,8.437e-06)
};

\addplot coordinates{
(9,7.881e-02)     (49,3.243e-02)    (209,1.232e-02)
(769,4.454e-03)   (2561,1.551e-03)  (7937,5.236e-04)
(23297,1.723e-04) (65537,5.545e-05) (178177,1.751e-05)
};

\addplot coordinates{
(11,6.887e-02)    (71,3.177e-02)     (351,1.341e-02)
(1471,5.334e-03)  (5503,2.027e-03)   (18943,7.415e-04)
(61183,2.628e-04) (187903,9.063e-05) (553983,3.053e-05)
};

\addplot coordinates{
(13,5.755e-02)     (97,2.925e-02)     (545,1.351e-02)
(2561,5.842e-03)   (10625,2.397e-03)  (40193,9.414e-04)
(141569,3.564e-04) (471041,1.308e-04) 
(1496065,4.670e-05)
};
}%
\maketitle
\begin{abstract}%
\PGFPlots\ draws high--quality function plots in normal or logarithmic scaling with a user-friendly interface directly in \TeX. The user supplies axis labels, legend entries and the plot coordinates for one or more plots and \PGFPlots\ applies axis scaling, computes any logarithms and axis ticks and draws the plots, supporting line plots, scatter plots, piecewise constant plots, bar plots, area plots, mesh-- and surface plots and some more. It is based on Till Tantau's package \PGF/\Tikz.
\end{abstract}
\tableofcontents
\section{Introduction}
This package provides tools to generate plots and labeled axes easily. It draws normal plots, logplots and semi-logplots, in two and three dimensions. Axis ticks, labels, legends (in case of multiple plots) can be added with key-value options. It can cycle through a set of predefined line/marker/color specifications. In summary, its purpose is to simplify the generation of high-quality function and/or data plots, and solving the problems of
\begin{itemize}
	\item consistency of document and font type and font size,
	\item direct use of \TeX\ math mode in axis descriptions,
	\item consistency of data and figures (no third party tool necessary),
	\item inter document consistency using preamble configurations and styles.
\end{itemize}
Although not necessary, separate |.pdf| or |.eps| graphics can be generated using the |external| library developed as part of \Tikz.

\PGFPlots\ is build completely on \Tikz/\PGF. Knowledge of \Tikz\ will simplify the work with \PGFPlots, although it is not required.

\section[About PGFPlots: Preliminaries]{About {\normalfont\PGFPlots}: Preliminaries}
This section contains information about upgrades, the team, the installation (in case you need to do it manually) and troubleshooting. You may skip it completely except for the upgrade remarks.

However, note that this library requires at least \PGF\ version $2.00$. At the time of this writing, many \TeX-distributions still contain the older \PGF\ version $1.18$, so it may be necessary to install a recent \PGF\ prior to using \PGFPlots.

\subsection{Components}
\PGFPlots\ comes with two components:
\begin{enumerate}
	\item the plotting component (which you are currently reading) and
	\item the \PGFPlotstable\ component which simplifies number formatting and postprocessing of numerical tables. It comes as a separate package and has its own manual \href{file:pgfplotstable.pdf}{pgfplotstable.pdf}.
\end{enumerate}

\subsection{Upgrade remarks}
This release provides a lot of improvements which can be found in all detail in \texttt{ChangeLog} for interested readers. However, some attention is useful with respect to the following changes.

\subsubsection{New Optional Features}
\begin{enumerate}
	\item \PGFPlots\ 1.3 comes with user interface improvements. The technical distinction between ``behavior options'' and ``style options'' of older versions is no longer necessary (although still fully supported).

	\item \PGFPlots\ 1.3 has a new feature which allows to \emph{move axis labels tight to tick labels} automatically.

	Since this affects the spacing, it is not enabled be default.

	Use
\begin{codeexample}[code only]
\usepackage{pgfplots}
\pgfplotsset{compat=1.3}
\end{codeexample}
	in your preamble to benefit from the improved spacing\footnote{The |compat=1.3| setting is necessary for new features which move axis labels around like reversed axis. However, \PGFPlots\ will still work as it does in earlier versions, your documents will remain the same if you don't set it explicitly.}.
	Take a look at the next page for the precise definition of |compat|.

	\item \PGFPlots\ 1.3 now supports reversed axes. It is no longer necessary to use work--arounds with negative units.
\pgfkeys{/pdflinks/search key prefixes in/.add={/pgfplots/,}{}}

	Take a look at the |x dir=reverse| key.

	Existing work--arounds will still function properly. Use |\pgfplotsset{compat=1.3}| together with |x dir=reverse| to switch to the new version.
\end{enumerate}

\subsubsection{Old Features Which May Need Attention}
\begin{enumerate}
	\item The |scatter/classes| feature produces proper legends as of version 1.3. This may change the appearance of existing legends of plots with |scatter/classes|.

	\item Due to a small math bug in \PGF\ $2.00$, you \emph{can't} use the math expression `|-x^2|'. It is necessary to use `|0-x^2|' instead. The same holds for
		`|exp(-x^2)|'; use `|exp(0-x^2)|' instead.

		This will be fixed with \PGF\ $>2.00$.

	\item Starting with \PGFPlots\ $1.1$, |\tikzstyle| should \emph{no longer be used} to set \PGFPlots\ options.
	
	Although |\tikzstyle| is still supported for some older \PGFPlots\ options, you should replace any occurance of |\tikzstyle| with |\pgfplotsset{|\meta{style name}|/.style={|\meta{key-value-list}|}}| or the associated |/.append style| variant. See section~\ref{sec:styles} for more detail.
\end{enumerate}
I apologize for any inconvenience caused by these changes.

\begin{pgfplotskey}{compat=\mchoice{1.3,pre 1.3,default,newest} (initially default)}
	The preamble configuration 
\begin{codeexample}[code only]
\usepackage{pgfplots}
\pgfplotsset{compat=1.3}
\end{codeexample}
	allows to choose between backwards compatibility and most recent features.

	\PGFPlots\ version 1.3 comes with new features which may lead to different spacings compared to earlier versions:
	\begin{enumerate}
		\item Version 1.3 comes with the |xlabel near ticks| feature which places (in this case $x$) axis labels ``near'' the tick labels, taking the size of tick labels into account.
		
		Older versions used |xlabel absolute|, i.e.\ an absolute distance between the axis and axis labels (now matter how large or small tick labels are).

		The initial setting \emph{keeps backwards compatibility}.

		You are encouraged to use
\begin{codeexample}[code only]
\usepackage{pgfplots}
\pgfplotsset{compat=1.3}
\end{codeexample}
		in your preamble.
		
		\item Version 1.3 fixes a bug with |anchor=south west| which occurred mainly for reversed axes: the anchors where upside--down. This is fixed now.

		
		If you have written some work--around and want to keep it going, use

		|\pgfplotsset{compat/anchors=pre 1.3}|\pgfmanualpdflabel{/pgfplots/compat/anchors}{} 

		to disable the bug fix.
	\end{enumerate}

	Use |\pgfplotsset{compat=default}| to restore the factory settings.

	The setting |\pgfplotsset{compat=newest}| will always use features of the most recent version. This might result in small changes in the document's appearance.
\end{pgfplotskey}

\subsection{The Team}
\PGFPlots\ has been written mainly by Christian Feuers�nger with many improvements of Pascal Wolkotte and Nick Papior Andersen as a spare time project. We hope it is useful and provides valuable plots.

If you are interested in writing something but don't know how, consider reading the auxiliary manual \href{file:TeX-programming-notes.pdf}{TeX-programming-notes.pdf} which comes with \PGFPlots. It is far from complete, but maybe it is a good starting point (at least for more literature).

\subsection{Acknowledgements}
I thank God for all hours of enjoyed programming. I thank Pascal Wolkotte and Nick Papior Andersen for their programming efforts and contributions as part of the development team. I thank Stefan Tibus, who contributed the |plot shell| feature. I thank Tom Cashman for the contribution of the |reverse legend| feature. Special thanks go to Stefan Pinnow whose tests of \PGFPlots\ lead to numerous quality improvements. Furthermore, I thank Dr.~Schweitzer for many fruitful discussions and Dr.~Meine for his ideas and suggestions. Thanks as well to the many international contributors who provided feature requests or identified bugs or simply manual improvements!

Last but not least, I thank Till Tantau and Mark Wibrow for their excellent graphics (and more) package \PGF\ and \Tikz\ which is the base of \PGFPlots.

\include{pgfplots.install}%
\include{pgfplots.intro}%
\include{pgfplots.reference}%
\include{pgfplots.libs}%
\include{pgfplots.resources}%
\include{pgfplots.importexport}%
\include{pgfplots.basic.reference}%

\printindex

\bibliographystyle{abbrv} %gerapali} %gerabbrv} %gerunsrt.bst} %gerabbrv}% gerplain}
\nocite{pgfplotstable}
\nocite{programmingnotes}
\bibliography{pgfplots}
\end{document}
%
%%%%%%%%%%%%%%%%%%%%%%%%%%%%%%%%%%%%%%%%%%%%%%%%%%%%%%%%%%%%%%%%%%%%%%%%%%%%%
%
% Package pgfplots.sty documentation. 
%
% Copyright 2007/2008 by Christian Feuersaenger.
%
% This program is free software: you can redistribute it and/or modify
% it under the terms of the GNU General Public License as published by
% the Free Software Foundation, either version 3 of the License, or
% (at your option) any later version.
% 
% This program is distributed in the hope that it will be useful,
% but WITHOUT ANY WARRANTY; without even the implied warranty of
% MERCHANTABILITY or FITNESS FOR A PARTICULAR PURPOSE.  See the
% GNU General Public License for more details.
% 
% You should have received a copy of the GNU General Public License
% along with this program.  If not, see <http://www.gnu.org/licenses/>.
%
%
%%%%%%%%%%%%%%%%%%%%%%%%%%%%%%%%%%%%%%%%%%%%%%%%%%%%%%%%%%%%%%%%%%%%%%%%%%%%%
\input{pgfplots.preamble.tex}
%\RequirePackage[german,english,francais]{babel}

\pgfplotsmanualexternalexpensivetrue

%\usetikzlibrary{external}
%\tikzexternalize[prefix=figures/]{pgfplots}

\title{%
	Manual for Package \PGFPlots\\
	{\small Version \pgfplotsversion}\\
	{\small\href{http://sourceforge.net/projects/pgfplots}{http://sourceforge.net/projects/pgfplots}}}

%\includeonly{pgfplots.libs}


\begin{document}

\def\plotcoords{%
\addplot coordinates {
(5,8.312e-02)    (17,2.547e-02)   (49,7.407e-03)
(129,2.102e-03)  (321,5.874e-04)  (769,1.623e-04)
(1793,4.442e-05) (4097,1.207e-05) (9217,3.261e-06)
};

\addplot coordinates{
(7,8.472e-02)    (31,3.044e-02)    (111,1.022e-02)
(351,3.303e-03)  (1023,1.039e-03)  (2815,3.196e-04)
(7423,9.658e-05) (18943,2.873e-05) (47103,8.437e-06)
};

\addplot coordinates{
(9,7.881e-02)     (49,3.243e-02)    (209,1.232e-02)
(769,4.454e-03)   (2561,1.551e-03)  (7937,5.236e-04)
(23297,1.723e-04) (65537,5.545e-05) (178177,1.751e-05)
};

\addplot coordinates{
(11,6.887e-02)    (71,3.177e-02)     (351,1.341e-02)
(1471,5.334e-03)  (5503,2.027e-03)   (18943,7.415e-04)
(61183,2.628e-04) (187903,9.063e-05) (553983,3.053e-05)
};

\addplot coordinates{
(13,5.755e-02)     (97,2.925e-02)     (545,1.351e-02)
(2561,5.842e-03)   (10625,2.397e-03)  (40193,9.414e-04)
(141569,3.564e-04) (471041,1.308e-04) 
(1496065,4.670e-05)
};
}%
\maketitle
\begin{abstract}%
\PGFPlots\ draws high--quality function plots in normal or logarithmic scaling with a user-friendly interface directly in \TeX. The user supplies axis labels, legend entries and the plot coordinates for one or more plots and \PGFPlots\ applies axis scaling, computes any logarithms and axis ticks and draws the plots, supporting line plots, scatter plots, piecewise constant plots, bar plots, area plots, mesh-- and surface plots and some more. It is based on Till Tantau's package \PGF/\Tikz.
\end{abstract}
\tableofcontents
\section{Introduction}
This package provides tools to generate plots and labeled axes easily. It draws normal plots, logplots and semi-logplots, in two and three dimensions. Axis ticks, labels, legends (in case of multiple plots) can be added with key-value options. It can cycle through a set of predefined line/marker/color specifications. In summary, its purpose is to simplify the generation of high-quality function and/or data plots, and solving the problems of
\begin{itemize}
	\item consistency of document and font type and font size,
	\item direct use of \TeX\ math mode in axis descriptions,
	\item consistency of data and figures (no third party tool necessary),
	\item inter document consistency using preamble configurations and styles.
\end{itemize}
Although not necessary, separate |.pdf| or |.eps| graphics can be generated using the |external| library developed as part of \Tikz.

\PGFPlots\ is build completely on \Tikz/\PGF. Knowledge of \Tikz\ will simplify the work with \PGFPlots, although it is not required.

\section[About PGFPlots: Preliminaries]{About {\normalfont\PGFPlots}: Preliminaries}
This section contains information about upgrades, the team, the installation (in case you need to do it manually) and troubleshooting. You may skip it completely except for the upgrade remarks.

However, note that this library requires at least \PGF\ version $2.00$. At the time of this writing, many \TeX-distributions still contain the older \PGF\ version $1.18$, so it may be necessary to install a recent \PGF\ prior to using \PGFPlots.

\subsection{Components}
\PGFPlots\ comes with two components:
\begin{enumerate}
	\item the plotting component (which you are currently reading) and
	\item the \PGFPlotstable\ component which simplifies number formatting and postprocessing of numerical tables. It comes as a separate package and has its own manual \href{file:pgfplotstable.pdf}{pgfplotstable.pdf}.
\end{enumerate}

\subsection{Upgrade remarks}
This release provides a lot of improvements which can be found in all detail in \texttt{ChangeLog} for interested readers. However, some attention is useful with respect to the following changes.

\subsubsection{New Optional Features}
\begin{enumerate}
	\item \PGFPlots\ 1.3 comes with user interface improvements. The technical distinction between ``behavior options'' and ``style options'' of older versions is no longer necessary (although still fully supported).

	\item \PGFPlots\ 1.3 has a new feature which allows to \emph{move axis labels tight to tick labels} automatically.

	Since this affects the spacing, it is not enabled be default.

	Use
\begin{codeexample}[code only]
\usepackage{pgfplots}
\pgfplotsset{compat=1.3}
\end{codeexample}
	in your preamble to benefit from the improved spacing\footnote{The |compat=1.3| setting is necessary for new features which move axis labels around like reversed axis. However, \PGFPlots\ will still work as it does in earlier versions, your documents will remain the same if you don't set it explicitly.}.
	Take a look at the next page for the precise definition of |compat|.

	\item \PGFPlots\ 1.3 now supports reversed axes. It is no longer necessary to use work--arounds with negative units.
\pgfkeys{/pdflinks/search key prefixes in/.add={/pgfplots/,}{}}

	Take a look at the |x dir=reverse| key.

	Existing work--arounds will still function properly. Use |\pgfplotsset{compat=1.3}| together with |x dir=reverse| to switch to the new version.
\end{enumerate}

\subsubsection{Old Features Which May Need Attention}
\begin{enumerate}
	\item The |scatter/classes| feature produces proper legends as of version 1.3. This may change the appearance of existing legends of plots with |scatter/classes|.

	\item Due to a small math bug in \PGF\ $2.00$, you \emph{can't} use the math expression `|-x^2|'. It is necessary to use `|0-x^2|' instead. The same holds for
		`|exp(-x^2)|'; use `|exp(0-x^2)|' instead.

		This will be fixed with \PGF\ $>2.00$.

	\item Starting with \PGFPlots\ $1.1$, |\tikzstyle| should \emph{no longer be used} to set \PGFPlots\ options.
	
	Although |\tikzstyle| is still supported for some older \PGFPlots\ options, you should replace any occurance of |\tikzstyle| with |\pgfplotsset{|\meta{style name}|/.style={|\meta{key-value-list}|}}| or the associated |/.append style| variant. See section~\ref{sec:styles} for more detail.
\end{enumerate}
I apologize for any inconvenience caused by these changes.

\begin{pgfplotskey}{compat=\mchoice{1.3,pre 1.3,default,newest} (initially default)}
	The preamble configuration 
\begin{codeexample}[code only]
\usepackage{pgfplots}
\pgfplotsset{compat=1.3}
\end{codeexample}
	allows to choose between backwards compatibility and most recent features.

	\PGFPlots\ version 1.3 comes with new features which may lead to different spacings compared to earlier versions:
	\begin{enumerate}
		\item Version 1.3 comes with the |xlabel near ticks| feature which places (in this case $x$) axis labels ``near'' the tick labels, taking the size of tick labels into account.
		
		Older versions used |xlabel absolute|, i.e.\ an absolute distance between the axis and axis labels (now matter how large or small tick labels are).

		The initial setting \emph{keeps backwards compatibility}.

		You are encouraged to use
\begin{codeexample}[code only]
\usepackage{pgfplots}
\pgfplotsset{compat=1.3}
\end{codeexample}
		in your preamble.
		
		\item Version 1.3 fixes a bug with |anchor=south west| which occurred mainly for reversed axes: the anchors where upside--down. This is fixed now.

		
		If you have written some work--around and want to keep it going, use

		|\pgfplotsset{compat/anchors=pre 1.3}|\pgfmanualpdflabel{/pgfplots/compat/anchors}{} 

		to disable the bug fix.
	\end{enumerate}

	Use |\pgfplotsset{compat=default}| to restore the factory settings.

	The setting |\pgfplotsset{compat=newest}| will always use features of the most recent version. This might result in small changes in the document's appearance.
\end{pgfplotskey}

\subsection{The Team}
\PGFPlots\ has been written mainly by Christian Feuers�nger with many improvements of Pascal Wolkotte and Nick Papior Andersen as a spare time project. We hope it is useful and provides valuable plots.

If you are interested in writing something but don't know how, consider reading the auxiliary manual \href{file:TeX-programming-notes.pdf}{TeX-programming-notes.pdf} which comes with \PGFPlots. It is far from complete, but maybe it is a good starting point (at least for more literature).

\subsection{Acknowledgements}
I thank God for all hours of enjoyed programming. I thank Pascal Wolkotte and Nick Papior Andersen for their programming efforts and contributions as part of the development team. I thank Stefan Tibus, who contributed the |plot shell| feature. I thank Tom Cashman for the contribution of the |reverse legend| feature. Special thanks go to Stefan Pinnow whose tests of \PGFPlots\ lead to numerous quality improvements. Furthermore, I thank Dr.~Schweitzer for many fruitful discussions and Dr.~Meine for his ideas and suggestions. Thanks as well to the many international contributors who provided feature requests or identified bugs or simply manual improvements!

Last but not least, I thank Till Tantau and Mark Wibrow for their excellent graphics (and more) package \PGF\ and \Tikz\ which is the base of \PGFPlots.

\include{pgfplots.install}%
\include{pgfplots.intro}%
\include{pgfplots.reference}%
\include{pgfplots.libs}%
\include{pgfplots.resources}%
\include{pgfplots.importexport}%
\include{pgfplots.basic.reference}%

\printindex

\bibliographystyle{abbrv} %gerapali} %gerabbrv} %gerunsrt.bst} %gerabbrv}% gerplain}
\nocite{pgfplotstable}
\nocite{programmingnotes}
\bibliography{pgfplots}
\end{document}
%
%%%%%%%%%%%%%%%%%%%%%%%%%%%%%%%%%%%%%%%%%%%%%%%%%%%%%%%%%%%%%%%%%%%%%%%%%%%%%
%
% Package pgfplots.sty documentation. 
%
% Copyright 2007/2008 by Christian Feuersaenger.
%
% This program is free software: you can redistribute it and/or modify
% it under the terms of the GNU General Public License as published by
% the Free Software Foundation, either version 3 of the License, or
% (at your option) any later version.
% 
% This program is distributed in the hope that it will be useful,
% but WITHOUT ANY WARRANTY; without even the implied warranty of
% MERCHANTABILITY or FITNESS FOR A PARTICULAR PURPOSE.  See the
% GNU General Public License for more details.
% 
% You should have received a copy of the GNU General Public License
% along with this program.  If not, see <http://www.gnu.org/licenses/>.
%
%
%%%%%%%%%%%%%%%%%%%%%%%%%%%%%%%%%%%%%%%%%%%%%%%%%%%%%%%%%%%%%%%%%%%%%%%%%%%%%
\input{pgfplots.preamble.tex}
%\RequirePackage[german,english,francais]{babel}

\pgfplotsmanualexternalexpensivetrue

%\usetikzlibrary{external}
%\tikzexternalize[prefix=figures/]{pgfplots}

\title{%
	Manual for Package \PGFPlots\\
	{\small Version \pgfplotsversion}\\
	{\small\href{http://sourceforge.net/projects/pgfplots}{http://sourceforge.net/projects/pgfplots}}}

%\includeonly{pgfplots.libs}


\begin{document}

\def\plotcoords{%
\addplot coordinates {
(5,8.312e-02)    (17,2.547e-02)   (49,7.407e-03)
(129,2.102e-03)  (321,5.874e-04)  (769,1.623e-04)
(1793,4.442e-05) (4097,1.207e-05) (9217,3.261e-06)
};

\addplot coordinates{
(7,8.472e-02)    (31,3.044e-02)    (111,1.022e-02)
(351,3.303e-03)  (1023,1.039e-03)  (2815,3.196e-04)
(7423,9.658e-05) (18943,2.873e-05) (47103,8.437e-06)
};

\addplot coordinates{
(9,7.881e-02)     (49,3.243e-02)    (209,1.232e-02)
(769,4.454e-03)   (2561,1.551e-03)  (7937,5.236e-04)
(23297,1.723e-04) (65537,5.545e-05) (178177,1.751e-05)
};

\addplot coordinates{
(11,6.887e-02)    (71,3.177e-02)     (351,1.341e-02)
(1471,5.334e-03)  (5503,2.027e-03)   (18943,7.415e-04)
(61183,2.628e-04) (187903,9.063e-05) (553983,3.053e-05)
};

\addplot coordinates{
(13,5.755e-02)     (97,2.925e-02)     (545,1.351e-02)
(2561,5.842e-03)   (10625,2.397e-03)  (40193,9.414e-04)
(141569,3.564e-04) (471041,1.308e-04) 
(1496065,4.670e-05)
};
}%
\maketitle
\begin{abstract}%
\PGFPlots\ draws high--quality function plots in normal or logarithmic scaling with a user-friendly interface directly in \TeX. The user supplies axis labels, legend entries and the plot coordinates for one or more plots and \PGFPlots\ applies axis scaling, computes any logarithms and axis ticks and draws the plots, supporting line plots, scatter plots, piecewise constant plots, bar plots, area plots, mesh-- and surface plots and some more. It is based on Till Tantau's package \PGF/\Tikz.
\end{abstract}
\tableofcontents
\section{Introduction}
This package provides tools to generate plots and labeled axes easily. It draws normal plots, logplots and semi-logplots, in two and three dimensions. Axis ticks, labels, legends (in case of multiple plots) can be added with key-value options. It can cycle through a set of predefined line/marker/color specifications. In summary, its purpose is to simplify the generation of high-quality function and/or data plots, and solving the problems of
\begin{itemize}
	\item consistency of document and font type and font size,
	\item direct use of \TeX\ math mode in axis descriptions,
	\item consistency of data and figures (no third party tool necessary),
	\item inter document consistency using preamble configurations and styles.
\end{itemize}
Although not necessary, separate |.pdf| or |.eps| graphics can be generated using the |external| library developed as part of \Tikz.

\PGFPlots\ is build completely on \Tikz/\PGF. Knowledge of \Tikz\ will simplify the work with \PGFPlots, although it is not required.

\section[About PGFPlots: Preliminaries]{About {\normalfont\PGFPlots}: Preliminaries}
This section contains information about upgrades, the team, the installation (in case you need to do it manually) and troubleshooting. You may skip it completely except for the upgrade remarks.

However, note that this library requires at least \PGF\ version $2.00$. At the time of this writing, many \TeX-distributions still contain the older \PGF\ version $1.18$, so it may be necessary to install a recent \PGF\ prior to using \PGFPlots.

\subsection{Components}
\PGFPlots\ comes with two components:
\begin{enumerate}
	\item the plotting component (which you are currently reading) and
	\item the \PGFPlotstable\ component which simplifies number formatting and postprocessing of numerical tables. It comes as a separate package and has its own manual \href{file:pgfplotstable.pdf}{pgfplotstable.pdf}.
\end{enumerate}

\subsection{Upgrade remarks}
This release provides a lot of improvements which can be found in all detail in \texttt{ChangeLog} for interested readers. However, some attention is useful with respect to the following changes.

\subsubsection{New Optional Features}
\begin{enumerate}
	\item \PGFPlots\ 1.3 comes with user interface improvements. The technical distinction between ``behavior options'' and ``style options'' of older versions is no longer necessary (although still fully supported).

	\item \PGFPlots\ 1.3 has a new feature which allows to \emph{move axis labels tight to tick labels} automatically.

	Since this affects the spacing, it is not enabled be default.

	Use
\begin{codeexample}[code only]
\usepackage{pgfplots}
\pgfplotsset{compat=1.3}
\end{codeexample}
	in your preamble to benefit from the improved spacing\footnote{The |compat=1.3| setting is necessary for new features which move axis labels around like reversed axis. However, \PGFPlots\ will still work as it does in earlier versions, your documents will remain the same if you don't set it explicitly.}.
	Take a look at the next page for the precise definition of |compat|.

	\item \PGFPlots\ 1.3 now supports reversed axes. It is no longer necessary to use work--arounds with negative units.
\pgfkeys{/pdflinks/search key prefixes in/.add={/pgfplots/,}{}}

	Take a look at the |x dir=reverse| key.

	Existing work--arounds will still function properly. Use |\pgfplotsset{compat=1.3}| together with |x dir=reverse| to switch to the new version.
\end{enumerate}

\subsubsection{Old Features Which May Need Attention}
\begin{enumerate}
	\item The |scatter/classes| feature produces proper legends as of version 1.3. This may change the appearance of existing legends of plots with |scatter/classes|.

	\item Due to a small math bug in \PGF\ $2.00$, you \emph{can't} use the math expression `|-x^2|'. It is necessary to use `|0-x^2|' instead. The same holds for
		`|exp(-x^2)|'; use `|exp(0-x^2)|' instead.

		This will be fixed with \PGF\ $>2.00$.

	\item Starting with \PGFPlots\ $1.1$, |\tikzstyle| should \emph{no longer be used} to set \PGFPlots\ options.
	
	Although |\tikzstyle| is still supported for some older \PGFPlots\ options, you should replace any occurance of |\tikzstyle| with |\pgfplotsset{|\meta{style name}|/.style={|\meta{key-value-list}|}}| or the associated |/.append style| variant. See section~\ref{sec:styles} for more detail.
\end{enumerate}
I apologize for any inconvenience caused by these changes.

\begin{pgfplotskey}{compat=\mchoice{1.3,pre 1.3,default,newest} (initially default)}
	The preamble configuration 
\begin{codeexample}[code only]
\usepackage{pgfplots}
\pgfplotsset{compat=1.3}
\end{codeexample}
	allows to choose between backwards compatibility and most recent features.

	\PGFPlots\ version 1.3 comes with new features which may lead to different spacings compared to earlier versions:
	\begin{enumerate}
		\item Version 1.3 comes with the |xlabel near ticks| feature which places (in this case $x$) axis labels ``near'' the tick labels, taking the size of tick labels into account.
		
		Older versions used |xlabel absolute|, i.e.\ an absolute distance between the axis and axis labels (now matter how large or small tick labels are).

		The initial setting \emph{keeps backwards compatibility}.

		You are encouraged to use
\begin{codeexample}[code only]
\usepackage{pgfplots}
\pgfplotsset{compat=1.3}
\end{codeexample}
		in your preamble.
		
		\item Version 1.3 fixes a bug with |anchor=south west| which occurred mainly for reversed axes: the anchors where upside--down. This is fixed now.

		
		If you have written some work--around and want to keep it going, use

		|\pgfplotsset{compat/anchors=pre 1.3}|\pgfmanualpdflabel{/pgfplots/compat/anchors}{} 

		to disable the bug fix.
	\end{enumerate}

	Use |\pgfplotsset{compat=default}| to restore the factory settings.

	The setting |\pgfplotsset{compat=newest}| will always use features of the most recent version. This might result in small changes in the document's appearance.
\end{pgfplotskey}

\subsection{The Team}
\PGFPlots\ has been written mainly by Christian Feuers�nger with many improvements of Pascal Wolkotte and Nick Papior Andersen as a spare time project. We hope it is useful and provides valuable plots.

If you are interested in writing something but don't know how, consider reading the auxiliary manual \href{file:TeX-programming-notes.pdf}{TeX-programming-notes.pdf} which comes with \PGFPlots. It is far from complete, but maybe it is a good starting point (at least for more literature).

\subsection{Acknowledgements}
I thank God for all hours of enjoyed programming. I thank Pascal Wolkotte and Nick Papior Andersen for their programming efforts and contributions as part of the development team. I thank Stefan Tibus, who contributed the |plot shell| feature. I thank Tom Cashman for the contribution of the |reverse legend| feature. Special thanks go to Stefan Pinnow whose tests of \PGFPlots\ lead to numerous quality improvements. Furthermore, I thank Dr.~Schweitzer for many fruitful discussions and Dr.~Meine for his ideas and suggestions. Thanks as well to the many international contributors who provided feature requests or identified bugs or simply manual improvements!

Last but not least, I thank Till Tantau and Mark Wibrow for their excellent graphics (and more) package \PGF\ and \Tikz\ which is the base of \PGFPlots.

\include{pgfplots.install}%
\include{pgfplots.intro}%
\include{pgfplots.reference}%
\include{pgfplots.libs}%
\include{pgfplots.resources}%
\include{pgfplots.importexport}%
\include{pgfplots.basic.reference}%

\printindex

\bibliographystyle{abbrv} %gerapali} %gerabbrv} %gerunsrt.bst} %gerabbrv}% gerplain}
\nocite{pgfplotstable}
\nocite{programmingnotes}
\bibliography{pgfplots}
\end{document}
%
\include{pgfplots.basic.reference}%

\printindex

\bibliographystyle{abbrv} %gerapali} %gerabbrv} %gerunsrt.bst} %gerabbrv}% gerplain}
\nocite{pgfplotstable}
\nocite{programmingnotes}
\bibliography{pgfplots}
\end{document}
%
%%%%%%%%%%%%%%%%%%%%%%%%%%%%%%%%%%%%%%%%%%%%%%%%%%%%%%%%%%%%%%%%%%%%%%%%%%%%%
%
% Package pgfplots.sty documentation. 
%
% Copyright 2007/2008 by Christian Feuersaenger.
%
% This program is free software: you can redistribute it and/or modify
% it under the terms of the GNU General Public License as published by
% the Free Software Foundation, either version 3 of the License, or
% (at your option) any later version.
% 
% This program is distributed in the hope that it will be useful,
% but WITHOUT ANY WARRANTY; without even the implied warranty of
% MERCHANTABILITY or FITNESS FOR A PARTICULAR PURPOSE.  See the
% GNU General Public License for more details.
% 
% You should have received a copy of the GNU General Public License
% along with this program.  If not, see <http://www.gnu.org/licenses/>.
%
%
%%%%%%%%%%%%%%%%%%%%%%%%%%%%%%%%%%%%%%%%%%%%%%%%%%%%%%%%%%%%%%%%%%%%%%%%%%%%%
%%%%%%%%%%%%%%%%%%%%%%%%%%%%%%%%%%%%%%%%%%%%%%%%%%%%%%%%%%%%%%%%%%%%%%%%%%%%%
%
% Package pgfplots.sty documentation. 
%
% Copyright 2007/2008 by Christian Feuersaenger.
%
% This program is free software: you can redistribute it and/or modify
% it under the terms of the GNU General Public License as published by
% the Free Software Foundation, either version 3 of the License, or
% (at your option) any later version.
% 
% This program is distributed in the hope that it will be useful,
% but WITHOUT ANY WARRANTY; without even the implied warranty of
% MERCHANTABILITY or FITNESS FOR A PARTICULAR PURPOSE.  See the
% GNU General Public License for more details.
% 
% You should have received a copy of the GNU General Public License
% along with this program.  If not, see <http://www.gnu.org/licenses/>.
%
%
%%%%%%%%%%%%%%%%%%%%%%%%%%%%%%%%%%%%%%%%%%%%%%%%%%%%%%%%%%%%%%%%%%%%%%%%%%%%%
\documentclass[a4paper]{ltxdoc}

\usepackage{makeidx}

% DON't let hyperref overload the format of index and glossary. 
% I want to do that on my own in the stylefiles for makeindex...
\makeatletter
\let\@old@wrindex=\@wrindex
\makeatother

\usepackage{ifpdf}
\ifpdf
	\usepackage[pdftex]{hyperref}
\else
	\usepackage[dvipdfm]{hyperref}
\fi
\hypersetup{pdfborder={0 0 0}}

\makeatletter
\let\@wrindex=\@old@wrindex
\makeatother

% Formatiere Seitennummern im Index:
\newcommand{\indexpageno}[1]{%
	{\bfseries\hyperpage{#1}}%
}


\newcommand{\R}{\mathbb{R}}
\newcommand{\N}{\mathbb{N}}
\newcommand{\Z}{\mathbb{Z}}

\long\def\COMMENTLOWLEVEL#1\ENDCOMMENT{}
\def\ENDCOMMENT{}

\usepackage{textcomp}
\usepackage{booktabs}

\usepackage{calc}
\usepackage[formats]{listings}
%\usepackage{courier} % don't use it - the '^' character can't be copy-pasted in courier

\usepackage{array}
\lstset{%
	basicstyle=\ttfamily,
	language=[LaTeX]tex, % Seems as if \lstset{language=tex} must be invoked BEFORE loading tikz!?
	tabsize=4,
	breaklines=true,
	breakindent=0pt
}

\ifpdf
	\pdfinfo {
		/Author	(Christian Feuersaenger)
	}

\else
	\def\pgfsysdriver{pgfsys-dvipdfm.def}
\fi
%\def\pgfsysdriver{pgfsys-pdftex.def}
\usepackage{pgfplots}

\ifpdf
	\usetikzlibrary{pgfplotsclickable}
\fi

%\usepackage{fp}
% ATTENTION:
% this requires pgf version NEWER than 2.00 :
%\usetikzlibrary{fixedpointarithmetic}

\usetikzlibrary{dateplot}

\usepackage[a4paper,left=2.25cm,right=2.25cm,top=2.5cm,bottom=2.5cm,nohead]{geometry}
\usepackage{amsmath,amssymb}
\usepackage{xxcolor}
\usepackage{pifont}
\usepackage[latin1]{inputenc}
\usepackage{amsmath}
\usepackage{eurosym}
\input{pgfplots-macros}

\pgfqkeys{/codeexample}{%
	every codeexample/.style={
		width=8cm,
		/pgfplots/every axis/.append style={legend style={fill=graphicbackground}}
	},
	tabsize=4,
}
\pgfplotsset{
	every axis/.append style={width=7cm},
	filter discard warning=false,
}

\usetikzlibrary{backgrounds,patterns}
% Global styles:
\tikzset{
  shape example/.style={
    color=black!30,
    draw,
    fill=yellow!30,
    line width=.5cm,
    inner xsep=2.5cm,
    inner ysep=0.5cm}
}

\newcommand{\FIXME}[1]{\textcolor{red}{(FIXME: #1)}}

% fuer endvironment 'sidewaysfigure' bspw
% \usepackage{rotating}

\newcommand\Tikz{Ti\textit kZ}
\newcommand\PGF{\textsc{pgf}}
\newcommand\PGFPlots{\textsc{pgfplots}}
\def\PGFPlotstable{\textsc{PgfplotsTable}}


\makeindex

% Fix overful hboxes automatically:
\tolerance=2000
\emergencystretch=10pt

\tikzset{prefix=gnuplot/pgfplots_} % prefix for 'plot function'

\author{%
	Christian Feuers\"anger\footnote{\url{http://wissrech.ins.uni-bonn.de/people/feuersaenger}}\\%
	Institut f\"ur Numerische Simulation\\
	Universit\"at Bonn, Germany}


%\RequirePackage[german,english,francais]{babel}

\pgfplotsmanualexternalexpensivetrue

%\usetikzlibrary{external}
%\tikzexternalize[prefix=figures/]{pgfplots}

\title{%
	Manual for Package \PGFPlots\\
	{\small Version \pgfplotsversion}\\
	{\small\href{http://sourceforge.net/projects/pgfplots}{http://sourceforge.net/projects/pgfplots}}}

%\includeonly{pgfplots.libs}


\begin{document}

\def\plotcoords{%
\addplot coordinates {
(5,8.312e-02)    (17,2.547e-02)   (49,7.407e-03)
(129,2.102e-03)  (321,5.874e-04)  (769,1.623e-04)
(1793,4.442e-05) (4097,1.207e-05) (9217,3.261e-06)
};

\addplot coordinates{
(7,8.472e-02)    (31,3.044e-02)    (111,1.022e-02)
(351,3.303e-03)  (1023,1.039e-03)  (2815,3.196e-04)
(7423,9.658e-05) (18943,2.873e-05) (47103,8.437e-06)
};

\addplot coordinates{
(9,7.881e-02)     (49,3.243e-02)    (209,1.232e-02)
(769,4.454e-03)   (2561,1.551e-03)  (7937,5.236e-04)
(23297,1.723e-04) (65537,5.545e-05) (178177,1.751e-05)
};

\addplot coordinates{
(11,6.887e-02)    (71,3.177e-02)     (351,1.341e-02)
(1471,5.334e-03)  (5503,2.027e-03)   (18943,7.415e-04)
(61183,2.628e-04) (187903,9.063e-05) (553983,3.053e-05)
};

\addplot coordinates{
(13,5.755e-02)     (97,2.925e-02)     (545,1.351e-02)
(2561,5.842e-03)   (10625,2.397e-03)  (40193,9.414e-04)
(141569,3.564e-04) (471041,1.308e-04) 
(1496065,4.670e-05)
};
}%
\maketitle
\begin{abstract}%
\PGFPlots\ draws high--quality function plots in normal or logarithmic scaling with a user-friendly interface directly in \TeX. The user supplies axis labels, legend entries and the plot coordinates for one or more plots and \PGFPlots\ applies axis scaling, computes any logarithms and axis ticks and draws the plots, supporting line plots, scatter plots, piecewise constant plots, bar plots, area plots, mesh-- and surface plots and some more. It is based on Till Tantau's package \PGF/\Tikz.
\end{abstract}
\tableofcontents
\section{Introduction}
This package provides tools to generate plots and labeled axes easily. It draws normal plots, logplots and semi-logplots, in two and three dimensions. Axis ticks, labels, legends (in case of multiple plots) can be added with key-value options. It can cycle through a set of predefined line/marker/color specifications. In summary, its purpose is to simplify the generation of high-quality function and/or data plots, and solving the problems of
\begin{itemize}
	\item consistency of document and font type and font size,
	\item direct use of \TeX\ math mode in axis descriptions,
	\item consistency of data and figures (no third party tool necessary),
	\item inter document consistency using preamble configurations and styles.
\end{itemize}
Although not necessary, separate |.pdf| or |.eps| graphics can be generated using the |external| library developed as part of \Tikz.

\PGFPlots\ is build completely on \Tikz/\PGF. Knowledge of \Tikz\ will simplify the work with \PGFPlots, although it is not required.

\section[About PGFPlots: Preliminaries]{About {\normalfont\PGFPlots}: Preliminaries}
This section contains information about upgrades, the team, the installation (in case you need to do it manually) and troubleshooting. You may skip it completely except for the upgrade remarks.

However, note that this library requires at least \PGF\ version $2.00$. At the time of this writing, many \TeX-distributions still contain the older \PGF\ version $1.18$, so it may be necessary to install a recent \PGF\ prior to using \PGFPlots.

\subsection{Components}
\PGFPlots\ comes with two components:
\begin{enumerate}
	\item the plotting component (which you are currently reading) and
	\item the \PGFPlotstable\ component which simplifies number formatting and postprocessing of numerical tables. It comes as a separate package and has its own manual \href{file:pgfplotstable.pdf}{pgfplotstable.pdf}.
\end{enumerate}

\subsection{Upgrade remarks}
This release provides a lot of improvements which can be found in all detail in \texttt{ChangeLog} for interested readers. However, some attention is useful with respect to the following changes.

\subsubsection{New Optional Features}
\begin{enumerate}
	\item \PGFPlots\ 1.3 comes with user interface improvements. The technical distinction between ``behavior options'' and ``style options'' of older versions is no longer necessary (although still fully supported).

	\item \PGFPlots\ 1.3 has a new feature which allows to \emph{move axis labels tight to tick labels} automatically.

	Since this affects the spacing, it is not enabled be default.

	Use
\begin{codeexample}[code only]
\usepackage{pgfplots}
\pgfplotsset{compat=1.3}
\end{codeexample}
	in your preamble to benefit from the improved spacing\footnote{The |compat=1.3| setting is necessary for new features which move axis labels around like reversed axis. However, \PGFPlots\ will still work as it does in earlier versions, your documents will remain the same if you don't set it explicitly.}.
	Take a look at the next page for the precise definition of |compat|.

	\item \PGFPlots\ 1.3 now supports reversed axes. It is no longer necessary to use work--arounds with negative units.
\pgfkeys{/pdflinks/search key prefixes in/.add={/pgfplots/,}{}}

	Take a look at the |x dir=reverse| key.

	Existing work--arounds will still function properly. Use |\pgfplotsset{compat=1.3}| together with |x dir=reverse| to switch to the new version.
\end{enumerate}

\subsubsection{Old Features Which May Need Attention}
\begin{enumerate}
	\item The |scatter/classes| feature produces proper legends as of version 1.3. This may change the appearance of existing legends of plots with |scatter/classes|.

	\item Due to a small math bug in \PGF\ $2.00$, you \emph{can't} use the math expression `|-x^2|'. It is necessary to use `|0-x^2|' instead. The same holds for
		`|exp(-x^2)|'; use `|exp(0-x^2)|' instead.

		This will be fixed with \PGF\ $>2.00$.

	\item Starting with \PGFPlots\ $1.1$, |\tikzstyle| should \emph{no longer be used} to set \PGFPlots\ options.
	
	Although |\tikzstyle| is still supported for some older \PGFPlots\ options, you should replace any occurance of |\tikzstyle| with |\pgfplotsset{|\meta{style name}|/.style={|\meta{key-value-list}|}}| or the associated |/.append style| variant. See section~\ref{sec:styles} for more detail.
\end{enumerate}
I apologize for any inconvenience caused by these changes.

\begin{pgfplotskey}{compat=\mchoice{1.3,pre 1.3,default,newest} (initially default)}
	The preamble configuration 
\begin{codeexample}[code only]
\usepackage{pgfplots}
\pgfplotsset{compat=1.3}
\end{codeexample}
	allows to choose between backwards compatibility and most recent features.

	\PGFPlots\ version 1.3 comes with new features which may lead to different spacings compared to earlier versions:
	\begin{enumerate}
		\item Version 1.3 comes with the |xlabel near ticks| feature which places (in this case $x$) axis labels ``near'' the tick labels, taking the size of tick labels into account.
		
		Older versions used |xlabel absolute|, i.e.\ an absolute distance between the axis and axis labels (now matter how large or small tick labels are).

		The initial setting \emph{keeps backwards compatibility}.

		You are encouraged to use
\begin{codeexample}[code only]
\usepackage{pgfplots}
\pgfplotsset{compat=1.3}
\end{codeexample}
		in your preamble.
		
		\item Version 1.3 fixes a bug with |anchor=south west| which occurred mainly for reversed axes: the anchors where upside--down. This is fixed now.

		
		If you have written some work--around and want to keep it going, use

		|\pgfplotsset{compat/anchors=pre 1.3}|\pgfmanualpdflabel{/pgfplots/compat/anchors}{} 

		to disable the bug fix.
	\end{enumerate}

	Use |\pgfplotsset{compat=default}| to restore the factory settings.

	The setting |\pgfplotsset{compat=newest}| will always use features of the most recent version. This might result in small changes in the document's appearance.
\end{pgfplotskey}

\subsection{The Team}
\PGFPlots\ has been written mainly by Christian Feuers�nger with many improvements of Pascal Wolkotte and Nick Papior Andersen as a spare time project. We hope it is useful and provides valuable plots.

If you are interested in writing something but don't know how, consider reading the auxiliary manual \href{file:TeX-programming-notes.pdf}{TeX-programming-notes.pdf} which comes with \PGFPlots. It is far from complete, but maybe it is a good starting point (at least for more literature).

\subsection{Acknowledgements}
I thank God for all hours of enjoyed programming. I thank Pascal Wolkotte and Nick Papior Andersen for their programming efforts and contributions as part of the development team. I thank Stefan Tibus, who contributed the |plot shell| feature. I thank Tom Cashman for the contribution of the |reverse legend| feature. Special thanks go to Stefan Pinnow whose tests of \PGFPlots\ lead to numerous quality improvements. Furthermore, I thank Dr.~Schweitzer for many fruitful discussions and Dr.~Meine for his ideas and suggestions. Thanks as well to the many international contributors who provided feature requests or identified bugs or simply manual improvements!

Last but not least, I thank Till Tantau and Mark Wibrow for their excellent graphics (and more) package \PGF\ and \Tikz\ which is the base of \PGFPlots.

%%%%%%%%%%%%%%%%%%%%%%%%%%%%%%%%%%%%%%%%%%%%%%%%%%%%%%%%%%%%%%%%%%%%%%%%%%%%%
%
% Package pgfplots.sty documentation. 
%
% Copyright 2007/2008 by Christian Feuersaenger.
%
% This program is free software: you can redistribute it and/or modify
% it under the terms of the GNU General Public License as published by
% the Free Software Foundation, either version 3 of the License, or
% (at your option) any later version.
% 
% This program is distributed in the hope that it will be useful,
% but WITHOUT ANY WARRANTY; without even the implied warranty of
% MERCHANTABILITY or FITNESS FOR A PARTICULAR PURPOSE.  See the
% GNU General Public License for more details.
% 
% You should have received a copy of the GNU General Public License
% along with this program.  If not, see <http://www.gnu.org/licenses/>.
%
%
%%%%%%%%%%%%%%%%%%%%%%%%%%%%%%%%%%%%%%%%%%%%%%%%%%%%%%%%%%%%%%%%%%%%%%%%%%%%%
\input{pgfplots.preamble.tex}
%\RequirePackage[german,english,francais]{babel}

\pgfplotsmanualexternalexpensivetrue

%\usetikzlibrary{external}
%\tikzexternalize[prefix=figures/]{pgfplots}

\title{%
	Manual for Package \PGFPlots\\
	{\small Version \pgfplotsversion}\\
	{\small\href{http://sourceforge.net/projects/pgfplots}{http://sourceforge.net/projects/pgfplots}}}

%\includeonly{pgfplots.libs}


\begin{document}

\def\plotcoords{%
\addplot coordinates {
(5,8.312e-02)    (17,2.547e-02)   (49,7.407e-03)
(129,2.102e-03)  (321,5.874e-04)  (769,1.623e-04)
(1793,4.442e-05) (4097,1.207e-05) (9217,3.261e-06)
};

\addplot coordinates{
(7,8.472e-02)    (31,3.044e-02)    (111,1.022e-02)
(351,3.303e-03)  (1023,1.039e-03)  (2815,3.196e-04)
(7423,9.658e-05) (18943,2.873e-05) (47103,8.437e-06)
};

\addplot coordinates{
(9,7.881e-02)     (49,3.243e-02)    (209,1.232e-02)
(769,4.454e-03)   (2561,1.551e-03)  (7937,5.236e-04)
(23297,1.723e-04) (65537,5.545e-05) (178177,1.751e-05)
};

\addplot coordinates{
(11,6.887e-02)    (71,3.177e-02)     (351,1.341e-02)
(1471,5.334e-03)  (5503,2.027e-03)   (18943,7.415e-04)
(61183,2.628e-04) (187903,9.063e-05) (553983,3.053e-05)
};

\addplot coordinates{
(13,5.755e-02)     (97,2.925e-02)     (545,1.351e-02)
(2561,5.842e-03)   (10625,2.397e-03)  (40193,9.414e-04)
(141569,3.564e-04) (471041,1.308e-04) 
(1496065,4.670e-05)
};
}%
\maketitle
\begin{abstract}%
\PGFPlots\ draws high--quality function plots in normal or logarithmic scaling with a user-friendly interface directly in \TeX. The user supplies axis labels, legend entries and the plot coordinates for one or more plots and \PGFPlots\ applies axis scaling, computes any logarithms and axis ticks and draws the plots, supporting line plots, scatter plots, piecewise constant plots, bar plots, area plots, mesh-- and surface plots and some more. It is based on Till Tantau's package \PGF/\Tikz.
\end{abstract}
\tableofcontents
\section{Introduction}
This package provides tools to generate plots and labeled axes easily. It draws normal plots, logplots and semi-logplots, in two and three dimensions. Axis ticks, labels, legends (in case of multiple plots) can be added with key-value options. It can cycle through a set of predefined line/marker/color specifications. In summary, its purpose is to simplify the generation of high-quality function and/or data plots, and solving the problems of
\begin{itemize}
	\item consistency of document and font type and font size,
	\item direct use of \TeX\ math mode in axis descriptions,
	\item consistency of data and figures (no third party tool necessary),
	\item inter document consistency using preamble configurations and styles.
\end{itemize}
Although not necessary, separate |.pdf| or |.eps| graphics can be generated using the |external| library developed as part of \Tikz.

\PGFPlots\ is build completely on \Tikz/\PGF. Knowledge of \Tikz\ will simplify the work with \PGFPlots, although it is not required.

\section[About PGFPlots: Preliminaries]{About {\normalfont\PGFPlots}: Preliminaries}
This section contains information about upgrades, the team, the installation (in case you need to do it manually) and troubleshooting. You may skip it completely except for the upgrade remarks.

However, note that this library requires at least \PGF\ version $2.00$. At the time of this writing, many \TeX-distributions still contain the older \PGF\ version $1.18$, so it may be necessary to install a recent \PGF\ prior to using \PGFPlots.

\subsection{Components}
\PGFPlots\ comes with two components:
\begin{enumerate}
	\item the plotting component (which you are currently reading) and
	\item the \PGFPlotstable\ component which simplifies number formatting and postprocessing of numerical tables. It comes as a separate package and has its own manual \href{file:pgfplotstable.pdf}{pgfplotstable.pdf}.
\end{enumerate}

\subsection{Upgrade remarks}
This release provides a lot of improvements which can be found in all detail in \texttt{ChangeLog} for interested readers. However, some attention is useful with respect to the following changes.

\subsubsection{New Optional Features}
\begin{enumerate}
	\item \PGFPlots\ 1.3 comes with user interface improvements. The technical distinction between ``behavior options'' and ``style options'' of older versions is no longer necessary (although still fully supported).

	\item \PGFPlots\ 1.3 has a new feature which allows to \emph{move axis labels tight to tick labels} automatically.

	Since this affects the spacing, it is not enabled be default.

	Use
\begin{codeexample}[code only]
\usepackage{pgfplots}
\pgfplotsset{compat=1.3}
\end{codeexample}
	in your preamble to benefit from the improved spacing\footnote{The |compat=1.3| setting is necessary for new features which move axis labels around like reversed axis. However, \PGFPlots\ will still work as it does in earlier versions, your documents will remain the same if you don't set it explicitly.}.
	Take a look at the next page for the precise definition of |compat|.

	\item \PGFPlots\ 1.3 now supports reversed axes. It is no longer necessary to use work--arounds with negative units.
\pgfkeys{/pdflinks/search key prefixes in/.add={/pgfplots/,}{}}

	Take a look at the |x dir=reverse| key.

	Existing work--arounds will still function properly. Use |\pgfplotsset{compat=1.3}| together with |x dir=reverse| to switch to the new version.
\end{enumerate}

\subsubsection{Old Features Which May Need Attention}
\begin{enumerate}
	\item The |scatter/classes| feature produces proper legends as of version 1.3. This may change the appearance of existing legends of plots with |scatter/classes|.

	\item Due to a small math bug in \PGF\ $2.00$, you \emph{can't} use the math expression `|-x^2|'. It is necessary to use `|0-x^2|' instead. The same holds for
		`|exp(-x^2)|'; use `|exp(0-x^2)|' instead.

		This will be fixed with \PGF\ $>2.00$.

	\item Starting with \PGFPlots\ $1.1$, |\tikzstyle| should \emph{no longer be used} to set \PGFPlots\ options.
	
	Although |\tikzstyle| is still supported for some older \PGFPlots\ options, you should replace any occurance of |\tikzstyle| with |\pgfplotsset{|\meta{style name}|/.style={|\meta{key-value-list}|}}| or the associated |/.append style| variant. See section~\ref{sec:styles} for more detail.
\end{enumerate}
I apologize for any inconvenience caused by these changes.

\begin{pgfplotskey}{compat=\mchoice{1.3,pre 1.3,default,newest} (initially default)}
	The preamble configuration 
\begin{codeexample}[code only]
\usepackage{pgfplots}
\pgfplotsset{compat=1.3}
\end{codeexample}
	allows to choose between backwards compatibility and most recent features.

	\PGFPlots\ version 1.3 comes with new features which may lead to different spacings compared to earlier versions:
	\begin{enumerate}
		\item Version 1.3 comes with the |xlabel near ticks| feature which places (in this case $x$) axis labels ``near'' the tick labels, taking the size of tick labels into account.
		
		Older versions used |xlabel absolute|, i.e.\ an absolute distance between the axis and axis labels (now matter how large or small tick labels are).

		The initial setting \emph{keeps backwards compatibility}.

		You are encouraged to use
\begin{codeexample}[code only]
\usepackage{pgfplots}
\pgfplotsset{compat=1.3}
\end{codeexample}
		in your preamble.
		
		\item Version 1.3 fixes a bug with |anchor=south west| which occurred mainly for reversed axes: the anchors where upside--down. This is fixed now.

		
		If you have written some work--around and want to keep it going, use

		|\pgfplotsset{compat/anchors=pre 1.3}|\pgfmanualpdflabel{/pgfplots/compat/anchors}{} 

		to disable the bug fix.
	\end{enumerate}

	Use |\pgfplotsset{compat=default}| to restore the factory settings.

	The setting |\pgfplotsset{compat=newest}| will always use features of the most recent version. This might result in small changes in the document's appearance.
\end{pgfplotskey}

\subsection{The Team}
\PGFPlots\ has been written mainly by Christian Feuers�nger with many improvements of Pascal Wolkotte and Nick Papior Andersen as a spare time project. We hope it is useful and provides valuable plots.

If you are interested in writing something but don't know how, consider reading the auxiliary manual \href{file:TeX-programming-notes.pdf}{TeX-programming-notes.pdf} which comes with \PGFPlots. It is far from complete, but maybe it is a good starting point (at least for more literature).

\subsection{Acknowledgements}
I thank God for all hours of enjoyed programming. I thank Pascal Wolkotte and Nick Papior Andersen for their programming efforts and contributions as part of the development team. I thank Stefan Tibus, who contributed the |plot shell| feature. I thank Tom Cashman for the contribution of the |reverse legend| feature. Special thanks go to Stefan Pinnow whose tests of \PGFPlots\ lead to numerous quality improvements. Furthermore, I thank Dr.~Schweitzer for many fruitful discussions and Dr.~Meine for his ideas and suggestions. Thanks as well to the many international contributors who provided feature requests or identified bugs or simply manual improvements!

Last but not least, I thank Till Tantau and Mark Wibrow for their excellent graphics (and more) package \PGF\ and \Tikz\ which is the base of \PGFPlots.

\include{pgfplots.install}%
\include{pgfplots.intro}%
\include{pgfplots.reference}%
\include{pgfplots.libs}%
\include{pgfplots.resources}%
\include{pgfplots.importexport}%
\include{pgfplots.basic.reference}%

\printindex

\bibliographystyle{abbrv} %gerapali} %gerabbrv} %gerunsrt.bst} %gerabbrv}% gerplain}
\nocite{pgfplotstable}
\nocite{programmingnotes}
\bibliography{pgfplots}
\end{document}
%
%%%%%%%%%%%%%%%%%%%%%%%%%%%%%%%%%%%%%%%%%%%%%%%%%%%%%%%%%%%%%%%%%%%%%%%%%%%%%
%
% Package pgfplots.sty documentation. 
%
% Copyright 2007/2008 by Christian Feuersaenger.
%
% This program is free software: you can redistribute it and/or modify
% it under the terms of the GNU General Public License as published by
% the Free Software Foundation, either version 3 of the License, or
% (at your option) any later version.
% 
% This program is distributed in the hope that it will be useful,
% but WITHOUT ANY WARRANTY; without even the implied warranty of
% MERCHANTABILITY or FITNESS FOR A PARTICULAR PURPOSE.  See the
% GNU General Public License for more details.
% 
% You should have received a copy of the GNU General Public License
% along with this program.  If not, see <http://www.gnu.org/licenses/>.
%
%
%%%%%%%%%%%%%%%%%%%%%%%%%%%%%%%%%%%%%%%%%%%%%%%%%%%%%%%%%%%%%%%%%%%%%%%%%%%%%
\input{pgfplots.preamble.tex}
%\RequirePackage[german,english,francais]{babel}

\pgfplotsmanualexternalexpensivetrue

%\usetikzlibrary{external}
%\tikzexternalize[prefix=figures/]{pgfplots}

\title{%
	Manual for Package \PGFPlots\\
	{\small Version \pgfplotsversion}\\
	{\small\href{http://sourceforge.net/projects/pgfplots}{http://sourceforge.net/projects/pgfplots}}}

%\includeonly{pgfplots.libs}


\begin{document}

\def\plotcoords{%
\addplot coordinates {
(5,8.312e-02)    (17,2.547e-02)   (49,7.407e-03)
(129,2.102e-03)  (321,5.874e-04)  (769,1.623e-04)
(1793,4.442e-05) (4097,1.207e-05) (9217,3.261e-06)
};

\addplot coordinates{
(7,8.472e-02)    (31,3.044e-02)    (111,1.022e-02)
(351,3.303e-03)  (1023,1.039e-03)  (2815,3.196e-04)
(7423,9.658e-05) (18943,2.873e-05) (47103,8.437e-06)
};

\addplot coordinates{
(9,7.881e-02)     (49,3.243e-02)    (209,1.232e-02)
(769,4.454e-03)   (2561,1.551e-03)  (7937,5.236e-04)
(23297,1.723e-04) (65537,5.545e-05) (178177,1.751e-05)
};

\addplot coordinates{
(11,6.887e-02)    (71,3.177e-02)     (351,1.341e-02)
(1471,5.334e-03)  (5503,2.027e-03)   (18943,7.415e-04)
(61183,2.628e-04) (187903,9.063e-05) (553983,3.053e-05)
};

\addplot coordinates{
(13,5.755e-02)     (97,2.925e-02)     (545,1.351e-02)
(2561,5.842e-03)   (10625,2.397e-03)  (40193,9.414e-04)
(141569,3.564e-04) (471041,1.308e-04) 
(1496065,4.670e-05)
};
}%
\maketitle
\begin{abstract}%
\PGFPlots\ draws high--quality function plots in normal or logarithmic scaling with a user-friendly interface directly in \TeX. The user supplies axis labels, legend entries and the plot coordinates for one or more plots and \PGFPlots\ applies axis scaling, computes any logarithms and axis ticks and draws the plots, supporting line plots, scatter plots, piecewise constant plots, bar plots, area plots, mesh-- and surface plots and some more. It is based on Till Tantau's package \PGF/\Tikz.
\end{abstract}
\tableofcontents
\section{Introduction}
This package provides tools to generate plots and labeled axes easily. It draws normal plots, logplots and semi-logplots, in two and three dimensions. Axis ticks, labels, legends (in case of multiple plots) can be added with key-value options. It can cycle through a set of predefined line/marker/color specifications. In summary, its purpose is to simplify the generation of high-quality function and/or data plots, and solving the problems of
\begin{itemize}
	\item consistency of document and font type and font size,
	\item direct use of \TeX\ math mode in axis descriptions,
	\item consistency of data and figures (no third party tool necessary),
	\item inter document consistency using preamble configurations and styles.
\end{itemize}
Although not necessary, separate |.pdf| or |.eps| graphics can be generated using the |external| library developed as part of \Tikz.

\PGFPlots\ is build completely on \Tikz/\PGF. Knowledge of \Tikz\ will simplify the work with \PGFPlots, although it is not required.

\section[About PGFPlots: Preliminaries]{About {\normalfont\PGFPlots}: Preliminaries}
This section contains information about upgrades, the team, the installation (in case you need to do it manually) and troubleshooting. You may skip it completely except for the upgrade remarks.

However, note that this library requires at least \PGF\ version $2.00$. At the time of this writing, many \TeX-distributions still contain the older \PGF\ version $1.18$, so it may be necessary to install a recent \PGF\ prior to using \PGFPlots.

\subsection{Components}
\PGFPlots\ comes with two components:
\begin{enumerate}
	\item the plotting component (which you are currently reading) and
	\item the \PGFPlotstable\ component which simplifies number formatting and postprocessing of numerical tables. It comes as a separate package and has its own manual \href{file:pgfplotstable.pdf}{pgfplotstable.pdf}.
\end{enumerate}

\subsection{Upgrade remarks}
This release provides a lot of improvements which can be found in all detail in \texttt{ChangeLog} for interested readers. However, some attention is useful with respect to the following changes.

\subsubsection{New Optional Features}
\begin{enumerate}
	\item \PGFPlots\ 1.3 comes with user interface improvements. The technical distinction between ``behavior options'' and ``style options'' of older versions is no longer necessary (although still fully supported).

	\item \PGFPlots\ 1.3 has a new feature which allows to \emph{move axis labels tight to tick labels} automatically.

	Since this affects the spacing, it is not enabled be default.

	Use
\begin{codeexample}[code only]
\usepackage{pgfplots}
\pgfplotsset{compat=1.3}
\end{codeexample}
	in your preamble to benefit from the improved spacing\footnote{The |compat=1.3| setting is necessary for new features which move axis labels around like reversed axis. However, \PGFPlots\ will still work as it does in earlier versions, your documents will remain the same if you don't set it explicitly.}.
	Take a look at the next page for the precise definition of |compat|.

	\item \PGFPlots\ 1.3 now supports reversed axes. It is no longer necessary to use work--arounds with negative units.
\pgfkeys{/pdflinks/search key prefixes in/.add={/pgfplots/,}{}}

	Take a look at the |x dir=reverse| key.

	Existing work--arounds will still function properly. Use |\pgfplotsset{compat=1.3}| together with |x dir=reverse| to switch to the new version.
\end{enumerate}

\subsubsection{Old Features Which May Need Attention}
\begin{enumerate}
	\item The |scatter/classes| feature produces proper legends as of version 1.3. This may change the appearance of existing legends of plots with |scatter/classes|.

	\item Due to a small math bug in \PGF\ $2.00$, you \emph{can't} use the math expression `|-x^2|'. It is necessary to use `|0-x^2|' instead. The same holds for
		`|exp(-x^2)|'; use `|exp(0-x^2)|' instead.

		This will be fixed with \PGF\ $>2.00$.

	\item Starting with \PGFPlots\ $1.1$, |\tikzstyle| should \emph{no longer be used} to set \PGFPlots\ options.
	
	Although |\tikzstyle| is still supported for some older \PGFPlots\ options, you should replace any occurance of |\tikzstyle| with |\pgfplotsset{|\meta{style name}|/.style={|\meta{key-value-list}|}}| or the associated |/.append style| variant. See section~\ref{sec:styles} for more detail.
\end{enumerate}
I apologize for any inconvenience caused by these changes.

\begin{pgfplotskey}{compat=\mchoice{1.3,pre 1.3,default,newest} (initially default)}
	The preamble configuration 
\begin{codeexample}[code only]
\usepackage{pgfplots}
\pgfplotsset{compat=1.3}
\end{codeexample}
	allows to choose between backwards compatibility and most recent features.

	\PGFPlots\ version 1.3 comes with new features which may lead to different spacings compared to earlier versions:
	\begin{enumerate}
		\item Version 1.3 comes with the |xlabel near ticks| feature which places (in this case $x$) axis labels ``near'' the tick labels, taking the size of tick labels into account.
		
		Older versions used |xlabel absolute|, i.e.\ an absolute distance between the axis and axis labels (now matter how large or small tick labels are).

		The initial setting \emph{keeps backwards compatibility}.

		You are encouraged to use
\begin{codeexample}[code only]
\usepackage{pgfplots}
\pgfplotsset{compat=1.3}
\end{codeexample}
		in your preamble.
		
		\item Version 1.3 fixes a bug with |anchor=south west| which occurred mainly for reversed axes: the anchors where upside--down. This is fixed now.

		
		If you have written some work--around and want to keep it going, use

		|\pgfplotsset{compat/anchors=pre 1.3}|\pgfmanualpdflabel{/pgfplots/compat/anchors}{} 

		to disable the bug fix.
	\end{enumerate}

	Use |\pgfplotsset{compat=default}| to restore the factory settings.

	The setting |\pgfplotsset{compat=newest}| will always use features of the most recent version. This might result in small changes in the document's appearance.
\end{pgfplotskey}

\subsection{The Team}
\PGFPlots\ has been written mainly by Christian Feuers�nger with many improvements of Pascal Wolkotte and Nick Papior Andersen as a spare time project. We hope it is useful and provides valuable plots.

If you are interested in writing something but don't know how, consider reading the auxiliary manual \href{file:TeX-programming-notes.pdf}{TeX-programming-notes.pdf} which comes with \PGFPlots. It is far from complete, but maybe it is a good starting point (at least for more literature).

\subsection{Acknowledgements}
I thank God for all hours of enjoyed programming. I thank Pascal Wolkotte and Nick Papior Andersen for their programming efforts and contributions as part of the development team. I thank Stefan Tibus, who contributed the |plot shell| feature. I thank Tom Cashman for the contribution of the |reverse legend| feature. Special thanks go to Stefan Pinnow whose tests of \PGFPlots\ lead to numerous quality improvements. Furthermore, I thank Dr.~Schweitzer for many fruitful discussions and Dr.~Meine for his ideas and suggestions. Thanks as well to the many international contributors who provided feature requests or identified bugs or simply manual improvements!

Last but not least, I thank Till Tantau and Mark Wibrow for their excellent graphics (and more) package \PGF\ and \Tikz\ which is the base of \PGFPlots.

\include{pgfplots.install}%
\include{pgfplots.intro}%
\include{pgfplots.reference}%
\include{pgfplots.libs}%
\include{pgfplots.resources}%
\include{pgfplots.importexport}%
\include{pgfplots.basic.reference}%

\printindex

\bibliographystyle{abbrv} %gerapali} %gerabbrv} %gerunsrt.bst} %gerabbrv}% gerplain}
\nocite{pgfplotstable}
\nocite{programmingnotes}
\bibliography{pgfplots}
\end{document}
%
%%%%%%%%%%%%%%%%%%%%%%%%%%%%%%%%%%%%%%%%%%%%%%%%%%%%%%%%%%%%%%%%%%%%%%%%%%%%%
%
% Package pgfplots.sty documentation. 
%
% Copyright 2007/2008 by Christian Feuersaenger.
%
% This program is free software: you can redistribute it and/or modify
% it under the terms of the GNU General Public License as published by
% the Free Software Foundation, either version 3 of the License, or
% (at your option) any later version.
% 
% This program is distributed in the hope that it will be useful,
% but WITHOUT ANY WARRANTY; without even the implied warranty of
% MERCHANTABILITY or FITNESS FOR A PARTICULAR PURPOSE.  See the
% GNU General Public License for more details.
% 
% You should have received a copy of the GNU General Public License
% along with this program.  If not, see <http://www.gnu.org/licenses/>.
%
%
%%%%%%%%%%%%%%%%%%%%%%%%%%%%%%%%%%%%%%%%%%%%%%%%%%%%%%%%%%%%%%%%%%%%%%%%%%%%%
\input{pgfplots.preamble.tex}
%\RequirePackage[german,english,francais]{babel}

\pgfplotsmanualexternalexpensivetrue

%\usetikzlibrary{external}
%\tikzexternalize[prefix=figures/]{pgfplots}

\title{%
	Manual for Package \PGFPlots\\
	{\small Version \pgfplotsversion}\\
	{\small\href{http://sourceforge.net/projects/pgfplots}{http://sourceforge.net/projects/pgfplots}}}

%\includeonly{pgfplots.libs}


\begin{document}

\def\plotcoords{%
\addplot coordinates {
(5,8.312e-02)    (17,2.547e-02)   (49,7.407e-03)
(129,2.102e-03)  (321,5.874e-04)  (769,1.623e-04)
(1793,4.442e-05) (4097,1.207e-05) (9217,3.261e-06)
};

\addplot coordinates{
(7,8.472e-02)    (31,3.044e-02)    (111,1.022e-02)
(351,3.303e-03)  (1023,1.039e-03)  (2815,3.196e-04)
(7423,9.658e-05) (18943,2.873e-05) (47103,8.437e-06)
};

\addplot coordinates{
(9,7.881e-02)     (49,3.243e-02)    (209,1.232e-02)
(769,4.454e-03)   (2561,1.551e-03)  (7937,5.236e-04)
(23297,1.723e-04) (65537,5.545e-05) (178177,1.751e-05)
};

\addplot coordinates{
(11,6.887e-02)    (71,3.177e-02)     (351,1.341e-02)
(1471,5.334e-03)  (5503,2.027e-03)   (18943,7.415e-04)
(61183,2.628e-04) (187903,9.063e-05) (553983,3.053e-05)
};

\addplot coordinates{
(13,5.755e-02)     (97,2.925e-02)     (545,1.351e-02)
(2561,5.842e-03)   (10625,2.397e-03)  (40193,9.414e-04)
(141569,3.564e-04) (471041,1.308e-04) 
(1496065,4.670e-05)
};
}%
\maketitle
\begin{abstract}%
\PGFPlots\ draws high--quality function plots in normal or logarithmic scaling with a user-friendly interface directly in \TeX. The user supplies axis labels, legend entries and the plot coordinates for one or more plots and \PGFPlots\ applies axis scaling, computes any logarithms and axis ticks and draws the plots, supporting line plots, scatter plots, piecewise constant plots, bar plots, area plots, mesh-- and surface plots and some more. It is based on Till Tantau's package \PGF/\Tikz.
\end{abstract}
\tableofcontents
\section{Introduction}
This package provides tools to generate plots and labeled axes easily. It draws normal plots, logplots and semi-logplots, in two and three dimensions. Axis ticks, labels, legends (in case of multiple plots) can be added with key-value options. It can cycle through a set of predefined line/marker/color specifications. In summary, its purpose is to simplify the generation of high-quality function and/or data plots, and solving the problems of
\begin{itemize}
	\item consistency of document and font type and font size,
	\item direct use of \TeX\ math mode in axis descriptions,
	\item consistency of data and figures (no third party tool necessary),
	\item inter document consistency using preamble configurations and styles.
\end{itemize}
Although not necessary, separate |.pdf| or |.eps| graphics can be generated using the |external| library developed as part of \Tikz.

\PGFPlots\ is build completely on \Tikz/\PGF. Knowledge of \Tikz\ will simplify the work with \PGFPlots, although it is not required.

\section[About PGFPlots: Preliminaries]{About {\normalfont\PGFPlots}: Preliminaries}
This section contains information about upgrades, the team, the installation (in case you need to do it manually) and troubleshooting. You may skip it completely except for the upgrade remarks.

However, note that this library requires at least \PGF\ version $2.00$. At the time of this writing, many \TeX-distributions still contain the older \PGF\ version $1.18$, so it may be necessary to install a recent \PGF\ prior to using \PGFPlots.

\subsection{Components}
\PGFPlots\ comes with two components:
\begin{enumerate}
	\item the plotting component (which you are currently reading) and
	\item the \PGFPlotstable\ component which simplifies number formatting and postprocessing of numerical tables. It comes as a separate package and has its own manual \href{file:pgfplotstable.pdf}{pgfplotstable.pdf}.
\end{enumerate}

\subsection{Upgrade remarks}
This release provides a lot of improvements which can be found in all detail in \texttt{ChangeLog} for interested readers. However, some attention is useful with respect to the following changes.

\subsubsection{New Optional Features}
\begin{enumerate}
	\item \PGFPlots\ 1.3 comes with user interface improvements. The technical distinction between ``behavior options'' and ``style options'' of older versions is no longer necessary (although still fully supported).

	\item \PGFPlots\ 1.3 has a new feature which allows to \emph{move axis labels tight to tick labels} automatically.

	Since this affects the spacing, it is not enabled be default.

	Use
\begin{codeexample}[code only]
\usepackage{pgfplots}
\pgfplotsset{compat=1.3}
\end{codeexample}
	in your preamble to benefit from the improved spacing\footnote{The |compat=1.3| setting is necessary for new features which move axis labels around like reversed axis. However, \PGFPlots\ will still work as it does in earlier versions, your documents will remain the same if you don't set it explicitly.}.
	Take a look at the next page for the precise definition of |compat|.

	\item \PGFPlots\ 1.3 now supports reversed axes. It is no longer necessary to use work--arounds with negative units.
\pgfkeys{/pdflinks/search key prefixes in/.add={/pgfplots/,}{}}

	Take a look at the |x dir=reverse| key.

	Existing work--arounds will still function properly. Use |\pgfplotsset{compat=1.3}| together with |x dir=reverse| to switch to the new version.
\end{enumerate}

\subsubsection{Old Features Which May Need Attention}
\begin{enumerate}
	\item The |scatter/classes| feature produces proper legends as of version 1.3. This may change the appearance of existing legends of plots with |scatter/classes|.

	\item Due to a small math bug in \PGF\ $2.00$, you \emph{can't} use the math expression `|-x^2|'. It is necessary to use `|0-x^2|' instead. The same holds for
		`|exp(-x^2)|'; use `|exp(0-x^2)|' instead.

		This will be fixed with \PGF\ $>2.00$.

	\item Starting with \PGFPlots\ $1.1$, |\tikzstyle| should \emph{no longer be used} to set \PGFPlots\ options.
	
	Although |\tikzstyle| is still supported for some older \PGFPlots\ options, you should replace any occurance of |\tikzstyle| with |\pgfplotsset{|\meta{style name}|/.style={|\meta{key-value-list}|}}| or the associated |/.append style| variant. See section~\ref{sec:styles} for more detail.
\end{enumerate}
I apologize for any inconvenience caused by these changes.

\begin{pgfplotskey}{compat=\mchoice{1.3,pre 1.3,default,newest} (initially default)}
	The preamble configuration 
\begin{codeexample}[code only]
\usepackage{pgfplots}
\pgfplotsset{compat=1.3}
\end{codeexample}
	allows to choose between backwards compatibility and most recent features.

	\PGFPlots\ version 1.3 comes with new features which may lead to different spacings compared to earlier versions:
	\begin{enumerate}
		\item Version 1.3 comes with the |xlabel near ticks| feature which places (in this case $x$) axis labels ``near'' the tick labels, taking the size of tick labels into account.
		
		Older versions used |xlabel absolute|, i.e.\ an absolute distance between the axis and axis labels (now matter how large or small tick labels are).

		The initial setting \emph{keeps backwards compatibility}.

		You are encouraged to use
\begin{codeexample}[code only]
\usepackage{pgfplots}
\pgfplotsset{compat=1.3}
\end{codeexample}
		in your preamble.
		
		\item Version 1.3 fixes a bug with |anchor=south west| which occurred mainly for reversed axes: the anchors where upside--down. This is fixed now.

		
		If you have written some work--around and want to keep it going, use

		|\pgfplotsset{compat/anchors=pre 1.3}|\pgfmanualpdflabel{/pgfplots/compat/anchors}{} 

		to disable the bug fix.
	\end{enumerate}

	Use |\pgfplotsset{compat=default}| to restore the factory settings.

	The setting |\pgfplotsset{compat=newest}| will always use features of the most recent version. This might result in small changes in the document's appearance.
\end{pgfplotskey}

\subsection{The Team}
\PGFPlots\ has been written mainly by Christian Feuers�nger with many improvements of Pascal Wolkotte and Nick Papior Andersen as a spare time project. We hope it is useful and provides valuable plots.

If you are interested in writing something but don't know how, consider reading the auxiliary manual \href{file:TeX-programming-notes.pdf}{TeX-programming-notes.pdf} which comes with \PGFPlots. It is far from complete, but maybe it is a good starting point (at least for more literature).

\subsection{Acknowledgements}
I thank God for all hours of enjoyed programming. I thank Pascal Wolkotte and Nick Papior Andersen for their programming efforts and contributions as part of the development team. I thank Stefan Tibus, who contributed the |plot shell| feature. I thank Tom Cashman for the contribution of the |reverse legend| feature. Special thanks go to Stefan Pinnow whose tests of \PGFPlots\ lead to numerous quality improvements. Furthermore, I thank Dr.~Schweitzer for many fruitful discussions and Dr.~Meine for his ideas and suggestions. Thanks as well to the many international contributors who provided feature requests or identified bugs or simply manual improvements!

Last but not least, I thank Till Tantau and Mark Wibrow for their excellent graphics (and more) package \PGF\ and \Tikz\ which is the base of \PGFPlots.

\include{pgfplots.install}%
\include{pgfplots.intro}%
\include{pgfplots.reference}%
\include{pgfplots.libs}%
\include{pgfplots.resources}%
\include{pgfplots.importexport}%
\include{pgfplots.basic.reference}%

\printindex

\bibliographystyle{abbrv} %gerapali} %gerabbrv} %gerunsrt.bst} %gerabbrv}% gerplain}
\nocite{pgfplotstable}
\nocite{programmingnotes}
\bibliography{pgfplots}
\end{document}
%
%%%%%%%%%%%%%%%%%%%%%%%%%%%%%%%%%%%%%%%%%%%%%%%%%%%%%%%%%%%%%%%%%%%%%%%%%%%%%
%
% Package pgfplots.sty documentation. 
%
% Copyright 2007/2008 by Christian Feuersaenger.
%
% This program is free software: you can redistribute it and/or modify
% it under the terms of the GNU General Public License as published by
% the Free Software Foundation, either version 3 of the License, or
% (at your option) any later version.
% 
% This program is distributed in the hope that it will be useful,
% but WITHOUT ANY WARRANTY; without even the implied warranty of
% MERCHANTABILITY or FITNESS FOR A PARTICULAR PURPOSE.  See the
% GNU General Public License for more details.
% 
% You should have received a copy of the GNU General Public License
% along with this program.  If not, see <http://www.gnu.org/licenses/>.
%
%
%%%%%%%%%%%%%%%%%%%%%%%%%%%%%%%%%%%%%%%%%%%%%%%%%%%%%%%%%%%%%%%%%%%%%%%%%%%%%
\input{pgfplots.preamble.tex}
%\RequirePackage[german,english,francais]{babel}

\pgfplotsmanualexternalexpensivetrue

%\usetikzlibrary{external}
%\tikzexternalize[prefix=figures/]{pgfplots}

\title{%
	Manual for Package \PGFPlots\\
	{\small Version \pgfplotsversion}\\
	{\small\href{http://sourceforge.net/projects/pgfplots}{http://sourceforge.net/projects/pgfplots}}}

%\includeonly{pgfplots.libs}


\begin{document}

\def\plotcoords{%
\addplot coordinates {
(5,8.312e-02)    (17,2.547e-02)   (49,7.407e-03)
(129,2.102e-03)  (321,5.874e-04)  (769,1.623e-04)
(1793,4.442e-05) (4097,1.207e-05) (9217,3.261e-06)
};

\addplot coordinates{
(7,8.472e-02)    (31,3.044e-02)    (111,1.022e-02)
(351,3.303e-03)  (1023,1.039e-03)  (2815,3.196e-04)
(7423,9.658e-05) (18943,2.873e-05) (47103,8.437e-06)
};

\addplot coordinates{
(9,7.881e-02)     (49,3.243e-02)    (209,1.232e-02)
(769,4.454e-03)   (2561,1.551e-03)  (7937,5.236e-04)
(23297,1.723e-04) (65537,5.545e-05) (178177,1.751e-05)
};

\addplot coordinates{
(11,6.887e-02)    (71,3.177e-02)     (351,1.341e-02)
(1471,5.334e-03)  (5503,2.027e-03)   (18943,7.415e-04)
(61183,2.628e-04) (187903,9.063e-05) (553983,3.053e-05)
};

\addplot coordinates{
(13,5.755e-02)     (97,2.925e-02)     (545,1.351e-02)
(2561,5.842e-03)   (10625,2.397e-03)  (40193,9.414e-04)
(141569,3.564e-04) (471041,1.308e-04) 
(1496065,4.670e-05)
};
}%
\maketitle
\begin{abstract}%
\PGFPlots\ draws high--quality function plots in normal or logarithmic scaling with a user-friendly interface directly in \TeX. The user supplies axis labels, legend entries and the plot coordinates for one or more plots and \PGFPlots\ applies axis scaling, computes any logarithms and axis ticks and draws the plots, supporting line plots, scatter plots, piecewise constant plots, bar plots, area plots, mesh-- and surface plots and some more. It is based on Till Tantau's package \PGF/\Tikz.
\end{abstract}
\tableofcontents
\section{Introduction}
This package provides tools to generate plots and labeled axes easily. It draws normal plots, logplots and semi-logplots, in two and three dimensions. Axis ticks, labels, legends (in case of multiple plots) can be added with key-value options. It can cycle through a set of predefined line/marker/color specifications. In summary, its purpose is to simplify the generation of high-quality function and/or data plots, and solving the problems of
\begin{itemize}
	\item consistency of document and font type and font size,
	\item direct use of \TeX\ math mode in axis descriptions,
	\item consistency of data and figures (no third party tool necessary),
	\item inter document consistency using preamble configurations and styles.
\end{itemize}
Although not necessary, separate |.pdf| or |.eps| graphics can be generated using the |external| library developed as part of \Tikz.

\PGFPlots\ is build completely on \Tikz/\PGF. Knowledge of \Tikz\ will simplify the work with \PGFPlots, although it is not required.

\section[About PGFPlots: Preliminaries]{About {\normalfont\PGFPlots}: Preliminaries}
This section contains information about upgrades, the team, the installation (in case you need to do it manually) and troubleshooting. You may skip it completely except for the upgrade remarks.

However, note that this library requires at least \PGF\ version $2.00$. At the time of this writing, many \TeX-distributions still contain the older \PGF\ version $1.18$, so it may be necessary to install a recent \PGF\ prior to using \PGFPlots.

\subsection{Components}
\PGFPlots\ comes with two components:
\begin{enumerate}
	\item the plotting component (which you are currently reading) and
	\item the \PGFPlotstable\ component which simplifies number formatting and postprocessing of numerical tables. It comes as a separate package and has its own manual \href{file:pgfplotstable.pdf}{pgfplotstable.pdf}.
\end{enumerate}

\subsection{Upgrade remarks}
This release provides a lot of improvements which can be found in all detail in \texttt{ChangeLog} for interested readers. However, some attention is useful with respect to the following changes.

\subsubsection{New Optional Features}
\begin{enumerate}
	\item \PGFPlots\ 1.3 comes with user interface improvements. The technical distinction between ``behavior options'' and ``style options'' of older versions is no longer necessary (although still fully supported).

	\item \PGFPlots\ 1.3 has a new feature which allows to \emph{move axis labels tight to tick labels} automatically.

	Since this affects the spacing, it is not enabled be default.

	Use
\begin{codeexample}[code only]
\usepackage{pgfplots}
\pgfplotsset{compat=1.3}
\end{codeexample}
	in your preamble to benefit from the improved spacing\footnote{The |compat=1.3| setting is necessary for new features which move axis labels around like reversed axis. However, \PGFPlots\ will still work as it does in earlier versions, your documents will remain the same if you don't set it explicitly.}.
	Take a look at the next page for the precise definition of |compat|.

	\item \PGFPlots\ 1.3 now supports reversed axes. It is no longer necessary to use work--arounds with negative units.
\pgfkeys{/pdflinks/search key prefixes in/.add={/pgfplots/,}{}}

	Take a look at the |x dir=reverse| key.

	Existing work--arounds will still function properly. Use |\pgfplotsset{compat=1.3}| together with |x dir=reverse| to switch to the new version.
\end{enumerate}

\subsubsection{Old Features Which May Need Attention}
\begin{enumerate}
	\item The |scatter/classes| feature produces proper legends as of version 1.3. This may change the appearance of existing legends of plots with |scatter/classes|.

	\item Due to a small math bug in \PGF\ $2.00$, you \emph{can't} use the math expression `|-x^2|'. It is necessary to use `|0-x^2|' instead. The same holds for
		`|exp(-x^2)|'; use `|exp(0-x^2)|' instead.

		This will be fixed with \PGF\ $>2.00$.

	\item Starting with \PGFPlots\ $1.1$, |\tikzstyle| should \emph{no longer be used} to set \PGFPlots\ options.
	
	Although |\tikzstyle| is still supported for some older \PGFPlots\ options, you should replace any occurance of |\tikzstyle| with |\pgfplotsset{|\meta{style name}|/.style={|\meta{key-value-list}|}}| or the associated |/.append style| variant. See section~\ref{sec:styles} for more detail.
\end{enumerate}
I apologize for any inconvenience caused by these changes.

\begin{pgfplotskey}{compat=\mchoice{1.3,pre 1.3,default,newest} (initially default)}
	The preamble configuration 
\begin{codeexample}[code only]
\usepackage{pgfplots}
\pgfplotsset{compat=1.3}
\end{codeexample}
	allows to choose between backwards compatibility and most recent features.

	\PGFPlots\ version 1.3 comes with new features which may lead to different spacings compared to earlier versions:
	\begin{enumerate}
		\item Version 1.3 comes with the |xlabel near ticks| feature which places (in this case $x$) axis labels ``near'' the tick labels, taking the size of tick labels into account.
		
		Older versions used |xlabel absolute|, i.e.\ an absolute distance between the axis and axis labels (now matter how large or small tick labels are).

		The initial setting \emph{keeps backwards compatibility}.

		You are encouraged to use
\begin{codeexample}[code only]
\usepackage{pgfplots}
\pgfplotsset{compat=1.3}
\end{codeexample}
		in your preamble.
		
		\item Version 1.3 fixes a bug with |anchor=south west| which occurred mainly for reversed axes: the anchors where upside--down. This is fixed now.

		
		If you have written some work--around and want to keep it going, use

		|\pgfplotsset{compat/anchors=pre 1.3}|\pgfmanualpdflabel{/pgfplots/compat/anchors}{} 

		to disable the bug fix.
	\end{enumerate}

	Use |\pgfplotsset{compat=default}| to restore the factory settings.

	The setting |\pgfplotsset{compat=newest}| will always use features of the most recent version. This might result in small changes in the document's appearance.
\end{pgfplotskey}

\subsection{The Team}
\PGFPlots\ has been written mainly by Christian Feuers�nger with many improvements of Pascal Wolkotte and Nick Papior Andersen as a spare time project. We hope it is useful and provides valuable plots.

If you are interested in writing something but don't know how, consider reading the auxiliary manual \href{file:TeX-programming-notes.pdf}{TeX-programming-notes.pdf} which comes with \PGFPlots. It is far from complete, but maybe it is a good starting point (at least for more literature).

\subsection{Acknowledgements}
I thank God for all hours of enjoyed programming. I thank Pascal Wolkotte and Nick Papior Andersen for their programming efforts and contributions as part of the development team. I thank Stefan Tibus, who contributed the |plot shell| feature. I thank Tom Cashman for the contribution of the |reverse legend| feature. Special thanks go to Stefan Pinnow whose tests of \PGFPlots\ lead to numerous quality improvements. Furthermore, I thank Dr.~Schweitzer for many fruitful discussions and Dr.~Meine for his ideas and suggestions. Thanks as well to the many international contributors who provided feature requests or identified bugs or simply manual improvements!

Last but not least, I thank Till Tantau and Mark Wibrow for their excellent graphics (and more) package \PGF\ and \Tikz\ which is the base of \PGFPlots.

\include{pgfplots.install}%
\include{pgfplots.intro}%
\include{pgfplots.reference}%
\include{pgfplots.libs}%
\include{pgfplots.resources}%
\include{pgfplots.importexport}%
\include{pgfplots.basic.reference}%

\printindex

\bibliographystyle{abbrv} %gerapali} %gerabbrv} %gerunsrt.bst} %gerabbrv}% gerplain}
\nocite{pgfplotstable}
\nocite{programmingnotes}
\bibliography{pgfplots}
\end{document}
%
%%%%%%%%%%%%%%%%%%%%%%%%%%%%%%%%%%%%%%%%%%%%%%%%%%%%%%%%%%%%%%%%%%%%%%%%%%%%%
%
% Package pgfplots.sty documentation. 
%
% Copyright 2007/2008 by Christian Feuersaenger.
%
% This program is free software: you can redistribute it and/or modify
% it under the terms of the GNU General Public License as published by
% the Free Software Foundation, either version 3 of the License, or
% (at your option) any later version.
% 
% This program is distributed in the hope that it will be useful,
% but WITHOUT ANY WARRANTY; without even the implied warranty of
% MERCHANTABILITY or FITNESS FOR A PARTICULAR PURPOSE.  See the
% GNU General Public License for more details.
% 
% You should have received a copy of the GNU General Public License
% along with this program.  If not, see <http://www.gnu.org/licenses/>.
%
%
%%%%%%%%%%%%%%%%%%%%%%%%%%%%%%%%%%%%%%%%%%%%%%%%%%%%%%%%%%%%%%%%%%%%%%%%%%%%%
\input{pgfplots.preamble.tex}
%\RequirePackage[german,english,francais]{babel}

\pgfplotsmanualexternalexpensivetrue

%\usetikzlibrary{external}
%\tikzexternalize[prefix=figures/]{pgfplots}

\title{%
	Manual for Package \PGFPlots\\
	{\small Version \pgfplotsversion}\\
	{\small\href{http://sourceforge.net/projects/pgfplots}{http://sourceforge.net/projects/pgfplots}}}

%\includeonly{pgfplots.libs}


\begin{document}

\def\plotcoords{%
\addplot coordinates {
(5,8.312e-02)    (17,2.547e-02)   (49,7.407e-03)
(129,2.102e-03)  (321,5.874e-04)  (769,1.623e-04)
(1793,4.442e-05) (4097,1.207e-05) (9217,3.261e-06)
};

\addplot coordinates{
(7,8.472e-02)    (31,3.044e-02)    (111,1.022e-02)
(351,3.303e-03)  (1023,1.039e-03)  (2815,3.196e-04)
(7423,9.658e-05) (18943,2.873e-05) (47103,8.437e-06)
};

\addplot coordinates{
(9,7.881e-02)     (49,3.243e-02)    (209,1.232e-02)
(769,4.454e-03)   (2561,1.551e-03)  (7937,5.236e-04)
(23297,1.723e-04) (65537,5.545e-05) (178177,1.751e-05)
};

\addplot coordinates{
(11,6.887e-02)    (71,3.177e-02)     (351,1.341e-02)
(1471,5.334e-03)  (5503,2.027e-03)   (18943,7.415e-04)
(61183,2.628e-04) (187903,9.063e-05) (553983,3.053e-05)
};

\addplot coordinates{
(13,5.755e-02)     (97,2.925e-02)     (545,1.351e-02)
(2561,5.842e-03)   (10625,2.397e-03)  (40193,9.414e-04)
(141569,3.564e-04) (471041,1.308e-04) 
(1496065,4.670e-05)
};
}%
\maketitle
\begin{abstract}%
\PGFPlots\ draws high--quality function plots in normal or logarithmic scaling with a user-friendly interface directly in \TeX. The user supplies axis labels, legend entries and the plot coordinates for one or more plots and \PGFPlots\ applies axis scaling, computes any logarithms and axis ticks and draws the plots, supporting line plots, scatter plots, piecewise constant plots, bar plots, area plots, mesh-- and surface plots and some more. It is based on Till Tantau's package \PGF/\Tikz.
\end{abstract}
\tableofcontents
\section{Introduction}
This package provides tools to generate plots and labeled axes easily. It draws normal plots, logplots and semi-logplots, in two and three dimensions. Axis ticks, labels, legends (in case of multiple plots) can be added with key-value options. It can cycle through a set of predefined line/marker/color specifications. In summary, its purpose is to simplify the generation of high-quality function and/or data plots, and solving the problems of
\begin{itemize}
	\item consistency of document and font type and font size,
	\item direct use of \TeX\ math mode in axis descriptions,
	\item consistency of data and figures (no third party tool necessary),
	\item inter document consistency using preamble configurations and styles.
\end{itemize}
Although not necessary, separate |.pdf| or |.eps| graphics can be generated using the |external| library developed as part of \Tikz.

\PGFPlots\ is build completely on \Tikz/\PGF. Knowledge of \Tikz\ will simplify the work with \PGFPlots, although it is not required.

\section[About PGFPlots: Preliminaries]{About {\normalfont\PGFPlots}: Preliminaries}
This section contains information about upgrades, the team, the installation (in case you need to do it manually) and troubleshooting. You may skip it completely except for the upgrade remarks.

However, note that this library requires at least \PGF\ version $2.00$. At the time of this writing, many \TeX-distributions still contain the older \PGF\ version $1.18$, so it may be necessary to install a recent \PGF\ prior to using \PGFPlots.

\subsection{Components}
\PGFPlots\ comes with two components:
\begin{enumerate}
	\item the plotting component (which you are currently reading) and
	\item the \PGFPlotstable\ component which simplifies number formatting and postprocessing of numerical tables. It comes as a separate package and has its own manual \href{file:pgfplotstable.pdf}{pgfplotstable.pdf}.
\end{enumerate}

\subsection{Upgrade remarks}
This release provides a lot of improvements which can be found in all detail in \texttt{ChangeLog} for interested readers. However, some attention is useful with respect to the following changes.

\subsubsection{New Optional Features}
\begin{enumerate}
	\item \PGFPlots\ 1.3 comes with user interface improvements. The technical distinction between ``behavior options'' and ``style options'' of older versions is no longer necessary (although still fully supported).

	\item \PGFPlots\ 1.3 has a new feature which allows to \emph{move axis labels tight to tick labels} automatically.

	Since this affects the spacing, it is not enabled be default.

	Use
\begin{codeexample}[code only]
\usepackage{pgfplots}
\pgfplotsset{compat=1.3}
\end{codeexample}
	in your preamble to benefit from the improved spacing\footnote{The |compat=1.3| setting is necessary for new features which move axis labels around like reversed axis. However, \PGFPlots\ will still work as it does in earlier versions, your documents will remain the same if you don't set it explicitly.}.
	Take a look at the next page for the precise definition of |compat|.

	\item \PGFPlots\ 1.3 now supports reversed axes. It is no longer necessary to use work--arounds with negative units.
\pgfkeys{/pdflinks/search key prefixes in/.add={/pgfplots/,}{}}

	Take a look at the |x dir=reverse| key.

	Existing work--arounds will still function properly. Use |\pgfplotsset{compat=1.3}| together with |x dir=reverse| to switch to the new version.
\end{enumerate}

\subsubsection{Old Features Which May Need Attention}
\begin{enumerate}
	\item The |scatter/classes| feature produces proper legends as of version 1.3. This may change the appearance of existing legends of plots with |scatter/classes|.

	\item Due to a small math bug in \PGF\ $2.00$, you \emph{can't} use the math expression `|-x^2|'. It is necessary to use `|0-x^2|' instead. The same holds for
		`|exp(-x^2)|'; use `|exp(0-x^2)|' instead.

		This will be fixed with \PGF\ $>2.00$.

	\item Starting with \PGFPlots\ $1.1$, |\tikzstyle| should \emph{no longer be used} to set \PGFPlots\ options.
	
	Although |\tikzstyle| is still supported for some older \PGFPlots\ options, you should replace any occurance of |\tikzstyle| with |\pgfplotsset{|\meta{style name}|/.style={|\meta{key-value-list}|}}| or the associated |/.append style| variant. See section~\ref{sec:styles} for more detail.
\end{enumerate}
I apologize for any inconvenience caused by these changes.

\begin{pgfplotskey}{compat=\mchoice{1.3,pre 1.3,default,newest} (initially default)}
	The preamble configuration 
\begin{codeexample}[code only]
\usepackage{pgfplots}
\pgfplotsset{compat=1.3}
\end{codeexample}
	allows to choose between backwards compatibility and most recent features.

	\PGFPlots\ version 1.3 comes with new features which may lead to different spacings compared to earlier versions:
	\begin{enumerate}
		\item Version 1.3 comes with the |xlabel near ticks| feature which places (in this case $x$) axis labels ``near'' the tick labels, taking the size of tick labels into account.
		
		Older versions used |xlabel absolute|, i.e.\ an absolute distance between the axis and axis labels (now matter how large or small tick labels are).

		The initial setting \emph{keeps backwards compatibility}.

		You are encouraged to use
\begin{codeexample}[code only]
\usepackage{pgfplots}
\pgfplotsset{compat=1.3}
\end{codeexample}
		in your preamble.
		
		\item Version 1.3 fixes a bug with |anchor=south west| which occurred mainly for reversed axes: the anchors where upside--down. This is fixed now.

		
		If you have written some work--around and want to keep it going, use

		|\pgfplotsset{compat/anchors=pre 1.3}|\pgfmanualpdflabel{/pgfplots/compat/anchors}{} 

		to disable the bug fix.
	\end{enumerate}

	Use |\pgfplotsset{compat=default}| to restore the factory settings.

	The setting |\pgfplotsset{compat=newest}| will always use features of the most recent version. This might result in small changes in the document's appearance.
\end{pgfplotskey}

\subsection{The Team}
\PGFPlots\ has been written mainly by Christian Feuers�nger with many improvements of Pascal Wolkotte and Nick Papior Andersen as a spare time project. We hope it is useful and provides valuable plots.

If you are interested in writing something but don't know how, consider reading the auxiliary manual \href{file:TeX-programming-notes.pdf}{TeX-programming-notes.pdf} which comes with \PGFPlots. It is far from complete, but maybe it is a good starting point (at least for more literature).

\subsection{Acknowledgements}
I thank God for all hours of enjoyed programming. I thank Pascal Wolkotte and Nick Papior Andersen for their programming efforts and contributions as part of the development team. I thank Stefan Tibus, who contributed the |plot shell| feature. I thank Tom Cashman for the contribution of the |reverse legend| feature. Special thanks go to Stefan Pinnow whose tests of \PGFPlots\ lead to numerous quality improvements. Furthermore, I thank Dr.~Schweitzer for many fruitful discussions and Dr.~Meine for his ideas and suggestions. Thanks as well to the many international contributors who provided feature requests or identified bugs or simply manual improvements!

Last but not least, I thank Till Tantau and Mark Wibrow for their excellent graphics (and more) package \PGF\ and \Tikz\ which is the base of \PGFPlots.

\include{pgfplots.install}%
\include{pgfplots.intro}%
\include{pgfplots.reference}%
\include{pgfplots.libs}%
\include{pgfplots.resources}%
\include{pgfplots.importexport}%
\include{pgfplots.basic.reference}%

\printindex

\bibliographystyle{abbrv} %gerapali} %gerabbrv} %gerunsrt.bst} %gerabbrv}% gerplain}
\nocite{pgfplotstable}
\nocite{programmingnotes}
\bibliography{pgfplots}
\end{document}
%
%%%%%%%%%%%%%%%%%%%%%%%%%%%%%%%%%%%%%%%%%%%%%%%%%%%%%%%%%%%%%%%%%%%%%%%%%%%%%
%
% Package pgfplots.sty documentation. 
%
% Copyright 2007/2008 by Christian Feuersaenger.
%
% This program is free software: you can redistribute it and/or modify
% it under the terms of the GNU General Public License as published by
% the Free Software Foundation, either version 3 of the License, or
% (at your option) any later version.
% 
% This program is distributed in the hope that it will be useful,
% but WITHOUT ANY WARRANTY; without even the implied warranty of
% MERCHANTABILITY or FITNESS FOR A PARTICULAR PURPOSE.  See the
% GNU General Public License for more details.
% 
% You should have received a copy of the GNU General Public License
% along with this program.  If not, see <http://www.gnu.org/licenses/>.
%
%
%%%%%%%%%%%%%%%%%%%%%%%%%%%%%%%%%%%%%%%%%%%%%%%%%%%%%%%%%%%%%%%%%%%%%%%%%%%%%
\input{pgfplots.preamble.tex}
%\RequirePackage[german,english,francais]{babel}

\pgfplotsmanualexternalexpensivetrue

%\usetikzlibrary{external}
%\tikzexternalize[prefix=figures/]{pgfplots}

\title{%
	Manual for Package \PGFPlots\\
	{\small Version \pgfplotsversion}\\
	{\small\href{http://sourceforge.net/projects/pgfplots}{http://sourceforge.net/projects/pgfplots}}}

%\includeonly{pgfplots.libs}


\begin{document}

\def\plotcoords{%
\addplot coordinates {
(5,8.312e-02)    (17,2.547e-02)   (49,7.407e-03)
(129,2.102e-03)  (321,5.874e-04)  (769,1.623e-04)
(1793,4.442e-05) (4097,1.207e-05) (9217,3.261e-06)
};

\addplot coordinates{
(7,8.472e-02)    (31,3.044e-02)    (111,1.022e-02)
(351,3.303e-03)  (1023,1.039e-03)  (2815,3.196e-04)
(7423,9.658e-05) (18943,2.873e-05) (47103,8.437e-06)
};

\addplot coordinates{
(9,7.881e-02)     (49,3.243e-02)    (209,1.232e-02)
(769,4.454e-03)   (2561,1.551e-03)  (7937,5.236e-04)
(23297,1.723e-04) (65537,5.545e-05) (178177,1.751e-05)
};

\addplot coordinates{
(11,6.887e-02)    (71,3.177e-02)     (351,1.341e-02)
(1471,5.334e-03)  (5503,2.027e-03)   (18943,7.415e-04)
(61183,2.628e-04) (187903,9.063e-05) (553983,3.053e-05)
};

\addplot coordinates{
(13,5.755e-02)     (97,2.925e-02)     (545,1.351e-02)
(2561,5.842e-03)   (10625,2.397e-03)  (40193,9.414e-04)
(141569,3.564e-04) (471041,1.308e-04) 
(1496065,4.670e-05)
};
}%
\maketitle
\begin{abstract}%
\PGFPlots\ draws high--quality function plots in normal or logarithmic scaling with a user-friendly interface directly in \TeX. The user supplies axis labels, legend entries and the plot coordinates for one or more plots and \PGFPlots\ applies axis scaling, computes any logarithms and axis ticks and draws the plots, supporting line plots, scatter plots, piecewise constant plots, bar plots, area plots, mesh-- and surface plots and some more. It is based on Till Tantau's package \PGF/\Tikz.
\end{abstract}
\tableofcontents
\section{Introduction}
This package provides tools to generate plots and labeled axes easily. It draws normal plots, logplots and semi-logplots, in two and three dimensions. Axis ticks, labels, legends (in case of multiple plots) can be added with key-value options. It can cycle through a set of predefined line/marker/color specifications. In summary, its purpose is to simplify the generation of high-quality function and/or data plots, and solving the problems of
\begin{itemize}
	\item consistency of document and font type and font size,
	\item direct use of \TeX\ math mode in axis descriptions,
	\item consistency of data and figures (no third party tool necessary),
	\item inter document consistency using preamble configurations and styles.
\end{itemize}
Although not necessary, separate |.pdf| or |.eps| graphics can be generated using the |external| library developed as part of \Tikz.

\PGFPlots\ is build completely on \Tikz/\PGF. Knowledge of \Tikz\ will simplify the work with \PGFPlots, although it is not required.

\section[About PGFPlots: Preliminaries]{About {\normalfont\PGFPlots}: Preliminaries}
This section contains information about upgrades, the team, the installation (in case you need to do it manually) and troubleshooting. You may skip it completely except for the upgrade remarks.

However, note that this library requires at least \PGF\ version $2.00$. At the time of this writing, many \TeX-distributions still contain the older \PGF\ version $1.18$, so it may be necessary to install a recent \PGF\ prior to using \PGFPlots.

\subsection{Components}
\PGFPlots\ comes with two components:
\begin{enumerate}
	\item the plotting component (which you are currently reading) and
	\item the \PGFPlotstable\ component which simplifies number formatting and postprocessing of numerical tables. It comes as a separate package and has its own manual \href{file:pgfplotstable.pdf}{pgfplotstable.pdf}.
\end{enumerate}

\subsection{Upgrade remarks}
This release provides a lot of improvements which can be found in all detail in \texttt{ChangeLog} for interested readers. However, some attention is useful with respect to the following changes.

\subsubsection{New Optional Features}
\begin{enumerate}
	\item \PGFPlots\ 1.3 comes with user interface improvements. The technical distinction between ``behavior options'' and ``style options'' of older versions is no longer necessary (although still fully supported).

	\item \PGFPlots\ 1.3 has a new feature which allows to \emph{move axis labels tight to tick labels} automatically.

	Since this affects the spacing, it is not enabled be default.

	Use
\begin{codeexample}[code only]
\usepackage{pgfplots}
\pgfplotsset{compat=1.3}
\end{codeexample}
	in your preamble to benefit from the improved spacing\footnote{The |compat=1.3| setting is necessary for new features which move axis labels around like reversed axis. However, \PGFPlots\ will still work as it does in earlier versions, your documents will remain the same if you don't set it explicitly.}.
	Take a look at the next page for the precise definition of |compat|.

	\item \PGFPlots\ 1.3 now supports reversed axes. It is no longer necessary to use work--arounds with negative units.
\pgfkeys{/pdflinks/search key prefixes in/.add={/pgfplots/,}{}}

	Take a look at the |x dir=reverse| key.

	Existing work--arounds will still function properly. Use |\pgfplotsset{compat=1.3}| together with |x dir=reverse| to switch to the new version.
\end{enumerate}

\subsubsection{Old Features Which May Need Attention}
\begin{enumerate}
	\item The |scatter/classes| feature produces proper legends as of version 1.3. This may change the appearance of existing legends of plots with |scatter/classes|.

	\item Due to a small math bug in \PGF\ $2.00$, you \emph{can't} use the math expression `|-x^2|'. It is necessary to use `|0-x^2|' instead. The same holds for
		`|exp(-x^2)|'; use `|exp(0-x^2)|' instead.

		This will be fixed with \PGF\ $>2.00$.

	\item Starting with \PGFPlots\ $1.1$, |\tikzstyle| should \emph{no longer be used} to set \PGFPlots\ options.
	
	Although |\tikzstyle| is still supported for some older \PGFPlots\ options, you should replace any occurance of |\tikzstyle| with |\pgfplotsset{|\meta{style name}|/.style={|\meta{key-value-list}|}}| or the associated |/.append style| variant. See section~\ref{sec:styles} for more detail.
\end{enumerate}
I apologize for any inconvenience caused by these changes.

\begin{pgfplotskey}{compat=\mchoice{1.3,pre 1.3,default,newest} (initially default)}
	The preamble configuration 
\begin{codeexample}[code only]
\usepackage{pgfplots}
\pgfplotsset{compat=1.3}
\end{codeexample}
	allows to choose between backwards compatibility and most recent features.

	\PGFPlots\ version 1.3 comes with new features which may lead to different spacings compared to earlier versions:
	\begin{enumerate}
		\item Version 1.3 comes with the |xlabel near ticks| feature which places (in this case $x$) axis labels ``near'' the tick labels, taking the size of tick labels into account.
		
		Older versions used |xlabel absolute|, i.e.\ an absolute distance between the axis and axis labels (now matter how large or small tick labels are).

		The initial setting \emph{keeps backwards compatibility}.

		You are encouraged to use
\begin{codeexample}[code only]
\usepackage{pgfplots}
\pgfplotsset{compat=1.3}
\end{codeexample}
		in your preamble.
		
		\item Version 1.3 fixes a bug with |anchor=south west| which occurred mainly for reversed axes: the anchors where upside--down. This is fixed now.

		
		If you have written some work--around and want to keep it going, use

		|\pgfplotsset{compat/anchors=pre 1.3}|\pgfmanualpdflabel{/pgfplots/compat/anchors}{} 

		to disable the bug fix.
	\end{enumerate}

	Use |\pgfplotsset{compat=default}| to restore the factory settings.

	The setting |\pgfplotsset{compat=newest}| will always use features of the most recent version. This might result in small changes in the document's appearance.
\end{pgfplotskey}

\subsection{The Team}
\PGFPlots\ has been written mainly by Christian Feuers�nger with many improvements of Pascal Wolkotte and Nick Papior Andersen as a spare time project. We hope it is useful and provides valuable plots.

If you are interested in writing something but don't know how, consider reading the auxiliary manual \href{file:TeX-programming-notes.pdf}{TeX-programming-notes.pdf} which comes with \PGFPlots. It is far from complete, but maybe it is a good starting point (at least for more literature).

\subsection{Acknowledgements}
I thank God for all hours of enjoyed programming. I thank Pascal Wolkotte and Nick Papior Andersen for their programming efforts and contributions as part of the development team. I thank Stefan Tibus, who contributed the |plot shell| feature. I thank Tom Cashman for the contribution of the |reverse legend| feature. Special thanks go to Stefan Pinnow whose tests of \PGFPlots\ lead to numerous quality improvements. Furthermore, I thank Dr.~Schweitzer for many fruitful discussions and Dr.~Meine for his ideas and suggestions. Thanks as well to the many international contributors who provided feature requests or identified bugs or simply manual improvements!

Last but not least, I thank Till Tantau and Mark Wibrow for their excellent graphics (and more) package \PGF\ and \Tikz\ which is the base of \PGFPlots.

\include{pgfplots.install}%
\include{pgfplots.intro}%
\include{pgfplots.reference}%
\include{pgfplots.libs}%
\include{pgfplots.resources}%
\include{pgfplots.importexport}%
\include{pgfplots.basic.reference}%

\printindex

\bibliographystyle{abbrv} %gerapali} %gerabbrv} %gerunsrt.bst} %gerabbrv}% gerplain}
\nocite{pgfplotstable}
\nocite{programmingnotes}
\bibliography{pgfplots}
\end{document}
%
\include{pgfplots.basic.reference}%

\printindex

\bibliographystyle{abbrv} %gerapali} %gerabbrv} %gerunsrt.bst} %gerabbrv}% gerplain}
\nocite{pgfplotstable}
\nocite{programmingnotes}
\bibliography{pgfplots}
\end{document}
%
%%%%%%%%%%%%%%%%%%%%%%%%%%%%%%%%%%%%%%%%%%%%%%%%%%%%%%%%%%%%%%%%%%%%%%%%%%%%%
%
% Package pgfplots.sty documentation. 
%
% Copyright 2007/2008 by Christian Feuersaenger.
%
% This program is free software: you can redistribute it and/or modify
% it under the terms of the GNU General Public License as published by
% the Free Software Foundation, either version 3 of the License, or
% (at your option) any later version.
% 
% This program is distributed in the hope that it will be useful,
% but WITHOUT ANY WARRANTY; without even the implied warranty of
% MERCHANTABILITY or FITNESS FOR A PARTICULAR PURPOSE.  See the
% GNU General Public License for more details.
% 
% You should have received a copy of the GNU General Public License
% along with this program.  If not, see <http://www.gnu.org/licenses/>.
%
%
%%%%%%%%%%%%%%%%%%%%%%%%%%%%%%%%%%%%%%%%%%%%%%%%%%%%%%%%%%%%%%%%%%%%%%%%%%%%%
%%%%%%%%%%%%%%%%%%%%%%%%%%%%%%%%%%%%%%%%%%%%%%%%%%%%%%%%%%%%%%%%%%%%%%%%%%%%%
%
% Package pgfplots.sty documentation. 
%
% Copyright 2007/2008 by Christian Feuersaenger.
%
% This program is free software: you can redistribute it and/or modify
% it under the terms of the GNU General Public License as published by
% the Free Software Foundation, either version 3 of the License, or
% (at your option) any later version.
% 
% This program is distributed in the hope that it will be useful,
% but WITHOUT ANY WARRANTY; without even the implied warranty of
% MERCHANTABILITY or FITNESS FOR A PARTICULAR PURPOSE.  See the
% GNU General Public License for more details.
% 
% You should have received a copy of the GNU General Public License
% along with this program.  If not, see <http://www.gnu.org/licenses/>.
%
%
%%%%%%%%%%%%%%%%%%%%%%%%%%%%%%%%%%%%%%%%%%%%%%%%%%%%%%%%%%%%%%%%%%%%%%%%%%%%%
\documentclass[a4paper]{ltxdoc}

\usepackage{makeidx}

% DON't let hyperref overload the format of index and glossary. 
% I want to do that on my own in the stylefiles for makeindex...
\makeatletter
\let\@old@wrindex=\@wrindex
\makeatother

\usepackage{ifpdf}
\ifpdf
	\usepackage[pdftex]{hyperref}
\else
	\usepackage[dvipdfm]{hyperref}
\fi
\hypersetup{pdfborder={0 0 0}}

\makeatletter
\let\@wrindex=\@old@wrindex
\makeatother

% Formatiere Seitennummern im Index:
\newcommand{\indexpageno}[1]{%
	{\bfseries\hyperpage{#1}}%
}


\newcommand{\R}{\mathbb{R}}
\newcommand{\N}{\mathbb{N}}
\newcommand{\Z}{\mathbb{Z}}

\long\def\COMMENTLOWLEVEL#1\ENDCOMMENT{}
\def\ENDCOMMENT{}

\usepackage{textcomp}
\usepackage{booktabs}

\usepackage{calc}
\usepackage[formats]{listings}
%\usepackage{courier} % don't use it - the '^' character can't be copy-pasted in courier

\usepackage{array}
\lstset{%
	basicstyle=\ttfamily,
	language=[LaTeX]tex, % Seems as if \lstset{language=tex} must be invoked BEFORE loading tikz!?
	tabsize=4,
	breaklines=true,
	breakindent=0pt
}

\ifpdf
	\pdfinfo {
		/Author	(Christian Feuersaenger)
	}

\else
	\def\pgfsysdriver{pgfsys-dvipdfm.def}
\fi
%\def\pgfsysdriver{pgfsys-pdftex.def}
\usepackage{pgfplots}

\ifpdf
	\usetikzlibrary{pgfplotsclickable}
\fi

%\usepackage{fp}
% ATTENTION:
% this requires pgf version NEWER than 2.00 :
%\usetikzlibrary{fixedpointarithmetic}

\usetikzlibrary{dateplot}

\usepackage[a4paper,left=2.25cm,right=2.25cm,top=2.5cm,bottom=2.5cm,nohead]{geometry}
\usepackage{amsmath,amssymb}
\usepackage{xxcolor}
\usepackage{pifont}
\usepackage[latin1]{inputenc}
\usepackage{amsmath}
\usepackage{eurosym}
\input{pgfplots-macros}

\pgfqkeys{/codeexample}{%
	every codeexample/.style={
		width=8cm,
		/pgfplots/every axis/.append style={legend style={fill=graphicbackground}}
	},
	tabsize=4,
}
\pgfplotsset{
	every axis/.append style={width=7cm},
	filter discard warning=false,
}

\usetikzlibrary{backgrounds,patterns}
% Global styles:
\tikzset{
  shape example/.style={
    color=black!30,
    draw,
    fill=yellow!30,
    line width=.5cm,
    inner xsep=2.5cm,
    inner ysep=0.5cm}
}

\newcommand{\FIXME}[1]{\textcolor{red}{(FIXME: #1)}}

% fuer endvironment 'sidewaysfigure' bspw
% \usepackage{rotating}

\newcommand\Tikz{Ti\textit kZ}
\newcommand\PGF{\textsc{pgf}}
\newcommand\PGFPlots{\textsc{pgfplots}}
\def\PGFPlotstable{\textsc{PgfplotsTable}}


\makeindex

% Fix overful hboxes automatically:
\tolerance=2000
\emergencystretch=10pt

\tikzset{prefix=gnuplot/pgfplots_} % prefix for 'plot function'

\author{%
	Christian Feuers\"anger\footnote{\url{http://wissrech.ins.uni-bonn.de/people/feuersaenger}}\\%
	Institut f\"ur Numerische Simulation\\
	Universit\"at Bonn, Germany}


%\RequirePackage[german,english,francais]{babel}

\pgfplotsmanualexternalexpensivetrue

%\usetikzlibrary{external}
%\tikzexternalize[prefix=figures/]{pgfplots}

\title{%
	Manual for Package \PGFPlots\\
	{\small Version \pgfplotsversion}\\
	{\small\href{http://sourceforge.net/projects/pgfplots}{http://sourceforge.net/projects/pgfplots}}}

%\includeonly{pgfplots.libs}


\begin{document}

\def\plotcoords{%
\addplot coordinates {
(5,8.312e-02)    (17,2.547e-02)   (49,7.407e-03)
(129,2.102e-03)  (321,5.874e-04)  (769,1.623e-04)
(1793,4.442e-05) (4097,1.207e-05) (9217,3.261e-06)
};

\addplot coordinates{
(7,8.472e-02)    (31,3.044e-02)    (111,1.022e-02)
(351,3.303e-03)  (1023,1.039e-03)  (2815,3.196e-04)
(7423,9.658e-05) (18943,2.873e-05) (47103,8.437e-06)
};

\addplot coordinates{
(9,7.881e-02)     (49,3.243e-02)    (209,1.232e-02)
(769,4.454e-03)   (2561,1.551e-03)  (7937,5.236e-04)
(23297,1.723e-04) (65537,5.545e-05) (178177,1.751e-05)
};

\addplot coordinates{
(11,6.887e-02)    (71,3.177e-02)     (351,1.341e-02)
(1471,5.334e-03)  (5503,2.027e-03)   (18943,7.415e-04)
(61183,2.628e-04) (187903,9.063e-05) (553983,3.053e-05)
};

\addplot coordinates{
(13,5.755e-02)     (97,2.925e-02)     (545,1.351e-02)
(2561,5.842e-03)   (10625,2.397e-03)  (40193,9.414e-04)
(141569,3.564e-04) (471041,1.308e-04) 
(1496065,4.670e-05)
};
}%
\maketitle
\begin{abstract}%
\PGFPlots\ draws high--quality function plots in normal or logarithmic scaling with a user-friendly interface directly in \TeX. The user supplies axis labels, legend entries and the plot coordinates for one or more plots and \PGFPlots\ applies axis scaling, computes any logarithms and axis ticks and draws the plots, supporting line plots, scatter plots, piecewise constant plots, bar plots, area plots, mesh-- and surface plots and some more. It is based on Till Tantau's package \PGF/\Tikz.
\end{abstract}
\tableofcontents
\section{Introduction}
This package provides tools to generate plots and labeled axes easily. It draws normal plots, logplots and semi-logplots, in two and three dimensions. Axis ticks, labels, legends (in case of multiple plots) can be added with key-value options. It can cycle through a set of predefined line/marker/color specifications. In summary, its purpose is to simplify the generation of high-quality function and/or data plots, and solving the problems of
\begin{itemize}
	\item consistency of document and font type and font size,
	\item direct use of \TeX\ math mode in axis descriptions,
	\item consistency of data and figures (no third party tool necessary),
	\item inter document consistency using preamble configurations and styles.
\end{itemize}
Although not necessary, separate |.pdf| or |.eps| graphics can be generated using the |external| library developed as part of \Tikz.

\PGFPlots\ is build completely on \Tikz/\PGF. Knowledge of \Tikz\ will simplify the work with \PGFPlots, although it is not required.

\section[About PGFPlots: Preliminaries]{About {\normalfont\PGFPlots}: Preliminaries}
This section contains information about upgrades, the team, the installation (in case you need to do it manually) and troubleshooting. You may skip it completely except for the upgrade remarks.

However, note that this library requires at least \PGF\ version $2.00$. At the time of this writing, many \TeX-distributions still contain the older \PGF\ version $1.18$, so it may be necessary to install a recent \PGF\ prior to using \PGFPlots.

\subsection{Components}
\PGFPlots\ comes with two components:
\begin{enumerate}
	\item the plotting component (which you are currently reading) and
	\item the \PGFPlotstable\ component which simplifies number formatting and postprocessing of numerical tables. It comes as a separate package and has its own manual \href{file:pgfplotstable.pdf}{pgfplotstable.pdf}.
\end{enumerate}

\subsection{Upgrade remarks}
This release provides a lot of improvements which can be found in all detail in \texttt{ChangeLog} for interested readers. However, some attention is useful with respect to the following changes.

\subsubsection{New Optional Features}
\begin{enumerate}
	\item \PGFPlots\ 1.3 comes with user interface improvements. The technical distinction between ``behavior options'' and ``style options'' of older versions is no longer necessary (although still fully supported).

	\item \PGFPlots\ 1.3 has a new feature which allows to \emph{move axis labels tight to tick labels} automatically.

	Since this affects the spacing, it is not enabled be default.

	Use
\begin{codeexample}[code only]
\usepackage{pgfplots}
\pgfplotsset{compat=1.3}
\end{codeexample}
	in your preamble to benefit from the improved spacing\footnote{The |compat=1.3| setting is necessary for new features which move axis labels around like reversed axis. However, \PGFPlots\ will still work as it does in earlier versions, your documents will remain the same if you don't set it explicitly.}.
	Take a look at the next page for the precise definition of |compat|.

	\item \PGFPlots\ 1.3 now supports reversed axes. It is no longer necessary to use work--arounds with negative units.
\pgfkeys{/pdflinks/search key prefixes in/.add={/pgfplots/,}{}}

	Take a look at the |x dir=reverse| key.

	Existing work--arounds will still function properly. Use |\pgfplotsset{compat=1.3}| together with |x dir=reverse| to switch to the new version.
\end{enumerate}

\subsubsection{Old Features Which May Need Attention}
\begin{enumerate}
	\item The |scatter/classes| feature produces proper legends as of version 1.3. This may change the appearance of existing legends of plots with |scatter/classes|.

	\item Due to a small math bug in \PGF\ $2.00$, you \emph{can't} use the math expression `|-x^2|'. It is necessary to use `|0-x^2|' instead. The same holds for
		`|exp(-x^2)|'; use `|exp(0-x^2)|' instead.

		This will be fixed with \PGF\ $>2.00$.

	\item Starting with \PGFPlots\ $1.1$, |\tikzstyle| should \emph{no longer be used} to set \PGFPlots\ options.
	
	Although |\tikzstyle| is still supported for some older \PGFPlots\ options, you should replace any occurance of |\tikzstyle| with |\pgfplotsset{|\meta{style name}|/.style={|\meta{key-value-list}|}}| or the associated |/.append style| variant. See section~\ref{sec:styles} for more detail.
\end{enumerate}
I apologize for any inconvenience caused by these changes.

\begin{pgfplotskey}{compat=\mchoice{1.3,pre 1.3,default,newest} (initially default)}
	The preamble configuration 
\begin{codeexample}[code only]
\usepackage{pgfplots}
\pgfplotsset{compat=1.3}
\end{codeexample}
	allows to choose between backwards compatibility and most recent features.

	\PGFPlots\ version 1.3 comes with new features which may lead to different spacings compared to earlier versions:
	\begin{enumerate}
		\item Version 1.3 comes with the |xlabel near ticks| feature which places (in this case $x$) axis labels ``near'' the tick labels, taking the size of tick labels into account.
		
		Older versions used |xlabel absolute|, i.e.\ an absolute distance between the axis and axis labels (now matter how large or small tick labels are).

		The initial setting \emph{keeps backwards compatibility}.

		You are encouraged to use
\begin{codeexample}[code only]
\usepackage{pgfplots}
\pgfplotsset{compat=1.3}
\end{codeexample}
		in your preamble.
		
		\item Version 1.3 fixes a bug with |anchor=south west| which occurred mainly for reversed axes: the anchors where upside--down. This is fixed now.

		
		If you have written some work--around and want to keep it going, use

		|\pgfplotsset{compat/anchors=pre 1.3}|\pgfmanualpdflabel{/pgfplots/compat/anchors}{} 

		to disable the bug fix.
	\end{enumerate}

	Use |\pgfplotsset{compat=default}| to restore the factory settings.

	The setting |\pgfplotsset{compat=newest}| will always use features of the most recent version. This might result in small changes in the document's appearance.
\end{pgfplotskey}

\subsection{The Team}
\PGFPlots\ has been written mainly by Christian Feuers�nger with many improvements of Pascal Wolkotte and Nick Papior Andersen as a spare time project. We hope it is useful and provides valuable plots.

If you are interested in writing something but don't know how, consider reading the auxiliary manual \href{file:TeX-programming-notes.pdf}{TeX-programming-notes.pdf} which comes with \PGFPlots. It is far from complete, but maybe it is a good starting point (at least for more literature).

\subsection{Acknowledgements}
I thank God for all hours of enjoyed programming. I thank Pascal Wolkotte and Nick Papior Andersen for their programming efforts and contributions as part of the development team. I thank Stefan Tibus, who contributed the |plot shell| feature. I thank Tom Cashman for the contribution of the |reverse legend| feature. Special thanks go to Stefan Pinnow whose tests of \PGFPlots\ lead to numerous quality improvements. Furthermore, I thank Dr.~Schweitzer for many fruitful discussions and Dr.~Meine for his ideas and suggestions. Thanks as well to the many international contributors who provided feature requests or identified bugs or simply manual improvements!

Last but not least, I thank Till Tantau and Mark Wibrow for their excellent graphics (and more) package \PGF\ and \Tikz\ which is the base of \PGFPlots.

%%%%%%%%%%%%%%%%%%%%%%%%%%%%%%%%%%%%%%%%%%%%%%%%%%%%%%%%%%%%%%%%%%%%%%%%%%%%%
%
% Package pgfplots.sty documentation. 
%
% Copyright 2007/2008 by Christian Feuersaenger.
%
% This program is free software: you can redistribute it and/or modify
% it under the terms of the GNU General Public License as published by
% the Free Software Foundation, either version 3 of the License, or
% (at your option) any later version.
% 
% This program is distributed in the hope that it will be useful,
% but WITHOUT ANY WARRANTY; without even the implied warranty of
% MERCHANTABILITY or FITNESS FOR A PARTICULAR PURPOSE.  See the
% GNU General Public License for more details.
% 
% You should have received a copy of the GNU General Public License
% along with this program.  If not, see <http://www.gnu.org/licenses/>.
%
%
%%%%%%%%%%%%%%%%%%%%%%%%%%%%%%%%%%%%%%%%%%%%%%%%%%%%%%%%%%%%%%%%%%%%%%%%%%%%%
\input{pgfplots.preamble.tex}
%\RequirePackage[german,english,francais]{babel}

\pgfplotsmanualexternalexpensivetrue

%\usetikzlibrary{external}
%\tikzexternalize[prefix=figures/]{pgfplots}

\title{%
	Manual for Package \PGFPlots\\
	{\small Version \pgfplotsversion}\\
	{\small\href{http://sourceforge.net/projects/pgfplots}{http://sourceforge.net/projects/pgfplots}}}

%\includeonly{pgfplots.libs}


\begin{document}

\def\plotcoords{%
\addplot coordinates {
(5,8.312e-02)    (17,2.547e-02)   (49,7.407e-03)
(129,2.102e-03)  (321,5.874e-04)  (769,1.623e-04)
(1793,4.442e-05) (4097,1.207e-05) (9217,3.261e-06)
};

\addplot coordinates{
(7,8.472e-02)    (31,3.044e-02)    (111,1.022e-02)
(351,3.303e-03)  (1023,1.039e-03)  (2815,3.196e-04)
(7423,9.658e-05) (18943,2.873e-05) (47103,8.437e-06)
};

\addplot coordinates{
(9,7.881e-02)     (49,3.243e-02)    (209,1.232e-02)
(769,4.454e-03)   (2561,1.551e-03)  (7937,5.236e-04)
(23297,1.723e-04) (65537,5.545e-05) (178177,1.751e-05)
};

\addplot coordinates{
(11,6.887e-02)    (71,3.177e-02)     (351,1.341e-02)
(1471,5.334e-03)  (5503,2.027e-03)   (18943,7.415e-04)
(61183,2.628e-04) (187903,9.063e-05) (553983,3.053e-05)
};

\addplot coordinates{
(13,5.755e-02)     (97,2.925e-02)     (545,1.351e-02)
(2561,5.842e-03)   (10625,2.397e-03)  (40193,9.414e-04)
(141569,3.564e-04) (471041,1.308e-04) 
(1496065,4.670e-05)
};
}%
\maketitle
\begin{abstract}%
\PGFPlots\ draws high--quality function plots in normal or logarithmic scaling with a user-friendly interface directly in \TeX. The user supplies axis labels, legend entries and the plot coordinates for one or more plots and \PGFPlots\ applies axis scaling, computes any logarithms and axis ticks and draws the plots, supporting line plots, scatter plots, piecewise constant plots, bar plots, area plots, mesh-- and surface plots and some more. It is based on Till Tantau's package \PGF/\Tikz.
\end{abstract}
\tableofcontents
\section{Introduction}
This package provides tools to generate plots and labeled axes easily. It draws normal plots, logplots and semi-logplots, in two and three dimensions. Axis ticks, labels, legends (in case of multiple plots) can be added with key-value options. It can cycle through a set of predefined line/marker/color specifications. In summary, its purpose is to simplify the generation of high-quality function and/or data plots, and solving the problems of
\begin{itemize}
	\item consistency of document and font type and font size,
	\item direct use of \TeX\ math mode in axis descriptions,
	\item consistency of data and figures (no third party tool necessary),
	\item inter document consistency using preamble configurations and styles.
\end{itemize}
Although not necessary, separate |.pdf| or |.eps| graphics can be generated using the |external| library developed as part of \Tikz.

\PGFPlots\ is build completely on \Tikz/\PGF. Knowledge of \Tikz\ will simplify the work with \PGFPlots, although it is not required.

\section[About PGFPlots: Preliminaries]{About {\normalfont\PGFPlots}: Preliminaries}
This section contains information about upgrades, the team, the installation (in case you need to do it manually) and troubleshooting. You may skip it completely except for the upgrade remarks.

However, note that this library requires at least \PGF\ version $2.00$. At the time of this writing, many \TeX-distributions still contain the older \PGF\ version $1.18$, so it may be necessary to install a recent \PGF\ prior to using \PGFPlots.

\subsection{Components}
\PGFPlots\ comes with two components:
\begin{enumerate}
	\item the plotting component (which you are currently reading) and
	\item the \PGFPlotstable\ component which simplifies number formatting and postprocessing of numerical tables. It comes as a separate package and has its own manual \href{file:pgfplotstable.pdf}{pgfplotstable.pdf}.
\end{enumerate}

\subsection{Upgrade remarks}
This release provides a lot of improvements which can be found in all detail in \texttt{ChangeLog} for interested readers. However, some attention is useful with respect to the following changes.

\subsubsection{New Optional Features}
\begin{enumerate}
	\item \PGFPlots\ 1.3 comes with user interface improvements. The technical distinction between ``behavior options'' and ``style options'' of older versions is no longer necessary (although still fully supported).

	\item \PGFPlots\ 1.3 has a new feature which allows to \emph{move axis labels tight to tick labels} automatically.

	Since this affects the spacing, it is not enabled be default.

	Use
\begin{codeexample}[code only]
\usepackage{pgfplots}
\pgfplotsset{compat=1.3}
\end{codeexample}
	in your preamble to benefit from the improved spacing\footnote{The |compat=1.3| setting is necessary for new features which move axis labels around like reversed axis. However, \PGFPlots\ will still work as it does in earlier versions, your documents will remain the same if you don't set it explicitly.}.
	Take a look at the next page for the precise definition of |compat|.

	\item \PGFPlots\ 1.3 now supports reversed axes. It is no longer necessary to use work--arounds with negative units.
\pgfkeys{/pdflinks/search key prefixes in/.add={/pgfplots/,}{}}

	Take a look at the |x dir=reverse| key.

	Existing work--arounds will still function properly. Use |\pgfplotsset{compat=1.3}| together with |x dir=reverse| to switch to the new version.
\end{enumerate}

\subsubsection{Old Features Which May Need Attention}
\begin{enumerate}
	\item The |scatter/classes| feature produces proper legends as of version 1.3. This may change the appearance of existing legends of plots with |scatter/classes|.

	\item Due to a small math bug in \PGF\ $2.00$, you \emph{can't} use the math expression `|-x^2|'. It is necessary to use `|0-x^2|' instead. The same holds for
		`|exp(-x^2)|'; use `|exp(0-x^2)|' instead.

		This will be fixed with \PGF\ $>2.00$.

	\item Starting with \PGFPlots\ $1.1$, |\tikzstyle| should \emph{no longer be used} to set \PGFPlots\ options.
	
	Although |\tikzstyle| is still supported for some older \PGFPlots\ options, you should replace any occurance of |\tikzstyle| with |\pgfplotsset{|\meta{style name}|/.style={|\meta{key-value-list}|}}| or the associated |/.append style| variant. See section~\ref{sec:styles} for more detail.
\end{enumerate}
I apologize for any inconvenience caused by these changes.

\begin{pgfplotskey}{compat=\mchoice{1.3,pre 1.3,default,newest} (initially default)}
	The preamble configuration 
\begin{codeexample}[code only]
\usepackage{pgfplots}
\pgfplotsset{compat=1.3}
\end{codeexample}
	allows to choose between backwards compatibility and most recent features.

	\PGFPlots\ version 1.3 comes with new features which may lead to different spacings compared to earlier versions:
	\begin{enumerate}
		\item Version 1.3 comes with the |xlabel near ticks| feature which places (in this case $x$) axis labels ``near'' the tick labels, taking the size of tick labels into account.
		
		Older versions used |xlabel absolute|, i.e.\ an absolute distance between the axis and axis labels (now matter how large or small tick labels are).

		The initial setting \emph{keeps backwards compatibility}.

		You are encouraged to use
\begin{codeexample}[code only]
\usepackage{pgfplots}
\pgfplotsset{compat=1.3}
\end{codeexample}
		in your preamble.
		
		\item Version 1.3 fixes a bug with |anchor=south west| which occurred mainly for reversed axes: the anchors where upside--down. This is fixed now.

		
		If you have written some work--around and want to keep it going, use

		|\pgfplotsset{compat/anchors=pre 1.3}|\pgfmanualpdflabel{/pgfplots/compat/anchors}{} 

		to disable the bug fix.
	\end{enumerate}

	Use |\pgfplotsset{compat=default}| to restore the factory settings.

	The setting |\pgfplotsset{compat=newest}| will always use features of the most recent version. This might result in small changes in the document's appearance.
\end{pgfplotskey}

\subsection{The Team}
\PGFPlots\ has been written mainly by Christian Feuers�nger with many improvements of Pascal Wolkotte and Nick Papior Andersen as a spare time project. We hope it is useful and provides valuable plots.

If you are interested in writing something but don't know how, consider reading the auxiliary manual \href{file:TeX-programming-notes.pdf}{TeX-programming-notes.pdf} which comes with \PGFPlots. It is far from complete, but maybe it is a good starting point (at least for more literature).

\subsection{Acknowledgements}
I thank God for all hours of enjoyed programming. I thank Pascal Wolkotte and Nick Papior Andersen for their programming efforts and contributions as part of the development team. I thank Stefan Tibus, who contributed the |plot shell| feature. I thank Tom Cashman for the contribution of the |reverse legend| feature. Special thanks go to Stefan Pinnow whose tests of \PGFPlots\ lead to numerous quality improvements. Furthermore, I thank Dr.~Schweitzer for many fruitful discussions and Dr.~Meine for his ideas and suggestions. Thanks as well to the many international contributors who provided feature requests or identified bugs or simply manual improvements!

Last but not least, I thank Till Tantau and Mark Wibrow for their excellent graphics (and more) package \PGF\ and \Tikz\ which is the base of \PGFPlots.

\include{pgfplots.install}%
\include{pgfplots.intro}%
\include{pgfplots.reference}%
\include{pgfplots.libs}%
\include{pgfplots.resources}%
\include{pgfplots.importexport}%
\include{pgfplots.basic.reference}%

\printindex

\bibliographystyle{abbrv} %gerapali} %gerabbrv} %gerunsrt.bst} %gerabbrv}% gerplain}
\nocite{pgfplotstable}
\nocite{programmingnotes}
\bibliography{pgfplots}
\end{document}
%
%%%%%%%%%%%%%%%%%%%%%%%%%%%%%%%%%%%%%%%%%%%%%%%%%%%%%%%%%%%%%%%%%%%%%%%%%%%%%
%
% Package pgfplots.sty documentation. 
%
% Copyright 2007/2008 by Christian Feuersaenger.
%
% This program is free software: you can redistribute it and/or modify
% it under the terms of the GNU General Public License as published by
% the Free Software Foundation, either version 3 of the License, or
% (at your option) any later version.
% 
% This program is distributed in the hope that it will be useful,
% but WITHOUT ANY WARRANTY; without even the implied warranty of
% MERCHANTABILITY or FITNESS FOR A PARTICULAR PURPOSE.  See the
% GNU General Public License for more details.
% 
% You should have received a copy of the GNU General Public License
% along with this program.  If not, see <http://www.gnu.org/licenses/>.
%
%
%%%%%%%%%%%%%%%%%%%%%%%%%%%%%%%%%%%%%%%%%%%%%%%%%%%%%%%%%%%%%%%%%%%%%%%%%%%%%
\input{pgfplots.preamble.tex}
%\RequirePackage[german,english,francais]{babel}

\pgfplotsmanualexternalexpensivetrue

%\usetikzlibrary{external}
%\tikzexternalize[prefix=figures/]{pgfplots}

\title{%
	Manual for Package \PGFPlots\\
	{\small Version \pgfplotsversion}\\
	{\small\href{http://sourceforge.net/projects/pgfplots}{http://sourceforge.net/projects/pgfplots}}}

%\includeonly{pgfplots.libs}


\begin{document}

\def\plotcoords{%
\addplot coordinates {
(5,8.312e-02)    (17,2.547e-02)   (49,7.407e-03)
(129,2.102e-03)  (321,5.874e-04)  (769,1.623e-04)
(1793,4.442e-05) (4097,1.207e-05) (9217,3.261e-06)
};

\addplot coordinates{
(7,8.472e-02)    (31,3.044e-02)    (111,1.022e-02)
(351,3.303e-03)  (1023,1.039e-03)  (2815,3.196e-04)
(7423,9.658e-05) (18943,2.873e-05) (47103,8.437e-06)
};

\addplot coordinates{
(9,7.881e-02)     (49,3.243e-02)    (209,1.232e-02)
(769,4.454e-03)   (2561,1.551e-03)  (7937,5.236e-04)
(23297,1.723e-04) (65537,5.545e-05) (178177,1.751e-05)
};

\addplot coordinates{
(11,6.887e-02)    (71,3.177e-02)     (351,1.341e-02)
(1471,5.334e-03)  (5503,2.027e-03)   (18943,7.415e-04)
(61183,2.628e-04) (187903,9.063e-05) (553983,3.053e-05)
};

\addplot coordinates{
(13,5.755e-02)     (97,2.925e-02)     (545,1.351e-02)
(2561,5.842e-03)   (10625,2.397e-03)  (40193,9.414e-04)
(141569,3.564e-04) (471041,1.308e-04) 
(1496065,4.670e-05)
};
}%
\maketitle
\begin{abstract}%
\PGFPlots\ draws high--quality function plots in normal or logarithmic scaling with a user-friendly interface directly in \TeX. The user supplies axis labels, legend entries and the plot coordinates for one or more plots and \PGFPlots\ applies axis scaling, computes any logarithms and axis ticks and draws the plots, supporting line plots, scatter plots, piecewise constant plots, bar plots, area plots, mesh-- and surface plots and some more. It is based on Till Tantau's package \PGF/\Tikz.
\end{abstract}
\tableofcontents
\section{Introduction}
This package provides tools to generate plots and labeled axes easily. It draws normal plots, logplots and semi-logplots, in two and three dimensions. Axis ticks, labels, legends (in case of multiple plots) can be added with key-value options. It can cycle through a set of predefined line/marker/color specifications. In summary, its purpose is to simplify the generation of high-quality function and/or data plots, and solving the problems of
\begin{itemize}
	\item consistency of document and font type and font size,
	\item direct use of \TeX\ math mode in axis descriptions,
	\item consistency of data and figures (no third party tool necessary),
	\item inter document consistency using preamble configurations and styles.
\end{itemize}
Although not necessary, separate |.pdf| or |.eps| graphics can be generated using the |external| library developed as part of \Tikz.

\PGFPlots\ is build completely on \Tikz/\PGF. Knowledge of \Tikz\ will simplify the work with \PGFPlots, although it is not required.

\section[About PGFPlots: Preliminaries]{About {\normalfont\PGFPlots}: Preliminaries}
This section contains information about upgrades, the team, the installation (in case you need to do it manually) and troubleshooting. You may skip it completely except for the upgrade remarks.

However, note that this library requires at least \PGF\ version $2.00$. At the time of this writing, many \TeX-distributions still contain the older \PGF\ version $1.18$, so it may be necessary to install a recent \PGF\ prior to using \PGFPlots.

\subsection{Components}
\PGFPlots\ comes with two components:
\begin{enumerate}
	\item the plotting component (which you are currently reading) and
	\item the \PGFPlotstable\ component which simplifies number formatting and postprocessing of numerical tables. It comes as a separate package and has its own manual \href{file:pgfplotstable.pdf}{pgfplotstable.pdf}.
\end{enumerate}

\subsection{Upgrade remarks}
This release provides a lot of improvements which can be found in all detail in \texttt{ChangeLog} for interested readers. However, some attention is useful with respect to the following changes.

\subsubsection{New Optional Features}
\begin{enumerate}
	\item \PGFPlots\ 1.3 comes with user interface improvements. The technical distinction between ``behavior options'' and ``style options'' of older versions is no longer necessary (although still fully supported).

	\item \PGFPlots\ 1.3 has a new feature which allows to \emph{move axis labels tight to tick labels} automatically.

	Since this affects the spacing, it is not enabled be default.

	Use
\begin{codeexample}[code only]
\usepackage{pgfplots}
\pgfplotsset{compat=1.3}
\end{codeexample}
	in your preamble to benefit from the improved spacing\footnote{The |compat=1.3| setting is necessary for new features which move axis labels around like reversed axis. However, \PGFPlots\ will still work as it does in earlier versions, your documents will remain the same if you don't set it explicitly.}.
	Take a look at the next page for the precise definition of |compat|.

	\item \PGFPlots\ 1.3 now supports reversed axes. It is no longer necessary to use work--arounds with negative units.
\pgfkeys{/pdflinks/search key prefixes in/.add={/pgfplots/,}{}}

	Take a look at the |x dir=reverse| key.

	Existing work--arounds will still function properly. Use |\pgfplotsset{compat=1.3}| together with |x dir=reverse| to switch to the new version.
\end{enumerate}

\subsubsection{Old Features Which May Need Attention}
\begin{enumerate}
	\item The |scatter/classes| feature produces proper legends as of version 1.3. This may change the appearance of existing legends of plots with |scatter/classes|.

	\item Due to a small math bug in \PGF\ $2.00$, you \emph{can't} use the math expression `|-x^2|'. It is necessary to use `|0-x^2|' instead. The same holds for
		`|exp(-x^2)|'; use `|exp(0-x^2)|' instead.

		This will be fixed with \PGF\ $>2.00$.

	\item Starting with \PGFPlots\ $1.1$, |\tikzstyle| should \emph{no longer be used} to set \PGFPlots\ options.
	
	Although |\tikzstyle| is still supported for some older \PGFPlots\ options, you should replace any occurance of |\tikzstyle| with |\pgfplotsset{|\meta{style name}|/.style={|\meta{key-value-list}|}}| or the associated |/.append style| variant. See section~\ref{sec:styles} for more detail.
\end{enumerate}
I apologize for any inconvenience caused by these changes.

\begin{pgfplotskey}{compat=\mchoice{1.3,pre 1.3,default,newest} (initially default)}
	The preamble configuration 
\begin{codeexample}[code only]
\usepackage{pgfplots}
\pgfplotsset{compat=1.3}
\end{codeexample}
	allows to choose between backwards compatibility and most recent features.

	\PGFPlots\ version 1.3 comes with new features which may lead to different spacings compared to earlier versions:
	\begin{enumerate}
		\item Version 1.3 comes with the |xlabel near ticks| feature which places (in this case $x$) axis labels ``near'' the tick labels, taking the size of tick labels into account.
		
		Older versions used |xlabel absolute|, i.e.\ an absolute distance between the axis and axis labels (now matter how large or small tick labels are).

		The initial setting \emph{keeps backwards compatibility}.

		You are encouraged to use
\begin{codeexample}[code only]
\usepackage{pgfplots}
\pgfplotsset{compat=1.3}
\end{codeexample}
		in your preamble.
		
		\item Version 1.3 fixes a bug with |anchor=south west| which occurred mainly for reversed axes: the anchors where upside--down. This is fixed now.

		
		If you have written some work--around and want to keep it going, use

		|\pgfplotsset{compat/anchors=pre 1.3}|\pgfmanualpdflabel{/pgfplots/compat/anchors}{} 

		to disable the bug fix.
	\end{enumerate}

	Use |\pgfplotsset{compat=default}| to restore the factory settings.

	The setting |\pgfplotsset{compat=newest}| will always use features of the most recent version. This might result in small changes in the document's appearance.
\end{pgfplotskey}

\subsection{The Team}
\PGFPlots\ has been written mainly by Christian Feuers�nger with many improvements of Pascal Wolkotte and Nick Papior Andersen as a spare time project. We hope it is useful and provides valuable plots.

If you are interested in writing something but don't know how, consider reading the auxiliary manual \href{file:TeX-programming-notes.pdf}{TeX-programming-notes.pdf} which comes with \PGFPlots. It is far from complete, but maybe it is a good starting point (at least for more literature).

\subsection{Acknowledgements}
I thank God for all hours of enjoyed programming. I thank Pascal Wolkotte and Nick Papior Andersen for their programming efforts and contributions as part of the development team. I thank Stefan Tibus, who contributed the |plot shell| feature. I thank Tom Cashman for the contribution of the |reverse legend| feature. Special thanks go to Stefan Pinnow whose tests of \PGFPlots\ lead to numerous quality improvements. Furthermore, I thank Dr.~Schweitzer for many fruitful discussions and Dr.~Meine for his ideas and suggestions. Thanks as well to the many international contributors who provided feature requests or identified bugs or simply manual improvements!

Last but not least, I thank Till Tantau and Mark Wibrow for their excellent graphics (and more) package \PGF\ and \Tikz\ which is the base of \PGFPlots.

\include{pgfplots.install}%
\include{pgfplots.intro}%
\include{pgfplots.reference}%
\include{pgfplots.libs}%
\include{pgfplots.resources}%
\include{pgfplots.importexport}%
\include{pgfplots.basic.reference}%

\printindex

\bibliographystyle{abbrv} %gerapali} %gerabbrv} %gerunsrt.bst} %gerabbrv}% gerplain}
\nocite{pgfplotstable}
\nocite{programmingnotes}
\bibliography{pgfplots}
\end{document}
%
%%%%%%%%%%%%%%%%%%%%%%%%%%%%%%%%%%%%%%%%%%%%%%%%%%%%%%%%%%%%%%%%%%%%%%%%%%%%%
%
% Package pgfplots.sty documentation. 
%
% Copyright 2007/2008 by Christian Feuersaenger.
%
% This program is free software: you can redistribute it and/or modify
% it under the terms of the GNU General Public License as published by
% the Free Software Foundation, either version 3 of the License, or
% (at your option) any later version.
% 
% This program is distributed in the hope that it will be useful,
% but WITHOUT ANY WARRANTY; without even the implied warranty of
% MERCHANTABILITY or FITNESS FOR A PARTICULAR PURPOSE.  See the
% GNU General Public License for more details.
% 
% You should have received a copy of the GNU General Public License
% along with this program.  If not, see <http://www.gnu.org/licenses/>.
%
%
%%%%%%%%%%%%%%%%%%%%%%%%%%%%%%%%%%%%%%%%%%%%%%%%%%%%%%%%%%%%%%%%%%%%%%%%%%%%%
\input{pgfplots.preamble.tex}
%\RequirePackage[german,english,francais]{babel}

\pgfplotsmanualexternalexpensivetrue

%\usetikzlibrary{external}
%\tikzexternalize[prefix=figures/]{pgfplots}

\title{%
	Manual for Package \PGFPlots\\
	{\small Version \pgfplotsversion}\\
	{\small\href{http://sourceforge.net/projects/pgfplots}{http://sourceforge.net/projects/pgfplots}}}

%\includeonly{pgfplots.libs}


\begin{document}

\def\plotcoords{%
\addplot coordinates {
(5,8.312e-02)    (17,2.547e-02)   (49,7.407e-03)
(129,2.102e-03)  (321,5.874e-04)  (769,1.623e-04)
(1793,4.442e-05) (4097,1.207e-05) (9217,3.261e-06)
};

\addplot coordinates{
(7,8.472e-02)    (31,3.044e-02)    (111,1.022e-02)
(351,3.303e-03)  (1023,1.039e-03)  (2815,3.196e-04)
(7423,9.658e-05) (18943,2.873e-05) (47103,8.437e-06)
};

\addplot coordinates{
(9,7.881e-02)     (49,3.243e-02)    (209,1.232e-02)
(769,4.454e-03)   (2561,1.551e-03)  (7937,5.236e-04)
(23297,1.723e-04) (65537,5.545e-05) (178177,1.751e-05)
};

\addplot coordinates{
(11,6.887e-02)    (71,3.177e-02)     (351,1.341e-02)
(1471,5.334e-03)  (5503,2.027e-03)   (18943,7.415e-04)
(61183,2.628e-04) (187903,9.063e-05) (553983,3.053e-05)
};

\addplot coordinates{
(13,5.755e-02)     (97,2.925e-02)     (545,1.351e-02)
(2561,5.842e-03)   (10625,2.397e-03)  (40193,9.414e-04)
(141569,3.564e-04) (471041,1.308e-04) 
(1496065,4.670e-05)
};
}%
\maketitle
\begin{abstract}%
\PGFPlots\ draws high--quality function plots in normal or logarithmic scaling with a user-friendly interface directly in \TeX. The user supplies axis labels, legend entries and the plot coordinates for one or more plots and \PGFPlots\ applies axis scaling, computes any logarithms and axis ticks and draws the plots, supporting line plots, scatter plots, piecewise constant plots, bar plots, area plots, mesh-- and surface plots and some more. It is based on Till Tantau's package \PGF/\Tikz.
\end{abstract}
\tableofcontents
\section{Introduction}
This package provides tools to generate plots and labeled axes easily. It draws normal plots, logplots and semi-logplots, in two and three dimensions. Axis ticks, labels, legends (in case of multiple plots) can be added with key-value options. It can cycle through a set of predefined line/marker/color specifications. In summary, its purpose is to simplify the generation of high-quality function and/or data plots, and solving the problems of
\begin{itemize}
	\item consistency of document and font type and font size,
	\item direct use of \TeX\ math mode in axis descriptions,
	\item consistency of data and figures (no third party tool necessary),
	\item inter document consistency using preamble configurations and styles.
\end{itemize}
Although not necessary, separate |.pdf| or |.eps| graphics can be generated using the |external| library developed as part of \Tikz.

\PGFPlots\ is build completely on \Tikz/\PGF. Knowledge of \Tikz\ will simplify the work with \PGFPlots, although it is not required.

\section[About PGFPlots: Preliminaries]{About {\normalfont\PGFPlots}: Preliminaries}
This section contains information about upgrades, the team, the installation (in case you need to do it manually) and troubleshooting. You may skip it completely except for the upgrade remarks.

However, note that this library requires at least \PGF\ version $2.00$. At the time of this writing, many \TeX-distributions still contain the older \PGF\ version $1.18$, so it may be necessary to install a recent \PGF\ prior to using \PGFPlots.

\subsection{Components}
\PGFPlots\ comes with two components:
\begin{enumerate}
	\item the plotting component (which you are currently reading) and
	\item the \PGFPlotstable\ component which simplifies number formatting and postprocessing of numerical tables. It comes as a separate package and has its own manual \href{file:pgfplotstable.pdf}{pgfplotstable.pdf}.
\end{enumerate}

\subsection{Upgrade remarks}
This release provides a lot of improvements which can be found in all detail in \texttt{ChangeLog} for interested readers. However, some attention is useful with respect to the following changes.

\subsubsection{New Optional Features}
\begin{enumerate}
	\item \PGFPlots\ 1.3 comes with user interface improvements. The technical distinction between ``behavior options'' and ``style options'' of older versions is no longer necessary (although still fully supported).

	\item \PGFPlots\ 1.3 has a new feature which allows to \emph{move axis labels tight to tick labels} automatically.

	Since this affects the spacing, it is not enabled be default.

	Use
\begin{codeexample}[code only]
\usepackage{pgfplots}
\pgfplotsset{compat=1.3}
\end{codeexample}
	in your preamble to benefit from the improved spacing\footnote{The |compat=1.3| setting is necessary for new features which move axis labels around like reversed axis. However, \PGFPlots\ will still work as it does in earlier versions, your documents will remain the same if you don't set it explicitly.}.
	Take a look at the next page for the precise definition of |compat|.

	\item \PGFPlots\ 1.3 now supports reversed axes. It is no longer necessary to use work--arounds with negative units.
\pgfkeys{/pdflinks/search key prefixes in/.add={/pgfplots/,}{}}

	Take a look at the |x dir=reverse| key.

	Existing work--arounds will still function properly. Use |\pgfplotsset{compat=1.3}| together with |x dir=reverse| to switch to the new version.
\end{enumerate}

\subsubsection{Old Features Which May Need Attention}
\begin{enumerate}
	\item The |scatter/classes| feature produces proper legends as of version 1.3. This may change the appearance of existing legends of plots with |scatter/classes|.

	\item Due to a small math bug in \PGF\ $2.00$, you \emph{can't} use the math expression `|-x^2|'. It is necessary to use `|0-x^2|' instead. The same holds for
		`|exp(-x^2)|'; use `|exp(0-x^2)|' instead.

		This will be fixed with \PGF\ $>2.00$.

	\item Starting with \PGFPlots\ $1.1$, |\tikzstyle| should \emph{no longer be used} to set \PGFPlots\ options.
	
	Although |\tikzstyle| is still supported for some older \PGFPlots\ options, you should replace any occurance of |\tikzstyle| with |\pgfplotsset{|\meta{style name}|/.style={|\meta{key-value-list}|}}| or the associated |/.append style| variant. See section~\ref{sec:styles} for more detail.
\end{enumerate}
I apologize for any inconvenience caused by these changes.

\begin{pgfplotskey}{compat=\mchoice{1.3,pre 1.3,default,newest} (initially default)}
	The preamble configuration 
\begin{codeexample}[code only]
\usepackage{pgfplots}
\pgfplotsset{compat=1.3}
\end{codeexample}
	allows to choose between backwards compatibility and most recent features.

	\PGFPlots\ version 1.3 comes with new features which may lead to different spacings compared to earlier versions:
	\begin{enumerate}
		\item Version 1.3 comes with the |xlabel near ticks| feature which places (in this case $x$) axis labels ``near'' the tick labels, taking the size of tick labels into account.
		
		Older versions used |xlabel absolute|, i.e.\ an absolute distance between the axis and axis labels (now matter how large or small tick labels are).

		The initial setting \emph{keeps backwards compatibility}.

		You are encouraged to use
\begin{codeexample}[code only]
\usepackage{pgfplots}
\pgfplotsset{compat=1.3}
\end{codeexample}
		in your preamble.
		
		\item Version 1.3 fixes a bug with |anchor=south west| which occurred mainly for reversed axes: the anchors where upside--down. This is fixed now.

		
		If you have written some work--around and want to keep it going, use

		|\pgfplotsset{compat/anchors=pre 1.3}|\pgfmanualpdflabel{/pgfplots/compat/anchors}{} 

		to disable the bug fix.
	\end{enumerate}

	Use |\pgfplotsset{compat=default}| to restore the factory settings.

	The setting |\pgfplotsset{compat=newest}| will always use features of the most recent version. This might result in small changes in the document's appearance.
\end{pgfplotskey}

\subsection{The Team}
\PGFPlots\ has been written mainly by Christian Feuers�nger with many improvements of Pascal Wolkotte and Nick Papior Andersen as a spare time project. We hope it is useful and provides valuable plots.

If you are interested in writing something but don't know how, consider reading the auxiliary manual \href{file:TeX-programming-notes.pdf}{TeX-programming-notes.pdf} which comes with \PGFPlots. It is far from complete, but maybe it is a good starting point (at least for more literature).

\subsection{Acknowledgements}
I thank God for all hours of enjoyed programming. I thank Pascal Wolkotte and Nick Papior Andersen for their programming efforts and contributions as part of the development team. I thank Stefan Tibus, who contributed the |plot shell| feature. I thank Tom Cashman for the contribution of the |reverse legend| feature. Special thanks go to Stefan Pinnow whose tests of \PGFPlots\ lead to numerous quality improvements. Furthermore, I thank Dr.~Schweitzer for many fruitful discussions and Dr.~Meine for his ideas and suggestions. Thanks as well to the many international contributors who provided feature requests or identified bugs or simply manual improvements!

Last but not least, I thank Till Tantau and Mark Wibrow for their excellent graphics (and more) package \PGF\ and \Tikz\ which is the base of \PGFPlots.

\include{pgfplots.install}%
\include{pgfplots.intro}%
\include{pgfplots.reference}%
\include{pgfplots.libs}%
\include{pgfplots.resources}%
\include{pgfplots.importexport}%
\include{pgfplots.basic.reference}%

\printindex

\bibliographystyle{abbrv} %gerapali} %gerabbrv} %gerunsrt.bst} %gerabbrv}% gerplain}
\nocite{pgfplotstable}
\nocite{programmingnotes}
\bibliography{pgfplots}
\end{document}
%
%%%%%%%%%%%%%%%%%%%%%%%%%%%%%%%%%%%%%%%%%%%%%%%%%%%%%%%%%%%%%%%%%%%%%%%%%%%%%
%
% Package pgfplots.sty documentation. 
%
% Copyright 2007/2008 by Christian Feuersaenger.
%
% This program is free software: you can redistribute it and/or modify
% it under the terms of the GNU General Public License as published by
% the Free Software Foundation, either version 3 of the License, or
% (at your option) any later version.
% 
% This program is distributed in the hope that it will be useful,
% but WITHOUT ANY WARRANTY; without even the implied warranty of
% MERCHANTABILITY or FITNESS FOR A PARTICULAR PURPOSE.  See the
% GNU General Public License for more details.
% 
% You should have received a copy of the GNU General Public License
% along with this program.  If not, see <http://www.gnu.org/licenses/>.
%
%
%%%%%%%%%%%%%%%%%%%%%%%%%%%%%%%%%%%%%%%%%%%%%%%%%%%%%%%%%%%%%%%%%%%%%%%%%%%%%
\input{pgfplots.preamble.tex}
%\RequirePackage[german,english,francais]{babel}

\pgfplotsmanualexternalexpensivetrue

%\usetikzlibrary{external}
%\tikzexternalize[prefix=figures/]{pgfplots}

\title{%
	Manual for Package \PGFPlots\\
	{\small Version \pgfplotsversion}\\
	{\small\href{http://sourceforge.net/projects/pgfplots}{http://sourceforge.net/projects/pgfplots}}}

%\includeonly{pgfplots.libs}


\begin{document}

\def\plotcoords{%
\addplot coordinates {
(5,8.312e-02)    (17,2.547e-02)   (49,7.407e-03)
(129,2.102e-03)  (321,5.874e-04)  (769,1.623e-04)
(1793,4.442e-05) (4097,1.207e-05) (9217,3.261e-06)
};

\addplot coordinates{
(7,8.472e-02)    (31,3.044e-02)    (111,1.022e-02)
(351,3.303e-03)  (1023,1.039e-03)  (2815,3.196e-04)
(7423,9.658e-05) (18943,2.873e-05) (47103,8.437e-06)
};

\addplot coordinates{
(9,7.881e-02)     (49,3.243e-02)    (209,1.232e-02)
(769,4.454e-03)   (2561,1.551e-03)  (7937,5.236e-04)
(23297,1.723e-04) (65537,5.545e-05) (178177,1.751e-05)
};

\addplot coordinates{
(11,6.887e-02)    (71,3.177e-02)     (351,1.341e-02)
(1471,5.334e-03)  (5503,2.027e-03)   (18943,7.415e-04)
(61183,2.628e-04) (187903,9.063e-05) (553983,3.053e-05)
};

\addplot coordinates{
(13,5.755e-02)     (97,2.925e-02)     (545,1.351e-02)
(2561,5.842e-03)   (10625,2.397e-03)  (40193,9.414e-04)
(141569,3.564e-04) (471041,1.308e-04) 
(1496065,4.670e-05)
};
}%
\maketitle
\begin{abstract}%
\PGFPlots\ draws high--quality function plots in normal or logarithmic scaling with a user-friendly interface directly in \TeX. The user supplies axis labels, legend entries and the plot coordinates for one or more plots and \PGFPlots\ applies axis scaling, computes any logarithms and axis ticks and draws the plots, supporting line plots, scatter plots, piecewise constant plots, bar plots, area plots, mesh-- and surface plots and some more. It is based on Till Tantau's package \PGF/\Tikz.
\end{abstract}
\tableofcontents
\section{Introduction}
This package provides tools to generate plots and labeled axes easily. It draws normal plots, logplots and semi-logplots, in two and three dimensions. Axis ticks, labels, legends (in case of multiple plots) can be added with key-value options. It can cycle through a set of predefined line/marker/color specifications. In summary, its purpose is to simplify the generation of high-quality function and/or data plots, and solving the problems of
\begin{itemize}
	\item consistency of document and font type and font size,
	\item direct use of \TeX\ math mode in axis descriptions,
	\item consistency of data and figures (no third party tool necessary),
	\item inter document consistency using preamble configurations and styles.
\end{itemize}
Although not necessary, separate |.pdf| or |.eps| graphics can be generated using the |external| library developed as part of \Tikz.

\PGFPlots\ is build completely on \Tikz/\PGF. Knowledge of \Tikz\ will simplify the work with \PGFPlots, although it is not required.

\section[About PGFPlots: Preliminaries]{About {\normalfont\PGFPlots}: Preliminaries}
This section contains information about upgrades, the team, the installation (in case you need to do it manually) and troubleshooting. You may skip it completely except for the upgrade remarks.

However, note that this library requires at least \PGF\ version $2.00$. At the time of this writing, many \TeX-distributions still contain the older \PGF\ version $1.18$, so it may be necessary to install a recent \PGF\ prior to using \PGFPlots.

\subsection{Components}
\PGFPlots\ comes with two components:
\begin{enumerate}
	\item the plotting component (which you are currently reading) and
	\item the \PGFPlotstable\ component which simplifies number formatting and postprocessing of numerical tables. It comes as a separate package and has its own manual \href{file:pgfplotstable.pdf}{pgfplotstable.pdf}.
\end{enumerate}

\subsection{Upgrade remarks}
This release provides a lot of improvements which can be found in all detail in \texttt{ChangeLog} for interested readers. However, some attention is useful with respect to the following changes.

\subsubsection{New Optional Features}
\begin{enumerate}
	\item \PGFPlots\ 1.3 comes with user interface improvements. The technical distinction between ``behavior options'' and ``style options'' of older versions is no longer necessary (although still fully supported).

	\item \PGFPlots\ 1.3 has a new feature which allows to \emph{move axis labels tight to tick labels} automatically.

	Since this affects the spacing, it is not enabled be default.

	Use
\begin{codeexample}[code only]
\usepackage{pgfplots}
\pgfplotsset{compat=1.3}
\end{codeexample}
	in your preamble to benefit from the improved spacing\footnote{The |compat=1.3| setting is necessary for new features which move axis labels around like reversed axis. However, \PGFPlots\ will still work as it does in earlier versions, your documents will remain the same if you don't set it explicitly.}.
	Take a look at the next page for the precise definition of |compat|.

	\item \PGFPlots\ 1.3 now supports reversed axes. It is no longer necessary to use work--arounds with negative units.
\pgfkeys{/pdflinks/search key prefixes in/.add={/pgfplots/,}{}}

	Take a look at the |x dir=reverse| key.

	Existing work--arounds will still function properly. Use |\pgfplotsset{compat=1.3}| together with |x dir=reverse| to switch to the new version.
\end{enumerate}

\subsubsection{Old Features Which May Need Attention}
\begin{enumerate}
	\item The |scatter/classes| feature produces proper legends as of version 1.3. This may change the appearance of existing legends of plots with |scatter/classes|.

	\item Due to a small math bug in \PGF\ $2.00$, you \emph{can't} use the math expression `|-x^2|'. It is necessary to use `|0-x^2|' instead. The same holds for
		`|exp(-x^2)|'; use `|exp(0-x^2)|' instead.

		This will be fixed with \PGF\ $>2.00$.

	\item Starting with \PGFPlots\ $1.1$, |\tikzstyle| should \emph{no longer be used} to set \PGFPlots\ options.
	
	Although |\tikzstyle| is still supported for some older \PGFPlots\ options, you should replace any occurance of |\tikzstyle| with |\pgfplotsset{|\meta{style name}|/.style={|\meta{key-value-list}|}}| or the associated |/.append style| variant. See section~\ref{sec:styles} for more detail.
\end{enumerate}
I apologize for any inconvenience caused by these changes.

\begin{pgfplotskey}{compat=\mchoice{1.3,pre 1.3,default,newest} (initially default)}
	The preamble configuration 
\begin{codeexample}[code only]
\usepackage{pgfplots}
\pgfplotsset{compat=1.3}
\end{codeexample}
	allows to choose between backwards compatibility and most recent features.

	\PGFPlots\ version 1.3 comes with new features which may lead to different spacings compared to earlier versions:
	\begin{enumerate}
		\item Version 1.3 comes with the |xlabel near ticks| feature which places (in this case $x$) axis labels ``near'' the tick labels, taking the size of tick labels into account.
		
		Older versions used |xlabel absolute|, i.e.\ an absolute distance between the axis and axis labels (now matter how large or small tick labels are).

		The initial setting \emph{keeps backwards compatibility}.

		You are encouraged to use
\begin{codeexample}[code only]
\usepackage{pgfplots}
\pgfplotsset{compat=1.3}
\end{codeexample}
		in your preamble.
		
		\item Version 1.3 fixes a bug with |anchor=south west| which occurred mainly for reversed axes: the anchors where upside--down. This is fixed now.

		
		If you have written some work--around and want to keep it going, use

		|\pgfplotsset{compat/anchors=pre 1.3}|\pgfmanualpdflabel{/pgfplots/compat/anchors}{} 

		to disable the bug fix.
	\end{enumerate}

	Use |\pgfplotsset{compat=default}| to restore the factory settings.

	The setting |\pgfplotsset{compat=newest}| will always use features of the most recent version. This might result in small changes in the document's appearance.
\end{pgfplotskey}

\subsection{The Team}
\PGFPlots\ has been written mainly by Christian Feuers�nger with many improvements of Pascal Wolkotte and Nick Papior Andersen as a spare time project. We hope it is useful and provides valuable plots.

If you are interested in writing something but don't know how, consider reading the auxiliary manual \href{file:TeX-programming-notes.pdf}{TeX-programming-notes.pdf} which comes with \PGFPlots. It is far from complete, but maybe it is a good starting point (at least for more literature).

\subsection{Acknowledgements}
I thank God for all hours of enjoyed programming. I thank Pascal Wolkotte and Nick Papior Andersen for their programming efforts and contributions as part of the development team. I thank Stefan Tibus, who contributed the |plot shell| feature. I thank Tom Cashman for the contribution of the |reverse legend| feature. Special thanks go to Stefan Pinnow whose tests of \PGFPlots\ lead to numerous quality improvements. Furthermore, I thank Dr.~Schweitzer for many fruitful discussions and Dr.~Meine for his ideas and suggestions. Thanks as well to the many international contributors who provided feature requests or identified bugs or simply manual improvements!

Last but not least, I thank Till Tantau and Mark Wibrow for their excellent graphics (and more) package \PGF\ and \Tikz\ which is the base of \PGFPlots.

\include{pgfplots.install}%
\include{pgfplots.intro}%
\include{pgfplots.reference}%
\include{pgfplots.libs}%
\include{pgfplots.resources}%
\include{pgfplots.importexport}%
\include{pgfplots.basic.reference}%

\printindex

\bibliographystyle{abbrv} %gerapali} %gerabbrv} %gerunsrt.bst} %gerabbrv}% gerplain}
\nocite{pgfplotstable}
\nocite{programmingnotes}
\bibliography{pgfplots}
\end{document}
%
%%%%%%%%%%%%%%%%%%%%%%%%%%%%%%%%%%%%%%%%%%%%%%%%%%%%%%%%%%%%%%%%%%%%%%%%%%%%%
%
% Package pgfplots.sty documentation. 
%
% Copyright 2007/2008 by Christian Feuersaenger.
%
% This program is free software: you can redistribute it and/or modify
% it under the terms of the GNU General Public License as published by
% the Free Software Foundation, either version 3 of the License, or
% (at your option) any later version.
% 
% This program is distributed in the hope that it will be useful,
% but WITHOUT ANY WARRANTY; without even the implied warranty of
% MERCHANTABILITY or FITNESS FOR A PARTICULAR PURPOSE.  See the
% GNU General Public License for more details.
% 
% You should have received a copy of the GNU General Public License
% along with this program.  If not, see <http://www.gnu.org/licenses/>.
%
%
%%%%%%%%%%%%%%%%%%%%%%%%%%%%%%%%%%%%%%%%%%%%%%%%%%%%%%%%%%%%%%%%%%%%%%%%%%%%%
\input{pgfplots.preamble.tex}
%\RequirePackage[german,english,francais]{babel}

\pgfplotsmanualexternalexpensivetrue

%\usetikzlibrary{external}
%\tikzexternalize[prefix=figures/]{pgfplots}

\title{%
	Manual for Package \PGFPlots\\
	{\small Version \pgfplotsversion}\\
	{\small\href{http://sourceforge.net/projects/pgfplots}{http://sourceforge.net/projects/pgfplots}}}

%\includeonly{pgfplots.libs}


\begin{document}

\def\plotcoords{%
\addplot coordinates {
(5,8.312e-02)    (17,2.547e-02)   (49,7.407e-03)
(129,2.102e-03)  (321,5.874e-04)  (769,1.623e-04)
(1793,4.442e-05) (4097,1.207e-05) (9217,3.261e-06)
};

\addplot coordinates{
(7,8.472e-02)    (31,3.044e-02)    (111,1.022e-02)
(351,3.303e-03)  (1023,1.039e-03)  (2815,3.196e-04)
(7423,9.658e-05) (18943,2.873e-05) (47103,8.437e-06)
};

\addplot coordinates{
(9,7.881e-02)     (49,3.243e-02)    (209,1.232e-02)
(769,4.454e-03)   (2561,1.551e-03)  (7937,5.236e-04)
(23297,1.723e-04) (65537,5.545e-05) (178177,1.751e-05)
};

\addplot coordinates{
(11,6.887e-02)    (71,3.177e-02)     (351,1.341e-02)
(1471,5.334e-03)  (5503,2.027e-03)   (18943,7.415e-04)
(61183,2.628e-04) (187903,9.063e-05) (553983,3.053e-05)
};

\addplot coordinates{
(13,5.755e-02)     (97,2.925e-02)     (545,1.351e-02)
(2561,5.842e-03)   (10625,2.397e-03)  (40193,9.414e-04)
(141569,3.564e-04) (471041,1.308e-04) 
(1496065,4.670e-05)
};
}%
\maketitle
\begin{abstract}%
\PGFPlots\ draws high--quality function plots in normal or logarithmic scaling with a user-friendly interface directly in \TeX. The user supplies axis labels, legend entries and the plot coordinates for one or more plots and \PGFPlots\ applies axis scaling, computes any logarithms and axis ticks and draws the plots, supporting line plots, scatter plots, piecewise constant plots, bar plots, area plots, mesh-- and surface plots and some more. It is based on Till Tantau's package \PGF/\Tikz.
\end{abstract}
\tableofcontents
\section{Introduction}
This package provides tools to generate plots and labeled axes easily. It draws normal plots, logplots and semi-logplots, in two and three dimensions. Axis ticks, labels, legends (in case of multiple plots) can be added with key-value options. It can cycle through a set of predefined line/marker/color specifications. In summary, its purpose is to simplify the generation of high-quality function and/or data plots, and solving the problems of
\begin{itemize}
	\item consistency of document and font type and font size,
	\item direct use of \TeX\ math mode in axis descriptions,
	\item consistency of data and figures (no third party tool necessary),
	\item inter document consistency using preamble configurations and styles.
\end{itemize}
Although not necessary, separate |.pdf| or |.eps| graphics can be generated using the |external| library developed as part of \Tikz.

\PGFPlots\ is build completely on \Tikz/\PGF. Knowledge of \Tikz\ will simplify the work with \PGFPlots, although it is not required.

\section[About PGFPlots: Preliminaries]{About {\normalfont\PGFPlots}: Preliminaries}
This section contains information about upgrades, the team, the installation (in case you need to do it manually) and troubleshooting. You may skip it completely except for the upgrade remarks.

However, note that this library requires at least \PGF\ version $2.00$. At the time of this writing, many \TeX-distributions still contain the older \PGF\ version $1.18$, so it may be necessary to install a recent \PGF\ prior to using \PGFPlots.

\subsection{Components}
\PGFPlots\ comes with two components:
\begin{enumerate}
	\item the plotting component (which you are currently reading) and
	\item the \PGFPlotstable\ component which simplifies number formatting and postprocessing of numerical tables. It comes as a separate package and has its own manual \href{file:pgfplotstable.pdf}{pgfplotstable.pdf}.
\end{enumerate}

\subsection{Upgrade remarks}
This release provides a lot of improvements which can be found in all detail in \texttt{ChangeLog} for interested readers. However, some attention is useful with respect to the following changes.

\subsubsection{New Optional Features}
\begin{enumerate}
	\item \PGFPlots\ 1.3 comes with user interface improvements. The technical distinction between ``behavior options'' and ``style options'' of older versions is no longer necessary (although still fully supported).

	\item \PGFPlots\ 1.3 has a new feature which allows to \emph{move axis labels tight to tick labels} automatically.

	Since this affects the spacing, it is not enabled be default.

	Use
\begin{codeexample}[code only]
\usepackage{pgfplots}
\pgfplotsset{compat=1.3}
\end{codeexample}
	in your preamble to benefit from the improved spacing\footnote{The |compat=1.3| setting is necessary for new features which move axis labels around like reversed axis. However, \PGFPlots\ will still work as it does in earlier versions, your documents will remain the same if you don't set it explicitly.}.
	Take a look at the next page for the precise definition of |compat|.

	\item \PGFPlots\ 1.3 now supports reversed axes. It is no longer necessary to use work--arounds with negative units.
\pgfkeys{/pdflinks/search key prefixes in/.add={/pgfplots/,}{}}

	Take a look at the |x dir=reverse| key.

	Existing work--arounds will still function properly. Use |\pgfplotsset{compat=1.3}| together with |x dir=reverse| to switch to the new version.
\end{enumerate}

\subsubsection{Old Features Which May Need Attention}
\begin{enumerate}
	\item The |scatter/classes| feature produces proper legends as of version 1.3. This may change the appearance of existing legends of plots with |scatter/classes|.

	\item Due to a small math bug in \PGF\ $2.00$, you \emph{can't} use the math expression `|-x^2|'. It is necessary to use `|0-x^2|' instead. The same holds for
		`|exp(-x^2)|'; use `|exp(0-x^2)|' instead.

		This will be fixed with \PGF\ $>2.00$.

	\item Starting with \PGFPlots\ $1.1$, |\tikzstyle| should \emph{no longer be used} to set \PGFPlots\ options.
	
	Although |\tikzstyle| is still supported for some older \PGFPlots\ options, you should replace any occurance of |\tikzstyle| with |\pgfplotsset{|\meta{style name}|/.style={|\meta{key-value-list}|}}| or the associated |/.append style| variant. See section~\ref{sec:styles} for more detail.
\end{enumerate}
I apologize for any inconvenience caused by these changes.

\begin{pgfplotskey}{compat=\mchoice{1.3,pre 1.3,default,newest} (initially default)}
	The preamble configuration 
\begin{codeexample}[code only]
\usepackage{pgfplots}
\pgfplotsset{compat=1.3}
\end{codeexample}
	allows to choose between backwards compatibility and most recent features.

	\PGFPlots\ version 1.3 comes with new features which may lead to different spacings compared to earlier versions:
	\begin{enumerate}
		\item Version 1.3 comes with the |xlabel near ticks| feature which places (in this case $x$) axis labels ``near'' the tick labels, taking the size of tick labels into account.
		
		Older versions used |xlabel absolute|, i.e.\ an absolute distance between the axis and axis labels (now matter how large or small tick labels are).

		The initial setting \emph{keeps backwards compatibility}.

		You are encouraged to use
\begin{codeexample}[code only]
\usepackage{pgfplots}
\pgfplotsset{compat=1.3}
\end{codeexample}
		in your preamble.
		
		\item Version 1.3 fixes a bug with |anchor=south west| which occurred mainly for reversed axes: the anchors where upside--down. This is fixed now.

		
		If you have written some work--around and want to keep it going, use

		|\pgfplotsset{compat/anchors=pre 1.3}|\pgfmanualpdflabel{/pgfplots/compat/anchors}{} 

		to disable the bug fix.
	\end{enumerate}

	Use |\pgfplotsset{compat=default}| to restore the factory settings.

	The setting |\pgfplotsset{compat=newest}| will always use features of the most recent version. This might result in small changes in the document's appearance.
\end{pgfplotskey}

\subsection{The Team}
\PGFPlots\ has been written mainly by Christian Feuers�nger with many improvements of Pascal Wolkotte and Nick Papior Andersen as a spare time project. We hope it is useful and provides valuable plots.

If you are interested in writing something but don't know how, consider reading the auxiliary manual \href{file:TeX-programming-notes.pdf}{TeX-programming-notes.pdf} which comes with \PGFPlots. It is far from complete, but maybe it is a good starting point (at least for more literature).

\subsection{Acknowledgements}
I thank God for all hours of enjoyed programming. I thank Pascal Wolkotte and Nick Papior Andersen for their programming efforts and contributions as part of the development team. I thank Stefan Tibus, who contributed the |plot shell| feature. I thank Tom Cashman for the contribution of the |reverse legend| feature. Special thanks go to Stefan Pinnow whose tests of \PGFPlots\ lead to numerous quality improvements. Furthermore, I thank Dr.~Schweitzer for many fruitful discussions and Dr.~Meine for his ideas and suggestions. Thanks as well to the many international contributors who provided feature requests or identified bugs or simply manual improvements!

Last but not least, I thank Till Tantau and Mark Wibrow for their excellent graphics (and more) package \PGF\ and \Tikz\ which is the base of \PGFPlots.

\include{pgfplots.install}%
\include{pgfplots.intro}%
\include{pgfplots.reference}%
\include{pgfplots.libs}%
\include{pgfplots.resources}%
\include{pgfplots.importexport}%
\include{pgfplots.basic.reference}%

\printindex

\bibliographystyle{abbrv} %gerapali} %gerabbrv} %gerunsrt.bst} %gerabbrv}% gerplain}
\nocite{pgfplotstable}
\nocite{programmingnotes}
\bibliography{pgfplots}
\end{document}
%
%%%%%%%%%%%%%%%%%%%%%%%%%%%%%%%%%%%%%%%%%%%%%%%%%%%%%%%%%%%%%%%%%%%%%%%%%%%%%
%
% Package pgfplots.sty documentation. 
%
% Copyright 2007/2008 by Christian Feuersaenger.
%
% This program is free software: you can redistribute it and/or modify
% it under the terms of the GNU General Public License as published by
% the Free Software Foundation, either version 3 of the License, or
% (at your option) any later version.
% 
% This program is distributed in the hope that it will be useful,
% but WITHOUT ANY WARRANTY; without even the implied warranty of
% MERCHANTABILITY or FITNESS FOR A PARTICULAR PURPOSE.  See the
% GNU General Public License for more details.
% 
% You should have received a copy of the GNU General Public License
% along with this program.  If not, see <http://www.gnu.org/licenses/>.
%
%
%%%%%%%%%%%%%%%%%%%%%%%%%%%%%%%%%%%%%%%%%%%%%%%%%%%%%%%%%%%%%%%%%%%%%%%%%%%%%
\input{pgfplots.preamble.tex}
%\RequirePackage[german,english,francais]{babel}

\pgfplotsmanualexternalexpensivetrue

%\usetikzlibrary{external}
%\tikzexternalize[prefix=figures/]{pgfplots}

\title{%
	Manual for Package \PGFPlots\\
	{\small Version \pgfplotsversion}\\
	{\small\href{http://sourceforge.net/projects/pgfplots}{http://sourceforge.net/projects/pgfplots}}}

%\includeonly{pgfplots.libs}


\begin{document}

\def\plotcoords{%
\addplot coordinates {
(5,8.312e-02)    (17,2.547e-02)   (49,7.407e-03)
(129,2.102e-03)  (321,5.874e-04)  (769,1.623e-04)
(1793,4.442e-05) (4097,1.207e-05) (9217,3.261e-06)
};

\addplot coordinates{
(7,8.472e-02)    (31,3.044e-02)    (111,1.022e-02)
(351,3.303e-03)  (1023,1.039e-03)  (2815,3.196e-04)
(7423,9.658e-05) (18943,2.873e-05) (47103,8.437e-06)
};

\addplot coordinates{
(9,7.881e-02)     (49,3.243e-02)    (209,1.232e-02)
(769,4.454e-03)   (2561,1.551e-03)  (7937,5.236e-04)
(23297,1.723e-04) (65537,5.545e-05) (178177,1.751e-05)
};

\addplot coordinates{
(11,6.887e-02)    (71,3.177e-02)     (351,1.341e-02)
(1471,5.334e-03)  (5503,2.027e-03)   (18943,7.415e-04)
(61183,2.628e-04) (187903,9.063e-05) (553983,3.053e-05)
};

\addplot coordinates{
(13,5.755e-02)     (97,2.925e-02)     (545,1.351e-02)
(2561,5.842e-03)   (10625,2.397e-03)  (40193,9.414e-04)
(141569,3.564e-04) (471041,1.308e-04) 
(1496065,4.670e-05)
};
}%
\maketitle
\begin{abstract}%
\PGFPlots\ draws high--quality function plots in normal or logarithmic scaling with a user-friendly interface directly in \TeX. The user supplies axis labels, legend entries and the plot coordinates for one or more plots and \PGFPlots\ applies axis scaling, computes any logarithms and axis ticks and draws the plots, supporting line plots, scatter plots, piecewise constant plots, bar plots, area plots, mesh-- and surface plots and some more. It is based on Till Tantau's package \PGF/\Tikz.
\end{abstract}
\tableofcontents
\section{Introduction}
This package provides tools to generate plots and labeled axes easily. It draws normal plots, logplots and semi-logplots, in two and three dimensions. Axis ticks, labels, legends (in case of multiple plots) can be added with key-value options. It can cycle through a set of predefined line/marker/color specifications. In summary, its purpose is to simplify the generation of high-quality function and/or data plots, and solving the problems of
\begin{itemize}
	\item consistency of document and font type and font size,
	\item direct use of \TeX\ math mode in axis descriptions,
	\item consistency of data and figures (no third party tool necessary),
	\item inter document consistency using preamble configurations and styles.
\end{itemize}
Although not necessary, separate |.pdf| or |.eps| graphics can be generated using the |external| library developed as part of \Tikz.

\PGFPlots\ is build completely on \Tikz/\PGF. Knowledge of \Tikz\ will simplify the work with \PGFPlots, although it is not required.

\section[About PGFPlots: Preliminaries]{About {\normalfont\PGFPlots}: Preliminaries}
This section contains information about upgrades, the team, the installation (in case you need to do it manually) and troubleshooting. You may skip it completely except for the upgrade remarks.

However, note that this library requires at least \PGF\ version $2.00$. At the time of this writing, many \TeX-distributions still contain the older \PGF\ version $1.18$, so it may be necessary to install a recent \PGF\ prior to using \PGFPlots.

\subsection{Components}
\PGFPlots\ comes with two components:
\begin{enumerate}
	\item the plotting component (which you are currently reading) and
	\item the \PGFPlotstable\ component which simplifies number formatting and postprocessing of numerical tables. It comes as a separate package and has its own manual \href{file:pgfplotstable.pdf}{pgfplotstable.pdf}.
\end{enumerate}

\subsection{Upgrade remarks}
This release provides a lot of improvements which can be found in all detail in \texttt{ChangeLog} for interested readers. However, some attention is useful with respect to the following changes.

\subsubsection{New Optional Features}
\begin{enumerate}
	\item \PGFPlots\ 1.3 comes with user interface improvements. The technical distinction between ``behavior options'' and ``style options'' of older versions is no longer necessary (although still fully supported).

	\item \PGFPlots\ 1.3 has a new feature which allows to \emph{move axis labels tight to tick labels} automatically.

	Since this affects the spacing, it is not enabled be default.

	Use
\begin{codeexample}[code only]
\usepackage{pgfplots}
\pgfplotsset{compat=1.3}
\end{codeexample}
	in your preamble to benefit from the improved spacing\footnote{The |compat=1.3| setting is necessary for new features which move axis labels around like reversed axis. However, \PGFPlots\ will still work as it does in earlier versions, your documents will remain the same if you don't set it explicitly.}.
	Take a look at the next page for the precise definition of |compat|.

	\item \PGFPlots\ 1.3 now supports reversed axes. It is no longer necessary to use work--arounds with negative units.
\pgfkeys{/pdflinks/search key prefixes in/.add={/pgfplots/,}{}}

	Take a look at the |x dir=reverse| key.

	Existing work--arounds will still function properly. Use |\pgfplotsset{compat=1.3}| together with |x dir=reverse| to switch to the new version.
\end{enumerate}

\subsubsection{Old Features Which May Need Attention}
\begin{enumerate}
	\item The |scatter/classes| feature produces proper legends as of version 1.3. This may change the appearance of existing legends of plots with |scatter/classes|.

	\item Due to a small math bug in \PGF\ $2.00$, you \emph{can't} use the math expression `|-x^2|'. It is necessary to use `|0-x^2|' instead. The same holds for
		`|exp(-x^2)|'; use `|exp(0-x^2)|' instead.

		This will be fixed with \PGF\ $>2.00$.

	\item Starting with \PGFPlots\ $1.1$, |\tikzstyle| should \emph{no longer be used} to set \PGFPlots\ options.
	
	Although |\tikzstyle| is still supported for some older \PGFPlots\ options, you should replace any occurance of |\tikzstyle| with |\pgfplotsset{|\meta{style name}|/.style={|\meta{key-value-list}|}}| or the associated |/.append style| variant. See section~\ref{sec:styles} for more detail.
\end{enumerate}
I apologize for any inconvenience caused by these changes.

\begin{pgfplotskey}{compat=\mchoice{1.3,pre 1.3,default,newest} (initially default)}
	The preamble configuration 
\begin{codeexample}[code only]
\usepackage{pgfplots}
\pgfplotsset{compat=1.3}
\end{codeexample}
	allows to choose between backwards compatibility and most recent features.

	\PGFPlots\ version 1.3 comes with new features which may lead to different spacings compared to earlier versions:
	\begin{enumerate}
		\item Version 1.3 comes with the |xlabel near ticks| feature which places (in this case $x$) axis labels ``near'' the tick labels, taking the size of tick labels into account.
		
		Older versions used |xlabel absolute|, i.e.\ an absolute distance between the axis and axis labels (now matter how large or small tick labels are).

		The initial setting \emph{keeps backwards compatibility}.

		You are encouraged to use
\begin{codeexample}[code only]
\usepackage{pgfplots}
\pgfplotsset{compat=1.3}
\end{codeexample}
		in your preamble.
		
		\item Version 1.3 fixes a bug with |anchor=south west| which occurred mainly for reversed axes: the anchors where upside--down. This is fixed now.

		
		If you have written some work--around and want to keep it going, use

		|\pgfplotsset{compat/anchors=pre 1.3}|\pgfmanualpdflabel{/pgfplots/compat/anchors}{} 

		to disable the bug fix.
	\end{enumerate}

	Use |\pgfplotsset{compat=default}| to restore the factory settings.

	The setting |\pgfplotsset{compat=newest}| will always use features of the most recent version. This might result in small changes in the document's appearance.
\end{pgfplotskey}

\subsection{The Team}
\PGFPlots\ has been written mainly by Christian Feuers�nger with many improvements of Pascal Wolkotte and Nick Papior Andersen as a spare time project. We hope it is useful and provides valuable plots.

If you are interested in writing something but don't know how, consider reading the auxiliary manual \href{file:TeX-programming-notes.pdf}{TeX-programming-notes.pdf} which comes with \PGFPlots. It is far from complete, but maybe it is a good starting point (at least for more literature).

\subsection{Acknowledgements}
I thank God for all hours of enjoyed programming. I thank Pascal Wolkotte and Nick Papior Andersen for their programming efforts and contributions as part of the development team. I thank Stefan Tibus, who contributed the |plot shell| feature. I thank Tom Cashman for the contribution of the |reverse legend| feature. Special thanks go to Stefan Pinnow whose tests of \PGFPlots\ lead to numerous quality improvements. Furthermore, I thank Dr.~Schweitzer for many fruitful discussions and Dr.~Meine for his ideas and suggestions. Thanks as well to the many international contributors who provided feature requests or identified bugs or simply manual improvements!

Last but not least, I thank Till Tantau and Mark Wibrow for their excellent graphics (and more) package \PGF\ and \Tikz\ which is the base of \PGFPlots.

\include{pgfplots.install}%
\include{pgfplots.intro}%
\include{pgfplots.reference}%
\include{pgfplots.libs}%
\include{pgfplots.resources}%
\include{pgfplots.importexport}%
\include{pgfplots.basic.reference}%

\printindex

\bibliographystyle{abbrv} %gerapali} %gerabbrv} %gerunsrt.bst} %gerabbrv}% gerplain}
\nocite{pgfplotstable}
\nocite{programmingnotes}
\bibliography{pgfplots}
\end{document}
%
\include{pgfplots.basic.reference}%

\printindex

\bibliographystyle{abbrv} %gerapali} %gerabbrv} %gerunsrt.bst} %gerabbrv}% gerplain}
\nocite{pgfplotstable}
\nocite{programmingnotes}
\bibliography{pgfplots}
\end{document}
%
%%%%%%%%%%%%%%%%%%%%%%%%%%%%%%%%%%%%%%%%%%%%%%%%%%%%%%%%%%%%%%%%%%%%%%%%%%%%%
%
% Package pgfplots.sty documentation. 
%
% Copyright 2007/2008 by Christian Feuersaenger.
%
% This program is free software: you can redistribute it and/or modify
% it under the terms of the GNU General Public License as published by
% the Free Software Foundation, either version 3 of the License, or
% (at your option) any later version.
% 
% This program is distributed in the hope that it will be useful,
% but WITHOUT ANY WARRANTY; without even the implied warranty of
% MERCHANTABILITY or FITNESS FOR A PARTICULAR PURPOSE.  See the
% GNU General Public License for more details.
% 
% You should have received a copy of the GNU General Public License
% along with this program.  If not, see <http://www.gnu.org/licenses/>.
%
%
%%%%%%%%%%%%%%%%%%%%%%%%%%%%%%%%%%%%%%%%%%%%%%%%%%%%%%%%%%%%%%%%%%%%%%%%%%%%%
%%%%%%%%%%%%%%%%%%%%%%%%%%%%%%%%%%%%%%%%%%%%%%%%%%%%%%%%%%%%%%%%%%%%%%%%%%%%%
%
% Package pgfplots.sty documentation. 
%
% Copyright 2007/2008 by Christian Feuersaenger.
%
% This program is free software: you can redistribute it and/or modify
% it under the terms of the GNU General Public License as published by
% the Free Software Foundation, either version 3 of the License, or
% (at your option) any later version.
% 
% This program is distributed in the hope that it will be useful,
% but WITHOUT ANY WARRANTY; without even the implied warranty of
% MERCHANTABILITY or FITNESS FOR A PARTICULAR PURPOSE.  See the
% GNU General Public License for more details.
% 
% You should have received a copy of the GNU General Public License
% along with this program.  If not, see <http://www.gnu.org/licenses/>.
%
%
%%%%%%%%%%%%%%%%%%%%%%%%%%%%%%%%%%%%%%%%%%%%%%%%%%%%%%%%%%%%%%%%%%%%%%%%%%%%%
\documentclass[a4paper]{ltxdoc}

\usepackage{makeidx}

% DON't let hyperref overload the format of index and glossary. 
% I want to do that on my own in the stylefiles for makeindex...
\makeatletter
\let\@old@wrindex=\@wrindex
\makeatother

\usepackage{ifpdf}
\ifpdf
	\usepackage[pdftex]{hyperref}
\else
	\usepackage[dvipdfm]{hyperref}
\fi
\hypersetup{pdfborder={0 0 0}}

\makeatletter
\let\@wrindex=\@old@wrindex
\makeatother

% Formatiere Seitennummern im Index:
\newcommand{\indexpageno}[1]{%
	{\bfseries\hyperpage{#1}}%
}


\newcommand{\R}{\mathbb{R}}
\newcommand{\N}{\mathbb{N}}
\newcommand{\Z}{\mathbb{Z}}

\long\def\COMMENTLOWLEVEL#1\ENDCOMMENT{}
\def\ENDCOMMENT{}

\usepackage{textcomp}
\usepackage{booktabs}

\usepackage{calc}
\usepackage[formats]{listings}
%\usepackage{courier} % don't use it - the '^' character can't be copy-pasted in courier

\usepackage{array}
\lstset{%
	basicstyle=\ttfamily,
	language=[LaTeX]tex, % Seems as if \lstset{language=tex} must be invoked BEFORE loading tikz!?
	tabsize=4,
	breaklines=true,
	breakindent=0pt
}

\ifpdf
	\pdfinfo {
		/Author	(Christian Feuersaenger)
	}

\else
	\def\pgfsysdriver{pgfsys-dvipdfm.def}
\fi
%\def\pgfsysdriver{pgfsys-pdftex.def}
\usepackage{pgfplots}

\ifpdf
	\usetikzlibrary{pgfplotsclickable}
\fi

%\usepackage{fp}
% ATTENTION:
% this requires pgf version NEWER than 2.00 :
%\usetikzlibrary{fixedpointarithmetic}

\usetikzlibrary{dateplot}

\usepackage[a4paper,left=2.25cm,right=2.25cm,top=2.5cm,bottom=2.5cm,nohead]{geometry}
\usepackage{amsmath,amssymb}
\usepackage{xxcolor}
\usepackage{pifont}
\usepackage[latin1]{inputenc}
\usepackage{amsmath}
\usepackage{eurosym}
\input{pgfplots-macros}

\pgfqkeys{/codeexample}{%
	every codeexample/.style={
		width=8cm,
		/pgfplots/every axis/.append style={legend style={fill=graphicbackground}}
	},
	tabsize=4,
}
\pgfplotsset{
	every axis/.append style={width=7cm},
	filter discard warning=false,
}

\usetikzlibrary{backgrounds,patterns}
% Global styles:
\tikzset{
  shape example/.style={
    color=black!30,
    draw,
    fill=yellow!30,
    line width=.5cm,
    inner xsep=2.5cm,
    inner ysep=0.5cm}
}

\newcommand{\FIXME}[1]{\textcolor{red}{(FIXME: #1)}}

% fuer endvironment 'sidewaysfigure' bspw
% \usepackage{rotating}

\newcommand\Tikz{Ti\textit kZ}
\newcommand\PGF{\textsc{pgf}}
\newcommand\PGFPlots{\textsc{pgfplots}}
\def\PGFPlotstable{\textsc{PgfplotsTable}}


\makeindex

% Fix overful hboxes automatically:
\tolerance=2000
\emergencystretch=10pt

\tikzset{prefix=gnuplot/pgfplots_} % prefix for 'plot function'

\author{%
	Christian Feuers\"anger\footnote{\url{http://wissrech.ins.uni-bonn.de/people/feuersaenger}}\\%
	Institut f\"ur Numerische Simulation\\
	Universit\"at Bonn, Germany}


%\RequirePackage[german,english,francais]{babel}

\pgfplotsmanualexternalexpensivetrue

%\usetikzlibrary{external}
%\tikzexternalize[prefix=figures/]{pgfplots}

\title{%
	Manual for Package \PGFPlots\\
	{\small Version \pgfplotsversion}\\
	{\small\href{http://sourceforge.net/projects/pgfplots}{http://sourceforge.net/projects/pgfplots}}}

%\includeonly{pgfplots.libs}


\begin{document}

\def\plotcoords{%
\addplot coordinates {
(5,8.312e-02)    (17,2.547e-02)   (49,7.407e-03)
(129,2.102e-03)  (321,5.874e-04)  (769,1.623e-04)
(1793,4.442e-05) (4097,1.207e-05) (9217,3.261e-06)
};

\addplot coordinates{
(7,8.472e-02)    (31,3.044e-02)    (111,1.022e-02)
(351,3.303e-03)  (1023,1.039e-03)  (2815,3.196e-04)
(7423,9.658e-05) (18943,2.873e-05) (47103,8.437e-06)
};

\addplot coordinates{
(9,7.881e-02)     (49,3.243e-02)    (209,1.232e-02)
(769,4.454e-03)   (2561,1.551e-03)  (7937,5.236e-04)
(23297,1.723e-04) (65537,5.545e-05) (178177,1.751e-05)
};

\addplot coordinates{
(11,6.887e-02)    (71,3.177e-02)     (351,1.341e-02)
(1471,5.334e-03)  (5503,2.027e-03)   (18943,7.415e-04)
(61183,2.628e-04) (187903,9.063e-05) (553983,3.053e-05)
};

\addplot coordinates{
(13,5.755e-02)     (97,2.925e-02)     (545,1.351e-02)
(2561,5.842e-03)   (10625,2.397e-03)  (40193,9.414e-04)
(141569,3.564e-04) (471041,1.308e-04) 
(1496065,4.670e-05)
};
}%
\maketitle
\begin{abstract}%
\PGFPlots\ draws high--quality function plots in normal or logarithmic scaling with a user-friendly interface directly in \TeX. The user supplies axis labels, legend entries and the plot coordinates for one or more plots and \PGFPlots\ applies axis scaling, computes any logarithms and axis ticks and draws the plots, supporting line plots, scatter plots, piecewise constant plots, bar plots, area plots, mesh-- and surface plots and some more. It is based on Till Tantau's package \PGF/\Tikz.
\end{abstract}
\tableofcontents
\section{Introduction}
This package provides tools to generate plots and labeled axes easily. It draws normal plots, logplots and semi-logplots, in two and three dimensions. Axis ticks, labels, legends (in case of multiple plots) can be added with key-value options. It can cycle through a set of predefined line/marker/color specifications. In summary, its purpose is to simplify the generation of high-quality function and/or data plots, and solving the problems of
\begin{itemize}
	\item consistency of document and font type and font size,
	\item direct use of \TeX\ math mode in axis descriptions,
	\item consistency of data and figures (no third party tool necessary),
	\item inter document consistency using preamble configurations and styles.
\end{itemize}
Although not necessary, separate |.pdf| or |.eps| graphics can be generated using the |external| library developed as part of \Tikz.

\PGFPlots\ is build completely on \Tikz/\PGF. Knowledge of \Tikz\ will simplify the work with \PGFPlots, although it is not required.

\section[About PGFPlots: Preliminaries]{About {\normalfont\PGFPlots}: Preliminaries}
This section contains information about upgrades, the team, the installation (in case you need to do it manually) and troubleshooting. You may skip it completely except for the upgrade remarks.

However, note that this library requires at least \PGF\ version $2.00$. At the time of this writing, many \TeX-distributions still contain the older \PGF\ version $1.18$, so it may be necessary to install a recent \PGF\ prior to using \PGFPlots.

\subsection{Components}
\PGFPlots\ comes with two components:
\begin{enumerate}
	\item the plotting component (which you are currently reading) and
	\item the \PGFPlotstable\ component which simplifies number formatting and postprocessing of numerical tables. It comes as a separate package and has its own manual \href{file:pgfplotstable.pdf}{pgfplotstable.pdf}.
\end{enumerate}

\subsection{Upgrade remarks}
This release provides a lot of improvements which can be found in all detail in \texttt{ChangeLog} for interested readers. However, some attention is useful with respect to the following changes.

\subsubsection{New Optional Features}
\begin{enumerate}
	\item \PGFPlots\ 1.3 comes with user interface improvements. The technical distinction between ``behavior options'' and ``style options'' of older versions is no longer necessary (although still fully supported).

	\item \PGFPlots\ 1.3 has a new feature which allows to \emph{move axis labels tight to tick labels} automatically.

	Since this affects the spacing, it is not enabled be default.

	Use
\begin{codeexample}[code only]
\usepackage{pgfplots}
\pgfplotsset{compat=1.3}
\end{codeexample}
	in your preamble to benefit from the improved spacing\footnote{The |compat=1.3| setting is necessary for new features which move axis labels around like reversed axis. However, \PGFPlots\ will still work as it does in earlier versions, your documents will remain the same if you don't set it explicitly.}.
	Take a look at the next page for the precise definition of |compat|.

	\item \PGFPlots\ 1.3 now supports reversed axes. It is no longer necessary to use work--arounds with negative units.
\pgfkeys{/pdflinks/search key prefixes in/.add={/pgfplots/,}{}}

	Take a look at the |x dir=reverse| key.

	Existing work--arounds will still function properly. Use |\pgfplotsset{compat=1.3}| together with |x dir=reverse| to switch to the new version.
\end{enumerate}

\subsubsection{Old Features Which May Need Attention}
\begin{enumerate}
	\item The |scatter/classes| feature produces proper legends as of version 1.3. This may change the appearance of existing legends of plots with |scatter/classes|.

	\item Due to a small math bug in \PGF\ $2.00$, you \emph{can't} use the math expression `|-x^2|'. It is necessary to use `|0-x^2|' instead. The same holds for
		`|exp(-x^2)|'; use `|exp(0-x^2)|' instead.

		This will be fixed with \PGF\ $>2.00$.

	\item Starting with \PGFPlots\ $1.1$, |\tikzstyle| should \emph{no longer be used} to set \PGFPlots\ options.
	
	Although |\tikzstyle| is still supported for some older \PGFPlots\ options, you should replace any occurance of |\tikzstyle| with |\pgfplotsset{|\meta{style name}|/.style={|\meta{key-value-list}|}}| or the associated |/.append style| variant. See section~\ref{sec:styles} for more detail.
\end{enumerate}
I apologize for any inconvenience caused by these changes.

\begin{pgfplotskey}{compat=\mchoice{1.3,pre 1.3,default,newest} (initially default)}
	The preamble configuration 
\begin{codeexample}[code only]
\usepackage{pgfplots}
\pgfplotsset{compat=1.3}
\end{codeexample}
	allows to choose between backwards compatibility and most recent features.

	\PGFPlots\ version 1.3 comes with new features which may lead to different spacings compared to earlier versions:
	\begin{enumerate}
		\item Version 1.3 comes with the |xlabel near ticks| feature which places (in this case $x$) axis labels ``near'' the tick labels, taking the size of tick labels into account.
		
		Older versions used |xlabel absolute|, i.e.\ an absolute distance between the axis and axis labels (now matter how large or small tick labels are).

		The initial setting \emph{keeps backwards compatibility}.

		You are encouraged to use
\begin{codeexample}[code only]
\usepackage{pgfplots}
\pgfplotsset{compat=1.3}
\end{codeexample}
		in your preamble.
		
		\item Version 1.3 fixes a bug with |anchor=south west| which occurred mainly for reversed axes: the anchors where upside--down. This is fixed now.

		
		If you have written some work--around and want to keep it going, use

		|\pgfplotsset{compat/anchors=pre 1.3}|\pgfmanualpdflabel{/pgfplots/compat/anchors}{} 

		to disable the bug fix.
	\end{enumerate}

	Use |\pgfplotsset{compat=default}| to restore the factory settings.

	The setting |\pgfplotsset{compat=newest}| will always use features of the most recent version. This might result in small changes in the document's appearance.
\end{pgfplotskey}

\subsection{The Team}
\PGFPlots\ has been written mainly by Christian Feuers�nger with many improvements of Pascal Wolkotte and Nick Papior Andersen as a spare time project. We hope it is useful and provides valuable plots.

If you are interested in writing something but don't know how, consider reading the auxiliary manual \href{file:TeX-programming-notes.pdf}{TeX-programming-notes.pdf} which comes with \PGFPlots. It is far from complete, but maybe it is a good starting point (at least for more literature).

\subsection{Acknowledgements}
I thank God for all hours of enjoyed programming. I thank Pascal Wolkotte and Nick Papior Andersen for their programming efforts and contributions as part of the development team. I thank Stefan Tibus, who contributed the |plot shell| feature. I thank Tom Cashman for the contribution of the |reverse legend| feature. Special thanks go to Stefan Pinnow whose tests of \PGFPlots\ lead to numerous quality improvements. Furthermore, I thank Dr.~Schweitzer for many fruitful discussions and Dr.~Meine for his ideas and suggestions. Thanks as well to the many international contributors who provided feature requests or identified bugs or simply manual improvements!

Last but not least, I thank Till Tantau and Mark Wibrow for their excellent graphics (and more) package \PGF\ and \Tikz\ which is the base of \PGFPlots.

%%%%%%%%%%%%%%%%%%%%%%%%%%%%%%%%%%%%%%%%%%%%%%%%%%%%%%%%%%%%%%%%%%%%%%%%%%%%%
%
% Package pgfplots.sty documentation. 
%
% Copyright 2007/2008 by Christian Feuersaenger.
%
% This program is free software: you can redistribute it and/or modify
% it under the terms of the GNU General Public License as published by
% the Free Software Foundation, either version 3 of the License, or
% (at your option) any later version.
% 
% This program is distributed in the hope that it will be useful,
% but WITHOUT ANY WARRANTY; without even the implied warranty of
% MERCHANTABILITY or FITNESS FOR A PARTICULAR PURPOSE.  See the
% GNU General Public License for more details.
% 
% You should have received a copy of the GNU General Public License
% along with this program.  If not, see <http://www.gnu.org/licenses/>.
%
%
%%%%%%%%%%%%%%%%%%%%%%%%%%%%%%%%%%%%%%%%%%%%%%%%%%%%%%%%%%%%%%%%%%%%%%%%%%%%%
\input{pgfplots.preamble.tex}
%\RequirePackage[german,english,francais]{babel}

\pgfplotsmanualexternalexpensivetrue

%\usetikzlibrary{external}
%\tikzexternalize[prefix=figures/]{pgfplots}

\title{%
	Manual for Package \PGFPlots\\
	{\small Version \pgfplotsversion}\\
	{\small\href{http://sourceforge.net/projects/pgfplots}{http://sourceforge.net/projects/pgfplots}}}

%\includeonly{pgfplots.libs}


\begin{document}

\def\plotcoords{%
\addplot coordinates {
(5,8.312e-02)    (17,2.547e-02)   (49,7.407e-03)
(129,2.102e-03)  (321,5.874e-04)  (769,1.623e-04)
(1793,4.442e-05) (4097,1.207e-05) (9217,3.261e-06)
};

\addplot coordinates{
(7,8.472e-02)    (31,3.044e-02)    (111,1.022e-02)
(351,3.303e-03)  (1023,1.039e-03)  (2815,3.196e-04)
(7423,9.658e-05) (18943,2.873e-05) (47103,8.437e-06)
};

\addplot coordinates{
(9,7.881e-02)     (49,3.243e-02)    (209,1.232e-02)
(769,4.454e-03)   (2561,1.551e-03)  (7937,5.236e-04)
(23297,1.723e-04) (65537,5.545e-05) (178177,1.751e-05)
};

\addplot coordinates{
(11,6.887e-02)    (71,3.177e-02)     (351,1.341e-02)
(1471,5.334e-03)  (5503,2.027e-03)   (18943,7.415e-04)
(61183,2.628e-04) (187903,9.063e-05) (553983,3.053e-05)
};

\addplot coordinates{
(13,5.755e-02)     (97,2.925e-02)     (545,1.351e-02)
(2561,5.842e-03)   (10625,2.397e-03)  (40193,9.414e-04)
(141569,3.564e-04) (471041,1.308e-04) 
(1496065,4.670e-05)
};
}%
\maketitle
\begin{abstract}%
\PGFPlots\ draws high--quality function plots in normal or logarithmic scaling with a user-friendly interface directly in \TeX. The user supplies axis labels, legend entries and the plot coordinates for one or more plots and \PGFPlots\ applies axis scaling, computes any logarithms and axis ticks and draws the plots, supporting line plots, scatter plots, piecewise constant plots, bar plots, area plots, mesh-- and surface plots and some more. It is based on Till Tantau's package \PGF/\Tikz.
\end{abstract}
\tableofcontents
\section{Introduction}
This package provides tools to generate plots and labeled axes easily. It draws normal plots, logplots and semi-logplots, in two and three dimensions. Axis ticks, labels, legends (in case of multiple plots) can be added with key-value options. It can cycle through a set of predefined line/marker/color specifications. In summary, its purpose is to simplify the generation of high-quality function and/or data plots, and solving the problems of
\begin{itemize}
	\item consistency of document and font type and font size,
	\item direct use of \TeX\ math mode in axis descriptions,
	\item consistency of data and figures (no third party tool necessary),
	\item inter document consistency using preamble configurations and styles.
\end{itemize}
Although not necessary, separate |.pdf| or |.eps| graphics can be generated using the |external| library developed as part of \Tikz.

\PGFPlots\ is build completely on \Tikz/\PGF. Knowledge of \Tikz\ will simplify the work with \PGFPlots, although it is not required.

\section[About PGFPlots: Preliminaries]{About {\normalfont\PGFPlots}: Preliminaries}
This section contains information about upgrades, the team, the installation (in case you need to do it manually) and troubleshooting. You may skip it completely except for the upgrade remarks.

However, note that this library requires at least \PGF\ version $2.00$. At the time of this writing, many \TeX-distributions still contain the older \PGF\ version $1.18$, so it may be necessary to install a recent \PGF\ prior to using \PGFPlots.

\subsection{Components}
\PGFPlots\ comes with two components:
\begin{enumerate}
	\item the plotting component (which you are currently reading) and
	\item the \PGFPlotstable\ component which simplifies number formatting and postprocessing of numerical tables. It comes as a separate package and has its own manual \href{file:pgfplotstable.pdf}{pgfplotstable.pdf}.
\end{enumerate}

\subsection{Upgrade remarks}
This release provides a lot of improvements which can be found in all detail in \texttt{ChangeLog} for interested readers. However, some attention is useful with respect to the following changes.

\subsubsection{New Optional Features}
\begin{enumerate}
	\item \PGFPlots\ 1.3 comes with user interface improvements. The technical distinction between ``behavior options'' and ``style options'' of older versions is no longer necessary (although still fully supported).

	\item \PGFPlots\ 1.3 has a new feature which allows to \emph{move axis labels tight to tick labels} automatically.

	Since this affects the spacing, it is not enabled be default.

	Use
\begin{codeexample}[code only]
\usepackage{pgfplots}
\pgfplotsset{compat=1.3}
\end{codeexample}
	in your preamble to benefit from the improved spacing\footnote{The |compat=1.3| setting is necessary for new features which move axis labels around like reversed axis. However, \PGFPlots\ will still work as it does in earlier versions, your documents will remain the same if you don't set it explicitly.}.
	Take a look at the next page for the precise definition of |compat|.

	\item \PGFPlots\ 1.3 now supports reversed axes. It is no longer necessary to use work--arounds with negative units.
\pgfkeys{/pdflinks/search key prefixes in/.add={/pgfplots/,}{}}

	Take a look at the |x dir=reverse| key.

	Existing work--arounds will still function properly. Use |\pgfplotsset{compat=1.3}| together with |x dir=reverse| to switch to the new version.
\end{enumerate}

\subsubsection{Old Features Which May Need Attention}
\begin{enumerate}
	\item The |scatter/classes| feature produces proper legends as of version 1.3. This may change the appearance of existing legends of plots with |scatter/classes|.

	\item Due to a small math bug in \PGF\ $2.00$, you \emph{can't} use the math expression `|-x^2|'. It is necessary to use `|0-x^2|' instead. The same holds for
		`|exp(-x^2)|'; use `|exp(0-x^2)|' instead.

		This will be fixed with \PGF\ $>2.00$.

	\item Starting with \PGFPlots\ $1.1$, |\tikzstyle| should \emph{no longer be used} to set \PGFPlots\ options.
	
	Although |\tikzstyle| is still supported for some older \PGFPlots\ options, you should replace any occurance of |\tikzstyle| with |\pgfplotsset{|\meta{style name}|/.style={|\meta{key-value-list}|}}| or the associated |/.append style| variant. See section~\ref{sec:styles} for more detail.
\end{enumerate}
I apologize for any inconvenience caused by these changes.

\begin{pgfplotskey}{compat=\mchoice{1.3,pre 1.3,default,newest} (initially default)}
	The preamble configuration 
\begin{codeexample}[code only]
\usepackage{pgfplots}
\pgfplotsset{compat=1.3}
\end{codeexample}
	allows to choose between backwards compatibility and most recent features.

	\PGFPlots\ version 1.3 comes with new features which may lead to different spacings compared to earlier versions:
	\begin{enumerate}
		\item Version 1.3 comes with the |xlabel near ticks| feature which places (in this case $x$) axis labels ``near'' the tick labels, taking the size of tick labels into account.
		
		Older versions used |xlabel absolute|, i.e.\ an absolute distance between the axis and axis labels (now matter how large or small tick labels are).

		The initial setting \emph{keeps backwards compatibility}.

		You are encouraged to use
\begin{codeexample}[code only]
\usepackage{pgfplots}
\pgfplotsset{compat=1.3}
\end{codeexample}
		in your preamble.
		
		\item Version 1.3 fixes a bug with |anchor=south west| which occurred mainly for reversed axes: the anchors where upside--down. This is fixed now.

		
		If you have written some work--around and want to keep it going, use

		|\pgfplotsset{compat/anchors=pre 1.3}|\pgfmanualpdflabel{/pgfplots/compat/anchors}{} 

		to disable the bug fix.
	\end{enumerate}

	Use |\pgfplotsset{compat=default}| to restore the factory settings.

	The setting |\pgfplotsset{compat=newest}| will always use features of the most recent version. This might result in small changes in the document's appearance.
\end{pgfplotskey}

\subsection{The Team}
\PGFPlots\ has been written mainly by Christian Feuers�nger with many improvements of Pascal Wolkotte and Nick Papior Andersen as a spare time project. We hope it is useful and provides valuable plots.

If you are interested in writing something but don't know how, consider reading the auxiliary manual \href{file:TeX-programming-notes.pdf}{TeX-programming-notes.pdf} which comes with \PGFPlots. It is far from complete, but maybe it is a good starting point (at least for more literature).

\subsection{Acknowledgements}
I thank God for all hours of enjoyed programming. I thank Pascal Wolkotte and Nick Papior Andersen for their programming efforts and contributions as part of the development team. I thank Stefan Tibus, who contributed the |plot shell| feature. I thank Tom Cashman for the contribution of the |reverse legend| feature. Special thanks go to Stefan Pinnow whose tests of \PGFPlots\ lead to numerous quality improvements. Furthermore, I thank Dr.~Schweitzer for many fruitful discussions and Dr.~Meine for his ideas and suggestions. Thanks as well to the many international contributors who provided feature requests or identified bugs or simply manual improvements!

Last but not least, I thank Till Tantau and Mark Wibrow for their excellent graphics (and more) package \PGF\ and \Tikz\ which is the base of \PGFPlots.

\include{pgfplots.install}%
\include{pgfplots.intro}%
\include{pgfplots.reference}%
\include{pgfplots.libs}%
\include{pgfplots.resources}%
\include{pgfplots.importexport}%
\include{pgfplots.basic.reference}%

\printindex

\bibliographystyle{abbrv} %gerapali} %gerabbrv} %gerunsrt.bst} %gerabbrv}% gerplain}
\nocite{pgfplotstable}
\nocite{programmingnotes}
\bibliography{pgfplots}
\end{document}
%
%%%%%%%%%%%%%%%%%%%%%%%%%%%%%%%%%%%%%%%%%%%%%%%%%%%%%%%%%%%%%%%%%%%%%%%%%%%%%
%
% Package pgfplots.sty documentation. 
%
% Copyright 2007/2008 by Christian Feuersaenger.
%
% This program is free software: you can redistribute it and/or modify
% it under the terms of the GNU General Public License as published by
% the Free Software Foundation, either version 3 of the License, or
% (at your option) any later version.
% 
% This program is distributed in the hope that it will be useful,
% but WITHOUT ANY WARRANTY; without even the implied warranty of
% MERCHANTABILITY or FITNESS FOR A PARTICULAR PURPOSE.  See the
% GNU General Public License for more details.
% 
% You should have received a copy of the GNU General Public License
% along with this program.  If not, see <http://www.gnu.org/licenses/>.
%
%
%%%%%%%%%%%%%%%%%%%%%%%%%%%%%%%%%%%%%%%%%%%%%%%%%%%%%%%%%%%%%%%%%%%%%%%%%%%%%
\input{pgfplots.preamble.tex}
%\RequirePackage[german,english,francais]{babel}

\pgfplotsmanualexternalexpensivetrue

%\usetikzlibrary{external}
%\tikzexternalize[prefix=figures/]{pgfplots}

\title{%
	Manual for Package \PGFPlots\\
	{\small Version \pgfplotsversion}\\
	{\small\href{http://sourceforge.net/projects/pgfplots}{http://sourceforge.net/projects/pgfplots}}}

%\includeonly{pgfplots.libs}


\begin{document}

\def\plotcoords{%
\addplot coordinates {
(5,8.312e-02)    (17,2.547e-02)   (49,7.407e-03)
(129,2.102e-03)  (321,5.874e-04)  (769,1.623e-04)
(1793,4.442e-05) (4097,1.207e-05) (9217,3.261e-06)
};

\addplot coordinates{
(7,8.472e-02)    (31,3.044e-02)    (111,1.022e-02)
(351,3.303e-03)  (1023,1.039e-03)  (2815,3.196e-04)
(7423,9.658e-05) (18943,2.873e-05) (47103,8.437e-06)
};

\addplot coordinates{
(9,7.881e-02)     (49,3.243e-02)    (209,1.232e-02)
(769,4.454e-03)   (2561,1.551e-03)  (7937,5.236e-04)
(23297,1.723e-04) (65537,5.545e-05) (178177,1.751e-05)
};

\addplot coordinates{
(11,6.887e-02)    (71,3.177e-02)     (351,1.341e-02)
(1471,5.334e-03)  (5503,2.027e-03)   (18943,7.415e-04)
(61183,2.628e-04) (187903,9.063e-05) (553983,3.053e-05)
};

\addplot coordinates{
(13,5.755e-02)     (97,2.925e-02)     (545,1.351e-02)
(2561,5.842e-03)   (10625,2.397e-03)  (40193,9.414e-04)
(141569,3.564e-04) (471041,1.308e-04) 
(1496065,4.670e-05)
};
}%
\maketitle
\begin{abstract}%
\PGFPlots\ draws high--quality function plots in normal or logarithmic scaling with a user-friendly interface directly in \TeX. The user supplies axis labels, legend entries and the plot coordinates for one or more plots and \PGFPlots\ applies axis scaling, computes any logarithms and axis ticks and draws the plots, supporting line plots, scatter plots, piecewise constant plots, bar plots, area plots, mesh-- and surface plots and some more. It is based on Till Tantau's package \PGF/\Tikz.
\end{abstract}
\tableofcontents
\section{Introduction}
This package provides tools to generate plots and labeled axes easily. It draws normal plots, logplots and semi-logplots, in two and three dimensions. Axis ticks, labels, legends (in case of multiple plots) can be added with key-value options. It can cycle through a set of predefined line/marker/color specifications. In summary, its purpose is to simplify the generation of high-quality function and/or data plots, and solving the problems of
\begin{itemize}
	\item consistency of document and font type and font size,
	\item direct use of \TeX\ math mode in axis descriptions,
	\item consistency of data and figures (no third party tool necessary),
	\item inter document consistency using preamble configurations and styles.
\end{itemize}
Although not necessary, separate |.pdf| or |.eps| graphics can be generated using the |external| library developed as part of \Tikz.

\PGFPlots\ is build completely on \Tikz/\PGF. Knowledge of \Tikz\ will simplify the work with \PGFPlots, although it is not required.

\section[About PGFPlots: Preliminaries]{About {\normalfont\PGFPlots}: Preliminaries}
This section contains information about upgrades, the team, the installation (in case you need to do it manually) and troubleshooting. You may skip it completely except for the upgrade remarks.

However, note that this library requires at least \PGF\ version $2.00$. At the time of this writing, many \TeX-distributions still contain the older \PGF\ version $1.18$, so it may be necessary to install a recent \PGF\ prior to using \PGFPlots.

\subsection{Components}
\PGFPlots\ comes with two components:
\begin{enumerate}
	\item the plotting component (which you are currently reading) and
	\item the \PGFPlotstable\ component which simplifies number formatting and postprocessing of numerical tables. It comes as a separate package and has its own manual \href{file:pgfplotstable.pdf}{pgfplotstable.pdf}.
\end{enumerate}

\subsection{Upgrade remarks}
This release provides a lot of improvements which can be found in all detail in \texttt{ChangeLog} for interested readers. However, some attention is useful with respect to the following changes.

\subsubsection{New Optional Features}
\begin{enumerate}
	\item \PGFPlots\ 1.3 comes with user interface improvements. The technical distinction between ``behavior options'' and ``style options'' of older versions is no longer necessary (although still fully supported).

	\item \PGFPlots\ 1.3 has a new feature which allows to \emph{move axis labels tight to tick labels} automatically.

	Since this affects the spacing, it is not enabled be default.

	Use
\begin{codeexample}[code only]
\usepackage{pgfplots}
\pgfplotsset{compat=1.3}
\end{codeexample}
	in your preamble to benefit from the improved spacing\footnote{The |compat=1.3| setting is necessary for new features which move axis labels around like reversed axis. However, \PGFPlots\ will still work as it does in earlier versions, your documents will remain the same if you don't set it explicitly.}.
	Take a look at the next page for the precise definition of |compat|.

	\item \PGFPlots\ 1.3 now supports reversed axes. It is no longer necessary to use work--arounds with negative units.
\pgfkeys{/pdflinks/search key prefixes in/.add={/pgfplots/,}{}}

	Take a look at the |x dir=reverse| key.

	Existing work--arounds will still function properly. Use |\pgfplotsset{compat=1.3}| together with |x dir=reverse| to switch to the new version.
\end{enumerate}

\subsubsection{Old Features Which May Need Attention}
\begin{enumerate}
	\item The |scatter/classes| feature produces proper legends as of version 1.3. This may change the appearance of existing legends of plots with |scatter/classes|.

	\item Due to a small math bug in \PGF\ $2.00$, you \emph{can't} use the math expression `|-x^2|'. It is necessary to use `|0-x^2|' instead. The same holds for
		`|exp(-x^2)|'; use `|exp(0-x^2)|' instead.

		This will be fixed with \PGF\ $>2.00$.

	\item Starting with \PGFPlots\ $1.1$, |\tikzstyle| should \emph{no longer be used} to set \PGFPlots\ options.
	
	Although |\tikzstyle| is still supported for some older \PGFPlots\ options, you should replace any occurance of |\tikzstyle| with |\pgfplotsset{|\meta{style name}|/.style={|\meta{key-value-list}|}}| or the associated |/.append style| variant. See section~\ref{sec:styles} for more detail.
\end{enumerate}
I apologize for any inconvenience caused by these changes.

\begin{pgfplotskey}{compat=\mchoice{1.3,pre 1.3,default,newest} (initially default)}
	The preamble configuration 
\begin{codeexample}[code only]
\usepackage{pgfplots}
\pgfplotsset{compat=1.3}
\end{codeexample}
	allows to choose between backwards compatibility and most recent features.

	\PGFPlots\ version 1.3 comes with new features which may lead to different spacings compared to earlier versions:
	\begin{enumerate}
		\item Version 1.3 comes with the |xlabel near ticks| feature which places (in this case $x$) axis labels ``near'' the tick labels, taking the size of tick labels into account.
		
		Older versions used |xlabel absolute|, i.e.\ an absolute distance between the axis and axis labels (now matter how large or small tick labels are).

		The initial setting \emph{keeps backwards compatibility}.

		You are encouraged to use
\begin{codeexample}[code only]
\usepackage{pgfplots}
\pgfplotsset{compat=1.3}
\end{codeexample}
		in your preamble.
		
		\item Version 1.3 fixes a bug with |anchor=south west| which occurred mainly for reversed axes: the anchors where upside--down. This is fixed now.

		
		If you have written some work--around and want to keep it going, use

		|\pgfplotsset{compat/anchors=pre 1.3}|\pgfmanualpdflabel{/pgfplots/compat/anchors}{} 

		to disable the bug fix.
	\end{enumerate}

	Use |\pgfplotsset{compat=default}| to restore the factory settings.

	The setting |\pgfplotsset{compat=newest}| will always use features of the most recent version. This might result in small changes in the document's appearance.
\end{pgfplotskey}

\subsection{The Team}
\PGFPlots\ has been written mainly by Christian Feuers�nger with many improvements of Pascal Wolkotte and Nick Papior Andersen as a spare time project. We hope it is useful and provides valuable plots.

If you are interested in writing something but don't know how, consider reading the auxiliary manual \href{file:TeX-programming-notes.pdf}{TeX-programming-notes.pdf} which comes with \PGFPlots. It is far from complete, but maybe it is a good starting point (at least for more literature).

\subsection{Acknowledgements}
I thank God for all hours of enjoyed programming. I thank Pascal Wolkotte and Nick Papior Andersen for their programming efforts and contributions as part of the development team. I thank Stefan Tibus, who contributed the |plot shell| feature. I thank Tom Cashman for the contribution of the |reverse legend| feature. Special thanks go to Stefan Pinnow whose tests of \PGFPlots\ lead to numerous quality improvements. Furthermore, I thank Dr.~Schweitzer for many fruitful discussions and Dr.~Meine for his ideas and suggestions. Thanks as well to the many international contributors who provided feature requests or identified bugs or simply manual improvements!

Last but not least, I thank Till Tantau and Mark Wibrow for their excellent graphics (and more) package \PGF\ and \Tikz\ which is the base of \PGFPlots.

\include{pgfplots.install}%
\include{pgfplots.intro}%
\include{pgfplots.reference}%
\include{pgfplots.libs}%
\include{pgfplots.resources}%
\include{pgfplots.importexport}%
\include{pgfplots.basic.reference}%

\printindex

\bibliographystyle{abbrv} %gerapali} %gerabbrv} %gerunsrt.bst} %gerabbrv}% gerplain}
\nocite{pgfplotstable}
\nocite{programmingnotes}
\bibliography{pgfplots}
\end{document}
%
%%%%%%%%%%%%%%%%%%%%%%%%%%%%%%%%%%%%%%%%%%%%%%%%%%%%%%%%%%%%%%%%%%%%%%%%%%%%%
%
% Package pgfplots.sty documentation. 
%
% Copyright 2007/2008 by Christian Feuersaenger.
%
% This program is free software: you can redistribute it and/or modify
% it under the terms of the GNU General Public License as published by
% the Free Software Foundation, either version 3 of the License, or
% (at your option) any later version.
% 
% This program is distributed in the hope that it will be useful,
% but WITHOUT ANY WARRANTY; without even the implied warranty of
% MERCHANTABILITY or FITNESS FOR A PARTICULAR PURPOSE.  See the
% GNU General Public License for more details.
% 
% You should have received a copy of the GNU General Public License
% along with this program.  If not, see <http://www.gnu.org/licenses/>.
%
%
%%%%%%%%%%%%%%%%%%%%%%%%%%%%%%%%%%%%%%%%%%%%%%%%%%%%%%%%%%%%%%%%%%%%%%%%%%%%%
\input{pgfplots.preamble.tex}
%\RequirePackage[german,english,francais]{babel}

\pgfplotsmanualexternalexpensivetrue

%\usetikzlibrary{external}
%\tikzexternalize[prefix=figures/]{pgfplots}

\title{%
	Manual for Package \PGFPlots\\
	{\small Version \pgfplotsversion}\\
	{\small\href{http://sourceforge.net/projects/pgfplots}{http://sourceforge.net/projects/pgfplots}}}

%\includeonly{pgfplots.libs}


\begin{document}

\def\plotcoords{%
\addplot coordinates {
(5,8.312e-02)    (17,2.547e-02)   (49,7.407e-03)
(129,2.102e-03)  (321,5.874e-04)  (769,1.623e-04)
(1793,4.442e-05) (4097,1.207e-05) (9217,3.261e-06)
};

\addplot coordinates{
(7,8.472e-02)    (31,3.044e-02)    (111,1.022e-02)
(351,3.303e-03)  (1023,1.039e-03)  (2815,3.196e-04)
(7423,9.658e-05) (18943,2.873e-05) (47103,8.437e-06)
};

\addplot coordinates{
(9,7.881e-02)     (49,3.243e-02)    (209,1.232e-02)
(769,4.454e-03)   (2561,1.551e-03)  (7937,5.236e-04)
(23297,1.723e-04) (65537,5.545e-05) (178177,1.751e-05)
};

\addplot coordinates{
(11,6.887e-02)    (71,3.177e-02)     (351,1.341e-02)
(1471,5.334e-03)  (5503,2.027e-03)   (18943,7.415e-04)
(61183,2.628e-04) (187903,9.063e-05) (553983,3.053e-05)
};

\addplot coordinates{
(13,5.755e-02)     (97,2.925e-02)     (545,1.351e-02)
(2561,5.842e-03)   (10625,2.397e-03)  (40193,9.414e-04)
(141569,3.564e-04) (471041,1.308e-04) 
(1496065,4.670e-05)
};
}%
\maketitle
\begin{abstract}%
\PGFPlots\ draws high--quality function plots in normal or logarithmic scaling with a user-friendly interface directly in \TeX. The user supplies axis labels, legend entries and the plot coordinates for one or more plots and \PGFPlots\ applies axis scaling, computes any logarithms and axis ticks and draws the plots, supporting line plots, scatter plots, piecewise constant plots, bar plots, area plots, mesh-- and surface plots and some more. It is based on Till Tantau's package \PGF/\Tikz.
\end{abstract}
\tableofcontents
\section{Introduction}
This package provides tools to generate plots and labeled axes easily. It draws normal plots, logplots and semi-logplots, in two and three dimensions. Axis ticks, labels, legends (in case of multiple plots) can be added with key-value options. It can cycle through a set of predefined line/marker/color specifications. In summary, its purpose is to simplify the generation of high-quality function and/or data plots, and solving the problems of
\begin{itemize}
	\item consistency of document and font type and font size,
	\item direct use of \TeX\ math mode in axis descriptions,
	\item consistency of data and figures (no third party tool necessary),
	\item inter document consistency using preamble configurations and styles.
\end{itemize}
Although not necessary, separate |.pdf| or |.eps| graphics can be generated using the |external| library developed as part of \Tikz.

\PGFPlots\ is build completely on \Tikz/\PGF. Knowledge of \Tikz\ will simplify the work with \PGFPlots, although it is not required.

\section[About PGFPlots: Preliminaries]{About {\normalfont\PGFPlots}: Preliminaries}
This section contains information about upgrades, the team, the installation (in case you need to do it manually) and troubleshooting. You may skip it completely except for the upgrade remarks.

However, note that this library requires at least \PGF\ version $2.00$. At the time of this writing, many \TeX-distributions still contain the older \PGF\ version $1.18$, so it may be necessary to install a recent \PGF\ prior to using \PGFPlots.

\subsection{Components}
\PGFPlots\ comes with two components:
\begin{enumerate}
	\item the plotting component (which you are currently reading) and
	\item the \PGFPlotstable\ component which simplifies number formatting and postprocessing of numerical tables. It comes as a separate package and has its own manual \href{file:pgfplotstable.pdf}{pgfplotstable.pdf}.
\end{enumerate}

\subsection{Upgrade remarks}
This release provides a lot of improvements which can be found in all detail in \texttt{ChangeLog} for interested readers. However, some attention is useful with respect to the following changes.

\subsubsection{New Optional Features}
\begin{enumerate}
	\item \PGFPlots\ 1.3 comes with user interface improvements. The technical distinction between ``behavior options'' and ``style options'' of older versions is no longer necessary (although still fully supported).

	\item \PGFPlots\ 1.3 has a new feature which allows to \emph{move axis labels tight to tick labels} automatically.

	Since this affects the spacing, it is not enabled be default.

	Use
\begin{codeexample}[code only]
\usepackage{pgfplots}
\pgfplotsset{compat=1.3}
\end{codeexample}
	in your preamble to benefit from the improved spacing\footnote{The |compat=1.3| setting is necessary for new features which move axis labels around like reversed axis. However, \PGFPlots\ will still work as it does in earlier versions, your documents will remain the same if you don't set it explicitly.}.
	Take a look at the next page for the precise definition of |compat|.

	\item \PGFPlots\ 1.3 now supports reversed axes. It is no longer necessary to use work--arounds with negative units.
\pgfkeys{/pdflinks/search key prefixes in/.add={/pgfplots/,}{}}

	Take a look at the |x dir=reverse| key.

	Existing work--arounds will still function properly. Use |\pgfplotsset{compat=1.3}| together with |x dir=reverse| to switch to the new version.
\end{enumerate}

\subsubsection{Old Features Which May Need Attention}
\begin{enumerate}
	\item The |scatter/classes| feature produces proper legends as of version 1.3. This may change the appearance of existing legends of plots with |scatter/classes|.

	\item Due to a small math bug in \PGF\ $2.00$, you \emph{can't} use the math expression `|-x^2|'. It is necessary to use `|0-x^2|' instead. The same holds for
		`|exp(-x^2)|'; use `|exp(0-x^2)|' instead.

		This will be fixed with \PGF\ $>2.00$.

	\item Starting with \PGFPlots\ $1.1$, |\tikzstyle| should \emph{no longer be used} to set \PGFPlots\ options.
	
	Although |\tikzstyle| is still supported for some older \PGFPlots\ options, you should replace any occurance of |\tikzstyle| with |\pgfplotsset{|\meta{style name}|/.style={|\meta{key-value-list}|}}| or the associated |/.append style| variant. See section~\ref{sec:styles} for more detail.
\end{enumerate}
I apologize for any inconvenience caused by these changes.

\begin{pgfplotskey}{compat=\mchoice{1.3,pre 1.3,default,newest} (initially default)}
	The preamble configuration 
\begin{codeexample}[code only]
\usepackage{pgfplots}
\pgfplotsset{compat=1.3}
\end{codeexample}
	allows to choose between backwards compatibility and most recent features.

	\PGFPlots\ version 1.3 comes with new features which may lead to different spacings compared to earlier versions:
	\begin{enumerate}
		\item Version 1.3 comes with the |xlabel near ticks| feature which places (in this case $x$) axis labels ``near'' the tick labels, taking the size of tick labels into account.
		
		Older versions used |xlabel absolute|, i.e.\ an absolute distance between the axis and axis labels (now matter how large or small tick labels are).

		The initial setting \emph{keeps backwards compatibility}.

		You are encouraged to use
\begin{codeexample}[code only]
\usepackage{pgfplots}
\pgfplotsset{compat=1.3}
\end{codeexample}
		in your preamble.
		
		\item Version 1.3 fixes a bug with |anchor=south west| which occurred mainly for reversed axes: the anchors where upside--down. This is fixed now.

		
		If you have written some work--around and want to keep it going, use

		|\pgfplotsset{compat/anchors=pre 1.3}|\pgfmanualpdflabel{/pgfplots/compat/anchors}{} 

		to disable the bug fix.
	\end{enumerate}

	Use |\pgfplotsset{compat=default}| to restore the factory settings.

	The setting |\pgfplotsset{compat=newest}| will always use features of the most recent version. This might result in small changes in the document's appearance.
\end{pgfplotskey}

\subsection{The Team}
\PGFPlots\ has been written mainly by Christian Feuers�nger with many improvements of Pascal Wolkotte and Nick Papior Andersen as a spare time project. We hope it is useful and provides valuable plots.

If you are interested in writing something but don't know how, consider reading the auxiliary manual \href{file:TeX-programming-notes.pdf}{TeX-programming-notes.pdf} which comes with \PGFPlots. It is far from complete, but maybe it is a good starting point (at least for more literature).

\subsection{Acknowledgements}
I thank God for all hours of enjoyed programming. I thank Pascal Wolkotte and Nick Papior Andersen for their programming efforts and contributions as part of the development team. I thank Stefan Tibus, who contributed the |plot shell| feature. I thank Tom Cashman for the contribution of the |reverse legend| feature. Special thanks go to Stefan Pinnow whose tests of \PGFPlots\ lead to numerous quality improvements. Furthermore, I thank Dr.~Schweitzer for many fruitful discussions and Dr.~Meine for his ideas and suggestions. Thanks as well to the many international contributors who provided feature requests or identified bugs or simply manual improvements!

Last but not least, I thank Till Tantau and Mark Wibrow for their excellent graphics (and more) package \PGF\ and \Tikz\ which is the base of \PGFPlots.

\include{pgfplots.install}%
\include{pgfplots.intro}%
\include{pgfplots.reference}%
\include{pgfplots.libs}%
\include{pgfplots.resources}%
\include{pgfplots.importexport}%
\include{pgfplots.basic.reference}%

\printindex

\bibliographystyle{abbrv} %gerapali} %gerabbrv} %gerunsrt.bst} %gerabbrv}% gerplain}
\nocite{pgfplotstable}
\nocite{programmingnotes}
\bibliography{pgfplots}
\end{document}
%
%%%%%%%%%%%%%%%%%%%%%%%%%%%%%%%%%%%%%%%%%%%%%%%%%%%%%%%%%%%%%%%%%%%%%%%%%%%%%
%
% Package pgfplots.sty documentation. 
%
% Copyright 2007/2008 by Christian Feuersaenger.
%
% This program is free software: you can redistribute it and/or modify
% it under the terms of the GNU General Public License as published by
% the Free Software Foundation, either version 3 of the License, or
% (at your option) any later version.
% 
% This program is distributed in the hope that it will be useful,
% but WITHOUT ANY WARRANTY; without even the implied warranty of
% MERCHANTABILITY or FITNESS FOR A PARTICULAR PURPOSE.  See the
% GNU General Public License for more details.
% 
% You should have received a copy of the GNU General Public License
% along with this program.  If not, see <http://www.gnu.org/licenses/>.
%
%
%%%%%%%%%%%%%%%%%%%%%%%%%%%%%%%%%%%%%%%%%%%%%%%%%%%%%%%%%%%%%%%%%%%%%%%%%%%%%
\input{pgfplots.preamble.tex}
%\RequirePackage[german,english,francais]{babel}

\pgfplotsmanualexternalexpensivetrue

%\usetikzlibrary{external}
%\tikzexternalize[prefix=figures/]{pgfplots}

\title{%
	Manual for Package \PGFPlots\\
	{\small Version \pgfplotsversion}\\
	{\small\href{http://sourceforge.net/projects/pgfplots}{http://sourceforge.net/projects/pgfplots}}}

%\includeonly{pgfplots.libs}


\begin{document}

\def\plotcoords{%
\addplot coordinates {
(5,8.312e-02)    (17,2.547e-02)   (49,7.407e-03)
(129,2.102e-03)  (321,5.874e-04)  (769,1.623e-04)
(1793,4.442e-05) (4097,1.207e-05) (9217,3.261e-06)
};

\addplot coordinates{
(7,8.472e-02)    (31,3.044e-02)    (111,1.022e-02)
(351,3.303e-03)  (1023,1.039e-03)  (2815,3.196e-04)
(7423,9.658e-05) (18943,2.873e-05) (47103,8.437e-06)
};

\addplot coordinates{
(9,7.881e-02)     (49,3.243e-02)    (209,1.232e-02)
(769,4.454e-03)   (2561,1.551e-03)  (7937,5.236e-04)
(23297,1.723e-04) (65537,5.545e-05) (178177,1.751e-05)
};

\addplot coordinates{
(11,6.887e-02)    (71,3.177e-02)     (351,1.341e-02)
(1471,5.334e-03)  (5503,2.027e-03)   (18943,7.415e-04)
(61183,2.628e-04) (187903,9.063e-05) (553983,3.053e-05)
};

\addplot coordinates{
(13,5.755e-02)     (97,2.925e-02)     (545,1.351e-02)
(2561,5.842e-03)   (10625,2.397e-03)  (40193,9.414e-04)
(141569,3.564e-04) (471041,1.308e-04) 
(1496065,4.670e-05)
};
}%
\maketitle
\begin{abstract}%
\PGFPlots\ draws high--quality function plots in normal or logarithmic scaling with a user-friendly interface directly in \TeX. The user supplies axis labels, legend entries and the plot coordinates for one or more plots and \PGFPlots\ applies axis scaling, computes any logarithms and axis ticks and draws the plots, supporting line plots, scatter plots, piecewise constant plots, bar plots, area plots, mesh-- and surface plots and some more. It is based on Till Tantau's package \PGF/\Tikz.
\end{abstract}
\tableofcontents
\section{Introduction}
This package provides tools to generate plots and labeled axes easily. It draws normal plots, logplots and semi-logplots, in two and three dimensions. Axis ticks, labels, legends (in case of multiple plots) can be added with key-value options. It can cycle through a set of predefined line/marker/color specifications. In summary, its purpose is to simplify the generation of high-quality function and/or data plots, and solving the problems of
\begin{itemize}
	\item consistency of document and font type and font size,
	\item direct use of \TeX\ math mode in axis descriptions,
	\item consistency of data and figures (no third party tool necessary),
	\item inter document consistency using preamble configurations and styles.
\end{itemize}
Although not necessary, separate |.pdf| or |.eps| graphics can be generated using the |external| library developed as part of \Tikz.

\PGFPlots\ is build completely on \Tikz/\PGF. Knowledge of \Tikz\ will simplify the work with \PGFPlots, although it is not required.

\section[About PGFPlots: Preliminaries]{About {\normalfont\PGFPlots}: Preliminaries}
This section contains information about upgrades, the team, the installation (in case you need to do it manually) and troubleshooting. You may skip it completely except for the upgrade remarks.

However, note that this library requires at least \PGF\ version $2.00$. At the time of this writing, many \TeX-distributions still contain the older \PGF\ version $1.18$, so it may be necessary to install a recent \PGF\ prior to using \PGFPlots.

\subsection{Components}
\PGFPlots\ comes with two components:
\begin{enumerate}
	\item the plotting component (which you are currently reading) and
	\item the \PGFPlotstable\ component which simplifies number formatting and postprocessing of numerical tables. It comes as a separate package and has its own manual \href{file:pgfplotstable.pdf}{pgfplotstable.pdf}.
\end{enumerate}

\subsection{Upgrade remarks}
This release provides a lot of improvements which can be found in all detail in \texttt{ChangeLog} for interested readers. However, some attention is useful with respect to the following changes.

\subsubsection{New Optional Features}
\begin{enumerate}
	\item \PGFPlots\ 1.3 comes with user interface improvements. The technical distinction between ``behavior options'' and ``style options'' of older versions is no longer necessary (although still fully supported).

	\item \PGFPlots\ 1.3 has a new feature which allows to \emph{move axis labels tight to tick labels} automatically.

	Since this affects the spacing, it is not enabled be default.

	Use
\begin{codeexample}[code only]
\usepackage{pgfplots}
\pgfplotsset{compat=1.3}
\end{codeexample}
	in your preamble to benefit from the improved spacing\footnote{The |compat=1.3| setting is necessary for new features which move axis labels around like reversed axis. However, \PGFPlots\ will still work as it does in earlier versions, your documents will remain the same if you don't set it explicitly.}.
	Take a look at the next page for the precise definition of |compat|.

	\item \PGFPlots\ 1.3 now supports reversed axes. It is no longer necessary to use work--arounds with negative units.
\pgfkeys{/pdflinks/search key prefixes in/.add={/pgfplots/,}{}}

	Take a look at the |x dir=reverse| key.

	Existing work--arounds will still function properly. Use |\pgfplotsset{compat=1.3}| together with |x dir=reverse| to switch to the new version.
\end{enumerate}

\subsubsection{Old Features Which May Need Attention}
\begin{enumerate}
	\item The |scatter/classes| feature produces proper legends as of version 1.3. This may change the appearance of existing legends of plots with |scatter/classes|.

	\item Due to a small math bug in \PGF\ $2.00$, you \emph{can't} use the math expression `|-x^2|'. It is necessary to use `|0-x^2|' instead. The same holds for
		`|exp(-x^2)|'; use `|exp(0-x^2)|' instead.

		This will be fixed with \PGF\ $>2.00$.

	\item Starting with \PGFPlots\ $1.1$, |\tikzstyle| should \emph{no longer be used} to set \PGFPlots\ options.
	
	Although |\tikzstyle| is still supported for some older \PGFPlots\ options, you should replace any occurance of |\tikzstyle| with |\pgfplotsset{|\meta{style name}|/.style={|\meta{key-value-list}|}}| or the associated |/.append style| variant. See section~\ref{sec:styles} for more detail.
\end{enumerate}
I apologize for any inconvenience caused by these changes.

\begin{pgfplotskey}{compat=\mchoice{1.3,pre 1.3,default,newest} (initially default)}
	The preamble configuration 
\begin{codeexample}[code only]
\usepackage{pgfplots}
\pgfplotsset{compat=1.3}
\end{codeexample}
	allows to choose between backwards compatibility and most recent features.

	\PGFPlots\ version 1.3 comes with new features which may lead to different spacings compared to earlier versions:
	\begin{enumerate}
		\item Version 1.3 comes with the |xlabel near ticks| feature which places (in this case $x$) axis labels ``near'' the tick labels, taking the size of tick labels into account.
		
		Older versions used |xlabel absolute|, i.e.\ an absolute distance between the axis and axis labels (now matter how large or small tick labels are).

		The initial setting \emph{keeps backwards compatibility}.

		You are encouraged to use
\begin{codeexample}[code only]
\usepackage{pgfplots}
\pgfplotsset{compat=1.3}
\end{codeexample}
		in your preamble.
		
		\item Version 1.3 fixes a bug with |anchor=south west| which occurred mainly for reversed axes: the anchors where upside--down. This is fixed now.

		
		If you have written some work--around and want to keep it going, use

		|\pgfplotsset{compat/anchors=pre 1.3}|\pgfmanualpdflabel{/pgfplots/compat/anchors}{} 

		to disable the bug fix.
	\end{enumerate}

	Use |\pgfplotsset{compat=default}| to restore the factory settings.

	The setting |\pgfplotsset{compat=newest}| will always use features of the most recent version. This might result in small changes in the document's appearance.
\end{pgfplotskey}

\subsection{The Team}
\PGFPlots\ has been written mainly by Christian Feuers�nger with many improvements of Pascal Wolkotte and Nick Papior Andersen as a spare time project. We hope it is useful and provides valuable plots.

If you are interested in writing something but don't know how, consider reading the auxiliary manual \href{file:TeX-programming-notes.pdf}{TeX-programming-notes.pdf} which comes with \PGFPlots. It is far from complete, but maybe it is a good starting point (at least for more literature).

\subsection{Acknowledgements}
I thank God for all hours of enjoyed programming. I thank Pascal Wolkotte and Nick Papior Andersen for their programming efforts and contributions as part of the development team. I thank Stefan Tibus, who contributed the |plot shell| feature. I thank Tom Cashman for the contribution of the |reverse legend| feature. Special thanks go to Stefan Pinnow whose tests of \PGFPlots\ lead to numerous quality improvements. Furthermore, I thank Dr.~Schweitzer for many fruitful discussions and Dr.~Meine for his ideas and suggestions. Thanks as well to the many international contributors who provided feature requests or identified bugs or simply manual improvements!

Last but not least, I thank Till Tantau and Mark Wibrow for their excellent graphics (and more) package \PGF\ and \Tikz\ which is the base of \PGFPlots.

\include{pgfplots.install}%
\include{pgfplots.intro}%
\include{pgfplots.reference}%
\include{pgfplots.libs}%
\include{pgfplots.resources}%
\include{pgfplots.importexport}%
\include{pgfplots.basic.reference}%

\printindex

\bibliographystyle{abbrv} %gerapali} %gerabbrv} %gerunsrt.bst} %gerabbrv}% gerplain}
\nocite{pgfplotstable}
\nocite{programmingnotes}
\bibliography{pgfplots}
\end{document}
%
%%%%%%%%%%%%%%%%%%%%%%%%%%%%%%%%%%%%%%%%%%%%%%%%%%%%%%%%%%%%%%%%%%%%%%%%%%%%%
%
% Package pgfplots.sty documentation. 
%
% Copyright 2007/2008 by Christian Feuersaenger.
%
% This program is free software: you can redistribute it and/or modify
% it under the terms of the GNU General Public License as published by
% the Free Software Foundation, either version 3 of the License, or
% (at your option) any later version.
% 
% This program is distributed in the hope that it will be useful,
% but WITHOUT ANY WARRANTY; without even the implied warranty of
% MERCHANTABILITY or FITNESS FOR A PARTICULAR PURPOSE.  See the
% GNU General Public License for more details.
% 
% You should have received a copy of the GNU General Public License
% along with this program.  If not, see <http://www.gnu.org/licenses/>.
%
%
%%%%%%%%%%%%%%%%%%%%%%%%%%%%%%%%%%%%%%%%%%%%%%%%%%%%%%%%%%%%%%%%%%%%%%%%%%%%%
\input{pgfplots.preamble.tex}
%\RequirePackage[german,english,francais]{babel}

\pgfplotsmanualexternalexpensivetrue

%\usetikzlibrary{external}
%\tikzexternalize[prefix=figures/]{pgfplots}

\title{%
	Manual for Package \PGFPlots\\
	{\small Version \pgfplotsversion}\\
	{\small\href{http://sourceforge.net/projects/pgfplots}{http://sourceforge.net/projects/pgfplots}}}

%\includeonly{pgfplots.libs}


\begin{document}

\def\plotcoords{%
\addplot coordinates {
(5,8.312e-02)    (17,2.547e-02)   (49,7.407e-03)
(129,2.102e-03)  (321,5.874e-04)  (769,1.623e-04)
(1793,4.442e-05) (4097,1.207e-05) (9217,3.261e-06)
};

\addplot coordinates{
(7,8.472e-02)    (31,3.044e-02)    (111,1.022e-02)
(351,3.303e-03)  (1023,1.039e-03)  (2815,3.196e-04)
(7423,9.658e-05) (18943,2.873e-05) (47103,8.437e-06)
};

\addplot coordinates{
(9,7.881e-02)     (49,3.243e-02)    (209,1.232e-02)
(769,4.454e-03)   (2561,1.551e-03)  (7937,5.236e-04)
(23297,1.723e-04) (65537,5.545e-05) (178177,1.751e-05)
};

\addplot coordinates{
(11,6.887e-02)    (71,3.177e-02)     (351,1.341e-02)
(1471,5.334e-03)  (5503,2.027e-03)   (18943,7.415e-04)
(61183,2.628e-04) (187903,9.063e-05) (553983,3.053e-05)
};

\addplot coordinates{
(13,5.755e-02)     (97,2.925e-02)     (545,1.351e-02)
(2561,5.842e-03)   (10625,2.397e-03)  (40193,9.414e-04)
(141569,3.564e-04) (471041,1.308e-04) 
(1496065,4.670e-05)
};
}%
\maketitle
\begin{abstract}%
\PGFPlots\ draws high--quality function plots in normal or logarithmic scaling with a user-friendly interface directly in \TeX. The user supplies axis labels, legend entries and the plot coordinates for one or more plots and \PGFPlots\ applies axis scaling, computes any logarithms and axis ticks and draws the plots, supporting line plots, scatter plots, piecewise constant plots, bar plots, area plots, mesh-- and surface plots and some more. It is based on Till Tantau's package \PGF/\Tikz.
\end{abstract}
\tableofcontents
\section{Introduction}
This package provides tools to generate plots and labeled axes easily. It draws normal plots, logplots and semi-logplots, in two and three dimensions. Axis ticks, labels, legends (in case of multiple plots) can be added with key-value options. It can cycle through a set of predefined line/marker/color specifications. In summary, its purpose is to simplify the generation of high-quality function and/or data plots, and solving the problems of
\begin{itemize}
	\item consistency of document and font type and font size,
	\item direct use of \TeX\ math mode in axis descriptions,
	\item consistency of data and figures (no third party tool necessary),
	\item inter document consistency using preamble configurations and styles.
\end{itemize}
Although not necessary, separate |.pdf| or |.eps| graphics can be generated using the |external| library developed as part of \Tikz.

\PGFPlots\ is build completely on \Tikz/\PGF. Knowledge of \Tikz\ will simplify the work with \PGFPlots, although it is not required.

\section[About PGFPlots: Preliminaries]{About {\normalfont\PGFPlots}: Preliminaries}
This section contains information about upgrades, the team, the installation (in case you need to do it manually) and troubleshooting. You may skip it completely except for the upgrade remarks.

However, note that this library requires at least \PGF\ version $2.00$. At the time of this writing, many \TeX-distributions still contain the older \PGF\ version $1.18$, so it may be necessary to install a recent \PGF\ prior to using \PGFPlots.

\subsection{Components}
\PGFPlots\ comes with two components:
\begin{enumerate}
	\item the plotting component (which you are currently reading) and
	\item the \PGFPlotstable\ component which simplifies number formatting and postprocessing of numerical tables. It comes as a separate package and has its own manual \href{file:pgfplotstable.pdf}{pgfplotstable.pdf}.
\end{enumerate}

\subsection{Upgrade remarks}
This release provides a lot of improvements which can be found in all detail in \texttt{ChangeLog} for interested readers. However, some attention is useful with respect to the following changes.

\subsubsection{New Optional Features}
\begin{enumerate}
	\item \PGFPlots\ 1.3 comes with user interface improvements. The technical distinction between ``behavior options'' and ``style options'' of older versions is no longer necessary (although still fully supported).

	\item \PGFPlots\ 1.3 has a new feature which allows to \emph{move axis labels tight to tick labels} automatically.

	Since this affects the spacing, it is not enabled be default.

	Use
\begin{codeexample}[code only]
\usepackage{pgfplots}
\pgfplotsset{compat=1.3}
\end{codeexample}
	in your preamble to benefit from the improved spacing\footnote{The |compat=1.3| setting is necessary for new features which move axis labels around like reversed axis. However, \PGFPlots\ will still work as it does in earlier versions, your documents will remain the same if you don't set it explicitly.}.
	Take a look at the next page for the precise definition of |compat|.

	\item \PGFPlots\ 1.3 now supports reversed axes. It is no longer necessary to use work--arounds with negative units.
\pgfkeys{/pdflinks/search key prefixes in/.add={/pgfplots/,}{}}

	Take a look at the |x dir=reverse| key.

	Existing work--arounds will still function properly. Use |\pgfplotsset{compat=1.3}| together with |x dir=reverse| to switch to the new version.
\end{enumerate}

\subsubsection{Old Features Which May Need Attention}
\begin{enumerate}
	\item The |scatter/classes| feature produces proper legends as of version 1.3. This may change the appearance of existing legends of plots with |scatter/classes|.

	\item Due to a small math bug in \PGF\ $2.00$, you \emph{can't} use the math expression `|-x^2|'. It is necessary to use `|0-x^2|' instead. The same holds for
		`|exp(-x^2)|'; use `|exp(0-x^2)|' instead.

		This will be fixed with \PGF\ $>2.00$.

	\item Starting with \PGFPlots\ $1.1$, |\tikzstyle| should \emph{no longer be used} to set \PGFPlots\ options.
	
	Although |\tikzstyle| is still supported for some older \PGFPlots\ options, you should replace any occurance of |\tikzstyle| with |\pgfplotsset{|\meta{style name}|/.style={|\meta{key-value-list}|}}| or the associated |/.append style| variant. See section~\ref{sec:styles} for more detail.
\end{enumerate}
I apologize for any inconvenience caused by these changes.

\begin{pgfplotskey}{compat=\mchoice{1.3,pre 1.3,default,newest} (initially default)}
	The preamble configuration 
\begin{codeexample}[code only]
\usepackage{pgfplots}
\pgfplotsset{compat=1.3}
\end{codeexample}
	allows to choose between backwards compatibility and most recent features.

	\PGFPlots\ version 1.3 comes with new features which may lead to different spacings compared to earlier versions:
	\begin{enumerate}
		\item Version 1.3 comes with the |xlabel near ticks| feature which places (in this case $x$) axis labels ``near'' the tick labels, taking the size of tick labels into account.
		
		Older versions used |xlabel absolute|, i.e.\ an absolute distance between the axis and axis labels (now matter how large or small tick labels are).

		The initial setting \emph{keeps backwards compatibility}.

		You are encouraged to use
\begin{codeexample}[code only]
\usepackage{pgfplots}
\pgfplotsset{compat=1.3}
\end{codeexample}
		in your preamble.
		
		\item Version 1.3 fixes a bug with |anchor=south west| which occurred mainly for reversed axes: the anchors where upside--down. This is fixed now.

		
		If you have written some work--around and want to keep it going, use

		|\pgfplotsset{compat/anchors=pre 1.3}|\pgfmanualpdflabel{/pgfplots/compat/anchors}{} 

		to disable the bug fix.
	\end{enumerate}

	Use |\pgfplotsset{compat=default}| to restore the factory settings.

	The setting |\pgfplotsset{compat=newest}| will always use features of the most recent version. This might result in small changes in the document's appearance.
\end{pgfplotskey}

\subsection{The Team}
\PGFPlots\ has been written mainly by Christian Feuers�nger with many improvements of Pascal Wolkotte and Nick Papior Andersen as a spare time project. We hope it is useful and provides valuable plots.

If you are interested in writing something but don't know how, consider reading the auxiliary manual \href{file:TeX-programming-notes.pdf}{TeX-programming-notes.pdf} which comes with \PGFPlots. It is far from complete, but maybe it is a good starting point (at least for more literature).

\subsection{Acknowledgements}
I thank God for all hours of enjoyed programming. I thank Pascal Wolkotte and Nick Papior Andersen for their programming efforts and contributions as part of the development team. I thank Stefan Tibus, who contributed the |plot shell| feature. I thank Tom Cashman for the contribution of the |reverse legend| feature. Special thanks go to Stefan Pinnow whose tests of \PGFPlots\ lead to numerous quality improvements. Furthermore, I thank Dr.~Schweitzer for many fruitful discussions and Dr.~Meine for his ideas and suggestions. Thanks as well to the many international contributors who provided feature requests or identified bugs or simply manual improvements!

Last but not least, I thank Till Tantau and Mark Wibrow for their excellent graphics (and more) package \PGF\ and \Tikz\ which is the base of \PGFPlots.

\include{pgfplots.install}%
\include{pgfplots.intro}%
\include{pgfplots.reference}%
\include{pgfplots.libs}%
\include{pgfplots.resources}%
\include{pgfplots.importexport}%
\include{pgfplots.basic.reference}%

\printindex

\bibliographystyle{abbrv} %gerapali} %gerabbrv} %gerunsrt.bst} %gerabbrv}% gerplain}
\nocite{pgfplotstable}
\nocite{programmingnotes}
\bibliography{pgfplots}
\end{document}
%
%%%%%%%%%%%%%%%%%%%%%%%%%%%%%%%%%%%%%%%%%%%%%%%%%%%%%%%%%%%%%%%%%%%%%%%%%%%%%
%
% Package pgfplots.sty documentation. 
%
% Copyright 2007/2008 by Christian Feuersaenger.
%
% This program is free software: you can redistribute it and/or modify
% it under the terms of the GNU General Public License as published by
% the Free Software Foundation, either version 3 of the License, or
% (at your option) any later version.
% 
% This program is distributed in the hope that it will be useful,
% but WITHOUT ANY WARRANTY; without even the implied warranty of
% MERCHANTABILITY or FITNESS FOR A PARTICULAR PURPOSE.  See the
% GNU General Public License for more details.
% 
% You should have received a copy of the GNU General Public License
% along with this program.  If not, see <http://www.gnu.org/licenses/>.
%
%
%%%%%%%%%%%%%%%%%%%%%%%%%%%%%%%%%%%%%%%%%%%%%%%%%%%%%%%%%%%%%%%%%%%%%%%%%%%%%
\input{pgfplots.preamble.tex}
%\RequirePackage[german,english,francais]{babel}

\pgfplotsmanualexternalexpensivetrue

%\usetikzlibrary{external}
%\tikzexternalize[prefix=figures/]{pgfplots}

\title{%
	Manual for Package \PGFPlots\\
	{\small Version \pgfplotsversion}\\
	{\small\href{http://sourceforge.net/projects/pgfplots}{http://sourceforge.net/projects/pgfplots}}}

%\includeonly{pgfplots.libs}


\begin{document}

\def\plotcoords{%
\addplot coordinates {
(5,8.312e-02)    (17,2.547e-02)   (49,7.407e-03)
(129,2.102e-03)  (321,5.874e-04)  (769,1.623e-04)
(1793,4.442e-05) (4097,1.207e-05) (9217,3.261e-06)
};

\addplot coordinates{
(7,8.472e-02)    (31,3.044e-02)    (111,1.022e-02)
(351,3.303e-03)  (1023,1.039e-03)  (2815,3.196e-04)
(7423,9.658e-05) (18943,2.873e-05) (47103,8.437e-06)
};

\addplot coordinates{
(9,7.881e-02)     (49,3.243e-02)    (209,1.232e-02)
(769,4.454e-03)   (2561,1.551e-03)  (7937,5.236e-04)
(23297,1.723e-04) (65537,5.545e-05) (178177,1.751e-05)
};

\addplot coordinates{
(11,6.887e-02)    (71,3.177e-02)     (351,1.341e-02)
(1471,5.334e-03)  (5503,2.027e-03)   (18943,7.415e-04)
(61183,2.628e-04) (187903,9.063e-05) (553983,3.053e-05)
};

\addplot coordinates{
(13,5.755e-02)     (97,2.925e-02)     (545,1.351e-02)
(2561,5.842e-03)   (10625,2.397e-03)  (40193,9.414e-04)
(141569,3.564e-04) (471041,1.308e-04) 
(1496065,4.670e-05)
};
}%
\maketitle
\begin{abstract}%
\PGFPlots\ draws high--quality function plots in normal or logarithmic scaling with a user-friendly interface directly in \TeX. The user supplies axis labels, legend entries and the plot coordinates for one or more plots and \PGFPlots\ applies axis scaling, computes any logarithms and axis ticks and draws the plots, supporting line plots, scatter plots, piecewise constant plots, bar plots, area plots, mesh-- and surface plots and some more. It is based on Till Tantau's package \PGF/\Tikz.
\end{abstract}
\tableofcontents
\section{Introduction}
This package provides tools to generate plots and labeled axes easily. It draws normal plots, logplots and semi-logplots, in two and three dimensions. Axis ticks, labels, legends (in case of multiple plots) can be added with key-value options. It can cycle through a set of predefined line/marker/color specifications. In summary, its purpose is to simplify the generation of high-quality function and/or data plots, and solving the problems of
\begin{itemize}
	\item consistency of document and font type and font size,
	\item direct use of \TeX\ math mode in axis descriptions,
	\item consistency of data and figures (no third party tool necessary),
	\item inter document consistency using preamble configurations and styles.
\end{itemize}
Although not necessary, separate |.pdf| or |.eps| graphics can be generated using the |external| library developed as part of \Tikz.

\PGFPlots\ is build completely on \Tikz/\PGF. Knowledge of \Tikz\ will simplify the work with \PGFPlots, although it is not required.

\section[About PGFPlots: Preliminaries]{About {\normalfont\PGFPlots}: Preliminaries}
This section contains information about upgrades, the team, the installation (in case you need to do it manually) and troubleshooting. You may skip it completely except for the upgrade remarks.

However, note that this library requires at least \PGF\ version $2.00$. At the time of this writing, many \TeX-distributions still contain the older \PGF\ version $1.18$, so it may be necessary to install a recent \PGF\ prior to using \PGFPlots.

\subsection{Components}
\PGFPlots\ comes with two components:
\begin{enumerate}
	\item the plotting component (which you are currently reading) and
	\item the \PGFPlotstable\ component which simplifies number formatting and postprocessing of numerical tables. It comes as a separate package and has its own manual \href{file:pgfplotstable.pdf}{pgfplotstable.pdf}.
\end{enumerate}

\subsection{Upgrade remarks}
This release provides a lot of improvements which can be found in all detail in \texttt{ChangeLog} for interested readers. However, some attention is useful with respect to the following changes.

\subsubsection{New Optional Features}
\begin{enumerate}
	\item \PGFPlots\ 1.3 comes with user interface improvements. The technical distinction between ``behavior options'' and ``style options'' of older versions is no longer necessary (although still fully supported).

	\item \PGFPlots\ 1.3 has a new feature which allows to \emph{move axis labels tight to tick labels} automatically.

	Since this affects the spacing, it is not enabled be default.

	Use
\begin{codeexample}[code only]
\usepackage{pgfplots}
\pgfplotsset{compat=1.3}
\end{codeexample}
	in your preamble to benefit from the improved spacing\footnote{The |compat=1.3| setting is necessary for new features which move axis labels around like reversed axis. However, \PGFPlots\ will still work as it does in earlier versions, your documents will remain the same if you don't set it explicitly.}.
	Take a look at the next page for the precise definition of |compat|.

	\item \PGFPlots\ 1.3 now supports reversed axes. It is no longer necessary to use work--arounds with negative units.
\pgfkeys{/pdflinks/search key prefixes in/.add={/pgfplots/,}{}}

	Take a look at the |x dir=reverse| key.

	Existing work--arounds will still function properly. Use |\pgfplotsset{compat=1.3}| together with |x dir=reverse| to switch to the new version.
\end{enumerate}

\subsubsection{Old Features Which May Need Attention}
\begin{enumerate}
	\item The |scatter/classes| feature produces proper legends as of version 1.3. This may change the appearance of existing legends of plots with |scatter/classes|.

	\item Due to a small math bug in \PGF\ $2.00$, you \emph{can't} use the math expression `|-x^2|'. It is necessary to use `|0-x^2|' instead. The same holds for
		`|exp(-x^2)|'; use `|exp(0-x^2)|' instead.

		This will be fixed with \PGF\ $>2.00$.

	\item Starting with \PGFPlots\ $1.1$, |\tikzstyle| should \emph{no longer be used} to set \PGFPlots\ options.
	
	Although |\tikzstyle| is still supported for some older \PGFPlots\ options, you should replace any occurance of |\tikzstyle| with |\pgfplotsset{|\meta{style name}|/.style={|\meta{key-value-list}|}}| or the associated |/.append style| variant. See section~\ref{sec:styles} for more detail.
\end{enumerate}
I apologize for any inconvenience caused by these changes.

\begin{pgfplotskey}{compat=\mchoice{1.3,pre 1.3,default,newest} (initially default)}
	The preamble configuration 
\begin{codeexample}[code only]
\usepackage{pgfplots}
\pgfplotsset{compat=1.3}
\end{codeexample}
	allows to choose between backwards compatibility and most recent features.

	\PGFPlots\ version 1.3 comes with new features which may lead to different spacings compared to earlier versions:
	\begin{enumerate}
		\item Version 1.3 comes with the |xlabel near ticks| feature which places (in this case $x$) axis labels ``near'' the tick labels, taking the size of tick labels into account.
		
		Older versions used |xlabel absolute|, i.e.\ an absolute distance between the axis and axis labels (now matter how large or small tick labels are).

		The initial setting \emph{keeps backwards compatibility}.

		You are encouraged to use
\begin{codeexample}[code only]
\usepackage{pgfplots}
\pgfplotsset{compat=1.3}
\end{codeexample}
		in your preamble.
		
		\item Version 1.3 fixes a bug with |anchor=south west| which occurred mainly for reversed axes: the anchors where upside--down. This is fixed now.

		
		If you have written some work--around and want to keep it going, use

		|\pgfplotsset{compat/anchors=pre 1.3}|\pgfmanualpdflabel{/pgfplots/compat/anchors}{} 

		to disable the bug fix.
	\end{enumerate}

	Use |\pgfplotsset{compat=default}| to restore the factory settings.

	The setting |\pgfplotsset{compat=newest}| will always use features of the most recent version. This might result in small changes in the document's appearance.
\end{pgfplotskey}

\subsection{The Team}
\PGFPlots\ has been written mainly by Christian Feuers�nger with many improvements of Pascal Wolkotte and Nick Papior Andersen as a spare time project. We hope it is useful and provides valuable plots.

If you are interested in writing something but don't know how, consider reading the auxiliary manual \href{file:TeX-programming-notes.pdf}{TeX-programming-notes.pdf} which comes with \PGFPlots. It is far from complete, but maybe it is a good starting point (at least for more literature).

\subsection{Acknowledgements}
I thank God for all hours of enjoyed programming. I thank Pascal Wolkotte and Nick Papior Andersen for their programming efforts and contributions as part of the development team. I thank Stefan Tibus, who contributed the |plot shell| feature. I thank Tom Cashman for the contribution of the |reverse legend| feature. Special thanks go to Stefan Pinnow whose tests of \PGFPlots\ lead to numerous quality improvements. Furthermore, I thank Dr.~Schweitzer for many fruitful discussions and Dr.~Meine for his ideas and suggestions. Thanks as well to the many international contributors who provided feature requests or identified bugs or simply manual improvements!

Last but not least, I thank Till Tantau and Mark Wibrow for their excellent graphics (and more) package \PGF\ and \Tikz\ which is the base of \PGFPlots.

\include{pgfplots.install}%
\include{pgfplots.intro}%
\include{pgfplots.reference}%
\include{pgfplots.libs}%
\include{pgfplots.resources}%
\include{pgfplots.importexport}%
\include{pgfplots.basic.reference}%

\printindex

\bibliographystyle{abbrv} %gerapali} %gerabbrv} %gerunsrt.bst} %gerabbrv}% gerplain}
\nocite{pgfplotstable}
\nocite{programmingnotes}
\bibliography{pgfplots}
\end{document}
%
\include{pgfplots.basic.reference}%

\printindex

\bibliographystyle{abbrv} %gerapali} %gerabbrv} %gerunsrt.bst} %gerabbrv}% gerplain}
\nocite{pgfplotstable}
\nocite{programmingnotes}
\bibliography{pgfplots}
\end{document}
%
%%%%%%%%%%%%%%%%%%%%%%%%%%%%%%%%%%%%%%%%%%%%%%%%%%%%%%%%%%%%%%%%%%%%%%%%%%%%%
%
% Package pgfplots.sty documentation. 
%
% Copyright 2007/2008 by Christian Feuersaenger.
%
% This program is free software: you can redistribute it and/or modify
% it under the terms of the GNU General Public License as published by
% the Free Software Foundation, either version 3 of the License, or
% (at your option) any later version.
% 
% This program is distributed in the hope that it will be useful,
% but WITHOUT ANY WARRANTY; without even the implied warranty of
% MERCHANTABILITY or FITNESS FOR A PARTICULAR PURPOSE.  See the
% GNU General Public License for more details.
% 
% You should have received a copy of the GNU General Public License
% along with this program.  If not, see <http://www.gnu.org/licenses/>.
%
%
%%%%%%%%%%%%%%%%%%%%%%%%%%%%%%%%%%%%%%%%%%%%%%%%%%%%%%%%%%%%%%%%%%%%%%%%%%%%%
%%%%%%%%%%%%%%%%%%%%%%%%%%%%%%%%%%%%%%%%%%%%%%%%%%%%%%%%%%%%%%%%%%%%%%%%%%%%%
%
% Package pgfplots.sty documentation. 
%
% Copyright 2007/2008 by Christian Feuersaenger.
%
% This program is free software: you can redistribute it and/or modify
% it under the terms of the GNU General Public License as published by
% the Free Software Foundation, either version 3 of the License, or
% (at your option) any later version.
% 
% This program is distributed in the hope that it will be useful,
% but WITHOUT ANY WARRANTY; without even the implied warranty of
% MERCHANTABILITY or FITNESS FOR A PARTICULAR PURPOSE.  See the
% GNU General Public License for more details.
% 
% You should have received a copy of the GNU General Public License
% along with this program.  If not, see <http://www.gnu.org/licenses/>.
%
%
%%%%%%%%%%%%%%%%%%%%%%%%%%%%%%%%%%%%%%%%%%%%%%%%%%%%%%%%%%%%%%%%%%%%%%%%%%%%%
\documentclass[a4paper]{ltxdoc}

\usepackage{makeidx}

% DON't let hyperref overload the format of index and glossary. 
% I want to do that on my own in the stylefiles for makeindex...
\makeatletter
\let\@old@wrindex=\@wrindex
\makeatother

\usepackage{ifpdf}
\ifpdf
	\usepackage[pdftex]{hyperref}
\else
	\usepackage[dvipdfm]{hyperref}
\fi
\hypersetup{pdfborder={0 0 0}}

\makeatletter
\let\@wrindex=\@old@wrindex
\makeatother

% Formatiere Seitennummern im Index:
\newcommand{\indexpageno}[1]{%
	{\bfseries\hyperpage{#1}}%
}


\newcommand{\R}{\mathbb{R}}
\newcommand{\N}{\mathbb{N}}
\newcommand{\Z}{\mathbb{Z}}

\long\def\COMMENTLOWLEVEL#1\ENDCOMMENT{}
\def\ENDCOMMENT{}

\usepackage{textcomp}
\usepackage{booktabs}

\usepackage{calc}
\usepackage[formats]{listings}
%\usepackage{courier} % don't use it - the '^' character can't be copy-pasted in courier

\usepackage{array}
\lstset{%
	basicstyle=\ttfamily,
	language=[LaTeX]tex, % Seems as if \lstset{language=tex} must be invoked BEFORE loading tikz!?
	tabsize=4,
	breaklines=true,
	breakindent=0pt
}

\ifpdf
	\pdfinfo {
		/Author	(Christian Feuersaenger)
	}

\else
	\def\pgfsysdriver{pgfsys-dvipdfm.def}
\fi
%\def\pgfsysdriver{pgfsys-pdftex.def}
\usepackage{pgfplots}

\ifpdf
	\usetikzlibrary{pgfplotsclickable}
\fi

%\usepackage{fp}
% ATTENTION:
% this requires pgf version NEWER than 2.00 :
%\usetikzlibrary{fixedpointarithmetic}

\usetikzlibrary{dateplot}

\usepackage[a4paper,left=2.25cm,right=2.25cm,top=2.5cm,bottom=2.5cm,nohead]{geometry}
\usepackage{amsmath,amssymb}
\usepackage{xxcolor}
\usepackage{pifont}
\usepackage[latin1]{inputenc}
\usepackage{amsmath}
\usepackage{eurosym}
\input{pgfplots-macros}

\pgfqkeys{/codeexample}{%
	every codeexample/.style={
		width=8cm,
		/pgfplots/every axis/.append style={legend style={fill=graphicbackground}}
	},
	tabsize=4,
}
\pgfplotsset{
	every axis/.append style={width=7cm},
	filter discard warning=false,
}

\usetikzlibrary{backgrounds,patterns}
% Global styles:
\tikzset{
  shape example/.style={
    color=black!30,
    draw,
    fill=yellow!30,
    line width=.5cm,
    inner xsep=2.5cm,
    inner ysep=0.5cm}
}

\newcommand{\FIXME}[1]{\textcolor{red}{(FIXME: #1)}}

% fuer endvironment 'sidewaysfigure' bspw
% \usepackage{rotating}

\newcommand\Tikz{Ti\textit kZ}
\newcommand\PGF{\textsc{pgf}}
\newcommand\PGFPlots{\textsc{pgfplots}}
\def\PGFPlotstable{\textsc{PgfplotsTable}}


\makeindex

% Fix overful hboxes automatically:
\tolerance=2000
\emergencystretch=10pt

\tikzset{prefix=gnuplot/pgfplots_} % prefix for 'plot function'

\author{%
	Christian Feuers\"anger\footnote{\url{http://wissrech.ins.uni-bonn.de/people/feuersaenger}}\\%
	Institut f\"ur Numerische Simulation\\
	Universit\"at Bonn, Germany}


%\RequirePackage[german,english,francais]{babel}

\pgfplotsmanualexternalexpensivetrue

%\usetikzlibrary{external}
%\tikzexternalize[prefix=figures/]{pgfplots}

\title{%
	Manual for Package \PGFPlots\\
	{\small Version \pgfplotsversion}\\
	{\small\href{http://sourceforge.net/projects/pgfplots}{http://sourceforge.net/projects/pgfplots}}}

%\includeonly{pgfplots.libs}


\begin{document}

\def\plotcoords{%
\addplot coordinates {
(5,8.312e-02)    (17,2.547e-02)   (49,7.407e-03)
(129,2.102e-03)  (321,5.874e-04)  (769,1.623e-04)
(1793,4.442e-05) (4097,1.207e-05) (9217,3.261e-06)
};

\addplot coordinates{
(7,8.472e-02)    (31,3.044e-02)    (111,1.022e-02)
(351,3.303e-03)  (1023,1.039e-03)  (2815,3.196e-04)
(7423,9.658e-05) (18943,2.873e-05) (47103,8.437e-06)
};

\addplot coordinates{
(9,7.881e-02)     (49,3.243e-02)    (209,1.232e-02)
(769,4.454e-03)   (2561,1.551e-03)  (7937,5.236e-04)
(23297,1.723e-04) (65537,5.545e-05) (178177,1.751e-05)
};

\addplot coordinates{
(11,6.887e-02)    (71,3.177e-02)     (351,1.341e-02)
(1471,5.334e-03)  (5503,2.027e-03)   (18943,7.415e-04)
(61183,2.628e-04) (187903,9.063e-05) (553983,3.053e-05)
};

\addplot coordinates{
(13,5.755e-02)     (97,2.925e-02)     (545,1.351e-02)
(2561,5.842e-03)   (10625,2.397e-03)  (40193,9.414e-04)
(141569,3.564e-04) (471041,1.308e-04) 
(1496065,4.670e-05)
};
}%
\maketitle
\begin{abstract}%
\PGFPlots\ draws high--quality function plots in normal or logarithmic scaling with a user-friendly interface directly in \TeX. The user supplies axis labels, legend entries and the plot coordinates for one or more plots and \PGFPlots\ applies axis scaling, computes any logarithms and axis ticks and draws the plots, supporting line plots, scatter plots, piecewise constant plots, bar plots, area plots, mesh-- and surface plots and some more. It is based on Till Tantau's package \PGF/\Tikz.
\end{abstract}
\tableofcontents
\section{Introduction}
This package provides tools to generate plots and labeled axes easily. It draws normal plots, logplots and semi-logplots, in two and three dimensions. Axis ticks, labels, legends (in case of multiple plots) can be added with key-value options. It can cycle through a set of predefined line/marker/color specifications. In summary, its purpose is to simplify the generation of high-quality function and/or data plots, and solving the problems of
\begin{itemize}
	\item consistency of document and font type and font size,
	\item direct use of \TeX\ math mode in axis descriptions,
	\item consistency of data and figures (no third party tool necessary),
	\item inter document consistency using preamble configurations and styles.
\end{itemize}
Although not necessary, separate |.pdf| or |.eps| graphics can be generated using the |external| library developed as part of \Tikz.

\PGFPlots\ is build completely on \Tikz/\PGF. Knowledge of \Tikz\ will simplify the work with \PGFPlots, although it is not required.

\section[About PGFPlots: Preliminaries]{About {\normalfont\PGFPlots}: Preliminaries}
This section contains information about upgrades, the team, the installation (in case you need to do it manually) and troubleshooting. You may skip it completely except for the upgrade remarks.

However, note that this library requires at least \PGF\ version $2.00$. At the time of this writing, many \TeX-distributions still contain the older \PGF\ version $1.18$, so it may be necessary to install a recent \PGF\ prior to using \PGFPlots.

\subsection{Components}
\PGFPlots\ comes with two components:
\begin{enumerate}
	\item the plotting component (which you are currently reading) and
	\item the \PGFPlotstable\ component which simplifies number formatting and postprocessing of numerical tables. It comes as a separate package and has its own manual \href{file:pgfplotstable.pdf}{pgfplotstable.pdf}.
\end{enumerate}

\subsection{Upgrade remarks}
This release provides a lot of improvements which can be found in all detail in \texttt{ChangeLog} for interested readers. However, some attention is useful with respect to the following changes.

\subsubsection{New Optional Features}
\begin{enumerate}
	\item \PGFPlots\ 1.3 comes with user interface improvements. The technical distinction between ``behavior options'' and ``style options'' of older versions is no longer necessary (although still fully supported).

	\item \PGFPlots\ 1.3 has a new feature which allows to \emph{move axis labels tight to tick labels} automatically.

	Since this affects the spacing, it is not enabled be default.

	Use
\begin{codeexample}[code only]
\usepackage{pgfplots}
\pgfplotsset{compat=1.3}
\end{codeexample}
	in your preamble to benefit from the improved spacing\footnote{The |compat=1.3| setting is necessary for new features which move axis labels around like reversed axis. However, \PGFPlots\ will still work as it does in earlier versions, your documents will remain the same if you don't set it explicitly.}.
	Take a look at the next page for the precise definition of |compat|.

	\item \PGFPlots\ 1.3 now supports reversed axes. It is no longer necessary to use work--arounds with negative units.
\pgfkeys{/pdflinks/search key prefixes in/.add={/pgfplots/,}{}}

	Take a look at the |x dir=reverse| key.

	Existing work--arounds will still function properly. Use |\pgfplotsset{compat=1.3}| together with |x dir=reverse| to switch to the new version.
\end{enumerate}

\subsubsection{Old Features Which May Need Attention}
\begin{enumerate}
	\item The |scatter/classes| feature produces proper legends as of version 1.3. This may change the appearance of existing legends of plots with |scatter/classes|.

	\item Due to a small math bug in \PGF\ $2.00$, you \emph{can't} use the math expression `|-x^2|'. It is necessary to use `|0-x^2|' instead. The same holds for
		`|exp(-x^2)|'; use `|exp(0-x^2)|' instead.

		This will be fixed with \PGF\ $>2.00$.

	\item Starting with \PGFPlots\ $1.1$, |\tikzstyle| should \emph{no longer be used} to set \PGFPlots\ options.
	
	Although |\tikzstyle| is still supported for some older \PGFPlots\ options, you should replace any occurance of |\tikzstyle| with |\pgfplotsset{|\meta{style name}|/.style={|\meta{key-value-list}|}}| or the associated |/.append style| variant. See section~\ref{sec:styles} for more detail.
\end{enumerate}
I apologize for any inconvenience caused by these changes.

\begin{pgfplotskey}{compat=\mchoice{1.3,pre 1.3,default,newest} (initially default)}
	The preamble configuration 
\begin{codeexample}[code only]
\usepackage{pgfplots}
\pgfplotsset{compat=1.3}
\end{codeexample}
	allows to choose between backwards compatibility and most recent features.

	\PGFPlots\ version 1.3 comes with new features which may lead to different spacings compared to earlier versions:
	\begin{enumerate}
		\item Version 1.3 comes with the |xlabel near ticks| feature which places (in this case $x$) axis labels ``near'' the tick labels, taking the size of tick labels into account.
		
		Older versions used |xlabel absolute|, i.e.\ an absolute distance between the axis and axis labels (now matter how large or small tick labels are).

		The initial setting \emph{keeps backwards compatibility}.

		You are encouraged to use
\begin{codeexample}[code only]
\usepackage{pgfplots}
\pgfplotsset{compat=1.3}
\end{codeexample}
		in your preamble.
		
		\item Version 1.3 fixes a bug with |anchor=south west| which occurred mainly for reversed axes: the anchors where upside--down. This is fixed now.

		
		If you have written some work--around and want to keep it going, use

		|\pgfplotsset{compat/anchors=pre 1.3}|\pgfmanualpdflabel{/pgfplots/compat/anchors}{} 

		to disable the bug fix.
	\end{enumerate}

	Use |\pgfplotsset{compat=default}| to restore the factory settings.

	The setting |\pgfplotsset{compat=newest}| will always use features of the most recent version. This might result in small changes in the document's appearance.
\end{pgfplotskey}

\subsection{The Team}
\PGFPlots\ has been written mainly by Christian Feuers�nger with many improvements of Pascal Wolkotte and Nick Papior Andersen as a spare time project. We hope it is useful and provides valuable plots.

If you are interested in writing something but don't know how, consider reading the auxiliary manual \href{file:TeX-programming-notes.pdf}{TeX-programming-notes.pdf} which comes with \PGFPlots. It is far from complete, but maybe it is a good starting point (at least for more literature).

\subsection{Acknowledgements}
I thank God for all hours of enjoyed programming. I thank Pascal Wolkotte and Nick Papior Andersen for their programming efforts and contributions as part of the development team. I thank Stefan Tibus, who contributed the |plot shell| feature. I thank Tom Cashman for the contribution of the |reverse legend| feature. Special thanks go to Stefan Pinnow whose tests of \PGFPlots\ lead to numerous quality improvements. Furthermore, I thank Dr.~Schweitzer for many fruitful discussions and Dr.~Meine for his ideas and suggestions. Thanks as well to the many international contributors who provided feature requests or identified bugs or simply manual improvements!

Last but not least, I thank Till Tantau and Mark Wibrow for their excellent graphics (and more) package \PGF\ and \Tikz\ which is the base of \PGFPlots.

%%%%%%%%%%%%%%%%%%%%%%%%%%%%%%%%%%%%%%%%%%%%%%%%%%%%%%%%%%%%%%%%%%%%%%%%%%%%%
%
% Package pgfplots.sty documentation. 
%
% Copyright 2007/2008 by Christian Feuersaenger.
%
% This program is free software: you can redistribute it and/or modify
% it under the terms of the GNU General Public License as published by
% the Free Software Foundation, either version 3 of the License, or
% (at your option) any later version.
% 
% This program is distributed in the hope that it will be useful,
% but WITHOUT ANY WARRANTY; without even the implied warranty of
% MERCHANTABILITY or FITNESS FOR A PARTICULAR PURPOSE.  See the
% GNU General Public License for more details.
% 
% You should have received a copy of the GNU General Public License
% along with this program.  If not, see <http://www.gnu.org/licenses/>.
%
%
%%%%%%%%%%%%%%%%%%%%%%%%%%%%%%%%%%%%%%%%%%%%%%%%%%%%%%%%%%%%%%%%%%%%%%%%%%%%%
\input{pgfplots.preamble.tex}
%\RequirePackage[german,english,francais]{babel}

\pgfplotsmanualexternalexpensivetrue

%\usetikzlibrary{external}
%\tikzexternalize[prefix=figures/]{pgfplots}

\title{%
	Manual for Package \PGFPlots\\
	{\small Version \pgfplotsversion}\\
	{\small\href{http://sourceforge.net/projects/pgfplots}{http://sourceforge.net/projects/pgfplots}}}

%\includeonly{pgfplots.libs}


\begin{document}

\def\plotcoords{%
\addplot coordinates {
(5,8.312e-02)    (17,2.547e-02)   (49,7.407e-03)
(129,2.102e-03)  (321,5.874e-04)  (769,1.623e-04)
(1793,4.442e-05) (4097,1.207e-05) (9217,3.261e-06)
};

\addplot coordinates{
(7,8.472e-02)    (31,3.044e-02)    (111,1.022e-02)
(351,3.303e-03)  (1023,1.039e-03)  (2815,3.196e-04)
(7423,9.658e-05) (18943,2.873e-05) (47103,8.437e-06)
};

\addplot coordinates{
(9,7.881e-02)     (49,3.243e-02)    (209,1.232e-02)
(769,4.454e-03)   (2561,1.551e-03)  (7937,5.236e-04)
(23297,1.723e-04) (65537,5.545e-05) (178177,1.751e-05)
};

\addplot coordinates{
(11,6.887e-02)    (71,3.177e-02)     (351,1.341e-02)
(1471,5.334e-03)  (5503,2.027e-03)   (18943,7.415e-04)
(61183,2.628e-04) (187903,9.063e-05) (553983,3.053e-05)
};

\addplot coordinates{
(13,5.755e-02)     (97,2.925e-02)     (545,1.351e-02)
(2561,5.842e-03)   (10625,2.397e-03)  (40193,9.414e-04)
(141569,3.564e-04) (471041,1.308e-04) 
(1496065,4.670e-05)
};
}%
\maketitle
\begin{abstract}%
\PGFPlots\ draws high--quality function plots in normal or logarithmic scaling with a user-friendly interface directly in \TeX. The user supplies axis labels, legend entries and the plot coordinates for one or more plots and \PGFPlots\ applies axis scaling, computes any logarithms and axis ticks and draws the plots, supporting line plots, scatter plots, piecewise constant plots, bar plots, area plots, mesh-- and surface plots and some more. It is based on Till Tantau's package \PGF/\Tikz.
\end{abstract}
\tableofcontents
\section{Introduction}
This package provides tools to generate plots and labeled axes easily. It draws normal plots, logplots and semi-logplots, in two and three dimensions. Axis ticks, labels, legends (in case of multiple plots) can be added with key-value options. It can cycle through a set of predefined line/marker/color specifications. In summary, its purpose is to simplify the generation of high-quality function and/or data plots, and solving the problems of
\begin{itemize}
	\item consistency of document and font type and font size,
	\item direct use of \TeX\ math mode in axis descriptions,
	\item consistency of data and figures (no third party tool necessary),
	\item inter document consistency using preamble configurations and styles.
\end{itemize}
Although not necessary, separate |.pdf| or |.eps| graphics can be generated using the |external| library developed as part of \Tikz.

\PGFPlots\ is build completely on \Tikz/\PGF. Knowledge of \Tikz\ will simplify the work with \PGFPlots, although it is not required.

\section[About PGFPlots: Preliminaries]{About {\normalfont\PGFPlots}: Preliminaries}
This section contains information about upgrades, the team, the installation (in case you need to do it manually) and troubleshooting. You may skip it completely except for the upgrade remarks.

However, note that this library requires at least \PGF\ version $2.00$. At the time of this writing, many \TeX-distributions still contain the older \PGF\ version $1.18$, so it may be necessary to install a recent \PGF\ prior to using \PGFPlots.

\subsection{Components}
\PGFPlots\ comes with two components:
\begin{enumerate}
	\item the plotting component (which you are currently reading) and
	\item the \PGFPlotstable\ component which simplifies number formatting and postprocessing of numerical tables. It comes as a separate package and has its own manual \href{file:pgfplotstable.pdf}{pgfplotstable.pdf}.
\end{enumerate}

\subsection{Upgrade remarks}
This release provides a lot of improvements which can be found in all detail in \texttt{ChangeLog} for interested readers. However, some attention is useful with respect to the following changes.

\subsubsection{New Optional Features}
\begin{enumerate}
	\item \PGFPlots\ 1.3 comes with user interface improvements. The technical distinction between ``behavior options'' and ``style options'' of older versions is no longer necessary (although still fully supported).

	\item \PGFPlots\ 1.3 has a new feature which allows to \emph{move axis labels tight to tick labels} automatically.

	Since this affects the spacing, it is not enabled be default.

	Use
\begin{codeexample}[code only]
\usepackage{pgfplots}
\pgfplotsset{compat=1.3}
\end{codeexample}
	in your preamble to benefit from the improved spacing\footnote{The |compat=1.3| setting is necessary for new features which move axis labels around like reversed axis. However, \PGFPlots\ will still work as it does in earlier versions, your documents will remain the same if you don't set it explicitly.}.
	Take a look at the next page for the precise definition of |compat|.

	\item \PGFPlots\ 1.3 now supports reversed axes. It is no longer necessary to use work--arounds with negative units.
\pgfkeys{/pdflinks/search key prefixes in/.add={/pgfplots/,}{}}

	Take a look at the |x dir=reverse| key.

	Existing work--arounds will still function properly. Use |\pgfplotsset{compat=1.3}| together with |x dir=reverse| to switch to the new version.
\end{enumerate}

\subsubsection{Old Features Which May Need Attention}
\begin{enumerate}
	\item The |scatter/classes| feature produces proper legends as of version 1.3. This may change the appearance of existing legends of plots with |scatter/classes|.

	\item Due to a small math bug in \PGF\ $2.00$, you \emph{can't} use the math expression `|-x^2|'. It is necessary to use `|0-x^2|' instead. The same holds for
		`|exp(-x^2)|'; use `|exp(0-x^2)|' instead.

		This will be fixed with \PGF\ $>2.00$.

	\item Starting with \PGFPlots\ $1.1$, |\tikzstyle| should \emph{no longer be used} to set \PGFPlots\ options.
	
	Although |\tikzstyle| is still supported for some older \PGFPlots\ options, you should replace any occurance of |\tikzstyle| with |\pgfplotsset{|\meta{style name}|/.style={|\meta{key-value-list}|}}| or the associated |/.append style| variant. See section~\ref{sec:styles} for more detail.
\end{enumerate}
I apologize for any inconvenience caused by these changes.

\begin{pgfplotskey}{compat=\mchoice{1.3,pre 1.3,default,newest} (initially default)}
	The preamble configuration 
\begin{codeexample}[code only]
\usepackage{pgfplots}
\pgfplotsset{compat=1.3}
\end{codeexample}
	allows to choose between backwards compatibility and most recent features.

	\PGFPlots\ version 1.3 comes with new features which may lead to different spacings compared to earlier versions:
	\begin{enumerate}
		\item Version 1.3 comes with the |xlabel near ticks| feature which places (in this case $x$) axis labels ``near'' the tick labels, taking the size of tick labels into account.
		
		Older versions used |xlabel absolute|, i.e.\ an absolute distance between the axis and axis labels (now matter how large or small tick labels are).

		The initial setting \emph{keeps backwards compatibility}.

		You are encouraged to use
\begin{codeexample}[code only]
\usepackage{pgfplots}
\pgfplotsset{compat=1.3}
\end{codeexample}
		in your preamble.
		
		\item Version 1.3 fixes a bug with |anchor=south west| which occurred mainly for reversed axes: the anchors where upside--down. This is fixed now.

		
		If you have written some work--around and want to keep it going, use

		|\pgfplotsset{compat/anchors=pre 1.3}|\pgfmanualpdflabel{/pgfplots/compat/anchors}{} 

		to disable the bug fix.
	\end{enumerate}

	Use |\pgfplotsset{compat=default}| to restore the factory settings.

	The setting |\pgfplotsset{compat=newest}| will always use features of the most recent version. This might result in small changes in the document's appearance.
\end{pgfplotskey}

\subsection{The Team}
\PGFPlots\ has been written mainly by Christian Feuers�nger with many improvements of Pascal Wolkotte and Nick Papior Andersen as a spare time project. We hope it is useful and provides valuable plots.

If you are interested in writing something but don't know how, consider reading the auxiliary manual \href{file:TeX-programming-notes.pdf}{TeX-programming-notes.pdf} which comes with \PGFPlots. It is far from complete, but maybe it is a good starting point (at least for more literature).

\subsection{Acknowledgements}
I thank God for all hours of enjoyed programming. I thank Pascal Wolkotte and Nick Papior Andersen for their programming efforts and contributions as part of the development team. I thank Stefan Tibus, who contributed the |plot shell| feature. I thank Tom Cashman for the contribution of the |reverse legend| feature. Special thanks go to Stefan Pinnow whose tests of \PGFPlots\ lead to numerous quality improvements. Furthermore, I thank Dr.~Schweitzer for many fruitful discussions and Dr.~Meine for his ideas and suggestions. Thanks as well to the many international contributors who provided feature requests or identified bugs or simply manual improvements!

Last but not least, I thank Till Tantau and Mark Wibrow for their excellent graphics (and more) package \PGF\ and \Tikz\ which is the base of \PGFPlots.

\include{pgfplots.install}%
\include{pgfplots.intro}%
\include{pgfplots.reference}%
\include{pgfplots.libs}%
\include{pgfplots.resources}%
\include{pgfplots.importexport}%
\include{pgfplots.basic.reference}%

\printindex

\bibliographystyle{abbrv} %gerapali} %gerabbrv} %gerunsrt.bst} %gerabbrv}% gerplain}
\nocite{pgfplotstable}
\nocite{programmingnotes}
\bibliography{pgfplots}
\end{document}
%
%%%%%%%%%%%%%%%%%%%%%%%%%%%%%%%%%%%%%%%%%%%%%%%%%%%%%%%%%%%%%%%%%%%%%%%%%%%%%
%
% Package pgfplots.sty documentation. 
%
% Copyright 2007/2008 by Christian Feuersaenger.
%
% This program is free software: you can redistribute it and/or modify
% it under the terms of the GNU General Public License as published by
% the Free Software Foundation, either version 3 of the License, or
% (at your option) any later version.
% 
% This program is distributed in the hope that it will be useful,
% but WITHOUT ANY WARRANTY; without even the implied warranty of
% MERCHANTABILITY or FITNESS FOR A PARTICULAR PURPOSE.  See the
% GNU General Public License for more details.
% 
% You should have received a copy of the GNU General Public License
% along with this program.  If not, see <http://www.gnu.org/licenses/>.
%
%
%%%%%%%%%%%%%%%%%%%%%%%%%%%%%%%%%%%%%%%%%%%%%%%%%%%%%%%%%%%%%%%%%%%%%%%%%%%%%
\input{pgfplots.preamble.tex}
%\RequirePackage[german,english,francais]{babel}

\pgfplotsmanualexternalexpensivetrue

%\usetikzlibrary{external}
%\tikzexternalize[prefix=figures/]{pgfplots}

\title{%
	Manual for Package \PGFPlots\\
	{\small Version \pgfplotsversion}\\
	{\small\href{http://sourceforge.net/projects/pgfplots}{http://sourceforge.net/projects/pgfplots}}}

%\includeonly{pgfplots.libs}


\begin{document}

\def\plotcoords{%
\addplot coordinates {
(5,8.312e-02)    (17,2.547e-02)   (49,7.407e-03)
(129,2.102e-03)  (321,5.874e-04)  (769,1.623e-04)
(1793,4.442e-05) (4097,1.207e-05) (9217,3.261e-06)
};

\addplot coordinates{
(7,8.472e-02)    (31,3.044e-02)    (111,1.022e-02)
(351,3.303e-03)  (1023,1.039e-03)  (2815,3.196e-04)
(7423,9.658e-05) (18943,2.873e-05) (47103,8.437e-06)
};

\addplot coordinates{
(9,7.881e-02)     (49,3.243e-02)    (209,1.232e-02)
(769,4.454e-03)   (2561,1.551e-03)  (7937,5.236e-04)
(23297,1.723e-04) (65537,5.545e-05) (178177,1.751e-05)
};

\addplot coordinates{
(11,6.887e-02)    (71,3.177e-02)     (351,1.341e-02)
(1471,5.334e-03)  (5503,2.027e-03)   (18943,7.415e-04)
(61183,2.628e-04) (187903,9.063e-05) (553983,3.053e-05)
};

\addplot coordinates{
(13,5.755e-02)     (97,2.925e-02)     (545,1.351e-02)
(2561,5.842e-03)   (10625,2.397e-03)  (40193,9.414e-04)
(141569,3.564e-04) (471041,1.308e-04) 
(1496065,4.670e-05)
};
}%
\maketitle
\begin{abstract}%
\PGFPlots\ draws high--quality function plots in normal or logarithmic scaling with a user-friendly interface directly in \TeX. The user supplies axis labels, legend entries and the plot coordinates for one or more plots and \PGFPlots\ applies axis scaling, computes any logarithms and axis ticks and draws the plots, supporting line plots, scatter plots, piecewise constant plots, bar plots, area plots, mesh-- and surface plots and some more. It is based on Till Tantau's package \PGF/\Tikz.
\end{abstract}
\tableofcontents
\section{Introduction}
This package provides tools to generate plots and labeled axes easily. It draws normal plots, logplots and semi-logplots, in two and three dimensions. Axis ticks, labels, legends (in case of multiple plots) can be added with key-value options. It can cycle through a set of predefined line/marker/color specifications. In summary, its purpose is to simplify the generation of high-quality function and/or data plots, and solving the problems of
\begin{itemize}
	\item consistency of document and font type and font size,
	\item direct use of \TeX\ math mode in axis descriptions,
	\item consistency of data and figures (no third party tool necessary),
	\item inter document consistency using preamble configurations and styles.
\end{itemize}
Although not necessary, separate |.pdf| or |.eps| graphics can be generated using the |external| library developed as part of \Tikz.

\PGFPlots\ is build completely on \Tikz/\PGF. Knowledge of \Tikz\ will simplify the work with \PGFPlots, although it is not required.

\section[About PGFPlots: Preliminaries]{About {\normalfont\PGFPlots}: Preliminaries}
This section contains information about upgrades, the team, the installation (in case you need to do it manually) and troubleshooting. You may skip it completely except for the upgrade remarks.

However, note that this library requires at least \PGF\ version $2.00$. At the time of this writing, many \TeX-distributions still contain the older \PGF\ version $1.18$, so it may be necessary to install a recent \PGF\ prior to using \PGFPlots.

\subsection{Components}
\PGFPlots\ comes with two components:
\begin{enumerate}
	\item the plotting component (which you are currently reading) and
	\item the \PGFPlotstable\ component which simplifies number formatting and postprocessing of numerical tables. It comes as a separate package and has its own manual \href{file:pgfplotstable.pdf}{pgfplotstable.pdf}.
\end{enumerate}

\subsection{Upgrade remarks}
This release provides a lot of improvements which can be found in all detail in \texttt{ChangeLog} for interested readers. However, some attention is useful with respect to the following changes.

\subsubsection{New Optional Features}
\begin{enumerate}
	\item \PGFPlots\ 1.3 comes with user interface improvements. The technical distinction between ``behavior options'' and ``style options'' of older versions is no longer necessary (although still fully supported).

	\item \PGFPlots\ 1.3 has a new feature which allows to \emph{move axis labels tight to tick labels} automatically.

	Since this affects the spacing, it is not enabled be default.

	Use
\begin{codeexample}[code only]
\usepackage{pgfplots}
\pgfplotsset{compat=1.3}
\end{codeexample}
	in your preamble to benefit from the improved spacing\footnote{The |compat=1.3| setting is necessary for new features which move axis labels around like reversed axis. However, \PGFPlots\ will still work as it does in earlier versions, your documents will remain the same if you don't set it explicitly.}.
	Take a look at the next page for the precise definition of |compat|.

	\item \PGFPlots\ 1.3 now supports reversed axes. It is no longer necessary to use work--arounds with negative units.
\pgfkeys{/pdflinks/search key prefixes in/.add={/pgfplots/,}{}}

	Take a look at the |x dir=reverse| key.

	Existing work--arounds will still function properly. Use |\pgfplotsset{compat=1.3}| together with |x dir=reverse| to switch to the new version.
\end{enumerate}

\subsubsection{Old Features Which May Need Attention}
\begin{enumerate}
	\item The |scatter/classes| feature produces proper legends as of version 1.3. This may change the appearance of existing legends of plots with |scatter/classes|.

	\item Due to a small math bug in \PGF\ $2.00$, you \emph{can't} use the math expression `|-x^2|'. It is necessary to use `|0-x^2|' instead. The same holds for
		`|exp(-x^2)|'; use `|exp(0-x^2)|' instead.

		This will be fixed with \PGF\ $>2.00$.

	\item Starting with \PGFPlots\ $1.1$, |\tikzstyle| should \emph{no longer be used} to set \PGFPlots\ options.
	
	Although |\tikzstyle| is still supported for some older \PGFPlots\ options, you should replace any occurance of |\tikzstyle| with |\pgfplotsset{|\meta{style name}|/.style={|\meta{key-value-list}|}}| or the associated |/.append style| variant. See section~\ref{sec:styles} for more detail.
\end{enumerate}
I apologize for any inconvenience caused by these changes.

\begin{pgfplotskey}{compat=\mchoice{1.3,pre 1.3,default,newest} (initially default)}
	The preamble configuration 
\begin{codeexample}[code only]
\usepackage{pgfplots}
\pgfplotsset{compat=1.3}
\end{codeexample}
	allows to choose between backwards compatibility and most recent features.

	\PGFPlots\ version 1.3 comes with new features which may lead to different spacings compared to earlier versions:
	\begin{enumerate}
		\item Version 1.3 comes with the |xlabel near ticks| feature which places (in this case $x$) axis labels ``near'' the tick labels, taking the size of tick labels into account.
		
		Older versions used |xlabel absolute|, i.e.\ an absolute distance between the axis and axis labels (now matter how large or small tick labels are).

		The initial setting \emph{keeps backwards compatibility}.

		You are encouraged to use
\begin{codeexample}[code only]
\usepackage{pgfplots}
\pgfplotsset{compat=1.3}
\end{codeexample}
		in your preamble.
		
		\item Version 1.3 fixes a bug with |anchor=south west| which occurred mainly for reversed axes: the anchors where upside--down. This is fixed now.

		
		If you have written some work--around and want to keep it going, use

		|\pgfplotsset{compat/anchors=pre 1.3}|\pgfmanualpdflabel{/pgfplots/compat/anchors}{} 

		to disable the bug fix.
	\end{enumerate}

	Use |\pgfplotsset{compat=default}| to restore the factory settings.

	The setting |\pgfplotsset{compat=newest}| will always use features of the most recent version. This might result in small changes in the document's appearance.
\end{pgfplotskey}

\subsection{The Team}
\PGFPlots\ has been written mainly by Christian Feuers�nger with many improvements of Pascal Wolkotte and Nick Papior Andersen as a spare time project. We hope it is useful and provides valuable plots.

If you are interested in writing something but don't know how, consider reading the auxiliary manual \href{file:TeX-programming-notes.pdf}{TeX-programming-notes.pdf} which comes with \PGFPlots. It is far from complete, but maybe it is a good starting point (at least for more literature).

\subsection{Acknowledgements}
I thank God for all hours of enjoyed programming. I thank Pascal Wolkotte and Nick Papior Andersen for their programming efforts and contributions as part of the development team. I thank Stefan Tibus, who contributed the |plot shell| feature. I thank Tom Cashman for the contribution of the |reverse legend| feature. Special thanks go to Stefan Pinnow whose tests of \PGFPlots\ lead to numerous quality improvements. Furthermore, I thank Dr.~Schweitzer for many fruitful discussions and Dr.~Meine for his ideas and suggestions. Thanks as well to the many international contributors who provided feature requests or identified bugs or simply manual improvements!

Last but not least, I thank Till Tantau and Mark Wibrow for their excellent graphics (and more) package \PGF\ and \Tikz\ which is the base of \PGFPlots.

\include{pgfplots.install}%
\include{pgfplots.intro}%
\include{pgfplots.reference}%
\include{pgfplots.libs}%
\include{pgfplots.resources}%
\include{pgfplots.importexport}%
\include{pgfplots.basic.reference}%

\printindex

\bibliographystyle{abbrv} %gerapali} %gerabbrv} %gerunsrt.bst} %gerabbrv}% gerplain}
\nocite{pgfplotstable}
\nocite{programmingnotes}
\bibliography{pgfplots}
\end{document}
%
%%%%%%%%%%%%%%%%%%%%%%%%%%%%%%%%%%%%%%%%%%%%%%%%%%%%%%%%%%%%%%%%%%%%%%%%%%%%%
%
% Package pgfplots.sty documentation. 
%
% Copyright 2007/2008 by Christian Feuersaenger.
%
% This program is free software: you can redistribute it and/or modify
% it under the terms of the GNU General Public License as published by
% the Free Software Foundation, either version 3 of the License, or
% (at your option) any later version.
% 
% This program is distributed in the hope that it will be useful,
% but WITHOUT ANY WARRANTY; without even the implied warranty of
% MERCHANTABILITY or FITNESS FOR A PARTICULAR PURPOSE.  See the
% GNU General Public License for more details.
% 
% You should have received a copy of the GNU General Public License
% along with this program.  If not, see <http://www.gnu.org/licenses/>.
%
%
%%%%%%%%%%%%%%%%%%%%%%%%%%%%%%%%%%%%%%%%%%%%%%%%%%%%%%%%%%%%%%%%%%%%%%%%%%%%%
\input{pgfplots.preamble.tex}
%\RequirePackage[german,english,francais]{babel}

\pgfplotsmanualexternalexpensivetrue

%\usetikzlibrary{external}
%\tikzexternalize[prefix=figures/]{pgfplots}

\title{%
	Manual for Package \PGFPlots\\
	{\small Version \pgfplotsversion}\\
	{\small\href{http://sourceforge.net/projects/pgfplots}{http://sourceforge.net/projects/pgfplots}}}

%\includeonly{pgfplots.libs}


\begin{document}

\def\plotcoords{%
\addplot coordinates {
(5,8.312e-02)    (17,2.547e-02)   (49,7.407e-03)
(129,2.102e-03)  (321,5.874e-04)  (769,1.623e-04)
(1793,4.442e-05) (4097,1.207e-05) (9217,3.261e-06)
};

\addplot coordinates{
(7,8.472e-02)    (31,3.044e-02)    (111,1.022e-02)
(351,3.303e-03)  (1023,1.039e-03)  (2815,3.196e-04)
(7423,9.658e-05) (18943,2.873e-05) (47103,8.437e-06)
};

\addplot coordinates{
(9,7.881e-02)     (49,3.243e-02)    (209,1.232e-02)
(769,4.454e-03)   (2561,1.551e-03)  (7937,5.236e-04)
(23297,1.723e-04) (65537,5.545e-05) (178177,1.751e-05)
};

\addplot coordinates{
(11,6.887e-02)    (71,3.177e-02)     (351,1.341e-02)
(1471,5.334e-03)  (5503,2.027e-03)   (18943,7.415e-04)
(61183,2.628e-04) (187903,9.063e-05) (553983,3.053e-05)
};

\addplot coordinates{
(13,5.755e-02)     (97,2.925e-02)     (545,1.351e-02)
(2561,5.842e-03)   (10625,2.397e-03)  (40193,9.414e-04)
(141569,3.564e-04) (471041,1.308e-04) 
(1496065,4.670e-05)
};
}%
\maketitle
\begin{abstract}%
\PGFPlots\ draws high--quality function plots in normal or logarithmic scaling with a user-friendly interface directly in \TeX. The user supplies axis labels, legend entries and the plot coordinates for one or more plots and \PGFPlots\ applies axis scaling, computes any logarithms and axis ticks and draws the plots, supporting line plots, scatter plots, piecewise constant plots, bar plots, area plots, mesh-- and surface plots and some more. It is based on Till Tantau's package \PGF/\Tikz.
\end{abstract}
\tableofcontents
\section{Introduction}
This package provides tools to generate plots and labeled axes easily. It draws normal plots, logplots and semi-logplots, in two and three dimensions. Axis ticks, labels, legends (in case of multiple plots) can be added with key-value options. It can cycle through a set of predefined line/marker/color specifications. In summary, its purpose is to simplify the generation of high-quality function and/or data plots, and solving the problems of
\begin{itemize}
	\item consistency of document and font type and font size,
	\item direct use of \TeX\ math mode in axis descriptions,
	\item consistency of data and figures (no third party tool necessary),
	\item inter document consistency using preamble configurations and styles.
\end{itemize}
Although not necessary, separate |.pdf| or |.eps| graphics can be generated using the |external| library developed as part of \Tikz.

\PGFPlots\ is build completely on \Tikz/\PGF. Knowledge of \Tikz\ will simplify the work with \PGFPlots, although it is not required.

\section[About PGFPlots: Preliminaries]{About {\normalfont\PGFPlots}: Preliminaries}
This section contains information about upgrades, the team, the installation (in case you need to do it manually) and troubleshooting. You may skip it completely except for the upgrade remarks.

However, note that this library requires at least \PGF\ version $2.00$. At the time of this writing, many \TeX-distributions still contain the older \PGF\ version $1.18$, so it may be necessary to install a recent \PGF\ prior to using \PGFPlots.

\subsection{Components}
\PGFPlots\ comes with two components:
\begin{enumerate}
	\item the plotting component (which you are currently reading) and
	\item the \PGFPlotstable\ component which simplifies number formatting and postprocessing of numerical tables. It comes as a separate package and has its own manual \href{file:pgfplotstable.pdf}{pgfplotstable.pdf}.
\end{enumerate}

\subsection{Upgrade remarks}
This release provides a lot of improvements which can be found in all detail in \texttt{ChangeLog} for interested readers. However, some attention is useful with respect to the following changes.

\subsubsection{New Optional Features}
\begin{enumerate}
	\item \PGFPlots\ 1.3 comes with user interface improvements. The technical distinction between ``behavior options'' and ``style options'' of older versions is no longer necessary (although still fully supported).

	\item \PGFPlots\ 1.3 has a new feature which allows to \emph{move axis labels tight to tick labels} automatically.

	Since this affects the spacing, it is not enabled be default.

	Use
\begin{codeexample}[code only]
\usepackage{pgfplots}
\pgfplotsset{compat=1.3}
\end{codeexample}
	in your preamble to benefit from the improved spacing\footnote{The |compat=1.3| setting is necessary for new features which move axis labels around like reversed axis. However, \PGFPlots\ will still work as it does in earlier versions, your documents will remain the same if you don't set it explicitly.}.
	Take a look at the next page for the precise definition of |compat|.

	\item \PGFPlots\ 1.3 now supports reversed axes. It is no longer necessary to use work--arounds with negative units.
\pgfkeys{/pdflinks/search key prefixes in/.add={/pgfplots/,}{}}

	Take a look at the |x dir=reverse| key.

	Existing work--arounds will still function properly. Use |\pgfplotsset{compat=1.3}| together with |x dir=reverse| to switch to the new version.
\end{enumerate}

\subsubsection{Old Features Which May Need Attention}
\begin{enumerate}
	\item The |scatter/classes| feature produces proper legends as of version 1.3. This may change the appearance of existing legends of plots with |scatter/classes|.

	\item Due to a small math bug in \PGF\ $2.00$, you \emph{can't} use the math expression `|-x^2|'. It is necessary to use `|0-x^2|' instead. The same holds for
		`|exp(-x^2)|'; use `|exp(0-x^2)|' instead.

		This will be fixed with \PGF\ $>2.00$.

	\item Starting with \PGFPlots\ $1.1$, |\tikzstyle| should \emph{no longer be used} to set \PGFPlots\ options.
	
	Although |\tikzstyle| is still supported for some older \PGFPlots\ options, you should replace any occurance of |\tikzstyle| with |\pgfplotsset{|\meta{style name}|/.style={|\meta{key-value-list}|}}| or the associated |/.append style| variant. See section~\ref{sec:styles} for more detail.
\end{enumerate}
I apologize for any inconvenience caused by these changes.

\begin{pgfplotskey}{compat=\mchoice{1.3,pre 1.3,default,newest} (initially default)}
	The preamble configuration 
\begin{codeexample}[code only]
\usepackage{pgfplots}
\pgfplotsset{compat=1.3}
\end{codeexample}
	allows to choose between backwards compatibility and most recent features.

	\PGFPlots\ version 1.3 comes with new features which may lead to different spacings compared to earlier versions:
	\begin{enumerate}
		\item Version 1.3 comes with the |xlabel near ticks| feature which places (in this case $x$) axis labels ``near'' the tick labels, taking the size of tick labels into account.
		
		Older versions used |xlabel absolute|, i.e.\ an absolute distance between the axis and axis labels (now matter how large or small tick labels are).

		The initial setting \emph{keeps backwards compatibility}.

		You are encouraged to use
\begin{codeexample}[code only]
\usepackage{pgfplots}
\pgfplotsset{compat=1.3}
\end{codeexample}
		in your preamble.
		
		\item Version 1.3 fixes a bug with |anchor=south west| which occurred mainly for reversed axes: the anchors where upside--down. This is fixed now.

		
		If you have written some work--around and want to keep it going, use

		|\pgfplotsset{compat/anchors=pre 1.3}|\pgfmanualpdflabel{/pgfplots/compat/anchors}{} 

		to disable the bug fix.
	\end{enumerate}

	Use |\pgfplotsset{compat=default}| to restore the factory settings.

	The setting |\pgfplotsset{compat=newest}| will always use features of the most recent version. This might result in small changes in the document's appearance.
\end{pgfplotskey}

\subsection{The Team}
\PGFPlots\ has been written mainly by Christian Feuers�nger with many improvements of Pascal Wolkotte and Nick Papior Andersen as a spare time project. We hope it is useful and provides valuable plots.

If you are interested in writing something but don't know how, consider reading the auxiliary manual \href{file:TeX-programming-notes.pdf}{TeX-programming-notes.pdf} which comes with \PGFPlots. It is far from complete, but maybe it is a good starting point (at least for more literature).

\subsection{Acknowledgements}
I thank God for all hours of enjoyed programming. I thank Pascal Wolkotte and Nick Papior Andersen for their programming efforts and contributions as part of the development team. I thank Stefan Tibus, who contributed the |plot shell| feature. I thank Tom Cashman for the contribution of the |reverse legend| feature. Special thanks go to Stefan Pinnow whose tests of \PGFPlots\ lead to numerous quality improvements. Furthermore, I thank Dr.~Schweitzer for many fruitful discussions and Dr.~Meine for his ideas and suggestions. Thanks as well to the many international contributors who provided feature requests or identified bugs or simply manual improvements!

Last but not least, I thank Till Tantau and Mark Wibrow for their excellent graphics (and more) package \PGF\ and \Tikz\ which is the base of \PGFPlots.

\include{pgfplots.install}%
\include{pgfplots.intro}%
\include{pgfplots.reference}%
\include{pgfplots.libs}%
\include{pgfplots.resources}%
\include{pgfplots.importexport}%
\include{pgfplots.basic.reference}%

\printindex

\bibliographystyle{abbrv} %gerapali} %gerabbrv} %gerunsrt.bst} %gerabbrv}% gerplain}
\nocite{pgfplotstable}
\nocite{programmingnotes}
\bibliography{pgfplots}
\end{document}
%
%%%%%%%%%%%%%%%%%%%%%%%%%%%%%%%%%%%%%%%%%%%%%%%%%%%%%%%%%%%%%%%%%%%%%%%%%%%%%
%
% Package pgfplots.sty documentation. 
%
% Copyright 2007/2008 by Christian Feuersaenger.
%
% This program is free software: you can redistribute it and/or modify
% it under the terms of the GNU General Public License as published by
% the Free Software Foundation, either version 3 of the License, or
% (at your option) any later version.
% 
% This program is distributed in the hope that it will be useful,
% but WITHOUT ANY WARRANTY; without even the implied warranty of
% MERCHANTABILITY or FITNESS FOR A PARTICULAR PURPOSE.  See the
% GNU General Public License for more details.
% 
% You should have received a copy of the GNU General Public License
% along with this program.  If not, see <http://www.gnu.org/licenses/>.
%
%
%%%%%%%%%%%%%%%%%%%%%%%%%%%%%%%%%%%%%%%%%%%%%%%%%%%%%%%%%%%%%%%%%%%%%%%%%%%%%
\input{pgfplots.preamble.tex}
%\RequirePackage[german,english,francais]{babel}

\pgfplotsmanualexternalexpensivetrue

%\usetikzlibrary{external}
%\tikzexternalize[prefix=figures/]{pgfplots}

\title{%
	Manual for Package \PGFPlots\\
	{\small Version \pgfplotsversion}\\
	{\small\href{http://sourceforge.net/projects/pgfplots}{http://sourceforge.net/projects/pgfplots}}}

%\includeonly{pgfplots.libs}


\begin{document}

\def\plotcoords{%
\addplot coordinates {
(5,8.312e-02)    (17,2.547e-02)   (49,7.407e-03)
(129,2.102e-03)  (321,5.874e-04)  (769,1.623e-04)
(1793,4.442e-05) (4097,1.207e-05) (9217,3.261e-06)
};

\addplot coordinates{
(7,8.472e-02)    (31,3.044e-02)    (111,1.022e-02)
(351,3.303e-03)  (1023,1.039e-03)  (2815,3.196e-04)
(7423,9.658e-05) (18943,2.873e-05) (47103,8.437e-06)
};

\addplot coordinates{
(9,7.881e-02)     (49,3.243e-02)    (209,1.232e-02)
(769,4.454e-03)   (2561,1.551e-03)  (7937,5.236e-04)
(23297,1.723e-04) (65537,5.545e-05) (178177,1.751e-05)
};

\addplot coordinates{
(11,6.887e-02)    (71,3.177e-02)     (351,1.341e-02)
(1471,5.334e-03)  (5503,2.027e-03)   (18943,7.415e-04)
(61183,2.628e-04) (187903,9.063e-05) (553983,3.053e-05)
};

\addplot coordinates{
(13,5.755e-02)     (97,2.925e-02)     (545,1.351e-02)
(2561,5.842e-03)   (10625,2.397e-03)  (40193,9.414e-04)
(141569,3.564e-04) (471041,1.308e-04) 
(1496065,4.670e-05)
};
}%
\maketitle
\begin{abstract}%
\PGFPlots\ draws high--quality function plots in normal or logarithmic scaling with a user-friendly interface directly in \TeX. The user supplies axis labels, legend entries and the plot coordinates for one or more plots and \PGFPlots\ applies axis scaling, computes any logarithms and axis ticks and draws the plots, supporting line plots, scatter plots, piecewise constant plots, bar plots, area plots, mesh-- and surface plots and some more. It is based on Till Tantau's package \PGF/\Tikz.
\end{abstract}
\tableofcontents
\section{Introduction}
This package provides tools to generate plots and labeled axes easily. It draws normal plots, logplots and semi-logplots, in two and three dimensions. Axis ticks, labels, legends (in case of multiple plots) can be added with key-value options. It can cycle through a set of predefined line/marker/color specifications. In summary, its purpose is to simplify the generation of high-quality function and/or data plots, and solving the problems of
\begin{itemize}
	\item consistency of document and font type and font size,
	\item direct use of \TeX\ math mode in axis descriptions,
	\item consistency of data and figures (no third party tool necessary),
	\item inter document consistency using preamble configurations and styles.
\end{itemize}
Although not necessary, separate |.pdf| or |.eps| graphics can be generated using the |external| library developed as part of \Tikz.

\PGFPlots\ is build completely on \Tikz/\PGF. Knowledge of \Tikz\ will simplify the work with \PGFPlots, although it is not required.

\section[About PGFPlots: Preliminaries]{About {\normalfont\PGFPlots}: Preliminaries}
This section contains information about upgrades, the team, the installation (in case you need to do it manually) and troubleshooting. You may skip it completely except for the upgrade remarks.

However, note that this library requires at least \PGF\ version $2.00$. At the time of this writing, many \TeX-distributions still contain the older \PGF\ version $1.18$, so it may be necessary to install a recent \PGF\ prior to using \PGFPlots.

\subsection{Components}
\PGFPlots\ comes with two components:
\begin{enumerate}
	\item the plotting component (which you are currently reading) and
	\item the \PGFPlotstable\ component which simplifies number formatting and postprocessing of numerical tables. It comes as a separate package and has its own manual \href{file:pgfplotstable.pdf}{pgfplotstable.pdf}.
\end{enumerate}

\subsection{Upgrade remarks}
This release provides a lot of improvements which can be found in all detail in \texttt{ChangeLog} for interested readers. However, some attention is useful with respect to the following changes.

\subsubsection{New Optional Features}
\begin{enumerate}
	\item \PGFPlots\ 1.3 comes with user interface improvements. The technical distinction between ``behavior options'' and ``style options'' of older versions is no longer necessary (although still fully supported).

	\item \PGFPlots\ 1.3 has a new feature which allows to \emph{move axis labels tight to tick labels} automatically.

	Since this affects the spacing, it is not enabled be default.

	Use
\begin{codeexample}[code only]
\usepackage{pgfplots}
\pgfplotsset{compat=1.3}
\end{codeexample}
	in your preamble to benefit from the improved spacing\footnote{The |compat=1.3| setting is necessary for new features which move axis labels around like reversed axis. However, \PGFPlots\ will still work as it does in earlier versions, your documents will remain the same if you don't set it explicitly.}.
	Take a look at the next page for the precise definition of |compat|.

	\item \PGFPlots\ 1.3 now supports reversed axes. It is no longer necessary to use work--arounds with negative units.
\pgfkeys{/pdflinks/search key prefixes in/.add={/pgfplots/,}{}}

	Take a look at the |x dir=reverse| key.

	Existing work--arounds will still function properly. Use |\pgfplotsset{compat=1.3}| together with |x dir=reverse| to switch to the new version.
\end{enumerate}

\subsubsection{Old Features Which May Need Attention}
\begin{enumerate}
	\item The |scatter/classes| feature produces proper legends as of version 1.3. This may change the appearance of existing legends of plots with |scatter/classes|.

	\item Due to a small math bug in \PGF\ $2.00$, you \emph{can't} use the math expression `|-x^2|'. It is necessary to use `|0-x^2|' instead. The same holds for
		`|exp(-x^2)|'; use `|exp(0-x^2)|' instead.

		This will be fixed with \PGF\ $>2.00$.

	\item Starting with \PGFPlots\ $1.1$, |\tikzstyle| should \emph{no longer be used} to set \PGFPlots\ options.
	
	Although |\tikzstyle| is still supported for some older \PGFPlots\ options, you should replace any occurance of |\tikzstyle| with |\pgfplotsset{|\meta{style name}|/.style={|\meta{key-value-list}|}}| or the associated |/.append style| variant. See section~\ref{sec:styles} for more detail.
\end{enumerate}
I apologize for any inconvenience caused by these changes.

\begin{pgfplotskey}{compat=\mchoice{1.3,pre 1.3,default,newest} (initially default)}
	The preamble configuration 
\begin{codeexample}[code only]
\usepackage{pgfplots}
\pgfplotsset{compat=1.3}
\end{codeexample}
	allows to choose between backwards compatibility and most recent features.

	\PGFPlots\ version 1.3 comes with new features which may lead to different spacings compared to earlier versions:
	\begin{enumerate}
		\item Version 1.3 comes with the |xlabel near ticks| feature which places (in this case $x$) axis labels ``near'' the tick labels, taking the size of tick labels into account.
		
		Older versions used |xlabel absolute|, i.e.\ an absolute distance between the axis and axis labels (now matter how large or small tick labels are).

		The initial setting \emph{keeps backwards compatibility}.

		You are encouraged to use
\begin{codeexample}[code only]
\usepackage{pgfplots}
\pgfplotsset{compat=1.3}
\end{codeexample}
		in your preamble.
		
		\item Version 1.3 fixes a bug with |anchor=south west| which occurred mainly for reversed axes: the anchors where upside--down. This is fixed now.

		
		If you have written some work--around and want to keep it going, use

		|\pgfplotsset{compat/anchors=pre 1.3}|\pgfmanualpdflabel{/pgfplots/compat/anchors}{} 

		to disable the bug fix.
	\end{enumerate}

	Use |\pgfplotsset{compat=default}| to restore the factory settings.

	The setting |\pgfplotsset{compat=newest}| will always use features of the most recent version. This might result in small changes in the document's appearance.
\end{pgfplotskey}

\subsection{The Team}
\PGFPlots\ has been written mainly by Christian Feuers�nger with many improvements of Pascal Wolkotte and Nick Papior Andersen as a spare time project. We hope it is useful and provides valuable plots.

If you are interested in writing something but don't know how, consider reading the auxiliary manual \href{file:TeX-programming-notes.pdf}{TeX-programming-notes.pdf} which comes with \PGFPlots. It is far from complete, but maybe it is a good starting point (at least for more literature).

\subsection{Acknowledgements}
I thank God for all hours of enjoyed programming. I thank Pascal Wolkotte and Nick Papior Andersen for their programming efforts and contributions as part of the development team. I thank Stefan Tibus, who contributed the |plot shell| feature. I thank Tom Cashman for the contribution of the |reverse legend| feature. Special thanks go to Stefan Pinnow whose tests of \PGFPlots\ lead to numerous quality improvements. Furthermore, I thank Dr.~Schweitzer for many fruitful discussions and Dr.~Meine for his ideas and suggestions. Thanks as well to the many international contributors who provided feature requests or identified bugs or simply manual improvements!

Last but not least, I thank Till Tantau and Mark Wibrow for their excellent graphics (and more) package \PGF\ and \Tikz\ which is the base of \PGFPlots.

\include{pgfplots.install}%
\include{pgfplots.intro}%
\include{pgfplots.reference}%
\include{pgfplots.libs}%
\include{pgfplots.resources}%
\include{pgfplots.importexport}%
\include{pgfplots.basic.reference}%

\printindex

\bibliographystyle{abbrv} %gerapali} %gerabbrv} %gerunsrt.bst} %gerabbrv}% gerplain}
\nocite{pgfplotstable}
\nocite{programmingnotes}
\bibliography{pgfplots}
\end{document}
%
%%%%%%%%%%%%%%%%%%%%%%%%%%%%%%%%%%%%%%%%%%%%%%%%%%%%%%%%%%%%%%%%%%%%%%%%%%%%%
%
% Package pgfplots.sty documentation. 
%
% Copyright 2007/2008 by Christian Feuersaenger.
%
% This program is free software: you can redistribute it and/or modify
% it under the terms of the GNU General Public License as published by
% the Free Software Foundation, either version 3 of the License, or
% (at your option) any later version.
% 
% This program is distributed in the hope that it will be useful,
% but WITHOUT ANY WARRANTY; without even the implied warranty of
% MERCHANTABILITY or FITNESS FOR A PARTICULAR PURPOSE.  See the
% GNU General Public License for more details.
% 
% You should have received a copy of the GNU General Public License
% along with this program.  If not, see <http://www.gnu.org/licenses/>.
%
%
%%%%%%%%%%%%%%%%%%%%%%%%%%%%%%%%%%%%%%%%%%%%%%%%%%%%%%%%%%%%%%%%%%%%%%%%%%%%%
\input{pgfplots.preamble.tex}
%\RequirePackage[german,english,francais]{babel}

\pgfplotsmanualexternalexpensivetrue

%\usetikzlibrary{external}
%\tikzexternalize[prefix=figures/]{pgfplots}

\title{%
	Manual for Package \PGFPlots\\
	{\small Version \pgfplotsversion}\\
	{\small\href{http://sourceforge.net/projects/pgfplots}{http://sourceforge.net/projects/pgfplots}}}

%\includeonly{pgfplots.libs}


\begin{document}

\def\plotcoords{%
\addplot coordinates {
(5,8.312e-02)    (17,2.547e-02)   (49,7.407e-03)
(129,2.102e-03)  (321,5.874e-04)  (769,1.623e-04)
(1793,4.442e-05) (4097,1.207e-05) (9217,3.261e-06)
};

\addplot coordinates{
(7,8.472e-02)    (31,3.044e-02)    (111,1.022e-02)
(351,3.303e-03)  (1023,1.039e-03)  (2815,3.196e-04)
(7423,9.658e-05) (18943,2.873e-05) (47103,8.437e-06)
};

\addplot coordinates{
(9,7.881e-02)     (49,3.243e-02)    (209,1.232e-02)
(769,4.454e-03)   (2561,1.551e-03)  (7937,5.236e-04)
(23297,1.723e-04) (65537,5.545e-05) (178177,1.751e-05)
};

\addplot coordinates{
(11,6.887e-02)    (71,3.177e-02)     (351,1.341e-02)
(1471,5.334e-03)  (5503,2.027e-03)   (18943,7.415e-04)
(61183,2.628e-04) (187903,9.063e-05) (553983,3.053e-05)
};

\addplot coordinates{
(13,5.755e-02)     (97,2.925e-02)     (545,1.351e-02)
(2561,5.842e-03)   (10625,2.397e-03)  (40193,9.414e-04)
(141569,3.564e-04) (471041,1.308e-04) 
(1496065,4.670e-05)
};
}%
\maketitle
\begin{abstract}%
\PGFPlots\ draws high--quality function plots in normal or logarithmic scaling with a user-friendly interface directly in \TeX. The user supplies axis labels, legend entries and the plot coordinates for one or more plots and \PGFPlots\ applies axis scaling, computes any logarithms and axis ticks and draws the plots, supporting line plots, scatter plots, piecewise constant plots, bar plots, area plots, mesh-- and surface plots and some more. It is based on Till Tantau's package \PGF/\Tikz.
\end{abstract}
\tableofcontents
\section{Introduction}
This package provides tools to generate plots and labeled axes easily. It draws normal plots, logplots and semi-logplots, in two and three dimensions. Axis ticks, labels, legends (in case of multiple plots) can be added with key-value options. It can cycle through a set of predefined line/marker/color specifications. In summary, its purpose is to simplify the generation of high-quality function and/or data plots, and solving the problems of
\begin{itemize}
	\item consistency of document and font type and font size,
	\item direct use of \TeX\ math mode in axis descriptions,
	\item consistency of data and figures (no third party tool necessary),
	\item inter document consistency using preamble configurations and styles.
\end{itemize}
Although not necessary, separate |.pdf| or |.eps| graphics can be generated using the |external| library developed as part of \Tikz.

\PGFPlots\ is build completely on \Tikz/\PGF. Knowledge of \Tikz\ will simplify the work with \PGFPlots, although it is not required.

\section[About PGFPlots: Preliminaries]{About {\normalfont\PGFPlots}: Preliminaries}
This section contains information about upgrades, the team, the installation (in case you need to do it manually) and troubleshooting. You may skip it completely except for the upgrade remarks.

However, note that this library requires at least \PGF\ version $2.00$. At the time of this writing, many \TeX-distributions still contain the older \PGF\ version $1.18$, so it may be necessary to install a recent \PGF\ prior to using \PGFPlots.

\subsection{Components}
\PGFPlots\ comes with two components:
\begin{enumerate}
	\item the plotting component (which you are currently reading) and
	\item the \PGFPlotstable\ component which simplifies number formatting and postprocessing of numerical tables. It comes as a separate package and has its own manual \href{file:pgfplotstable.pdf}{pgfplotstable.pdf}.
\end{enumerate}

\subsection{Upgrade remarks}
This release provides a lot of improvements which can be found in all detail in \texttt{ChangeLog} for interested readers. However, some attention is useful with respect to the following changes.

\subsubsection{New Optional Features}
\begin{enumerate}
	\item \PGFPlots\ 1.3 comes with user interface improvements. The technical distinction between ``behavior options'' and ``style options'' of older versions is no longer necessary (although still fully supported).

	\item \PGFPlots\ 1.3 has a new feature which allows to \emph{move axis labels tight to tick labels} automatically.

	Since this affects the spacing, it is not enabled be default.

	Use
\begin{codeexample}[code only]
\usepackage{pgfplots}
\pgfplotsset{compat=1.3}
\end{codeexample}
	in your preamble to benefit from the improved spacing\footnote{The |compat=1.3| setting is necessary for new features which move axis labels around like reversed axis. However, \PGFPlots\ will still work as it does in earlier versions, your documents will remain the same if you don't set it explicitly.}.
	Take a look at the next page for the precise definition of |compat|.

	\item \PGFPlots\ 1.3 now supports reversed axes. It is no longer necessary to use work--arounds with negative units.
\pgfkeys{/pdflinks/search key prefixes in/.add={/pgfplots/,}{}}

	Take a look at the |x dir=reverse| key.

	Existing work--arounds will still function properly. Use |\pgfplotsset{compat=1.3}| together with |x dir=reverse| to switch to the new version.
\end{enumerate}

\subsubsection{Old Features Which May Need Attention}
\begin{enumerate}
	\item The |scatter/classes| feature produces proper legends as of version 1.3. This may change the appearance of existing legends of plots with |scatter/classes|.

	\item Due to a small math bug in \PGF\ $2.00$, you \emph{can't} use the math expression `|-x^2|'. It is necessary to use `|0-x^2|' instead. The same holds for
		`|exp(-x^2)|'; use `|exp(0-x^2)|' instead.

		This will be fixed with \PGF\ $>2.00$.

	\item Starting with \PGFPlots\ $1.1$, |\tikzstyle| should \emph{no longer be used} to set \PGFPlots\ options.
	
	Although |\tikzstyle| is still supported for some older \PGFPlots\ options, you should replace any occurance of |\tikzstyle| with |\pgfplotsset{|\meta{style name}|/.style={|\meta{key-value-list}|}}| or the associated |/.append style| variant. See section~\ref{sec:styles} for more detail.
\end{enumerate}
I apologize for any inconvenience caused by these changes.

\begin{pgfplotskey}{compat=\mchoice{1.3,pre 1.3,default,newest} (initially default)}
	The preamble configuration 
\begin{codeexample}[code only]
\usepackage{pgfplots}
\pgfplotsset{compat=1.3}
\end{codeexample}
	allows to choose between backwards compatibility and most recent features.

	\PGFPlots\ version 1.3 comes with new features which may lead to different spacings compared to earlier versions:
	\begin{enumerate}
		\item Version 1.3 comes with the |xlabel near ticks| feature which places (in this case $x$) axis labels ``near'' the tick labels, taking the size of tick labels into account.
		
		Older versions used |xlabel absolute|, i.e.\ an absolute distance between the axis and axis labels (now matter how large or small tick labels are).

		The initial setting \emph{keeps backwards compatibility}.

		You are encouraged to use
\begin{codeexample}[code only]
\usepackage{pgfplots}
\pgfplotsset{compat=1.3}
\end{codeexample}
		in your preamble.
		
		\item Version 1.3 fixes a bug with |anchor=south west| which occurred mainly for reversed axes: the anchors where upside--down. This is fixed now.

		
		If you have written some work--around and want to keep it going, use

		|\pgfplotsset{compat/anchors=pre 1.3}|\pgfmanualpdflabel{/pgfplots/compat/anchors}{} 

		to disable the bug fix.
	\end{enumerate}

	Use |\pgfplotsset{compat=default}| to restore the factory settings.

	The setting |\pgfplotsset{compat=newest}| will always use features of the most recent version. This might result in small changes in the document's appearance.
\end{pgfplotskey}

\subsection{The Team}
\PGFPlots\ has been written mainly by Christian Feuers�nger with many improvements of Pascal Wolkotte and Nick Papior Andersen as a spare time project. We hope it is useful and provides valuable plots.

If you are interested in writing something but don't know how, consider reading the auxiliary manual \href{file:TeX-programming-notes.pdf}{TeX-programming-notes.pdf} which comes with \PGFPlots. It is far from complete, but maybe it is a good starting point (at least for more literature).

\subsection{Acknowledgements}
I thank God for all hours of enjoyed programming. I thank Pascal Wolkotte and Nick Papior Andersen for their programming efforts and contributions as part of the development team. I thank Stefan Tibus, who contributed the |plot shell| feature. I thank Tom Cashman for the contribution of the |reverse legend| feature. Special thanks go to Stefan Pinnow whose tests of \PGFPlots\ lead to numerous quality improvements. Furthermore, I thank Dr.~Schweitzer for many fruitful discussions and Dr.~Meine for his ideas and suggestions. Thanks as well to the many international contributors who provided feature requests or identified bugs or simply manual improvements!

Last but not least, I thank Till Tantau and Mark Wibrow for their excellent graphics (and more) package \PGF\ and \Tikz\ which is the base of \PGFPlots.

\include{pgfplots.install}%
\include{pgfplots.intro}%
\include{pgfplots.reference}%
\include{pgfplots.libs}%
\include{pgfplots.resources}%
\include{pgfplots.importexport}%
\include{pgfplots.basic.reference}%

\printindex

\bibliographystyle{abbrv} %gerapali} %gerabbrv} %gerunsrt.bst} %gerabbrv}% gerplain}
\nocite{pgfplotstable}
\nocite{programmingnotes}
\bibliography{pgfplots}
\end{document}
%
%%%%%%%%%%%%%%%%%%%%%%%%%%%%%%%%%%%%%%%%%%%%%%%%%%%%%%%%%%%%%%%%%%%%%%%%%%%%%
%
% Package pgfplots.sty documentation. 
%
% Copyright 2007/2008 by Christian Feuersaenger.
%
% This program is free software: you can redistribute it and/or modify
% it under the terms of the GNU General Public License as published by
% the Free Software Foundation, either version 3 of the License, or
% (at your option) any later version.
% 
% This program is distributed in the hope that it will be useful,
% but WITHOUT ANY WARRANTY; without even the implied warranty of
% MERCHANTABILITY or FITNESS FOR A PARTICULAR PURPOSE.  See the
% GNU General Public License for more details.
% 
% You should have received a copy of the GNU General Public License
% along with this program.  If not, see <http://www.gnu.org/licenses/>.
%
%
%%%%%%%%%%%%%%%%%%%%%%%%%%%%%%%%%%%%%%%%%%%%%%%%%%%%%%%%%%%%%%%%%%%%%%%%%%%%%
\input{pgfplots.preamble.tex}
%\RequirePackage[german,english,francais]{babel}

\pgfplotsmanualexternalexpensivetrue

%\usetikzlibrary{external}
%\tikzexternalize[prefix=figures/]{pgfplots}

\title{%
	Manual for Package \PGFPlots\\
	{\small Version \pgfplotsversion}\\
	{\small\href{http://sourceforge.net/projects/pgfplots}{http://sourceforge.net/projects/pgfplots}}}

%\includeonly{pgfplots.libs}


\begin{document}

\def\plotcoords{%
\addplot coordinates {
(5,8.312e-02)    (17,2.547e-02)   (49,7.407e-03)
(129,2.102e-03)  (321,5.874e-04)  (769,1.623e-04)
(1793,4.442e-05) (4097,1.207e-05) (9217,3.261e-06)
};

\addplot coordinates{
(7,8.472e-02)    (31,3.044e-02)    (111,1.022e-02)
(351,3.303e-03)  (1023,1.039e-03)  (2815,3.196e-04)
(7423,9.658e-05) (18943,2.873e-05) (47103,8.437e-06)
};

\addplot coordinates{
(9,7.881e-02)     (49,3.243e-02)    (209,1.232e-02)
(769,4.454e-03)   (2561,1.551e-03)  (7937,5.236e-04)
(23297,1.723e-04) (65537,5.545e-05) (178177,1.751e-05)
};

\addplot coordinates{
(11,6.887e-02)    (71,3.177e-02)     (351,1.341e-02)
(1471,5.334e-03)  (5503,2.027e-03)   (18943,7.415e-04)
(61183,2.628e-04) (187903,9.063e-05) (553983,3.053e-05)
};

\addplot coordinates{
(13,5.755e-02)     (97,2.925e-02)     (545,1.351e-02)
(2561,5.842e-03)   (10625,2.397e-03)  (40193,9.414e-04)
(141569,3.564e-04) (471041,1.308e-04) 
(1496065,4.670e-05)
};
}%
\maketitle
\begin{abstract}%
\PGFPlots\ draws high--quality function plots in normal or logarithmic scaling with a user-friendly interface directly in \TeX. The user supplies axis labels, legend entries and the plot coordinates for one or more plots and \PGFPlots\ applies axis scaling, computes any logarithms and axis ticks and draws the plots, supporting line plots, scatter plots, piecewise constant plots, bar plots, area plots, mesh-- and surface plots and some more. It is based on Till Tantau's package \PGF/\Tikz.
\end{abstract}
\tableofcontents
\section{Introduction}
This package provides tools to generate plots and labeled axes easily. It draws normal plots, logplots and semi-logplots, in two and three dimensions. Axis ticks, labels, legends (in case of multiple plots) can be added with key-value options. It can cycle through a set of predefined line/marker/color specifications. In summary, its purpose is to simplify the generation of high-quality function and/or data plots, and solving the problems of
\begin{itemize}
	\item consistency of document and font type and font size,
	\item direct use of \TeX\ math mode in axis descriptions,
	\item consistency of data and figures (no third party tool necessary),
	\item inter document consistency using preamble configurations and styles.
\end{itemize}
Although not necessary, separate |.pdf| or |.eps| graphics can be generated using the |external| library developed as part of \Tikz.

\PGFPlots\ is build completely on \Tikz/\PGF. Knowledge of \Tikz\ will simplify the work with \PGFPlots, although it is not required.

\section[About PGFPlots: Preliminaries]{About {\normalfont\PGFPlots}: Preliminaries}
This section contains information about upgrades, the team, the installation (in case you need to do it manually) and troubleshooting. You may skip it completely except for the upgrade remarks.

However, note that this library requires at least \PGF\ version $2.00$. At the time of this writing, many \TeX-distributions still contain the older \PGF\ version $1.18$, so it may be necessary to install a recent \PGF\ prior to using \PGFPlots.

\subsection{Components}
\PGFPlots\ comes with two components:
\begin{enumerate}
	\item the plotting component (which you are currently reading) and
	\item the \PGFPlotstable\ component which simplifies number formatting and postprocessing of numerical tables. It comes as a separate package and has its own manual \href{file:pgfplotstable.pdf}{pgfplotstable.pdf}.
\end{enumerate}

\subsection{Upgrade remarks}
This release provides a lot of improvements which can be found in all detail in \texttt{ChangeLog} for interested readers. However, some attention is useful with respect to the following changes.

\subsubsection{New Optional Features}
\begin{enumerate}
	\item \PGFPlots\ 1.3 comes with user interface improvements. The technical distinction between ``behavior options'' and ``style options'' of older versions is no longer necessary (although still fully supported).

	\item \PGFPlots\ 1.3 has a new feature which allows to \emph{move axis labels tight to tick labels} automatically.

	Since this affects the spacing, it is not enabled be default.

	Use
\begin{codeexample}[code only]
\usepackage{pgfplots}
\pgfplotsset{compat=1.3}
\end{codeexample}
	in your preamble to benefit from the improved spacing\footnote{The |compat=1.3| setting is necessary for new features which move axis labels around like reversed axis. However, \PGFPlots\ will still work as it does in earlier versions, your documents will remain the same if you don't set it explicitly.}.
	Take a look at the next page for the precise definition of |compat|.

	\item \PGFPlots\ 1.3 now supports reversed axes. It is no longer necessary to use work--arounds with negative units.
\pgfkeys{/pdflinks/search key prefixes in/.add={/pgfplots/,}{}}

	Take a look at the |x dir=reverse| key.

	Existing work--arounds will still function properly. Use |\pgfplotsset{compat=1.3}| together with |x dir=reverse| to switch to the new version.
\end{enumerate}

\subsubsection{Old Features Which May Need Attention}
\begin{enumerate}
	\item The |scatter/classes| feature produces proper legends as of version 1.3. This may change the appearance of existing legends of plots with |scatter/classes|.

	\item Due to a small math bug in \PGF\ $2.00$, you \emph{can't} use the math expression `|-x^2|'. It is necessary to use `|0-x^2|' instead. The same holds for
		`|exp(-x^2)|'; use `|exp(0-x^2)|' instead.

		This will be fixed with \PGF\ $>2.00$.

	\item Starting with \PGFPlots\ $1.1$, |\tikzstyle| should \emph{no longer be used} to set \PGFPlots\ options.
	
	Although |\tikzstyle| is still supported for some older \PGFPlots\ options, you should replace any occurance of |\tikzstyle| with |\pgfplotsset{|\meta{style name}|/.style={|\meta{key-value-list}|}}| or the associated |/.append style| variant. See section~\ref{sec:styles} for more detail.
\end{enumerate}
I apologize for any inconvenience caused by these changes.

\begin{pgfplotskey}{compat=\mchoice{1.3,pre 1.3,default,newest} (initially default)}
	The preamble configuration 
\begin{codeexample}[code only]
\usepackage{pgfplots}
\pgfplotsset{compat=1.3}
\end{codeexample}
	allows to choose between backwards compatibility and most recent features.

	\PGFPlots\ version 1.3 comes with new features which may lead to different spacings compared to earlier versions:
	\begin{enumerate}
		\item Version 1.3 comes with the |xlabel near ticks| feature which places (in this case $x$) axis labels ``near'' the tick labels, taking the size of tick labels into account.
		
		Older versions used |xlabel absolute|, i.e.\ an absolute distance between the axis and axis labels (now matter how large or small tick labels are).

		The initial setting \emph{keeps backwards compatibility}.

		You are encouraged to use
\begin{codeexample}[code only]
\usepackage{pgfplots}
\pgfplotsset{compat=1.3}
\end{codeexample}
		in your preamble.
		
		\item Version 1.3 fixes a bug with |anchor=south west| which occurred mainly for reversed axes: the anchors where upside--down. This is fixed now.

		
		If you have written some work--around and want to keep it going, use

		|\pgfplotsset{compat/anchors=pre 1.3}|\pgfmanualpdflabel{/pgfplots/compat/anchors}{} 

		to disable the bug fix.
	\end{enumerate}

	Use |\pgfplotsset{compat=default}| to restore the factory settings.

	The setting |\pgfplotsset{compat=newest}| will always use features of the most recent version. This might result in small changes in the document's appearance.
\end{pgfplotskey}

\subsection{The Team}
\PGFPlots\ has been written mainly by Christian Feuers�nger with many improvements of Pascal Wolkotte and Nick Papior Andersen as a spare time project. We hope it is useful and provides valuable plots.

If you are interested in writing something but don't know how, consider reading the auxiliary manual \href{file:TeX-programming-notes.pdf}{TeX-programming-notes.pdf} which comes with \PGFPlots. It is far from complete, but maybe it is a good starting point (at least for more literature).

\subsection{Acknowledgements}
I thank God for all hours of enjoyed programming. I thank Pascal Wolkotte and Nick Papior Andersen for their programming efforts and contributions as part of the development team. I thank Stefan Tibus, who contributed the |plot shell| feature. I thank Tom Cashman for the contribution of the |reverse legend| feature. Special thanks go to Stefan Pinnow whose tests of \PGFPlots\ lead to numerous quality improvements. Furthermore, I thank Dr.~Schweitzer for many fruitful discussions and Dr.~Meine for his ideas and suggestions. Thanks as well to the many international contributors who provided feature requests or identified bugs or simply manual improvements!

Last but not least, I thank Till Tantau and Mark Wibrow for their excellent graphics (and more) package \PGF\ and \Tikz\ which is the base of \PGFPlots.

\include{pgfplots.install}%
\include{pgfplots.intro}%
\include{pgfplots.reference}%
\include{pgfplots.libs}%
\include{pgfplots.resources}%
\include{pgfplots.importexport}%
\include{pgfplots.basic.reference}%

\printindex

\bibliographystyle{abbrv} %gerapali} %gerabbrv} %gerunsrt.bst} %gerabbrv}% gerplain}
\nocite{pgfplotstable}
\nocite{programmingnotes}
\bibliography{pgfplots}
\end{document}
%
\include{pgfplots.basic.reference}%

\printindex

\bibliographystyle{abbrv} %gerapali} %gerabbrv} %gerunsrt.bst} %gerabbrv}% gerplain}
\nocite{pgfplotstable}
\nocite{programmingnotes}
\bibliography{pgfplots}
\end{document}
%
%%%%%%%%%%%%%%%%%%%%%%%%%%%%%%%%%%%%%%%%%%%%%%%%%%%%%%%%%%%%%%%%%%%%%%%%%%%%%
%
% Package pgfplots.sty documentation. 
%
% Copyright 2007/2008 by Christian Feuersaenger.
%
% This program is free software: you can redistribute it and/or modify
% it under the terms of the GNU General Public License as published by
% the Free Software Foundation, either version 3 of the License, or
% (at your option) any later version.
% 
% This program is distributed in the hope that it will be useful,
% but WITHOUT ANY WARRANTY; without even the implied warranty of
% MERCHANTABILITY or FITNESS FOR A PARTICULAR PURPOSE.  See the
% GNU General Public License for more details.
% 
% You should have received a copy of the GNU General Public License
% along with this program.  If not, see <http://www.gnu.org/licenses/>.
%
%
%%%%%%%%%%%%%%%%%%%%%%%%%%%%%%%%%%%%%%%%%%%%%%%%%%%%%%%%%%%%%%%%%%%%%%%%%%%%%
%%%%%%%%%%%%%%%%%%%%%%%%%%%%%%%%%%%%%%%%%%%%%%%%%%%%%%%%%%%%%%%%%%%%%%%%%%%%%
%
% Package pgfplots.sty documentation. 
%
% Copyright 2007/2008 by Christian Feuersaenger.
%
% This program is free software: you can redistribute it and/or modify
% it under the terms of the GNU General Public License as published by
% the Free Software Foundation, either version 3 of the License, or
% (at your option) any later version.
% 
% This program is distributed in the hope that it will be useful,
% but WITHOUT ANY WARRANTY; without even the implied warranty of
% MERCHANTABILITY or FITNESS FOR A PARTICULAR PURPOSE.  See the
% GNU General Public License for more details.
% 
% You should have received a copy of the GNU General Public License
% along with this program.  If not, see <http://www.gnu.org/licenses/>.
%
%
%%%%%%%%%%%%%%%%%%%%%%%%%%%%%%%%%%%%%%%%%%%%%%%%%%%%%%%%%%%%%%%%%%%%%%%%%%%%%
\documentclass[a4paper]{ltxdoc}

\usepackage{makeidx}

% DON't let hyperref overload the format of index and glossary. 
% I want to do that on my own in the stylefiles for makeindex...
\makeatletter
\let\@old@wrindex=\@wrindex
\makeatother

\usepackage{ifpdf}
\ifpdf
	\usepackage[pdftex]{hyperref}
\else
	\usepackage[dvipdfm]{hyperref}
\fi
\hypersetup{pdfborder={0 0 0}}

\makeatletter
\let\@wrindex=\@old@wrindex
\makeatother

% Formatiere Seitennummern im Index:
\newcommand{\indexpageno}[1]{%
	{\bfseries\hyperpage{#1}}%
}


\newcommand{\R}{\mathbb{R}}
\newcommand{\N}{\mathbb{N}}
\newcommand{\Z}{\mathbb{Z}}

\long\def\COMMENTLOWLEVEL#1\ENDCOMMENT{}
\def\ENDCOMMENT{}

\usepackage{textcomp}
\usepackage{booktabs}

\usepackage{calc}
\usepackage[formats]{listings}
%\usepackage{courier} % don't use it - the '^' character can't be copy-pasted in courier

\usepackage{array}
\lstset{%
	basicstyle=\ttfamily,
	language=[LaTeX]tex, % Seems as if \lstset{language=tex} must be invoked BEFORE loading tikz!?
	tabsize=4,
	breaklines=true,
	breakindent=0pt
}

\ifpdf
	\pdfinfo {
		/Author	(Christian Feuersaenger)
	}

\else
	\def\pgfsysdriver{pgfsys-dvipdfm.def}
\fi
%\def\pgfsysdriver{pgfsys-pdftex.def}
\usepackage{pgfplots}

\ifpdf
	\usetikzlibrary{pgfplotsclickable}
\fi

%\usepackage{fp}
% ATTENTION:
% this requires pgf version NEWER than 2.00 :
%\usetikzlibrary{fixedpointarithmetic}

\usetikzlibrary{dateplot}

\usepackage[a4paper,left=2.25cm,right=2.25cm,top=2.5cm,bottom=2.5cm,nohead]{geometry}
\usepackage{amsmath,amssymb}
\usepackage{xxcolor}
\usepackage{pifont}
\usepackage[latin1]{inputenc}
\usepackage{amsmath}
\usepackage{eurosym}
\input{pgfplots-macros}

\pgfqkeys{/codeexample}{%
	every codeexample/.style={
		width=8cm,
		/pgfplots/every axis/.append style={legend style={fill=graphicbackground}}
	},
	tabsize=4,
}
\pgfplotsset{
	every axis/.append style={width=7cm},
	filter discard warning=false,
}

\usetikzlibrary{backgrounds,patterns}
% Global styles:
\tikzset{
  shape example/.style={
    color=black!30,
    draw,
    fill=yellow!30,
    line width=.5cm,
    inner xsep=2.5cm,
    inner ysep=0.5cm}
}

\newcommand{\FIXME}[1]{\textcolor{red}{(FIXME: #1)}}

% fuer endvironment 'sidewaysfigure' bspw
% \usepackage{rotating}

\newcommand\Tikz{Ti\textit kZ}
\newcommand\PGF{\textsc{pgf}}
\newcommand\PGFPlots{\textsc{pgfplots}}
\def\PGFPlotstable{\textsc{PgfplotsTable}}


\makeindex

% Fix overful hboxes automatically:
\tolerance=2000
\emergencystretch=10pt

\tikzset{prefix=gnuplot/pgfplots_} % prefix for 'plot function'

\author{%
	Christian Feuers\"anger\footnote{\url{http://wissrech.ins.uni-bonn.de/people/feuersaenger}}\\%
	Institut f\"ur Numerische Simulation\\
	Universit\"at Bonn, Germany}


%\RequirePackage[german,english,francais]{babel}

\pgfplotsmanualexternalexpensivetrue

%\usetikzlibrary{external}
%\tikzexternalize[prefix=figures/]{pgfplots}

\title{%
	Manual for Package \PGFPlots\\
	{\small Version \pgfplotsversion}\\
	{\small\href{http://sourceforge.net/projects/pgfplots}{http://sourceforge.net/projects/pgfplots}}}

%\includeonly{pgfplots.libs}


\begin{document}

\def\plotcoords{%
\addplot coordinates {
(5,8.312e-02)    (17,2.547e-02)   (49,7.407e-03)
(129,2.102e-03)  (321,5.874e-04)  (769,1.623e-04)
(1793,4.442e-05) (4097,1.207e-05) (9217,3.261e-06)
};

\addplot coordinates{
(7,8.472e-02)    (31,3.044e-02)    (111,1.022e-02)
(351,3.303e-03)  (1023,1.039e-03)  (2815,3.196e-04)
(7423,9.658e-05) (18943,2.873e-05) (47103,8.437e-06)
};

\addplot coordinates{
(9,7.881e-02)     (49,3.243e-02)    (209,1.232e-02)
(769,4.454e-03)   (2561,1.551e-03)  (7937,5.236e-04)
(23297,1.723e-04) (65537,5.545e-05) (178177,1.751e-05)
};

\addplot coordinates{
(11,6.887e-02)    (71,3.177e-02)     (351,1.341e-02)
(1471,5.334e-03)  (5503,2.027e-03)   (18943,7.415e-04)
(61183,2.628e-04) (187903,9.063e-05) (553983,3.053e-05)
};

\addplot coordinates{
(13,5.755e-02)     (97,2.925e-02)     (545,1.351e-02)
(2561,5.842e-03)   (10625,2.397e-03)  (40193,9.414e-04)
(141569,3.564e-04) (471041,1.308e-04) 
(1496065,4.670e-05)
};
}%
\maketitle
\begin{abstract}%
\PGFPlots\ draws high--quality function plots in normal or logarithmic scaling with a user-friendly interface directly in \TeX. The user supplies axis labels, legend entries and the plot coordinates for one or more plots and \PGFPlots\ applies axis scaling, computes any logarithms and axis ticks and draws the plots, supporting line plots, scatter plots, piecewise constant plots, bar plots, area plots, mesh-- and surface plots and some more. It is based on Till Tantau's package \PGF/\Tikz.
\end{abstract}
\tableofcontents
\section{Introduction}
This package provides tools to generate plots and labeled axes easily. It draws normal plots, logplots and semi-logplots, in two and three dimensions. Axis ticks, labels, legends (in case of multiple plots) can be added with key-value options. It can cycle through a set of predefined line/marker/color specifications. In summary, its purpose is to simplify the generation of high-quality function and/or data plots, and solving the problems of
\begin{itemize}
	\item consistency of document and font type and font size,
	\item direct use of \TeX\ math mode in axis descriptions,
	\item consistency of data and figures (no third party tool necessary),
	\item inter document consistency using preamble configurations and styles.
\end{itemize}
Although not necessary, separate |.pdf| or |.eps| graphics can be generated using the |external| library developed as part of \Tikz.

\PGFPlots\ is build completely on \Tikz/\PGF. Knowledge of \Tikz\ will simplify the work with \PGFPlots, although it is not required.

\section[About PGFPlots: Preliminaries]{About {\normalfont\PGFPlots}: Preliminaries}
This section contains information about upgrades, the team, the installation (in case you need to do it manually) and troubleshooting. You may skip it completely except for the upgrade remarks.

However, note that this library requires at least \PGF\ version $2.00$. At the time of this writing, many \TeX-distributions still contain the older \PGF\ version $1.18$, so it may be necessary to install a recent \PGF\ prior to using \PGFPlots.

\subsection{Components}
\PGFPlots\ comes with two components:
\begin{enumerate}
	\item the plotting component (which you are currently reading) and
	\item the \PGFPlotstable\ component which simplifies number formatting and postprocessing of numerical tables. It comes as a separate package and has its own manual \href{file:pgfplotstable.pdf}{pgfplotstable.pdf}.
\end{enumerate}

\subsection{Upgrade remarks}
This release provides a lot of improvements which can be found in all detail in \texttt{ChangeLog} for interested readers. However, some attention is useful with respect to the following changes.

\subsubsection{New Optional Features}
\begin{enumerate}
	\item \PGFPlots\ 1.3 comes with user interface improvements. The technical distinction between ``behavior options'' and ``style options'' of older versions is no longer necessary (although still fully supported).

	\item \PGFPlots\ 1.3 has a new feature which allows to \emph{move axis labels tight to tick labels} automatically.

	Since this affects the spacing, it is not enabled be default.

	Use
\begin{codeexample}[code only]
\usepackage{pgfplots}
\pgfplotsset{compat=1.3}
\end{codeexample}
	in your preamble to benefit from the improved spacing\footnote{The |compat=1.3| setting is necessary for new features which move axis labels around like reversed axis. However, \PGFPlots\ will still work as it does in earlier versions, your documents will remain the same if you don't set it explicitly.}.
	Take a look at the next page for the precise definition of |compat|.

	\item \PGFPlots\ 1.3 now supports reversed axes. It is no longer necessary to use work--arounds with negative units.
\pgfkeys{/pdflinks/search key prefixes in/.add={/pgfplots/,}{}}

	Take a look at the |x dir=reverse| key.

	Existing work--arounds will still function properly. Use |\pgfplotsset{compat=1.3}| together with |x dir=reverse| to switch to the new version.
\end{enumerate}

\subsubsection{Old Features Which May Need Attention}
\begin{enumerate}
	\item The |scatter/classes| feature produces proper legends as of version 1.3. This may change the appearance of existing legends of plots with |scatter/classes|.

	\item Due to a small math bug in \PGF\ $2.00$, you \emph{can't} use the math expression `|-x^2|'. It is necessary to use `|0-x^2|' instead. The same holds for
		`|exp(-x^2)|'; use `|exp(0-x^2)|' instead.

		This will be fixed with \PGF\ $>2.00$.

	\item Starting with \PGFPlots\ $1.1$, |\tikzstyle| should \emph{no longer be used} to set \PGFPlots\ options.
	
	Although |\tikzstyle| is still supported for some older \PGFPlots\ options, you should replace any occurance of |\tikzstyle| with |\pgfplotsset{|\meta{style name}|/.style={|\meta{key-value-list}|}}| or the associated |/.append style| variant. See section~\ref{sec:styles} for more detail.
\end{enumerate}
I apologize for any inconvenience caused by these changes.

\begin{pgfplotskey}{compat=\mchoice{1.3,pre 1.3,default,newest} (initially default)}
	The preamble configuration 
\begin{codeexample}[code only]
\usepackage{pgfplots}
\pgfplotsset{compat=1.3}
\end{codeexample}
	allows to choose between backwards compatibility and most recent features.

	\PGFPlots\ version 1.3 comes with new features which may lead to different spacings compared to earlier versions:
	\begin{enumerate}
		\item Version 1.3 comes with the |xlabel near ticks| feature which places (in this case $x$) axis labels ``near'' the tick labels, taking the size of tick labels into account.
		
		Older versions used |xlabel absolute|, i.e.\ an absolute distance between the axis and axis labels (now matter how large or small tick labels are).

		The initial setting \emph{keeps backwards compatibility}.

		You are encouraged to use
\begin{codeexample}[code only]
\usepackage{pgfplots}
\pgfplotsset{compat=1.3}
\end{codeexample}
		in your preamble.
		
		\item Version 1.3 fixes a bug with |anchor=south west| which occurred mainly for reversed axes: the anchors where upside--down. This is fixed now.

		
		If you have written some work--around and want to keep it going, use

		|\pgfplotsset{compat/anchors=pre 1.3}|\pgfmanualpdflabel{/pgfplots/compat/anchors}{} 

		to disable the bug fix.
	\end{enumerate}

	Use |\pgfplotsset{compat=default}| to restore the factory settings.

	The setting |\pgfplotsset{compat=newest}| will always use features of the most recent version. This might result in small changes in the document's appearance.
\end{pgfplotskey}

\subsection{The Team}
\PGFPlots\ has been written mainly by Christian Feuers�nger with many improvements of Pascal Wolkotte and Nick Papior Andersen as a spare time project. We hope it is useful and provides valuable plots.

If you are interested in writing something but don't know how, consider reading the auxiliary manual \href{file:TeX-programming-notes.pdf}{TeX-programming-notes.pdf} which comes with \PGFPlots. It is far from complete, but maybe it is a good starting point (at least for more literature).

\subsection{Acknowledgements}
I thank God for all hours of enjoyed programming. I thank Pascal Wolkotte and Nick Papior Andersen for their programming efforts and contributions as part of the development team. I thank Stefan Tibus, who contributed the |plot shell| feature. I thank Tom Cashman for the contribution of the |reverse legend| feature. Special thanks go to Stefan Pinnow whose tests of \PGFPlots\ lead to numerous quality improvements. Furthermore, I thank Dr.~Schweitzer for many fruitful discussions and Dr.~Meine for his ideas and suggestions. Thanks as well to the many international contributors who provided feature requests or identified bugs or simply manual improvements!

Last but not least, I thank Till Tantau and Mark Wibrow for their excellent graphics (and more) package \PGF\ and \Tikz\ which is the base of \PGFPlots.

%%%%%%%%%%%%%%%%%%%%%%%%%%%%%%%%%%%%%%%%%%%%%%%%%%%%%%%%%%%%%%%%%%%%%%%%%%%%%
%
% Package pgfplots.sty documentation. 
%
% Copyright 2007/2008 by Christian Feuersaenger.
%
% This program is free software: you can redistribute it and/or modify
% it under the terms of the GNU General Public License as published by
% the Free Software Foundation, either version 3 of the License, or
% (at your option) any later version.
% 
% This program is distributed in the hope that it will be useful,
% but WITHOUT ANY WARRANTY; without even the implied warranty of
% MERCHANTABILITY or FITNESS FOR A PARTICULAR PURPOSE.  See the
% GNU General Public License for more details.
% 
% You should have received a copy of the GNU General Public License
% along with this program.  If not, see <http://www.gnu.org/licenses/>.
%
%
%%%%%%%%%%%%%%%%%%%%%%%%%%%%%%%%%%%%%%%%%%%%%%%%%%%%%%%%%%%%%%%%%%%%%%%%%%%%%
\input{pgfplots.preamble.tex}
%\RequirePackage[german,english,francais]{babel}

\pgfplotsmanualexternalexpensivetrue

%\usetikzlibrary{external}
%\tikzexternalize[prefix=figures/]{pgfplots}

\title{%
	Manual for Package \PGFPlots\\
	{\small Version \pgfplotsversion}\\
	{\small\href{http://sourceforge.net/projects/pgfplots}{http://sourceforge.net/projects/pgfplots}}}

%\includeonly{pgfplots.libs}


\begin{document}

\def\plotcoords{%
\addplot coordinates {
(5,8.312e-02)    (17,2.547e-02)   (49,7.407e-03)
(129,2.102e-03)  (321,5.874e-04)  (769,1.623e-04)
(1793,4.442e-05) (4097,1.207e-05) (9217,3.261e-06)
};

\addplot coordinates{
(7,8.472e-02)    (31,3.044e-02)    (111,1.022e-02)
(351,3.303e-03)  (1023,1.039e-03)  (2815,3.196e-04)
(7423,9.658e-05) (18943,2.873e-05) (47103,8.437e-06)
};

\addplot coordinates{
(9,7.881e-02)     (49,3.243e-02)    (209,1.232e-02)
(769,4.454e-03)   (2561,1.551e-03)  (7937,5.236e-04)
(23297,1.723e-04) (65537,5.545e-05) (178177,1.751e-05)
};

\addplot coordinates{
(11,6.887e-02)    (71,3.177e-02)     (351,1.341e-02)
(1471,5.334e-03)  (5503,2.027e-03)   (18943,7.415e-04)
(61183,2.628e-04) (187903,9.063e-05) (553983,3.053e-05)
};

\addplot coordinates{
(13,5.755e-02)     (97,2.925e-02)     (545,1.351e-02)
(2561,5.842e-03)   (10625,2.397e-03)  (40193,9.414e-04)
(141569,3.564e-04) (471041,1.308e-04) 
(1496065,4.670e-05)
};
}%
\maketitle
\begin{abstract}%
\PGFPlots\ draws high--quality function plots in normal or logarithmic scaling with a user-friendly interface directly in \TeX. The user supplies axis labels, legend entries and the plot coordinates for one or more plots and \PGFPlots\ applies axis scaling, computes any logarithms and axis ticks and draws the plots, supporting line plots, scatter plots, piecewise constant plots, bar plots, area plots, mesh-- and surface plots and some more. It is based on Till Tantau's package \PGF/\Tikz.
\end{abstract}
\tableofcontents
\section{Introduction}
This package provides tools to generate plots and labeled axes easily. It draws normal plots, logplots and semi-logplots, in two and three dimensions. Axis ticks, labels, legends (in case of multiple plots) can be added with key-value options. It can cycle through a set of predefined line/marker/color specifications. In summary, its purpose is to simplify the generation of high-quality function and/or data plots, and solving the problems of
\begin{itemize}
	\item consistency of document and font type and font size,
	\item direct use of \TeX\ math mode in axis descriptions,
	\item consistency of data and figures (no third party tool necessary),
	\item inter document consistency using preamble configurations and styles.
\end{itemize}
Although not necessary, separate |.pdf| or |.eps| graphics can be generated using the |external| library developed as part of \Tikz.

\PGFPlots\ is build completely on \Tikz/\PGF. Knowledge of \Tikz\ will simplify the work with \PGFPlots, although it is not required.

\section[About PGFPlots: Preliminaries]{About {\normalfont\PGFPlots}: Preliminaries}
This section contains information about upgrades, the team, the installation (in case you need to do it manually) and troubleshooting. You may skip it completely except for the upgrade remarks.

However, note that this library requires at least \PGF\ version $2.00$. At the time of this writing, many \TeX-distributions still contain the older \PGF\ version $1.18$, so it may be necessary to install a recent \PGF\ prior to using \PGFPlots.

\subsection{Components}
\PGFPlots\ comes with two components:
\begin{enumerate}
	\item the plotting component (which you are currently reading) and
	\item the \PGFPlotstable\ component which simplifies number formatting and postprocessing of numerical tables. It comes as a separate package and has its own manual \href{file:pgfplotstable.pdf}{pgfplotstable.pdf}.
\end{enumerate}

\subsection{Upgrade remarks}
This release provides a lot of improvements which can be found in all detail in \texttt{ChangeLog} for interested readers. However, some attention is useful with respect to the following changes.

\subsubsection{New Optional Features}
\begin{enumerate}
	\item \PGFPlots\ 1.3 comes with user interface improvements. The technical distinction between ``behavior options'' and ``style options'' of older versions is no longer necessary (although still fully supported).

	\item \PGFPlots\ 1.3 has a new feature which allows to \emph{move axis labels tight to tick labels} automatically.

	Since this affects the spacing, it is not enabled be default.

	Use
\begin{codeexample}[code only]
\usepackage{pgfplots}
\pgfplotsset{compat=1.3}
\end{codeexample}
	in your preamble to benefit from the improved spacing\footnote{The |compat=1.3| setting is necessary for new features which move axis labels around like reversed axis. However, \PGFPlots\ will still work as it does in earlier versions, your documents will remain the same if you don't set it explicitly.}.
	Take a look at the next page for the precise definition of |compat|.

	\item \PGFPlots\ 1.3 now supports reversed axes. It is no longer necessary to use work--arounds with negative units.
\pgfkeys{/pdflinks/search key prefixes in/.add={/pgfplots/,}{}}

	Take a look at the |x dir=reverse| key.

	Existing work--arounds will still function properly. Use |\pgfplotsset{compat=1.3}| together with |x dir=reverse| to switch to the new version.
\end{enumerate}

\subsubsection{Old Features Which May Need Attention}
\begin{enumerate}
	\item The |scatter/classes| feature produces proper legends as of version 1.3. This may change the appearance of existing legends of plots with |scatter/classes|.

	\item Due to a small math bug in \PGF\ $2.00$, you \emph{can't} use the math expression `|-x^2|'. It is necessary to use `|0-x^2|' instead. The same holds for
		`|exp(-x^2)|'; use `|exp(0-x^2)|' instead.

		This will be fixed with \PGF\ $>2.00$.

	\item Starting with \PGFPlots\ $1.1$, |\tikzstyle| should \emph{no longer be used} to set \PGFPlots\ options.
	
	Although |\tikzstyle| is still supported for some older \PGFPlots\ options, you should replace any occurance of |\tikzstyle| with |\pgfplotsset{|\meta{style name}|/.style={|\meta{key-value-list}|}}| or the associated |/.append style| variant. See section~\ref{sec:styles} for more detail.
\end{enumerate}
I apologize for any inconvenience caused by these changes.

\begin{pgfplotskey}{compat=\mchoice{1.3,pre 1.3,default,newest} (initially default)}
	The preamble configuration 
\begin{codeexample}[code only]
\usepackage{pgfplots}
\pgfplotsset{compat=1.3}
\end{codeexample}
	allows to choose between backwards compatibility and most recent features.

	\PGFPlots\ version 1.3 comes with new features which may lead to different spacings compared to earlier versions:
	\begin{enumerate}
		\item Version 1.3 comes with the |xlabel near ticks| feature which places (in this case $x$) axis labels ``near'' the tick labels, taking the size of tick labels into account.
		
		Older versions used |xlabel absolute|, i.e.\ an absolute distance between the axis and axis labels (now matter how large or small tick labels are).

		The initial setting \emph{keeps backwards compatibility}.

		You are encouraged to use
\begin{codeexample}[code only]
\usepackage{pgfplots}
\pgfplotsset{compat=1.3}
\end{codeexample}
		in your preamble.
		
		\item Version 1.3 fixes a bug with |anchor=south west| which occurred mainly for reversed axes: the anchors where upside--down. This is fixed now.

		
		If you have written some work--around and want to keep it going, use

		|\pgfplotsset{compat/anchors=pre 1.3}|\pgfmanualpdflabel{/pgfplots/compat/anchors}{} 

		to disable the bug fix.
	\end{enumerate}

	Use |\pgfplotsset{compat=default}| to restore the factory settings.

	The setting |\pgfplotsset{compat=newest}| will always use features of the most recent version. This might result in small changes in the document's appearance.
\end{pgfplotskey}

\subsection{The Team}
\PGFPlots\ has been written mainly by Christian Feuers�nger with many improvements of Pascal Wolkotte and Nick Papior Andersen as a spare time project. We hope it is useful and provides valuable plots.

If you are interested in writing something but don't know how, consider reading the auxiliary manual \href{file:TeX-programming-notes.pdf}{TeX-programming-notes.pdf} which comes with \PGFPlots. It is far from complete, but maybe it is a good starting point (at least for more literature).

\subsection{Acknowledgements}
I thank God for all hours of enjoyed programming. I thank Pascal Wolkotte and Nick Papior Andersen for their programming efforts and contributions as part of the development team. I thank Stefan Tibus, who contributed the |plot shell| feature. I thank Tom Cashman for the contribution of the |reverse legend| feature. Special thanks go to Stefan Pinnow whose tests of \PGFPlots\ lead to numerous quality improvements. Furthermore, I thank Dr.~Schweitzer for many fruitful discussions and Dr.~Meine for his ideas and suggestions. Thanks as well to the many international contributors who provided feature requests or identified bugs or simply manual improvements!

Last but not least, I thank Till Tantau and Mark Wibrow for their excellent graphics (and more) package \PGF\ and \Tikz\ which is the base of \PGFPlots.

\include{pgfplots.install}%
\include{pgfplots.intro}%
\include{pgfplots.reference}%
\include{pgfplots.libs}%
\include{pgfplots.resources}%
\include{pgfplots.importexport}%
\include{pgfplots.basic.reference}%

\printindex

\bibliographystyle{abbrv} %gerapali} %gerabbrv} %gerunsrt.bst} %gerabbrv}% gerplain}
\nocite{pgfplotstable}
\nocite{programmingnotes}
\bibliography{pgfplots}
\end{document}
%
%%%%%%%%%%%%%%%%%%%%%%%%%%%%%%%%%%%%%%%%%%%%%%%%%%%%%%%%%%%%%%%%%%%%%%%%%%%%%
%
% Package pgfplots.sty documentation. 
%
% Copyright 2007/2008 by Christian Feuersaenger.
%
% This program is free software: you can redistribute it and/or modify
% it under the terms of the GNU General Public License as published by
% the Free Software Foundation, either version 3 of the License, or
% (at your option) any later version.
% 
% This program is distributed in the hope that it will be useful,
% but WITHOUT ANY WARRANTY; without even the implied warranty of
% MERCHANTABILITY or FITNESS FOR A PARTICULAR PURPOSE.  See the
% GNU General Public License for more details.
% 
% You should have received a copy of the GNU General Public License
% along with this program.  If not, see <http://www.gnu.org/licenses/>.
%
%
%%%%%%%%%%%%%%%%%%%%%%%%%%%%%%%%%%%%%%%%%%%%%%%%%%%%%%%%%%%%%%%%%%%%%%%%%%%%%
\input{pgfplots.preamble.tex}
%\RequirePackage[german,english,francais]{babel}

\pgfplotsmanualexternalexpensivetrue

%\usetikzlibrary{external}
%\tikzexternalize[prefix=figures/]{pgfplots}

\title{%
	Manual for Package \PGFPlots\\
	{\small Version \pgfplotsversion}\\
	{\small\href{http://sourceforge.net/projects/pgfplots}{http://sourceforge.net/projects/pgfplots}}}

%\includeonly{pgfplots.libs}


\begin{document}

\def\plotcoords{%
\addplot coordinates {
(5,8.312e-02)    (17,2.547e-02)   (49,7.407e-03)
(129,2.102e-03)  (321,5.874e-04)  (769,1.623e-04)
(1793,4.442e-05) (4097,1.207e-05) (9217,3.261e-06)
};

\addplot coordinates{
(7,8.472e-02)    (31,3.044e-02)    (111,1.022e-02)
(351,3.303e-03)  (1023,1.039e-03)  (2815,3.196e-04)
(7423,9.658e-05) (18943,2.873e-05) (47103,8.437e-06)
};

\addplot coordinates{
(9,7.881e-02)     (49,3.243e-02)    (209,1.232e-02)
(769,4.454e-03)   (2561,1.551e-03)  (7937,5.236e-04)
(23297,1.723e-04) (65537,5.545e-05) (178177,1.751e-05)
};

\addplot coordinates{
(11,6.887e-02)    (71,3.177e-02)     (351,1.341e-02)
(1471,5.334e-03)  (5503,2.027e-03)   (18943,7.415e-04)
(61183,2.628e-04) (187903,9.063e-05) (553983,3.053e-05)
};

\addplot coordinates{
(13,5.755e-02)     (97,2.925e-02)     (545,1.351e-02)
(2561,5.842e-03)   (10625,2.397e-03)  (40193,9.414e-04)
(141569,3.564e-04) (471041,1.308e-04) 
(1496065,4.670e-05)
};
}%
\maketitle
\begin{abstract}%
\PGFPlots\ draws high--quality function plots in normal or logarithmic scaling with a user-friendly interface directly in \TeX. The user supplies axis labels, legend entries and the plot coordinates for one or more plots and \PGFPlots\ applies axis scaling, computes any logarithms and axis ticks and draws the plots, supporting line plots, scatter plots, piecewise constant plots, bar plots, area plots, mesh-- and surface plots and some more. It is based on Till Tantau's package \PGF/\Tikz.
\end{abstract}
\tableofcontents
\section{Introduction}
This package provides tools to generate plots and labeled axes easily. It draws normal plots, logplots and semi-logplots, in two and three dimensions. Axis ticks, labels, legends (in case of multiple plots) can be added with key-value options. It can cycle through a set of predefined line/marker/color specifications. In summary, its purpose is to simplify the generation of high-quality function and/or data plots, and solving the problems of
\begin{itemize}
	\item consistency of document and font type and font size,
	\item direct use of \TeX\ math mode in axis descriptions,
	\item consistency of data and figures (no third party tool necessary),
	\item inter document consistency using preamble configurations and styles.
\end{itemize}
Although not necessary, separate |.pdf| or |.eps| graphics can be generated using the |external| library developed as part of \Tikz.

\PGFPlots\ is build completely on \Tikz/\PGF. Knowledge of \Tikz\ will simplify the work with \PGFPlots, although it is not required.

\section[About PGFPlots: Preliminaries]{About {\normalfont\PGFPlots}: Preliminaries}
This section contains information about upgrades, the team, the installation (in case you need to do it manually) and troubleshooting. You may skip it completely except for the upgrade remarks.

However, note that this library requires at least \PGF\ version $2.00$. At the time of this writing, many \TeX-distributions still contain the older \PGF\ version $1.18$, so it may be necessary to install a recent \PGF\ prior to using \PGFPlots.

\subsection{Components}
\PGFPlots\ comes with two components:
\begin{enumerate}
	\item the plotting component (which you are currently reading) and
	\item the \PGFPlotstable\ component which simplifies number formatting and postprocessing of numerical tables. It comes as a separate package and has its own manual \href{file:pgfplotstable.pdf}{pgfplotstable.pdf}.
\end{enumerate}

\subsection{Upgrade remarks}
This release provides a lot of improvements which can be found in all detail in \texttt{ChangeLog} for interested readers. However, some attention is useful with respect to the following changes.

\subsubsection{New Optional Features}
\begin{enumerate}
	\item \PGFPlots\ 1.3 comes with user interface improvements. The technical distinction between ``behavior options'' and ``style options'' of older versions is no longer necessary (although still fully supported).

	\item \PGFPlots\ 1.3 has a new feature which allows to \emph{move axis labels tight to tick labels} automatically.

	Since this affects the spacing, it is not enabled be default.

	Use
\begin{codeexample}[code only]
\usepackage{pgfplots}
\pgfplotsset{compat=1.3}
\end{codeexample}
	in your preamble to benefit from the improved spacing\footnote{The |compat=1.3| setting is necessary for new features which move axis labels around like reversed axis. However, \PGFPlots\ will still work as it does in earlier versions, your documents will remain the same if you don't set it explicitly.}.
	Take a look at the next page for the precise definition of |compat|.

	\item \PGFPlots\ 1.3 now supports reversed axes. It is no longer necessary to use work--arounds with negative units.
\pgfkeys{/pdflinks/search key prefixes in/.add={/pgfplots/,}{}}

	Take a look at the |x dir=reverse| key.

	Existing work--arounds will still function properly. Use |\pgfplotsset{compat=1.3}| together with |x dir=reverse| to switch to the new version.
\end{enumerate}

\subsubsection{Old Features Which May Need Attention}
\begin{enumerate}
	\item The |scatter/classes| feature produces proper legends as of version 1.3. This may change the appearance of existing legends of plots with |scatter/classes|.

	\item Due to a small math bug in \PGF\ $2.00$, you \emph{can't} use the math expression `|-x^2|'. It is necessary to use `|0-x^2|' instead. The same holds for
		`|exp(-x^2)|'; use `|exp(0-x^2)|' instead.

		This will be fixed with \PGF\ $>2.00$.

	\item Starting with \PGFPlots\ $1.1$, |\tikzstyle| should \emph{no longer be used} to set \PGFPlots\ options.
	
	Although |\tikzstyle| is still supported for some older \PGFPlots\ options, you should replace any occurance of |\tikzstyle| with |\pgfplotsset{|\meta{style name}|/.style={|\meta{key-value-list}|}}| or the associated |/.append style| variant. See section~\ref{sec:styles} for more detail.
\end{enumerate}
I apologize for any inconvenience caused by these changes.

\begin{pgfplotskey}{compat=\mchoice{1.3,pre 1.3,default,newest} (initially default)}
	The preamble configuration 
\begin{codeexample}[code only]
\usepackage{pgfplots}
\pgfplotsset{compat=1.3}
\end{codeexample}
	allows to choose between backwards compatibility and most recent features.

	\PGFPlots\ version 1.3 comes with new features which may lead to different spacings compared to earlier versions:
	\begin{enumerate}
		\item Version 1.3 comes with the |xlabel near ticks| feature which places (in this case $x$) axis labels ``near'' the tick labels, taking the size of tick labels into account.
		
		Older versions used |xlabel absolute|, i.e.\ an absolute distance between the axis and axis labels (now matter how large or small tick labels are).

		The initial setting \emph{keeps backwards compatibility}.

		You are encouraged to use
\begin{codeexample}[code only]
\usepackage{pgfplots}
\pgfplotsset{compat=1.3}
\end{codeexample}
		in your preamble.
		
		\item Version 1.3 fixes a bug with |anchor=south west| which occurred mainly for reversed axes: the anchors where upside--down. This is fixed now.

		
		If you have written some work--around and want to keep it going, use

		|\pgfplotsset{compat/anchors=pre 1.3}|\pgfmanualpdflabel{/pgfplots/compat/anchors}{} 

		to disable the bug fix.
	\end{enumerate}

	Use |\pgfplotsset{compat=default}| to restore the factory settings.

	The setting |\pgfplotsset{compat=newest}| will always use features of the most recent version. This might result in small changes in the document's appearance.
\end{pgfplotskey}

\subsection{The Team}
\PGFPlots\ has been written mainly by Christian Feuers�nger with many improvements of Pascal Wolkotte and Nick Papior Andersen as a spare time project. We hope it is useful and provides valuable plots.

If you are interested in writing something but don't know how, consider reading the auxiliary manual \href{file:TeX-programming-notes.pdf}{TeX-programming-notes.pdf} which comes with \PGFPlots. It is far from complete, but maybe it is a good starting point (at least for more literature).

\subsection{Acknowledgements}
I thank God for all hours of enjoyed programming. I thank Pascal Wolkotte and Nick Papior Andersen for their programming efforts and contributions as part of the development team. I thank Stefan Tibus, who contributed the |plot shell| feature. I thank Tom Cashman for the contribution of the |reverse legend| feature. Special thanks go to Stefan Pinnow whose tests of \PGFPlots\ lead to numerous quality improvements. Furthermore, I thank Dr.~Schweitzer for many fruitful discussions and Dr.~Meine for his ideas and suggestions. Thanks as well to the many international contributors who provided feature requests or identified bugs or simply manual improvements!

Last but not least, I thank Till Tantau and Mark Wibrow for their excellent graphics (and more) package \PGF\ and \Tikz\ which is the base of \PGFPlots.

\include{pgfplots.install}%
\include{pgfplots.intro}%
\include{pgfplots.reference}%
\include{pgfplots.libs}%
\include{pgfplots.resources}%
\include{pgfplots.importexport}%
\include{pgfplots.basic.reference}%

\printindex

\bibliographystyle{abbrv} %gerapali} %gerabbrv} %gerunsrt.bst} %gerabbrv}% gerplain}
\nocite{pgfplotstable}
\nocite{programmingnotes}
\bibliography{pgfplots}
\end{document}
%
%%%%%%%%%%%%%%%%%%%%%%%%%%%%%%%%%%%%%%%%%%%%%%%%%%%%%%%%%%%%%%%%%%%%%%%%%%%%%
%
% Package pgfplots.sty documentation. 
%
% Copyright 2007/2008 by Christian Feuersaenger.
%
% This program is free software: you can redistribute it and/or modify
% it under the terms of the GNU General Public License as published by
% the Free Software Foundation, either version 3 of the License, or
% (at your option) any later version.
% 
% This program is distributed in the hope that it will be useful,
% but WITHOUT ANY WARRANTY; without even the implied warranty of
% MERCHANTABILITY or FITNESS FOR A PARTICULAR PURPOSE.  See the
% GNU General Public License for more details.
% 
% You should have received a copy of the GNU General Public License
% along with this program.  If not, see <http://www.gnu.org/licenses/>.
%
%
%%%%%%%%%%%%%%%%%%%%%%%%%%%%%%%%%%%%%%%%%%%%%%%%%%%%%%%%%%%%%%%%%%%%%%%%%%%%%
\input{pgfplots.preamble.tex}
%\RequirePackage[german,english,francais]{babel}

\pgfplotsmanualexternalexpensivetrue

%\usetikzlibrary{external}
%\tikzexternalize[prefix=figures/]{pgfplots}

\title{%
	Manual for Package \PGFPlots\\
	{\small Version \pgfplotsversion}\\
	{\small\href{http://sourceforge.net/projects/pgfplots}{http://sourceforge.net/projects/pgfplots}}}

%\includeonly{pgfplots.libs}


\begin{document}

\def\plotcoords{%
\addplot coordinates {
(5,8.312e-02)    (17,2.547e-02)   (49,7.407e-03)
(129,2.102e-03)  (321,5.874e-04)  (769,1.623e-04)
(1793,4.442e-05) (4097,1.207e-05) (9217,3.261e-06)
};

\addplot coordinates{
(7,8.472e-02)    (31,3.044e-02)    (111,1.022e-02)
(351,3.303e-03)  (1023,1.039e-03)  (2815,3.196e-04)
(7423,9.658e-05) (18943,2.873e-05) (47103,8.437e-06)
};

\addplot coordinates{
(9,7.881e-02)     (49,3.243e-02)    (209,1.232e-02)
(769,4.454e-03)   (2561,1.551e-03)  (7937,5.236e-04)
(23297,1.723e-04) (65537,5.545e-05) (178177,1.751e-05)
};

\addplot coordinates{
(11,6.887e-02)    (71,3.177e-02)     (351,1.341e-02)
(1471,5.334e-03)  (5503,2.027e-03)   (18943,7.415e-04)
(61183,2.628e-04) (187903,9.063e-05) (553983,3.053e-05)
};

\addplot coordinates{
(13,5.755e-02)     (97,2.925e-02)     (545,1.351e-02)
(2561,5.842e-03)   (10625,2.397e-03)  (40193,9.414e-04)
(141569,3.564e-04) (471041,1.308e-04) 
(1496065,4.670e-05)
};
}%
\maketitle
\begin{abstract}%
\PGFPlots\ draws high--quality function plots in normal or logarithmic scaling with a user-friendly interface directly in \TeX. The user supplies axis labels, legend entries and the plot coordinates for one or more plots and \PGFPlots\ applies axis scaling, computes any logarithms and axis ticks and draws the plots, supporting line plots, scatter plots, piecewise constant plots, bar plots, area plots, mesh-- and surface plots and some more. It is based on Till Tantau's package \PGF/\Tikz.
\end{abstract}
\tableofcontents
\section{Introduction}
This package provides tools to generate plots and labeled axes easily. It draws normal plots, logplots and semi-logplots, in two and three dimensions. Axis ticks, labels, legends (in case of multiple plots) can be added with key-value options. It can cycle through a set of predefined line/marker/color specifications. In summary, its purpose is to simplify the generation of high-quality function and/or data plots, and solving the problems of
\begin{itemize}
	\item consistency of document and font type and font size,
	\item direct use of \TeX\ math mode in axis descriptions,
	\item consistency of data and figures (no third party tool necessary),
	\item inter document consistency using preamble configurations and styles.
\end{itemize}
Although not necessary, separate |.pdf| or |.eps| graphics can be generated using the |external| library developed as part of \Tikz.

\PGFPlots\ is build completely on \Tikz/\PGF. Knowledge of \Tikz\ will simplify the work with \PGFPlots, although it is not required.

\section[About PGFPlots: Preliminaries]{About {\normalfont\PGFPlots}: Preliminaries}
This section contains information about upgrades, the team, the installation (in case you need to do it manually) and troubleshooting. You may skip it completely except for the upgrade remarks.

However, note that this library requires at least \PGF\ version $2.00$. At the time of this writing, many \TeX-distributions still contain the older \PGF\ version $1.18$, so it may be necessary to install a recent \PGF\ prior to using \PGFPlots.

\subsection{Components}
\PGFPlots\ comes with two components:
\begin{enumerate}
	\item the plotting component (which you are currently reading) and
	\item the \PGFPlotstable\ component which simplifies number formatting and postprocessing of numerical tables. It comes as a separate package and has its own manual \href{file:pgfplotstable.pdf}{pgfplotstable.pdf}.
\end{enumerate}

\subsection{Upgrade remarks}
This release provides a lot of improvements which can be found in all detail in \texttt{ChangeLog} for interested readers. However, some attention is useful with respect to the following changes.

\subsubsection{New Optional Features}
\begin{enumerate}
	\item \PGFPlots\ 1.3 comes with user interface improvements. The technical distinction between ``behavior options'' and ``style options'' of older versions is no longer necessary (although still fully supported).

	\item \PGFPlots\ 1.3 has a new feature which allows to \emph{move axis labels tight to tick labels} automatically.

	Since this affects the spacing, it is not enabled be default.

	Use
\begin{codeexample}[code only]
\usepackage{pgfplots}
\pgfplotsset{compat=1.3}
\end{codeexample}
	in your preamble to benefit from the improved spacing\footnote{The |compat=1.3| setting is necessary for new features which move axis labels around like reversed axis. However, \PGFPlots\ will still work as it does in earlier versions, your documents will remain the same if you don't set it explicitly.}.
	Take a look at the next page for the precise definition of |compat|.

	\item \PGFPlots\ 1.3 now supports reversed axes. It is no longer necessary to use work--arounds with negative units.
\pgfkeys{/pdflinks/search key prefixes in/.add={/pgfplots/,}{}}

	Take a look at the |x dir=reverse| key.

	Existing work--arounds will still function properly. Use |\pgfplotsset{compat=1.3}| together with |x dir=reverse| to switch to the new version.
\end{enumerate}

\subsubsection{Old Features Which May Need Attention}
\begin{enumerate}
	\item The |scatter/classes| feature produces proper legends as of version 1.3. This may change the appearance of existing legends of plots with |scatter/classes|.

	\item Due to a small math bug in \PGF\ $2.00$, you \emph{can't} use the math expression `|-x^2|'. It is necessary to use `|0-x^2|' instead. The same holds for
		`|exp(-x^2)|'; use `|exp(0-x^2)|' instead.

		This will be fixed with \PGF\ $>2.00$.

	\item Starting with \PGFPlots\ $1.1$, |\tikzstyle| should \emph{no longer be used} to set \PGFPlots\ options.
	
	Although |\tikzstyle| is still supported for some older \PGFPlots\ options, you should replace any occurance of |\tikzstyle| with |\pgfplotsset{|\meta{style name}|/.style={|\meta{key-value-list}|}}| or the associated |/.append style| variant. See section~\ref{sec:styles} for more detail.
\end{enumerate}
I apologize for any inconvenience caused by these changes.

\begin{pgfplotskey}{compat=\mchoice{1.3,pre 1.3,default,newest} (initially default)}
	The preamble configuration 
\begin{codeexample}[code only]
\usepackage{pgfplots}
\pgfplotsset{compat=1.3}
\end{codeexample}
	allows to choose between backwards compatibility and most recent features.

	\PGFPlots\ version 1.3 comes with new features which may lead to different spacings compared to earlier versions:
	\begin{enumerate}
		\item Version 1.3 comes with the |xlabel near ticks| feature which places (in this case $x$) axis labels ``near'' the tick labels, taking the size of tick labels into account.
		
		Older versions used |xlabel absolute|, i.e.\ an absolute distance between the axis and axis labels (now matter how large or small tick labels are).

		The initial setting \emph{keeps backwards compatibility}.

		You are encouraged to use
\begin{codeexample}[code only]
\usepackage{pgfplots}
\pgfplotsset{compat=1.3}
\end{codeexample}
		in your preamble.
		
		\item Version 1.3 fixes a bug with |anchor=south west| which occurred mainly for reversed axes: the anchors where upside--down. This is fixed now.

		
		If you have written some work--around and want to keep it going, use

		|\pgfplotsset{compat/anchors=pre 1.3}|\pgfmanualpdflabel{/pgfplots/compat/anchors}{} 

		to disable the bug fix.
	\end{enumerate}

	Use |\pgfplotsset{compat=default}| to restore the factory settings.

	The setting |\pgfplotsset{compat=newest}| will always use features of the most recent version. This might result in small changes in the document's appearance.
\end{pgfplotskey}

\subsection{The Team}
\PGFPlots\ has been written mainly by Christian Feuers�nger with many improvements of Pascal Wolkotte and Nick Papior Andersen as a spare time project. We hope it is useful and provides valuable plots.

If you are interested in writing something but don't know how, consider reading the auxiliary manual \href{file:TeX-programming-notes.pdf}{TeX-programming-notes.pdf} which comes with \PGFPlots. It is far from complete, but maybe it is a good starting point (at least for more literature).

\subsection{Acknowledgements}
I thank God for all hours of enjoyed programming. I thank Pascal Wolkotte and Nick Papior Andersen for their programming efforts and contributions as part of the development team. I thank Stefan Tibus, who contributed the |plot shell| feature. I thank Tom Cashman for the contribution of the |reverse legend| feature. Special thanks go to Stefan Pinnow whose tests of \PGFPlots\ lead to numerous quality improvements. Furthermore, I thank Dr.~Schweitzer for many fruitful discussions and Dr.~Meine for his ideas and suggestions. Thanks as well to the many international contributors who provided feature requests or identified bugs or simply manual improvements!

Last but not least, I thank Till Tantau and Mark Wibrow for their excellent graphics (and more) package \PGF\ and \Tikz\ which is the base of \PGFPlots.

\include{pgfplots.install}%
\include{pgfplots.intro}%
\include{pgfplots.reference}%
\include{pgfplots.libs}%
\include{pgfplots.resources}%
\include{pgfplots.importexport}%
\include{pgfplots.basic.reference}%

\printindex

\bibliographystyle{abbrv} %gerapali} %gerabbrv} %gerunsrt.bst} %gerabbrv}% gerplain}
\nocite{pgfplotstable}
\nocite{programmingnotes}
\bibliography{pgfplots}
\end{document}
%
%%%%%%%%%%%%%%%%%%%%%%%%%%%%%%%%%%%%%%%%%%%%%%%%%%%%%%%%%%%%%%%%%%%%%%%%%%%%%
%
% Package pgfplots.sty documentation. 
%
% Copyright 2007/2008 by Christian Feuersaenger.
%
% This program is free software: you can redistribute it and/or modify
% it under the terms of the GNU General Public License as published by
% the Free Software Foundation, either version 3 of the License, or
% (at your option) any later version.
% 
% This program is distributed in the hope that it will be useful,
% but WITHOUT ANY WARRANTY; without even the implied warranty of
% MERCHANTABILITY or FITNESS FOR A PARTICULAR PURPOSE.  See the
% GNU General Public License for more details.
% 
% You should have received a copy of the GNU General Public License
% along with this program.  If not, see <http://www.gnu.org/licenses/>.
%
%
%%%%%%%%%%%%%%%%%%%%%%%%%%%%%%%%%%%%%%%%%%%%%%%%%%%%%%%%%%%%%%%%%%%%%%%%%%%%%
\input{pgfplots.preamble.tex}
%\RequirePackage[german,english,francais]{babel}

\pgfplotsmanualexternalexpensivetrue

%\usetikzlibrary{external}
%\tikzexternalize[prefix=figures/]{pgfplots}

\title{%
	Manual for Package \PGFPlots\\
	{\small Version \pgfplotsversion}\\
	{\small\href{http://sourceforge.net/projects/pgfplots}{http://sourceforge.net/projects/pgfplots}}}

%\includeonly{pgfplots.libs}


\begin{document}

\def\plotcoords{%
\addplot coordinates {
(5,8.312e-02)    (17,2.547e-02)   (49,7.407e-03)
(129,2.102e-03)  (321,5.874e-04)  (769,1.623e-04)
(1793,4.442e-05) (4097,1.207e-05) (9217,3.261e-06)
};

\addplot coordinates{
(7,8.472e-02)    (31,3.044e-02)    (111,1.022e-02)
(351,3.303e-03)  (1023,1.039e-03)  (2815,3.196e-04)
(7423,9.658e-05) (18943,2.873e-05) (47103,8.437e-06)
};

\addplot coordinates{
(9,7.881e-02)     (49,3.243e-02)    (209,1.232e-02)
(769,4.454e-03)   (2561,1.551e-03)  (7937,5.236e-04)
(23297,1.723e-04) (65537,5.545e-05) (178177,1.751e-05)
};

\addplot coordinates{
(11,6.887e-02)    (71,3.177e-02)     (351,1.341e-02)
(1471,5.334e-03)  (5503,2.027e-03)   (18943,7.415e-04)
(61183,2.628e-04) (187903,9.063e-05) (553983,3.053e-05)
};

\addplot coordinates{
(13,5.755e-02)     (97,2.925e-02)     (545,1.351e-02)
(2561,5.842e-03)   (10625,2.397e-03)  (40193,9.414e-04)
(141569,3.564e-04) (471041,1.308e-04) 
(1496065,4.670e-05)
};
}%
\maketitle
\begin{abstract}%
\PGFPlots\ draws high--quality function plots in normal or logarithmic scaling with a user-friendly interface directly in \TeX. The user supplies axis labels, legend entries and the plot coordinates for one or more plots and \PGFPlots\ applies axis scaling, computes any logarithms and axis ticks and draws the plots, supporting line plots, scatter plots, piecewise constant plots, bar plots, area plots, mesh-- and surface plots and some more. It is based on Till Tantau's package \PGF/\Tikz.
\end{abstract}
\tableofcontents
\section{Introduction}
This package provides tools to generate plots and labeled axes easily. It draws normal plots, logplots and semi-logplots, in two and three dimensions. Axis ticks, labels, legends (in case of multiple plots) can be added with key-value options. It can cycle through a set of predefined line/marker/color specifications. In summary, its purpose is to simplify the generation of high-quality function and/or data plots, and solving the problems of
\begin{itemize}
	\item consistency of document and font type and font size,
	\item direct use of \TeX\ math mode in axis descriptions,
	\item consistency of data and figures (no third party tool necessary),
	\item inter document consistency using preamble configurations and styles.
\end{itemize}
Although not necessary, separate |.pdf| or |.eps| graphics can be generated using the |external| library developed as part of \Tikz.

\PGFPlots\ is build completely on \Tikz/\PGF. Knowledge of \Tikz\ will simplify the work with \PGFPlots, although it is not required.

\section[About PGFPlots: Preliminaries]{About {\normalfont\PGFPlots}: Preliminaries}
This section contains information about upgrades, the team, the installation (in case you need to do it manually) and troubleshooting. You may skip it completely except for the upgrade remarks.

However, note that this library requires at least \PGF\ version $2.00$. At the time of this writing, many \TeX-distributions still contain the older \PGF\ version $1.18$, so it may be necessary to install a recent \PGF\ prior to using \PGFPlots.

\subsection{Components}
\PGFPlots\ comes with two components:
\begin{enumerate}
	\item the plotting component (which you are currently reading) and
	\item the \PGFPlotstable\ component which simplifies number formatting and postprocessing of numerical tables. It comes as a separate package and has its own manual \href{file:pgfplotstable.pdf}{pgfplotstable.pdf}.
\end{enumerate}

\subsection{Upgrade remarks}
This release provides a lot of improvements which can be found in all detail in \texttt{ChangeLog} for interested readers. However, some attention is useful with respect to the following changes.

\subsubsection{New Optional Features}
\begin{enumerate}
	\item \PGFPlots\ 1.3 comes with user interface improvements. The technical distinction between ``behavior options'' and ``style options'' of older versions is no longer necessary (although still fully supported).

	\item \PGFPlots\ 1.3 has a new feature which allows to \emph{move axis labels tight to tick labels} automatically.

	Since this affects the spacing, it is not enabled be default.

	Use
\begin{codeexample}[code only]
\usepackage{pgfplots}
\pgfplotsset{compat=1.3}
\end{codeexample}
	in your preamble to benefit from the improved spacing\footnote{The |compat=1.3| setting is necessary for new features which move axis labels around like reversed axis. However, \PGFPlots\ will still work as it does in earlier versions, your documents will remain the same if you don't set it explicitly.}.
	Take a look at the next page for the precise definition of |compat|.

	\item \PGFPlots\ 1.3 now supports reversed axes. It is no longer necessary to use work--arounds with negative units.
\pgfkeys{/pdflinks/search key prefixes in/.add={/pgfplots/,}{}}

	Take a look at the |x dir=reverse| key.

	Existing work--arounds will still function properly. Use |\pgfplotsset{compat=1.3}| together with |x dir=reverse| to switch to the new version.
\end{enumerate}

\subsubsection{Old Features Which May Need Attention}
\begin{enumerate}
	\item The |scatter/classes| feature produces proper legends as of version 1.3. This may change the appearance of existing legends of plots with |scatter/classes|.

	\item Due to a small math bug in \PGF\ $2.00$, you \emph{can't} use the math expression `|-x^2|'. It is necessary to use `|0-x^2|' instead. The same holds for
		`|exp(-x^2)|'; use `|exp(0-x^2)|' instead.

		This will be fixed with \PGF\ $>2.00$.

	\item Starting with \PGFPlots\ $1.1$, |\tikzstyle| should \emph{no longer be used} to set \PGFPlots\ options.
	
	Although |\tikzstyle| is still supported for some older \PGFPlots\ options, you should replace any occurance of |\tikzstyle| with |\pgfplotsset{|\meta{style name}|/.style={|\meta{key-value-list}|}}| or the associated |/.append style| variant. See section~\ref{sec:styles} for more detail.
\end{enumerate}
I apologize for any inconvenience caused by these changes.

\begin{pgfplotskey}{compat=\mchoice{1.3,pre 1.3,default,newest} (initially default)}
	The preamble configuration 
\begin{codeexample}[code only]
\usepackage{pgfplots}
\pgfplotsset{compat=1.3}
\end{codeexample}
	allows to choose between backwards compatibility and most recent features.

	\PGFPlots\ version 1.3 comes with new features which may lead to different spacings compared to earlier versions:
	\begin{enumerate}
		\item Version 1.3 comes with the |xlabel near ticks| feature which places (in this case $x$) axis labels ``near'' the tick labels, taking the size of tick labels into account.
		
		Older versions used |xlabel absolute|, i.e.\ an absolute distance between the axis and axis labels (now matter how large or small tick labels are).

		The initial setting \emph{keeps backwards compatibility}.

		You are encouraged to use
\begin{codeexample}[code only]
\usepackage{pgfplots}
\pgfplotsset{compat=1.3}
\end{codeexample}
		in your preamble.
		
		\item Version 1.3 fixes a bug with |anchor=south west| which occurred mainly for reversed axes: the anchors where upside--down. This is fixed now.

		
		If you have written some work--around and want to keep it going, use

		|\pgfplotsset{compat/anchors=pre 1.3}|\pgfmanualpdflabel{/pgfplots/compat/anchors}{} 

		to disable the bug fix.
	\end{enumerate}

	Use |\pgfplotsset{compat=default}| to restore the factory settings.

	The setting |\pgfplotsset{compat=newest}| will always use features of the most recent version. This might result in small changes in the document's appearance.
\end{pgfplotskey}

\subsection{The Team}
\PGFPlots\ has been written mainly by Christian Feuers�nger with many improvements of Pascal Wolkotte and Nick Papior Andersen as a spare time project. We hope it is useful and provides valuable plots.

If you are interested in writing something but don't know how, consider reading the auxiliary manual \href{file:TeX-programming-notes.pdf}{TeX-programming-notes.pdf} which comes with \PGFPlots. It is far from complete, but maybe it is a good starting point (at least for more literature).

\subsection{Acknowledgements}
I thank God for all hours of enjoyed programming. I thank Pascal Wolkotte and Nick Papior Andersen for their programming efforts and contributions as part of the development team. I thank Stefan Tibus, who contributed the |plot shell| feature. I thank Tom Cashman for the contribution of the |reverse legend| feature. Special thanks go to Stefan Pinnow whose tests of \PGFPlots\ lead to numerous quality improvements. Furthermore, I thank Dr.~Schweitzer for many fruitful discussions and Dr.~Meine for his ideas and suggestions. Thanks as well to the many international contributors who provided feature requests or identified bugs or simply manual improvements!

Last but not least, I thank Till Tantau and Mark Wibrow for their excellent graphics (and more) package \PGF\ and \Tikz\ which is the base of \PGFPlots.

\include{pgfplots.install}%
\include{pgfplots.intro}%
\include{pgfplots.reference}%
\include{pgfplots.libs}%
\include{pgfplots.resources}%
\include{pgfplots.importexport}%
\include{pgfplots.basic.reference}%

\printindex

\bibliographystyle{abbrv} %gerapali} %gerabbrv} %gerunsrt.bst} %gerabbrv}% gerplain}
\nocite{pgfplotstable}
\nocite{programmingnotes}
\bibliography{pgfplots}
\end{document}
%
%%%%%%%%%%%%%%%%%%%%%%%%%%%%%%%%%%%%%%%%%%%%%%%%%%%%%%%%%%%%%%%%%%%%%%%%%%%%%
%
% Package pgfplots.sty documentation. 
%
% Copyright 2007/2008 by Christian Feuersaenger.
%
% This program is free software: you can redistribute it and/or modify
% it under the terms of the GNU General Public License as published by
% the Free Software Foundation, either version 3 of the License, or
% (at your option) any later version.
% 
% This program is distributed in the hope that it will be useful,
% but WITHOUT ANY WARRANTY; without even the implied warranty of
% MERCHANTABILITY or FITNESS FOR A PARTICULAR PURPOSE.  See the
% GNU General Public License for more details.
% 
% You should have received a copy of the GNU General Public License
% along with this program.  If not, see <http://www.gnu.org/licenses/>.
%
%
%%%%%%%%%%%%%%%%%%%%%%%%%%%%%%%%%%%%%%%%%%%%%%%%%%%%%%%%%%%%%%%%%%%%%%%%%%%%%
\input{pgfplots.preamble.tex}
%\RequirePackage[german,english,francais]{babel}

\pgfplotsmanualexternalexpensivetrue

%\usetikzlibrary{external}
%\tikzexternalize[prefix=figures/]{pgfplots}

\title{%
	Manual for Package \PGFPlots\\
	{\small Version \pgfplotsversion}\\
	{\small\href{http://sourceforge.net/projects/pgfplots}{http://sourceforge.net/projects/pgfplots}}}

%\includeonly{pgfplots.libs}


\begin{document}

\def\plotcoords{%
\addplot coordinates {
(5,8.312e-02)    (17,2.547e-02)   (49,7.407e-03)
(129,2.102e-03)  (321,5.874e-04)  (769,1.623e-04)
(1793,4.442e-05) (4097,1.207e-05) (9217,3.261e-06)
};

\addplot coordinates{
(7,8.472e-02)    (31,3.044e-02)    (111,1.022e-02)
(351,3.303e-03)  (1023,1.039e-03)  (2815,3.196e-04)
(7423,9.658e-05) (18943,2.873e-05) (47103,8.437e-06)
};

\addplot coordinates{
(9,7.881e-02)     (49,3.243e-02)    (209,1.232e-02)
(769,4.454e-03)   (2561,1.551e-03)  (7937,5.236e-04)
(23297,1.723e-04) (65537,5.545e-05) (178177,1.751e-05)
};

\addplot coordinates{
(11,6.887e-02)    (71,3.177e-02)     (351,1.341e-02)
(1471,5.334e-03)  (5503,2.027e-03)   (18943,7.415e-04)
(61183,2.628e-04) (187903,9.063e-05) (553983,3.053e-05)
};

\addplot coordinates{
(13,5.755e-02)     (97,2.925e-02)     (545,1.351e-02)
(2561,5.842e-03)   (10625,2.397e-03)  (40193,9.414e-04)
(141569,3.564e-04) (471041,1.308e-04) 
(1496065,4.670e-05)
};
}%
\maketitle
\begin{abstract}%
\PGFPlots\ draws high--quality function plots in normal or logarithmic scaling with a user-friendly interface directly in \TeX. The user supplies axis labels, legend entries and the plot coordinates for one or more plots and \PGFPlots\ applies axis scaling, computes any logarithms and axis ticks and draws the plots, supporting line plots, scatter plots, piecewise constant plots, bar plots, area plots, mesh-- and surface plots and some more. It is based on Till Tantau's package \PGF/\Tikz.
\end{abstract}
\tableofcontents
\section{Introduction}
This package provides tools to generate plots and labeled axes easily. It draws normal plots, logplots and semi-logplots, in two and three dimensions. Axis ticks, labels, legends (in case of multiple plots) can be added with key-value options. It can cycle through a set of predefined line/marker/color specifications. In summary, its purpose is to simplify the generation of high-quality function and/or data plots, and solving the problems of
\begin{itemize}
	\item consistency of document and font type and font size,
	\item direct use of \TeX\ math mode in axis descriptions,
	\item consistency of data and figures (no third party tool necessary),
	\item inter document consistency using preamble configurations and styles.
\end{itemize}
Although not necessary, separate |.pdf| or |.eps| graphics can be generated using the |external| library developed as part of \Tikz.

\PGFPlots\ is build completely on \Tikz/\PGF. Knowledge of \Tikz\ will simplify the work with \PGFPlots, although it is not required.

\section[About PGFPlots: Preliminaries]{About {\normalfont\PGFPlots}: Preliminaries}
This section contains information about upgrades, the team, the installation (in case you need to do it manually) and troubleshooting. You may skip it completely except for the upgrade remarks.

However, note that this library requires at least \PGF\ version $2.00$. At the time of this writing, many \TeX-distributions still contain the older \PGF\ version $1.18$, so it may be necessary to install a recent \PGF\ prior to using \PGFPlots.

\subsection{Components}
\PGFPlots\ comes with two components:
\begin{enumerate}
	\item the plotting component (which you are currently reading) and
	\item the \PGFPlotstable\ component which simplifies number formatting and postprocessing of numerical tables. It comes as a separate package and has its own manual \href{file:pgfplotstable.pdf}{pgfplotstable.pdf}.
\end{enumerate}

\subsection{Upgrade remarks}
This release provides a lot of improvements which can be found in all detail in \texttt{ChangeLog} for interested readers. However, some attention is useful with respect to the following changes.

\subsubsection{New Optional Features}
\begin{enumerate}
	\item \PGFPlots\ 1.3 comes with user interface improvements. The technical distinction between ``behavior options'' and ``style options'' of older versions is no longer necessary (although still fully supported).

	\item \PGFPlots\ 1.3 has a new feature which allows to \emph{move axis labels tight to tick labels} automatically.

	Since this affects the spacing, it is not enabled be default.

	Use
\begin{codeexample}[code only]
\usepackage{pgfplots}
\pgfplotsset{compat=1.3}
\end{codeexample}
	in your preamble to benefit from the improved spacing\footnote{The |compat=1.3| setting is necessary for new features which move axis labels around like reversed axis. However, \PGFPlots\ will still work as it does in earlier versions, your documents will remain the same if you don't set it explicitly.}.
	Take a look at the next page for the precise definition of |compat|.

	\item \PGFPlots\ 1.3 now supports reversed axes. It is no longer necessary to use work--arounds with negative units.
\pgfkeys{/pdflinks/search key prefixes in/.add={/pgfplots/,}{}}

	Take a look at the |x dir=reverse| key.

	Existing work--arounds will still function properly. Use |\pgfplotsset{compat=1.3}| together with |x dir=reverse| to switch to the new version.
\end{enumerate}

\subsubsection{Old Features Which May Need Attention}
\begin{enumerate}
	\item The |scatter/classes| feature produces proper legends as of version 1.3. This may change the appearance of existing legends of plots with |scatter/classes|.

	\item Due to a small math bug in \PGF\ $2.00$, you \emph{can't} use the math expression `|-x^2|'. It is necessary to use `|0-x^2|' instead. The same holds for
		`|exp(-x^2)|'; use `|exp(0-x^2)|' instead.

		This will be fixed with \PGF\ $>2.00$.

	\item Starting with \PGFPlots\ $1.1$, |\tikzstyle| should \emph{no longer be used} to set \PGFPlots\ options.
	
	Although |\tikzstyle| is still supported for some older \PGFPlots\ options, you should replace any occurance of |\tikzstyle| with |\pgfplotsset{|\meta{style name}|/.style={|\meta{key-value-list}|}}| or the associated |/.append style| variant. See section~\ref{sec:styles} for more detail.
\end{enumerate}
I apologize for any inconvenience caused by these changes.

\begin{pgfplotskey}{compat=\mchoice{1.3,pre 1.3,default,newest} (initially default)}
	The preamble configuration 
\begin{codeexample}[code only]
\usepackage{pgfplots}
\pgfplotsset{compat=1.3}
\end{codeexample}
	allows to choose between backwards compatibility and most recent features.

	\PGFPlots\ version 1.3 comes with new features which may lead to different spacings compared to earlier versions:
	\begin{enumerate}
		\item Version 1.3 comes with the |xlabel near ticks| feature which places (in this case $x$) axis labels ``near'' the tick labels, taking the size of tick labels into account.
		
		Older versions used |xlabel absolute|, i.e.\ an absolute distance between the axis and axis labels (now matter how large or small tick labels are).

		The initial setting \emph{keeps backwards compatibility}.

		You are encouraged to use
\begin{codeexample}[code only]
\usepackage{pgfplots}
\pgfplotsset{compat=1.3}
\end{codeexample}
		in your preamble.
		
		\item Version 1.3 fixes a bug with |anchor=south west| which occurred mainly for reversed axes: the anchors where upside--down. This is fixed now.

		
		If you have written some work--around and want to keep it going, use

		|\pgfplotsset{compat/anchors=pre 1.3}|\pgfmanualpdflabel{/pgfplots/compat/anchors}{} 

		to disable the bug fix.
	\end{enumerate}

	Use |\pgfplotsset{compat=default}| to restore the factory settings.

	The setting |\pgfplotsset{compat=newest}| will always use features of the most recent version. This might result in small changes in the document's appearance.
\end{pgfplotskey}

\subsection{The Team}
\PGFPlots\ has been written mainly by Christian Feuers�nger with many improvements of Pascal Wolkotte and Nick Papior Andersen as a spare time project. We hope it is useful and provides valuable plots.

If you are interested in writing something but don't know how, consider reading the auxiliary manual \href{file:TeX-programming-notes.pdf}{TeX-programming-notes.pdf} which comes with \PGFPlots. It is far from complete, but maybe it is a good starting point (at least for more literature).

\subsection{Acknowledgements}
I thank God for all hours of enjoyed programming. I thank Pascal Wolkotte and Nick Papior Andersen for their programming efforts and contributions as part of the development team. I thank Stefan Tibus, who contributed the |plot shell| feature. I thank Tom Cashman for the contribution of the |reverse legend| feature. Special thanks go to Stefan Pinnow whose tests of \PGFPlots\ lead to numerous quality improvements. Furthermore, I thank Dr.~Schweitzer for many fruitful discussions and Dr.~Meine for his ideas and suggestions. Thanks as well to the many international contributors who provided feature requests or identified bugs or simply manual improvements!

Last but not least, I thank Till Tantau and Mark Wibrow for their excellent graphics (and more) package \PGF\ and \Tikz\ which is the base of \PGFPlots.

\include{pgfplots.install}%
\include{pgfplots.intro}%
\include{pgfplots.reference}%
\include{pgfplots.libs}%
\include{pgfplots.resources}%
\include{pgfplots.importexport}%
\include{pgfplots.basic.reference}%

\printindex

\bibliographystyle{abbrv} %gerapali} %gerabbrv} %gerunsrt.bst} %gerabbrv}% gerplain}
\nocite{pgfplotstable}
\nocite{programmingnotes}
\bibliography{pgfplots}
\end{document}
%
%%%%%%%%%%%%%%%%%%%%%%%%%%%%%%%%%%%%%%%%%%%%%%%%%%%%%%%%%%%%%%%%%%%%%%%%%%%%%
%
% Package pgfplots.sty documentation. 
%
% Copyright 2007/2008 by Christian Feuersaenger.
%
% This program is free software: you can redistribute it and/or modify
% it under the terms of the GNU General Public License as published by
% the Free Software Foundation, either version 3 of the License, or
% (at your option) any later version.
% 
% This program is distributed in the hope that it will be useful,
% but WITHOUT ANY WARRANTY; without even the implied warranty of
% MERCHANTABILITY or FITNESS FOR A PARTICULAR PURPOSE.  See the
% GNU General Public License for more details.
% 
% You should have received a copy of the GNU General Public License
% along with this program.  If not, see <http://www.gnu.org/licenses/>.
%
%
%%%%%%%%%%%%%%%%%%%%%%%%%%%%%%%%%%%%%%%%%%%%%%%%%%%%%%%%%%%%%%%%%%%%%%%%%%%%%
\input{pgfplots.preamble.tex}
%\RequirePackage[german,english,francais]{babel}

\pgfplotsmanualexternalexpensivetrue

%\usetikzlibrary{external}
%\tikzexternalize[prefix=figures/]{pgfplots}

\title{%
	Manual for Package \PGFPlots\\
	{\small Version \pgfplotsversion}\\
	{\small\href{http://sourceforge.net/projects/pgfplots}{http://sourceforge.net/projects/pgfplots}}}

%\includeonly{pgfplots.libs}


\begin{document}

\def\plotcoords{%
\addplot coordinates {
(5,8.312e-02)    (17,2.547e-02)   (49,7.407e-03)
(129,2.102e-03)  (321,5.874e-04)  (769,1.623e-04)
(1793,4.442e-05) (4097,1.207e-05) (9217,3.261e-06)
};

\addplot coordinates{
(7,8.472e-02)    (31,3.044e-02)    (111,1.022e-02)
(351,3.303e-03)  (1023,1.039e-03)  (2815,3.196e-04)
(7423,9.658e-05) (18943,2.873e-05) (47103,8.437e-06)
};

\addplot coordinates{
(9,7.881e-02)     (49,3.243e-02)    (209,1.232e-02)
(769,4.454e-03)   (2561,1.551e-03)  (7937,5.236e-04)
(23297,1.723e-04) (65537,5.545e-05) (178177,1.751e-05)
};

\addplot coordinates{
(11,6.887e-02)    (71,3.177e-02)     (351,1.341e-02)
(1471,5.334e-03)  (5503,2.027e-03)   (18943,7.415e-04)
(61183,2.628e-04) (187903,9.063e-05) (553983,3.053e-05)
};

\addplot coordinates{
(13,5.755e-02)     (97,2.925e-02)     (545,1.351e-02)
(2561,5.842e-03)   (10625,2.397e-03)  (40193,9.414e-04)
(141569,3.564e-04) (471041,1.308e-04) 
(1496065,4.670e-05)
};
}%
\maketitle
\begin{abstract}%
\PGFPlots\ draws high--quality function plots in normal or logarithmic scaling with a user-friendly interface directly in \TeX. The user supplies axis labels, legend entries and the plot coordinates for one or more plots and \PGFPlots\ applies axis scaling, computes any logarithms and axis ticks and draws the plots, supporting line plots, scatter plots, piecewise constant plots, bar plots, area plots, mesh-- and surface plots and some more. It is based on Till Tantau's package \PGF/\Tikz.
\end{abstract}
\tableofcontents
\section{Introduction}
This package provides tools to generate plots and labeled axes easily. It draws normal plots, logplots and semi-logplots, in two and three dimensions. Axis ticks, labels, legends (in case of multiple plots) can be added with key-value options. It can cycle through a set of predefined line/marker/color specifications. In summary, its purpose is to simplify the generation of high-quality function and/or data plots, and solving the problems of
\begin{itemize}
	\item consistency of document and font type and font size,
	\item direct use of \TeX\ math mode in axis descriptions,
	\item consistency of data and figures (no third party tool necessary),
	\item inter document consistency using preamble configurations and styles.
\end{itemize}
Although not necessary, separate |.pdf| or |.eps| graphics can be generated using the |external| library developed as part of \Tikz.

\PGFPlots\ is build completely on \Tikz/\PGF. Knowledge of \Tikz\ will simplify the work with \PGFPlots, although it is not required.

\section[About PGFPlots: Preliminaries]{About {\normalfont\PGFPlots}: Preliminaries}
This section contains information about upgrades, the team, the installation (in case you need to do it manually) and troubleshooting. You may skip it completely except for the upgrade remarks.

However, note that this library requires at least \PGF\ version $2.00$. At the time of this writing, many \TeX-distributions still contain the older \PGF\ version $1.18$, so it may be necessary to install a recent \PGF\ prior to using \PGFPlots.

\subsection{Components}
\PGFPlots\ comes with two components:
\begin{enumerate}
	\item the plotting component (which you are currently reading) and
	\item the \PGFPlotstable\ component which simplifies number formatting and postprocessing of numerical tables. It comes as a separate package and has its own manual \href{file:pgfplotstable.pdf}{pgfplotstable.pdf}.
\end{enumerate}

\subsection{Upgrade remarks}
This release provides a lot of improvements which can be found in all detail in \texttt{ChangeLog} for interested readers. However, some attention is useful with respect to the following changes.

\subsubsection{New Optional Features}
\begin{enumerate}
	\item \PGFPlots\ 1.3 comes with user interface improvements. The technical distinction between ``behavior options'' and ``style options'' of older versions is no longer necessary (although still fully supported).

	\item \PGFPlots\ 1.3 has a new feature which allows to \emph{move axis labels tight to tick labels} automatically.

	Since this affects the spacing, it is not enabled be default.

	Use
\begin{codeexample}[code only]
\usepackage{pgfplots}
\pgfplotsset{compat=1.3}
\end{codeexample}
	in your preamble to benefit from the improved spacing\footnote{The |compat=1.3| setting is necessary for new features which move axis labels around like reversed axis. However, \PGFPlots\ will still work as it does in earlier versions, your documents will remain the same if you don't set it explicitly.}.
	Take a look at the next page for the precise definition of |compat|.

	\item \PGFPlots\ 1.3 now supports reversed axes. It is no longer necessary to use work--arounds with negative units.
\pgfkeys{/pdflinks/search key prefixes in/.add={/pgfplots/,}{}}

	Take a look at the |x dir=reverse| key.

	Existing work--arounds will still function properly. Use |\pgfplotsset{compat=1.3}| together with |x dir=reverse| to switch to the new version.
\end{enumerate}

\subsubsection{Old Features Which May Need Attention}
\begin{enumerate}
	\item The |scatter/classes| feature produces proper legends as of version 1.3. This may change the appearance of existing legends of plots with |scatter/classes|.

	\item Due to a small math bug in \PGF\ $2.00$, you \emph{can't} use the math expression `|-x^2|'. It is necessary to use `|0-x^2|' instead. The same holds for
		`|exp(-x^2)|'; use `|exp(0-x^2)|' instead.

		This will be fixed with \PGF\ $>2.00$.

	\item Starting with \PGFPlots\ $1.1$, |\tikzstyle| should \emph{no longer be used} to set \PGFPlots\ options.
	
	Although |\tikzstyle| is still supported for some older \PGFPlots\ options, you should replace any occurance of |\tikzstyle| with |\pgfplotsset{|\meta{style name}|/.style={|\meta{key-value-list}|}}| or the associated |/.append style| variant. See section~\ref{sec:styles} for more detail.
\end{enumerate}
I apologize for any inconvenience caused by these changes.

\begin{pgfplotskey}{compat=\mchoice{1.3,pre 1.3,default,newest} (initially default)}
	The preamble configuration 
\begin{codeexample}[code only]
\usepackage{pgfplots}
\pgfplotsset{compat=1.3}
\end{codeexample}
	allows to choose between backwards compatibility and most recent features.

	\PGFPlots\ version 1.3 comes with new features which may lead to different spacings compared to earlier versions:
	\begin{enumerate}
		\item Version 1.3 comes with the |xlabel near ticks| feature which places (in this case $x$) axis labels ``near'' the tick labels, taking the size of tick labels into account.
		
		Older versions used |xlabel absolute|, i.e.\ an absolute distance between the axis and axis labels (now matter how large or small tick labels are).

		The initial setting \emph{keeps backwards compatibility}.

		You are encouraged to use
\begin{codeexample}[code only]
\usepackage{pgfplots}
\pgfplotsset{compat=1.3}
\end{codeexample}
		in your preamble.
		
		\item Version 1.3 fixes a bug with |anchor=south west| which occurred mainly for reversed axes: the anchors where upside--down. This is fixed now.

		
		If you have written some work--around and want to keep it going, use

		|\pgfplotsset{compat/anchors=pre 1.3}|\pgfmanualpdflabel{/pgfplots/compat/anchors}{} 

		to disable the bug fix.
	\end{enumerate}

	Use |\pgfplotsset{compat=default}| to restore the factory settings.

	The setting |\pgfplotsset{compat=newest}| will always use features of the most recent version. This might result in small changes in the document's appearance.
\end{pgfplotskey}

\subsection{The Team}
\PGFPlots\ has been written mainly by Christian Feuers�nger with many improvements of Pascal Wolkotte and Nick Papior Andersen as a spare time project. We hope it is useful and provides valuable plots.

If you are interested in writing something but don't know how, consider reading the auxiliary manual \href{file:TeX-programming-notes.pdf}{TeX-programming-notes.pdf} which comes with \PGFPlots. It is far from complete, but maybe it is a good starting point (at least for more literature).

\subsection{Acknowledgements}
I thank God for all hours of enjoyed programming. I thank Pascal Wolkotte and Nick Papior Andersen for their programming efforts and contributions as part of the development team. I thank Stefan Tibus, who contributed the |plot shell| feature. I thank Tom Cashman for the contribution of the |reverse legend| feature. Special thanks go to Stefan Pinnow whose tests of \PGFPlots\ lead to numerous quality improvements. Furthermore, I thank Dr.~Schweitzer for many fruitful discussions and Dr.~Meine for his ideas and suggestions. Thanks as well to the many international contributors who provided feature requests or identified bugs or simply manual improvements!

Last but not least, I thank Till Tantau and Mark Wibrow for their excellent graphics (and more) package \PGF\ and \Tikz\ which is the base of \PGFPlots.

\include{pgfplots.install}%
\include{pgfplots.intro}%
\include{pgfplots.reference}%
\include{pgfplots.libs}%
\include{pgfplots.resources}%
\include{pgfplots.importexport}%
\include{pgfplots.basic.reference}%

\printindex

\bibliographystyle{abbrv} %gerapali} %gerabbrv} %gerunsrt.bst} %gerabbrv}% gerplain}
\nocite{pgfplotstable}
\nocite{programmingnotes}
\bibliography{pgfplots}
\end{document}
%
\include{pgfplots.basic.reference}%

\printindex

\bibliographystyle{abbrv} %gerapali} %gerabbrv} %gerunsrt.bst} %gerabbrv}% gerplain}
\nocite{pgfplotstable}
\nocite{programmingnotes}
\bibliography{pgfplots}
\end{document}
%
\include{pgfplots.basic.reference}%

\printindex

\bibliographystyle{abbrv} %gerapali} %gerabbrv} %gerunsrt.bst} %gerabbrv}% gerplain}
\nocite{pgfplotstable}
\nocite{programmingnotes}
\bibliography{pgfplots}
\end{document}
%
%%%%%%%%%%%%%%%%%%%%%%%%%%%%%%%%%%%%%%%%%%%%%%%%%%%%%%%%%%%%%%%%%%%%%%%%%%%%%
%
% Package pgfplots.sty documentation. 
%
% Copyright 2007/2008 by Christian Feuersaenger.
%
% This program is free software: you can redistribute it and/or modify
% it under the terms of the GNU General Public License as published by
% the Free Software Foundation, either version 3 of the License, or
% (at your option) any later version.
% 
% This program is distributed in the hope that it will be useful,
% but WITHOUT ANY WARRANTY; without even the implied warranty of
% MERCHANTABILITY or FITNESS FOR A PARTICULAR PURPOSE.  See the
% GNU General Public License for more details.
% 
% You should have received a copy of the GNU General Public License
% along with this program.  If not, see <http://www.gnu.org/licenses/>.
%
%
%%%%%%%%%%%%%%%%%%%%%%%%%%%%%%%%%%%%%%%%%%%%%%%%%%%%%%%%%%%%%%%%%%%%%%%%%%%%%
%%%%%%%%%%%%%%%%%%%%%%%%%%%%%%%%%%%%%%%%%%%%%%%%%%%%%%%%%%%%%%%%%%%%%%%%%%%%%
%
% Package pgfplots.sty documentation. 
%
% Copyright 2007/2008 by Christian Feuersaenger.
%
% This program is free software: you can redistribute it and/or modify
% it under the terms of the GNU General Public License as published by
% the Free Software Foundation, either version 3 of the License, or
% (at your option) any later version.
% 
% This program is distributed in the hope that it will be useful,
% but WITHOUT ANY WARRANTY; without even the implied warranty of
% MERCHANTABILITY or FITNESS FOR A PARTICULAR PURPOSE.  See the
% GNU General Public License for more details.
% 
% You should have received a copy of the GNU General Public License
% along with this program.  If not, see <http://www.gnu.org/licenses/>.
%
%
%%%%%%%%%%%%%%%%%%%%%%%%%%%%%%%%%%%%%%%%%%%%%%%%%%%%%%%%%%%%%%%%%%%%%%%%%%%%%
\documentclass[a4paper]{ltxdoc}

\usepackage{makeidx}

% DON't let hyperref overload the format of index and glossary. 
% I want to do that on my own in the stylefiles for makeindex...
\makeatletter
\let\@old@wrindex=\@wrindex
\makeatother

\usepackage{ifpdf}
\ifpdf
	\usepackage[pdftex]{hyperref}
\else
	\usepackage[dvipdfm]{hyperref}
\fi
\hypersetup{pdfborder={0 0 0}}

\makeatletter
\let\@wrindex=\@old@wrindex
\makeatother

% Formatiere Seitennummern im Index:
\newcommand{\indexpageno}[1]{%
	{\bfseries\hyperpage{#1}}%
}


\newcommand{\R}{\mathbb{R}}
\newcommand{\N}{\mathbb{N}}
\newcommand{\Z}{\mathbb{Z}}

\long\def\COMMENTLOWLEVEL#1\ENDCOMMENT{}
\def\ENDCOMMENT{}

\usepackage{textcomp}
\usepackage{booktabs}

\usepackage{calc}
\usepackage[formats]{listings}
%\usepackage{courier} % don't use it - the '^' character can't be copy-pasted in courier

\usepackage{array}
\lstset{%
	basicstyle=\ttfamily,
	language=[LaTeX]tex, % Seems as if \lstset{language=tex} must be invoked BEFORE loading tikz!?
	tabsize=4,
	breaklines=true,
	breakindent=0pt
}

\ifpdf
	\pdfinfo {
		/Author	(Christian Feuersaenger)
	}

\else
	\def\pgfsysdriver{pgfsys-dvipdfm.def}
\fi
%\def\pgfsysdriver{pgfsys-pdftex.def}
\usepackage{pgfplots}

\ifpdf
	\usetikzlibrary{pgfplotsclickable}
\fi

%\usepackage{fp}
% ATTENTION:
% this requires pgf version NEWER than 2.00 :
%\usetikzlibrary{fixedpointarithmetic}

\usetikzlibrary{dateplot}

\usepackage[a4paper,left=2.25cm,right=2.25cm,top=2.5cm,bottom=2.5cm,nohead]{geometry}
\usepackage{amsmath,amssymb}
\usepackage{xxcolor}
\usepackage{pifont}
\usepackage[latin1]{inputenc}
\usepackage{amsmath}
\usepackage{eurosym}

\def\eps{\textsc{eps}}

\input pgfmanual-en-macros.tex

\def\pgfmanualbar{\char`\|}
\makeatletter
\newenvironment{addplotoperation}[3][]{
  \begin{pgfmanualentry}
  	{%
	\let\ltxdoc@marg=\marg
	\let\ltxdoc@oarg=\oarg
	\let\ltxdoc@parg=\parg
	\let\ltxdoc@meta=\meta
	\def\marg##1{{\normalfont\ltxdoc@marg{##1}}}%
	\def\oarg##1{{\normalfont\ltxdoc@oarg{##1}}}%
	\def\parg##1{{\normalfont\ltxdoc@parg{##1}}}%
	\def\meta##1{{\normalfont\ltxdoc@meta{##1}}}%
    \pgfmanualentryheadline{\textcolor{gray}{{\ttfamily\char`\\addplot\ }}%
      \declare{\texttt{#2}} \texttt{#3;}}%
	  \unskip
	 \nobreak
    \pgfmanualentryheadline{\textcolor{gray}{{\texttt{\char`\\addplot}\oarg{style options} \texttt{plot}\oarg{behavior options}}\ }%
      \declare{\texttt{#2}} \texttt{#3} \textcolor{gray}{\meta{trailing path commands}}\texttt{;}}%
    \def\pgfmanualtest{#1}%
    \ifx\pgfmanualtest\@empty%
      \index{#2@\protect\textcolor{gray}{\protect\texttt{plot}}\protect\texttt{ #2}}%
      \index{Plot operations!plot #2@\protect\texttt{plot #2}}%
    \fi%
	}%
    \pgfmanualbody
}
{
  \end{pgfmanualentry}
}

\newenvironment{codekey}[1]{%
  \begin{pgfmanualentry}
 	\pgfmanualentryheadline{{\ttfamily\declare{#1}\textcolor{gray}{/.code}=\marg{...}}\hfill}%
	\def\mykey{#1}%
	\def\mypath{}%
	\def\myname{}%
	\firsttimetrue%
	\decompose#1/\nil%
    \pgfmanualbody
}
{
  \end{pgfmanualentry}
}

\newenvironment{pgfplotscodekey}[1]{%
	\begin{codekey}{/pgfplots/#1}%
}
{
  \end{codekey}
}
\newenvironment{pgfplotscodetwokey}[1]{%
  \begin{pgfmanualentry}
 	\pgfmanualentryheadline{{\ttfamily\declare{/pgfplots/#1}\textcolor{gray}{/.code 2 args}=\marg{...}}\hfill}%
	\def\mykey{/pgfplots/#1}%
	\def\mypath{}%
	\def\myname{}%
	\firsttimetrue%
	\decompose/pgfplots/#1/\nil%
    \pgfmanualbody
}
{
  \end{pgfmanualentry}
}

\newenvironment{pgfplotsxycodekeylist}[1]{%
	\begingroup
	\let\oldpgfmanualentryheadline=\pgfmanualentryheadline
	\def\pgfmanualentryheadline##1{%
		\pgfmanualentryheadline@##1\pgfplots@EOI
	}%
	\def\pgfmanualentryheadline@##1\hfill##2\pgfplots@EOI{%
		\oldpgfmanualentryheadline{{\ttfamily\declare{##1}\textcolor{gray}{/.code}=\marg{...}}\hfill}%
	}
	\begin{pgfplotsxykeylist}{#1}%
}
{
	\end{pgfplotsxykeylist}
	\endgroup
}

\newenvironment{pgfplotskey}[1]{%
  \begin{key}{/pgfplots/#1}%
}
{
  \end{key}
}

\def\choicesep{$\vert$}%
\def\choicearg#1{\texttt{#1}}

\newif\iffirstchoice
\newcommand\mchoice[1]{%
	\begingroup
	\firstchoicetrue
	\foreach \mchoice@ in {#1} {%
		\iffirstchoice
			\global\firstchoicefalse
		\else
			\choicesep
		\fi
		\choicearg{\mchoice@}%
	}%
	\endgroup
}%

% \begin{xykey}{/path/\x label=value}
% \end{xykey}
%
% has same features with 'default', 'initially' etc as key environment
\newenvironment{xykey}[2][]{%
	\begin{pgfmanualentry}
    \def\extrakeytext{}
	\insertpathifneeded{#2}{#1}%
	\expandafter\pgfutil@in@\expandafter=\expandafter{\mykey}%
	\ifpgfutil@in@%
		\expandafter\xykey@eq\mykey\@nil
	\else
		\expandafter\xykey@noeq\mykey\@nil
	\fi
	\pgfmanualbody
}{%
	\end{pgfmanualentry}
}%

% \begin{xystylekey}{/path/\x label=value}
% \end{xystylekey}
%
% has same features with 'default', 'initially' etc as key environment
\newenvironment{xystylekey}[2][]{%
	\begin{pgfmanualentry}
    \def\extrakeytext{style, }
	\insertpathifneeded{#2}{#1}%
	\expandafter\pgfutil@in@\expandafter=\expandafter{\mykey}%
	\ifpgfutil@in@%
		\expandafter\xykey@eq\mykey\@nil
	\else
		\expandafter\xykey@noeq\mykey\@nil
	\fi
	\pgfmanualbody
}{%
	\end{pgfmanualentry}
}%

% \insertpathifneeded{a key}{/pgfplots} -> assign mykey={/pgfplots/a key}
% \insertpathifneeded{/tikz/a key}{/pgfplots} -> assign mykey={/tikz/a key}
%
% #1: the key
% #2: a default path (or empty)
\def\insertpathifneeded#1#2{%
	\def\insertpathifneeded@@{#2}%
	\ifx\insertpathifneeded@@\empty
		\def\mykey{#1}%
	\else
		\insertpathifneeded@#2\@nil
		\ifpgfutil@in@
			\def\mykey{#2/#1}%
		\else
			\def\mykey{#1}%
		\fi
	\fi
}%
\def\insertpathifneeded@#1#2\@nil{%
	\def\insertpathifneeded@@{#1}%
	\def\insertpathifneeded@@@{/}%
	\ifx\insertpathifneeded@@\insertpathifneeded@@@
		\pgfutil@in@true
	\else
		\pgfutil@in@false
	\fi
}%

% \begin{keylist}[default path]
% 	{/path/option 1=value,/path/option 2=value2}
% \end{keylist}
\newenvironment{keylist}[2][]{%
	\begin{pgfmanualentry}
    \def\extrakeytext{}%
	\foreach \xx in {#2} {%
		\expandafter\insertpathifneeded\expandafter{\xx}{#1}%
		\expandafter\extractkey\mykey\@nil%
	}%
	\pgfmanualbody
}{%
  \end{pgfmanualentry}
}%

\newenvironment{pgfplotskeylist}[1]{%
	\begin{keylist}[/pgfplots]{#1}%
}{%
	\end{keylist}%
}

% \begin{xykeylist}[default path]
% 	{/path/option \x1=value,/path/option \x2=value2,/path/option \x3=value}
% \end{xykeylist}
\newenvironment{xykeylist}[2][]{%
	\begin{pgfmanualentry}
    \def\extrakeytext{}
	\foreach \xx in {#2} {%
		\expandafter\insertpathifneeded\expandafter{\xx}{#1}%
		\expandafter\pgfutil@in@\expandafter=\expandafter{\mykey}%
		\ifpgfutil@in@%
			\expandafter\xykey@eq\mykey\@nil
		\else
			\expandafter\xykey@noeq\mykey\@nil
		\fi
	}%
	\pgfmanualbody
}{%
  \end{pgfmanualentry}
}%

\newif\ifxykeyfound

\def\xykey@eq#1=#2\@nil{%
	\def\x{x}%
	\xdef\mykey{#1}%
	\def\xykey@@{#1}%
	\ifx\xykey@@\mykey
		\xykeyfoundfalse
	\else
		\xykeyfoundtrue
	\fi
    \expandafter\extractkey\mykey=#2\@nil%
	\ifxykeyfound
		\def\x{y}%
		\xdef\mykey{#1}%
		\expandafter\extractkey\mykey=#2\@nil%
	\fi
}
\def\xykey@noeq#1\@nil{%
	\def\x{x}%
	\xdef\mykey{#1}%
	\def\xykey@@{#1}%
	\ifx\xykey@@\mykey
		\xykeyfoundfalse
	\else
		\xykeyfoundtrue
	\fi
    \expandafter\extractkey\mykey\@nil%
	\ifxykeyfound
		\def\x{y}%
		\xdef\mykey{#1}%
		\expandafter\extractkey\mykey\@nil%
	\fi
}

% \begin{pgfplotsxykey}{\x label=value}
% \end{pgfplotsxykey}
%
% It introduces the path /pgfplots/ automatically.
%
% has same features with 'default', 'initially' etc as key environment
\newenvironment{pgfplotsxykey}[1]{%
	\begin{xykey}[/pgfplots]{#1}%
}{%
	\end{xykey}%
}


\newenvironment{pgfplotsxykeylist}[1]{%
	\begin{xykeylist}[/pgfplots]{#1}%
}{%
	\end{xykeylist}%
}


\newenvironment{plottype}[1]{%
	\begin{key}{/tikz/#1}%
	\end{key}
  \begin{pgfmanualentry}
    \pgfmanualentryheadline{\textcolor{gray}{{\ttfamily\char`\\addplot+[\declare{#1}]}}}%
    \pgfmanualbody
}
{
  \end{pgfmanualentry}
}

\def\index@prologue{\section*{Index}\addcontentsline{toc}{section}{Index}
}

\newenvironment{pgfplotstablecolumnkey}{%
  \begin{pgfmanualentry}
 	\pgfmanualentryheadline{{\ttfamily\textcolor{gray}{/pgfplots/table/}\declare{columns/\meta{column name}}\textcolor{gray}{/.style}=\marg{key-value-list}}\hfill}%
	\pgfplotsmanualkeyindex{/pgfplots/table/columns}%
    \pgfmanualbody
}
{
  \end{pgfmanualentry}
}
\newenvironment{pgfplotstabledisplaycolumnkey}{%
  \begin{pgfmanualentry}
 	\pgfmanualentryheadline{{\ttfamily\textcolor{gray}{/pgfplots/table/}\declare{display columns/\meta{index}}\textcolor{gray}{/.style}=\marg{key-value-list}}\hfill}%
	\pgfplotsmanualkeyindex{/pgfplots/table/display columns}%
    \pgfmanualbody
}
{
  \end{pgfmanualentry}
}
\newenvironment{pgfplotstablealiaskey}{%
  \begin{pgfmanualentry}
 	\pgfmanualentryheadline{{\ttfamily\textcolor{gray}{/pgfplots/table/}\declare{alias/\meta{col name}}\textcolor{gray}{/.initial}=\marg{real col name}}\hfill}%
	\pgfplotsmanualkeyindex{/pgfplots/table/alias}%
    \pgfmanualbody
}
{
  \end{pgfmanualentry}
}


\def\pgfplotsmanualkeyindex#1{%
	\def\mypath{#1}%
	\def\myname{}%
	\firsttimetrue%
	\decompose#1/\nil%
}
\newenvironment{pgfplotstablecreateonusekey}{%
  \begin{pgfmanualentry}
 	\pgfmanualentryheadline{{\ttfamily\textcolor{gray}{/pgfplots/table/}\declare{create on use/\meta{col name}}\textcolor{gray}{/.style}=\marg{create options}}\hfill}%
	\def\mykey{/pgfplots/table/create on use}%
    \pgfmanualbody
	\pgfplotsmanualkeyindex{/pgfplots/table/create on use}%
}
{
  \end{pgfmanualentry}
}

\def\pgfplotsassertcmdkeyexists#1{%
	\pgfkeysifdefined{/pgfplots/#1/.@cmd}\relax{%
		\pgfplots@error{DOCUMENTATION ERROR: command key /pgfplots/#1 does not exist!}%
	}%
}%

{
\catcode`\ =12%
\gdef\makespaceexpandable{\def\ { }}}%

\def\pgfplotsassertXYcmdkeyexists#1{%
	{\makespaceexpandable\def\x{x}\edef\pgfplotsassertXYcmdkeyexists@tmp{#1}%
	\pgfkeysifdefined{/pgfplots/\pgfplotsassertXYcmdkeyexists@tmp/.@cmd}\relax{%
		\pgfplots@error{DOCUMENTATION ERROR: command key /pgfplots/#1 does not exist!}%
	}}%
	{\makespaceexpandable\def\x{y}\edef\pgfplotsassertXYcmdkeyexists@tmp{#1}%
	\pgfkeysifdefined{/pgfplots/\pgfplotsassertXYcmdkeyexists@tmp/.@cmd}\relax{%
		\pgfplots@error{DOCUMENTATION ERROR: command key /pgfplots/#1 does not exist!}%
	}}%
}%

\def\pgfplotsshortstylekey #1=#2\pgfeov{%
	\pgfplotsassertcmdkeyexists{#1}%
	\pgfplotsassertcmdkeyexists{#2}%
	\begin{pgfplotskey}{#1=\marg{key-value-list}}
		An abbreviation for \texttt{#2/.append style=}\marg{key-value-list}.
	\end{pgfplotskey}
}
\def\pgfplotsshortxystylekey #1=#2\pgfeov{%
	\pgfplotsassertXYcmdkeyexists{#1}%
	\pgfplotsassertXYcmdkeyexists{#2}%
	\begin{pgfplotsxykey}{#1=\marg{key-value-list}}
		An abbreviation for {\def\x{x}\texttt{#2/.append style=}}\marg{key-value-list} (or the respective style for $y$, {\def\x{y}\texttt{#2}}).
	\end{pgfplotsxykey}
}
\def\pgfplotsshortstylekeys #1,#2=#3\pgfeov{%
	\pgfplotsassertcmdkeyexists{#1}%
	\pgfplotsassertcmdkeyexists{#2}%
	\pgfplotsassertcmdkeyexists{#3}%
	\begin{pgfplotskeylist}{%
		#1=\marg{key-value-list},
		#2=\marg{key-value-list}}
		Different abbreviations for \texttt{#3/.append style=}\marg{key-value-list}.
	\end{pgfplotskeylist}
}
\def\pgfplotsshortxystylekeys #1,#2=#3\pgfeov{%
	\pgfplotsassertXYcmdkeyexists{#1}%
	\pgfplotsassertXYcmdkeyexists{#2}%
	\pgfplotsassertXYcmdkeyexists{#3}%
	\begin{pgfplotsxykeylist}{%
		#1=\marg{key-value-list},
		#2=\marg{key-value-list}}
		Different abbreviations for {\def\x{x}\texttt{#3/.append style=}}\marg{key-value-list} (or the respective style for $y$, {\def\x{y}\texttt{#3}}).
	\end{pgfplotsxykeylist}
}

%
% Creates and shows a colormap with specification '#1'.
\def\pgfplotsshowcolormapexample#1{%
	\pgfplotscreatecolormap{tempcolormap}{#1}%
	\pgfplotsshowcolormap{tempcolormap}%
}

% Shows the colormap named '#1'.
\def\pgfplotsshowcolormap#1{%
	\pgfplotscolormaptoshadingspec{#1}{8cm}\result
	\def\tempb{\pgfdeclarehorizontalshading{tempshading}{1cm}}%
	\expandafter\tempb\expandafter{\result}%
	\pgfuseshading{tempshading}%
}

\makeatother


\pgfqkeys{/codeexample}{%
	every codeexample/.style={
		width=8cm,
		/pgfplots/every axis/.append style={legend style={fill=graphicbackground}}
	},
	tabsize=4,
}
\pgfplotsset{
	every axis/.append style={width=7cm},
	filter discard warning=false,
}

\usetikzlibrary{backgrounds,patterns}
% Global styles:
\tikzset{
  shape example/.style={
    color=black!30,
    draw,
    fill=yellow!30,
    line width=.5cm,
    inner xsep=2.5cm,
    inner ysep=0.5cm}
}

\newcommand{\FIXME}[1]{\textcolor{red}{(FIXME: #1)}}

% fuer endvironment 'sidewaysfigure' bspw
% \usepackage{rotating}

\newcommand\Tikz{Ti\textit kZ}
\newcommand\PGF{\textsc{pgf}}
\newcommand\PGFPlots{\textsc{pgfplots}}
\def\PGFPlotstable{\textsc{PgfplotsTable}}


\makeindex

% Fix overful hboxes automatically:
\tolerance=2000
\emergencystretch=10pt

\tikzset{prefix=gnuplot/pgfplots_} % prefix for 'plot function'

\author{%
	Christian Feuers\"anger\footnote{\url{http://wissrech.ins.uni-bonn.de/people/feuersaenger}}\\%
	Institut f\"ur Numerische Simulation\\
	Universit\"at Bonn, Germany}


%\RequirePackage[german,english,francais]{babel}

\pgfplotsmanualexternalexpensivetrue

%\usetikzlibrary{external}
%\tikzexternalize[prefix=figures/]{pgfplots}

\title{%
	Manual for Package \PGFPlots\\
	{\small Version \pgfplotsversion}\\
	{\small\href{http://sourceforge.net/projects/pgfplots}{http://sourceforge.net/projects/pgfplots}}}

%\includeonly{pgfplots.libs}


\begin{document}

\def\plotcoords{%
\addplot coordinates {
(5,8.312e-02)    (17,2.547e-02)   (49,7.407e-03)
(129,2.102e-03)  (321,5.874e-04)  (769,1.623e-04)
(1793,4.442e-05) (4097,1.207e-05) (9217,3.261e-06)
};

\addplot coordinates{
(7,8.472e-02)    (31,3.044e-02)    (111,1.022e-02)
(351,3.303e-03)  (1023,1.039e-03)  (2815,3.196e-04)
(7423,9.658e-05) (18943,2.873e-05) (47103,8.437e-06)
};

\addplot coordinates{
(9,7.881e-02)     (49,3.243e-02)    (209,1.232e-02)
(769,4.454e-03)   (2561,1.551e-03)  (7937,5.236e-04)
(23297,1.723e-04) (65537,5.545e-05) (178177,1.751e-05)
};

\addplot coordinates{
(11,6.887e-02)    (71,3.177e-02)     (351,1.341e-02)
(1471,5.334e-03)  (5503,2.027e-03)   (18943,7.415e-04)
(61183,2.628e-04) (187903,9.063e-05) (553983,3.053e-05)
};

\addplot coordinates{
(13,5.755e-02)     (97,2.925e-02)     (545,1.351e-02)
(2561,5.842e-03)   (10625,2.397e-03)  (40193,9.414e-04)
(141569,3.564e-04) (471041,1.308e-04) 
(1496065,4.670e-05)
};
}%
\maketitle
\begin{abstract}%
\PGFPlots\ draws high--quality function plots in normal or logarithmic scaling with a user-friendly interface directly in \TeX. The user supplies axis labels, legend entries and the plot coordinates for one or more plots and \PGFPlots\ applies axis scaling, computes any logarithms and axis ticks and draws the plots, supporting line plots, scatter plots, piecewise constant plots, bar plots, area plots, mesh-- and surface plots and some more. It is based on Till Tantau's package \PGF/\Tikz.
\end{abstract}
\tableofcontents
\section{Introduction}
This package provides tools to generate plots and labeled axes easily. It draws normal plots, logplots and semi-logplots, in two and three dimensions. Axis ticks, labels, legends (in case of multiple plots) can be added with key-value options. It can cycle through a set of predefined line/marker/color specifications. In summary, its purpose is to simplify the generation of high-quality function and/or data plots, and solving the problems of
\begin{itemize}
	\item consistency of document and font type and font size,
	\item direct use of \TeX\ math mode in axis descriptions,
	\item consistency of data and figures (no third party tool necessary),
	\item inter document consistency using preamble configurations and styles.
\end{itemize}
Although not necessary, separate |.pdf| or |.eps| graphics can be generated using the |external| library developed as part of \Tikz.

\PGFPlots\ is build completely on \Tikz/\PGF. Knowledge of \Tikz\ will simplify the work with \PGFPlots, although it is not required.

\section[About PGFPlots: Preliminaries]{About {\normalfont\PGFPlots}: Preliminaries}
This section contains information about upgrades, the team, the installation (in case you need to do it manually) and troubleshooting. You may skip it completely except for the upgrade remarks.

However, note that this library requires at least \PGF\ version $2.00$. At the time of this writing, many \TeX-distributions still contain the older \PGF\ version $1.18$, so it may be necessary to install a recent \PGF\ prior to using \PGFPlots.

\subsection{Components}
\PGFPlots\ comes with two components:
\begin{enumerate}
	\item the plotting component (which you are currently reading) and
	\item the \PGFPlotstable\ component which simplifies number formatting and postprocessing of numerical tables. It comes as a separate package and has its own manual \href{file:pgfplotstable.pdf}{pgfplotstable.pdf}.
\end{enumerate}

\subsection{Upgrade remarks}
This release provides a lot of improvements which can be found in all detail in \texttt{ChangeLog} for interested readers. However, some attention is useful with respect to the following changes.

\subsubsection{New Optional Features}
\begin{enumerate}
	\item \PGFPlots\ 1.3 comes with user interface improvements. The technical distinction between ``behavior options'' and ``style options'' of older versions is no longer necessary (although still fully supported).

	\item \PGFPlots\ 1.3 has a new feature which allows to \emph{move axis labels tight to tick labels} automatically.

	Since this affects the spacing, it is not enabled be default.

	Use
\begin{codeexample}[code only]
\usepackage{pgfplots}
\pgfplotsset{compat=1.3}
\end{codeexample}
	in your preamble to benefit from the improved spacing\footnote{The |compat=1.3| setting is necessary for new features which move axis labels around like reversed axis. However, \PGFPlots\ will still work as it does in earlier versions, your documents will remain the same if you don't set it explicitly.}.
	Take a look at the next page for the precise definition of |compat|.

	\item \PGFPlots\ 1.3 now supports reversed axes. It is no longer necessary to use work--arounds with negative units.
\pgfkeys{/pdflinks/search key prefixes in/.add={/pgfplots/,}{}}

	Take a look at the |x dir=reverse| key.

	Existing work--arounds will still function properly. Use |\pgfplotsset{compat=1.3}| together with |x dir=reverse| to switch to the new version.
\end{enumerate}

\subsubsection{Old Features Which May Need Attention}
\begin{enumerate}
	\item The |scatter/classes| feature produces proper legends as of version 1.3. This may change the appearance of existing legends of plots with |scatter/classes|.

	\item Due to a small math bug in \PGF\ $2.00$, you \emph{can't} use the math expression `|-x^2|'. It is necessary to use `|0-x^2|' instead. The same holds for
		`|exp(-x^2)|'; use `|exp(0-x^2)|' instead.

		This will be fixed with \PGF\ $>2.00$.

	\item Starting with \PGFPlots\ $1.1$, |\tikzstyle| should \emph{no longer be used} to set \PGFPlots\ options.
	
	Although |\tikzstyle| is still supported for some older \PGFPlots\ options, you should replace any occurance of |\tikzstyle| with |\pgfplotsset{|\meta{style name}|/.style={|\meta{key-value-list}|}}| or the associated |/.append style| variant. See section~\ref{sec:styles} for more detail.
\end{enumerate}
I apologize for any inconvenience caused by these changes.

\begin{pgfplotskey}{compat=\mchoice{1.3,pre 1.3,default,newest} (initially default)}
	The preamble configuration 
\begin{codeexample}[code only]
\usepackage{pgfplots}
\pgfplotsset{compat=1.3}
\end{codeexample}
	allows to choose between backwards compatibility and most recent features.

	\PGFPlots\ version 1.3 comes with new features which may lead to different spacings compared to earlier versions:
	\begin{enumerate}
		\item Version 1.3 comes with the |xlabel near ticks| feature which places (in this case $x$) axis labels ``near'' the tick labels, taking the size of tick labels into account.
		
		Older versions used |xlabel absolute|, i.e.\ an absolute distance between the axis and axis labels (now matter how large or small tick labels are).

		The initial setting \emph{keeps backwards compatibility}.

		You are encouraged to use
\begin{codeexample}[code only]
\usepackage{pgfplots}
\pgfplotsset{compat=1.3}
\end{codeexample}
		in your preamble.
		
		\item Version 1.3 fixes a bug with |anchor=south west| which occurred mainly for reversed axes: the anchors where upside--down. This is fixed now.

		
		If you have written some work--around and want to keep it going, use

		|\pgfplotsset{compat/anchors=pre 1.3}|\pgfmanualpdflabel{/pgfplots/compat/anchors}{} 

		to disable the bug fix.
	\end{enumerate}

	Use |\pgfplotsset{compat=default}| to restore the factory settings.

	The setting |\pgfplotsset{compat=newest}| will always use features of the most recent version. This might result in small changes in the document's appearance.
\end{pgfplotskey}

\subsection{The Team}
\PGFPlots\ has been written mainly by Christian Feuers�nger with many improvements of Pascal Wolkotte and Nick Papior Andersen as a spare time project. We hope it is useful and provides valuable plots.

If you are interested in writing something but don't know how, consider reading the auxiliary manual \href{file:TeX-programming-notes.pdf}{TeX-programming-notes.pdf} which comes with \PGFPlots. It is far from complete, but maybe it is a good starting point (at least for more literature).

\subsection{Acknowledgements}
I thank God for all hours of enjoyed programming. I thank Pascal Wolkotte and Nick Papior Andersen for their programming efforts and contributions as part of the development team. I thank Stefan Tibus, who contributed the |plot shell| feature. I thank Tom Cashman for the contribution of the |reverse legend| feature. Special thanks go to Stefan Pinnow whose tests of \PGFPlots\ lead to numerous quality improvements. Furthermore, I thank Dr.~Schweitzer for many fruitful discussions and Dr.~Meine for his ideas and suggestions. Thanks as well to the many international contributors who provided feature requests or identified bugs or simply manual improvements!

Last but not least, I thank Till Tantau and Mark Wibrow for their excellent graphics (and more) package \PGF\ and \Tikz\ which is the base of \PGFPlots.

%%%%%%%%%%%%%%%%%%%%%%%%%%%%%%%%%%%%%%%%%%%%%%%%%%%%%%%%%%%%%%%%%%%%%%%%%%%%%
%
% Package pgfplots.sty documentation. 
%
% Copyright 2007/2008 by Christian Feuersaenger.
%
% This program is free software: you can redistribute it and/or modify
% it under the terms of the GNU General Public License as published by
% the Free Software Foundation, either version 3 of the License, or
% (at your option) any later version.
% 
% This program is distributed in the hope that it will be useful,
% but WITHOUT ANY WARRANTY; without even the implied warranty of
% MERCHANTABILITY or FITNESS FOR A PARTICULAR PURPOSE.  See the
% GNU General Public License for more details.
% 
% You should have received a copy of the GNU General Public License
% along with this program.  If not, see <http://www.gnu.org/licenses/>.
%
%
%%%%%%%%%%%%%%%%%%%%%%%%%%%%%%%%%%%%%%%%%%%%%%%%%%%%%%%%%%%%%%%%%%%%%%%%%%%%%
%%%%%%%%%%%%%%%%%%%%%%%%%%%%%%%%%%%%%%%%%%%%%%%%%%%%%%%%%%%%%%%%%%%%%%%%%%%%%
%
% Package pgfplots.sty documentation. 
%
% Copyright 2007/2008 by Christian Feuersaenger.
%
% This program is free software: you can redistribute it and/or modify
% it under the terms of the GNU General Public License as published by
% the Free Software Foundation, either version 3 of the License, or
% (at your option) any later version.
% 
% This program is distributed in the hope that it will be useful,
% but WITHOUT ANY WARRANTY; without even the implied warranty of
% MERCHANTABILITY or FITNESS FOR A PARTICULAR PURPOSE.  See the
% GNU General Public License for more details.
% 
% You should have received a copy of the GNU General Public License
% along with this program.  If not, see <http://www.gnu.org/licenses/>.
%
%
%%%%%%%%%%%%%%%%%%%%%%%%%%%%%%%%%%%%%%%%%%%%%%%%%%%%%%%%%%%%%%%%%%%%%%%%%%%%%
\documentclass[a4paper]{ltxdoc}

\usepackage{makeidx}

% DON't let hyperref overload the format of index and glossary. 
% I want to do that on my own in the stylefiles for makeindex...
\makeatletter
\let\@old@wrindex=\@wrindex
\makeatother

\usepackage{ifpdf}
\ifpdf
	\usepackage[pdftex]{hyperref}
\else
	\usepackage[dvipdfm]{hyperref}
\fi
\hypersetup{pdfborder={0 0 0}}

\makeatletter
\let\@wrindex=\@old@wrindex
\makeatother

% Formatiere Seitennummern im Index:
\newcommand{\indexpageno}[1]{%
	{\bfseries\hyperpage{#1}}%
}


\newcommand{\R}{\mathbb{R}}
\newcommand{\N}{\mathbb{N}}
\newcommand{\Z}{\mathbb{Z}}

\long\def\COMMENTLOWLEVEL#1\ENDCOMMENT{}
\def\ENDCOMMENT{}

\usepackage{textcomp}
\usepackage{booktabs}

\usepackage{calc}
\usepackage[formats]{listings}
%\usepackage{courier} % don't use it - the '^' character can't be copy-pasted in courier

\usepackage{array}
\lstset{%
	basicstyle=\ttfamily,
	language=[LaTeX]tex, % Seems as if \lstset{language=tex} must be invoked BEFORE loading tikz!?
	tabsize=4,
	breaklines=true,
	breakindent=0pt
}

\ifpdf
	\pdfinfo {
		/Author	(Christian Feuersaenger)
	}

\else
	\def\pgfsysdriver{pgfsys-dvipdfm.def}
\fi
%\def\pgfsysdriver{pgfsys-pdftex.def}
\usepackage{pgfplots}

\ifpdf
	\usetikzlibrary{pgfplotsclickable}
\fi

%\usepackage{fp}
% ATTENTION:
% this requires pgf version NEWER than 2.00 :
%\usetikzlibrary{fixedpointarithmetic}

\usetikzlibrary{dateplot}

\usepackage[a4paper,left=2.25cm,right=2.25cm,top=2.5cm,bottom=2.5cm,nohead]{geometry}
\usepackage{amsmath,amssymb}
\usepackage{xxcolor}
\usepackage{pifont}
\usepackage[latin1]{inputenc}
\usepackage{amsmath}
\usepackage{eurosym}
\input{pgfplots-macros}

\pgfqkeys{/codeexample}{%
	every codeexample/.style={
		width=8cm,
		/pgfplots/every axis/.append style={legend style={fill=graphicbackground}}
	},
	tabsize=4,
}
\pgfplotsset{
	every axis/.append style={width=7cm},
	filter discard warning=false,
}

\usetikzlibrary{backgrounds,patterns}
% Global styles:
\tikzset{
  shape example/.style={
    color=black!30,
    draw,
    fill=yellow!30,
    line width=.5cm,
    inner xsep=2.5cm,
    inner ysep=0.5cm}
}

\newcommand{\FIXME}[1]{\textcolor{red}{(FIXME: #1)}}

% fuer endvironment 'sidewaysfigure' bspw
% \usepackage{rotating}

\newcommand\Tikz{Ti\textit kZ}
\newcommand\PGF{\textsc{pgf}}
\newcommand\PGFPlots{\textsc{pgfplots}}
\def\PGFPlotstable{\textsc{PgfplotsTable}}


\makeindex

% Fix overful hboxes automatically:
\tolerance=2000
\emergencystretch=10pt

\tikzset{prefix=gnuplot/pgfplots_} % prefix for 'plot function'

\author{%
	Christian Feuers\"anger\footnote{\url{http://wissrech.ins.uni-bonn.de/people/feuersaenger}}\\%
	Institut f\"ur Numerische Simulation\\
	Universit\"at Bonn, Germany}


%\RequirePackage[german,english,francais]{babel}

\pgfplotsmanualexternalexpensivetrue

%\usetikzlibrary{external}
%\tikzexternalize[prefix=figures/]{pgfplots}

\title{%
	Manual for Package \PGFPlots\\
	{\small Version \pgfplotsversion}\\
	{\small\href{http://sourceforge.net/projects/pgfplots}{http://sourceforge.net/projects/pgfplots}}}

%\includeonly{pgfplots.libs}


\begin{document}

\def\plotcoords{%
\addplot coordinates {
(5,8.312e-02)    (17,2.547e-02)   (49,7.407e-03)
(129,2.102e-03)  (321,5.874e-04)  (769,1.623e-04)
(1793,4.442e-05) (4097,1.207e-05) (9217,3.261e-06)
};

\addplot coordinates{
(7,8.472e-02)    (31,3.044e-02)    (111,1.022e-02)
(351,3.303e-03)  (1023,1.039e-03)  (2815,3.196e-04)
(7423,9.658e-05) (18943,2.873e-05) (47103,8.437e-06)
};

\addplot coordinates{
(9,7.881e-02)     (49,3.243e-02)    (209,1.232e-02)
(769,4.454e-03)   (2561,1.551e-03)  (7937,5.236e-04)
(23297,1.723e-04) (65537,5.545e-05) (178177,1.751e-05)
};

\addplot coordinates{
(11,6.887e-02)    (71,3.177e-02)     (351,1.341e-02)
(1471,5.334e-03)  (5503,2.027e-03)   (18943,7.415e-04)
(61183,2.628e-04) (187903,9.063e-05) (553983,3.053e-05)
};

\addplot coordinates{
(13,5.755e-02)     (97,2.925e-02)     (545,1.351e-02)
(2561,5.842e-03)   (10625,2.397e-03)  (40193,9.414e-04)
(141569,3.564e-04) (471041,1.308e-04) 
(1496065,4.670e-05)
};
}%
\maketitle
\begin{abstract}%
\PGFPlots\ draws high--quality function plots in normal or logarithmic scaling with a user-friendly interface directly in \TeX. The user supplies axis labels, legend entries and the plot coordinates for one or more plots and \PGFPlots\ applies axis scaling, computes any logarithms and axis ticks and draws the plots, supporting line plots, scatter plots, piecewise constant plots, bar plots, area plots, mesh-- and surface plots and some more. It is based on Till Tantau's package \PGF/\Tikz.
\end{abstract}
\tableofcontents
\section{Introduction}
This package provides tools to generate plots and labeled axes easily. It draws normal plots, logplots and semi-logplots, in two and three dimensions. Axis ticks, labels, legends (in case of multiple plots) can be added with key-value options. It can cycle through a set of predefined line/marker/color specifications. In summary, its purpose is to simplify the generation of high-quality function and/or data plots, and solving the problems of
\begin{itemize}
	\item consistency of document and font type and font size,
	\item direct use of \TeX\ math mode in axis descriptions,
	\item consistency of data and figures (no third party tool necessary),
	\item inter document consistency using preamble configurations and styles.
\end{itemize}
Although not necessary, separate |.pdf| or |.eps| graphics can be generated using the |external| library developed as part of \Tikz.

\PGFPlots\ is build completely on \Tikz/\PGF. Knowledge of \Tikz\ will simplify the work with \PGFPlots, although it is not required.

\section[About PGFPlots: Preliminaries]{About {\normalfont\PGFPlots}: Preliminaries}
This section contains information about upgrades, the team, the installation (in case you need to do it manually) and troubleshooting. You may skip it completely except for the upgrade remarks.

However, note that this library requires at least \PGF\ version $2.00$. At the time of this writing, many \TeX-distributions still contain the older \PGF\ version $1.18$, so it may be necessary to install a recent \PGF\ prior to using \PGFPlots.

\subsection{Components}
\PGFPlots\ comes with two components:
\begin{enumerate}
	\item the plotting component (which you are currently reading) and
	\item the \PGFPlotstable\ component which simplifies number formatting and postprocessing of numerical tables. It comes as a separate package and has its own manual \href{file:pgfplotstable.pdf}{pgfplotstable.pdf}.
\end{enumerate}

\subsection{Upgrade remarks}
This release provides a lot of improvements which can be found in all detail in \texttt{ChangeLog} for interested readers. However, some attention is useful with respect to the following changes.

\subsubsection{New Optional Features}
\begin{enumerate}
	\item \PGFPlots\ 1.3 comes with user interface improvements. The technical distinction between ``behavior options'' and ``style options'' of older versions is no longer necessary (although still fully supported).

	\item \PGFPlots\ 1.3 has a new feature which allows to \emph{move axis labels tight to tick labels} automatically.

	Since this affects the spacing, it is not enabled be default.

	Use
\begin{codeexample}[code only]
\usepackage{pgfplots}
\pgfplotsset{compat=1.3}
\end{codeexample}
	in your preamble to benefit from the improved spacing\footnote{The |compat=1.3| setting is necessary for new features which move axis labels around like reversed axis. However, \PGFPlots\ will still work as it does in earlier versions, your documents will remain the same if you don't set it explicitly.}.
	Take a look at the next page for the precise definition of |compat|.

	\item \PGFPlots\ 1.3 now supports reversed axes. It is no longer necessary to use work--arounds with negative units.
\pgfkeys{/pdflinks/search key prefixes in/.add={/pgfplots/,}{}}

	Take a look at the |x dir=reverse| key.

	Existing work--arounds will still function properly. Use |\pgfplotsset{compat=1.3}| together with |x dir=reverse| to switch to the new version.
\end{enumerate}

\subsubsection{Old Features Which May Need Attention}
\begin{enumerate}
	\item The |scatter/classes| feature produces proper legends as of version 1.3. This may change the appearance of existing legends of plots with |scatter/classes|.

	\item Due to a small math bug in \PGF\ $2.00$, you \emph{can't} use the math expression `|-x^2|'. It is necessary to use `|0-x^2|' instead. The same holds for
		`|exp(-x^2)|'; use `|exp(0-x^2)|' instead.

		This will be fixed with \PGF\ $>2.00$.

	\item Starting with \PGFPlots\ $1.1$, |\tikzstyle| should \emph{no longer be used} to set \PGFPlots\ options.
	
	Although |\tikzstyle| is still supported for some older \PGFPlots\ options, you should replace any occurance of |\tikzstyle| with |\pgfplotsset{|\meta{style name}|/.style={|\meta{key-value-list}|}}| or the associated |/.append style| variant. See section~\ref{sec:styles} for more detail.
\end{enumerate}
I apologize for any inconvenience caused by these changes.

\begin{pgfplotskey}{compat=\mchoice{1.3,pre 1.3,default,newest} (initially default)}
	The preamble configuration 
\begin{codeexample}[code only]
\usepackage{pgfplots}
\pgfplotsset{compat=1.3}
\end{codeexample}
	allows to choose between backwards compatibility and most recent features.

	\PGFPlots\ version 1.3 comes with new features which may lead to different spacings compared to earlier versions:
	\begin{enumerate}
		\item Version 1.3 comes with the |xlabel near ticks| feature which places (in this case $x$) axis labels ``near'' the tick labels, taking the size of tick labels into account.
		
		Older versions used |xlabel absolute|, i.e.\ an absolute distance between the axis and axis labels (now matter how large or small tick labels are).

		The initial setting \emph{keeps backwards compatibility}.

		You are encouraged to use
\begin{codeexample}[code only]
\usepackage{pgfplots}
\pgfplotsset{compat=1.3}
\end{codeexample}
		in your preamble.
		
		\item Version 1.3 fixes a bug with |anchor=south west| which occurred mainly for reversed axes: the anchors where upside--down. This is fixed now.

		
		If you have written some work--around and want to keep it going, use

		|\pgfplotsset{compat/anchors=pre 1.3}|\pgfmanualpdflabel{/pgfplots/compat/anchors}{} 

		to disable the bug fix.
	\end{enumerate}

	Use |\pgfplotsset{compat=default}| to restore the factory settings.

	The setting |\pgfplotsset{compat=newest}| will always use features of the most recent version. This might result in small changes in the document's appearance.
\end{pgfplotskey}

\subsection{The Team}
\PGFPlots\ has been written mainly by Christian Feuers�nger with many improvements of Pascal Wolkotte and Nick Papior Andersen as a spare time project. We hope it is useful and provides valuable plots.

If you are interested in writing something but don't know how, consider reading the auxiliary manual \href{file:TeX-programming-notes.pdf}{TeX-programming-notes.pdf} which comes with \PGFPlots. It is far from complete, but maybe it is a good starting point (at least for more literature).

\subsection{Acknowledgements}
I thank God for all hours of enjoyed programming. I thank Pascal Wolkotte and Nick Papior Andersen for their programming efforts and contributions as part of the development team. I thank Stefan Tibus, who contributed the |plot shell| feature. I thank Tom Cashman for the contribution of the |reverse legend| feature. Special thanks go to Stefan Pinnow whose tests of \PGFPlots\ lead to numerous quality improvements. Furthermore, I thank Dr.~Schweitzer for many fruitful discussions and Dr.~Meine for his ideas and suggestions. Thanks as well to the many international contributors who provided feature requests or identified bugs or simply manual improvements!

Last but not least, I thank Till Tantau and Mark Wibrow for their excellent graphics (and more) package \PGF\ and \Tikz\ which is the base of \PGFPlots.

%%%%%%%%%%%%%%%%%%%%%%%%%%%%%%%%%%%%%%%%%%%%%%%%%%%%%%%%%%%%%%%%%%%%%%%%%%%%%
%
% Package pgfplots.sty documentation. 
%
% Copyright 2007/2008 by Christian Feuersaenger.
%
% This program is free software: you can redistribute it and/or modify
% it under the terms of the GNU General Public License as published by
% the Free Software Foundation, either version 3 of the License, or
% (at your option) any later version.
% 
% This program is distributed in the hope that it will be useful,
% but WITHOUT ANY WARRANTY; without even the implied warranty of
% MERCHANTABILITY or FITNESS FOR A PARTICULAR PURPOSE.  See the
% GNU General Public License for more details.
% 
% You should have received a copy of the GNU General Public License
% along with this program.  If not, see <http://www.gnu.org/licenses/>.
%
%
%%%%%%%%%%%%%%%%%%%%%%%%%%%%%%%%%%%%%%%%%%%%%%%%%%%%%%%%%%%%%%%%%%%%%%%%%%%%%
\input{pgfplots.preamble.tex}
%\RequirePackage[german,english,francais]{babel}

\pgfplotsmanualexternalexpensivetrue

%\usetikzlibrary{external}
%\tikzexternalize[prefix=figures/]{pgfplots}

\title{%
	Manual for Package \PGFPlots\\
	{\small Version \pgfplotsversion}\\
	{\small\href{http://sourceforge.net/projects/pgfplots}{http://sourceforge.net/projects/pgfplots}}}

%\includeonly{pgfplots.libs}


\begin{document}

\def\plotcoords{%
\addplot coordinates {
(5,8.312e-02)    (17,2.547e-02)   (49,7.407e-03)
(129,2.102e-03)  (321,5.874e-04)  (769,1.623e-04)
(1793,4.442e-05) (4097,1.207e-05) (9217,3.261e-06)
};

\addplot coordinates{
(7,8.472e-02)    (31,3.044e-02)    (111,1.022e-02)
(351,3.303e-03)  (1023,1.039e-03)  (2815,3.196e-04)
(7423,9.658e-05) (18943,2.873e-05) (47103,8.437e-06)
};

\addplot coordinates{
(9,7.881e-02)     (49,3.243e-02)    (209,1.232e-02)
(769,4.454e-03)   (2561,1.551e-03)  (7937,5.236e-04)
(23297,1.723e-04) (65537,5.545e-05) (178177,1.751e-05)
};

\addplot coordinates{
(11,6.887e-02)    (71,3.177e-02)     (351,1.341e-02)
(1471,5.334e-03)  (5503,2.027e-03)   (18943,7.415e-04)
(61183,2.628e-04) (187903,9.063e-05) (553983,3.053e-05)
};

\addplot coordinates{
(13,5.755e-02)     (97,2.925e-02)     (545,1.351e-02)
(2561,5.842e-03)   (10625,2.397e-03)  (40193,9.414e-04)
(141569,3.564e-04) (471041,1.308e-04) 
(1496065,4.670e-05)
};
}%
\maketitle
\begin{abstract}%
\PGFPlots\ draws high--quality function plots in normal or logarithmic scaling with a user-friendly interface directly in \TeX. The user supplies axis labels, legend entries and the plot coordinates for one or more plots and \PGFPlots\ applies axis scaling, computes any logarithms and axis ticks and draws the plots, supporting line plots, scatter plots, piecewise constant plots, bar plots, area plots, mesh-- and surface plots and some more. It is based on Till Tantau's package \PGF/\Tikz.
\end{abstract}
\tableofcontents
\section{Introduction}
This package provides tools to generate plots and labeled axes easily. It draws normal plots, logplots and semi-logplots, in two and three dimensions. Axis ticks, labels, legends (in case of multiple plots) can be added with key-value options. It can cycle through a set of predefined line/marker/color specifications. In summary, its purpose is to simplify the generation of high-quality function and/or data plots, and solving the problems of
\begin{itemize}
	\item consistency of document and font type and font size,
	\item direct use of \TeX\ math mode in axis descriptions,
	\item consistency of data and figures (no third party tool necessary),
	\item inter document consistency using preamble configurations and styles.
\end{itemize}
Although not necessary, separate |.pdf| or |.eps| graphics can be generated using the |external| library developed as part of \Tikz.

\PGFPlots\ is build completely on \Tikz/\PGF. Knowledge of \Tikz\ will simplify the work with \PGFPlots, although it is not required.

\section[About PGFPlots: Preliminaries]{About {\normalfont\PGFPlots}: Preliminaries}
This section contains information about upgrades, the team, the installation (in case you need to do it manually) and troubleshooting. You may skip it completely except for the upgrade remarks.

However, note that this library requires at least \PGF\ version $2.00$. At the time of this writing, many \TeX-distributions still contain the older \PGF\ version $1.18$, so it may be necessary to install a recent \PGF\ prior to using \PGFPlots.

\subsection{Components}
\PGFPlots\ comes with two components:
\begin{enumerate}
	\item the plotting component (which you are currently reading) and
	\item the \PGFPlotstable\ component which simplifies number formatting and postprocessing of numerical tables. It comes as a separate package and has its own manual \href{file:pgfplotstable.pdf}{pgfplotstable.pdf}.
\end{enumerate}

\subsection{Upgrade remarks}
This release provides a lot of improvements which can be found in all detail in \texttt{ChangeLog} for interested readers. However, some attention is useful with respect to the following changes.

\subsubsection{New Optional Features}
\begin{enumerate}
	\item \PGFPlots\ 1.3 comes with user interface improvements. The technical distinction between ``behavior options'' and ``style options'' of older versions is no longer necessary (although still fully supported).

	\item \PGFPlots\ 1.3 has a new feature which allows to \emph{move axis labels tight to tick labels} automatically.

	Since this affects the spacing, it is not enabled be default.

	Use
\begin{codeexample}[code only]
\usepackage{pgfplots}
\pgfplotsset{compat=1.3}
\end{codeexample}
	in your preamble to benefit from the improved spacing\footnote{The |compat=1.3| setting is necessary for new features which move axis labels around like reversed axis. However, \PGFPlots\ will still work as it does in earlier versions, your documents will remain the same if you don't set it explicitly.}.
	Take a look at the next page for the precise definition of |compat|.

	\item \PGFPlots\ 1.3 now supports reversed axes. It is no longer necessary to use work--arounds with negative units.
\pgfkeys{/pdflinks/search key prefixes in/.add={/pgfplots/,}{}}

	Take a look at the |x dir=reverse| key.

	Existing work--arounds will still function properly. Use |\pgfplotsset{compat=1.3}| together with |x dir=reverse| to switch to the new version.
\end{enumerate}

\subsubsection{Old Features Which May Need Attention}
\begin{enumerate}
	\item The |scatter/classes| feature produces proper legends as of version 1.3. This may change the appearance of existing legends of plots with |scatter/classes|.

	\item Due to a small math bug in \PGF\ $2.00$, you \emph{can't} use the math expression `|-x^2|'. It is necessary to use `|0-x^2|' instead. The same holds for
		`|exp(-x^2)|'; use `|exp(0-x^2)|' instead.

		This will be fixed with \PGF\ $>2.00$.

	\item Starting with \PGFPlots\ $1.1$, |\tikzstyle| should \emph{no longer be used} to set \PGFPlots\ options.
	
	Although |\tikzstyle| is still supported for some older \PGFPlots\ options, you should replace any occurance of |\tikzstyle| with |\pgfplotsset{|\meta{style name}|/.style={|\meta{key-value-list}|}}| or the associated |/.append style| variant. See section~\ref{sec:styles} for more detail.
\end{enumerate}
I apologize for any inconvenience caused by these changes.

\begin{pgfplotskey}{compat=\mchoice{1.3,pre 1.3,default,newest} (initially default)}
	The preamble configuration 
\begin{codeexample}[code only]
\usepackage{pgfplots}
\pgfplotsset{compat=1.3}
\end{codeexample}
	allows to choose between backwards compatibility and most recent features.

	\PGFPlots\ version 1.3 comes with new features which may lead to different spacings compared to earlier versions:
	\begin{enumerate}
		\item Version 1.3 comes with the |xlabel near ticks| feature which places (in this case $x$) axis labels ``near'' the tick labels, taking the size of tick labels into account.
		
		Older versions used |xlabel absolute|, i.e.\ an absolute distance between the axis and axis labels (now matter how large or small tick labels are).

		The initial setting \emph{keeps backwards compatibility}.

		You are encouraged to use
\begin{codeexample}[code only]
\usepackage{pgfplots}
\pgfplotsset{compat=1.3}
\end{codeexample}
		in your preamble.
		
		\item Version 1.3 fixes a bug with |anchor=south west| which occurred mainly for reversed axes: the anchors where upside--down. This is fixed now.

		
		If you have written some work--around and want to keep it going, use

		|\pgfplotsset{compat/anchors=pre 1.3}|\pgfmanualpdflabel{/pgfplots/compat/anchors}{} 

		to disable the bug fix.
	\end{enumerate}

	Use |\pgfplotsset{compat=default}| to restore the factory settings.

	The setting |\pgfplotsset{compat=newest}| will always use features of the most recent version. This might result in small changes in the document's appearance.
\end{pgfplotskey}

\subsection{The Team}
\PGFPlots\ has been written mainly by Christian Feuers�nger with many improvements of Pascal Wolkotte and Nick Papior Andersen as a spare time project. We hope it is useful and provides valuable plots.

If you are interested in writing something but don't know how, consider reading the auxiliary manual \href{file:TeX-programming-notes.pdf}{TeX-programming-notes.pdf} which comes with \PGFPlots. It is far from complete, but maybe it is a good starting point (at least for more literature).

\subsection{Acknowledgements}
I thank God for all hours of enjoyed programming. I thank Pascal Wolkotte and Nick Papior Andersen for their programming efforts and contributions as part of the development team. I thank Stefan Tibus, who contributed the |plot shell| feature. I thank Tom Cashman for the contribution of the |reverse legend| feature. Special thanks go to Stefan Pinnow whose tests of \PGFPlots\ lead to numerous quality improvements. Furthermore, I thank Dr.~Schweitzer for many fruitful discussions and Dr.~Meine for his ideas and suggestions. Thanks as well to the many international contributors who provided feature requests or identified bugs or simply manual improvements!

Last but not least, I thank Till Tantau and Mark Wibrow for their excellent graphics (and more) package \PGF\ and \Tikz\ which is the base of \PGFPlots.

\include{pgfplots.install}%
\include{pgfplots.intro}%
\include{pgfplots.reference}%
\include{pgfplots.libs}%
\include{pgfplots.resources}%
\include{pgfplots.importexport}%
\include{pgfplots.basic.reference}%

\printindex

\bibliographystyle{abbrv} %gerapali} %gerabbrv} %gerunsrt.bst} %gerabbrv}% gerplain}
\nocite{pgfplotstable}
\nocite{programmingnotes}
\bibliography{pgfplots}
\end{document}
%
%%%%%%%%%%%%%%%%%%%%%%%%%%%%%%%%%%%%%%%%%%%%%%%%%%%%%%%%%%%%%%%%%%%%%%%%%%%%%
%
% Package pgfplots.sty documentation. 
%
% Copyright 2007/2008 by Christian Feuersaenger.
%
% This program is free software: you can redistribute it and/or modify
% it under the terms of the GNU General Public License as published by
% the Free Software Foundation, either version 3 of the License, or
% (at your option) any later version.
% 
% This program is distributed in the hope that it will be useful,
% but WITHOUT ANY WARRANTY; without even the implied warranty of
% MERCHANTABILITY or FITNESS FOR A PARTICULAR PURPOSE.  See the
% GNU General Public License for more details.
% 
% You should have received a copy of the GNU General Public License
% along with this program.  If not, see <http://www.gnu.org/licenses/>.
%
%
%%%%%%%%%%%%%%%%%%%%%%%%%%%%%%%%%%%%%%%%%%%%%%%%%%%%%%%%%%%%%%%%%%%%%%%%%%%%%
\input{pgfplots.preamble.tex}
%\RequirePackage[german,english,francais]{babel}

\pgfplotsmanualexternalexpensivetrue

%\usetikzlibrary{external}
%\tikzexternalize[prefix=figures/]{pgfplots}

\title{%
	Manual for Package \PGFPlots\\
	{\small Version \pgfplotsversion}\\
	{\small\href{http://sourceforge.net/projects/pgfplots}{http://sourceforge.net/projects/pgfplots}}}

%\includeonly{pgfplots.libs}


\begin{document}

\def\plotcoords{%
\addplot coordinates {
(5,8.312e-02)    (17,2.547e-02)   (49,7.407e-03)
(129,2.102e-03)  (321,5.874e-04)  (769,1.623e-04)
(1793,4.442e-05) (4097,1.207e-05) (9217,3.261e-06)
};

\addplot coordinates{
(7,8.472e-02)    (31,3.044e-02)    (111,1.022e-02)
(351,3.303e-03)  (1023,1.039e-03)  (2815,3.196e-04)
(7423,9.658e-05) (18943,2.873e-05) (47103,8.437e-06)
};

\addplot coordinates{
(9,7.881e-02)     (49,3.243e-02)    (209,1.232e-02)
(769,4.454e-03)   (2561,1.551e-03)  (7937,5.236e-04)
(23297,1.723e-04) (65537,5.545e-05) (178177,1.751e-05)
};

\addplot coordinates{
(11,6.887e-02)    (71,3.177e-02)     (351,1.341e-02)
(1471,5.334e-03)  (5503,2.027e-03)   (18943,7.415e-04)
(61183,2.628e-04) (187903,9.063e-05) (553983,3.053e-05)
};

\addplot coordinates{
(13,5.755e-02)     (97,2.925e-02)     (545,1.351e-02)
(2561,5.842e-03)   (10625,2.397e-03)  (40193,9.414e-04)
(141569,3.564e-04) (471041,1.308e-04) 
(1496065,4.670e-05)
};
}%
\maketitle
\begin{abstract}%
\PGFPlots\ draws high--quality function plots in normal or logarithmic scaling with a user-friendly interface directly in \TeX. The user supplies axis labels, legend entries and the plot coordinates for one or more plots and \PGFPlots\ applies axis scaling, computes any logarithms and axis ticks and draws the plots, supporting line plots, scatter plots, piecewise constant plots, bar plots, area plots, mesh-- and surface plots and some more. It is based on Till Tantau's package \PGF/\Tikz.
\end{abstract}
\tableofcontents
\section{Introduction}
This package provides tools to generate plots and labeled axes easily. It draws normal plots, logplots and semi-logplots, in two and three dimensions. Axis ticks, labels, legends (in case of multiple plots) can be added with key-value options. It can cycle through a set of predefined line/marker/color specifications. In summary, its purpose is to simplify the generation of high-quality function and/or data plots, and solving the problems of
\begin{itemize}
	\item consistency of document and font type and font size,
	\item direct use of \TeX\ math mode in axis descriptions,
	\item consistency of data and figures (no third party tool necessary),
	\item inter document consistency using preamble configurations and styles.
\end{itemize}
Although not necessary, separate |.pdf| or |.eps| graphics can be generated using the |external| library developed as part of \Tikz.

\PGFPlots\ is build completely on \Tikz/\PGF. Knowledge of \Tikz\ will simplify the work with \PGFPlots, although it is not required.

\section[About PGFPlots: Preliminaries]{About {\normalfont\PGFPlots}: Preliminaries}
This section contains information about upgrades, the team, the installation (in case you need to do it manually) and troubleshooting. You may skip it completely except for the upgrade remarks.

However, note that this library requires at least \PGF\ version $2.00$. At the time of this writing, many \TeX-distributions still contain the older \PGF\ version $1.18$, so it may be necessary to install a recent \PGF\ prior to using \PGFPlots.

\subsection{Components}
\PGFPlots\ comes with two components:
\begin{enumerate}
	\item the plotting component (which you are currently reading) and
	\item the \PGFPlotstable\ component which simplifies number formatting and postprocessing of numerical tables. It comes as a separate package and has its own manual \href{file:pgfplotstable.pdf}{pgfplotstable.pdf}.
\end{enumerate}

\subsection{Upgrade remarks}
This release provides a lot of improvements which can be found in all detail in \texttt{ChangeLog} for interested readers. However, some attention is useful with respect to the following changes.

\subsubsection{New Optional Features}
\begin{enumerate}
	\item \PGFPlots\ 1.3 comes with user interface improvements. The technical distinction between ``behavior options'' and ``style options'' of older versions is no longer necessary (although still fully supported).

	\item \PGFPlots\ 1.3 has a new feature which allows to \emph{move axis labels tight to tick labels} automatically.

	Since this affects the spacing, it is not enabled be default.

	Use
\begin{codeexample}[code only]
\usepackage{pgfplots}
\pgfplotsset{compat=1.3}
\end{codeexample}
	in your preamble to benefit from the improved spacing\footnote{The |compat=1.3| setting is necessary for new features which move axis labels around like reversed axis. However, \PGFPlots\ will still work as it does in earlier versions, your documents will remain the same if you don't set it explicitly.}.
	Take a look at the next page for the precise definition of |compat|.

	\item \PGFPlots\ 1.3 now supports reversed axes. It is no longer necessary to use work--arounds with negative units.
\pgfkeys{/pdflinks/search key prefixes in/.add={/pgfplots/,}{}}

	Take a look at the |x dir=reverse| key.

	Existing work--arounds will still function properly. Use |\pgfplotsset{compat=1.3}| together with |x dir=reverse| to switch to the new version.
\end{enumerate}

\subsubsection{Old Features Which May Need Attention}
\begin{enumerate}
	\item The |scatter/classes| feature produces proper legends as of version 1.3. This may change the appearance of existing legends of plots with |scatter/classes|.

	\item Due to a small math bug in \PGF\ $2.00$, you \emph{can't} use the math expression `|-x^2|'. It is necessary to use `|0-x^2|' instead. The same holds for
		`|exp(-x^2)|'; use `|exp(0-x^2)|' instead.

		This will be fixed with \PGF\ $>2.00$.

	\item Starting with \PGFPlots\ $1.1$, |\tikzstyle| should \emph{no longer be used} to set \PGFPlots\ options.
	
	Although |\tikzstyle| is still supported for some older \PGFPlots\ options, you should replace any occurance of |\tikzstyle| with |\pgfplotsset{|\meta{style name}|/.style={|\meta{key-value-list}|}}| or the associated |/.append style| variant. See section~\ref{sec:styles} for more detail.
\end{enumerate}
I apologize for any inconvenience caused by these changes.

\begin{pgfplotskey}{compat=\mchoice{1.3,pre 1.3,default,newest} (initially default)}
	The preamble configuration 
\begin{codeexample}[code only]
\usepackage{pgfplots}
\pgfplotsset{compat=1.3}
\end{codeexample}
	allows to choose between backwards compatibility and most recent features.

	\PGFPlots\ version 1.3 comes with new features which may lead to different spacings compared to earlier versions:
	\begin{enumerate}
		\item Version 1.3 comes with the |xlabel near ticks| feature which places (in this case $x$) axis labels ``near'' the tick labels, taking the size of tick labels into account.
		
		Older versions used |xlabel absolute|, i.e.\ an absolute distance between the axis and axis labels (now matter how large or small tick labels are).

		The initial setting \emph{keeps backwards compatibility}.

		You are encouraged to use
\begin{codeexample}[code only]
\usepackage{pgfplots}
\pgfplotsset{compat=1.3}
\end{codeexample}
		in your preamble.
		
		\item Version 1.3 fixes a bug with |anchor=south west| which occurred mainly for reversed axes: the anchors where upside--down. This is fixed now.

		
		If you have written some work--around and want to keep it going, use

		|\pgfplotsset{compat/anchors=pre 1.3}|\pgfmanualpdflabel{/pgfplots/compat/anchors}{} 

		to disable the bug fix.
	\end{enumerate}

	Use |\pgfplotsset{compat=default}| to restore the factory settings.

	The setting |\pgfplotsset{compat=newest}| will always use features of the most recent version. This might result in small changes in the document's appearance.
\end{pgfplotskey}

\subsection{The Team}
\PGFPlots\ has been written mainly by Christian Feuers�nger with many improvements of Pascal Wolkotte and Nick Papior Andersen as a spare time project. We hope it is useful and provides valuable plots.

If you are interested in writing something but don't know how, consider reading the auxiliary manual \href{file:TeX-programming-notes.pdf}{TeX-programming-notes.pdf} which comes with \PGFPlots. It is far from complete, but maybe it is a good starting point (at least for more literature).

\subsection{Acknowledgements}
I thank God for all hours of enjoyed programming. I thank Pascal Wolkotte and Nick Papior Andersen for their programming efforts and contributions as part of the development team. I thank Stefan Tibus, who contributed the |plot shell| feature. I thank Tom Cashman for the contribution of the |reverse legend| feature. Special thanks go to Stefan Pinnow whose tests of \PGFPlots\ lead to numerous quality improvements. Furthermore, I thank Dr.~Schweitzer for many fruitful discussions and Dr.~Meine for his ideas and suggestions. Thanks as well to the many international contributors who provided feature requests or identified bugs or simply manual improvements!

Last but not least, I thank Till Tantau and Mark Wibrow for their excellent graphics (and more) package \PGF\ and \Tikz\ which is the base of \PGFPlots.

\include{pgfplots.install}%
\include{pgfplots.intro}%
\include{pgfplots.reference}%
\include{pgfplots.libs}%
\include{pgfplots.resources}%
\include{pgfplots.importexport}%
\include{pgfplots.basic.reference}%

\printindex

\bibliographystyle{abbrv} %gerapali} %gerabbrv} %gerunsrt.bst} %gerabbrv}% gerplain}
\nocite{pgfplotstable}
\nocite{programmingnotes}
\bibliography{pgfplots}
\end{document}
%
%%%%%%%%%%%%%%%%%%%%%%%%%%%%%%%%%%%%%%%%%%%%%%%%%%%%%%%%%%%%%%%%%%%%%%%%%%%%%
%
% Package pgfplots.sty documentation. 
%
% Copyright 2007/2008 by Christian Feuersaenger.
%
% This program is free software: you can redistribute it and/or modify
% it under the terms of the GNU General Public License as published by
% the Free Software Foundation, either version 3 of the License, or
% (at your option) any later version.
% 
% This program is distributed in the hope that it will be useful,
% but WITHOUT ANY WARRANTY; without even the implied warranty of
% MERCHANTABILITY or FITNESS FOR A PARTICULAR PURPOSE.  See the
% GNU General Public License for more details.
% 
% You should have received a copy of the GNU General Public License
% along with this program.  If not, see <http://www.gnu.org/licenses/>.
%
%
%%%%%%%%%%%%%%%%%%%%%%%%%%%%%%%%%%%%%%%%%%%%%%%%%%%%%%%%%%%%%%%%%%%%%%%%%%%%%
\input{pgfplots.preamble.tex}
%\RequirePackage[german,english,francais]{babel}

\pgfplotsmanualexternalexpensivetrue

%\usetikzlibrary{external}
%\tikzexternalize[prefix=figures/]{pgfplots}

\title{%
	Manual for Package \PGFPlots\\
	{\small Version \pgfplotsversion}\\
	{\small\href{http://sourceforge.net/projects/pgfplots}{http://sourceforge.net/projects/pgfplots}}}

%\includeonly{pgfplots.libs}


\begin{document}

\def\plotcoords{%
\addplot coordinates {
(5,8.312e-02)    (17,2.547e-02)   (49,7.407e-03)
(129,2.102e-03)  (321,5.874e-04)  (769,1.623e-04)
(1793,4.442e-05) (4097,1.207e-05) (9217,3.261e-06)
};

\addplot coordinates{
(7,8.472e-02)    (31,3.044e-02)    (111,1.022e-02)
(351,3.303e-03)  (1023,1.039e-03)  (2815,3.196e-04)
(7423,9.658e-05) (18943,2.873e-05) (47103,8.437e-06)
};

\addplot coordinates{
(9,7.881e-02)     (49,3.243e-02)    (209,1.232e-02)
(769,4.454e-03)   (2561,1.551e-03)  (7937,5.236e-04)
(23297,1.723e-04) (65537,5.545e-05) (178177,1.751e-05)
};

\addplot coordinates{
(11,6.887e-02)    (71,3.177e-02)     (351,1.341e-02)
(1471,5.334e-03)  (5503,2.027e-03)   (18943,7.415e-04)
(61183,2.628e-04) (187903,9.063e-05) (553983,3.053e-05)
};

\addplot coordinates{
(13,5.755e-02)     (97,2.925e-02)     (545,1.351e-02)
(2561,5.842e-03)   (10625,2.397e-03)  (40193,9.414e-04)
(141569,3.564e-04) (471041,1.308e-04) 
(1496065,4.670e-05)
};
}%
\maketitle
\begin{abstract}%
\PGFPlots\ draws high--quality function plots in normal or logarithmic scaling with a user-friendly interface directly in \TeX. The user supplies axis labels, legend entries and the plot coordinates for one or more plots and \PGFPlots\ applies axis scaling, computes any logarithms and axis ticks and draws the plots, supporting line plots, scatter plots, piecewise constant plots, bar plots, area plots, mesh-- and surface plots and some more. It is based on Till Tantau's package \PGF/\Tikz.
\end{abstract}
\tableofcontents
\section{Introduction}
This package provides tools to generate plots and labeled axes easily. It draws normal plots, logplots and semi-logplots, in two and three dimensions. Axis ticks, labels, legends (in case of multiple plots) can be added with key-value options. It can cycle through a set of predefined line/marker/color specifications. In summary, its purpose is to simplify the generation of high-quality function and/or data plots, and solving the problems of
\begin{itemize}
	\item consistency of document and font type and font size,
	\item direct use of \TeX\ math mode in axis descriptions,
	\item consistency of data and figures (no third party tool necessary),
	\item inter document consistency using preamble configurations and styles.
\end{itemize}
Although not necessary, separate |.pdf| or |.eps| graphics can be generated using the |external| library developed as part of \Tikz.

\PGFPlots\ is build completely on \Tikz/\PGF. Knowledge of \Tikz\ will simplify the work with \PGFPlots, although it is not required.

\section[About PGFPlots: Preliminaries]{About {\normalfont\PGFPlots}: Preliminaries}
This section contains information about upgrades, the team, the installation (in case you need to do it manually) and troubleshooting. You may skip it completely except for the upgrade remarks.

However, note that this library requires at least \PGF\ version $2.00$. At the time of this writing, many \TeX-distributions still contain the older \PGF\ version $1.18$, so it may be necessary to install a recent \PGF\ prior to using \PGFPlots.

\subsection{Components}
\PGFPlots\ comes with two components:
\begin{enumerate}
	\item the plotting component (which you are currently reading) and
	\item the \PGFPlotstable\ component which simplifies number formatting and postprocessing of numerical tables. It comes as a separate package and has its own manual \href{file:pgfplotstable.pdf}{pgfplotstable.pdf}.
\end{enumerate}

\subsection{Upgrade remarks}
This release provides a lot of improvements which can be found in all detail in \texttt{ChangeLog} for interested readers. However, some attention is useful with respect to the following changes.

\subsubsection{New Optional Features}
\begin{enumerate}
	\item \PGFPlots\ 1.3 comes with user interface improvements. The technical distinction between ``behavior options'' and ``style options'' of older versions is no longer necessary (although still fully supported).

	\item \PGFPlots\ 1.3 has a new feature which allows to \emph{move axis labels tight to tick labels} automatically.

	Since this affects the spacing, it is not enabled be default.

	Use
\begin{codeexample}[code only]
\usepackage{pgfplots}
\pgfplotsset{compat=1.3}
\end{codeexample}
	in your preamble to benefit from the improved spacing\footnote{The |compat=1.3| setting is necessary for new features which move axis labels around like reversed axis. However, \PGFPlots\ will still work as it does in earlier versions, your documents will remain the same if you don't set it explicitly.}.
	Take a look at the next page for the precise definition of |compat|.

	\item \PGFPlots\ 1.3 now supports reversed axes. It is no longer necessary to use work--arounds with negative units.
\pgfkeys{/pdflinks/search key prefixes in/.add={/pgfplots/,}{}}

	Take a look at the |x dir=reverse| key.

	Existing work--arounds will still function properly. Use |\pgfplotsset{compat=1.3}| together with |x dir=reverse| to switch to the new version.
\end{enumerate}

\subsubsection{Old Features Which May Need Attention}
\begin{enumerate}
	\item The |scatter/classes| feature produces proper legends as of version 1.3. This may change the appearance of existing legends of plots with |scatter/classes|.

	\item Due to a small math bug in \PGF\ $2.00$, you \emph{can't} use the math expression `|-x^2|'. It is necessary to use `|0-x^2|' instead. The same holds for
		`|exp(-x^2)|'; use `|exp(0-x^2)|' instead.

		This will be fixed with \PGF\ $>2.00$.

	\item Starting with \PGFPlots\ $1.1$, |\tikzstyle| should \emph{no longer be used} to set \PGFPlots\ options.
	
	Although |\tikzstyle| is still supported for some older \PGFPlots\ options, you should replace any occurance of |\tikzstyle| with |\pgfplotsset{|\meta{style name}|/.style={|\meta{key-value-list}|}}| or the associated |/.append style| variant. See section~\ref{sec:styles} for more detail.
\end{enumerate}
I apologize for any inconvenience caused by these changes.

\begin{pgfplotskey}{compat=\mchoice{1.3,pre 1.3,default,newest} (initially default)}
	The preamble configuration 
\begin{codeexample}[code only]
\usepackage{pgfplots}
\pgfplotsset{compat=1.3}
\end{codeexample}
	allows to choose between backwards compatibility and most recent features.

	\PGFPlots\ version 1.3 comes with new features which may lead to different spacings compared to earlier versions:
	\begin{enumerate}
		\item Version 1.3 comes with the |xlabel near ticks| feature which places (in this case $x$) axis labels ``near'' the tick labels, taking the size of tick labels into account.
		
		Older versions used |xlabel absolute|, i.e.\ an absolute distance between the axis and axis labels (now matter how large or small tick labels are).

		The initial setting \emph{keeps backwards compatibility}.

		You are encouraged to use
\begin{codeexample}[code only]
\usepackage{pgfplots}
\pgfplotsset{compat=1.3}
\end{codeexample}
		in your preamble.
		
		\item Version 1.3 fixes a bug with |anchor=south west| which occurred mainly for reversed axes: the anchors where upside--down. This is fixed now.

		
		If you have written some work--around and want to keep it going, use

		|\pgfplotsset{compat/anchors=pre 1.3}|\pgfmanualpdflabel{/pgfplots/compat/anchors}{} 

		to disable the bug fix.
	\end{enumerate}

	Use |\pgfplotsset{compat=default}| to restore the factory settings.

	The setting |\pgfplotsset{compat=newest}| will always use features of the most recent version. This might result in small changes in the document's appearance.
\end{pgfplotskey}

\subsection{The Team}
\PGFPlots\ has been written mainly by Christian Feuers�nger with many improvements of Pascal Wolkotte and Nick Papior Andersen as a spare time project. We hope it is useful and provides valuable plots.

If you are interested in writing something but don't know how, consider reading the auxiliary manual \href{file:TeX-programming-notes.pdf}{TeX-programming-notes.pdf} which comes with \PGFPlots. It is far from complete, but maybe it is a good starting point (at least for more literature).

\subsection{Acknowledgements}
I thank God for all hours of enjoyed programming. I thank Pascal Wolkotte and Nick Papior Andersen for their programming efforts and contributions as part of the development team. I thank Stefan Tibus, who contributed the |plot shell| feature. I thank Tom Cashman for the contribution of the |reverse legend| feature. Special thanks go to Stefan Pinnow whose tests of \PGFPlots\ lead to numerous quality improvements. Furthermore, I thank Dr.~Schweitzer for many fruitful discussions and Dr.~Meine for his ideas and suggestions. Thanks as well to the many international contributors who provided feature requests or identified bugs or simply manual improvements!

Last but not least, I thank Till Tantau and Mark Wibrow for their excellent graphics (and more) package \PGF\ and \Tikz\ which is the base of \PGFPlots.

\include{pgfplots.install}%
\include{pgfplots.intro}%
\include{pgfplots.reference}%
\include{pgfplots.libs}%
\include{pgfplots.resources}%
\include{pgfplots.importexport}%
\include{pgfplots.basic.reference}%

\printindex

\bibliographystyle{abbrv} %gerapali} %gerabbrv} %gerunsrt.bst} %gerabbrv}% gerplain}
\nocite{pgfplotstable}
\nocite{programmingnotes}
\bibliography{pgfplots}
\end{document}
%
%%%%%%%%%%%%%%%%%%%%%%%%%%%%%%%%%%%%%%%%%%%%%%%%%%%%%%%%%%%%%%%%%%%%%%%%%%%%%
%
% Package pgfplots.sty documentation. 
%
% Copyright 2007/2008 by Christian Feuersaenger.
%
% This program is free software: you can redistribute it and/or modify
% it under the terms of the GNU General Public License as published by
% the Free Software Foundation, either version 3 of the License, or
% (at your option) any later version.
% 
% This program is distributed in the hope that it will be useful,
% but WITHOUT ANY WARRANTY; without even the implied warranty of
% MERCHANTABILITY or FITNESS FOR A PARTICULAR PURPOSE.  See the
% GNU General Public License for more details.
% 
% You should have received a copy of the GNU General Public License
% along with this program.  If not, see <http://www.gnu.org/licenses/>.
%
%
%%%%%%%%%%%%%%%%%%%%%%%%%%%%%%%%%%%%%%%%%%%%%%%%%%%%%%%%%%%%%%%%%%%%%%%%%%%%%
\input{pgfplots.preamble.tex}
%\RequirePackage[german,english,francais]{babel}

\pgfplotsmanualexternalexpensivetrue

%\usetikzlibrary{external}
%\tikzexternalize[prefix=figures/]{pgfplots}

\title{%
	Manual for Package \PGFPlots\\
	{\small Version \pgfplotsversion}\\
	{\small\href{http://sourceforge.net/projects/pgfplots}{http://sourceforge.net/projects/pgfplots}}}

%\includeonly{pgfplots.libs}


\begin{document}

\def\plotcoords{%
\addplot coordinates {
(5,8.312e-02)    (17,2.547e-02)   (49,7.407e-03)
(129,2.102e-03)  (321,5.874e-04)  (769,1.623e-04)
(1793,4.442e-05) (4097,1.207e-05) (9217,3.261e-06)
};

\addplot coordinates{
(7,8.472e-02)    (31,3.044e-02)    (111,1.022e-02)
(351,3.303e-03)  (1023,1.039e-03)  (2815,3.196e-04)
(7423,9.658e-05) (18943,2.873e-05) (47103,8.437e-06)
};

\addplot coordinates{
(9,7.881e-02)     (49,3.243e-02)    (209,1.232e-02)
(769,4.454e-03)   (2561,1.551e-03)  (7937,5.236e-04)
(23297,1.723e-04) (65537,5.545e-05) (178177,1.751e-05)
};

\addplot coordinates{
(11,6.887e-02)    (71,3.177e-02)     (351,1.341e-02)
(1471,5.334e-03)  (5503,2.027e-03)   (18943,7.415e-04)
(61183,2.628e-04) (187903,9.063e-05) (553983,3.053e-05)
};

\addplot coordinates{
(13,5.755e-02)     (97,2.925e-02)     (545,1.351e-02)
(2561,5.842e-03)   (10625,2.397e-03)  (40193,9.414e-04)
(141569,3.564e-04) (471041,1.308e-04) 
(1496065,4.670e-05)
};
}%
\maketitle
\begin{abstract}%
\PGFPlots\ draws high--quality function plots in normal or logarithmic scaling with a user-friendly interface directly in \TeX. The user supplies axis labels, legend entries and the plot coordinates for one or more plots and \PGFPlots\ applies axis scaling, computes any logarithms and axis ticks and draws the plots, supporting line plots, scatter plots, piecewise constant plots, bar plots, area plots, mesh-- and surface plots and some more. It is based on Till Tantau's package \PGF/\Tikz.
\end{abstract}
\tableofcontents
\section{Introduction}
This package provides tools to generate plots and labeled axes easily. It draws normal plots, logplots and semi-logplots, in two and three dimensions. Axis ticks, labels, legends (in case of multiple plots) can be added with key-value options. It can cycle through a set of predefined line/marker/color specifications. In summary, its purpose is to simplify the generation of high-quality function and/or data plots, and solving the problems of
\begin{itemize}
	\item consistency of document and font type and font size,
	\item direct use of \TeX\ math mode in axis descriptions,
	\item consistency of data and figures (no third party tool necessary),
	\item inter document consistency using preamble configurations and styles.
\end{itemize}
Although not necessary, separate |.pdf| or |.eps| graphics can be generated using the |external| library developed as part of \Tikz.

\PGFPlots\ is build completely on \Tikz/\PGF. Knowledge of \Tikz\ will simplify the work with \PGFPlots, although it is not required.

\section[About PGFPlots: Preliminaries]{About {\normalfont\PGFPlots}: Preliminaries}
This section contains information about upgrades, the team, the installation (in case you need to do it manually) and troubleshooting. You may skip it completely except for the upgrade remarks.

However, note that this library requires at least \PGF\ version $2.00$. At the time of this writing, many \TeX-distributions still contain the older \PGF\ version $1.18$, so it may be necessary to install a recent \PGF\ prior to using \PGFPlots.

\subsection{Components}
\PGFPlots\ comes with two components:
\begin{enumerate}
	\item the plotting component (which you are currently reading) and
	\item the \PGFPlotstable\ component which simplifies number formatting and postprocessing of numerical tables. It comes as a separate package and has its own manual \href{file:pgfplotstable.pdf}{pgfplotstable.pdf}.
\end{enumerate}

\subsection{Upgrade remarks}
This release provides a lot of improvements which can be found in all detail in \texttt{ChangeLog} for interested readers. However, some attention is useful with respect to the following changes.

\subsubsection{New Optional Features}
\begin{enumerate}
	\item \PGFPlots\ 1.3 comes with user interface improvements. The technical distinction between ``behavior options'' and ``style options'' of older versions is no longer necessary (although still fully supported).

	\item \PGFPlots\ 1.3 has a new feature which allows to \emph{move axis labels tight to tick labels} automatically.

	Since this affects the spacing, it is not enabled be default.

	Use
\begin{codeexample}[code only]
\usepackage{pgfplots}
\pgfplotsset{compat=1.3}
\end{codeexample}
	in your preamble to benefit from the improved spacing\footnote{The |compat=1.3| setting is necessary for new features which move axis labels around like reversed axis. However, \PGFPlots\ will still work as it does in earlier versions, your documents will remain the same if you don't set it explicitly.}.
	Take a look at the next page for the precise definition of |compat|.

	\item \PGFPlots\ 1.3 now supports reversed axes. It is no longer necessary to use work--arounds with negative units.
\pgfkeys{/pdflinks/search key prefixes in/.add={/pgfplots/,}{}}

	Take a look at the |x dir=reverse| key.

	Existing work--arounds will still function properly. Use |\pgfplotsset{compat=1.3}| together with |x dir=reverse| to switch to the new version.
\end{enumerate}

\subsubsection{Old Features Which May Need Attention}
\begin{enumerate}
	\item The |scatter/classes| feature produces proper legends as of version 1.3. This may change the appearance of existing legends of plots with |scatter/classes|.

	\item Due to a small math bug in \PGF\ $2.00$, you \emph{can't} use the math expression `|-x^2|'. It is necessary to use `|0-x^2|' instead. The same holds for
		`|exp(-x^2)|'; use `|exp(0-x^2)|' instead.

		This will be fixed with \PGF\ $>2.00$.

	\item Starting with \PGFPlots\ $1.1$, |\tikzstyle| should \emph{no longer be used} to set \PGFPlots\ options.
	
	Although |\tikzstyle| is still supported for some older \PGFPlots\ options, you should replace any occurance of |\tikzstyle| with |\pgfplotsset{|\meta{style name}|/.style={|\meta{key-value-list}|}}| or the associated |/.append style| variant. See section~\ref{sec:styles} for more detail.
\end{enumerate}
I apologize for any inconvenience caused by these changes.

\begin{pgfplotskey}{compat=\mchoice{1.3,pre 1.3,default,newest} (initially default)}
	The preamble configuration 
\begin{codeexample}[code only]
\usepackage{pgfplots}
\pgfplotsset{compat=1.3}
\end{codeexample}
	allows to choose between backwards compatibility and most recent features.

	\PGFPlots\ version 1.3 comes with new features which may lead to different spacings compared to earlier versions:
	\begin{enumerate}
		\item Version 1.3 comes with the |xlabel near ticks| feature which places (in this case $x$) axis labels ``near'' the tick labels, taking the size of tick labels into account.
		
		Older versions used |xlabel absolute|, i.e.\ an absolute distance between the axis and axis labels (now matter how large or small tick labels are).

		The initial setting \emph{keeps backwards compatibility}.

		You are encouraged to use
\begin{codeexample}[code only]
\usepackage{pgfplots}
\pgfplotsset{compat=1.3}
\end{codeexample}
		in your preamble.
		
		\item Version 1.3 fixes a bug with |anchor=south west| which occurred mainly for reversed axes: the anchors where upside--down. This is fixed now.

		
		If you have written some work--around and want to keep it going, use

		|\pgfplotsset{compat/anchors=pre 1.3}|\pgfmanualpdflabel{/pgfplots/compat/anchors}{} 

		to disable the bug fix.
	\end{enumerate}

	Use |\pgfplotsset{compat=default}| to restore the factory settings.

	The setting |\pgfplotsset{compat=newest}| will always use features of the most recent version. This might result in small changes in the document's appearance.
\end{pgfplotskey}

\subsection{The Team}
\PGFPlots\ has been written mainly by Christian Feuers�nger with many improvements of Pascal Wolkotte and Nick Papior Andersen as a spare time project. We hope it is useful and provides valuable plots.

If you are interested in writing something but don't know how, consider reading the auxiliary manual \href{file:TeX-programming-notes.pdf}{TeX-programming-notes.pdf} which comes with \PGFPlots. It is far from complete, but maybe it is a good starting point (at least for more literature).

\subsection{Acknowledgements}
I thank God for all hours of enjoyed programming. I thank Pascal Wolkotte and Nick Papior Andersen for their programming efforts and contributions as part of the development team. I thank Stefan Tibus, who contributed the |plot shell| feature. I thank Tom Cashman for the contribution of the |reverse legend| feature. Special thanks go to Stefan Pinnow whose tests of \PGFPlots\ lead to numerous quality improvements. Furthermore, I thank Dr.~Schweitzer for many fruitful discussions and Dr.~Meine for his ideas and suggestions. Thanks as well to the many international contributors who provided feature requests or identified bugs or simply manual improvements!

Last but not least, I thank Till Tantau and Mark Wibrow for their excellent graphics (and more) package \PGF\ and \Tikz\ which is the base of \PGFPlots.

\include{pgfplots.install}%
\include{pgfplots.intro}%
\include{pgfplots.reference}%
\include{pgfplots.libs}%
\include{pgfplots.resources}%
\include{pgfplots.importexport}%
\include{pgfplots.basic.reference}%

\printindex

\bibliographystyle{abbrv} %gerapali} %gerabbrv} %gerunsrt.bst} %gerabbrv}% gerplain}
\nocite{pgfplotstable}
\nocite{programmingnotes}
\bibliography{pgfplots}
\end{document}
%
%%%%%%%%%%%%%%%%%%%%%%%%%%%%%%%%%%%%%%%%%%%%%%%%%%%%%%%%%%%%%%%%%%%%%%%%%%%%%
%
% Package pgfplots.sty documentation. 
%
% Copyright 2007/2008 by Christian Feuersaenger.
%
% This program is free software: you can redistribute it and/or modify
% it under the terms of the GNU General Public License as published by
% the Free Software Foundation, either version 3 of the License, or
% (at your option) any later version.
% 
% This program is distributed in the hope that it will be useful,
% but WITHOUT ANY WARRANTY; without even the implied warranty of
% MERCHANTABILITY or FITNESS FOR A PARTICULAR PURPOSE.  See the
% GNU General Public License for more details.
% 
% You should have received a copy of the GNU General Public License
% along with this program.  If not, see <http://www.gnu.org/licenses/>.
%
%
%%%%%%%%%%%%%%%%%%%%%%%%%%%%%%%%%%%%%%%%%%%%%%%%%%%%%%%%%%%%%%%%%%%%%%%%%%%%%
\input{pgfplots.preamble.tex}
%\RequirePackage[german,english,francais]{babel}

\pgfplotsmanualexternalexpensivetrue

%\usetikzlibrary{external}
%\tikzexternalize[prefix=figures/]{pgfplots}

\title{%
	Manual for Package \PGFPlots\\
	{\small Version \pgfplotsversion}\\
	{\small\href{http://sourceforge.net/projects/pgfplots}{http://sourceforge.net/projects/pgfplots}}}

%\includeonly{pgfplots.libs}


\begin{document}

\def\plotcoords{%
\addplot coordinates {
(5,8.312e-02)    (17,2.547e-02)   (49,7.407e-03)
(129,2.102e-03)  (321,5.874e-04)  (769,1.623e-04)
(1793,4.442e-05) (4097,1.207e-05) (9217,3.261e-06)
};

\addplot coordinates{
(7,8.472e-02)    (31,3.044e-02)    (111,1.022e-02)
(351,3.303e-03)  (1023,1.039e-03)  (2815,3.196e-04)
(7423,9.658e-05) (18943,2.873e-05) (47103,8.437e-06)
};

\addplot coordinates{
(9,7.881e-02)     (49,3.243e-02)    (209,1.232e-02)
(769,4.454e-03)   (2561,1.551e-03)  (7937,5.236e-04)
(23297,1.723e-04) (65537,5.545e-05) (178177,1.751e-05)
};

\addplot coordinates{
(11,6.887e-02)    (71,3.177e-02)     (351,1.341e-02)
(1471,5.334e-03)  (5503,2.027e-03)   (18943,7.415e-04)
(61183,2.628e-04) (187903,9.063e-05) (553983,3.053e-05)
};

\addplot coordinates{
(13,5.755e-02)     (97,2.925e-02)     (545,1.351e-02)
(2561,5.842e-03)   (10625,2.397e-03)  (40193,9.414e-04)
(141569,3.564e-04) (471041,1.308e-04) 
(1496065,4.670e-05)
};
}%
\maketitle
\begin{abstract}%
\PGFPlots\ draws high--quality function plots in normal or logarithmic scaling with a user-friendly interface directly in \TeX. The user supplies axis labels, legend entries and the plot coordinates for one or more plots and \PGFPlots\ applies axis scaling, computes any logarithms and axis ticks and draws the plots, supporting line plots, scatter plots, piecewise constant plots, bar plots, area plots, mesh-- and surface plots and some more. It is based on Till Tantau's package \PGF/\Tikz.
\end{abstract}
\tableofcontents
\section{Introduction}
This package provides tools to generate plots and labeled axes easily. It draws normal plots, logplots and semi-logplots, in two and three dimensions. Axis ticks, labels, legends (in case of multiple plots) can be added with key-value options. It can cycle through a set of predefined line/marker/color specifications. In summary, its purpose is to simplify the generation of high-quality function and/or data plots, and solving the problems of
\begin{itemize}
	\item consistency of document and font type and font size,
	\item direct use of \TeX\ math mode in axis descriptions,
	\item consistency of data and figures (no third party tool necessary),
	\item inter document consistency using preamble configurations and styles.
\end{itemize}
Although not necessary, separate |.pdf| or |.eps| graphics can be generated using the |external| library developed as part of \Tikz.

\PGFPlots\ is build completely on \Tikz/\PGF. Knowledge of \Tikz\ will simplify the work with \PGFPlots, although it is not required.

\section[About PGFPlots: Preliminaries]{About {\normalfont\PGFPlots}: Preliminaries}
This section contains information about upgrades, the team, the installation (in case you need to do it manually) and troubleshooting. You may skip it completely except for the upgrade remarks.

However, note that this library requires at least \PGF\ version $2.00$. At the time of this writing, many \TeX-distributions still contain the older \PGF\ version $1.18$, so it may be necessary to install a recent \PGF\ prior to using \PGFPlots.

\subsection{Components}
\PGFPlots\ comes with two components:
\begin{enumerate}
	\item the plotting component (which you are currently reading) and
	\item the \PGFPlotstable\ component which simplifies number formatting and postprocessing of numerical tables. It comes as a separate package and has its own manual \href{file:pgfplotstable.pdf}{pgfplotstable.pdf}.
\end{enumerate}

\subsection{Upgrade remarks}
This release provides a lot of improvements which can be found in all detail in \texttt{ChangeLog} for interested readers. However, some attention is useful with respect to the following changes.

\subsubsection{New Optional Features}
\begin{enumerate}
	\item \PGFPlots\ 1.3 comes with user interface improvements. The technical distinction between ``behavior options'' and ``style options'' of older versions is no longer necessary (although still fully supported).

	\item \PGFPlots\ 1.3 has a new feature which allows to \emph{move axis labels tight to tick labels} automatically.

	Since this affects the spacing, it is not enabled be default.

	Use
\begin{codeexample}[code only]
\usepackage{pgfplots}
\pgfplotsset{compat=1.3}
\end{codeexample}
	in your preamble to benefit from the improved spacing\footnote{The |compat=1.3| setting is necessary for new features which move axis labels around like reversed axis. However, \PGFPlots\ will still work as it does in earlier versions, your documents will remain the same if you don't set it explicitly.}.
	Take a look at the next page for the precise definition of |compat|.

	\item \PGFPlots\ 1.3 now supports reversed axes. It is no longer necessary to use work--arounds with negative units.
\pgfkeys{/pdflinks/search key prefixes in/.add={/pgfplots/,}{}}

	Take a look at the |x dir=reverse| key.

	Existing work--arounds will still function properly. Use |\pgfplotsset{compat=1.3}| together with |x dir=reverse| to switch to the new version.
\end{enumerate}

\subsubsection{Old Features Which May Need Attention}
\begin{enumerate}
	\item The |scatter/classes| feature produces proper legends as of version 1.3. This may change the appearance of existing legends of plots with |scatter/classes|.

	\item Due to a small math bug in \PGF\ $2.00$, you \emph{can't} use the math expression `|-x^2|'. It is necessary to use `|0-x^2|' instead. The same holds for
		`|exp(-x^2)|'; use `|exp(0-x^2)|' instead.

		This will be fixed with \PGF\ $>2.00$.

	\item Starting with \PGFPlots\ $1.1$, |\tikzstyle| should \emph{no longer be used} to set \PGFPlots\ options.
	
	Although |\tikzstyle| is still supported for some older \PGFPlots\ options, you should replace any occurance of |\tikzstyle| with |\pgfplotsset{|\meta{style name}|/.style={|\meta{key-value-list}|}}| or the associated |/.append style| variant. See section~\ref{sec:styles} for more detail.
\end{enumerate}
I apologize for any inconvenience caused by these changes.

\begin{pgfplotskey}{compat=\mchoice{1.3,pre 1.3,default,newest} (initially default)}
	The preamble configuration 
\begin{codeexample}[code only]
\usepackage{pgfplots}
\pgfplotsset{compat=1.3}
\end{codeexample}
	allows to choose between backwards compatibility and most recent features.

	\PGFPlots\ version 1.3 comes with new features which may lead to different spacings compared to earlier versions:
	\begin{enumerate}
		\item Version 1.3 comes with the |xlabel near ticks| feature which places (in this case $x$) axis labels ``near'' the tick labels, taking the size of tick labels into account.
		
		Older versions used |xlabel absolute|, i.e.\ an absolute distance between the axis and axis labels (now matter how large or small tick labels are).

		The initial setting \emph{keeps backwards compatibility}.

		You are encouraged to use
\begin{codeexample}[code only]
\usepackage{pgfplots}
\pgfplotsset{compat=1.3}
\end{codeexample}
		in your preamble.
		
		\item Version 1.3 fixes a bug with |anchor=south west| which occurred mainly for reversed axes: the anchors where upside--down. This is fixed now.

		
		If you have written some work--around and want to keep it going, use

		|\pgfplotsset{compat/anchors=pre 1.3}|\pgfmanualpdflabel{/pgfplots/compat/anchors}{} 

		to disable the bug fix.
	\end{enumerate}

	Use |\pgfplotsset{compat=default}| to restore the factory settings.

	The setting |\pgfplotsset{compat=newest}| will always use features of the most recent version. This might result in small changes in the document's appearance.
\end{pgfplotskey}

\subsection{The Team}
\PGFPlots\ has been written mainly by Christian Feuers�nger with many improvements of Pascal Wolkotte and Nick Papior Andersen as a spare time project. We hope it is useful and provides valuable plots.

If you are interested in writing something but don't know how, consider reading the auxiliary manual \href{file:TeX-programming-notes.pdf}{TeX-programming-notes.pdf} which comes with \PGFPlots. It is far from complete, but maybe it is a good starting point (at least for more literature).

\subsection{Acknowledgements}
I thank God for all hours of enjoyed programming. I thank Pascal Wolkotte and Nick Papior Andersen for their programming efforts and contributions as part of the development team. I thank Stefan Tibus, who contributed the |plot shell| feature. I thank Tom Cashman for the contribution of the |reverse legend| feature. Special thanks go to Stefan Pinnow whose tests of \PGFPlots\ lead to numerous quality improvements. Furthermore, I thank Dr.~Schweitzer for many fruitful discussions and Dr.~Meine for his ideas and suggestions. Thanks as well to the many international contributors who provided feature requests or identified bugs or simply manual improvements!

Last but not least, I thank Till Tantau and Mark Wibrow for their excellent graphics (and more) package \PGF\ and \Tikz\ which is the base of \PGFPlots.

\include{pgfplots.install}%
\include{pgfplots.intro}%
\include{pgfplots.reference}%
\include{pgfplots.libs}%
\include{pgfplots.resources}%
\include{pgfplots.importexport}%
\include{pgfplots.basic.reference}%

\printindex

\bibliographystyle{abbrv} %gerapali} %gerabbrv} %gerunsrt.bst} %gerabbrv}% gerplain}
\nocite{pgfplotstable}
\nocite{programmingnotes}
\bibliography{pgfplots}
\end{document}
%
%%%%%%%%%%%%%%%%%%%%%%%%%%%%%%%%%%%%%%%%%%%%%%%%%%%%%%%%%%%%%%%%%%%%%%%%%%%%%
%
% Package pgfplots.sty documentation. 
%
% Copyright 2007/2008 by Christian Feuersaenger.
%
% This program is free software: you can redistribute it and/or modify
% it under the terms of the GNU General Public License as published by
% the Free Software Foundation, either version 3 of the License, or
% (at your option) any later version.
% 
% This program is distributed in the hope that it will be useful,
% but WITHOUT ANY WARRANTY; without even the implied warranty of
% MERCHANTABILITY or FITNESS FOR A PARTICULAR PURPOSE.  See the
% GNU General Public License for more details.
% 
% You should have received a copy of the GNU General Public License
% along with this program.  If not, see <http://www.gnu.org/licenses/>.
%
%
%%%%%%%%%%%%%%%%%%%%%%%%%%%%%%%%%%%%%%%%%%%%%%%%%%%%%%%%%%%%%%%%%%%%%%%%%%%%%
\input{pgfplots.preamble.tex}
%\RequirePackage[german,english,francais]{babel}

\pgfplotsmanualexternalexpensivetrue

%\usetikzlibrary{external}
%\tikzexternalize[prefix=figures/]{pgfplots}

\title{%
	Manual for Package \PGFPlots\\
	{\small Version \pgfplotsversion}\\
	{\small\href{http://sourceforge.net/projects/pgfplots}{http://sourceforge.net/projects/pgfplots}}}

%\includeonly{pgfplots.libs}


\begin{document}

\def\plotcoords{%
\addplot coordinates {
(5,8.312e-02)    (17,2.547e-02)   (49,7.407e-03)
(129,2.102e-03)  (321,5.874e-04)  (769,1.623e-04)
(1793,4.442e-05) (4097,1.207e-05) (9217,3.261e-06)
};

\addplot coordinates{
(7,8.472e-02)    (31,3.044e-02)    (111,1.022e-02)
(351,3.303e-03)  (1023,1.039e-03)  (2815,3.196e-04)
(7423,9.658e-05) (18943,2.873e-05) (47103,8.437e-06)
};

\addplot coordinates{
(9,7.881e-02)     (49,3.243e-02)    (209,1.232e-02)
(769,4.454e-03)   (2561,1.551e-03)  (7937,5.236e-04)
(23297,1.723e-04) (65537,5.545e-05) (178177,1.751e-05)
};

\addplot coordinates{
(11,6.887e-02)    (71,3.177e-02)     (351,1.341e-02)
(1471,5.334e-03)  (5503,2.027e-03)   (18943,7.415e-04)
(61183,2.628e-04) (187903,9.063e-05) (553983,3.053e-05)
};

\addplot coordinates{
(13,5.755e-02)     (97,2.925e-02)     (545,1.351e-02)
(2561,5.842e-03)   (10625,2.397e-03)  (40193,9.414e-04)
(141569,3.564e-04) (471041,1.308e-04) 
(1496065,4.670e-05)
};
}%
\maketitle
\begin{abstract}%
\PGFPlots\ draws high--quality function plots in normal or logarithmic scaling with a user-friendly interface directly in \TeX. The user supplies axis labels, legend entries and the plot coordinates for one or more plots and \PGFPlots\ applies axis scaling, computes any logarithms and axis ticks and draws the plots, supporting line plots, scatter plots, piecewise constant plots, bar plots, area plots, mesh-- and surface plots and some more. It is based on Till Tantau's package \PGF/\Tikz.
\end{abstract}
\tableofcontents
\section{Introduction}
This package provides tools to generate plots and labeled axes easily. It draws normal plots, logplots and semi-logplots, in two and three dimensions. Axis ticks, labels, legends (in case of multiple plots) can be added with key-value options. It can cycle through a set of predefined line/marker/color specifications. In summary, its purpose is to simplify the generation of high-quality function and/or data plots, and solving the problems of
\begin{itemize}
	\item consistency of document and font type and font size,
	\item direct use of \TeX\ math mode in axis descriptions,
	\item consistency of data and figures (no third party tool necessary),
	\item inter document consistency using preamble configurations and styles.
\end{itemize}
Although not necessary, separate |.pdf| or |.eps| graphics can be generated using the |external| library developed as part of \Tikz.

\PGFPlots\ is build completely on \Tikz/\PGF. Knowledge of \Tikz\ will simplify the work with \PGFPlots, although it is not required.

\section[About PGFPlots: Preliminaries]{About {\normalfont\PGFPlots}: Preliminaries}
This section contains information about upgrades, the team, the installation (in case you need to do it manually) and troubleshooting. You may skip it completely except for the upgrade remarks.

However, note that this library requires at least \PGF\ version $2.00$. At the time of this writing, many \TeX-distributions still contain the older \PGF\ version $1.18$, so it may be necessary to install a recent \PGF\ prior to using \PGFPlots.

\subsection{Components}
\PGFPlots\ comes with two components:
\begin{enumerate}
	\item the plotting component (which you are currently reading) and
	\item the \PGFPlotstable\ component which simplifies number formatting and postprocessing of numerical tables. It comes as a separate package and has its own manual \href{file:pgfplotstable.pdf}{pgfplotstable.pdf}.
\end{enumerate}

\subsection{Upgrade remarks}
This release provides a lot of improvements which can be found in all detail in \texttt{ChangeLog} for interested readers. However, some attention is useful with respect to the following changes.

\subsubsection{New Optional Features}
\begin{enumerate}
	\item \PGFPlots\ 1.3 comes with user interface improvements. The technical distinction between ``behavior options'' and ``style options'' of older versions is no longer necessary (although still fully supported).

	\item \PGFPlots\ 1.3 has a new feature which allows to \emph{move axis labels tight to tick labels} automatically.

	Since this affects the spacing, it is not enabled be default.

	Use
\begin{codeexample}[code only]
\usepackage{pgfplots}
\pgfplotsset{compat=1.3}
\end{codeexample}
	in your preamble to benefit from the improved spacing\footnote{The |compat=1.3| setting is necessary for new features which move axis labels around like reversed axis. However, \PGFPlots\ will still work as it does in earlier versions, your documents will remain the same if you don't set it explicitly.}.
	Take a look at the next page for the precise definition of |compat|.

	\item \PGFPlots\ 1.3 now supports reversed axes. It is no longer necessary to use work--arounds with negative units.
\pgfkeys{/pdflinks/search key prefixes in/.add={/pgfplots/,}{}}

	Take a look at the |x dir=reverse| key.

	Existing work--arounds will still function properly. Use |\pgfplotsset{compat=1.3}| together with |x dir=reverse| to switch to the new version.
\end{enumerate}

\subsubsection{Old Features Which May Need Attention}
\begin{enumerate}
	\item The |scatter/classes| feature produces proper legends as of version 1.3. This may change the appearance of existing legends of plots with |scatter/classes|.

	\item Due to a small math bug in \PGF\ $2.00$, you \emph{can't} use the math expression `|-x^2|'. It is necessary to use `|0-x^2|' instead. The same holds for
		`|exp(-x^2)|'; use `|exp(0-x^2)|' instead.

		This will be fixed with \PGF\ $>2.00$.

	\item Starting with \PGFPlots\ $1.1$, |\tikzstyle| should \emph{no longer be used} to set \PGFPlots\ options.
	
	Although |\tikzstyle| is still supported for some older \PGFPlots\ options, you should replace any occurance of |\tikzstyle| with |\pgfplotsset{|\meta{style name}|/.style={|\meta{key-value-list}|}}| or the associated |/.append style| variant. See section~\ref{sec:styles} for more detail.
\end{enumerate}
I apologize for any inconvenience caused by these changes.

\begin{pgfplotskey}{compat=\mchoice{1.3,pre 1.3,default,newest} (initially default)}
	The preamble configuration 
\begin{codeexample}[code only]
\usepackage{pgfplots}
\pgfplotsset{compat=1.3}
\end{codeexample}
	allows to choose between backwards compatibility and most recent features.

	\PGFPlots\ version 1.3 comes with new features which may lead to different spacings compared to earlier versions:
	\begin{enumerate}
		\item Version 1.3 comes with the |xlabel near ticks| feature which places (in this case $x$) axis labels ``near'' the tick labels, taking the size of tick labels into account.
		
		Older versions used |xlabel absolute|, i.e.\ an absolute distance between the axis and axis labels (now matter how large or small tick labels are).

		The initial setting \emph{keeps backwards compatibility}.

		You are encouraged to use
\begin{codeexample}[code only]
\usepackage{pgfplots}
\pgfplotsset{compat=1.3}
\end{codeexample}
		in your preamble.
		
		\item Version 1.3 fixes a bug with |anchor=south west| which occurred mainly for reversed axes: the anchors where upside--down. This is fixed now.

		
		If you have written some work--around and want to keep it going, use

		|\pgfplotsset{compat/anchors=pre 1.3}|\pgfmanualpdflabel{/pgfplots/compat/anchors}{} 

		to disable the bug fix.
	\end{enumerate}

	Use |\pgfplotsset{compat=default}| to restore the factory settings.

	The setting |\pgfplotsset{compat=newest}| will always use features of the most recent version. This might result in small changes in the document's appearance.
\end{pgfplotskey}

\subsection{The Team}
\PGFPlots\ has been written mainly by Christian Feuers�nger with many improvements of Pascal Wolkotte and Nick Papior Andersen as a spare time project. We hope it is useful and provides valuable plots.

If you are interested in writing something but don't know how, consider reading the auxiliary manual \href{file:TeX-programming-notes.pdf}{TeX-programming-notes.pdf} which comes with \PGFPlots. It is far from complete, but maybe it is a good starting point (at least for more literature).

\subsection{Acknowledgements}
I thank God for all hours of enjoyed programming. I thank Pascal Wolkotte and Nick Papior Andersen for their programming efforts and contributions as part of the development team. I thank Stefan Tibus, who contributed the |plot shell| feature. I thank Tom Cashman for the contribution of the |reverse legend| feature. Special thanks go to Stefan Pinnow whose tests of \PGFPlots\ lead to numerous quality improvements. Furthermore, I thank Dr.~Schweitzer for many fruitful discussions and Dr.~Meine for his ideas and suggestions. Thanks as well to the many international contributors who provided feature requests or identified bugs or simply manual improvements!

Last but not least, I thank Till Tantau and Mark Wibrow for their excellent graphics (and more) package \PGF\ and \Tikz\ which is the base of \PGFPlots.

\include{pgfplots.install}%
\include{pgfplots.intro}%
\include{pgfplots.reference}%
\include{pgfplots.libs}%
\include{pgfplots.resources}%
\include{pgfplots.importexport}%
\include{pgfplots.basic.reference}%

\printindex

\bibliographystyle{abbrv} %gerapali} %gerabbrv} %gerunsrt.bst} %gerabbrv}% gerplain}
\nocite{pgfplotstable}
\nocite{programmingnotes}
\bibliography{pgfplots}
\end{document}
%
\include{pgfplots.basic.reference}%

\printindex

\bibliographystyle{abbrv} %gerapali} %gerabbrv} %gerunsrt.bst} %gerabbrv}% gerplain}
\nocite{pgfplotstable}
\nocite{programmingnotes}
\bibliography{pgfplots}
\end{document}
%
%%%%%%%%%%%%%%%%%%%%%%%%%%%%%%%%%%%%%%%%%%%%%%%%%%%%%%%%%%%%%%%%%%%%%%%%%%%%%
%
% Package pgfplots.sty documentation. 
%
% Copyright 2007/2008 by Christian Feuersaenger.
%
% This program is free software: you can redistribute it and/or modify
% it under the terms of the GNU General Public License as published by
% the Free Software Foundation, either version 3 of the License, or
% (at your option) any later version.
% 
% This program is distributed in the hope that it will be useful,
% but WITHOUT ANY WARRANTY; without even the implied warranty of
% MERCHANTABILITY or FITNESS FOR A PARTICULAR PURPOSE.  See the
% GNU General Public License for more details.
% 
% You should have received a copy of the GNU General Public License
% along with this program.  If not, see <http://www.gnu.org/licenses/>.
%
%
%%%%%%%%%%%%%%%%%%%%%%%%%%%%%%%%%%%%%%%%%%%%%%%%%%%%%%%%%%%%%%%%%%%%%%%%%%%%%
%%%%%%%%%%%%%%%%%%%%%%%%%%%%%%%%%%%%%%%%%%%%%%%%%%%%%%%%%%%%%%%%%%%%%%%%%%%%%
%
% Package pgfplots.sty documentation. 
%
% Copyright 2007/2008 by Christian Feuersaenger.
%
% This program is free software: you can redistribute it and/or modify
% it under the terms of the GNU General Public License as published by
% the Free Software Foundation, either version 3 of the License, or
% (at your option) any later version.
% 
% This program is distributed in the hope that it will be useful,
% but WITHOUT ANY WARRANTY; without even the implied warranty of
% MERCHANTABILITY or FITNESS FOR A PARTICULAR PURPOSE.  See the
% GNU General Public License for more details.
% 
% You should have received a copy of the GNU General Public License
% along with this program.  If not, see <http://www.gnu.org/licenses/>.
%
%
%%%%%%%%%%%%%%%%%%%%%%%%%%%%%%%%%%%%%%%%%%%%%%%%%%%%%%%%%%%%%%%%%%%%%%%%%%%%%
\documentclass[a4paper]{ltxdoc}

\usepackage{makeidx}

% DON't let hyperref overload the format of index and glossary. 
% I want to do that on my own in the stylefiles for makeindex...
\makeatletter
\let\@old@wrindex=\@wrindex
\makeatother

\usepackage{ifpdf}
\ifpdf
	\usepackage[pdftex]{hyperref}
\else
	\usepackage[dvipdfm]{hyperref}
\fi
\hypersetup{pdfborder={0 0 0}}

\makeatletter
\let\@wrindex=\@old@wrindex
\makeatother

% Formatiere Seitennummern im Index:
\newcommand{\indexpageno}[1]{%
	{\bfseries\hyperpage{#1}}%
}


\newcommand{\R}{\mathbb{R}}
\newcommand{\N}{\mathbb{N}}
\newcommand{\Z}{\mathbb{Z}}

\long\def\COMMENTLOWLEVEL#1\ENDCOMMENT{}
\def\ENDCOMMENT{}

\usepackage{textcomp}
\usepackage{booktabs}

\usepackage{calc}
\usepackage[formats]{listings}
%\usepackage{courier} % don't use it - the '^' character can't be copy-pasted in courier

\usepackage{array}
\lstset{%
	basicstyle=\ttfamily,
	language=[LaTeX]tex, % Seems as if \lstset{language=tex} must be invoked BEFORE loading tikz!?
	tabsize=4,
	breaklines=true,
	breakindent=0pt
}

\ifpdf
	\pdfinfo {
		/Author	(Christian Feuersaenger)
	}

\else
	\def\pgfsysdriver{pgfsys-dvipdfm.def}
\fi
%\def\pgfsysdriver{pgfsys-pdftex.def}
\usepackage{pgfplots}

\ifpdf
	\usetikzlibrary{pgfplotsclickable}
\fi

%\usepackage{fp}
% ATTENTION:
% this requires pgf version NEWER than 2.00 :
%\usetikzlibrary{fixedpointarithmetic}

\usetikzlibrary{dateplot}

\usepackage[a4paper,left=2.25cm,right=2.25cm,top=2.5cm,bottom=2.5cm,nohead]{geometry}
\usepackage{amsmath,amssymb}
\usepackage{xxcolor}
\usepackage{pifont}
\usepackage[latin1]{inputenc}
\usepackage{amsmath}
\usepackage{eurosym}
\input{pgfplots-macros}

\pgfqkeys{/codeexample}{%
	every codeexample/.style={
		width=8cm,
		/pgfplots/every axis/.append style={legend style={fill=graphicbackground}}
	},
	tabsize=4,
}
\pgfplotsset{
	every axis/.append style={width=7cm},
	filter discard warning=false,
}

\usetikzlibrary{backgrounds,patterns}
% Global styles:
\tikzset{
  shape example/.style={
    color=black!30,
    draw,
    fill=yellow!30,
    line width=.5cm,
    inner xsep=2.5cm,
    inner ysep=0.5cm}
}

\newcommand{\FIXME}[1]{\textcolor{red}{(FIXME: #1)}}

% fuer endvironment 'sidewaysfigure' bspw
% \usepackage{rotating}

\newcommand\Tikz{Ti\textit kZ}
\newcommand\PGF{\textsc{pgf}}
\newcommand\PGFPlots{\textsc{pgfplots}}
\def\PGFPlotstable{\textsc{PgfplotsTable}}


\makeindex

% Fix overful hboxes automatically:
\tolerance=2000
\emergencystretch=10pt

\tikzset{prefix=gnuplot/pgfplots_} % prefix for 'plot function'

\author{%
	Christian Feuers\"anger\footnote{\url{http://wissrech.ins.uni-bonn.de/people/feuersaenger}}\\%
	Institut f\"ur Numerische Simulation\\
	Universit\"at Bonn, Germany}


%\RequirePackage[german,english,francais]{babel}

\pgfplotsmanualexternalexpensivetrue

%\usetikzlibrary{external}
%\tikzexternalize[prefix=figures/]{pgfplots}

\title{%
	Manual for Package \PGFPlots\\
	{\small Version \pgfplotsversion}\\
	{\small\href{http://sourceforge.net/projects/pgfplots}{http://sourceforge.net/projects/pgfplots}}}

%\includeonly{pgfplots.libs}


\begin{document}

\def\plotcoords{%
\addplot coordinates {
(5,8.312e-02)    (17,2.547e-02)   (49,7.407e-03)
(129,2.102e-03)  (321,5.874e-04)  (769,1.623e-04)
(1793,4.442e-05) (4097,1.207e-05) (9217,3.261e-06)
};

\addplot coordinates{
(7,8.472e-02)    (31,3.044e-02)    (111,1.022e-02)
(351,3.303e-03)  (1023,1.039e-03)  (2815,3.196e-04)
(7423,9.658e-05) (18943,2.873e-05) (47103,8.437e-06)
};

\addplot coordinates{
(9,7.881e-02)     (49,3.243e-02)    (209,1.232e-02)
(769,4.454e-03)   (2561,1.551e-03)  (7937,5.236e-04)
(23297,1.723e-04) (65537,5.545e-05) (178177,1.751e-05)
};

\addplot coordinates{
(11,6.887e-02)    (71,3.177e-02)     (351,1.341e-02)
(1471,5.334e-03)  (5503,2.027e-03)   (18943,7.415e-04)
(61183,2.628e-04) (187903,9.063e-05) (553983,3.053e-05)
};

\addplot coordinates{
(13,5.755e-02)     (97,2.925e-02)     (545,1.351e-02)
(2561,5.842e-03)   (10625,2.397e-03)  (40193,9.414e-04)
(141569,3.564e-04) (471041,1.308e-04) 
(1496065,4.670e-05)
};
}%
\maketitle
\begin{abstract}%
\PGFPlots\ draws high--quality function plots in normal or logarithmic scaling with a user-friendly interface directly in \TeX. The user supplies axis labels, legend entries and the plot coordinates for one or more plots and \PGFPlots\ applies axis scaling, computes any logarithms and axis ticks and draws the plots, supporting line plots, scatter plots, piecewise constant plots, bar plots, area plots, mesh-- and surface plots and some more. It is based on Till Tantau's package \PGF/\Tikz.
\end{abstract}
\tableofcontents
\section{Introduction}
This package provides tools to generate plots and labeled axes easily. It draws normal plots, logplots and semi-logplots, in two and three dimensions. Axis ticks, labels, legends (in case of multiple plots) can be added with key-value options. It can cycle through a set of predefined line/marker/color specifications. In summary, its purpose is to simplify the generation of high-quality function and/or data plots, and solving the problems of
\begin{itemize}
	\item consistency of document and font type and font size,
	\item direct use of \TeX\ math mode in axis descriptions,
	\item consistency of data and figures (no third party tool necessary),
	\item inter document consistency using preamble configurations and styles.
\end{itemize}
Although not necessary, separate |.pdf| or |.eps| graphics can be generated using the |external| library developed as part of \Tikz.

\PGFPlots\ is build completely on \Tikz/\PGF. Knowledge of \Tikz\ will simplify the work with \PGFPlots, although it is not required.

\section[About PGFPlots: Preliminaries]{About {\normalfont\PGFPlots}: Preliminaries}
This section contains information about upgrades, the team, the installation (in case you need to do it manually) and troubleshooting. You may skip it completely except for the upgrade remarks.

However, note that this library requires at least \PGF\ version $2.00$. At the time of this writing, many \TeX-distributions still contain the older \PGF\ version $1.18$, so it may be necessary to install a recent \PGF\ prior to using \PGFPlots.

\subsection{Components}
\PGFPlots\ comes with two components:
\begin{enumerate}
	\item the plotting component (which you are currently reading) and
	\item the \PGFPlotstable\ component which simplifies number formatting and postprocessing of numerical tables. It comes as a separate package and has its own manual \href{file:pgfplotstable.pdf}{pgfplotstable.pdf}.
\end{enumerate}

\subsection{Upgrade remarks}
This release provides a lot of improvements which can be found in all detail in \texttt{ChangeLog} for interested readers. However, some attention is useful with respect to the following changes.

\subsubsection{New Optional Features}
\begin{enumerate}
	\item \PGFPlots\ 1.3 comes with user interface improvements. The technical distinction between ``behavior options'' and ``style options'' of older versions is no longer necessary (although still fully supported).

	\item \PGFPlots\ 1.3 has a new feature which allows to \emph{move axis labels tight to tick labels} automatically.

	Since this affects the spacing, it is not enabled be default.

	Use
\begin{codeexample}[code only]
\usepackage{pgfplots}
\pgfplotsset{compat=1.3}
\end{codeexample}
	in your preamble to benefit from the improved spacing\footnote{The |compat=1.3| setting is necessary for new features which move axis labels around like reversed axis. However, \PGFPlots\ will still work as it does in earlier versions, your documents will remain the same if you don't set it explicitly.}.
	Take a look at the next page for the precise definition of |compat|.

	\item \PGFPlots\ 1.3 now supports reversed axes. It is no longer necessary to use work--arounds with negative units.
\pgfkeys{/pdflinks/search key prefixes in/.add={/pgfplots/,}{}}

	Take a look at the |x dir=reverse| key.

	Existing work--arounds will still function properly. Use |\pgfplotsset{compat=1.3}| together with |x dir=reverse| to switch to the new version.
\end{enumerate}

\subsubsection{Old Features Which May Need Attention}
\begin{enumerate}
	\item The |scatter/classes| feature produces proper legends as of version 1.3. This may change the appearance of existing legends of plots with |scatter/classes|.

	\item Due to a small math bug in \PGF\ $2.00$, you \emph{can't} use the math expression `|-x^2|'. It is necessary to use `|0-x^2|' instead. The same holds for
		`|exp(-x^2)|'; use `|exp(0-x^2)|' instead.

		This will be fixed with \PGF\ $>2.00$.

	\item Starting with \PGFPlots\ $1.1$, |\tikzstyle| should \emph{no longer be used} to set \PGFPlots\ options.
	
	Although |\tikzstyle| is still supported for some older \PGFPlots\ options, you should replace any occurance of |\tikzstyle| with |\pgfplotsset{|\meta{style name}|/.style={|\meta{key-value-list}|}}| or the associated |/.append style| variant. See section~\ref{sec:styles} for more detail.
\end{enumerate}
I apologize for any inconvenience caused by these changes.

\begin{pgfplotskey}{compat=\mchoice{1.3,pre 1.3,default,newest} (initially default)}
	The preamble configuration 
\begin{codeexample}[code only]
\usepackage{pgfplots}
\pgfplotsset{compat=1.3}
\end{codeexample}
	allows to choose between backwards compatibility and most recent features.

	\PGFPlots\ version 1.3 comes with new features which may lead to different spacings compared to earlier versions:
	\begin{enumerate}
		\item Version 1.3 comes with the |xlabel near ticks| feature which places (in this case $x$) axis labels ``near'' the tick labels, taking the size of tick labels into account.
		
		Older versions used |xlabel absolute|, i.e.\ an absolute distance between the axis and axis labels (now matter how large or small tick labels are).

		The initial setting \emph{keeps backwards compatibility}.

		You are encouraged to use
\begin{codeexample}[code only]
\usepackage{pgfplots}
\pgfplotsset{compat=1.3}
\end{codeexample}
		in your preamble.
		
		\item Version 1.3 fixes a bug with |anchor=south west| which occurred mainly for reversed axes: the anchors where upside--down. This is fixed now.

		
		If you have written some work--around and want to keep it going, use

		|\pgfplotsset{compat/anchors=pre 1.3}|\pgfmanualpdflabel{/pgfplots/compat/anchors}{} 

		to disable the bug fix.
	\end{enumerate}

	Use |\pgfplotsset{compat=default}| to restore the factory settings.

	The setting |\pgfplotsset{compat=newest}| will always use features of the most recent version. This might result in small changes in the document's appearance.
\end{pgfplotskey}

\subsection{The Team}
\PGFPlots\ has been written mainly by Christian Feuers�nger with many improvements of Pascal Wolkotte and Nick Papior Andersen as a spare time project. We hope it is useful and provides valuable plots.

If you are interested in writing something but don't know how, consider reading the auxiliary manual \href{file:TeX-programming-notes.pdf}{TeX-programming-notes.pdf} which comes with \PGFPlots. It is far from complete, but maybe it is a good starting point (at least for more literature).

\subsection{Acknowledgements}
I thank God for all hours of enjoyed programming. I thank Pascal Wolkotte and Nick Papior Andersen for their programming efforts and contributions as part of the development team. I thank Stefan Tibus, who contributed the |plot shell| feature. I thank Tom Cashman for the contribution of the |reverse legend| feature. Special thanks go to Stefan Pinnow whose tests of \PGFPlots\ lead to numerous quality improvements. Furthermore, I thank Dr.~Schweitzer for many fruitful discussions and Dr.~Meine for his ideas and suggestions. Thanks as well to the many international contributors who provided feature requests or identified bugs or simply manual improvements!

Last but not least, I thank Till Tantau and Mark Wibrow for their excellent graphics (and more) package \PGF\ and \Tikz\ which is the base of \PGFPlots.

%%%%%%%%%%%%%%%%%%%%%%%%%%%%%%%%%%%%%%%%%%%%%%%%%%%%%%%%%%%%%%%%%%%%%%%%%%%%%
%
% Package pgfplots.sty documentation. 
%
% Copyright 2007/2008 by Christian Feuersaenger.
%
% This program is free software: you can redistribute it and/or modify
% it under the terms of the GNU General Public License as published by
% the Free Software Foundation, either version 3 of the License, or
% (at your option) any later version.
% 
% This program is distributed in the hope that it will be useful,
% but WITHOUT ANY WARRANTY; without even the implied warranty of
% MERCHANTABILITY or FITNESS FOR A PARTICULAR PURPOSE.  See the
% GNU General Public License for more details.
% 
% You should have received a copy of the GNU General Public License
% along with this program.  If not, see <http://www.gnu.org/licenses/>.
%
%
%%%%%%%%%%%%%%%%%%%%%%%%%%%%%%%%%%%%%%%%%%%%%%%%%%%%%%%%%%%%%%%%%%%%%%%%%%%%%
\input{pgfplots.preamble.tex}
%\RequirePackage[german,english,francais]{babel}

\pgfplotsmanualexternalexpensivetrue

%\usetikzlibrary{external}
%\tikzexternalize[prefix=figures/]{pgfplots}

\title{%
	Manual for Package \PGFPlots\\
	{\small Version \pgfplotsversion}\\
	{\small\href{http://sourceforge.net/projects/pgfplots}{http://sourceforge.net/projects/pgfplots}}}

%\includeonly{pgfplots.libs}


\begin{document}

\def\plotcoords{%
\addplot coordinates {
(5,8.312e-02)    (17,2.547e-02)   (49,7.407e-03)
(129,2.102e-03)  (321,5.874e-04)  (769,1.623e-04)
(1793,4.442e-05) (4097,1.207e-05) (9217,3.261e-06)
};

\addplot coordinates{
(7,8.472e-02)    (31,3.044e-02)    (111,1.022e-02)
(351,3.303e-03)  (1023,1.039e-03)  (2815,3.196e-04)
(7423,9.658e-05) (18943,2.873e-05) (47103,8.437e-06)
};

\addplot coordinates{
(9,7.881e-02)     (49,3.243e-02)    (209,1.232e-02)
(769,4.454e-03)   (2561,1.551e-03)  (7937,5.236e-04)
(23297,1.723e-04) (65537,5.545e-05) (178177,1.751e-05)
};

\addplot coordinates{
(11,6.887e-02)    (71,3.177e-02)     (351,1.341e-02)
(1471,5.334e-03)  (5503,2.027e-03)   (18943,7.415e-04)
(61183,2.628e-04) (187903,9.063e-05) (553983,3.053e-05)
};

\addplot coordinates{
(13,5.755e-02)     (97,2.925e-02)     (545,1.351e-02)
(2561,5.842e-03)   (10625,2.397e-03)  (40193,9.414e-04)
(141569,3.564e-04) (471041,1.308e-04) 
(1496065,4.670e-05)
};
}%
\maketitle
\begin{abstract}%
\PGFPlots\ draws high--quality function plots in normal or logarithmic scaling with a user-friendly interface directly in \TeX. The user supplies axis labels, legend entries and the plot coordinates for one or more plots and \PGFPlots\ applies axis scaling, computes any logarithms and axis ticks and draws the plots, supporting line plots, scatter plots, piecewise constant plots, bar plots, area plots, mesh-- and surface plots and some more. It is based on Till Tantau's package \PGF/\Tikz.
\end{abstract}
\tableofcontents
\section{Introduction}
This package provides tools to generate plots and labeled axes easily. It draws normal plots, logplots and semi-logplots, in two and three dimensions. Axis ticks, labels, legends (in case of multiple plots) can be added with key-value options. It can cycle through a set of predefined line/marker/color specifications. In summary, its purpose is to simplify the generation of high-quality function and/or data plots, and solving the problems of
\begin{itemize}
	\item consistency of document and font type and font size,
	\item direct use of \TeX\ math mode in axis descriptions,
	\item consistency of data and figures (no third party tool necessary),
	\item inter document consistency using preamble configurations and styles.
\end{itemize}
Although not necessary, separate |.pdf| or |.eps| graphics can be generated using the |external| library developed as part of \Tikz.

\PGFPlots\ is build completely on \Tikz/\PGF. Knowledge of \Tikz\ will simplify the work with \PGFPlots, although it is not required.

\section[About PGFPlots: Preliminaries]{About {\normalfont\PGFPlots}: Preliminaries}
This section contains information about upgrades, the team, the installation (in case you need to do it manually) and troubleshooting. You may skip it completely except for the upgrade remarks.

However, note that this library requires at least \PGF\ version $2.00$. At the time of this writing, many \TeX-distributions still contain the older \PGF\ version $1.18$, so it may be necessary to install a recent \PGF\ prior to using \PGFPlots.

\subsection{Components}
\PGFPlots\ comes with two components:
\begin{enumerate}
	\item the plotting component (which you are currently reading) and
	\item the \PGFPlotstable\ component which simplifies number formatting and postprocessing of numerical tables. It comes as a separate package and has its own manual \href{file:pgfplotstable.pdf}{pgfplotstable.pdf}.
\end{enumerate}

\subsection{Upgrade remarks}
This release provides a lot of improvements which can be found in all detail in \texttt{ChangeLog} for interested readers. However, some attention is useful with respect to the following changes.

\subsubsection{New Optional Features}
\begin{enumerate}
	\item \PGFPlots\ 1.3 comes with user interface improvements. The technical distinction between ``behavior options'' and ``style options'' of older versions is no longer necessary (although still fully supported).

	\item \PGFPlots\ 1.3 has a new feature which allows to \emph{move axis labels tight to tick labels} automatically.

	Since this affects the spacing, it is not enabled be default.

	Use
\begin{codeexample}[code only]
\usepackage{pgfplots}
\pgfplotsset{compat=1.3}
\end{codeexample}
	in your preamble to benefit from the improved spacing\footnote{The |compat=1.3| setting is necessary for new features which move axis labels around like reversed axis. However, \PGFPlots\ will still work as it does in earlier versions, your documents will remain the same if you don't set it explicitly.}.
	Take a look at the next page for the precise definition of |compat|.

	\item \PGFPlots\ 1.3 now supports reversed axes. It is no longer necessary to use work--arounds with negative units.
\pgfkeys{/pdflinks/search key prefixes in/.add={/pgfplots/,}{}}

	Take a look at the |x dir=reverse| key.

	Existing work--arounds will still function properly. Use |\pgfplotsset{compat=1.3}| together with |x dir=reverse| to switch to the new version.
\end{enumerate}

\subsubsection{Old Features Which May Need Attention}
\begin{enumerate}
	\item The |scatter/classes| feature produces proper legends as of version 1.3. This may change the appearance of existing legends of plots with |scatter/classes|.

	\item Due to a small math bug in \PGF\ $2.00$, you \emph{can't} use the math expression `|-x^2|'. It is necessary to use `|0-x^2|' instead. The same holds for
		`|exp(-x^2)|'; use `|exp(0-x^2)|' instead.

		This will be fixed with \PGF\ $>2.00$.

	\item Starting with \PGFPlots\ $1.1$, |\tikzstyle| should \emph{no longer be used} to set \PGFPlots\ options.
	
	Although |\tikzstyle| is still supported for some older \PGFPlots\ options, you should replace any occurance of |\tikzstyle| with |\pgfplotsset{|\meta{style name}|/.style={|\meta{key-value-list}|}}| or the associated |/.append style| variant. See section~\ref{sec:styles} for more detail.
\end{enumerate}
I apologize for any inconvenience caused by these changes.

\begin{pgfplotskey}{compat=\mchoice{1.3,pre 1.3,default,newest} (initially default)}
	The preamble configuration 
\begin{codeexample}[code only]
\usepackage{pgfplots}
\pgfplotsset{compat=1.3}
\end{codeexample}
	allows to choose between backwards compatibility and most recent features.

	\PGFPlots\ version 1.3 comes with new features which may lead to different spacings compared to earlier versions:
	\begin{enumerate}
		\item Version 1.3 comes with the |xlabel near ticks| feature which places (in this case $x$) axis labels ``near'' the tick labels, taking the size of tick labels into account.
		
		Older versions used |xlabel absolute|, i.e.\ an absolute distance between the axis and axis labels (now matter how large or small tick labels are).

		The initial setting \emph{keeps backwards compatibility}.

		You are encouraged to use
\begin{codeexample}[code only]
\usepackage{pgfplots}
\pgfplotsset{compat=1.3}
\end{codeexample}
		in your preamble.
		
		\item Version 1.3 fixes a bug with |anchor=south west| which occurred mainly for reversed axes: the anchors where upside--down. This is fixed now.

		
		If you have written some work--around and want to keep it going, use

		|\pgfplotsset{compat/anchors=pre 1.3}|\pgfmanualpdflabel{/pgfplots/compat/anchors}{} 

		to disable the bug fix.
	\end{enumerate}

	Use |\pgfplotsset{compat=default}| to restore the factory settings.

	The setting |\pgfplotsset{compat=newest}| will always use features of the most recent version. This might result in small changes in the document's appearance.
\end{pgfplotskey}

\subsection{The Team}
\PGFPlots\ has been written mainly by Christian Feuers�nger with many improvements of Pascal Wolkotte and Nick Papior Andersen as a spare time project. We hope it is useful and provides valuable plots.

If you are interested in writing something but don't know how, consider reading the auxiliary manual \href{file:TeX-programming-notes.pdf}{TeX-programming-notes.pdf} which comes with \PGFPlots. It is far from complete, but maybe it is a good starting point (at least for more literature).

\subsection{Acknowledgements}
I thank God for all hours of enjoyed programming. I thank Pascal Wolkotte and Nick Papior Andersen for their programming efforts and contributions as part of the development team. I thank Stefan Tibus, who contributed the |plot shell| feature. I thank Tom Cashman for the contribution of the |reverse legend| feature. Special thanks go to Stefan Pinnow whose tests of \PGFPlots\ lead to numerous quality improvements. Furthermore, I thank Dr.~Schweitzer for many fruitful discussions and Dr.~Meine for his ideas and suggestions. Thanks as well to the many international contributors who provided feature requests or identified bugs or simply manual improvements!

Last but not least, I thank Till Tantau and Mark Wibrow for their excellent graphics (and more) package \PGF\ and \Tikz\ which is the base of \PGFPlots.

\include{pgfplots.install}%
\include{pgfplots.intro}%
\include{pgfplots.reference}%
\include{pgfplots.libs}%
\include{pgfplots.resources}%
\include{pgfplots.importexport}%
\include{pgfplots.basic.reference}%

\printindex

\bibliographystyle{abbrv} %gerapali} %gerabbrv} %gerunsrt.bst} %gerabbrv}% gerplain}
\nocite{pgfplotstable}
\nocite{programmingnotes}
\bibliography{pgfplots}
\end{document}
%
%%%%%%%%%%%%%%%%%%%%%%%%%%%%%%%%%%%%%%%%%%%%%%%%%%%%%%%%%%%%%%%%%%%%%%%%%%%%%
%
% Package pgfplots.sty documentation. 
%
% Copyright 2007/2008 by Christian Feuersaenger.
%
% This program is free software: you can redistribute it and/or modify
% it under the terms of the GNU General Public License as published by
% the Free Software Foundation, either version 3 of the License, or
% (at your option) any later version.
% 
% This program is distributed in the hope that it will be useful,
% but WITHOUT ANY WARRANTY; without even the implied warranty of
% MERCHANTABILITY or FITNESS FOR A PARTICULAR PURPOSE.  See the
% GNU General Public License for more details.
% 
% You should have received a copy of the GNU General Public License
% along with this program.  If not, see <http://www.gnu.org/licenses/>.
%
%
%%%%%%%%%%%%%%%%%%%%%%%%%%%%%%%%%%%%%%%%%%%%%%%%%%%%%%%%%%%%%%%%%%%%%%%%%%%%%
\input{pgfplots.preamble.tex}
%\RequirePackage[german,english,francais]{babel}

\pgfplotsmanualexternalexpensivetrue

%\usetikzlibrary{external}
%\tikzexternalize[prefix=figures/]{pgfplots}

\title{%
	Manual for Package \PGFPlots\\
	{\small Version \pgfplotsversion}\\
	{\small\href{http://sourceforge.net/projects/pgfplots}{http://sourceforge.net/projects/pgfplots}}}

%\includeonly{pgfplots.libs}


\begin{document}

\def\plotcoords{%
\addplot coordinates {
(5,8.312e-02)    (17,2.547e-02)   (49,7.407e-03)
(129,2.102e-03)  (321,5.874e-04)  (769,1.623e-04)
(1793,4.442e-05) (4097,1.207e-05) (9217,3.261e-06)
};

\addplot coordinates{
(7,8.472e-02)    (31,3.044e-02)    (111,1.022e-02)
(351,3.303e-03)  (1023,1.039e-03)  (2815,3.196e-04)
(7423,9.658e-05) (18943,2.873e-05) (47103,8.437e-06)
};

\addplot coordinates{
(9,7.881e-02)     (49,3.243e-02)    (209,1.232e-02)
(769,4.454e-03)   (2561,1.551e-03)  (7937,5.236e-04)
(23297,1.723e-04) (65537,5.545e-05) (178177,1.751e-05)
};

\addplot coordinates{
(11,6.887e-02)    (71,3.177e-02)     (351,1.341e-02)
(1471,5.334e-03)  (5503,2.027e-03)   (18943,7.415e-04)
(61183,2.628e-04) (187903,9.063e-05) (553983,3.053e-05)
};

\addplot coordinates{
(13,5.755e-02)     (97,2.925e-02)     (545,1.351e-02)
(2561,5.842e-03)   (10625,2.397e-03)  (40193,9.414e-04)
(141569,3.564e-04) (471041,1.308e-04) 
(1496065,4.670e-05)
};
}%
\maketitle
\begin{abstract}%
\PGFPlots\ draws high--quality function plots in normal or logarithmic scaling with a user-friendly interface directly in \TeX. The user supplies axis labels, legend entries and the plot coordinates for one or more plots and \PGFPlots\ applies axis scaling, computes any logarithms and axis ticks and draws the plots, supporting line plots, scatter plots, piecewise constant plots, bar plots, area plots, mesh-- and surface plots and some more. It is based on Till Tantau's package \PGF/\Tikz.
\end{abstract}
\tableofcontents
\section{Introduction}
This package provides tools to generate plots and labeled axes easily. It draws normal plots, logplots and semi-logplots, in two and three dimensions. Axis ticks, labels, legends (in case of multiple plots) can be added with key-value options. It can cycle through a set of predefined line/marker/color specifications. In summary, its purpose is to simplify the generation of high-quality function and/or data plots, and solving the problems of
\begin{itemize}
	\item consistency of document and font type and font size,
	\item direct use of \TeX\ math mode in axis descriptions,
	\item consistency of data and figures (no third party tool necessary),
	\item inter document consistency using preamble configurations and styles.
\end{itemize}
Although not necessary, separate |.pdf| or |.eps| graphics can be generated using the |external| library developed as part of \Tikz.

\PGFPlots\ is build completely on \Tikz/\PGF. Knowledge of \Tikz\ will simplify the work with \PGFPlots, although it is not required.

\section[About PGFPlots: Preliminaries]{About {\normalfont\PGFPlots}: Preliminaries}
This section contains information about upgrades, the team, the installation (in case you need to do it manually) and troubleshooting. You may skip it completely except for the upgrade remarks.

However, note that this library requires at least \PGF\ version $2.00$. At the time of this writing, many \TeX-distributions still contain the older \PGF\ version $1.18$, so it may be necessary to install a recent \PGF\ prior to using \PGFPlots.

\subsection{Components}
\PGFPlots\ comes with two components:
\begin{enumerate}
	\item the plotting component (which you are currently reading) and
	\item the \PGFPlotstable\ component which simplifies number formatting and postprocessing of numerical tables. It comes as a separate package and has its own manual \href{file:pgfplotstable.pdf}{pgfplotstable.pdf}.
\end{enumerate}

\subsection{Upgrade remarks}
This release provides a lot of improvements which can be found in all detail in \texttt{ChangeLog} for interested readers. However, some attention is useful with respect to the following changes.

\subsubsection{New Optional Features}
\begin{enumerate}
	\item \PGFPlots\ 1.3 comes with user interface improvements. The technical distinction between ``behavior options'' and ``style options'' of older versions is no longer necessary (although still fully supported).

	\item \PGFPlots\ 1.3 has a new feature which allows to \emph{move axis labels tight to tick labels} automatically.

	Since this affects the spacing, it is not enabled be default.

	Use
\begin{codeexample}[code only]
\usepackage{pgfplots}
\pgfplotsset{compat=1.3}
\end{codeexample}
	in your preamble to benefit from the improved spacing\footnote{The |compat=1.3| setting is necessary for new features which move axis labels around like reversed axis. However, \PGFPlots\ will still work as it does in earlier versions, your documents will remain the same if you don't set it explicitly.}.
	Take a look at the next page for the precise definition of |compat|.

	\item \PGFPlots\ 1.3 now supports reversed axes. It is no longer necessary to use work--arounds with negative units.
\pgfkeys{/pdflinks/search key prefixes in/.add={/pgfplots/,}{}}

	Take a look at the |x dir=reverse| key.

	Existing work--arounds will still function properly. Use |\pgfplotsset{compat=1.3}| together with |x dir=reverse| to switch to the new version.
\end{enumerate}

\subsubsection{Old Features Which May Need Attention}
\begin{enumerate}
	\item The |scatter/classes| feature produces proper legends as of version 1.3. This may change the appearance of existing legends of plots with |scatter/classes|.

	\item Due to a small math bug in \PGF\ $2.00$, you \emph{can't} use the math expression `|-x^2|'. It is necessary to use `|0-x^2|' instead. The same holds for
		`|exp(-x^2)|'; use `|exp(0-x^2)|' instead.

		This will be fixed with \PGF\ $>2.00$.

	\item Starting with \PGFPlots\ $1.1$, |\tikzstyle| should \emph{no longer be used} to set \PGFPlots\ options.
	
	Although |\tikzstyle| is still supported for some older \PGFPlots\ options, you should replace any occurance of |\tikzstyle| with |\pgfplotsset{|\meta{style name}|/.style={|\meta{key-value-list}|}}| or the associated |/.append style| variant. See section~\ref{sec:styles} for more detail.
\end{enumerate}
I apologize for any inconvenience caused by these changes.

\begin{pgfplotskey}{compat=\mchoice{1.3,pre 1.3,default,newest} (initially default)}
	The preamble configuration 
\begin{codeexample}[code only]
\usepackage{pgfplots}
\pgfplotsset{compat=1.3}
\end{codeexample}
	allows to choose between backwards compatibility and most recent features.

	\PGFPlots\ version 1.3 comes with new features which may lead to different spacings compared to earlier versions:
	\begin{enumerate}
		\item Version 1.3 comes with the |xlabel near ticks| feature which places (in this case $x$) axis labels ``near'' the tick labels, taking the size of tick labels into account.
		
		Older versions used |xlabel absolute|, i.e.\ an absolute distance between the axis and axis labels (now matter how large or small tick labels are).

		The initial setting \emph{keeps backwards compatibility}.

		You are encouraged to use
\begin{codeexample}[code only]
\usepackage{pgfplots}
\pgfplotsset{compat=1.3}
\end{codeexample}
		in your preamble.
		
		\item Version 1.3 fixes a bug with |anchor=south west| which occurred mainly for reversed axes: the anchors where upside--down. This is fixed now.

		
		If you have written some work--around and want to keep it going, use

		|\pgfplotsset{compat/anchors=pre 1.3}|\pgfmanualpdflabel{/pgfplots/compat/anchors}{} 

		to disable the bug fix.
	\end{enumerate}

	Use |\pgfplotsset{compat=default}| to restore the factory settings.

	The setting |\pgfplotsset{compat=newest}| will always use features of the most recent version. This might result in small changes in the document's appearance.
\end{pgfplotskey}

\subsection{The Team}
\PGFPlots\ has been written mainly by Christian Feuers�nger with many improvements of Pascal Wolkotte and Nick Papior Andersen as a spare time project. We hope it is useful and provides valuable plots.

If you are interested in writing something but don't know how, consider reading the auxiliary manual \href{file:TeX-programming-notes.pdf}{TeX-programming-notes.pdf} which comes with \PGFPlots. It is far from complete, but maybe it is a good starting point (at least for more literature).

\subsection{Acknowledgements}
I thank God for all hours of enjoyed programming. I thank Pascal Wolkotte and Nick Papior Andersen for their programming efforts and contributions as part of the development team. I thank Stefan Tibus, who contributed the |plot shell| feature. I thank Tom Cashman for the contribution of the |reverse legend| feature. Special thanks go to Stefan Pinnow whose tests of \PGFPlots\ lead to numerous quality improvements. Furthermore, I thank Dr.~Schweitzer for many fruitful discussions and Dr.~Meine for his ideas and suggestions. Thanks as well to the many international contributors who provided feature requests or identified bugs or simply manual improvements!

Last but not least, I thank Till Tantau and Mark Wibrow for their excellent graphics (and more) package \PGF\ and \Tikz\ which is the base of \PGFPlots.

\include{pgfplots.install}%
\include{pgfplots.intro}%
\include{pgfplots.reference}%
\include{pgfplots.libs}%
\include{pgfplots.resources}%
\include{pgfplots.importexport}%
\include{pgfplots.basic.reference}%

\printindex

\bibliographystyle{abbrv} %gerapali} %gerabbrv} %gerunsrt.bst} %gerabbrv}% gerplain}
\nocite{pgfplotstable}
\nocite{programmingnotes}
\bibliography{pgfplots}
\end{document}
%
%%%%%%%%%%%%%%%%%%%%%%%%%%%%%%%%%%%%%%%%%%%%%%%%%%%%%%%%%%%%%%%%%%%%%%%%%%%%%
%
% Package pgfplots.sty documentation. 
%
% Copyright 2007/2008 by Christian Feuersaenger.
%
% This program is free software: you can redistribute it and/or modify
% it under the terms of the GNU General Public License as published by
% the Free Software Foundation, either version 3 of the License, or
% (at your option) any later version.
% 
% This program is distributed in the hope that it will be useful,
% but WITHOUT ANY WARRANTY; without even the implied warranty of
% MERCHANTABILITY or FITNESS FOR A PARTICULAR PURPOSE.  See the
% GNU General Public License for more details.
% 
% You should have received a copy of the GNU General Public License
% along with this program.  If not, see <http://www.gnu.org/licenses/>.
%
%
%%%%%%%%%%%%%%%%%%%%%%%%%%%%%%%%%%%%%%%%%%%%%%%%%%%%%%%%%%%%%%%%%%%%%%%%%%%%%
\input{pgfplots.preamble.tex}
%\RequirePackage[german,english,francais]{babel}

\pgfplotsmanualexternalexpensivetrue

%\usetikzlibrary{external}
%\tikzexternalize[prefix=figures/]{pgfplots}

\title{%
	Manual for Package \PGFPlots\\
	{\small Version \pgfplotsversion}\\
	{\small\href{http://sourceforge.net/projects/pgfplots}{http://sourceforge.net/projects/pgfplots}}}

%\includeonly{pgfplots.libs}


\begin{document}

\def\plotcoords{%
\addplot coordinates {
(5,8.312e-02)    (17,2.547e-02)   (49,7.407e-03)
(129,2.102e-03)  (321,5.874e-04)  (769,1.623e-04)
(1793,4.442e-05) (4097,1.207e-05) (9217,3.261e-06)
};

\addplot coordinates{
(7,8.472e-02)    (31,3.044e-02)    (111,1.022e-02)
(351,3.303e-03)  (1023,1.039e-03)  (2815,3.196e-04)
(7423,9.658e-05) (18943,2.873e-05) (47103,8.437e-06)
};

\addplot coordinates{
(9,7.881e-02)     (49,3.243e-02)    (209,1.232e-02)
(769,4.454e-03)   (2561,1.551e-03)  (7937,5.236e-04)
(23297,1.723e-04) (65537,5.545e-05) (178177,1.751e-05)
};

\addplot coordinates{
(11,6.887e-02)    (71,3.177e-02)     (351,1.341e-02)
(1471,5.334e-03)  (5503,2.027e-03)   (18943,7.415e-04)
(61183,2.628e-04) (187903,9.063e-05) (553983,3.053e-05)
};

\addplot coordinates{
(13,5.755e-02)     (97,2.925e-02)     (545,1.351e-02)
(2561,5.842e-03)   (10625,2.397e-03)  (40193,9.414e-04)
(141569,3.564e-04) (471041,1.308e-04) 
(1496065,4.670e-05)
};
}%
\maketitle
\begin{abstract}%
\PGFPlots\ draws high--quality function plots in normal or logarithmic scaling with a user-friendly interface directly in \TeX. The user supplies axis labels, legend entries and the plot coordinates for one or more plots and \PGFPlots\ applies axis scaling, computes any logarithms and axis ticks and draws the plots, supporting line plots, scatter plots, piecewise constant plots, bar plots, area plots, mesh-- and surface plots and some more. It is based on Till Tantau's package \PGF/\Tikz.
\end{abstract}
\tableofcontents
\section{Introduction}
This package provides tools to generate plots and labeled axes easily. It draws normal plots, logplots and semi-logplots, in two and three dimensions. Axis ticks, labels, legends (in case of multiple plots) can be added with key-value options. It can cycle through a set of predefined line/marker/color specifications. In summary, its purpose is to simplify the generation of high-quality function and/or data plots, and solving the problems of
\begin{itemize}
	\item consistency of document and font type and font size,
	\item direct use of \TeX\ math mode in axis descriptions,
	\item consistency of data and figures (no third party tool necessary),
	\item inter document consistency using preamble configurations and styles.
\end{itemize}
Although not necessary, separate |.pdf| or |.eps| graphics can be generated using the |external| library developed as part of \Tikz.

\PGFPlots\ is build completely on \Tikz/\PGF. Knowledge of \Tikz\ will simplify the work with \PGFPlots, although it is not required.

\section[About PGFPlots: Preliminaries]{About {\normalfont\PGFPlots}: Preliminaries}
This section contains information about upgrades, the team, the installation (in case you need to do it manually) and troubleshooting. You may skip it completely except for the upgrade remarks.

However, note that this library requires at least \PGF\ version $2.00$. At the time of this writing, many \TeX-distributions still contain the older \PGF\ version $1.18$, so it may be necessary to install a recent \PGF\ prior to using \PGFPlots.

\subsection{Components}
\PGFPlots\ comes with two components:
\begin{enumerate}
	\item the plotting component (which you are currently reading) and
	\item the \PGFPlotstable\ component which simplifies number formatting and postprocessing of numerical tables. It comes as a separate package and has its own manual \href{file:pgfplotstable.pdf}{pgfplotstable.pdf}.
\end{enumerate}

\subsection{Upgrade remarks}
This release provides a lot of improvements which can be found in all detail in \texttt{ChangeLog} for interested readers. However, some attention is useful with respect to the following changes.

\subsubsection{New Optional Features}
\begin{enumerate}
	\item \PGFPlots\ 1.3 comes with user interface improvements. The technical distinction between ``behavior options'' and ``style options'' of older versions is no longer necessary (although still fully supported).

	\item \PGFPlots\ 1.3 has a new feature which allows to \emph{move axis labels tight to tick labels} automatically.

	Since this affects the spacing, it is not enabled be default.

	Use
\begin{codeexample}[code only]
\usepackage{pgfplots}
\pgfplotsset{compat=1.3}
\end{codeexample}
	in your preamble to benefit from the improved spacing\footnote{The |compat=1.3| setting is necessary for new features which move axis labels around like reversed axis. However, \PGFPlots\ will still work as it does in earlier versions, your documents will remain the same if you don't set it explicitly.}.
	Take a look at the next page for the precise definition of |compat|.

	\item \PGFPlots\ 1.3 now supports reversed axes. It is no longer necessary to use work--arounds with negative units.
\pgfkeys{/pdflinks/search key prefixes in/.add={/pgfplots/,}{}}

	Take a look at the |x dir=reverse| key.

	Existing work--arounds will still function properly. Use |\pgfplotsset{compat=1.3}| together with |x dir=reverse| to switch to the new version.
\end{enumerate}

\subsubsection{Old Features Which May Need Attention}
\begin{enumerate}
	\item The |scatter/classes| feature produces proper legends as of version 1.3. This may change the appearance of existing legends of plots with |scatter/classes|.

	\item Due to a small math bug in \PGF\ $2.00$, you \emph{can't} use the math expression `|-x^2|'. It is necessary to use `|0-x^2|' instead. The same holds for
		`|exp(-x^2)|'; use `|exp(0-x^2)|' instead.

		This will be fixed with \PGF\ $>2.00$.

	\item Starting with \PGFPlots\ $1.1$, |\tikzstyle| should \emph{no longer be used} to set \PGFPlots\ options.
	
	Although |\tikzstyle| is still supported for some older \PGFPlots\ options, you should replace any occurance of |\tikzstyle| with |\pgfplotsset{|\meta{style name}|/.style={|\meta{key-value-list}|}}| or the associated |/.append style| variant. See section~\ref{sec:styles} for more detail.
\end{enumerate}
I apologize for any inconvenience caused by these changes.

\begin{pgfplotskey}{compat=\mchoice{1.3,pre 1.3,default,newest} (initially default)}
	The preamble configuration 
\begin{codeexample}[code only]
\usepackage{pgfplots}
\pgfplotsset{compat=1.3}
\end{codeexample}
	allows to choose between backwards compatibility and most recent features.

	\PGFPlots\ version 1.3 comes with new features which may lead to different spacings compared to earlier versions:
	\begin{enumerate}
		\item Version 1.3 comes with the |xlabel near ticks| feature which places (in this case $x$) axis labels ``near'' the tick labels, taking the size of tick labels into account.
		
		Older versions used |xlabel absolute|, i.e.\ an absolute distance between the axis and axis labels (now matter how large or small tick labels are).

		The initial setting \emph{keeps backwards compatibility}.

		You are encouraged to use
\begin{codeexample}[code only]
\usepackage{pgfplots}
\pgfplotsset{compat=1.3}
\end{codeexample}
		in your preamble.
		
		\item Version 1.3 fixes a bug with |anchor=south west| which occurred mainly for reversed axes: the anchors where upside--down. This is fixed now.

		
		If you have written some work--around and want to keep it going, use

		|\pgfplotsset{compat/anchors=pre 1.3}|\pgfmanualpdflabel{/pgfplots/compat/anchors}{} 

		to disable the bug fix.
	\end{enumerate}

	Use |\pgfplotsset{compat=default}| to restore the factory settings.

	The setting |\pgfplotsset{compat=newest}| will always use features of the most recent version. This might result in small changes in the document's appearance.
\end{pgfplotskey}

\subsection{The Team}
\PGFPlots\ has been written mainly by Christian Feuers�nger with many improvements of Pascal Wolkotte and Nick Papior Andersen as a spare time project. We hope it is useful and provides valuable plots.

If you are interested in writing something but don't know how, consider reading the auxiliary manual \href{file:TeX-programming-notes.pdf}{TeX-programming-notes.pdf} which comes with \PGFPlots. It is far from complete, but maybe it is a good starting point (at least for more literature).

\subsection{Acknowledgements}
I thank God for all hours of enjoyed programming. I thank Pascal Wolkotte and Nick Papior Andersen for their programming efforts and contributions as part of the development team. I thank Stefan Tibus, who contributed the |plot shell| feature. I thank Tom Cashman for the contribution of the |reverse legend| feature. Special thanks go to Stefan Pinnow whose tests of \PGFPlots\ lead to numerous quality improvements. Furthermore, I thank Dr.~Schweitzer for many fruitful discussions and Dr.~Meine for his ideas and suggestions. Thanks as well to the many international contributors who provided feature requests or identified bugs or simply manual improvements!

Last but not least, I thank Till Tantau and Mark Wibrow for their excellent graphics (and more) package \PGF\ and \Tikz\ which is the base of \PGFPlots.

\include{pgfplots.install}%
\include{pgfplots.intro}%
\include{pgfplots.reference}%
\include{pgfplots.libs}%
\include{pgfplots.resources}%
\include{pgfplots.importexport}%
\include{pgfplots.basic.reference}%

\printindex

\bibliographystyle{abbrv} %gerapali} %gerabbrv} %gerunsrt.bst} %gerabbrv}% gerplain}
\nocite{pgfplotstable}
\nocite{programmingnotes}
\bibliography{pgfplots}
\end{document}
%
%%%%%%%%%%%%%%%%%%%%%%%%%%%%%%%%%%%%%%%%%%%%%%%%%%%%%%%%%%%%%%%%%%%%%%%%%%%%%
%
% Package pgfplots.sty documentation. 
%
% Copyright 2007/2008 by Christian Feuersaenger.
%
% This program is free software: you can redistribute it and/or modify
% it under the terms of the GNU General Public License as published by
% the Free Software Foundation, either version 3 of the License, or
% (at your option) any later version.
% 
% This program is distributed in the hope that it will be useful,
% but WITHOUT ANY WARRANTY; without even the implied warranty of
% MERCHANTABILITY or FITNESS FOR A PARTICULAR PURPOSE.  See the
% GNU General Public License for more details.
% 
% You should have received a copy of the GNU General Public License
% along with this program.  If not, see <http://www.gnu.org/licenses/>.
%
%
%%%%%%%%%%%%%%%%%%%%%%%%%%%%%%%%%%%%%%%%%%%%%%%%%%%%%%%%%%%%%%%%%%%%%%%%%%%%%
\input{pgfplots.preamble.tex}
%\RequirePackage[german,english,francais]{babel}

\pgfplotsmanualexternalexpensivetrue

%\usetikzlibrary{external}
%\tikzexternalize[prefix=figures/]{pgfplots}

\title{%
	Manual for Package \PGFPlots\\
	{\small Version \pgfplotsversion}\\
	{\small\href{http://sourceforge.net/projects/pgfplots}{http://sourceforge.net/projects/pgfplots}}}

%\includeonly{pgfplots.libs}


\begin{document}

\def\plotcoords{%
\addplot coordinates {
(5,8.312e-02)    (17,2.547e-02)   (49,7.407e-03)
(129,2.102e-03)  (321,5.874e-04)  (769,1.623e-04)
(1793,4.442e-05) (4097,1.207e-05) (9217,3.261e-06)
};

\addplot coordinates{
(7,8.472e-02)    (31,3.044e-02)    (111,1.022e-02)
(351,3.303e-03)  (1023,1.039e-03)  (2815,3.196e-04)
(7423,9.658e-05) (18943,2.873e-05) (47103,8.437e-06)
};

\addplot coordinates{
(9,7.881e-02)     (49,3.243e-02)    (209,1.232e-02)
(769,4.454e-03)   (2561,1.551e-03)  (7937,5.236e-04)
(23297,1.723e-04) (65537,5.545e-05) (178177,1.751e-05)
};

\addplot coordinates{
(11,6.887e-02)    (71,3.177e-02)     (351,1.341e-02)
(1471,5.334e-03)  (5503,2.027e-03)   (18943,7.415e-04)
(61183,2.628e-04) (187903,9.063e-05) (553983,3.053e-05)
};

\addplot coordinates{
(13,5.755e-02)     (97,2.925e-02)     (545,1.351e-02)
(2561,5.842e-03)   (10625,2.397e-03)  (40193,9.414e-04)
(141569,3.564e-04) (471041,1.308e-04) 
(1496065,4.670e-05)
};
}%
\maketitle
\begin{abstract}%
\PGFPlots\ draws high--quality function plots in normal or logarithmic scaling with a user-friendly interface directly in \TeX. The user supplies axis labels, legend entries and the plot coordinates for one or more plots and \PGFPlots\ applies axis scaling, computes any logarithms and axis ticks and draws the plots, supporting line plots, scatter plots, piecewise constant plots, bar plots, area plots, mesh-- and surface plots and some more. It is based on Till Tantau's package \PGF/\Tikz.
\end{abstract}
\tableofcontents
\section{Introduction}
This package provides tools to generate plots and labeled axes easily. It draws normal plots, logplots and semi-logplots, in two and three dimensions. Axis ticks, labels, legends (in case of multiple plots) can be added with key-value options. It can cycle through a set of predefined line/marker/color specifications. In summary, its purpose is to simplify the generation of high-quality function and/or data plots, and solving the problems of
\begin{itemize}
	\item consistency of document and font type and font size,
	\item direct use of \TeX\ math mode in axis descriptions,
	\item consistency of data and figures (no third party tool necessary),
	\item inter document consistency using preamble configurations and styles.
\end{itemize}
Although not necessary, separate |.pdf| or |.eps| graphics can be generated using the |external| library developed as part of \Tikz.

\PGFPlots\ is build completely on \Tikz/\PGF. Knowledge of \Tikz\ will simplify the work with \PGFPlots, although it is not required.

\section[About PGFPlots: Preliminaries]{About {\normalfont\PGFPlots}: Preliminaries}
This section contains information about upgrades, the team, the installation (in case you need to do it manually) and troubleshooting. You may skip it completely except for the upgrade remarks.

However, note that this library requires at least \PGF\ version $2.00$. At the time of this writing, many \TeX-distributions still contain the older \PGF\ version $1.18$, so it may be necessary to install a recent \PGF\ prior to using \PGFPlots.

\subsection{Components}
\PGFPlots\ comes with two components:
\begin{enumerate}
	\item the plotting component (which you are currently reading) and
	\item the \PGFPlotstable\ component which simplifies number formatting and postprocessing of numerical tables. It comes as a separate package and has its own manual \href{file:pgfplotstable.pdf}{pgfplotstable.pdf}.
\end{enumerate}

\subsection{Upgrade remarks}
This release provides a lot of improvements which can be found in all detail in \texttt{ChangeLog} for interested readers. However, some attention is useful with respect to the following changes.

\subsubsection{New Optional Features}
\begin{enumerate}
	\item \PGFPlots\ 1.3 comes with user interface improvements. The technical distinction between ``behavior options'' and ``style options'' of older versions is no longer necessary (although still fully supported).

	\item \PGFPlots\ 1.3 has a new feature which allows to \emph{move axis labels tight to tick labels} automatically.

	Since this affects the spacing, it is not enabled be default.

	Use
\begin{codeexample}[code only]
\usepackage{pgfplots}
\pgfplotsset{compat=1.3}
\end{codeexample}
	in your preamble to benefit from the improved spacing\footnote{The |compat=1.3| setting is necessary for new features which move axis labels around like reversed axis. However, \PGFPlots\ will still work as it does in earlier versions, your documents will remain the same if you don't set it explicitly.}.
	Take a look at the next page for the precise definition of |compat|.

	\item \PGFPlots\ 1.3 now supports reversed axes. It is no longer necessary to use work--arounds with negative units.
\pgfkeys{/pdflinks/search key prefixes in/.add={/pgfplots/,}{}}

	Take a look at the |x dir=reverse| key.

	Existing work--arounds will still function properly. Use |\pgfplotsset{compat=1.3}| together with |x dir=reverse| to switch to the new version.
\end{enumerate}

\subsubsection{Old Features Which May Need Attention}
\begin{enumerate}
	\item The |scatter/classes| feature produces proper legends as of version 1.3. This may change the appearance of existing legends of plots with |scatter/classes|.

	\item Due to a small math bug in \PGF\ $2.00$, you \emph{can't} use the math expression `|-x^2|'. It is necessary to use `|0-x^2|' instead. The same holds for
		`|exp(-x^2)|'; use `|exp(0-x^2)|' instead.

		This will be fixed with \PGF\ $>2.00$.

	\item Starting with \PGFPlots\ $1.1$, |\tikzstyle| should \emph{no longer be used} to set \PGFPlots\ options.
	
	Although |\tikzstyle| is still supported for some older \PGFPlots\ options, you should replace any occurance of |\tikzstyle| with |\pgfplotsset{|\meta{style name}|/.style={|\meta{key-value-list}|}}| or the associated |/.append style| variant. See section~\ref{sec:styles} for more detail.
\end{enumerate}
I apologize for any inconvenience caused by these changes.

\begin{pgfplotskey}{compat=\mchoice{1.3,pre 1.3,default,newest} (initially default)}
	The preamble configuration 
\begin{codeexample}[code only]
\usepackage{pgfplots}
\pgfplotsset{compat=1.3}
\end{codeexample}
	allows to choose between backwards compatibility and most recent features.

	\PGFPlots\ version 1.3 comes with new features which may lead to different spacings compared to earlier versions:
	\begin{enumerate}
		\item Version 1.3 comes with the |xlabel near ticks| feature which places (in this case $x$) axis labels ``near'' the tick labels, taking the size of tick labels into account.
		
		Older versions used |xlabel absolute|, i.e.\ an absolute distance between the axis and axis labels (now matter how large or small tick labels are).

		The initial setting \emph{keeps backwards compatibility}.

		You are encouraged to use
\begin{codeexample}[code only]
\usepackage{pgfplots}
\pgfplotsset{compat=1.3}
\end{codeexample}
		in your preamble.
		
		\item Version 1.3 fixes a bug with |anchor=south west| which occurred mainly for reversed axes: the anchors where upside--down. This is fixed now.

		
		If you have written some work--around and want to keep it going, use

		|\pgfplotsset{compat/anchors=pre 1.3}|\pgfmanualpdflabel{/pgfplots/compat/anchors}{} 

		to disable the bug fix.
	\end{enumerate}

	Use |\pgfplotsset{compat=default}| to restore the factory settings.

	The setting |\pgfplotsset{compat=newest}| will always use features of the most recent version. This might result in small changes in the document's appearance.
\end{pgfplotskey}

\subsection{The Team}
\PGFPlots\ has been written mainly by Christian Feuers�nger with many improvements of Pascal Wolkotte and Nick Papior Andersen as a spare time project. We hope it is useful and provides valuable plots.

If you are interested in writing something but don't know how, consider reading the auxiliary manual \href{file:TeX-programming-notes.pdf}{TeX-programming-notes.pdf} which comes with \PGFPlots. It is far from complete, but maybe it is a good starting point (at least for more literature).

\subsection{Acknowledgements}
I thank God for all hours of enjoyed programming. I thank Pascal Wolkotte and Nick Papior Andersen for their programming efforts and contributions as part of the development team. I thank Stefan Tibus, who contributed the |plot shell| feature. I thank Tom Cashman for the contribution of the |reverse legend| feature. Special thanks go to Stefan Pinnow whose tests of \PGFPlots\ lead to numerous quality improvements. Furthermore, I thank Dr.~Schweitzer for many fruitful discussions and Dr.~Meine for his ideas and suggestions. Thanks as well to the many international contributors who provided feature requests or identified bugs or simply manual improvements!

Last but not least, I thank Till Tantau and Mark Wibrow for their excellent graphics (and more) package \PGF\ and \Tikz\ which is the base of \PGFPlots.

\include{pgfplots.install}%
\include{pgfplots.intro}%
\include{pgfplots.reference}%
\include{pgfplots.libs}%
\include{pgfplots.resources}%
\include{pgfplots.importexport}%
\include{pgfplots.basic.reference}%

\printindex

\bibliographystyle{abbrv} %gerapali} %gerabbrv} %gerunsrt.bst} %gerabbrv}% gerplain}
\nocite{pgfplotstable}
\nocite{programmingnotes}
\bibliography{pgfplots}
\end{document}
%
%%%%%%%%%%%%%%%%%%%%%%%%%%%%%%%%%%%%%%%%%%%%%%%%%%%%%%%%%%%%%%%%%%%%%%%%%%%%%
%
% Package pgfplots.sty documentation. 
%
% Copyright 2007/2008 by Christian Feuersaenger.
%
% This program is free software: you can redistribute it and/or modify
% it under the terms of the GNU General Public License as published by
% the Free Software Foundation, either version 3 of the License, or
% (at your option) any later version.
% 
% This program is distributed in the hope that it will be useful,
% but WITHOUT ANY WARRANTY; without even the implied warranty of
% MERCHANTABILITY or FITNESS FOR A PARTICULAR PURPOSE.  See the
% GNU General Public License for more details.
% 
% You should have received a copy of the GNU General Public License
% along with this program.  If not, see <http://www.gnu.org/licenses/>.
%
%
%%%%%%%%%%%%%%%%%%%%%%%%%%%%%%%%%%%%%%%%%%%%%%%%%%%%%%%%%%%%%%%%%%%%%%%%%%%%%
\input{pgfplots.preamble.tex}
%\RequirePackage[german,english,francais]{babel}

\pgfplotsmanualexternalexpensivetrue

%\usetikzlibrary{external}
%\tikzexternalize[prefix=figures/]{pgfplots}

\title{%
	Manual for Package \PGFPlots\\
	{\small Version \pgfplotsversion}\\
	{\small\href{http://sourceforge.net/projects/pgfplots}{http://sourceforge.net/projects/pgfplots}}}

%\includeonly{pgfplots.libs}


\begin{document}

\def\plotcoords{%
\addplot coordinates {
(5,8.312e-02)    (17,2.547e-02)   (49,7.407e-03)
(129,2.102e-03)  (321,5.874e-04)  (769,1.623e-04)
(1793,4.442e-05) (4097,1.207e-05) (9217,3.261e-06)
};

\addplot coordinates{
(7,8.472e-02)    (31,3.044e-02)    (111,1.022e-02)
(351,3.303e-03)  (1023,1.039e-03)  (2815,3.196e-04)
(7423,9.658e-05) (18943,2.873e-05) (47103,8.437e-06)
};

\addplot coordinates{
(9,7.881e-02)     (49,3.243e-02)    (209,1.232e-02)
(769,4.454e-03)   (2561,1.551e-03)  (7937,5.236e-04)
(23297,1.723e-04) (65537,5.545e-05) (178177,1.751e-05)
};

\addplot coordinates{
(11,6.887e-02)    (71,3.177e-02)     (351,1.341e-02)
(1471,5.334e-03)  (5503,2.027e-03)   (18943,7.415e-04)
(61183,2.628e-04) (187903,9.063e-05) (553983,3.053e-05)
};

\addplot coordinates{
(13,5.755e-02)     (97,2.925e-02)     (545,1.351e-02)
(2561,5.842e-03)   (10625,2.397e-03)  (40193,9.414e-04)
(141569,3.564e-04) (471041,1.308e-04) 
(1496065,4.670e-05)
};
}%
\maketitle
\begin{abstract}%
\PGFPlots\ draws high--quality function plots in normal or logarithmic scaling with a user-friendly interface directly in \TeX. The user supplies axis labels, legend entries and the plot coordinates for one or more plots and \PGFPlots\ applies axis scaling, computes any logarithms and axis ticks and draws the plots, supporting line plots, scatter plots, piecewise constant plots, bar plots, area plots, mesh-- and surface plots and some more. It is based on Till Tantau's package \PGF/\Tikz.
\end{abstract}
\tableofcontents
\section{Introduction}
This package provides tools to generate plots and labeled axes easily. It draws normal plots, logplots and semi-logplots, in two and three dimensions. Axis ticks, labels, legends (in case of multiple plots) can be added with key-value options. It can cycle through a set of predefined line/marker/color specifications. In summary, its purpose is to simplify the generation of high-quality function and/or data plots, and solving the problems of
\begin{itemize}
	\item consistency of document and font type and font size,
	\item direct use of \TeX\ math mode in axis descriptions,
	\item consistency of data and figures (no third party tool necessary),
	\item inter document consistency using preamble configurations and styles.
\end{itemize}
Although not necessary, separate |.pdf| or |.eps| graphics can be generated using the |external| library developed as part of \Tikz.

\PGFPlots\ is build completely on \Tikz/\PGF. Knowledge of \Tikz\ will simplify the work with \PGFPlots, although it is not required.

\section[About PGFPlots: Preliminaries]{About {\normalfont\PGFPlots}: Preliminaries}
This section contains information about upgrades, the team, the installation (in case you need to do it manually) and troubleshooting. You may skip it completely except for the upgrade remarks.

However, note that this library requires at least \PGF\ version $2.00$. At the time of this writing, many \TeX-distributions still contain the older \PGF\ version $1.18$, so it may be necessary to install a recent \PGF\ prior to using \PGFPlots.

\subsection{Components}
\PGFPlots\ comes with two components:
\begin{enumerate}
	\item the plotting component (which you are currently reading) and
	\item the \PGFPlotstable\ component which simplifies number formatting and postprocessing of numerical tables. It comes as a separate package and has its own manual \href{file:pgfplotstable.pdf}{pgfplotstable.pdf}.
\end{enumerate}

\subsection{Upgrade remarks}
This release provides a lot of improvements which can be found in all detail in \texttt{ChangeLog} for interested readers. However, some attention is useful with respect to the following changes.

\subsubsection{New Optional Features}
\begin{enumerate}
	\item \PGFPlots\ 1.3 comes with user interface improvements. The technical distinction between ``behavior options'' and ``style options'' of older versions is no longer necessary (although still fully supported).

	\item \PGFPlots\ 1.3 has a new feature which allows to \emph{move axis labels tight to tick labels} automatically.

	Since this affects the spacing, it is not enabled be default.

	Use
\begin{codeexample}[code only]
\usepackage{pgfplots}
\pgfplotsset{compat=1.3}
\end{codeexample}
	in your preamble to benefit from the improved spacing\footnote{The |compat=1.3| setting is necessary for new features which move axis labels around like reversed axis. However, \PGFPlots\ will still work as it does in earlier versions, your documents will remain the same if you don't set it explicitly.}.
	Take a look at the next page for the precise definition of |compat|.

	\item \PGFPlots\ 1.3 now supports reversed axes. It is no longer necessary to use work--arounds with negative units.
\pgfkeys{/pdflinks/search key prefixes in/.add={/pgfplots/,}{}}

	Take a look at the |x dir=reverse| key.

	Existing work--arounds will still function properly. Use |\pgfplotsset{compat=1.3}| together with |x dir=reverse| to switch to the new version.
\end{enumerate}

\subsubsection{Old Features Which May Need Attention}
\begin{enumerate}
	\item The |scatter/classes| feature produces proper legends as of version 1.3. This may change the appearance of existing legends of plots with |scatter/classes|.

	\item Due to a small math bug in \PGF\ $2.00$, you \emph{can't} use the math expression `|-x^2|'. It is necessary to use `|0-x^2|' instead. The same holds for
		`|exp(-x^2)|'; use `|exp(0-x^2)|' instead.

		This will be fixed with \PGF\ $>2.00$.

	\item Starting with \PGFPlots\ $1.1$, |\tikzstyle| should \emph{no longer be used} to set \PGFPlots\ options.
	
	Although |\tikzstyle| is still supported for some older \PGFPlots\ options, you should replace any occurance of |\tikzstyle| with |\pgfplotsset{|\meta{style name}|/.style={|\meta{key-value-list}|}}| or the associated |/.append style| variant. See section~\ref{sec:styles} for more detail.
\end{enumerate}
I apologize for any inconvenience caused by these changes.

\begin{pgfplotskey}{compat=\mchoice{1.3,pre 1.3,default,newest} (initially default)}
	The preamble configuration 
\begin{codeexample}[code only]
\usepackage{pgfplots}
\pgfplotsset{compat=1.3}
\end{codeexample}
	allows to choose between backwards compatibility and most recent features.

	\PGFPlots\ version 1.3 comes with new features which may lead to different spacings compared to earlier versions:
	\begin{enumerate}
		\item Version 1.3 comes with the |xlabel near ticks| feature which places (in this case $x$) axis labels ``near'' the tick labels, taking the size of tick labels into account.
		
		Older versions used |xlabel absolute|, i.e.\ an absolute distance between the axis and axis labels (now matter how large or small tick labels are).

		The initial setting \emph{keeps backwards compatibility}.

		You are encouraged to use
\begin{codeexample}[code only]
\usepackage{pgfplots}
\pgfplotsset{compat=1.3}
\end{codeexample}
		in your preamble.
		
		\item Version 1.3 fixes a bug with |anchor=south west| which occurred mainly for reversed axes: the anchors where upside--down. This is fixed now.

		
		If you have written some work--around and want to keep it going, use

		|\pgfplotsset{compat/anchors=pre 1.3}|\pgfmanualpdflabel{/pgfplots/compat/anchors}{} 

		to disable the bug fix.
	\end{enumerate}

	Use |\pgfplotsset{compat=default}| to restore the factory settings.

	The setting |\pgfplotsset{compat=newest}| will always use features of the most recent version. This might result in small changes in the document's appearance.
\end{pgfplotskey}

\subsection{The Team}
\PGFPlots\ has been written mainly by Christian Feuers�nger with many improvements of Pascal Wolkotte and Nick Papior Andersen as a spare time project. We hope it is useful and provides valuable plots.

If you are interested in writing something but don't know how, consider reading the auxiliary manual \href{file:TeX-programming-notes.pdf}{TeX-programming-notes.pdf} which comes with \PGFPlots. It is far from complete, but maybe it is a good starting point (at least for more literature).

\subsection{Acknowledgements}
I thank God for all hours of enjoyed programming. I thank Pascal Wolkotte and Nick Papior Andersen for their programming efforts and contributions as part of the development team. I thank Stefan Tibus, who contributed the |plot shell| feature. I thank Tom Cashman for the contribution of the |reverse legend| feature. Special thanks go to Stefan Pinnow whose tests of \PGFPlots\ lead to numerous quality improvements. Furthermore, I thank Dr.~Schweitzer for many fruitful discussions and Dr.~Meine for his ideas and suggestions. Thanks as well to the many international contributors who provided feature requests or identified bugs or simply manual improvements!

Last but not least, I thank Till Tantau and Mark Wibrow for their excellent graphics (and more) package \PGF\ and \Tikz\ which is the base of \PGFPlots.

\include{pgfplots.install}%
\include{pgfplots.intro}%
\include{pgfplots.reference}%
\include{pgfplots.libs}%
\include{pgfplots.resources}%
\include{pgfplots.importexport}%
\include{pgfplots.basic.reference}%

\printindex

\bibliographystyle{abbrv} %gerapali} %gerabbrv} %gerunsrt.bst} %gerabbrv}% gerplain}
\nocite{pgfplotstable}
\nocite{programmingnotes}
\bibliography{pgfplots}
\end{document}
%
%%%%%%%%%%%%%%%%%%%%%%%%%%%%%%%%%%%%%%%%%%%%%%%%%%%%%%%%%%%%%%%%%%%%%%%%%%%%%
%
% Package pgfplots.sty documentation. 
%
% Copyright 2007/2008 by Christian Feuersaenger.
%
% This program is free software: you can redistribute it and/or modify
% it under the terms of the GNU General Public License as published by
% the Free Software Foundation, either version 3 of the License, or
% (at your option) any later version.
% 
% This program is distributed in the hope that it will be useful,
% but WITHOUT ANY WARRANTY; without even the implied warranty of
% MERCHANTABILITY or FITNESS FOR A PARTICULAR PURPOSE.  See the
% GNU General Public License for more details.
% 
% You should have received a copy of the GNU General Public License
% along with this program.  If not, see <http://www.gnu.org/licenses/>.
%
%
%%%%%%%%%%%%%%%%%%%%%%%%%%%%%%%%%%%%%%%%%%%%%%%%%%%%%%%%%%%%%%%%%%%%%%%%%%%%%
\input{pgfplots.preamble.tex}
%\RequirePackage[german,english,francais]{babel}

\pgfplotsmanualexternalexpensivetrue

%\usetikzlibrary{external}
%\tikzexternalize[prefix=figures/]{pgfplots}

\title{%
	Manual for Package \PGFPlots\\
	{\small Version \pgfplotsversion}\\
	{\small\href{http://sourceforge.net/projects/pgfplots}{http://sourceforge.net/projects/pgfplots}}}

%\includeonly{pgfplots.libs}


\begin{document}

\def\plotcoords{%
\addplot coordinates {
(5,8.312e-02)    (17,2.547e-02)   (49,7.407e-03)
(129,2.102e-03)  (321,5.874e-04)  (769,1.623e-04)
(1793,4.442e-05) (4097,1.207e-05) (9217,3.261e-06)
};

\addplot coordinates{
(7,8.472e-02)    (31,3.044e-02)    (111,1.022e-02)
(351,3.303e-03)  (1023,1.039e-03)  (2815,3.196e-04)
(7423,9.658e-05) (18943,2.873e-05) (47103,8.437e-06)
};

\addplot coordinates{
(9,7.881e-02)     (49,3.243e-02)    (209,1.232e-02)
(769,4.454e-03)   (2561,1.551e-03)  (7937,5.236e-04)
(23297,1.723e-04) (65537,5.545e-05) (178177,1.751e-05)
};

\addplot coordinates{
(11,6.887e-02)    (71,3.177e-02)     (351,1.341e-02)
(1471,5.334e-03)  (5503,2.027e-03)   (18943,7.415e-04)
(61183,2.628e-04) (187903,9.063e-05) (553983,3.053e-05)
};

\addplot coordinates{
(13,5.755e-02)     (97,2.925e-02)     (545,1.351e-02)
(2561,5.842e-03)   (10625,2.397e-03)  (40193,9.414e-04)
(141569,3.564e-04) (471041,1.308e-04) 
(1496065,4.670e-05)
};
}%
\maketitle
\begin{abstract}%
\PGFPlots\ draws high--quality function plots in normal or logarithmic scaling with a user-friendly interface directly in \TeX. The user supplies axis labels, legend entries and the plot coordinates for one or more plots and \PGFPlots\ applies axis scaling, computes any logarithms and axis ticks and draws the plots, supporting line plots, scatter plots, piecewise constant plots, bar plots, area plots, mesh-- and surface plots and some more. It is based on Till Tantau's package \PGF/\Tikz.
\end{abstract}
\tableofcontents
\section{Introduction}
This package provides tools to generate plots and labeled axes easily. It draws normal plots, logplots and semi-logplots, in two and three dimensions. Axis ticks, labels, legends (in case of multiple plots) can be added with key-value options. It can cycle through a set of predefined line/marker/color specifications. In summary, its purpose is to simplify the generation of high-quality function and/or data plots, and solving the problems of
\begin{itemize}
	\item consistency of document and font type and font size,
	\item direct use of \TeX\ math mode in axis descriptions,
	\item consistency of data and figures (no third party tool necessary),
	\item inter document consistency using preamble configurations and styles.
\end{itemize}
Although not necessary, separate |.pdf| or |.eps| graphics can be generated using the |external| library developed as part of \Tikz.

\PGFPlots\ is build completely on \Tikz/\PGF. Knowledge of \Tikz\ will simplify the work with \PGFPlots, although it is not required.

\section[About PGFPlots: Preliminaries]{About {\normalfont\PGFPlots}: Preliminaries}
This section contains information about upgrades, the team, the installation (in case you need to do it manually) and troubleshooting. You may skip it completely except for the upgrade remarks.

However, note that this library requires at least \PGF\ version $2.00$. At the time of this writing, many \TeX-distributions still contain the older \PGF\ version $1.18$, so it may be necessary to install a recent \PGF\ prior to using \PGFPlots.

\subsection{Components}
\PGFPlots\ comes with two components:
\begin{enumerate}
	\item the plotting component (which you are currently reading) and
	\item the \PGFPlotstable\ component which simplifies number formatting and postprocessing of numerical tables. It comes as a separate package and has its own manual \href{file:pgfplotstable.pdf}{pgfplotstable.pdf}.
\end{enumerate}

\subsection{Upgrade remarks}
This release provides a lot of improvements which can be found in all detail in \texttt{ChangeLog} for interested readers. However, some attention is useful with respect to the following changes.

\subsubsection{New Optional Features}
\begin{enumerate}
	\item \PGFPlots\ 1.3 comes with user interface improvements. The technical distinction between ``behavior options'' and ``style options'' of older versions is no longer necessary (although still fully supported).

	\item \PGFPlots\ 1.3 has a new feature which allows to \emph{move axis labels tight to tick labels} automatically.

	Since this affects the spacing, it is not enabled be default.

	Use
\begin{codeexample}[code only]
\usepackage{pgfplots}
\pgfplotsset{compat=1.3}
\end{codeexample}
	in your preamble to benefit from the improved spacing\footnote{The |compat=1.3| setting is necessary for new features which move axis labels around like reversed axis. However, \PGFPlots\ will still work as it does in earlier versions, your documents will remain the same if you don't set it explicitly.}.
	Take a look at the next page for the precise definition of |compat|.

	\item \PGFPlots\ 1.3 now supports reversed axes. It is no longer necessary to use work--arounds with negative units.
\pgfkeys{/pdflinks/search key prefixes in/.add={/pgfplots/,}{}}

	Take a look at the |x dir=reverse| key.

	Existing work--arounds will still function properly. Use |\pgfplotsset{compat=1.3}| together with |x dir=reverse| to switch to the new version.
\end{enumerate}

\subsubsection{Old Features Which May Need Attention}
\begin{enumerate}
	\item The |scatter/classes| feature produces proper legends as of version 1.3. This may change the appearance of existing legends of plots with |scatter/classes|.

	\item Due to a small math bug in \PGF\ $2.00$, you \emph{can't} use the math expression `|-x^2|'. It is necessary to use `|0-x^2|' instead. The same holds for
		`|exp(-x^2)|'; use `|exp(0-x^2)|' instead.

		This will be fixed with \PGF\ $>2.00$.

	\item Starting with \PGFPlots\ $1.1$, |\tikzstyle| should \emph{no longer be used} to set \PGFPlots\ options.
	
	Although |\tikzstyle| is still supported for some older \PGFPlots\ options, you should replace any occurance of |\tikzstyle| with |\pgfplotsset{|\meta{style name}|/.style={|\meta{key-value-list}|}}| or the associated |/.append style| variant. See section~\ref{sec:styles} for more detail.
\end{enumerate}
I apologize for any inconvenience caused by these changes.

\begin{pgfplotskey}{compat=\mchoice{1.3,pre 1.3,default,newest} (initially default)}
	The preamble configuration 
\begin{codeexample}[code only]
\usepackage{pgfplots}
\pgfplotsset{compat=1.3}
\end{codeexample}
	allows to choose between backwards compatibility and most recent features.

	\PGFPlots\ version 1.3 comes with new features which may lead to different spacings compared to earlier versions:
	\begin{enumerate}
		\item Version 1.3 comes with the |xlabel near ticks| feature which places (in this case $x$) axis labels ``near'' the tick labels, taking the size of tick labels into account.
		
		Older versions used |xlabel absolute|, i.e.\ an absolute distance between the axis and axis labels (now matter how large or small tick labels are).

		The initial setting \emph{keeps backwards compatibility}.

		You are encouraged to use
\begin{codeexample}[code only]
\usepackage{pgfplots}
\pgfplotsset{compat=1.3}
\end{codeexample}
		in your preamble.
		
		\item Version 1.3 fixes a bug with |anchor=south west| which occurred mainly for reversed axes: the anchors where upside--down. This is fixed now.

		
		If you have written some work--around and want to keep it going, use

		|\pgfplotsset{compat/anchors=pre 1.3}|\pgfmanualpdflabel{/pgfplots/compat/anchors}{} 

		to disable the bug fix.
	\end{enumerate}

	Use |\pgfplotsset{compat=default}| to restore the factory settings.

	The setting |\pgfplotsset{compat=newest}| will always use features of the most recent version. This might result in small changes in the document's appearance.
\end{pgfplotskey}

\subsection{The Team}
\PGFPlots\ has been written mainly by Christian Feuers�nger with many improvements of Pascal Wolkotte and Nick Papior Andersen as a spare time project. We hope it is useful and provides valuable plots.

If you are interested in writing something but don't know how, consider reading the auxiliary manual \href{file:TeX-programming-notes.pdf}{TeX-programming-notes.pdf} which comes with \PGFPlots. It is far from complete, but maybe it is a good starting point (at least for more literature).

\subsection{Acknowledgements}
I thank God for all hours of enjoyed programming. I thank Pascal Wolkotte and Nick Papior Andersen for their programming efforts and contributions as part of the development team. I thank Stefan Tibus, who contributed the |plot shell| feature. I thank Tom Cashman for the contribution of the |reverse legend| feature. Special thanks go to Stefan Pinnow whose tests of \PGFPlots\ lead to numerous quality improvements. Furthermore, I thank Dr.~Schweitzer for many fruitful discussions and Dr.~Meine for his ideas and suggestions. Thanks as well to the many international contributors who provided feature requests or identified bugs or simply manual improvements!

Last but not least, I thank Till Tantau and Mark Wibrow for their excellent graphics (and more) package \PGF\ and \Tikz\ which is the base of \PGFPlots.

\include{pgfplots.install}%
\include{pgfplots.intro}%
\include{pgfplots.reference}%
\include{pgfplots.libs}%
\include{pgfplots.resources}%
\include{pgfplots.importexport}%
\include{pgfplots.basic.reference}%

\printindex

\bibliographystyle{abbrv} %gerapali} %gerabbrv} %gerunsrt.bst} %gerabbrv}% gerplain}
\nocite{pgfplotstable}
\nocite{programmingnotes}
\bibliography{pgfplots}
\end{document}
%
\include{pgfplots.basic.reference}%

\printindex

\bibliographystyle{abbrv} %gerapali} %gerabbrv} %gerunsrt.bst} %gerabbrv}% gerplain}
\nocite{pgfplotstable}
\nocite{programmingnotes}
\bibliography{pgfplots}
\end{document}
%
%%%%%%%%%%%%%%%%%%%%%%%%%%%%%%%%%%%%%%%%%%%%%%%%%%%%%%%%%%%%%%%%%%%%%%%%%%%%%
%
% Package pgfplots.sty documentation. 
%
% Copyright 2007/2008 by Christian Feuersaenger.
%
% This program is free software: you can redistribute it and/or modify
% it under the terms of the GNU General Public License as published by
% the Free Software Foundation, either version 3 of the License, or
% (at your option) any later version.
% 
% This program is distributed in the hope that it will be useful,
% but WITHOUT ANY WARRANTY; without even the implied warranty of
% MERCHANTABILITY or FITNESS FOR A PARTICULAR PURPOSE.  See the
% GNU General Public License for more details.
% 
% You should have received a copy of the GNU General Public License
% along with this program.  If not, see <http://www.gnu.org/licenses/>.
%
%
%%%%%%%%%%%%%%%%%%%%%%%%%%%%%%%%%%%%%%%%%%%%%%%%%%%%%%%%%%%%%%%%%%%%%%%%%%%%%
%%%%%%%%%%%%%%%%%%%%%%%%%%%%%%%%%%%%%%%%%%%%%%%%%%%%%%%%%%%%%%%%%%%%%%%%%%%%%
%
% Package pgfplots.sty documentation. 
%
% Copyright 2007/2008 by Christian Feuersaenger.
%
% This program is free software: you can redistribute it and/or modify
% it under the terms of the GNU General Public License as published by
% the Free Software Foundation, either version 3 of the License, or
% (at your option) any later version.
% 
% This program is distributed in the hope that it will be useful,
% but WITHOUT ANY WARRANTY; without even the implied warranty of
% MERCHANTABILITY or FITNESS FOR A PARTICULAR PURPOSE.  See the
% GNU General Public License for more details.
% 
% You should have received a copy of the GNU General Public License
% along with this program.  If not, see <http://www.gnu.org/licenses/>.
%
%
%%%%%%%%%%%%%%%%%%%%%%%%%%%%%%%%%%%%%%%%%%%%%%%%%%%%%%%%%%%%%%%%%%%%%%%%%%%%%
\documentclass[a4paper]{ltxdoc}

\usepackage{makeidx}

% DON't let hyperref overload the format of index and glossary. 
% I want to do that on my own in the stylefiles for makeindex...
\makeatletter
\let\@old@wrindex=\@wrindex
\makeatother

\usepackage{ifpdf}
\ifpdf
	\usepackage[pdftex]{hyperref}
\else
	\usepackage[dvipdfm]{hyperref}
\fi
\hypersetup{pdfborder={0 0 0}}

\makeatletter
\let\@wrindex=\@old@wrindex
\makeatother

% Formatiere Seitennummern im Index:
\newcommand{\indexpageno}[1]{%
	{\bfseries\hyperpage{#1}}%
}


\newcommand{\R}{\mathbb{R}}
\newcommand{\N}{\mathbb{N}}
\newcommand{\Z}{\mathbb{Z}}

\long\def\COMMENTLOWLEVEL#1\ENDCOMMENT{}
\def\ENDCOMMENT{}

\usepackage{textcomp}
\usepackage{booktabs}

\usepackage{calc}
\usepackage[formats]{listings}
%\usepackage{courier} % don't use it - the '^' character can't be copy-pasted in courier

\usepackage{array}
\lstset{%
	basicstyle=\ttfamily,
	language=[LaTeX]tex, % Seems as if \lstset{language=tex} must be invoked BEFORE loading tikz!?
	tabsize=4,
	breaklines=true,
	breakindent=0pt
}

\ifpdf
	\pdfinfo {
		/Author	(Christian Feuersaenger)
	}

\else
	\def\pgfsysdriver{pgfsys-dvipdfm.def}
\fi
%\def\pgfsysdriver{pgfsys-pdftex.def}
\usepackage{pgfplots}

\ifpdf
	\usetikzlibrary{pgfplotsclickable}
\fi

%\usepackage{fp}
% ATTENTION:
% this requires pgf version NEWER than 2.00 :
%\usetikzlibrary{fixedpointarithmetic}

\usetikzlibrary{dateplot}

\usepackage[a4paper,left=2.25cm,right=2.25cm,top=2.5cm,bottom=2.5cm,nohead]{geometry}
\usepackage{amsmath,amssymb}
\usepackage{xxcolor}
\usepackage{pifont}
\usepackage[latin1]{inputenc}
\usepackage{amsmath}
\usepackage{eurosym}
\input{pgfplots-macros}

\pgfqkeys{/codeexample}{%
	every codeexample/.style={
		width=8cm,
		/pgfplots/every axis/.append style={legend style={fill=graphicbackground}}
	},
	tabsize=4,
}
\pgfplotsset{
	every axis/.append style={width=7cm},
	filter discard warning=false,
}

\usetikzlibrary{backgrounds,patterns}
% Global styles:
\tikzset{
  shape example/.style={
    color=black!30,
    draw,
    fill=yellow!30,
    line width=.5cm,
    inner xsep=2.5cm,
    inner ysep=0.5cm}
}

\newcommand{\FIXME}[1]{\textcolor{red}{(FIXME: #1)}}

% fuer endvironment 'sidewaysfigure' bspw
% \usepackage{rotating}

\newcommand\Tikz{Ti\textit kZ}
\newcommand\PGF{\textsc{pgf}}
\newcommand\PGFPlots{\textsc{pgfplots}}
\def\PGFPlotstable{\textsc{PgfplotsTable}}


\makeindex

% Fix overful hboxes automatically:
\tolerance=2000
\emergencystretch=10pt

\tikzset{prefix=gnuplot/pgfplots_} % prefix for 'plot function'

\author{%
	Christian Feuers\"anger\footnote{\url{http://wissrech.ins.uni-bonn.de/people/feuersaenger}}\\%
	Institut f\"ur Numerische Simulation\\
	Universit\"at Bonn, Germany}


%\RequirePackage[german,english,francais]{babel}

\pgfplotsmanualexternalexpensivetrue

%\usetikzlibrary{external}
%\tikzexternalize[prefix=figures/]{pgfplots}

\title{%
	Manual for Package \PGFPlots\\
	{\small Version \pgfplotsversion}\\
	{\small\href{http://sourceforge.net/projects/pgfplots}{http://sourceforge.net/projects/pgfplots}}}

%\includeonly{pgfplots.libs}


\begin{document}

\def\plotcoords{%
\addplot coordinates {
(5,8.312e-02)    (17,2.547e-02)   (49,7.407e-03)
(129,2.102e-03)  (321,5.874e-04)  (769,1.623e-04)
(1793,4.442e-05) (4097,1.207e-05) (9217,3.261e-06)
};

\addplot coordinates{
(7,8.472e-02)    (31,3.044e-02)    (111,1.022e-02)
(351,3.303e-03)  (1023,1.039e-03)  (2815,3.196e-04)
(7423,9.658e-05) (18943,2.873e-05) (47103,8.437e-06)
};

\addplot coordinates{
(9,7.881e-02)     (49,3.243e-02)    (209,1.232e-02)
(769,4.454e-03)   (2561,1.551e-03)  (7937,5.236e-04)
(23297,1.723e-04) (65537,5.545e-05) (178177,1.751e-05)
};

\addplot coordinates{
(11,6.887e-02)    (71,3.177e-02)     (351,1.341e-02)
(1471,5.334e-03)  (5503,2.027e-03)   (18943,7.415e-04)
(61183,2.628e-04) (187903,9.063e-05) (553983,3.053e-05)
};

\addplot coordinates{
(13,5.755e-02)     (97,2.925e-02)     (545,1.351e-02)
(2561,5.842e-03)   (10625,2.397e-03)  (40193,9.414e-04)
(141569,3.564e-04) (471041,1.308e-04) 
(1496065,4.670e-05)
};
}%
\maketitle
\begin{abstract}%
\PGFPlots\ draws high--quality function plots in normal or logarithmic scaling with a user-friendly interface directly in \TeX. The user supplies axis labels, legend entries and the plot coordinates for one or more plots and \PGFPlots\ applies axis scaling, computes any logarithms and axis ticks and draws the plots, supporting line plots, scatter plots, piecewise constant plots, bar plots, area plots, mesh-- and surface plots and some more. It is based on Till Tantau's package \PGF/\Tikz.
\end{abstract}
\tableofcontents
\section{Introduction}
This package provides tools to generate plots and labeled axes easily. It draws normal plots, logplots and semi-logplots, in two and three dimensions. Axis ticks, labels, legends (in case of multiple plots) can be added with key-value options. It can cycle through a set of predefined line/marker/color specifications. In summary, its purpose is to simplify the generation of high-quality function and/or data plots, and solving the problems of
\begin{itemize}
	\item consistency of document and font type and font size,
	\item direct use of \TeX\ math mode in axis descriptions,
	\item consistency of data and figures (no third party tool necessary),
	\item inter document consistency using preamble configurations and styles.
\end{itemize}
Although not necessary, separate |.pdf| or |.eps| graphics can be generated using the |external| library developed as part of \Tikz.

\PGFPlots\ is build completely on \Tikz/\PGF. Knowledge of \Tikz\ will simplify the work with \PGFPlots, although it is not required.

\section[About PGFPlots: Preliminaries]{About {\normalfont\PGFPlots}: Preliminaries}
This section contains information about upgrades, the team, the installation (in case you need to do it manually) and troubleshooting. You may skip it completely except for the upgrade remarks.

However, note that this library requires at least \PGF\ version $2.00$. At the time of this writing, many \TeX-distributions still contain the older \PGF\ version $1.18$, so it may be necessary to install a recent \PGF\ prior to using \PGFPlots.

\subsection{Components}
\PGFPlots\ comes with two components:
\begin{enumerate}
	\item the plotting component (which you are currently reading) and
	\item the \PGFPlotstable\ component which simplifies number formatting and postprocessing of numerical tables. It comes as a separate package and has its own manual \href{file:pgfplotstable.pdf}{pgfplotstable.pdf}.
\end{enumerate}

\subsection{Upgrade remarks}
This release provides a lot of improvements which can be found in all detail in \texttt{ChangeLog} for interested readers. However, some attention is useful with respect to the following changes.

\subsubsection{New Optional Features}
\begin{enumerate}
	\item \PGFPlots\ 1.3 comes with user interface improvements. The technical distinction between ``behavior options'' and ``style options'' of older versions is no longer necessary (although still fully supported).

	\item \PGFPlots\ 1.3 has a new feature which allows to \emph{move axis labels tight to tick labels} automatically.

	Since this affects the spacing, it is not enabled be default.

	Use
\begin{codeexample}[code only]
\usepackage{pgfplots}
\pgfplotsset{compat=1.3}
\end{codeexample}
	in your preamble to benefit from the improved spacing\footnote{The |compat=1.3| setting is necessary for new features which move axis labels around like reversed axis. However, \PGFPlots\ will still work as it does in earlier versions, your documents will remain the same if you don't set it explicitly.}.
	Take a look at the next page for the precise definition of |compat|.

	\item \PGFPlots\ 1.3 now supports reversed axes. It is no longer necessary to use work--arounds with negative units.
\pgfkeys{/pdflinks/search key prefixes in/.add={/pgfplots/,}{}}

	Take a look at the |x dir=reverse| key.

	Existing work--arounds will still function properly. Use |\pgfplotsset{compat=1.3}| together with |x dir=reverse| to switch to the new version.
\end{enumerate}

\subsubsection{Old Features Which May Need Attention}
\begin{enumerate}
	\item The |scatter/classes| feature produces proper legends as of version 1.3. This may change the appearance of existing legends of plots with |scatter/classes|.

	\item Due to a small math bug in \PGF\ $2.00$, you \emph{can't} use the math expression `|-x^2|'. It is necessary to use `|0-x^2|' instead. The same holds for
		`|exp(-x^2)|'; use `|exp(0-x^2)|' instead.

		This will be fixed with \PGF\ $>2.00$.

	\item Starting with \PGFPlots\ $1.1$, |\tikzstyle| should \emph{no longer be used} to set \PGFPlots\ options.
	
	Although |\tikzstyle| is still supported for some older \PGFPlots\ options, you should replace any occurance of |\tikzstyle| with |\pgfplotsset{|\meta{style name}|/.style={|\meta{key-value-list}|}}| or the associated |/.append style| variant. See section~\ref{sec:styles} for more detail.
\end{enumerate}
I apologize for any inconvenience caused by these changes.

\begin{pgfplotskey}{compat=\mchoice{1.3,pre 1.3,default,newest} (initially default)}
	The preamble configuration 
\begin{codeexample}[code only]
\usepackage{pgfplots}
\pgfplotsset{compat=1.3}
\end{codeexample}
	allows to choose between backwards compatibility and most recent features.

	\PGFPlots\ version 1.3 comes with new features which may lead to different spacings compared to earlier versions:
	\begin{enumerate}
		\item Version 1.3 comes with the |xlabel near ticks| feature which places (in this case $x$) axis labels ``near'' the tick labels, taking the size of tick labels into account.
		
		Older versions used |xlabel absolute|, i.e.\ an absolute distance between the axis and axis labels (now matter how large or small tick labels are).

		The initial setting \emph{keeps backwards compatibility}.

		You are encouraged to use
\begin{codeexample}[code only]
\usepackage{pgfplots}
\pgfplotsset{compat=1.3}
\end{codeexample}
		in your preamble.
		
		\item Version 1.3 fixes a bug with |anchor=south west| which occurred mainly for reversed axes: the anchors where upside--down. This is fixed now.

		
		If you have written some work--around and want to keep it going, use

		|\pgfplotsset{compat/anchors=pre 1.3}|\pgfmanualpdflabel{/pgfplots/compat/anchors}{} 

		to disable the bug fix.
	\end{enumerate}

	Use |\pgfplotsset{compat=default}| to restore the factory settings.

	The setting |\pgfplotsset{compat=newest}| will always use features of the most recent version. This might result in small changes in the document's appearance.
\end{pgfplotskey}

\subsection{The Team}
\PGFPlots\ has been written mainly by Christian Feuers�nger with many improvements of Pascal Wolkotte and Nick Papior Andersen as a spare time project. We hope it is useful and provides valuable plots.

If you are interested in writing something but don't know how, consider reading the auxiliary manual \href{file:TeX-programming-notes.pdf}{TeX-programming-notes.pdf} which comes with \PGFPlots. It is far from complete, but maybe it is a good starting point (at least for more literature).

\subsection{Acknowledgements}
I thank God for all hours of enjoyed programming. I thank Pascal Wolkotte and Nick Papior Andersen for their programming efforts and contributions as part of the development team. I thank Stefan Tibus, who contributed the |plot shell| feature. I thank Tom Cashman for the contribution of the |reverse legend| feature. Special thanks go to Stefan Pinnow whose tests of \PGFPlots\ lead to numerous quality improvements. Furthermore, I thank Dr.~Schweitzer for many fruitful discussions and Dr.~Meine for his ideas and suggestions. Thanks as well to the many international contributors who provided feature requests or identified bugs or simply manual improvements!

Last but not least, I thank Till Tantau and Mark Wibrow for their excellent graphics (and more) package \PGF\ and \Tikz\ which is the base of \PGFPlots.

%%%%%%%%%%%%%%%%%%%%%%%%%%%%%%%%%%%%%%%%%%%%%%%%%%%%%%%%%%%%%%%%%%%%%%%%%%%%%
%
% Package pgfplots.sty documentation. 
%
% Copyright 2007/2008 by Christian Feuersaenger.
%
% This program is free software: you can redistribute it and/or modify
% it under the terms of the GNU General Public License as published by
% the Free Software Foundation, either version 3 of the License, or
% (at your option) any later version.
% 
% This program is distributed in the hope that it will be useful,
% but WITHOUT ANY WARRANTY; without even the implied warranty of
% MERCHANTABILITY or FITNESS FOR A PARTICULAR PURPOSE.  See the
% GNU General Public License for more details.
% 
% You should have received a copy of the GNU General Public License
% along with this program.  If not, see <http://www.gnu.org/licenses/>.
%
%
%%%%%%%%%%%%%%%%%%%%%%%%%%%%%%%%%%%%%%%%%%%%%%%%%%%%%%%%%%%%%%%%%%%%%%%%%%%%%
\input{pgfplots.preamble.tex}
%\RequirePackage[german,english,francais]{babel}

\pgfplotsmanualexternalexpensivetrue

%\usetikzlibrary{external}
%\tikzexternalize[prefix=figures/]{pgfplots}

\title{%
	Manual for Package \PGFPlots\\
	{\small Version \pgfplotsversion}\\
	{\small\href{http://sourceforge.net/projects/pgfplots}{http://sourceforge.net/projects/pgfplots}}}

%\includeonly{pgfplots.libs}


\begin{document}

\def\plotcoords{%
\addplot coordinates {
(5,8.312e-02)    (17,2.547e-02)   (49,7.407e-03)
(129,2.102e-03)  (321,5.874e-04)  (769,1.623e-04)
(1793,4.442e-05) (4097,1.207e-05) (9217,3.261e-06)
};

\addplot coordinates{
(7,8.472e-02)    (31,3.044e-02)    (111,1.022e-02)
(351,3.303e-03)  (1023,1.039e-03)  (2815,3.196e-04)
(7423,9.658e-05) (18943,2.873e-05) (47103,8.437e-06)
};

\addplot coordinates{
(9,7.881e-02)     (49,3.243e-02)    (209,1.232e-02)
(769,4.454e-03)   (2561,1.551e-03)  (7937,5.236e-04)
(23297,1.723e-04) (65537,5.545e-05) (178177,1.751e-05)
};

\addplot coordinates{
(11,6.887e-02)    (71,3.177e-02)     (351,1.341e-02)
(1471,5.334e-03)  (5503,2.027e-03)   (18943,7.415e-04)
(61183,2.628e-04) (187903,9.063e-05) (553983,3.053e-05)
};

\addplot coordinates{
(13,5.755e-02)     (97,2.925e-02)     (545,1.351e-02)
(2561,5.842e-03)   (10625,2.397e-03)  (40193,9.414e-04)
(141569,3.564e-04) (471041,1.308e-04) 
(1496065,4.670e-05)
};
}%
\maketitle
\begin{abstract}%
\PGFPlots\ draws high--quality function plots in normal or logarithmic scaling with a user-friendly interface directly in \TeX. The user supplies axis labels, legend entries and the plot coordinates for one or more plots and \PGFPlots\ applies axis scaling, computes any logarithms and axis ticks and draws the plots, supporting line plots, scatter plots, piecewise constant plots, bar plots, area plots, mesh-- and surface plots and some more. It is based on Till Tantau's package \PGF/\Tikz.
\end{abstract}
\tableofcontents
\section{Introduction}
This package provides tools to generate plots and labeled axes easily. It draws normal plots, logplots and semi-logplots, in two and three dimensions. Axis ticks, labels, legends (in case of multiple plots) can be added with key-value options. It can cycle through a set of predefined line/marker/color specifications. In summary, its purpose is to simplify the generation of high-quality function and/or data plots, and solving the problems of
\begin{itemize}
	\item consistency of document and font type and font size,
	\item direct use of \TeX\ math mode in axis descriptions,
	\item consistency of data and figures (no third party tool necessary),
	\item inter document consistency using preamble configurations and styles.
\end{itemize}
Although not necessary, separate |.pdf| or |.eps| graphics can be generated using the |external| library developed as part of \Tikz.

\PGFPlots\ is build completely on \Tikz/\PGF. Knowledge of \Tikz\ will simplify the work with \PGFPlots, although it is not required.

\section[About PGFPlots: Preliminaries]{About {\normalfont\PGFPlots}: Preliminaries}
This section contains information about upgrades, the team, the installation (in case you need to do it manually) and troubleshooting. You may skip it completely except for the upgrade remarks.

However, note that this library requires at least \PGF\ version $2.00$. At the time of this writing, many \TeX-distributions still contain the older \PGF\ version $1.18$, so it may be necessary to install a recent \PGF\ prior to using \PGFPlots.

\subsection{Components}
\PGFPlots\ comes with two components:
\begin{enumerate}
	\item the plotting component (which you are currently reading) and
	\item the \PGFPlotstable\ component which simplifies number formatting and postprocessing of numerical tables. It comes as a separate package and has its own manual \href{file:pgfplotstable.pdf}{pgfplotstable.pdf}.
\end{enumerate}

\subsection{Upgrade remarks}
This release provides a lot of improvements which can be found in all detail in \texttt{ChangeLog} for interested readers. However, some attention is useful with respect to the following changes.

\subsubsection{New Optional Features}
\begin{enumerate}
	\item \PGFPlots\ 1.3 comes with user interface improvements. The technical distinction between ``behavior options'' and ``style options'' of older versions is no longer necessary (although still fully supported).

	\item \PGFPlots\ 1.3 has a new feature which allows to \emph{move axis labels tight to tick labels} automatically.

	Since this affects the spacing, it is not enabled be default.

	Use
\begin{codeexample}[code only]
\usepackage{pgfplots}
\pgfplotsset{compat=1.3}
\end{codeexample}
	in your preamble to benefit from the improved spacing\footnote{The |compat=1.3| setting is necessary for new features which move axis labels around like reversed axis. However, \PGFPlots\ will still work as it does in earlier versions, your documents will remain the same if you don't set it explicitly.}.
	Take a look at the next page for the precise definition of |compat|.

	\item \PGFPlots\ 1.3 now supports reversed axes. It is no longer necessary to use work--arounds with negative units.
\pgfkeys{/pdflinks/search key prefixes in/.add={/pgfplots/,}{}}

	Take a look at the |x dir=reverse| key.

	Existing work--arounds will still function properly. Use |\pgfplotsset{compat=1.3}| together with |x dir=reverse| to switch to the new version.
\end{enumerate}

\subsubsection{Old Features Which May Need Attention}
\begin{enumerate}
	\item The |scatter/classes| feature produces proper legends as of version 1.3. This may change the appearance of existing legends of plots with |scatter/classes|.

	\item Due to a small math bug in \PGF\ $2.00$, you \emph{can't} use the math expression `|-x^2|'. It is necessary to use `|0-x^2|' instead. The same holds for
		`|exp(-x^2)|'; use `|exp(0-x^2)|' instead.

		This will be fixed with \PGF\ $>2.00$.

	\item Starting with \PGFPlots\ $1.1$, |\tikzstyle| should \emph{no longer be used} to set \PGFPlots\ options.
	
	Although |\tikzstyle| is still supported for some older \PGFPlots\ options, you should replace any occurance of |\tikzstyle| with |\pgfplotsset{|\meta{style name}|/.style={|\meta{key-value-list}|}}| or the associated |/.append style| variant. See section~\ref{sec:styles} for more detail.
\end{enumerate}
I apologize for any inconvenience caused by these changes.

\begin{pgfplotskey}{compat=\mchoice{1.3,pre 1.3,default,newest} (initially default)}
	The preamble configuration 
\begin{codeexample}[code only]
\usepackage{pgfplots}
\pgfplotsset{compat=1.3}
\end{codeexample}
	allows to choose between backwards compatibility and most recent features.

	\PGFPlots\ version 1.3 comes with new features which may lead to different spacings compared to earlier versions:
	\begin{enumerate}
		\item Version 1.3 comes with the |xlabel near ticks| feature which places (in this case $x$) axis labels ``near'' the tick labels, taking the size of tick labels into account.
		
		Older versions used |xlabel absolute|, i.e.\ an absolute distance between the axis and axis labels (now matter how large or small tick labels are).

		The initial setting \emph{keeps backwards compatibility}.

		You are encouraged to use
\begin{codeexample}[code only]
\usepackage{pgfplots}
\pgfplotsset{compat=1.3}
\end{codeexample}
		in your preamble.
		
		\item Version 1.3 fixes a bug with |anchor=south west| which occurred mainly for reversed axes: the anchors where upside--down. This is fixed now.

		
		If you have written some work--around and want to keep it going, use

		|\pgfplotsset{compat/anchors=pre 1.3}|\pgfmanualpdflabel{/pgfplots/compat/anchors}{} 

		to disable the bug fix.
	\end{enumerate}

	Use |\pgfplotsset{compat=default}| to restore the factory settings.

	The setting |\pgfplotsset{compat=newest}| will always use features of the most recent version. This might result in small changes in the document's appearance.
\end{pgfplotskey}

\subsection{The Team}
\PGFPlots\ has been written mainly by Christian Feuers�nger with many improvements of Pascal Wolkotte and Nick Papior Andersen as a spare time project. We hope it is useful and provides valuable plots.

If you are interested in writing something but don't know how, consider reading the auxiliary manual \href{file:TeX-programming-notes.pdf}{TeX-programming-notes.pdf} which comes with \PGFPlots. It is far from complete, but maybe it is a good starting point (at least for more literature).

\subsection{Acknowledgements}
I thank God for all hours of enjoyed programming. I thank Pascal Wolkotte and Nick Papior Andersen for their programming efforts and contributions as part of the development team. I thank Stefan Tibus, who contributed the |plot shell| feature. I thank Tom Cashman for the contribution of the |reverse legend| feature. Special thanks go to Stefan Pinnow whose tests of \PGFPlots\ lead to numerous quality improvements. Furthermore, I thank Dr.~Schweitzer for many fruitful discussions and Dr.~Meine for his ideas and suggestions. Thanks as well to the many international contributors who provided feature requests or identified bugs or simply manual improvements!

Last but not least, I thank Till Tantau and Mark Wibrow for their excellent graphics (and more) package \PGF\ and \Tikz\ which is the base of \PGFPlots.

\include{pgfplots.install}%
\include{pgfplots.intro}%
\include{pgfplots.reference}%
\include{pgfplots.libs}%
\include{pgfplots.resources}%
\include{pgfplots.importexport}%
\include{pgfplots.basic.reference}%

\printindex

\bibliographystyle{abbrv} %gerapali} %gerabbrv} %gerunsrt.bst} %gerabbrv}% gerplain}
\nocite{pgfplotstable}
\nocite{programmingnotes}
\bibliography{pgfplots}
\end{document}
%
%%%%%%%%%%%%%%%%%%%%%%%%%%%%%%%%%%%%%%%%%%%%%%%%%%%%%%%%%%%%%%%%%%%%%%%%%%%%%
%
% Package pgfplots.sty documentation. 
%
% Copyright 2007/2008 by Christian Feuersaenger.
%
% This program is free software: you can redistribute it and/or modify
% it under the terms of the GNU General Public License as published by
% the Free Software Foundation, either version 3 of the License, or
% (at your option) any later version.
% 
% This program is distributed in the hope that it will be useful,
% but WITHOUT ANY WARRANTY; without even the implied warranty of
% MERCHANTABILITY or FITNESS FOR A PARTICULAR PURPOSE.  See the
% GNU General Public License for more details.
% 
% You should have received a copy of the GNU General Public License
% along with this program.  If not, see <http://www.gnu.org/licenses/>.
%
%
%%%%%%%%%%%%%%%%%%%%%%%%%%%%%%%%%%%%%%%%%%%%%%%%%%%%%%%%%%%%%%%%%%%%%%%%%%%%%
\input{pgfplots.preamble.tex}
%\RequirePackage[german,english,francais]{babel}

\pgfplotsmanualexternalexpensivetrue

%\usetikzlibrary{external}
%\tikzexternalize[prefix=figures/]{pgfplots}

\title{%
	Manual for Package \PGFPlots\\
	{\small Version \pgfplotsversion}\\
	{\small\href{http://sourceforge.net/projects/pgfplots}{http://sourceforge.net/projects/pgfplots}}}

%\includeonly{pgfplots.libs}


\begin{document}

\def\plotcoords{%
\addplot coordinates {
(5,8.312e-02)    (17,2.547e-02)   (49,7.407e-03)
(129,2.102e-03)  (321,5.874e-04)  (769,1.623e-04)
(1793,4.442e-05) (4097,1.207e-05) (9217,3.261e-06)
};

\addplot coordinates{
(7,8.472e-02)    (31,3.044e-02)    (111,1.022e-02)
(351,3.303e-03)  (1023,1.039e-03)  (2815,3.196e-04)
(7423,9.658e-05) (18943,2.873e-05) (47103,8.437e-06)
};

\addplot coordinates{
(9,7.881e-02)     (49,3.243e-02)    (209,1.232e-02)
(769,4.454e-03)   (2561,1.551e-03)  (7937,5.236e-04)
(23297,1.723e-04) (65537,5.545e-05) (178177,1.751e-05)
};

\addplot coordinates{
(11,6.887e-02)    (71,3.177e-02)     (351,1.341e-02)
(1471,5.334e-03)  (5503,2.027e-03)   (18943,7.415e-04)
(61183,2.628e-04) (187903,9.063e-05) (553983,3.053e-05)
};

\addplot coordinates{
(13,5.755e-02)     (97,2.925e-02)     (545,1.351e-02)
(2561,5.842e-03)   (10625,2.397e-03)  (40193,9.414e-04)
(141569,3.564e-04) (471041,1.308e-04) 
(1496065,4.670e-05)
};
}%
\maketitle
\begin{abstract}%
\PGFPlots\ draws high--quality function plots in normal or logarithmic scaling with a user-friendly interface directly in \TeX. The user supplies axis labels, legend entries and the plot coordinates for one or more plots and \PGFPlots\ applies axis scaling, computes any logarithms and axis ticks and draws the plots, supporting line plots, scatter plots, piecewise constant plots, bar plots, area plots, mesh-- and surface plots and some more. It is based on Till Tantau's package \PGF/\Tikz.
\end{abstract}
\tableofcontents
\section{Introduction}
This package provides tools to generate plots and labeled axes easily. It draws normal plots, logplots and semi-logplots, in two and three dimensions. Axis ticks, labels, legends (in case of multiple plots) can be added with key-value options. It can cycle through a set of predefined line/marker/color specifications. In summary, its purpose is to simplify the generation of high-quality function and/or data plots, and solving the problems of
\begin{itemize}
	\item consistency of document and font type and font size,
	\item direct use of \TeX\ math mode in axis descriptions,
	\item consistency of data and figures (no third party tool necessary),
	\item inter document consistency using preamble configurations and styles.
\end{itemize}
Although not necessary, separate |.pdf| or |.eps| graphics can be generated using the |external| library developed as part of \Tikz.

\PGFPlots\ is build completely on \Tikz/\PGF. Knowledge of \Tikz\ will simplify the work with \PGFPlots, although it is not required.

\section[About PGFPlots: Preliminaries]{About {\normalfont\PGFPlots}: Preliminaries}
This section contains information about upgrades, the team, the installation (in case you need to do it manually) and troubleshooting. You may skip it completely except for the upgrade remarks.

However, note that this library requires at least \PGF\ version $2.00$. At the time of this writing, many \TeX-distributions still contain the older \PGF\ version $1.18$, so it may be necessary to install a recent \PGF\ prior to using \PGFPlots.

\subsection{Components}
\PGFPlots\ comes with two components:
\begin{enumerate}
	\item the plotting component (which you are currently reading) and
	\item the \PGFPlotstable\ component which simplifies number formatting and postprocessing of numerical tables. It comes as a separate package and has its own manual \href{file:pgfplotstable.pdf}{pgfplotstable.pdf}.
\end{enumerate}

\subsection{Upgrade remarks}
This release provides a lot of improvements which can be found in all detail in \texttt{ChangeLog} for interested readers. However, some attention is useful with respect to the following changes.

\subsubsection{New Optional Features}
\begin{enumerate}
	\item \PGFPlots\ 1.3 comes with user interface improvements. The technical distinction between ``behavior options'' and ``style options'' of older versions is no longer necessary (although still fully supported).

	\item \PGFPlots\ 1.3 has a new feature which allows to \emph{move axis labels tight to tick labels} automatically.

	Since this affects the spacing, it is not enabled be default.

	Use
\begin{codeexample}[code only]
\usepackage{pgfplots}
\pgfplotsset{compat=1.3}
\end{codeexample}
	in your preamble to benefit from the improved spacing\footnote{The |compat=1.3| setting is necessary for new features which move axis labels around like reversed axis. However, \PGFPlots\ will still work as it does in earlier versions, your documents will remain the same if you don't set it explicitly.}.
	Take a look at the next page for the precise definition of |compat|.

	\item \PGFPlots\ 1.3 now supports reversed axes. It is no longer necessary to use work--arounds with negative units.
\pgfkeys{/pdflinks/search key prefixes in/.add={/pgfplots/,}{}}

	Take a look at the |x dir=reverse| key.

	Existing work--arounds will still function properly. Use |\pgfplotsset{compat=1.3}| together with |x dir=reverse| to switch to the new version.
\end{enumerate}

\subsubsection{Old Features Which May Need Attention}
\begin{enumerate}
	\item The |scatter/classes| feature produces proper legends as of version 1.3. This may change the appearance of existing legends of plots with |scatter/classes|.

	\item Due to a small math bug in \PGF\ $2.00$, you \emph{can't} use the math expression `|-x^2|'. It is necessary to use `|0-x^2|' instead. The same holds for
		`|exp(-x^2)|'; use `|exp(0-x^2)|' instead.

		This will be fixed with \PGF\ $>2.00$.

	\item Starting with \PGFPlots\ $1.1$, |\tikzstyle| should \emph{no longer be used} to set \PGFPlots\ options.
	
	Although |\tikzstyle| is still supported for some older \PGFPlots\ options, you should replace any occurance of |\tikzstyle| with |\pgfplotsset{|\meta{style name}|/.style={|\meta{key-value-list}|}}| or the associated |/.append style| variant. See section~\ref{sec:styles} for more detail.
\end{enumerate}
I apologize for any inconvenience caused by these changes.

\begin{pgfplotskey}{compat=\mchoice{1.3,pre 1.3,default,newest} (initially default)}
	The preamble configuration 
\begin{codeexample}[code only]
\usepackage{pgfplots}
\pgfplotsset{compat=1.3}
\end{codeexample}
	allows to choose between backwards compatibility and most recent features.

	\PGFPlots\ version 1.3 comes with new features which may lead to different spacings compared to earlier versions:
	\begin{enumerate}
		\item Version 1.3 comes with the |xlabel near ticks| feature which places (in this case $x$) axis labels ``near'' the tick labels, taking the size of tick labels into account.
		
		Older versions used |xlabel absolute|, i.e.\ an absolute distance between the axis and axis labels (now matter how large or small tick labels are).

		The initial setting \emph{keeps backwards compatibility}.

		You are encouraged to use
\begin{codeexample}[code only]
\usepackage{pgfplots}
\pgfplotsset{compat=1.3}
\end{codeexample}
		in your preamble.
		
		\item Version 1.3 fixes a bug with |anchor=south west| which occurred mainly for reversed axes: the anchors where upside--down. This is fixed now.

		
		If you have written some work--around and want to keep it going, use

		|\pgfplotsset{compat/anchors=pre 1.3}|\pgfmanualpdflabel{/pgfplots/compat/anchors}{} 

		to disable the bug fix.
	\end{enumerate}

	Use |\pgfplotsset{compat=default}| to restore the factory settings.

	The setting |\pgfplotsset{compat=newest}| will always use features of the most recent version. This might result in small changes in the document's appearance.
\end{pgfplotskey}

\subsection{The Team}
\PGFPlots\ has been written mainly by Christian Feuers�nger with many improvements of Pascal Wolkotte and Nick Papior Andersen as a spare time project. We hope it is useful and provides valuable plots.

If you are interested in writing something but don't know how, consider reading the auxiliary manual \href{file:TeX-programming-notes.pdf}{TeX-programming-notes.pdf} which comes with \PGFPlots. It is far from complete, but maybe it is a good starting point (at least for more literature).

\subsection{Acknowledgements}
I thank God for all hours of enjoyed programming. I thank Pascal Wolkotte and Nick Papior Andersen for their programming efforts and contributions as part of the development team. I thank Stefan Tibus, who contributed the |plot shell| feature. I thank Tom Cashman for the contribution of the |reverse legend| feature. Special thanks go to Stefan Pinnow whose tests of \PGFPlots\ lead to numerous quality improvements. Furthermore, I thank Dr.~Schweitzer for many fruitful discussions and Dr.~Meine for his ideas and suggestions. Thanks as well to the many international contributors who provided feature requests or identified bugs or simply manual improvements!

Last but not least, I thank Till Tantau and Mark Wibrow for their excellent graphics (and more) package \PGF\ and \Tikz\ which is the base of \PGFPlots.

\include{pgfplots.install}%
\include{pgfplots.intro}%
\include{pgfplots.reference}%
\include{pgfplots.libs}%
\include{pgfplots.resources}%
\include{pgfplots.importexport}%
\include{pgfplots.basic.reference}%

\printindex

\bibliographystyle{abbrv} %gerapali} %gerabbrv} %gerunsrt.bst} %gerabbrv}% gerplain}
\nocite{pgfplotstable}
\nocite{programmingnotes}
\bibliography{pgfplots}
\end{document}
%
%%%%%%%%%%%%%%%%%%%%%%%%%%%%%%%%%%%%%%%%%%%%%%%%%%%%%%%%%%%%%%%%%%%%%%%%%%%%%
%
% Package pgfplots.sty documentation. 
%
% Copyright 2007/2008 by Christian Feuersaenger.
%
% This program is free software: you can redistribute it and/or modify
% it under the terms of the GNU General Public License as published by
% the Free Software Foundation, either version 3 of the License, or
% (at your option) any later version.
% 
% This program is distributed in the hope that it will be useful,
% but WITHOUT ANY WARRANTY; without even the implied warranty of
% MERCHANTABILITY or FITNESS FOR A PARTICULAR PURPOSE.  See the
% GNU General Public License for more details.
% 
% You should have received a copy of the GNU General Public License
% along with this program.  If not, see <http://www.gnu.org/licenses/>.
%
%
%%%%%%%%%%%%%%%%%%%%%%%%%%%%%%%%%%%%%%%%%%%%%%%%%%%%%%%%%%%%%%%%%%%%%%%%%%%%%
\input{pgfplots.preamble.tex}
%\RequirePackage[german,english,francais]{babel}

\pgfplotsmanualexternalexpensivetrue

%\usetikzlibrary{external}
%\tikzexternalize[prefix=figures/]{pgfplots}

\title{%
	Manual for Package \PGFPlots\\
	{\small Version \pgfplotsversion}\\
	{\small\href{http://sourceforge.net/projects/pgfplots}{http://sourceforge.net/projects/pgfplots}}}

%\includeonly{pgfplots.libs}


\begin{document}

\def\plotcoords{%
\addplot coordinates {
(5,8.312e-02)    (17,2.547e-02)   (49,7.407e-03)
(129,2.102e-03)  (321,5.874e-04)  (769,1.623e-04)
(1793,4.442e-05) (4097,1.207e-05) (9217,3.261e-06)
};

\addplot coordinates{
(7,8.472e-02)    (31,3.044e-02)    (111,1.022e-02)
(351,3.303e-03)  (1023,1.039e-03)  (2815,3.196e-04)
(7423,9.658e-05) (18943,2.873e-05) (47103,8.437e-06)
};

\addplot coordinates{
(9,7.881e-02)     (49,3.243e-02)    (209,1.232e-02)
(769,4.454e-03)   (2561,1.551e-03)  (7937,5.236e-04)
(23297,1.723e-04) (65537,5.545e-05) (178177,1.751e-05)
};

\addplot coordinates{
(11,6.887e-02)    (71,3.177e-02)     (351,1.341e-02)
(1471,5.334e-03)  (5503,2.027e-03)   (18943,7.415e-04)
(61183,2.628e-04) (187903,9.063e-05) (553983,3.053e-05)
};

\addplot coordinates{
(13,5.755e-02)     (97,2.925e-02)     (545,1.351e-02)
(2561,5.842e-03)   (10625,2.397e-03)  (40193,9.414e-04)
(141569,3.564e-04) (471041,1.308e-04) 
(1496065,4.670e-05)
};
}%
\maketitle
\begin{abstract}%
\PGFPlots\ draws high--quality function plots in normal or logarithmic scaling with a user-friendly interface directly in \TeX. The user supplies axis labels, legend entries and the plot coordinates for one or more plots and \PGFPlots\ applies axis scaling, computes any logarithms and axis ticks and draws the plots, supporting line plots, scatter plots, piecewise constant plots, bar plots, area plots, mesh-- and surface plots and some more. It is based on Till Tantau's package \PGF/\Tikz.
\end{abstract}
\tableofcontents
\section{Introduction}
This package provides tools to generate plots and labeled axes easily. It draws normal plots, logplots and semi-logplots, in two and three dimensions. Axis ticks, labels, legends (in case of multiple plots) can be added with key-value options. It can cycle through a set of predefined line/marker/color specifications. In summary, its purpose is to simplify the generation of high-quality function and/or data plots, and solving the problems of
\begin{itemize}
	\item consistency of document and font type and font size,
	\item direct use of \TeX\ math mode in axis descriptions,
	\item consistency of data and figures (no third party tool necessary),
	\item inter document consistency using preamble configurations and styles.
\end{itemize}
Although not necessary, separate |.pdf| or |.eps| graphics can be generated using the |external| library developed as part of \Tikz.

\PGFPlots\ is build completely on \Tikz/\PGF. Knowledge of \Tikz\ will simplify the work with \PGFPlots, although it is not required.

\section[About PGFPlots: Preliminaries]{About {\normalfont\PGFPlots}: Preliminaries}
This section contains information about upgrades, the team, the installation (in case you need to do it manually) and troubleshooting. You may skip it completely except for the upgrade remarks.

However, note that this library requires at least \PGF\ version $2.00$. At the time of this writing, many \TeX-distributions still contain the older \PGF\ version $1.18$, so it may be necessary to install a recent \PGF\ prior to using \PGFPlots.

\subsection{Components}
\PGFPlots\ comes with two components:
\begin{enumerate}
	\item the plotting component (which you are currently reading) and
	\item the \PGFPlotstable\ component which simplifies number formatting and postprocessing of numerical tables. It comes as a separate package and has its own manual \href{file:pgfplotstable.pdf}{pgfplotstable.pdf}.
\end{enumerate}

\subsection{Upgrade remarks}
This release provides a lot of improvements which can be found in all detail in \texttt{ChangeLog} for interested readers. However, some attention is useful with respect to the following changes.

\subsubsection{New Optional Features}
\begin{enumerate}
	\item \PGFPlots\ 1.3 comes with user interface improvements. The technical distinction between ``behavior options'' and ``style options'' of older versions is no longer necessary (although still fully supported).

	\item \PGFPlots\ 1.3 has a new feature which allows to \emph{move axis labels tight to tick labels} automatically.

	Since this affects the spacing, it is not enabled be default.

	Use
\begin{codeexample}[code only]
\usepackage{pgfplots}
\pgfplotsset{compat=1.3}
\end{codeexample}
	in your preamble to benefit from the improved spacing\footnote{The |compat=1.3| setting is necessary for new features which move axis labels around like reversed axis. However, \PGFPlots\ will still work as it does in earlier versions, your documents will remain the same if you don't set it explicitly.}.
	Take a look at the next page for the precise definition of |compat|.

	\item \PGFPlots\ 1.3 now supports reversed axes. It is no longer necessary to use work--arounds with negative units.
\pgfkeys{/pdflinks/search key prefixes in/.add={/pgfplots/,}{}}

	Take a look at the |x dir=reverse| key.

	Existing work--arounds will still function properly. Use |\pgfplotsset{compat=1.3}| together with |x dir=reverse| to switch to the new version.
\end{enumerate}

\subsubsection{Old Features Which May Need Attention}
\begin{enumerate}
	\item The |scatter/classes| feature produces proper legends as of version 1.3. This may change the appearance of existing legends of plots with |scatter/classes|.

	\item Due to a small math bug in \PGF\ $2.00$, you \emph{can't} use the math expression `|-x^2|'. It is necessary to use `|0-x^2|' instead. The same holds for
		`|exp(-x^2)|'; use `|exp(0-x^2)|' instead.

		This will be fixed with \PGF\ $>2.00$.

	\item Starting with \PGFPlots\ $1.1$, |\tikzstyle| should \emph{no longer be used} to set \PGFPlots\ options.
	
	Although |\tikzstyle| is still supported for some older \PGFPlots\ options, you should replace any occurance of |\tikzstyle| with |\pgfplotsset{|\meta{style name}|/.style={|\meta{key-value-list}|}}| or the associated |/.append style| variant. See section~\ref{sec:styles} for more detail.
\end{enumerate}
I apologize for any inconvenience caused by these changes.

\begin{pgfplotskey}{compat=\mchoice{1.3,pre 1.3,default,newest} (initially default)}
	The preamble configuration 
\begin{codeexample}[code only]
\usepackage{pgfplots}
\pgfplotsset{compat=1.3}
\end{codeexample}
	allows to choose between backwards compatibility and most recent features.

	\PGFPlots\ version 1.3 comes with new features which may lead to different spacings compared to earlier versions:
	\begin{enumerate}
		\item Version 1.3 comes with the |xlabel near ticks| feature which places (in this case $x$) axis labels ``near'' the tick labels, taking the size of tick labels into account.
		
		Older versions used |xlabel absolute|, i.e.\ an absolute distance between the axis and axis labels (now matter how large or small tick labels are).

		The initial setting \emph{keeps backwards compatibility}.

		You are encouraged to use
\begin{codeexample}[code only]
\usepackage{pgfplots}
\pgfplotsset{compat=1.3}
\end{codeexample}
		in your preamble.
		
		\item Version 1.3 fixes a bug with |anchor=south west| which occurred mainly for reversed axes: the anchors where upside--down. This is fixed now.

		
		If you have written some work--around and want to keep it going, use

		|\pgfplotsset{compat/anchors=pre 1.3}|\pgfmanualpdflabel{/pgfplots/compat/anchors}{} 

		to disable the bug fix.
	\end{enumerate}

	Use |\pgfplotsset{compat=default}| to restore the factory settings.

	The setting |\pgfplotsset{compat=newest}| will always use features of the most recent version. This might result in small changes in the document's appearance.
\end{pgfplotskey}

\subsection{The Team}
\PGFPlots\ has been written mainly by Christian Feuers�nger with many improvements of Pascal Wolkotte and Nick Papior Andersen as a spare time project. We hope it is useful and provides valuable plots.

If you are interested in writing something but don't know how, consider reading the auxiliary manual \href{file:TeX-programming-notes.pdf}{TeX-programming-notes.pdf} which comes with \PGFPlots. It is far from complete, but maybe it is a good starting point (at least for more literature).

\subsection{Acknowledgements}
I thank God for all hours of enjoyed programming. I thank Pascal Wolkotte and Nick Papior Andersen for their programming efforts and contributions as part of the development team. I thank Stefan Tibus, who contributed the |plot shell| feature. I thank Tom Cashman for the contribution of the |reverse legend| feature. Special thanks go to Stefan Pinnow whose tests of \PGFPlots\ lead to numerous quality improvements. Furthermore, I thank Dr.~Schweitzer for many fruitful discussions and Dr.~Meine for his ideas and suggestions. Thanks as well to the many international contributors who provided feature requests or identified bugs or simply manual improvements!

Last but not least, I thank Till Tantau and Mark Wibrow for their excellent graphics (and more) package \PGF\ and \Tikz\ which is the base of \PGFPlots.

\include{pgfplots.install}%
\include{pgfplots.intro}%
\include{pgfplots.reference}%
\include{pgfplots.libs}%
\include{pgfplots.resources}%
\include{pgfplots.importexport}%
\include{pgfplots.basic.reference}%

\printindex

\bibliographystyle{abbrv} %gerapali} %gerabbrv} %gerunsrt.bst} %gerabbrv}% gerplain}
\nocite{pgfplotstable}
\nocite{programmingnotes}
\bibliography{pgfplots}
\end{document}
%
%%%%%%%%%%%%%%%%%%%%%%%%%%%%%%%%%%%%%%%%%%%%%%%%%%%%%%%%%%%%%%%%%%%%%%%%%%%%%
%
% Package pgfplots.sty documentation. 
%
% Copyright 2007/2008 by Christian Feuersaenger.
%
% This program is free software: you can redistribute it and/or modify
% it under the terms of the GNU General Public License as published by
% the Free Software Foundation, either version 3 of the License, or
% (at your option) any later version.
% 
% This program is distributed in the hope that it will be useful,
% but WITHOUT ANY WARRANTY; without even the implied warranty of
% MERCHANTABILITY or FITNESS FOR A PARTICULAR PURPOSE.  See the
% GNU General Public License for more details.
% 
% You should have received a copy of the GNU General Public License
% along with this program.  If not, see <http://www.gnu.org/licenses/>.
%
%
%%%%%%%%%%%%%%%%%%%%%%%%%%%%%%%%%%%%%%%%%%%%%%%%%%%%%%%%%%%%%%%%%%%%%%%%%%%%%
\input{pgfplots.preamble.tex}
%\RequirePackage[german,english,francais]{babel}

\pgfplotsmanualexternalexpensivetrue

%\usetikzlibrary{external}
%\tikzexternalize[prefix=figures/]{pgfplots}

\title{%
	Manual for Package \PGFPlots\\
	{\small Version \pgfplotsversion}\\
	{\small\href{http://sourceforge.net/projects/pgfplots}{http://sourceforge.net/projects/pgfplots}}}

%\includeonly{pgfplots.libs}


\begin{document}

\def\plotcoords{%
\addplot coordinates {
(5,8.312e-02)    (17,2.547e-02)   (49,7.407e-03)
(129,2.102e-03)  (321,5.874e-04)  (769,1.623e-04)
(1793,4.442e-05) (4097,1.207e-05) (9217,3.261e-06)
};

\addplot coordinates{
(7,8.472e-02)    (31,3.044e-02)    (111,1.022e-02)
(351,3.303e-03)  (1023,1.039e-03)  (2815,3.196e-04)
(7423,9.658e-05) (18943,2.873e-05) (47103,8.437e-06)
};

\addplot coordinates{
(9,7.881e-02)     (49,3.243e-02)    (209,1.232e-02)
(769,4.454e-03)   (2561,1.551e-03)  (7937,5.236e-04)
(23297,1.723e-04) (65537,5.545e-05) (178177,1.751e-05)
};

\addplot coordinates{
(11,6.887e-02)    (71,3.177e-02)     (351,1.341e-02)
(1471,5.334e-03)  (5503,2.027e-03)   (18943,7.415e-04)
(61183,2.628e-04) (187903,9.063e-05) (553983,3.053e-05)
};

\addplot coordinates{
(13,5.755e-02)     (97,2.925e-02)     (545,1.351e-02)
(2561,5.842e-03)   (10625,2.397e-03)  (40193,9.414e-04)
(141569,3.564e-04) (471041,1.308e-04) 
(1496065,4.670e-05)
};
}%
\maketitle
\begin{abstract}%
\PGFPlots\ draws high--quality function plots in normal or logarithmic scaling with a user-friendly interface directly in \TeX. The user supplies axis labels, legend entries and the plot coordinates for one or more plots and \PGFPlots\ applies axis scaling, computes any logarithms and axis ticks and draws the plots, supporting line plots, scatter plots, piecewise constant plots, bar plots, area plots, mesh-- and surface plots and some more. It is based on Till Tantau's package \PGF/\Tikz.
\end{abstract}
\tableofcontents
\section{Introduction}
This package provides tools to generate plots and labeled axes easily. It draws normal plots, logplots and semi-logplots, in two and three dimensions. Axis ticks, labels, legends (in case of multiple plots) can be added with key-value options. It can cycle through a set of predefined line/marker/color specifications. In summary, its purpose is to simplify the generation of high-quality function and/or data plots, and solving the problems of
\begin{itemize}
	\item consistency of document and font type and font size,
	\item direct use of \TeX\ math mode in axis descriptions,
	\item consistency of data and figures (no third party tool necessary),
	\item inter document consistency using preamble configurations and styles.
\end{itemize}
Although not necessary, separate |.pdf| or |.eps| graphics can be generated using the |external| library developed as part of \Tikz.

\PGFPlots\ is build completely on \Tikz/\PGF. Knowledge of \Tikz\ will simplify the work with \PGFPlots, although it is not required.

\section[About PGFPlots: Preliminaries]{About {\normalfont\PGFPlots}: Preliminaries}
This section contains information about upgrades, the team, the installation (in case you need to do it manually) and troubleshooting. You may skip it completely except for the upgrade remarks.

However, note that this library requires at least \PGF\ version $2.00$. At the time of this writing, many \TeX-distributions still contain the older \PGF\ version $1.18$, so it may be necessary to install a recent \PGF\ prior to using \PGFPlots.

\subsection{Components}
\PGFPlots\ comes with two components:
\begin{enumerate}
	\item the plotting component (which you are currently reading) and
	\item the \PGFPlotstable\ component which simplifies number formatting and postprocessing of numerical tables. It comes as a separate package and has its own manual \href{file:pgfplotstable.pdf}{pgfplotstable.pdf}.
\end{enumerate}

\subsection{Upgrade remarks}
This release provides a lot of improvements which can be found in all detail in \texttt{ChangeLog} for interested readers. However, some attention is useful with respect to the following changes.

\subsubsection{New Optional Features}
\begin{enumerate}
	\item \PGFPlots\ 1.3 comes with user interface improvements. The technical distinction between ``behavior options'' and ``style options'' of older versions is no longer necessary (although still fully supported).

	\item \PGFPlots\ 1.3 has a new feature which allows to \emph{move axis labels tight to tick labels} automatically.

	Since this affects the spacing, it is not enabled be default.

	Use
\begin{codeexample}[code only]
\usepackage{pgfplots}
\pgfplotsset{compat=1.3}
\end{codeexample}
	in your preamble to benefit from the improved spacing\footnote{The |compat=1.3| setting is necessary for new features which move axis labels around like reversed axis. However, \PGFPlots\ will still work as it does in earlier versions, your documents will remain the same if you don't set it explicitly.}.
	Take a look at the next page for the precise definition of |compat|.

	\item \PGFPlots\ 1.3 now supports reversed axes. It is no longer necessary to use work--arounds with negative units.
\pgfkeys{/pdflinks/search key prefixes in/.add={/pgfplots/,}{}}

	Take a look at the |x dir=reverse| key.

	Existing work--arounds will still function properly. Use |\pgfplotsset{compat=1.3}| together with |x dir=reverse| to switch to the new version.
\end{enumerate}

\subsubsection{Old Features Which May Need Attention}
\begin{enumerate}
	\item The |scatter/classes| feature produces proper legends as of version 1.3. This may change the appearance of existing legends of plots with |scatter/classes|.

	\item Due to a small math bug in \PGF\ $2.00$, you \emph{can't} use the math expression `|-x^2|'. It is necessary to use `|0-x^2|' instead. The same holds for
		`|exp(-x^2)|'; use `|exp(0-x^2)|' instead.

		This will be fixed with \PGF\ $>2.00$.

	\item Starting with \PGFPlots\ $1.1$, |\tikzstyle| should \emph{no longer be used} to set \PGFPlots\ options.
	
	Although |\tikzstyle| is still supported for some older \PGFPlots\ options, you should replace any occurance of |\tikzstyle| with |\pgfplotsset{|\meta{style name}|/.style={|\meta{key-value-list}|}}| or the associated |/.append style| variant. See section~\ref{sec:styles} for more detail.
\end{enumerate}
I apologize for any inconvenience caused by these changes.

\begin{pgfplotskey}{compat=\mchoice{1.3,pre 1.3,default,newest} (initially default)}
	The preamble configuration 
\begin{codeexample}[code only]
\usepackage{pgfplots}
\pgfplotsset{compat=1.3}
\end{codeexample}
	allows to choose between backwards compatibility and most recent features.

	\PGFPlots\ version 1.3 comes with new features which may lead to different spacings compared to earlier versions:
	\begin{enumerate}
		\item Version 1.3 comes with the |xlabel near ticks| feature which places (in this case $x$) axis labels ``near'' the tick labels, taking the size of tick labels into account.
		
		Older versions used |xlabel absolute|, i.e.\ an absolute distance between the axis and axis labels (now matter how large or small tick labels are).

		The initial setting \emph{keeps backwards compatibility}.

		You are encouraged to use
\begin{codeexample}[code only]
\usepackage{pgfplots}
\pgfplotsset{compat=1.3}
\end{codeexample}
		in your preamble.
		
		\item Version 1.3 fixes a bug with |anchor=south west| which occurred mainly for reversed axes: the anchors where upside--down. This is fixed now.

		
		If you have written some work--around and want to keep it going, use

		|\pgfplotsset{compat/anchors=pre 1.3}|\pgfmanualpdflabel{/pgfplots/compat/anchors}{} 

		to disable the bug fix.
	\end{enumerate}

	Use |\pgfplotsset{compat=default}| to restore the factory settings.

	The setting |\pgfplotsset{compat=newest}| will always use features of the most recent version. This might result in small changes in the document's appearance.
\end{pgfplotskey}

\subsection{The Team}
\PGFPlots\ has been written mainly by Christian Feuers�nger with many improvements of Pascal Wolkotte and Nick Papior Andersen as a spare time project. We hope it is useful and provides valuable plots.

If you are interested in writing something but don't know how, consider reading the auxiliary manual \href{file:TeX-programming-notes.pdf}{TeX-programming-notes.pdf} which comes with \PGFPlots. It is far from complete, but maybe it is a good starting point (at least for more literature).

\subsection{Acknowledgements}
I thank God for all hours of enjoyed programming. I thank Pascal Wolkotte and Nick Papior Andersen for their programming efforts and contributions as part of the development team. I thank Stefan Tibus, who contributed the |plot shell| feature. I thank Tom Cashman for the contribution of the |reverse legend| feature. Special thanks go to Stefan Pinnow whose tests of \PGFPlots\ lead to numerous quality improvements. Furthermore, I thank Dr.~Schweitzer for many fruitful discussions and Dr.~Meine for his ideas and suggestions. Thanks as well to the many international contributors who provided feature requests or identified bugs or simply manual improvements!

Last but not least, I thank Till Tantau and Mark Wibrow for their excellent graphics (and more) package \PGF\ and \Tikz\ which is the base of \PGFPlots.

\include{pgfplots.install}%
\include{pgfplots.intro}%
\include{pgfplots.reference}%
\include{pgfplots.libs}%
\include{pgfplots.resources}%
\include{pgfplots.importexport}%
\include{pgfplots.basic.reference}%

\printindex

\bibliographystyle{abbrv} %gerapali} %gerabbrv} %gerunsrt.bst} %gerabbrv}% gerplain}
\nocite{pgfplotstable}
\nocite{programmingnotes}
\bibliography{pgfplots}
\end{document}
%
%%%%%%%%%%%%%%%%%%%%%%%%%%%%%%%%%%%%%%%%%%%%%%%%%%%%%%%%%%%%%%%%%%%%%%%%%%%%%
%
% Package pgfplots.sty documentation. 
%
% Copyright 2007/2008 by Christian Feuersaenger.
%
% This program is free software: you can redistribute it and/or modify
% it under the terms of the GNU General Public License as published by
% the Free Software Foundation, either version 3 of the License, or
% (at your option) any later version.
% 
% This program is distributed in the hope that it will be useful,
% but WITHOUT ANY WARRANTY; without even the implied warranty of
% MERCHANTABILITY or FITNESS FOR A PARTICULAR PURPOSE.  See the
% GNU General Public License for more details.
% 
% You should have received a copy of the GNU General Public License
% along with this program.  If not, see <http://www.gnu.org/licenses/>.
%
%
%%%%%%%%%%%%%%%%%%%%%%%%%%%%%%%%%%%%%%%%%%%%%%%%%%%%%%%%%%%%%%%%%%%%%%%%%%%%%
\input{pgfplots.preamble.tex}
%\RequirePackage[german,english,francais]{babel}

\pgfplotsmanualexternalexpensivetrue

%\usetikzlibrary{external}
%\tikzexternalize[prefix=figures/]{pgfplots}

\title{%
	Manual for Package \PGFPlots\\
	{\small Version \pgfplotsversion}\\
	{\small\href{http://sourceforge.net/projects/pgfplots}{http://sourceforge.net/projects/pgfplots}}}

%\includeonly{pgfplots.libs}


\begin{document}

\def\plotcoords{%
\addplot coordinates {
(5,8.312e-02)    (17,2.547e-02)   (49,7.407e-03)
(129,2.102e-03)  (321,5.874e-04)  (769,1.623e-04)
(1793,4.442e-05) (4097,1.207e-05) (9217,3.261e-06)
};

\addplot coordinates{
(7,8.472e-02)    (31,3.044e-02)    (111,1.022e-02)
(351,3.303e-03)  (1023,1.039e-03)  (2815,3.196e-04)
(7423,9.658e-05) (18943,2.873e-05) (47103,8.437e-06)
};

\addplot coordinates{
(9,7.881e-02)     (49,3.243e-02)    (209,1.232e-02)
(769,4.454e-03)   (2561,1.551e-03)  (7937,5.236e-04)
(23297,1.723e-04) (65537,5.545e-05) (178177,1.751e-05)
};

\addplot coordinates{
(11,6.887e-02)    (71,3.177e-02)     (351,1.341e-02)
(1471,5.334e-03)  (5503,2.027e-03)   (18943,7.415e-04)
(61183,2.628e-04) (187903,9.063e-05) (553983,3.053e-05)
};

\addplot coordinates{
(13,5.755e-02)     (97,2.925e-02)     (545,1.351e-02)
(2561,5.842e-03)   (10625,2.397e-03)  (40193,9.414e-04)
(141569,3.564e-04) (471041,1.308e-04) 
(1496065,4.670e-05)
};
}%
\maketitle
\begin{abstract}%
\PGFPlots\ draws high--quality function plots in normal or logarithmic scaling with a user-friendly interface directly in \TeX. The user supplies axis labels, legend entries and the plot coordinates for one or more plots and \PGFPlots\ applies axis scaling, computes any logarithms and axis ticks and draws the plots, supporting line plots, scatter plots, piecewise constant plots, bar plots, area plots, mesh-- and surface plots and some more. It is based on Till Tantau's package \PGF/\Tikz.
\end{abstract}
\tableofcontents
\section{Introduction}
This package provides tools to generate plots and labeled axes easily. It draws normal plots, logplots and semi-logplots, in two and three dimensions. Axis ticks, labels, legends (in case of multiple plots) can be added with key-value options. It can cycle through a set of predefined line/marker/color specifications. In summary, its purpose is to simplify the generation of high-quality function and/or data plots, and solving the problems of
\begin{itemize}
	\item consistency of document and font type and font size,
	\item direct use of \TeX\ math mode in axis descriptions,
	\item consistency of data and figures (no third party tool necessary),
	\item inter document consistency using preamble configurations and styles.
\end{itemize}
Although not necessary, separate |.pdf| or |.eps| graphics can be generated using the |external| library developed as part of \Tikz.

\PGFPlots\ is build completely on \Tikz/\PGF. Knowledge of \Tikz\ will simplify the work with \PGFPlots, although it is not required.

\section[About PGFPlots: Preliminaries]{About {\normalfont\PGFPlots}: Preliminaries}
This section contains information about upgrades, the team, the installation (in case you need to do it manually) and troubleshooting. You may skip it completely except for the upgrade remarks.

However, note that this library requires at least \PGF\ version $2.00$. At the time of this writing, many \TeX-distributions still contain the older \PGF\ version $1.18$, so it may be necessary to install a recent \PGF\ prior to using \PGFPlots.

\subsection{Components}
\PGFPlots\ comes with two components:
\begin{enumerate}
	\item the plotting component (which you are currently reading) and
	\item the \PGFPlotstable\ component which simplifies number formatting and postprocessing of numerical tables. It comes as a separate package and has its own manual \href{file:pgfplotstable.pdf}{pgfplotstable.pdf}.
\end{enumerate}

\subsection{Upgrade remarks}
This release provides a lot of improvements which can be found in all detail in \texttt{ChangeLog} for interested readers. However, some attention is useful with respect to the following changes.

\subsubsection{New Optional Features}
\begin{enumerate}
	\item \PGFPlots\ 1.3 comes with user interface improvements. The technical distinction between ``behavior options'' and ``style options'' of older versions is no longer necessary (although still fully supported).

	\item \PGFPlots\ 1.3 has a new feature which allows to \emph{move axis labels tight to tick labels} automatically.

	Since this affects the spacing, it is not enabled be default.

	Use
\begin{codeexample}[code only]
\usepackage{pgfplots}
\pgfplotsset{compat=1.3}
\end{codeexample}
	in your preamble to benefit from the improved spacing\footnote{The |compat=1.3| setting is necessary for new features which move axis labels around like reversed axis. However, \PGFPlots\ will still work as it does in earlier versions, your documents will remain the same if you don't set it explicitly.}.
	Take a look at the next page for the precise definition of |compat|.

	\item \PGFPlots\ 1.3 now supports reversed axes. It is no longer necessary to use work--arounds with negative units.
\pgfkeys{/pdflinks/search key prefixes in/.add={/pgfplots/,}{}}

	Take a look at the |x dir=reverse| key.

	Existing work--arounds will still function properly. Use |\pgfplotsset{compat=1.3}| together with |x dir=reverse| to switch to the new version.
\end{enumerate}

\subsubsection{Old Features Which May Need Attention}
\begin{enumerate}
	\item The |scatter/classes| feature produces proper legends as of version 1.3. This may change the appearance of existing legends of plots with |scatter/classes|.

	\item Due to a small math bug in \PGF\ $2.00$, you \emph{can't} use the math expression `|-x^2|'. It is necessary to use `|0-x^2|' instead. The same holds for
		`|exp(-x^2)|'; use `|exp(0-x^2)|' instead.

		This will be fixed with \PGF\ $>2.00$.

	\item Starting with \PGFPlots\ $1.1$, |\tikzstyle| should \emph{no longer be used} to set \PGFPlots\ options.
	
	Although |\tikzstyle| is still supported for some older \PGFPlots\ options, you should replace any occurance of |\tikzstyle| with |\pgfplotsset{|\meta{style name}|/.style={|\meta{key-value-list}|}}| or the associated |/.append style| variant. See section~\ref{sec:styles} for more detail.
\end{enumerate}
I apologize for any inconvenience caused by these changes.

\begin{pgfplotskey}{compat=\mchoice{1.3,pre 1.3,default,newest} (initially default)}
	The preamble configuration 
\begin{codeexample}[code only]
\usepackage{pgfplots}
\pgfplotsset{compat=1.3}
\end{codeexample}
	allows to choose between backwards compatibility and most recent features.

	\PGFPlots\ version 1.3 comes with new features which may lead to different spacings compared to earlier versions:
	\begin{enumerate}
		\item Version 1.3 comes with the |xlabel near ticks| feature which places (in this case $x$) axis labels ``near'' the tick labels, taking the size of tick labels into account.
		
		Older versions used |xlabel absolute|, i.e.\ an absolute distance between the axis and axis labels (now matter how large or small tick labels are).

		The initial setting \emph{keeps backwards compatibility}.

		You are encouraged to use
\begin{codeexample}[code only]
\usepackage{pgfplots}
\pgfplotsset{compat=1.3}
\end{codeexample}
		in your preamble.
		
		\item Version 1.3 fixes a bug with |anchor=south west| which occurred mainly for reversed axes: the anchors where upside--down. This is fixed now.

		
		If you have written some work--around and want to keep it going, use

		|\pgfplotsset{compat/anchors=pre 1.3}|\pgfmanualpdflabel{/pgfplots/compat/anchors}{} 

		to disable the bug fix.
	\end{enumerate}

	Use |\pgfplotsset{compat=default}| to restore the factory settings.

	The setting |\pgfplotsset{compat=newest}| will always use features of the most recent version. This might result in small changes in the document's appearance.
\end{pgfplotskey}

\subsection{The Team}
\PGFPlots\ has been written mainly by Christian Feuers�nger with many improvements of Pascal Wolkotte and Nick Papior Andersen as a spare time project. We hope it is useful and provides valuable plots.

If you are interested in writing something but don't know how, consider reading the auxiliary manual \href{file:TeX-programming-notes.pdf}{TeX-programming-notes.pdf} which comes with \PGFPlots. It is far from complete, but maybe it is a good starting point (at least for more literature).

\subsection{Acknowledgements}
I thank God for all hours of enjoyed programming. I thank Pascal Wolkotte and Nick Papior Andersen for their programming efforts and contributions as part of the development team. I thank Stefan Tibus, who contributed the |plot shell| feature. I thank Tom Cashman for the contribution of the |reverse legend| feature. Special thanks go to Stefan Pinnow whose tests of \PGFPlots\ lead to numerous quality improvements. Furthermore, I thank Dr.~Schweitzer for many fruitful discussions and Dr.~Meine for his ideas and suggestions. Thanks as well to the many international contributors who provided feature requests or identified bugs or simply manual improvements!

Last but not least, I thank Till Tantau and Mark Wibrow for their excellent graphics (and more) package \PGF\ and \Tikz\ which is the base of \PGFPlots.

\include{pgfplots.install}%
\include{pgfplots.intro}%
\include{pgfplots.reference}%
\include{pgfplots.libs}%
\include{pgfplots.resources}%
\include{pgfplots.importexport}%
\include{pgfplots.basic.reference}%

\printindex

\bibliographystyle{abbrv} %gerapali} %gerabbrv} %gerunsrt.bst} %gerabbrv}% gerplain}
\nocite{pgfplotstable}
\nocite{programmingnotes}
\bibliography{pgfplots}
\end{document}
%
%%%%%%%%%%%%%%%%%%%%%%%%%%%%%%%%%%%%%%%%%%%%%%%%%%%%%%%%%%%%%%%%%%%%%%%%%%%%%
%
% Package pgfplots.sty documentation. 
%
% Copyright 2007/2008 by Christian Feuersaenger.
%
% This program is free software: you can redistribute it and/or modify
% it under the terms of the GNU General Public License as published by
% the Free Software Foundation, either version 3 of the License, or
% (at your option) any later version.
% 
% This program is distributed in the hope that it will be useful,
% but WITHOUT ANY WARRANTY; without even the implied warranty of
% MERCHANTABILITY or FITNESS FOR A PARTICULAR PURPOSE.  See the
% GNU General Public License for more details.
% 
% You should have received a copy of the GNU General Public License
% along with this program.  If not, see <http://www.gnu.org/licenses/>.
%
%
%%%%%%%%%%%%%%%%%%%%%%%%%%%%%%%%%%%%%%%%%%%%%%%%%%%%%%%%%%%%%%%%%%%%%%%%%%%%%
\input{pgfplots.preamble.tex}
%\RequirePackage[german,english,francais]{babel}

\pgfplotsmanualexternalexpensivetrue

%\usetikzlibrary{external}
%\tikzexternalize[prefix=figures/]{pgfplots}

\title{%
	Manual for Package \PGFPlots\\
	{\small Version \pgfplotsversion}\\
	{\small\href{http://sourceforge.net/projects/pgfplots}{http://sourceforge.net/projects/pgfplots}}}

%\includeonly{pgfplots.libs}


\begin{document}

\def\plotcoords{%
\addplot coordinates {
(5,8.312e-02)    (17,2.547e-02)   (49,7.407e-03)
(129,2.102e-03)  (321,5.874e-04)  (769,1.623e-04)
(1793,4.442e-05) (4097,1.207e-05) (9217,3.261e-06)
};

\addplot coordinates{
(7,8.472e-02)    (31,3.044e-02)    (111,1.022e-02)
(351,3.303e-03)  (1023,1.039e-03)  (2815,3.196e-04)
(7423,9.658e-05) (18943,2.873e-05) (47103,8.437e-06)
};

\addplot coordinates{
(9,7.881e-02)     (49,3.243e-02)    (209,1.232e-02)
(769,4.454e-03)   (2561,1.551e-03)  (7937,5.236e-04)
(23297,1.723e-04) (65537,5.545e-05) (178177,1.751e-05)
};

\addplot coordinates{
(11,6.887e-02)    (71,3.177e-02)     (351,1.341e-02)
(1471,5.334e-03)  (5503,2.027e-03)   (18943,7.415e-04)
(61183,2.628e-04) (187903,9.063e-05) (553983,3.053e-05)
};

\addplot coordinates{
(13,5.755e-02)     (97,2.925e-02)     (545,1.351e-02)
(2561,5.842e-03)   (10625,2.397e-03)  (40193,9.414e-04)
(141569,3.564e-04) (471041,1.308e-04) 
(1496065,4.670e-05)
};
}%
\maketitle
\begin{abstract}%
\PGFPlots\ draws high--quality function plots in normal or logarithmic scaling with a user-friendly interface directly in \TeX. The user supplies axis labels, legend entries and the plot coordinates for one or more plots and \PGFPlots\ applies axis scaling, computes any logarithms and axis ticks and draws the plots, supporting line plots, scatter plots, piecewise constant plots, bar plots, area plots, mesh-- and surface plots and some more. It is based on Till Tantau's package \PGF/\Tikz.
\end{abstract}
\tableofcontents
\section{Introduction}
This package provides tools to generate plots and labeled axes easily. It draws normal plots, logplots and semi-logplots, in two and three dimensions. Axis ticks, labels, legends (in case of multiple plots) can be added with key-value options. It can cycle through a set of predefined line/marker/color specifications. In summary, its purpose is to simplify the generation of high-quality function and/or data plots, and solving the problems of
\begin{itemize}
	\item consistency of document and font type and font size,
	\item direct use of \TeX\ math mode in axis descriptions,
	\item consistency of data and figures (no third party tool necessary),
	\item inter document consistency using preamble configurations and styles.
\end{itemize}
Although not necessary, separate |.pdf| or |.eps| graphics can be generated using the |external| library developed as part of \Tikz.

\PGFPlots\ is build completely on \Tikz/\PGF. Knowledge of \Tikz\ will simplify the work with \PGFPlots, although it is not required.

\section[About PGFPlots: Preliminaries]{About {\normalfont\PGFPlots}: Preliminaries}
This section contains information about upgrades, the team, the installation (in case you need to do it manually) and troubleshooting. You may skip it completely except for the upgrade remarks.

However, note that this library requires at least \PGF\ version $2.00$. At the time of this writing, many \TeX-distributions still contain the older \PGF\ version $1.18$, so it may be necessary to install a recent \PGF\ prior to using \PGFPlots.

\subsection{Components}
\PGFPlots\ comes with two components:
\begin{enumerate}
	\item the plotting component (which you are currently reading) and
	\item the \PGFPlotstable\ component which simplifies number formatting and postprocessing of numerical tables. It comes as a separate package and has its own manual \href{file:pgfplotstable.pdf}{pgfplotstable.pdf}.
\end{enumerate}

\subsection{Upgrade remarks}
This release provides a lot of improvements which can be found in all detail in \texttt{ChangeLog} for interested readers. However, some attention is useful with respect to the following changes.

\subsubsection{New Optional Features}
\begin{enumerate}
	\item \PGFPlots\ 1.3 comes with user interface improvements. The technical distinction between ``behavior options'' and ``style options'' of older versions is no longer necessary (although still fully supported).

	\item \PGFPlots\ 1.3 has a new feature which allows to \emph{move axis labels tight to tick labels} automatically.

	Since this affects the spacing, it is not enabled be default.

	Use
\begin{codeexample}[code only]
\usepackage{pgfplots}
\pgfplotsset{compat=1.3}
\end{codeexample}
	in your preamble to benefit from the improved spacing\footnote{The |compat=1.3| setting is necessary for new features which move axis labels around like reversed axis. However, \PGFPlots\ will still work as it does in earlier versions, your documents will remain the same if you don't set it explicitly.}.
	Take a look at the next page for the precise definition of |compat|.

	\item \PGFPlots\ 1.3 now supports reversed axes. It is no longer necessary to use work--arounds with negative units.
\pgfkeys{/pdflinks/search key prefixes in/.add={/pgfplots/,}{}}

	Take a look at the |x dir=reverse| key.

	Existing work--arounds will still function properly. Use |\pgfplotsset{compat=1.3}| together with |x dir=reverse| to switch to the new version.
\end{enumerate}

\subsubsection{Old Features Which May Need Attention}
\begin{enumerate}
	\item The |scatter/classes| feature produces proper legends as of version 1.3. This may change the appearance of existing legends of plots with |scatter/classes|.

	\item Due to a small math bug in \PGF\ $2.00$, you \emph{can't} use the math expression `|-x^2|'. It is necessary to use `|0-x^2|' instead. The same holds for
		`|exp(-x^2)|'; use `|exp(0-x^2)|' instead.

		This will be fixed with \PGF\ $>2.00$.

	\item Starting with \PGFPlots\ $1.1$, |\tikzstyle| should \emph{no longer be used} to set \PGFPlots\ options.
	
	Although |\tikzstyle| is still supported for some older \PGFPlots\ options, you should replace any occurance of |\tikzstyle| with |\pgfplotsset{|\meta{style name}|/.style={|\meta{key-value-list}|}}| or the associated |/.append style| variant. See section~\ref{sec:styles} for more detail.
\end{enumerate}
I apologize for any inconvenience caused by these changes.

\begin{pgfplotskey}{compat=\mchoice{1.3,pre 1.3,default,newest} (initially default)}
	The preamble configuration 
\begin{codeexample}[code only]
\usepackage{pgfplots}
\pgfplotsset{compat=1.3}
\end{codeexample}
	allows to choose between backwards compatibility and most recent features.

	\PGFPlots\ version 1.3 comes with new features which may lead to different spacings compared to earlier versions:
	\begin{enumerate}
		\item Version 1.3 comes with the |xlabel near ticks| feature which places (in this case $x$) axis labels ``near'' the tick labels, taking the size of tick labels into account.
		
		Older versions used |xlabel absolute|, i.e.\ an absolute distance between the axis and axis labels (now matter how large or small tick labels are).

		The initial setting \emph{keeps backwards compatibility}.

		You are encouraged to use
\begin{codeexample}[code only]
\usepackage{pgfplots}
\pgfplotsset{compat=1.3}
\end{codeexample}
		in your preamble.
		
		\item Version 1.3 fixes a bug with |anchor=south west| which occurred mainly for reversed axes: the anchors where upside--down. This is fixed now.

		
		If you have written some work--around and want to keep it going, use

		|\pgfplotsset{compat/anchors=pre 1.3}|\pgfmanualpdflabel{/pgfplots/compat/anchors}{} 

		to disable the bug fix.
	\end{enumerate}

	Use |\pgfplotsset{compat=default}| to restore the factory settings.

	The setting |\pgfplotsset{compat=newest}| will always use features of the most recent version. This might result in small changes in the document's appearance.
\end{pgfplotskey}

\subsection{The Team}
\PGFPlots\ has been written mainly by Christian Feuers�nger with many improvements of Pascal Wolkotte and Nick Papior Andersen as a spare time project. We hope it is useful and provides valuable plots.

If you are interested in writing something but don't know how, consider reading the auxiliary manual \href{file:TeX-programming-notes.pdf}{TeX-programming-notes.pdf} which comes with \PGFPlots. It is far from complete, but maybe it is a good starting point (at least for more literature).

\subsection{Acknowledgements}
I thank God for all hours of enjoyed programming. I thank Pascal Wolkotte and Nick Papior Andersen for their programming efforts and contributions as part of the development team. I thank Stefan Tibus, who contributed the |plot shell| feature. I thank Tom Cashman for the contribution of the |reverse legend| feature. Special thanks go to Stefan Pinnow whose tests of \PGFPlots\ lead to numerous quality improvements. Furthermore, I thank Dr.~Schweitzer for many fruitful discussions and Dr.~Meine for his ideas and suggestions. Thanks as well to the many international contributors who provided feature requests or identified bugs or simply manual improvements!

Last but not least, I thank Till Tantau and Mark Wibrow for their excellent graphics (and more) package \PGF\ and \Tikz\ which is the base of \PGFPlots.

\include{pgfplots.install}%
\include{pgfplots.intro}%
\include{pgfplots.reference}%
\include{pgfplots.libs}%
\include{pgfplots.resources}%
\include{pgfplots.importexport}%
\include{pgfplots.basic.reference}%

\printindex

\bibliographystyle{abbrv} %gerapali} %gerabbrv} %gerunsrt.bst} %gerabbrv}% gerplain}
\nocite{pgfplotstable}
\nocite{programmingnotes}
\bibliography{pgfplots}
\end{document}
%
\include{pgfplots.basic.reference}%

\printindex

\bibliographystyle{abbrv} %gerapali} %gerabbrv} %gerunsrt.bst} %gerabbrv}% gerplain}
\nocite{pgfplotstable}
\nocite{programmingnotes}
\bibliography{pgfplots}
\end{document}
%
%%%%%%%%%%%%%%%%%%%%%%%%%%%%%%%%%%%%%%%%%%%%%%%%%%%%%%%%%%%%%%%%%%%%%%%%%%%%%
%
% Package pgfplots.sty documentation. 
%
% Copyright 2007/2008 by Christian Feuersaenger.
%
% This program is free software: you can redistribute it and/or modify
% it under the terms of the GNU General Public License as published by
% the Free Software Foundation, either version 3 of the License, or
% (at your option) any later version.
% 
% This program is distributed in the hope that it will be useful,
% but WITHOUT ANY WARRANTY; without even the implied warranty of
% MERCHANTABILITY or FITNESS FOR A PARTICULAR PURPOSE.  See the
% GNU General Public License for more details.
% 
% You should have received a copy of the GNU General Public License
% along with this program.  If not, see <http://www.gnu.org/licenses/>.
%
%
%%%%%%%%%%%%%%%%%%%%%%%%%%%%%%%%%%%%%%%%%%%%%%%%%%%%%%%%%%%%%%%%%%%%%%%%%%%%%
%%%%%%%%%%%%%%%%%%%%%%%%%%%%%%%%%%%%%%%%%%%%%%%%%%%%%%%%%%%%%%%%%%%%%%%%%%%%%
%
% Package pgfplots.sty documentation. 
%
% Copyright 2007/2008 by Christian Feuersaenger.
%
% This program is free software: you can redistribute it and/or modify
% it under the terms of the GNU General Public License as published by
% the Free Software Foundation, either version 3 of the License, or
% (at your option) any later version.
% 
% This program is distributed in the hope that it will be useful,
% but WITHOUT ANY WARRANTY; without even the implied warranty of
% MERCHANTABILITY or FITNESS FOR A PARTICULAR PURPOSE.  See the
% GNU General Public License for more details.
% 
% You should have received a copy of the GNU General Public License
% along with this program.  If not, see <http://www.gnu.org/licenses/>.
%
%
%%%%%%%%%%%%%%%%%%%%%%%%%%%%%%%%%%%%%%%%%%%%%%%%%%%%%%%%%%%%%%%%%%%%%%%%%%%%%
\documentclass[a4paper]{ltxdoc}

\usepackage{makeidx}

% DON't let hyperref overload the format of index and glossary. 
% I want to do that on my own in the stylefiles for makeindex...
\makeatletter
\let\@old@wrindex=\@wrindex
\makeatother

\usepackage{ifpdf}
\ifpdf
	\usepackage[pdftex]{hyperref}
\else
	\usepackage[dvipdfm]{hyperref}
\fi
\hypersetup{pdfborder={0 0 0}}

\makeatletter
\let\@wrindex=\@old@wrindex
\makeatother

% Formatiere Seitennummern im Index:
\newcommand{\indexpageno}[1]{%
	{\bfseries\hyperpage{#1}}%
}


\newcommand{\R}{\mathbb{R}}
\newcommand{\N}{\mathbb{N}}
\newcommand{\Z}{\mathbb{Z}}

\long\def\COMMENTLOWLEVEL#1\ENDCOMMENT{}
\def\ENDCOMMENT{}

\usepackage{textcomp}
\usepackage{booktabs}

\usepackage{calc}
\usepackage[formats]{listings}
%\usepackage{courier} % don't use it - the '^' character can't be copy-pasted in courier

\usepackage{array}
\lstset{%
	basicstyle=\ttfamily,
	language=[LaTeX]tex, % Seems as if \lstset{language=tex} must be invoked BEFORE loading tikz!?
	tabsize=4,
	breaklines=true,
	breakindent=0pt
}

\ifpdf
	\pdfinfo {
		/Author	(Christian Feuersaenger)
	}

\else
	\def\pgfsysdriver{pgfsys-dvipdfm.def}
\fi
%\def\pgfsysdriver{pgfsys-pdftex.def}
\usepackage{pgfplots}

\ifpdf
	\usetikzlibrary{pgfplotsclickable}
\fi

%\usepackage{fp}
% ATTENTION:
% this requires pgf version NEWER than 2.00 :
%\usetikzlibrary{fixedpointarithmetic}

\usetikzlibrary{dateplot}

\usepackage[a4paper,left=2.25cm,right=2.25cm,top=2.5cm,bottom=2.5cm,nohead]{geometry}
\usepackage{amsmath,amssymb}
\usepackage{xxcolor}
\usepackage{pifont}
\usepackage[latin1]{inputenc}
\usepackage{amsmath}
\usepackage{eurosym}
\input{pgfplots-macros}

\pgfqkeys{/codeexample}{%
	every codeexample/.style={
		width=8cm,
		/pgfplots/every axis/.append style={legend style={fill=graphicbackground}}
	},
	tabsize=4,
}
\pgfplotsset{
	every axis/.append style={width=7cm},
	filter discard warning=false,
}

\usetikzlibrary{backgrounds,patterns}
% Global styles:
\tikzset{
  shape example/.style={
    color=black!30,
    draw,
    fill=yellow!30,
    line width=.5cm,
    inner xsep=2.5cm,
    inner ysep=0.5cm}
}

\newcommand{\FIXME}[1]{\textcolor{red}{(FIXME: #1)}}

% fuer endvironment 'sidewaysfigure' bspw
% \usepackage{rotating}

\newcommand\Tikz{Ti\textit kZ}
\newcommand\PGF{\textsc{pgf}}
\newcommand\PGFPlots{\textsc{pgfplots}}
\def\PGFPlotstable{\textsc{PgfplotsTable}}


\makeindex

% Fix overful hboxes automatically:
\tolerance=2000
\emergencystretch=10pt

\tikzset{prefix=gnuplot/pgfplots_} % prefix for 'plot function'

\author{%
	Christian Feuers\"anger\footnote{\url{http://wissrech.ins.uni-bonn.de/people/feuersaenger}}\\%
	Institut f\"ur Numerische Simulation\\
	Universit\"at Bonn, Germany}


%\RequirePackage[german,english,francais]{babel}

\pgfplotsmanualexternalexpensivetrue

%\usetikzlibrary{external}
%\tikzexternalize[prefix=figures/]{pgfplots}

\title{%
	Manual for Package \PGFPlots\\
	{\small Version \pgfplotsversion}\\
	{\small\href{http://sourceforge.net/projects/pgfplots}{http://sourceforge.net/projects/pgfplots}}}

%\includeonly{pgfplots.libs}


\begin{document}

\def\plotcoords{%
\addplot coordinates {
(5,8.312e-02)    (17,2.547e-02)   (49,7.407e-03)
(129,2.102e-03)  (321,5.874e-04)  (769,1.623e-04)
(1793,4.442e-05) (4097,1.207e-05) (9217,3.261e-06)
};

\addplot coordinates{
(7,8.472e-02)    (31,3.044e-02)    (111,1.022e-02)
(351,3.303e-03)  (1023,1.039e-03)  (2815,3.196e-04)
(7423,9.658e-05) (18943,2.873e-05) (47103,8.437e-06)
};

\addplot coordinates{
(9,7.881e-02)     (49,3.243e-02)    (209,1.232e-02)
(769,4.454e-03)   (2561,1.551e-03)  (7937,5.236e-04)
(23297,1.723e-04) (65537,5.545e-05) (178177,1.751e-05)
};

\addplot coordinates{
(11,6.887e-02)    (71,3.177e-02)     (351,1.341e-02)
(1471,5.334e-03)  (5503,2.027e-03)   (18943,7.415e-04)
(61183,2.628e-04) (187903,9.063e-05) (553983,3.053e-05)
};

\addplot coordinates{
(13,5.755e-02)     (97,2.925e-02)     (545,1.351e-02)
(2561,5.842e-03)   (10625,2.397e-03)  (40193,9.414e-04)
(141569,3.564e-04) (471041,1.308e-04) 
(1496065,4.670e-05)
};
}%
\maketitle
\begin{abstract}%
\PGFPlots\ draws high--quality function plots in normal or logarithmic scaling with a user-friendly interface directly in \TeX. The user supplies axis labels, legend entries and the plot coordinates for one or more plots and \PGFPlots\ applies axis scaling, computes any logarithms and axis ticks and draws the plots, supporting line plots, scatter plots, piecewise constant plots, bar plots, area plots, mesh-- and surface plots and some more. It is based on Till Tantau's package \PGF/\Tikz.
\end{abstract}
\tableofcontents
\section{Introduction}
This package provides tools to generate plots and labeled axes easily. It draws normal plots, logplots and semi-logplots, in two and three dimensions. Axis ticks, labels, legends (in case of multiple plots) can be added with key-value options. It can cycle through a set of predefined line/marker/color specifications. In summary, its purpose is to simplify the generation of high-quality function and/or data plots, and solving the problems of
\begin{itemize}
	\item consistency of document and font type and font size,
	\item direct use of \TeX\ math mode in axis descriptions,
	\item consistency of data and figures (no third party tool necessary),
	\item inter document consistency using preamble configurations and styles.
\end{itemize}
Although not necessary, separate |.pdf| or |.eps| graphics can be generated using the |external| library developed as part of \Tikz.

\PGFPlots\ is build completely on \Tikz/\PGF. Knowledge of \Tikz\ will simplify the work with \PGFPlots, although it is not required.

\section[About PGFPlots: Preliminaries]{About {\normalfont\PGFPlots}: Preliminaries}
This section contains information about upgrades, the team, the installation (in case you need to do it manually) and troubleshooting. You may skip it completely except for the upgrade remarks.

However, note that this library requires at least \PGF\ version $2.00$. At the time of this writing, many \TeX-distributions still contain the older \PGF\ version $1.18$, so it may be necessary to install a recent \PGF\ prior to using \PGFPlots.

\subsection{Components}
\PGFPlots\ comes with two components:
\begin{enumerate}
	\item the plotting component (which you are currently reading) and
	\item the \PGFPlotstable\ component which simplifies number formatting and postprocessing of numerical tables. It comes as a separate package and has its own manual \href{file:pgfplotstable.pdf}{pgfplotstable.pdf}.
\end{enumerate}

\subsection{Upgrade remarks}
This release provides a lot of improvements which can be found in all detail in \texttt{ChangeLog} for interested readers. However, some attention is useful with respect to the following changes.

\subsubsection{New Optional Features}
\begin{enumerate}
	\item \PGFPlots\ 1.3 comes with user interface improvements. The technical distinction between ``behavior options'' and ``style options'' of older versions is no longer necessary (although still fully supported).

	\item \PGFPlots\ 1.3 has a new feature which allows to \emph{move axis labels tight to tick labels} automatically.

	Since this affects the spacing, it is not enabled be default.

	Use
\begin{codeexample}[code only]
\usepackage{pgfplots}
\pgfplotsset{compat=1.3}
\end{codeexample}
	in your preamble to benefit from the improved spacing\footnote{The |compat=1.3| setting is necessary for new features which move axis labels around like reversed axis. However, \PGFPlots\ will still work as it does in earlier versions, your documents will remain the same if you don't set it explicitly.}.
	Take a look at the next page for the precise definition of |compat|.

	\item \PGFPlots\ 1.3 now supports reversed axes. It is no longer necessary to use work--arounds with negative units.
\pgfkeys{/pdflinks/search key prefixes in/.add={/pgfplots/,}{}}

	Take a look at the |x dir=reverse| key.

	Existing work--arounds will still function properly. Use |\pgfplotsset{compat=1.3}| together with |x dir=reverse| to switch to the new version.
\end{enumerate}

\subsubsection{Old Features Which May Need Attention}
\begin{enumerate}
	\item The |scatter/classes| feature produces proper legends as of version 1.3. This may change the appearance of existing legends of plots with |scatter/classes|.

	\item Due to a small math bug in \PGF\ $2.00$, you \emph{can't} use the math expression `|-x^2|'. It is necessary to use `|0-x^2|' instead. The same holds for
		`|exp(-x^2)|'; use `|exp(0-x^2)|' instead.

		This will be fixed with \PGF\ $>2.00$.

	\item Starting with \PGFPlots\ $1.1$, |\tikzstyle| should \emph{no longer be used} to set \PGFPlots\ options.
	
	Although |\tikzstyle| is still supported for some older \PGFPlots\ options, you should replace any occurance of |\tikzstyle| with |\pgfplotsset{|\meta{style name}|/.style={|\meta{key-value-list}|}}| or the associated |/.append style| variant. See section~\ref{sec:styles} for more detail.
\end{enumerate}
I apologize for any inconvenience caused by these changes.

\begin{pgfplotskey}{compat=\mchoice{1.3,pre 1.3,default,newest} (initially default)}
	The preamble configuration 
\begin{codeexample}[code only]
\usepackage{pgfplots}
\pgfplotsset{compat=1.3}
\end{codeexample}
	allows to choose between backwards compatibility and most recent features.

	\PGFPlots\ version 1.3 comes with new features which may lead to different spacings compared to earlier versions:
	\begin{enumerate}
		\item Version 1.3 comes with the |xlabel near ticks| feature which places (in this case $x$) axis labels ``near'' the tick labels, taking the size of tick labels into account.
		
		Older versions used |xlabel absolute|, i.e.\ an absolute distance between the axis and axis labels (now matter how large or small tick labels are).

		The initial setting \emph{keeps backwards compatibility}.

		You are encouraged to use
\begin{codeexample}[code only]
\usepackage{pgfplots}
\pgfplotsset{compat=1.3}
\end{codeexample}
		in your preamble.
		
		\item Version 1.3 fixes a bug with |anchor=south west| which occurred mainly for reversed axes: the anchors where upside--down. This is fixed now.

		
		If you have written some work--around and want to keep it going, use

		|\pgfplotsset{compat/anchors=pre 1.3}|\pgfmanualpdflabel{/pgfplots/compat/anchors}{} 

		to disable the bug fix.
	\end{enumerate}

	Use |\pgfplotsset{compat=default}| to restore the factory settings.

	The setting |\pgfplotsset{compat=newest}| will always use features of the most recent version. This might result in small changes in the document's appearance.
\end{pgfplotskey}

\subsection{The Team}
\PGFPlots\ has been written mainly by Christian Feuers�nger with many improvements of Pascal Wolkotte and Nick Papior Andersen as a spare time project. We hope it is useful and provides valuable plots.

If you are interested in writing something but don't know how, consider reading the auxiliary manual \href{file:TeX-programming-notes.pdf}{TeX-programming-notes.pdf} which comes with \PGFPlots. It is far from complete, but maybe it is a good starting point (at least for more literature).

\subsection{Acknowledgements}
I thank God for all hours of enjoyed programming. I thank Pascal Wolkotte and Nick Papior Andersen for their programming efforts and contributions as part of the development team. I thank Stefan Tibus, who contributed the |plot shell| feature. I thank Tom Cashman for the contribution of the |reverse legend| feature. Special thanks go to Stefan Pinnow whose tests of \PGFPlots\ lead to numerous quality improvements. Furthermore, I thank Dr.~Schweitzer for many fruitful discussions and Dr.~Meine for his ideas and suggestions. Thanks as well to the many international contributors who provided feature requests or identified bugs or simply manual improvements!

Last but not least, I thank Till Tantau and Mark Wibrow for their excellent graphics (and more) package \PGF\ and \Tikz\ which is the base of \PGFPlots.

%%%%%%%%%%%%%%%%%%%%%%%%%%%%%%%%%%%%%%%%%%%%%%%%%%%%%%%%%%%%%%%%%%%%%%%%%%%%%
%
% Package pgfplots.sty documentation. 
%
% Copyright 2007/2008 by Christian Feuersaenger.
%
% This program is free software: you can redistribute it and/or modify
% it under the terms of the GNU General Public License as published by
% the Free Software Foundation, either version 3 of the License, or
% (at your option) any later version.
% 
% This program is distributed in the hope that it will be useful,
% but WITHOUT ANY WARRANTY; without even the implied warranty of
% MERCHANTABILITY or FITNESS FOR A PARTICULAR PURPOSE.  See the
% GNU General Public License for more details.
% 
% You should have received a copy of the GNU General Public License
% along with this program.  If not, see <http://www.gnu.org/licenses/>.
%
%
%%%%%%%%%%%%%%%%%%%%%%%%%%%%%%%%%%%%%%%%%%%%%%%%%%%%%%%%%%%%%%%%%%%%%%%%%%%%%
\input{pgfplots.preamble.tex}
%\RequirePackage[german,english,francais]{babel}

\pgfplotsmanualexternalexpensivetrue

%\usetikzlibrary{external}
%\tikzexternalize[prefix=figures/]{pgfplots}

\title{%
	Manual for Package \PGFPlots\\
	{\small Version \pgfplotsversion}\\
	{\small\href{http://sourceforge.net/projects/pgfplots}{http://sourceforge.net/projects/pgfplots}}}

%\includeonly{pgfplots.libs}


\begin{document}

\def\plotcoords{%
\addplot coordinates {
(5,8.312e-02)    (17,2.547e-02)   (49,7.407e-03)
(129,2.102e-03)  (321,5.874e-04)  (769,1.623e-04)
(1793,4.442e-05) (4097,1.207e-05) (9217,3.261e-06)
};

\addplot coordinates{
(7,8.472e-02)    (31,3.044e-02)    (111,1.022e-02)
(351,3.303e-03)  (1023,1.039e-03)  (2815,3.196e-04)
(7423,9.658e-05) (18943,2.873e-05) (47103,8.437e-06)
};

\addplot coordinates{
(9,7.881e-02)     (49,3.243e-02)    (209,1.232e-02)
(769,4.454e-03)   (2561,1.551e-03)  (7937,5.236e-04)
(23297,1.723e-04) (65537,5.545e-05) (178177,1.751e-05)
};

\addplot coordinates{
(11,6.887e-02)    (71,3.177e-02)     (351,1.341e-02)
(1471,5.334e-03)  (5503,2.027e-03)   (18943,7.415e-04)
(61183,2.628e-04) (187903,9.063e-05) (553983,3.053e-05)
};

\addplot coordinates{
(13,5.755e-02)     (97,2.925e-02)     (545,1.351e-02)
(2561,5.842e-03)   (10625,2.397e-03)  (40193,9.414e-04)
(141569,3.564e-04) (471041,1.308e-04) 
(1496065,4.670e-05)
};
}%
\maketitle
\begin{abstract}%
\PGFPlots\ draws high--quality function plots in normal or logarithmic scaling with a user-friendly interface directly in \TeX. The user supplies axis labels, legend entries and the plot coordinates for one or more plots and \PGFPlots\ applies axis scaling, computes any logarithms and axis ticks and draws the plots, supporting line plots, scatter plots, piecewise constant plots, bar plots, area plots, mesh-- and surface plots and some more. It is based on Till Tantau's package \PGF/\Tikz.
\end{abstract}
\tableofcontents
\section{Introduction}
This package provides tools to generate plots and labeled axes easily. It draws normal plots, logplots and semi-logplots, in two and three dimensions. Axis ticks, labels, legends (in case of multiple plots) can be added with key-value options. It can cycle through a set of predefined line/marker/color specifications. In summary, its purpose is to simplify the generation of high-quality function and/or data plots, and solving the problems of
\begin{itemize}
	\item consistency of document and font type and font size,
	\item direct use of \TeX\ math mode in axis descriptions,
	\item consistency of data and figures (no third party tool necessary),
	\item inter document consistency using preamble configurations and styles.
\end{itemize}
Although not necessary, separate |.pdf| or |.eps| graphics can be generated using the |external| library developed as part of \Tikz.

\PGFPlots\ is build completely on \Tikz/\PGF. Knowledge of \Tikz\ will simplify the work with \PGFPlots, although it is not required.

\section[About PGFPlots: Preliminaries]{About {\normalfont\PGFPlots}: Preliminaries}
This section contains information about upgrades, the team, the installation (in case you need to do it manually) and troubleshooting. You may skip it completely except for the upgrade remarks.

However, note that this library requires at least \PGF\ version $2.00$. At the time of this writing, many \TeX-distributions still contain the older \PGF\ version $1.18$, so it may be necessary to install a recent \PGF\ prior to using \PGFPlots.

\subsection{Components}
\PGFPlots\ comes with two components:
\begin{enumerate}
	\item the plotting component (which you are currently reading) and
	\item the \PGFPlotstable\ component which simplifies number formatting and postprocessing of numerical tables. It comes as a separate package and has its own manual \href{file:pgfplotstable.pdf}{pgfplotstable.pdf}.
\end{enumerate}

\subsection{Upgrade remarks}
This release provides a lot of improvements which can be found in all detail in \texttt{ChangeLog} for interested readers. However, some attention is useful with respect to the following changes.

\subsubsection{New Optional Features}
\begin{enumerate}
	\item \PGFPlots\ 1.3 comes with user interface improvements. The technical distinction between ``behavior options'' and ``style options'' of older versions is no longer necessary (although still fully supported).

	\item \PGFPlots\ 1.3 has a new feature which allows to \emph{move axis labels tight to tick labels} automatically.

	Since this affects the spacing, it is not enabled be default.

	Use
\begin{codeexample}[code only]
\usepackage{pgfplots}
\pgfplotsset{compat=1.3}
\end{codeexample}
	in your preamble to benefit from the improved spacing\footnote{The |compat=1.3| setting is necessary for new features which move axis labels around like reversed axis. However, \PGFPlots\ will still work as it does in earlier versions, your documents will remain the same if you don't set it explicitly.}.
	Take a look at the next page for the precise definition of |compat|.

	\item \PGFPlots\ 1.3 now supports reversed axes. It is no longer necessary to use work--arounds with negative units.
\pgfkeys{/pdflinks/search key prefixes in/.add={/pgfplots/,}{}}

	Take a look at the |x dir=reverse| key.

	Existing work--arounds will still function properly. Use |\pgfplotsset{compat=1.3}| together with |x dir=reverse| to switch to the new version.
\end{enumerate}

\subsubsection{Old Features Which May Need Attention}
\begin{enumerate}
	\item The |scatter/classes| feature produces proper legends as of version 1.3. This may change the appearance of existing legends of plots with |scatter/classes|.

	\item Due to a small math bug in \PGF\ $2.00$, you \emph{can't} use the math expression `|-x^2|'. It is necessary to use `|0-x^2|' instead. The same holds for
		`|exp(-x^2)|'; use `|exp(0-x^2)|' instead.

		This will be fixed with \PGF\ $>2.00$.

	\item Starting with \PGFPlots\ $1.1$, |\tikzstyle| should \emph{no longer be used} to set \PGFPlots\ options.
	
	Although |\tikzstyle| is still supported for some older \PGFPlots\ options, you should replace any occurance of |\tikzstyle| with |\pgfplotsset{|\meta{style name}|/.style={|\meta{key-value-list}|}}| or the associated |/.append style| variant. See section~\ref{sec:styles} for more detail.
\end{enumerate}
I apologize for any inconvenience caused by these changes.

\begin{pgfplotskey}{compat=\mchoice{1.3,pre 1.3,default,newest} (initially default)}
	The preamble configuration 
\begin{codeexample}[code only]
\usepackage{pgfplots}
\pgfplotsset{compat=1.3}
\end{codeexample}
	allows to choose between backwards compatibility and most recent features.

	\PGFPlots\ version 1.3 comes with new features which may lead to different spacings compared to earlier versions:
	\begin{enumerate}
		\item Version 1.3 comes with the |xlabel near ticks| feature which places (in this case $x$) axis labels ``near'' the tick labels, taking the size of tick labels into account.
		
		Older versions used |xlabel absolute|, i.e.\ an absolute distance between the axis and axis labels (now matter how large or small tick labels are).

		The initial setting \emph{keeps backwards compatibility}.

		You are encouraged to use
\begin{codeexample}[code only]
\usepackage{pgfplots}
\pgfplotsset{compat=1.3}
\end{codeexample}
		in your preamble.
		
		\item Version 1.3 fixes a bug with |anchor=south west| which occurred mainly for reversed axes: the anchors where upside--down. This is fixed now.

		
		If you have written some work--around and want to keep it going, use

		|\pgfplotsset{compat/anchors=pre 1.3}|\pgfmanualpdflabel{/pgfplots/compat/anchors}{} 

		to disable the bug fix.
	\end{enumerate}

	Use |\pgfplotsset{compat=default}| to restore the factory settings.

	The setting |\pgfplotsset{compat=newest}| will always use features of the most recent version. This might result in small changes in the document's appearance.
\end{pgfplotskey}

\subsection{The Team}
\PGFPlots\ has been written mainly by Christian Feuers�nger with many improvements of Pascal Wolkotte and Nick Papior Andersen as a spare time project. We hope it is useful and provides valuable plots.

If you are interested in writing something but don't know how, consider reading the auxiliary manual \href{file:TeX-programming-notes.pdf}{TeX-programming-notes.pdf} which comes with \PGFPlots. It is far from complete, but maybe it is a good starting point (at least for more literature).

\subsection{Acknowledgements}
I thank God for all hours of enjoyed programming. I thank Pascal Wolkotte and Nick Papior Andersen for their programming efforts and contributions as part of the development team. I thank Stefan Tibus, who contributed the |plot shell| feature. I thank Tom Cashman for the contribution of the |reverse legend| feature. Special thanks go to Stefan Pinnow whose tests of \PGFPlots\ lead to numerous quality improvements. Furthermore, I thank Dr.~Schweitzer for many fruitful discussions and Dr.~Meine for his ideas and suggestions. Thanks as well to the many international contributors who provided feature requests or identified bugs or simply manual improvements!

Last but not least, I thank Till Tantau and Mark Wibrow for their excellent graphics (and more) package \PGF\ and \Tikz\ which is the base of \PGFPlots.

\include{pgfplots.install}%
\include{pgfplots.intro}%
\include{pgfplots.reference}%
\include{pgfplots.libs}%
\include{pgfplots.resources}%
\include{pgfplots.importexport}%
\include{pgfplots.basic.reference}%

\printindex

\bibliographystyle{abbrv} %gerapali} %gerabbrv} %gerunsrt.bst} %gerabbrv}% gerplain}
\nocite{pgfplotstable}
\nocite{programmingnotes}
\bibliography{pgfplots}
\end{document}
%
%%%%%%%%%%%%%%%%%%%%%%%%%%%%%%%%%%%%%%%%%%%%%%%%%%%%%%%%%%%%%%%%%%%%%%%%%%%%%
%
% Package pgfplots.sty documentation. 
%
% Copyright 2007/2008 by Christian Feuersaenger.
%
% This program is free software: you can redistribute it and/or modify
% it under the terms of the GNU General Public License as published by
% the Free Software Foundation, either version 3 of the License, or
% (at your option) any later version.
% 
% This program is distributed in the hope that it will be useful,
% but WITHOUT ANY WARRANTY; without even the implied warranty of
% MERCHANTABILITY or FITNESS FOR A PARTICULAR PURPOSE.  See the
% GNU General Public License for more details.
% 
% You should have received a copy of the GNU General Public License
% along with this program.  If not, see <http://www.gnu.org/licenses/>.
%
%
%%%%%%%%%%%%%%%%%%%%%%%%%%%%%%%%%%%%%%%%%%%%%%%%%%%%%%%%%%%%%%%%%%%%%%%%%%%%%
\input{pgfplots.preamble.tex}
%\RequirePackage[german,english,francais]{babel}

\pgfplotsmanualexternalexpensivetrue

%\usetikzlibrary{external}
%\tikzexternalize[prefix=figures/]{pgfplots}

\title{%
	Manual for Package \PGFPlots\\
	{\small Version \pgfplotsversion}\\
	{\small\href{http://sourceforge.net/projects/pgfplots}{http://sourceforge.net/projects/pgfplots}}}

%\includeonly{pgfplots.libs}


\begin{document}

\def\plotcoords{%
\addplot coordinates {
(5,8.312e-02)    (17,2.547e-02)   (49,7.407e-03)
(129,2.102e-03)  (321,5.874e-04)  (769,1.623e-04)
(1793,4.442e-05) (4097,1.207e-05) (9217,3.261e-06)
};

\addplot coordinates{
(7,8.472e-02)    (31,3.044e-02)    (111,1.022e-02)
(351,3.303e-03)  (1023,1.039e-03)  (2815,3.196e-04)
(7423,9.658e-05) (18943,2.873e-05) (47103,8.437e-06)
};

\addplot coordinates{
(9,7.881e-02)     (49,3.243e-02)    (209,1.232e-02)
(769,4.454e-03)   (2561,1.551e-03)  (7937,5.236e-04)
(23297,1.723e-04) (65537,5.545e-05) (178177,1.751e-05)
};

\addplot coordinates{
(11,6.887e-02)    (71,3.177e-02)     (351,1.341e-02)
(1471,5.334e-03)  (5503,2.027e-03)   (18943,7.415e-04)
(61183,2.628e-04) (187903,9.063e-05) (553983,3.053e-05)
};

\addplot coordinates{
(13,5.755e-02)     (97,2.925e-02)     (545,1.351e-02)
(2561,5.842e-03)   (10625,2.397e-03)  (40193,9.414e-04)
(141569,3.564e-04) (471041,1.308e-04) 
(1496065,4.670e-05)
};
}%
\maketitle
\begin{abstract}%
\PGFPlots\ draws high--quality function plots in normal or logarithmic scaling with a user-friendly interface directly in \TeX. The user supplies axis labels, legend entries and the plot coordinates for one or more plots and \PGFPlots\ applies axis scaling, computes any logarithms and axis ticks and draws the plots, supporting line plots, scatter plots, piecewise constant plots, bar plots, area plots, mesh-- and surface plots and some more. It is based on Till Tantau's package \PGF/\Tikz.
\end{abstract}
\tableofcontents
\section{Introduction}
This package provides tools to generate plots and labeled axes easily. It draws normal plots, logplots and semi-logplots, in two and three dimensions. Axis ticks, labels, legends (in case of multiple plots) can be added with key-value options. It can cycle through a set of predefined line/marker/color specifications. In summary, its purpose is to simplify the generation of high-quality function and/or data plots, and solving the problems of
\begin{itemize}
	\item consistency of document and font type and font size,
	\item direct use of \TeX\ math mode in axis descriptions,
	\item consistency of data and figures (no third party tool necessary),
	\item inter document consistency using preamble configurations and styles.
\end{itemize}
Although not necessary, separate |.pdf| or |.eps| graphics can be generated using the |external| library developed as part of \Tikz.

\PGFPlots\ is build completely on \Tikz/\PGF. Knowledge of \Tikz\ will simplify the work with \PGFPlots, although it is not required.

\section[About PGFPlots: Preliminaries]{About {\normalfont\PGFPlots}: Preliminaries}
This section contains information about upgrades, the team, the installation (in case you need to do it manually) and troubleshooting. You may skip it completely except for the upgrade remarks.

However, note that this library requires at least \PGF\ version $2.00$. At the time of this writing, many \TeX-distributions still contain the older \PGF\ version $1.18$, so it may be necessary to install a recent \PGF\ prior to using \PGFPlots.

\subsection{Components}
\PGFPlots\ comes with two components:
\begin{enumerate}
	\item the plotting component (which you are currently reading) and
	\item the \PGFPlotstable\ component which simplifies number formatting and postprocessing of numerical tables. It comes as a separate package and has its own manual \href{file:pgfplotstable.pdf}{pgfplotstable.pdf}.
\end{enumerate}

\subsection{Upgrade remarks}
This release provides a lot of improvements which can be found in all detail in \texttt{ChangeLog} for interested readers. However, some attention is useful with respect to the following changes.

\subsubsection{New Optional Features}
\begin{enumerate}
	\item \PGFPlots\ 1.3 comes with user interface improvements. The technical distinction between ``behavior options'' and ``style options'' of older versions is no longer necessary (although still fully supported).

	\item \PGFPlots\ 1.3 has a new feature which allows to \emph{move axis labels tight to tick labels} automatically.

	Since this affects the spacing, it is not enabled be default.

	Use
\begin{codeexample}[code only]
\usepackage{pgfplots}
\pgfplotsset{compat=1.3}
\end{codeexample}
	in your preamble to benefit from the improved spacing\footnote{The |compat=1.3| setting is necessary for new features which move axis labels around like reversed axis. However, \PGFPlots\ will still work as it does in earlier versions, your documents will remain the same if you don't set it explicitly.}.
	Take a look at the next page for the precise definition of |compat|.

	\item \PGFPlots\ 1.3 now supports reversed axes. It is no longer necessary to use work--arounds with negative units.
\pgfkeys{/pdflinks/search key prefixes in/.add={/pgfplots/,}{}}

	Take a look at the |x dir=reverse| key.

	Existing work--arounds will still function properly. Use |\pgfplotsset{compat=1.3}| together with |x dir=reverse| to switch to the new version.
\end{enumerate}

\subsubsection{Old Features Which May Need Attention}
\begin{enumerate}
	\item The |scatter/classes| feature produces proper legends as of version 1.3. This may change the appearance of existing legends of plots with |scatter/classes|.

	\item Due to a small math bug in \PGF\ $2.00$, you \emph{can't} use the math expression `|-x^2|'. It is necessary to use `|0-x^2|' instead. The same holds for
		`|exp(-x^2)|'; use `|exp(0-x^2)|' instead.

		This will be fixed with \PGF\ $>2.00$.

	\item Starting with \PGFPlots\ $1.1$, |\tikzstyle| should \emph{no longer be used} to set \PGFPlots\ options.
	
	Although |\tikzstyle| is still supported for some older \PGFPlots\ options, you should replace any occurance of |\tikzstyle| with |\pgfplotsset{|\meta{style name}|/.style={|\meta{key-value-list}|}}| or the associated |/.append style| variant. See section~\ref{sec:styles} for more detail.
\end{enumerate}
I apologize for any inconvenience caused by these changes.

\begin{pgfplotskey}{compat=\mchoice{1.3,pre 1.3,default,newest} (initially default)}
	The preamble configuration 
\begin{codeexample}[code only]
\usepackage{pgfplots}
\pgfplotsset{compat=1.3}
\end{codeexample}
	allows to choose between backwards compatibility and most recent features.

	\PGFPlots\ version 1.3 comes with new features which may lead to different spacings compared to earlier versions:
	\begin{enumerate}
		\item Version 1.3 comes with the |xlabel near ticks| feature which places (in this case $x$) axis labels ``near'' the tick labels, taking the size of tick labels into account.
		
		Older versions used |xlabel absolute|, i.e.\ an absolute distance between the axis and axis labels (now matter how large or small tick labels are).

		The initial setting \emph{keeps backwards compatibility}.

		You are encouraged to use
\begin{codeexample}[code only]
\usepackage{pgfplots}
\pgfplotsset{compat=1.3}
\end{codeexample}
		in your preamble.
		
		\item Version 1.3 fixes a bug with |anchor=south west| which occurred mainly for reversed axes: the anchors where upside--down. This is fixed now.

		
		If you have written some work--around and want to keep it going, use

		|\pgfplotsset{compat/anchors=pre 1.3}|\pgfmanualpdflabel{/pgfplots/compat/anchors}{} 

		to disable the bug fix.
	\end{enumerate}

	Use |\pgfplotsset{compat=default}| to restore the factory settings.

	The setting |\pgfplotsset{compat=newest}| will always use features of the most recent version. This might result in small changes in the document's appearance.
\end{pgfplotskey}

\subsection{The Team}
\PGFPlots\ has been written mainly by Christian Feuers�nger with many improvements of Pascal Wolkotte and Nick Papior Andersen as a spare time project. We hope it is useful and provides valuable plots.

If you are interested in writing something but don't know how, consider reading the auxiliary manual \href{file:TeX-programming-notes.pdf}{TeX-programming-notes.pdf} which comes with \PGFPlots. It is far from complete, but maybe it is a good starting point (at least for more literature).

\subsection{Acknowledgements}
I thank God for all hours of enjoyed programming. I thank Pascal Wolkotte and Nick Papior Andersen for their programming efforts and contributions as part of the development team. I thank Stefan Tibus, who contributed the |plot shell| feature. I thank Tom Cashman for the contribution of the |reverse legend| feature. Special thanks go to Stefan Pinnow whose tests of \PGFPlots\ lead to numerous quality improvements. Furthermore, I thank Dr.~Schweitzer for many fruitful discussions and Dr.~Meine for his ideas and suggestions. Thanks as well to the many international contributors who provided feature requests or identified bugs or simply manual improvements!

Last but not least, I thank Till Tantau and Mark Wibrow for their excellent graphics (and more) package \PGF\ and \Tikz\ which is the base of \PGFPlots.

\include{pgfplots.install}%
\include{pgfplots.intro}%
\include{pgfplots.reference}%
\include{pgfplots.libs}%
\include{pgfplots.resources}%
\include{pgfplots.importexport}%
\include{pgfplots.basic.reference}%

\printindex

\bibliographystyle{abbrv} %gerapali} %gerabbrv} %gerunsrt.bst} %gerabbrv}% gerplain}
\nocite{pgfplotstable}
\nocite{programmingnotes}
\bibliography{pgfplots}
\end{document}
%
%%%%%%%%%%%%%%%%%%%%%%%%%%%%%%%%%%%%%%%%%%%%%%%%%%%%%%%%%%%%%%%%%%%%%%%%%%%%%
%
% Package pgfplots.sty documentation. 
%
% Copyright 2007/2008 by Christian Feuersaenger.
%
% This program is free software: you can redistribute it and/or modify
% it under the terms of the GNU General Public License as published by
% the Free Software Foundation, either version 3 of the License, or
% (at your option) any later version.
% 
% This program is distributed in the hope that it will be useful,
% but WITHOUT ANY WARRANTY; without even the implied warranty of
% MERCHANTABILITY or FITNESS FOR A PARTICULAR PURPOSE.  See the
% GNU General Public License for more details.
% 
% You should have received a copy of the GNU General Public License
% along with this program.  If not, see <http://www.gnu.org/licenses/>.
%
%
%%%%%%%%%%%%%%%%%%%%%%%%%%%%%%%%%%%%%%%%%%%%%%%%%%%%%%%%%%%%%%%%%%%%%%%%%%%%%
\input{pgfplots.preamble.tex}
%\RequirePackage[german,english,francais]{babel}

\pgfplotsmanualexternalexpensivetrue

%\usetikzlibrary{external}
%\tikzexternalize[prefix=figures/]{pgfplots}

\title{%
	Manual for Package \PGFPlots\\
	{\small Version \pgfplotsversion}\\
	{\small\href{http://sourceforge.net/projects/pgfplots}{http://sourceforge.net/projects/pgfplots}}}

%\includeonly{pgfplots.libs}


\begin{document}

\def\plotcoords{%
\addplot coordinates {
(5,8.312e-02)    (17,2.547e-02)   (49,7.407e-03)
(129,2.102e-03)  (321,5.874e-04)  (769,1.623e-04)
(1793,4.442e-05) (4097,1.207e-05) (9217,3.261e-06)
};

\addplot coordinates{
(7,8.472e-02)    (31,3.044e-02)    (111,1.022e-02)
(351,3.303e-03)  (1023,1.039e-03)  (2815,3.196e-04)
(7423,9.658e-05) (18943,2.873e-05) (47103,8.437e-06)
};

\addplot coordinates{
(9,7.881e-02)     (49,3.243e-02)    (209,1.232e-02)
(769,4.454e-03)   (2561,1.551e-03)  (7937,5.236e-04)
(23297,1.723e-04) (65537,5.545e-05) (178177,1.751e-05)
};

\addplot coordinates{
(11,6.887e-02)    (71,3.177e-02)     (351,1.341e-02)
(1471,5.334e-03)  (5503,2.027e-03)   (18943,7.415e-04)
(61183,2.628e-04) (187903,9.063e-05) (553983,3.053e-05)
};

\addplot coordinates{
(13,5.755e-02)     (97,2.925e-02)     (545,1.351e-02)
(2561,5.842e-03)   (10625,2.397e-03)  (40193,9.414e-04)
(141569,3.564e-04) (471041,1.308e-04) 
(1496065,4.670e-05)
};
}%
\maketitle
\begin{abstract}%
\PGFPlots\ draws high--quality function plots in normal or logarithmic scaling with a user-friendly interface directly in \TeX. The user supplies axis labels, legend entries and the plot coordinates for one or more plots and \PGFPlots\ applies axis scaling, computes any logarithms and axis ticks and draws the plots, supporting line plots, scatter plots, piecewise constant plots, bar plots, area plots, mesh-- and surface plots and some more. It is based on Till Tantau's package \PGF/\Tikz.
\end{abstract}
\tableofcontents
\section{Introduction}
This package provides tools to generate plots and labeled axes easily. It draws normal plots, logplots and semi-logplots, in two and three dimensions. Axis ticks, labels, legends (in case of multiple plots) can be added with key-value options. It can cycle through a set of predefined line/marker/color specifications. In summary, its purpose is to simplify the generation of high-quality function and/or data plots, and solving the problems of
\begin{itemize}
	\item consistency of document and font type and font size,
	\item direct use of \TeX\ math mode in axis descriptions,
	\item consistency of data and figures (no third party tool necessary),
	\item inter document consistency using preamble configurations and styles.
\end{itemize}
Although not necessary, separate |.pdf| or |.eps| graphics can be generated using the |external| library developed as part of \Tikz.

\PGFPlots\ is build completely on \Tikz/\PGF. Knowledge of \Tikz\ will simplify the work with \PGFPlots, although it is not required.

\section[About PGFPlots: Preliminaries]{About {\normalfont\PGFPlots}: Preliminaries}
This section contains information about upgrades, the team, the installation (in case you need to do it manually) and troubleshooting. You may skip it completely except for the upgrade remarks.

However, note that this library requires at least \PGF\ version $2.00$. At the time of this writing, many \TeX-distributions still contain the older \PGF\ version $1.18$, so it may be necessary to install a recent \PGF\ prior to using \PGFPlots.

\subsection{Components}
\PGFPlots\ comes with two components:
\begin{enumerate}
	\item the plotting component (which you are currently reading) and
	\item the \PGFPlotstable\ component which simplifies number formatting and postprocessing of numerical tables. It comes as a separate package and has its own manual \href{file:pgfplotstable.pdf}{pgfplotstable.pdf}.
\end{enumerate}

\subsection{Upgrade remarks}
This release provides a lot of improvements which can be found in all detail in \texttt{ChangeLog} for interested readers. However, some attention is useful with respect to the following changes.

\subsubsection{New Optional Features}
\begin{enumerate}
	\item \PGFPlots\ 1.3 comes with user interface improvements. The technical distinction between ``behavior options'' and ``style options'' of older versions is no longer necessary (although still fully supported).

	\item \PGFPlots\ 1.3 has a new feature which allows to \emph{move axis labels tight to tick labels} automatically.

	Since this affects the spacing, it is not enabled be default.

	Use
\begin{codeexample}[code only]
\usepackage{pgfplots}
\pgfplotsset{compat=1.3}
\end{codeexample}
	in your preamble to benefit from the improved spacing\footnote{The |compat=1.3| setting is necessary for new features which move axis labels around like reversed axis. However, \PGFPlots\ will still work as it does in earlier versions, your documents will remain the same if you don't set it explicitly.}.
	Take a look at the next page for the precise definition of |compat|.

	\item \PGFPlots\ 1.3 now supports reversed axes. It is no longer necessary to use work--arounds with negative units.
\pgfkeys{/pdflinks/search key prefixes in/.add={/pgfplots/,}{}}

	Take a look at the |x dir=reverse| key.

	Existing work--arounds will still function properly. Use |\pgfplotsset{compat=1.3}| together with |x dir=reverse| to switch to the new version.
\end{enumerate}

\subsubsection{Old Features Which May Need Attention}
\begin{enumerate}
	\item The |scatter/classes| feature produces proper legends as of version 1.3. This may change the appearance of existing legends of plots with |scatter/classes|.

	\item Due to a small math bug in \PGF\ $2.00$, you \emph{can't} use the math expression `|-x^2|'. It is necessary to use `|0-x^2|' instead. The same holds for
		`|exp(-x^2)|'; use `|exp(0-x^2)|' instead.

		This will be fixed with \PGF\ $>2.00$.

	\item Starting with \PGFPlots\ $1.1$, |\tikzstyle| should \emph{no longer be used} to set \PGFPlots\ options.
	
	Although |\tikzstyle| is still supported for some older \PGFPlots\ options, you should replace any occurance of |\tikzstyle| with |\pgfplotsset{|\meta{style name}|/.style={|\meta{key-value-list}|}}| or the associated |/.append style| variant. See section~\ref{sec:styles} for more detail.
\end{enumerate}
I apologize for any inconvenience caused by these changes.

\begin{pgfplotskey}{compat=\mchoice{1.3,pre 1.3,default,newest} (initially default)}
	The preamble configuration 
\begin{codeexample}[code only]
\usepackage{pgfplots}
\pgfplotsset{compat=1.3}
\end{codeexample}
	allows to choose between backwards compatibility and most recent features.

	\PGFPlots\ version 1.3 comes with new features which may lead to different spacings compared to earlier versions:
	\begin{enumerate}
		\item Version 1.3 comes with the |xlabel near ticks| feature which places (in this case $x$) axis labels ``near'' the tick labels, taking the size of tick labels into account.
		
		Older versions used |xlabel absolute|, i.e.\ an absolute distance between the axis and axis labels (now matter how large or small tick labels are).

		The initial setting \emph{keeps backwards compatibility}.

		You are encouraged to use
\begin{codeexample}[code only]
\usepackage{pgfplots}
\pgfplotsset{compat=1.3}
\end{codeexample}
		in your preamble.
		
		\item Version 1.3 fixes a bug with |anchor=south west| which occurred mainly for reversed axes: the anchors where upside--down. This is fixed now.

		
		If you have written some work--around and want to keep it going, use

		|\pgfplotsset{compat/anchors=pre 1.3}|\pgfmanualpdflabel{/pgfplots/compat/anchors}{} 

		to disable the bug fix.
	\end{enumerate}

	Use |\pgfplotsset{compat=default}| to restore the factory settings.

	The setting |\pgfplotsset{compat=newest}| will always use features of the most recent version. This might result in small changes in the document's appearance.
\end{pgfplotskey}

\subsection{The Team}
\PGFPlots\ has been written mainly by Christian Feuers�nger with many improvements of Pascal Wolkotte and Nick Papior Andersen as a spare time project. We hope it is useful and provides valuable plots.

If you are interested in writing something but don't know how, consider reading the auxiliary manual \href{file:TeX-programming-notes.pdf}{TeX-programming-notes.pdf} which comes with \PGFPlots. It is far from complete, but maybe it is a good starting point (at least for more literature).

\subsection{Acknowledgements}
I thank God for all hours of enjoyed programming. I thank Pascal Wolkotte and Nick Papior Andersen for their programming efforts and contributions as part of the development team. I thank Stefan Tibus, who contributed the |plot shell| feature. I thank Tom Cashman for the contribution of the |reverse legend| feature. Special thanks go to Stefan Pinnow whose tests of \PGFPlots\ lead to numerous quality improvements. Furthermore, I thank Dr.~Schweitzer for many fruitful discussions and Dr.~Meine for his ideas and suggestions. Thanks as well to the many international contributors who provided feature requests or identified bugs or simply manual improvements!

Last but not least, I thank Till Tantau and Mark Wibrow for their excellent graphics (and more) package \PGF\ and \Tikz\ which is the base of \PGFPlots.

\include{pgfplots.install}%
\include{pgfplots.intro}%
\include{pgfplots.reference}%
\include{pgfplots.libs}%
\include{pgfplots.resources}%
\include{pgfplots.importexport}%
\include{pgfplots.basic.reference}%

\printindex

\bibliographystyle{abbrv} %gerapali} %gerabbrv} %gerunsrt.bst} %gerabbrv}% gerplain}
\nocite{pgfplotstable}
\nocite{programmingnotes}
\bibliography{pgfplots}
\end{document}
%
%%%%%%%%%%%%%%%%%%%%%%%%%%%%%%%%%%%%%%%%%%%%%%%%%%%%%%%%%%%%%%%%%%%%%%%%%%%%%
%
% Package pgfplots.sty documentation. 
%
% Copyright 2007/2008 by Christian Feuersaenger.
%
% This program is free software: you can redistribute it and/or modify
% it under the terms of the GNU General Public License as published by
% the Free Software Foundation, either version 3 of the License, or
% (at your option) any later version.
% 
% This program is distributed in the hope that it will be useful,
% but WITHOUT ANY WARRANTY; without even the implied warranty of
% MERCHANTABILITY or FITNESS FOR A PARTICULAR PURPOSE.  See the
% GNU General Public License for more details.
% 
% You should have received a copy of the GNU General Public License
% along with this program.  If not, see <http://www.gnu.org/licenses/>.
%
%
%%%%%%%%%%%%%%%%%%%%%%%%%%%%%%%%%%%%%%%%%%%%%%%%%%%%%%%%%%%%%%%%%%%%%%%%%%%%%
\input{pgfplots.preamble.tex}
%\RequirePackage[german,english,francais]{babel}

\pgfplotsmanualexternalexpensivetrue

%\usetikzlibrary{external}
%\tikzexternalize[prefix=figures/]{pgfplots}

\title{%
	Manual for Package \PGFPlots\\
	{\small Version \pgfplotsversion}\\
	{\small\href{http://sourceforge.net/projects/pgfplots}{http://sourceforge.net/projects/pgfplots}}}

%\includeonly{pgfplots.libs}


\begin{document}

\def\plotcoords{%
\addplot coordinates {
(5,8.312e-02)    (17,2.547e-02)   (49,7.407e-03)
(129,2.102e-03)  (321,5.874e-04)  (769,1.623e-04)
(1793,4.442e-05) (4097,1.207e-05) (9217,3.261e-06)
};

\addplot coordinates{
(7,8.472e-02)    (31,3.044e-02)    (111,1.022e-02)
(351,3.303e-03)  (1023,1.039e-03)  (2815,3.196e-04)
(7423,9.658e-05) (18943,2.873e-05) (47103,8.437e-06)
};

\addplot coordinates{
(9,7.881e-02)     (49,3.243e-02)    (209,1.232e-02)
(769,4.454e-03)   (2561,1.551e-03)  (7937,5.236e-04)
(23297,1.723e-04) (65537,5.545e-05) (178177,1.751e-05)
};

\addplot coordinates{
(11,6.887e-02)    (71,3.177e-02)     (351,1.341e-02)
(1471,5.334e-03)  (5503,2.027e-03)   (18943,7.415e-04)
(61183,2.628e-04) (187903,9.063e-05) (553983,3.053e-05)
};

\addplot coordinates{
(13,5.755e-02)     (97,2.925e-02)     (545,1.351e-02)
(2561,5.842e-03)   (10625,2.397e-03)  (40193,9.414e-04)
(141569,3.564e-04) (471041,1.308e-04) 
(1496065,4.670e-05)
};
}%
\maketitle
\begin{abstract}%
\PGFPlots\ draws high--quality function plots in normal or logarithmic scaling with a user-friendly interface directly in \TeX. The user supplies axis labels, legend entries and the plot coordinates for one or more plots and \PGFPlots\ applies axis scaling, computes any logarithms and axis ticks and draws the plots, supporting line plots, scatter plots, piecewise constant plots, bar plots, area plots, mesh-- and surface plots and some more. It is based on Till Tantau's package \PGF/\Tikz.
\end{abstract}
\tableofcontents
\section{Introduction}
This package provides tools to generate plots and labeled axes easily. It draws normal plots, logplots and semi-logplots, in two and three dimensions. Axis ticks, labels, legends (in case of multiple plots) can be added with key-value options. It can cycle through a set of predefined line/marker/color specifications. In summary, its purpose is to simplify the generation of high-quality function and/or data plots, and solving the problems of
\begin{itemize}
	\item consistency of document and font type and font size,
	\item direct use of \TeX\ math mode in axis descriptions,
	\item consistency of data and figures (no third party tool necessary),
	\item inter document consistency using preamble configurations and styles.
\end{itemize}
Although not necessary, separate |.pdf| or |.eps| graphics can be generated using the |external| library developed as part of \Tikz.

\PGFPlots\ is build completely on \Tikz/\PGF. Knowledge of \Tikz\ will simplify the work with \PGFPlots, although it is not required.

\section[About PGFPlots: Preliminaries]{About {\normalfont\PGFPlots}: Preliminaries}
This section contains information about upgrades, the team, the installation (in case you need to do it manually) and troubleshooting. You may skip it completely except for the upgrade remarks.

However, note that this library requires at least \PGF\ version $2.00$. At the time of this writing, many \TeX-distributions still contain the older \PGF\ version $1.18$, so it may be necessary to install a recent \PGF\ prior to using \PGFPlots.

\subsection{Components}
\PGFPlots\ comes with two components:
\begin{enumerate}
	\item the plotting component (which you are currently reading) and
	\item the \PGFPlotstable\ component which simplifies number formatting and postprocessing of numerical tables. It comes as a separate package and has its own manual \href{file:pgfplotstable.pdf}{pgfplotstable.pdf}.
\end{enumerate}

\subsection{Upgrade remarks}
This release provides a lot of improvements which can be found in all detail in \texttt{ChangeLog} for interested readers. However, some attention is useful with respect to the following changes.

\subsubsection{New Optional Features}
\begin{enumerate}
	\item \PGFPlots\ 1.3 comes with user interface improvements. The technical distinction between ``behavior options'' and ``style options'' of older versions is no longer necessary (although still fully supported).

	\item \PGFPlots\ 1.3 has a new feature which allows to \emph{move axis labels tight to tick labels} automatically.

	Since this affects the spacing, it is not enabled be default.

	Use
\begin{codeexample}[code only]
\usepackage{pgfplots}
\pgfplotsset{compat=1.3}
\end{codeexample}
	in your preamble to benefit from the improved spacing\footnote{The |compat=1.3| setting is necessary for new features which move axis labels around like reversed axis. However, \PGFPlots\ will still work as it does in earlier versions, your documents will remain the same if you don't set it explicitly.}.
	Take a look at the next page for the precise definition of |compat|.

	\item \PGFPlots\ 1.3 now supports reversed axes. It is no longer necessary to use work--arounds with negative units.
\pgfkeys{/pdflinks/search key prefixes in/.add={/pgfplots/,}{}}

	Take a look at the |x dir=reverse| key.

	Existing work--arounds will still function properly. Use |\pgfplotsset{compat=1.3}| together with |x dir=reverse| to switch to the new version.
\end{enumerate}

\subsubsection{Old Features Which May Need Attention}
\begin{enumerate}
	\item The |scatter/classes| feature produces proper legends as of version 1.3. This may change the appearance of existing legends of plots with |scatter/classes|.

	\item Due to a small math bug in \PGF\ $2.00$, you \emph{can't} use the math expression `|-x^2|'. It is necessary to use `|0-x^2|' instead. The same holds for
		`|exp(-x^2)|'; use `|exp(0-x^2)|' instead.

		This will be fixed with \PGF\ $>2.00$.

	\item Starting with \PGFPlots\ $1.1$, |\tikzstyle| should \emph{no longer be used} to set \PGFPlots\ options.
	
	Although |\tikzstyle| is still supported for some older \PGFPlots\ options, you should replace any occurance of |\tikzstyle| with |\pgfplotsset{|\meta{style name}|/.style={|\meta{key-value-list}|}}| or the associated |/.append style| variant. See section~\ref{sec:styles} for more detail.
\end{enumerate}
I apologize for any inconvenience caused by these changes.

\begin{pgfplotskey}{compat=\mchoice{1.3,pre 1.3,default,newest} (initially default)}
	The preamble configuration 
\begin{codeexample}[code only]
\usepackage{pgfplots}
\pgfplotsset{compat=1.3}
\end{codeexample}
	allows to choose between backwards compatibility and most recent features.

	\PGFPlots\ version 1.3 comes with new features which may lead to different spacings compared to earlier versions:
	\begin{enumerate}
		\item Version 1.3 comes with the |xlabel near ticks| feature which places (in this case $x$) axis labels ``near'' the tick labels, taking the size of tick labels into account.
		
		Older versions used |xlabel absolute|, i.e.\ an absolute distance between the axis and axis labels (now matter how large or small tick labels are).

		The initial setting \emph{keeps backwards compatibility}.

		You are encouraged to use
\begin{codeexample}[code only]
\usepackage{pgfplots}
\pgfplotsset{compat=1.3}
\end{codeexample}
		in your preamble.
		
		\item Version 1.3 fixes a bug with |anchor=south west| which occurred mainly for reversed axes: the anchors where upside--down. This is fixed now.

		
		If you have written some work--around and want to keep it going, use

		|\pgfplotsset{compat/anchors=pre 1.3}|\pgfmanualpdflabel{/pgfplots/compat/anchors}{} 

		to disable the bug fix.
	\end{enumerate}

	Use |\pgfplotsset{compat=default}| to restore the factory settings.

	The setting |\pgfplotsset{compat=newest}| will always use features of the most recent version. This might result in small changes in the document's appearance.
\end{pgfplotskey}

\subsection{The Team}
\PGFPlots\ has been written mainly by Christian Feuers�nger with many improvements of Pascal Wolkotte and Nick Papior Andersen as a spare time project. We hope it is useful and provides valuable plots.

If you are interested in writing something but don't know how, consider reading the auxiliary manual \href{file:TeX-programming-notes.pdf}{TeX-programming-notes.pdf} which comes with \PGFPlots. It is far from complete, but maybe it is a good starting point (at least for more literature).

\subsection{Acknowledgements}
I thank God for all hours of enjoyed programming. I thank Pascal Wolkotte and Nick Papior Andersen for their programming efforts and contributions as part of the development team. I thank Stefan Tibus, who contributed the |plot shell| feature. I thank Tom Cashman for the contribution of the |reverse legend| feature. Special thanks go to Stefan Pinnow whose tests of \PGFPlots\ lead to numerous quality improvements. Furthermore, I thank Dr.~Schweitzer for many fruitful discussions and Dr.~Meine for his ideas and suggestions. Thanks as well to the many international contributors who provided feature requests or identified bugs or simply manual improvements!

Last but not least, I thank Till Tantau and Mark Wibrow for their excellent graphics (and more) package \PGF\ and \Tikz\ which is the base of \PGFPlots.

\include{pgfplots.install}%
\include{pgfplots.intro}%
\include{pgfplots.reference}%
\include{pgfplots.libs}%
\include{pgfplots.resources}%
\include{pgfplots.importexport}%
\include{pgfplots.basic.reference}%

\printindex

\bibliographystyle{abbrv} %gerapali} %gerabbrv} %gerunsrt.bst} %gerabbrv}% gerplain}
\nocite{pgfplotstable}
\nocite{programmingnotes}
\bibliography{pgfplots}
\end{document}
%
%%%%%%%%%%%%%%%%%%%%%%%%%%%%%%%%%%%%%%%%%%%%%%%%%%%%%%%%%%%%%%%%%%%%%%%%%%%%%
%
% Package pgfplots.sty documentation. 
%
% Copyright 2007/2008 by Christian Feuersaenger.
%
% This program is free software: you can redistribute it and/or modify
% it under the terms of the GNU General Public License as published by
% the Free Software Foundation, either version 3 of the License, or
% (at your option) any later version.
% 
% This program is distributed in the hope that it will be useful,
% but WITHOUT ANY WARRANTY; without even the implied warranty of
% MERCHANTABILITY or FITNESS FOR A PARTICULAR PURPOSE.  See the
% GNU General Public License for more details.
% 
% You should have received a copy of the GNU General Public License
% along with this program.  If not, see <http://www.gnu.org/licenses/>.
%
%
%%%%%%%%%%%%%%%%%%%%%%%%%%%%%%%%%%%%%%%%%%%%%%%%%%%%%%%%%%%%%%%%%%%%%%%%%%%%%
\input{pgfplots.preamble.tex}
%\RequirePackage[german,english,francais]{babel}

\pgfplotsmanualexternalexpensivetrue

%\usetikzlibrary{external}
%\tikzexternalize[prefix=figures/]{pgfplots}

\title{%
	Manual for Package \PGFPlots\\
	{\small Version \pgfplotsversion}\\
	{\small\href{http://sourceforge.net/projects/pgfplots}{http://sourceforge.net/projects/pgfplots}}}

%\includeonly{pgfplots.libs}


\begin{document}

\def\plotcoords{%
\addplot coordinates {
(5,8.312e-02)    (17,2.547e-02)   (49,7.407e-03)
(129,2.102e-03)  (321,5.874e-04)  (769,1.623e-04)
(1793,4.442e-05) (4097,1.207e-05) (9217,3.261e-06)
};

\addplot coordinates{
(7,8.472e-02)    (31,3.044e-02)    (111,1.022e-02)
(351,3.303e-03)  (1023,1.039e-03)  (2815,3.196e-04)
(7423,9.658e-05) (18943,2.873e-05) (47103,8.437e-06)
};

\addplot coordinates{
(9,7.881e-02)     (49,3.243e-02)    (209,1.232e-02)
(769,4.454e-03)   (2561,1.551e-03)  (7937,5.236e-04)
(23297,1.723e-04) (65537,5.545e-05) (178177,1.751e-05)
};

\addplot coordinates{
(11,6.887e-02)    (71,3.177e-02)     (351,1.341e-02)
(1471,5.334e-03)  (5503,2.027e-03)   (18943,7.415e-04)
(61183,2.628e-04) (187903,9.063e-05) (553983,3.053e-05)
};

\addplot coordinates{
(13,5.755e-02)     (97,2.925e-02)     (545,1.351e-02)
(2561,5.842e-03)   (10625,2.397e-03)  (40193,9.414e-04)
(141569,3.564e-04) (471041,1.308e-04) 
(1496065,4.670e-05)
};
}%
\maketitle
\begin{abstract}%
\PGFPlots\ draws high--quality function plots in normal or logarithmic scaling with a user-friendly interface directly in \TeX. The user supplies axis labels, legend entries and the plot coordinates for one or more plots and \PGFPlots\ applies axis scaling, computes any logarithms and axis ticks and draws the plots, supporting line plots, scatter plots, piecewise constant plots, bar plots, area plots, mesh-- and surface plots and some more. It is based on Till Tantau's package \PGF/\Tikz.
\end{abstract}
\tableofcontents
\section{Introduction}
This package provides tools to generate plots and labeled axes easily. It draws normal plots, logplots and semi-logplots, in two and three dimensions. Axis ticks, labels, legends (in case of multiple plots) can be added with key-value options. It can cycle through a set of predefined line/marker/color specifications. In summary, its purpose is to simplify the generation of high-quality function and/or data plots, and solving the problems of
\begin{itemize}
	\item consistency of document and font type and font size,
	\item direct use of \TeX\ math mode in axis descriptions,
	\item consistency of data and figures (no third party tool necessary),
	\item inter document consistency using preamble configurations and styles.
\end{itemize}
Although not necessary, separate |.pdf| or |.eps| graphics can be generated using the |external| library developed as part of \Tikz.

\PGFPlots\ is build completely on \Tikz/\PGF. Knowledge of \Tikz\ will simplify the work with \PGFPlots, although it is not required.

\section[About PGFPlots: Preliminaries]{About {\normalfont\PGFPlots}: Preliminaries}
This section contains information about upgrades, the team, the installation (in case you need to do it manually) and troubleshooting. You may skip it completely except for the upgrade remarks.

However, note that this library requires at least \PGF\ version $2.00$. At the time of this writing, many \TeX-distributions still contain the older \PGF\ version $1.18$, so it may be necessary to install a recent \PGF\ prior to using \PGFPlots.

\subsection{Components}
\PGFPlots\ comes with two components:
\begin{enumerate}
	\item the plotting component (which you are currently reading) and
	\item the \PGFPlotstable\ component which simplifies number formatting and postprocessing of numerical tables. It comes as a separate package and has its own manual \href{file:pgfplotstable.pdf}{pgfplotstable.pdf}.
\end{enumerate}

\subsection{Upgrade remarks}
This release provides a lot of improvements which can be found in all detail in \texttt{ChangeLog} for interested readers. However, some attention is useful with respect to the following changes.

\subsubsection{New Optional Features}
\begin{enumerate}
	\item \PGFPlots\ 1.3 comes with user interface improvements. The technical distinction between ``behavior options'' and ``style options'' of older versions is no longer necessary (although still fully supported).

	\item \PGFPlots\ 1.3 has a new feature which allows to \emph{move axis labels tight to tick labels} automatically.

	Since this affects the spacing, it is not enabled be default.

	Use
\begin{codeexample}[code only]
\usepackage{pgfplots}
\pgfplotsset{compat=1.3}
\end{codeexample}
	in your preamble to benefit from the improved spacing\footnote{The |compat=1.3| setting is necessary for new features which move axis labels around like reversed axis. However, \PGFPlots\ will still work as it does in earlier versions, your documents will remain the same if you don't set it explicitly.}.
	Take a look at the next page for the precise definition of |compat|.

	\item \PGFPlots\ 1.3 now supports reversed axes. It is no longer necessary to use work--arounds with negative units.
\pgfkeys{/pdflinks/search key prefixes in/.add={/pgfplots/,}{}}

	Take a look at the |x dir=reverse| key.

	Existing work--arounds will still function properly. Use |\pgfplotsset{compat=1.3}| together with |x dir=reverse| to switch to the new version.
\end{enumerate}

\subsubsection{Old Features Which May Need Attention}
\begin{enumerate}
	\item The |scatter/classes| feature produces proper legends as of version 1.3. This may change the appearance of existing legends of plots with |scatter/classes|.

	\item Due to a small math bug in \PGF\ $2.00$, you \emph{can't} use the math expression `|-x^2|'. It is necessary to use `|0-x^2|' instead. The same holds for
		`|exp(-x^2)|'; use `|exp(0-x^2)|' instead.

		This will be fixed with \PGF\ $>2.00$.

	\item Starting with \PGFPlots\ $1.1$, |\tikzstyle| should \emph{no longer be used} to set \PGFPlots\ options.
	
	Although |\tikzstyle| is still supported for some older \PGFPlots\ options, you should replace any occurance of |\tikzstyle| with |\pgfplotsset{|\meta{style name}|/.style={|\meta{key-value-list}|}}| or the associated |/.append style| variant. See section~\ref{sec:styles} for more detail.
\end{enumerate}
I apologize for any inconvenience caused by these changes.

\begin{pgfplotskey}{compat=\mchoice{1.3,pre 1.3,default,newest} (initially default)}
	The preamble configuration 
\begin{codeexample}[code only]
\usepackage{pgfplots}
\pgfplotsset{compat=1.3}
\end{codeexample}
	allows to choose between backwards compatibility and most recent features.

	\PGFPlots\ version 1.3 comes with new features which may lead to different spacings compared to earlier versions:
	\begin{enumerate}
		\item Version 1.3 comes with the |xlabel near ticks| feature which places (in this case $x$) axis labels ``near'' the tick labels, taking the size of tick labels into account.
		
		Older versions used |xlabel absolute|, i.e.\ an absolute distance between the axis and axis labels (now matter how large or small tick labels are).

		The initial setting \emph{keeps backwards compatibility}.

		You are encouraged to use
\begin{codeexample}[code only]
\usepackage{pgfplots}
\pgfplotsset{compat=1.3}
\end{codeexample}
		in your preamble.
		
		\item Version 1.3 fixes a bug with |anchor=south west| which occurred mainly for reversed axes: the anchors where upside--down. This is fixed now.

		
		If you have written some work--around and want to keep it going, use

		|\pgfplotsset{compat/anchors=pre 1.3}|\pgfmanualpdflabel{/pgfplots/compat/anchors}{} 

		to disable the bug fix.
	\end{enumerate}

	Use |\pgfplotsset{compat=default}| to restore the factory settings.

	The setting |\pgfplotsset{compat=newest}| will always use features of the most recent version. This might result in small changes in the document's appearance.
\end{pgfplotskey}

\subsection{The Team}
\PGFPlots\ has been written mainly by Christian Feuers�nger with many improvements of Pascal Wolkotte and Nick Papior Andersen as a spare time project. We hope it is useful and provides valuable plots.

If you are interested in writing something but don't know how, consider reading the auxiliary manual \href{file:TeX-programming-notes.pdf}{TeX-programming-notes.pdf} which comes with \PGFPlots. It is far from complete, but maybe it is a good starting point (at least for more literature).

\subsection{Acknowledgements}
I thank God for all hours of enjoyed programming. I thank Pascal Wolkotte and Nick Papior Andersen for their programming efforts and contributions as part of the development team. I thank Stefan Tibus, who contributed the |plot shell| feature. I thank Tom Cashman for the contribution of the |reverse legend| feature. Special thanks go to Stefan Pinnow whose tests of \PGFPlots\ lead to numerous quality improvements. Furthermore, I thank Dr.~Schweitzer for many fruitful discussions and Dr.~Meine for his ideas and suggestions. Thanks as well to the many international contributors who provided feature requests or identified bugs or simply manual improvements!

Last but not least, I thank Till Tantau and Mark Wibrow for their excellent graphics (and more) package \PGF\ and \Tikz\ which is the base of \PGFPlots.

\include{pgfplots.install}%
\include{pgfplots.intro}%
\include{pgfplots.reference}%
\include{pgfplots.libs}%
\include{pgfplots.resources}%
\include{pgfplots.importexport}%
\include{pgfplots.basic.reference}%

\printindex

\bibliographystyle{abbrv} %gerapali} %gerabbrv} %gerunsrt.bst} %gerabbrv}% gerplain}
\nocite{pgfplotstable}
\nocite{programmingnotes}
\bibliography{pgfplots}
\end{document}
%
%%%%%%%%%%%%%%%%%%%%%%%%%%%%%%%%%%%%%%%%%%%%%%%%%%%%%%%%%%%%%%%%%%%%%%%%%%%%%
%
% Package pgfplots.sty documentation. 
%
% Copyright 2007/2008 by Christian Feuersaenger.
%
% This program is free software: you can redistribute it and/or modify
% it under the terms of the GNU General Public License as published by
% the Free Software Foundation, either version 3 of the License, or
% (at your option) any later version.
% 
% This program is distributed in the hope that it will be useful,
% but WITHOUT ANY WARRANTY; without even the implied warranty of
% MERCHANTABILITY or FITNESS FOR A PARTICULAR PURPOSE.  See the
% GNU General Public License for more details.
% 
% You should have received a copy of the GNU General Public License
% along with this program.  If not, see <http://www.gnu.org/licenses/>.
%
%
%%%%%%%%%%%%%%%%%%%%%%%%%%%%%%%%%%%%%%%%%%%%%%%%%%%%%%%%%%%%%%%%%%%%%%%%%%%%%
\input{pgfplots.preamble.tex}
%\RequirePackage[german,english,francais]{babel}

\pgfplotsmanualexternalexpensivetrue

%\usetikzlibrary{external}
%\tikzexternalize[prefix=figures/]{pgfplots}

\title{%
	Manual for Package \PGFPlots\\
	{\small Version \pgfplotsversion}\\
	{\small\href{http://sourceforge.net/projects/pgfplots}{http://sourceforge.net/projects/pgfplots}}}

%\includeonly{pgfplots.libs}


\begin{document}

\def\plotcoords{%
\addplot coordinates {
(5,8.312e-02)    (17,2.547e-02)   (49,7.407e-03)
(129,2.102e-03)  (321,5.874e-04)  (769,1.623e-04)
(1793,4.442e-05) (4097,1.207e-05) (9217,3.261e-06)
};

\addplot coordinates{
(7,8.472e-02)    (31,3.044e-02)    (111,1.022e-02)
(351,3.303e-03)  (1023,1.039e-03)  (2815,3.196e-04)
(7423,9.658e-05) (18943,2.873e-05) (47103,8.437e-06)
};

\addplot coordinates{
(9,7.881e-02)     (49,3.243e-02)    (209,1.232e-02)
(769,4.454e-03)   (2561,1.551e-03)  (7937,5.236e-04)
(23297,1.723e-04) (65537,5.545e-05) (178177,1.751e-05)
};

\addplot coordinates{
(11,6.887e-02)    (71,3.177e-02)     (351,1.341e-02)
(1471,5.334e-03)  (5503,2.027e-03)   (18943,7.415e-04)
(61183,2.628e-04) (187903,9.063e-05) (553983,3.053e-05)
};

\addplot coordinates{
(13,5.755e-02)     (97,2.925e-02)     (545,1.351e-02)
(2561,5.842e-03)   (10625,2.397e-03)  (40193,9.414e-04)
(141569,3.564e-04) (471041,1.308e-04) 
(1496065,4.670e-05)
};
}%
\maketitle
\begin{abstract}%
\PGFPlots\ draws high--quality function plots in normal or logarithmic scaling with a user-friendly interface directly in \TeX. The user supplies axis labels, legend entries and the plot coordinates for one or more plots and \PGFPlots\ applies axis scaling, computes any logarithms and axis ticks and draws the plots, supporting line plots, scatter plots, piecewise constant plots, bar plots, area plots, mesh-- and surface plots and some more. It is based on Till Tantau's package \PGF/\Tikz.
\end{abstract}
\tableofcontents
\section{Introduction}
This package provides tools to generate plots and labeled axes easily. It draws normal plots, logplots and semi-logplots, in two and three dimensions. Axis ticks, labels, legends (in case of multiple plots) can be added with key-value options. It can cycle through a set of predefined line/marker/color specifications. In summary, its purpose is to simplify the generation of high-quality function and/or data plots, and solving the problems of
\begin{itemize}
	\item consistency of document and font type and font size,
	\item direct use of \TeX\ math mode in axis descriptions,
	\item consistency of data and figures (no third party tool necessary),
	\item inter document consistency using preamble configurations and styles.
\end{itemize}
Although not necessary, separate |.pdf| or |.eps| graphics can be generated using the |external| library developed as part of \Tikz.

\PGFPlots\ is build completely on \Tikz/\PGF. Knowledge of \Tikz\ will simplify the work with \PGFPlots, although it is not required.

\section[About PGFPlots: Preliminaries]{About {\normalfont\PGFPlots}: Preliminaries}
This section contains information about upgrades, the team, the installation (in case you need to do it manually) and troubleshooting. You may skip it completely except for the upgrade remarks.

However, note that this library requires at least \PGF\ version $2.00$. At the time of this writing, many \TeX-distributions still contain the older \PGF\ version $1.18$, so it may be necessary to install a recent \PGF\ prior to using \PGFPlots.

\subsection{Components}
\PGFPlots\ comes with two components:
\begin{enumerate}
	\item the plotting component (which you are currently reading) and
	\item the \PGFPlotstable\ component which simplifies number formatting and postprocessing of numerical tables. It comes as a separate package and has its own manual \href{file:pgfplotstable.pdf}{pgfplotstable.pdf}.
\end{enumerate}

\subsection{Upgrade remarks}
This release provides a lot of improvements which can be found in all detail in \texttt{ChangeLog} for interested readers. However, some attention is useful with respect to the following changes.

\subsubsection{New Optional Features}
\begin{enumerate}
	\item \PGFPlots\ 1.3 comes with user interface improvements. The technical distinction between ``behavior options'' and ``style options'' of older versions is no longer necessary (although still fully supported).

	\item \PGFPlots\ 1.3 has a new feature which allows to \emph{move axis labels tight to tick labels} automatically.

	Since this affects the spacing, it is not enabled be default.

	Use
\begin{codeexample}[code only]
\usepackage{pgfplots}
\pgfplotsset{compat=1.3}
\end{codeexample}
	in your preamble to benefit from the improved spacing\footnote{The |compat=1.3| setting is necessary for new features which move axis labels around like reversed axis. However, \PGFPlots\ will still work as it does in earlier versions, your documents will remain the same if you don't set it explicitly.}.
	Take a look at the next page for the precise definition of |compat|.

	\item \PGFPlots\ 1.3 now supports reversed axes. It is no longer necessary to use work--arounds with negative units.
\pgfkeys{/pdflinks/search key prefixes in/.add={/pgfplots/,}{}}

	Take a look at the |x dir=reverse| key.

	Existing work--arounds will still function properly. Use |\pgfplotsset{compat=1.3}| together with |x dir=reverse| to switch to the new version.
\end{enumerate}

\subsubsection{Old Features Which May Need Attention}
\begin{enumerate}
	\item The |scatter/classes| feature produces proper legends as of version 1.3. This may change the appearance of existing legends of plots with |scatter/classes|.

	\item Due to a small math bug in \PGF\ $2.00$, you \emph{can't} use the math expression `|-x^2|'. It is necessary to use `|0-x^2|' instead. The same holds for
		`|exp(-x^2)|'; use `|exp(0-x^2)|' instead.

		This will be fixed with \PGF\ $>2.00$.

	\item Starting with \PGFPlots\ $1.1$, |\tikzstyle| should \emph{no longer be used} to set \PGFPlots\ options.
	
	Although |\tikzstyle| is still supported for some older \PGFPlots\ options, you should replace any occurance of |\tikzstyle| with |\pgfplotsset{|\meta{style name}|/.style={|\meta{key-value-list}|}}| or the associated |/.append style| variant. See section~\ref{sec:styles} for more detail.
\end{enumerate}
I apologize for any inconvenience caused by these changes.

\begin{pgfplotskey}{compat=\mchoice{1.3,pre 1.3,default,newest} (initially default)}
	The preamble configuration 
\begin{codeexample}[code only]
\usepackage{pgfplots}
\pgfplotsset{compat=1.3}
\end{codeexample}
	allows to choose between backwards compatibility and most recent features.

	\PGFPlots\ version 1.3 comes with new features which may lead to different spacings compared to earlier versions:
	\begin{enumerate}
		\item Version 1.3 comes with the |xlabel near ticks| feature which places (in this case $x$) axis labels ``near'' the tick labels, taking the size of tick labels into account.
		
		Older versions used |xlabel absolute|, i.e.\ an absolute distance between the axis and axis labels (now matter how large or small tick labels are).

		The initial setting \emph{keeps backwards compatibility}.

		You are encouraged to use
\begin{codeexample}[code only]
\usepackage{pgfplots}
\pgfplotsset{compat=1.3}
\end{codeexample}
		in your preamble.
		
		\item Version 1.3 fixes a bug with |anchor=south west| which occurred mainly for reversed axes: the anchors where upside--down. This is fixed now.

		
		If you have written some work--around and want to keep it going, use

		|\pgfplotsset{compat/anchors=pre 1.3}|\pgfmanualpdflabel{/pgfplots/compat/anchors}{} 

		to disable the bug fix.
	\end{enumerate}

	Use |\pgfplotsset{compat=default}| to restore the factory settings.

	The setting |\pgfplotsset{compat=newest}| will always use features of the most recent version. This might result in small changes in the document's appearance.
\end{pgfplotskey}

\subsection{The Team}
\PGFPlots\ has been written mainly by Christian Feuers�nger with many improvements of Pascal Wolkotte and Nick Papior Andersen as a spare time project. We hope it is useful and provides valuable plots.

If you are interested in writing something but don't know how, consider reading the auxiliary manual \href{file:TeX-programming-notes.pdf}{TeX-programming-notes.pdf} which comes with \PGFPlots. It is far from complete, but maybe it is a good starting point (at least for more literature).

\subsection{Acknowledgements}
I thank God for all hours of enjoyed programming. I thank Pascal Wolkotte and Nick Papior Andersen for their programming efforts and contributions as part of the development team. I thank Stefan Tibus, who contributed the |plot shell| feature. I thank Tom Cashman for the contribution of the |reverse legend| feature. Special thanks go to Stefan Pinnow whose tests of \PGFPlots\ lead to numerous quality improvements. Furthermore, I thank Dr.~Schweitzer for many fruitful discussions and Dr.~Meine for his ideas and suggestions. Thanks as well to the many international contributors who provided feature requests or identified bugs or simply manual improvements!

Last but not least, I thank Till Tantau and Mark Wibrow for their excellent graphics (and more) package \PGF\ and \Tikz\ which is the base of \PGFPlots.

\include{pgfplots.install}%
\include{pgfplots.intro}%
\include{pgfplots.reference}%
\include{pgfplots.libs}%
\include{pgfplots.resources}%
\include{pgfplots.importexport}%
\include{pgfplots.basic.reference}%

\printindex

\bibliographystyle{abbrv} %gerapali} %gerabbrv} %gerunsrt.bst} %gerabbrv}% gerplain}
\nocite{pgfplotstable}
\nocite{programmingnotes}
\bibliography{pgfplots}
\end{document}
%
\include{pgfplots.basic.reference}%

\printindex

\bibliographystyle{abbrv} %gerapali} %gerabbrv} %gerunsrt.bst} %gerabbrv}% gerplain}
\nocite{pgfplotstable}
\nocite{programmingnotes}
\bibliography{pgfplots}
\end{document}
%
%%%%%%%%%%%%%%%%%%%%%%%%%%%%%%%%%%%%%%%%%%%%%%%%%%%%%%%%%%%%%%%%%%%%%%%%%%%%%
%
% Package pgfplots.sty documentation. 
%
% Copyright 2007/2008 by Christian Feuersaenger.
%
% This program is free software: you can redistribute it and/or modify
% it under the terms of the GNU General Public License as published by
% the Free Software Foundation, either version 3 of the License, or
% (at your option) any later version.
% 
% This program is distributed in the hope that it will be useful,
% but WITHOUT ANY WARRANTY; without even the implied warranty of
% MERCHANTABILITY or FITNESS FOR A PARTICULAR PURPOSE.  See the
% GNU General Public License for more details.
% 
% You should have received a copy of the GNU General Public License
% along with this program.  If not, see <http://www.gnu.org/licenses/>.
%
%
%%%%%%%%%%%%%%%%%%%%%%%%%%%%%%%%%%%%%%%%%%%%%%%%%%%%%%%%%%%%%%%%%%%%%%%%%%%%%
%%%%%%%%%%%%%%%%%%%%%%%%%%%%%%%%%%%%%%%%%%%%%%%%%%%%%%%%%%%%%%%%%%%%%%%%%%%%%
%
% Package pgfplots.sty documentation. 
%
% Copyright 2007/2008 by Christian Feuersaenger.
%
% This program is free software: you can redistribute it and/or modify
% it under the terms of the GNU General Public License as published by
% the Free Software Foundation, either version 3 of the License, or
% (at your option) any later version.
% 
% This program is distributed in the hope that it will be useful,
% but WITHOUT ANY WARRANTY; without even the implied warranty of
% MERCHANTABILITY or FITNESS FOR A PARTICULAR PURPOSE.  See the
% GNU General Public License for more details.
% 
% You should have received a copy of the GNU General Public License
% along with this program.  If not, see <http://www.gnu.org/licenses/>.
%
%
%%%%%%%%%%%%%%%%%%%%%%%%%%%%%%%%%%%%%%%%%%%%%%%%%%%%%%%%%%%%%%%%%%%%%%%%%%%%%
\documentclass[a4paper]{ltxdoc}

\usepackage{makeidx}

% DON't let hyperref overload the format of index and glossary. 
% I want to do that on my own in the stylefiles for makeindex...
\makeatletter
\let\@old@wrindex=\@wrindex
\makeatother

\usepackage{ifpdf}
\ifpdf
	\usepackage[pdftex]{hyperref}
\else
	\usepackage[dvipdfm]{hyperref}
\fi
\hypersetup{pdfborder={0 0 0}}

\makeatletter
\let\@wrindex=\@old@wrindex
\makeatother

% Formatiere Seitennummern im Index:
\newcommand{\indexpageno}[1]{%
	{\bfseries\hyperpage{#1}}%
}


\newcommand{\R}{\mathbb{R}}
\newcommand{\N}{\mathbb{N}}
\newcommand{\Z}{\mathbb{Z}}

\long\def\COMMENTLOWLEVEL#1\ENDCOMMENT{}
\def\ENDCOMMENT{}

\usepackage{textcomp}
\usepackage{booktabs}

\usepackage{calc}
\usepackage[formats]{listings}
%\usepackage{courier} % don't use it - the '^' character can't be copy-pasted in courier

\usepackage{array}
\lstset{%
	basicstyle=\ttfamily,
	language=[LaTeX]tex, % Seems as if \lstset{language=tex} must be invoked BEFORE loading tikz!?
	tabsize=4,
	breaklines=true,
	breakindent=0pt
}

\ifpdf
	\pdfinfo {
		/Author	(Christian Feuersaenger)
	}

\else
	\def\pgfsysdriver{pgfsys-dvipdfm.def}
\fi
%\def\pgfsysdriver{pgfsys-pdftex.def}
\usepackage{pgfplots}

\ifpdf
	\usetikzlibrary{pgfplotsclickable}
\fi

%\usepackage{fp}
% ATTENTION:
% this requires pgf version NEWER than 2.00 :
%\usetikzlibrary{fixedpointarithmetic}

\usetikzlibrary{dateplot}

\usepackage[a4paper,left=2.25cm,right=2.25cm,top=2.5cm,bottom=2.5cm,nohead]{geometry}
\usepackage{amsmath,amssymb}
\usepackage{xxcolor}
\usepackage{pifont}
\usepackage[latin1]{inputenc}
\usepackage{amsmath}
\usepackage{eurosym}
\input{pgfplots-macros}

\pgfqkeys{/codeexample}{%
	every codeexample/.style={
		width=8cm,
		/pgfplots/every axis/.append style={legend style={fill=graphicbackground}}
	},
	tabsize=4,
}
\pgfplotsset{
	every axis/.append style={width=7cm},
	filter discard warning=false,
}

\usetikzlibrary{backgrounds,patterns}
% Global styles:
\tikzset{
  shape example/.style={
    color=black!30,
    draw,
    fill=yellow!30,
    line width=.5cm,
    inner xsep=2.5cm,
    inner ysep=0.5cm}
}

\newcommand{\FIXME}[1]{\textcolor{red}{(FIXME: #1)}}

% fuer endvironment 'sidewaysfigure' bspw
% \usepackage{rotating}

\newcommand\Tikz{Ti\textit kZ}
\newcommand\PGF{\textsc{pgf}}
\newcommand\PGFPlots{\textsc{pgfplots}}
\def\PGFPlotstable{\textsc{PgfplotsTable}}


\makeindex

% Fix overful hboxes automatically:
\tolerance=2000
\emergencystretch=10pt

\tikzset{prefix=gnuplot/pgfplots_} % prefix for 'plot function'

\author{%
	Christian Feuers\"anger\footnote{\url{http://wissrech.ins.uni-bonn.de/people/feuersaenger}}\\%
	Institut f\"ur Numerische Simulation\\
	Universit\"at Bonn, Germany}


%\RequirePackage[german,english,francais]{babel}

\pgfplotsmanualexternalexpensivetrue

%\usetikzlibrary{external}
%\tikzexternalize[prefix=figures/]{pgfplots}

\title{%
	Manual for Package \PGFPlots\\
	{\small Version \pgfplotsversion}\\
	{\small\href{http://sourceforge.net/projects/pgfplots}{http://sourceforge.net/projects/pgfplots}}}

%\includeonly{pgfplots.libs}


\begin{document}

\def\plotcoords{%
\addplot coordinates {
(5,8.312e-02)    (17,2.547e-02)   (49,7.407e-03)
(129,2.102e-03)  (321,5.874e-04)  (769,1.623e-04)
(1793,4.442e-05) (4097,1.207e-05) (9217,3.261e-06)
};

\addplot coordinates{
(7,8.472e-02)    (31,3.044e-02)    (111,1.022e-02)
(351,3.303e-03)  (1023,1.039e-03)  (2815,3.196e-04)
(7423,9.658e-05) (18943,2.873e-05) (47103,8.437e-06)
};

\addplot coordinates{
(9,7.881e-02)     (49,3.243e-02)    (209,1.232e-02)
(769,4.454e-03)   (2561,1.551e-03)  (7937,5.236e-04)
(23297,1.723e-04) (65537,5.545e-05) (178177,1.751e-05)
};

\addplot coordinates{
(11,6.887e-02)    (71,3.177e-02)     (351,1.341e-02)
(1471,5.334e-03)  (5503,2.027e-03)   (18943,7.415e-04)
(61183,2.628e-04) (187903,9.063e-05) (553983,3.053e-05)
};

\addplot coordinates{
(13,5.755e-02)     (97,2.925e-02)     (545,1.351e-02)
(2561,5.842e-03)   (10625,2.397e-03)  (40193,9.414e-04)
(141569,3.564e-04) (471041,1.308e-04) 
(1496065,4.670e-05)
};
}%
\maketitle
\begin{abstract}%
\PGFPlots\ draws high--quality function plots in normal or logarithmic scaling with a user-friendly interface directly in \TeX. The user supplies axis labels, legend entries and the plot coordinates for one or more plots and \PGFPlots\ applies axis scaling, computes any logarithms and axis ticks and draws the plots, supporting line plots, scatter plots, piecewise constant plots, bar plots, area plots, mesh-- and surface plots and some more. It is based on Till Tantau's package \PGF/\Tikz.
\end{abstract}
\tableofcontents
\section{Introduction}
This package provides tools to generate plots and labeled axes easily. It draws normal plots, logplots and semi-logplots, in two and three dimensions. Axis ticks, labels, legends (in case of multiple plots) can be added with key-value options. It can cycle through a set of predefined line/marker/color specifications. In summary, its purpose is to simplify the generation of high-quality function and/or data plots, and solving the problems of
\begin{itemize}
	\item consistency of document and font type and font size,
	\item direct use of \TeX\ math mode in axis descriptions,
	\item consistency of data and figures (no third party tool necessary),
	\item inter document consistency using preamble configurations and styles.
\end{itemize}
Although not necessary, separate |.pdf| or |.eps| graphics can be generated using the |external| library developed as part of \Tikz.

\PGFPlots\ is build completely on \Tikz/\PGF. Knowledge of \Tikz\ will simplify the work with \PGFPlots, although it is not required.

\section[About PGFPlots: Preliminaries]{About {\normalfont\PGFPlots}: Preliminaries}
This section contains information about upgrades, the team, the installation (in case you need to do it manually) and troubleshooting. You may skip it completely except for the upgrade remarks.

However, note that this library requires at least \PGF\ version $2.00$. At the time of this writing, many \TeX-distributions still contain the older \PGF\ version $1.18$, so it may be necessary to install a recent \PGF\ prior to using \PGFPlots.

\subsection{Components}
\PGFPlots\ comes with two components:
\begin{enumerate}
	\item the plotting component (which you are currently reading) and
	\item the \PGFPlotstable\ component which simplifies number formatting and postprocessing of numerical tables. It comes as a separate package and has its own manual \href{file:pgfplotstable.pdf}{pgfplotstable.pdf}.
\end{enumerate}

\subsection{Upgrade remarks}
This release provides a lot of improvements which can be found in all detail in \texttt{ChangeLog} for interested readers. However, some attention is useful with respect to the following changes.

\subsubsection{New Optional Features}
\begin{enumerate}
	\item \PGFPlots\ 1.3 comes with user interface improvements. The technical distinction between ``behavior options'' and ``style options'' of older versions is no longer necessary (although still fully supported).

	\item \PGFPlots\ 1.3 has a new feature which allows to \emph{move axis labels tight to tick labels} automatically.

	Since this affects the spacing, it is not enabled be default.

	Use
\begin{codeexample}[code only]
\usepackage{pgfplots}
\pgfplotsset{compat=1.3}
\end{codeexample}
	in your preamble to benefit from the improved spacing\footnote{The |compat=1.3| setting is necessary for new features which move axis labels around like reversed axis. However, \PGFPlots\ will still work as it does in earlier versions, your documents will remain the same if you don't set it explicitly.}.
	Take a look at the next page for the precise definition of |compat|.

	\item \PGFPlots\ 1.3 now supports reversed axes. It is no longer necessary to use work--arounds with negative units.
\pgfkeys{/pdflinks/search key prefixes in/.add={/pgfplots/,}{}}

	Take a look at the |x dir=reverse| key.

	Existing work--arounds will still function properly. Use |\pgfplotsset{compat=1.3}| together with |x dir=reverse| to switch to the new version.
\end{enumerate}

\subsubsection{Old Features Which May Need Attention}
\begin{enumerate}
	\item The |scatter/classes| feature produces proper legends as of version 1.3. This may change the appearance of existing legends of plots with |scatter/classes|.

	\item Due to a small math bug in \PGF\ $2.00$, you \emph{can't} use the math expression `|-x^2|'. It is necessary to use `|0-x^2|' instead. The same holds for
		`|exp(-x^2)|'; use `|exp(0-x^2)|' instead.

		This will be fixed with \PGF\ $>2.00$.

	\item Starting with \PGFPlots\ $1.1$, |\tikzstyle| should \emph{no longer be used} to set \PGFPlots\ options.
	
	Although |\tikzstyle| is still supported for some older \PGFPlots\ options, you should replace any occurance of |\tikzstyle| with |\pgfplotsset{|\meta{style name}|/.style={|\meta{key-value-list}|}}| or the associated |/.append style| variant. See section~\ref{sec:styles} for more detail.
\end{enumerate}
I apologize for any inconvenience caused by these changes.

\begin{pgfplotskey}{compat=\mchoice{1.3,pre 1.3,default,newest} (initially default)}
	The preamble configuration 
\begin{codeexample}[code only]
\usepackage{pgfplots}
\pgfplotsset{compat=1.3}
\end{codeexample}
	allows to choose between backwards compatibility and most recent features.

	\PGFPlots\ version 1.3 comes with new features which may lead to different spacings compared to earlier versions:
	\begin{enumerate}
		\item Version 1.3 comes with the |xlabel near ticks| feature which places (in this case $x$) axis labels ``near'' the tick labels, taking the size of tick labels into account.
		
		Older versions used |xlabel absolute|, i.e.\ an absolute distance between the axis and axis labels (now matter how large or small tick labels are).

		The initial setting \emph{keeps backwards compatibility}.

		You are encouraged to use
\begin{codeexample}[code only]
\usepackage{pgfplots}
\pgfplotsset{compat=1.3}
\end{codeexample}
		in your preamble.
		
		\item Version 1.3 fixes a bug with |anchor=south west| which occurred mainly for reversed axes: the anchors where upside--down. This is fixed now.

		
		If you have written some work--around and want to keep it going, use

		|\pgfplotsset{compat/anchors=pre 1.3}|\pgfmanualpdflabel{/pgfplots/compat/anchors}{} 

		to disable the bug fix.
	\end{enumerate}

	Use |\pgfplotsset{compat=default}| to restore the factory settings.

	The setting |\pgfplotsset{compat=newest}| will always use features of the most recent version. This might result in small changes in the document's appearance.
\end{pgfplotskey}

\subsection{The Team}
\PGFPlots\ has been written mainly by Christian Feuers�nger with many improvements of Pascal Wolkotte and Nick Papior Andersen as a spare time project. We hope it is useful and provides valuable plots.

If you are interested in writing something but don't know how, consider reading the auxiliary manual \href{file:TeX-programming-notes.pdf}{TeX-programming-notes.pdf} which comes with \PGFPlots. It is far from complete, but maybe it is a good starting point (at least for more literature).

\subsection{Acknowledgements}
I thank God for all hours of enjoyed programming. I thank Pascal Wolkotte and Nick Papior Andersen for their programming efforts and contributions as part of the development team. I thank Stefan Tibus, who contributed the |plot shell| feature. I thank Tom Cashman for the contribution of the |reverse legend| feature. Special thanks go to Stefan Pinnow whose tests of \PGFPlots\ lead to numerous quality improvements. Furthermore, I thank Dr.~Schweitzer for many fruitful discussions and Dr.~Meine for his ideas and suggestions. Thanks as well to the many international contributors who provided feature requests or identified bugs or simply manual improvements!

Last but not least, I thank Till Tantau and Mark Wibrow for their excellent graphics (and more) package \PGF\ and \Tikz\ which is the base of \PGFPlots.

%%%%%%%%%%%%%%%%%%%%%%%%%%%%%%%%%%%%%%%%%%%%%%%%%%%%%%%%%%%%%%%%%%%%%%%%%%%%%
%
% Package pgfplots.sty documentation. 
%
% Copyright 2007/2008 by Christian Feuersaenger.
%
% This program is free software: you can redistribute it and/or modify
% it under the terms of the GNU General Public License as published by
% the Free Software Foundation, either version 3 of the License, or
% (at your option) any later version.
% 
% This program is distributed in the hope that it will be useful,
% but WITHOUT ANY WARRANTY; without even the implied warranty of
% MERCHANTABILITY or FITNESS FOR A PARTICULAR PURPOSE.  See the
% GNU General Public License for more details.
% 
% You should have received a copy of the GNU General Public License
% along with this program.  If not, see <http://www.gnu.org/licenses/>.
%
%
%%%%%%%%%%%%%%%%%%%%%%%%%%%%%%%%%%%%%%%%%%%%%%%%%%%%%%%%%%%%%%%%%%%%%%%%%%%%%
\input{pgfplots.preamble.tex}
%\RequirePackage[german,english,francais]{babel}

\pgfplotsmanualexternalexpensivetrue

%\usetikzlibrary{external}
%\tikzexternalize[prefix=figures/]{pgfplots}

\title{%
	Manual for Package \PGFPlots\\
	{\small Version \pgfplotsversion}\\
	{\small\href{http://sourceforge.net/projects/pgfplots}{http://sourceforge.net/projects/pgfplots}}}

%\includeonly{pgfplots.libs}


\begin{document}

\def\plotcoords{%
\addplot coordinates {
(5,8.312e-02)    (17,2.547e-02)   (49,7.407e-03)
(129,2.102e-03)  (321,5.874e-04)  (769,1.623e-04)
(1793,4.442e-05) (4097,1.207e-05) (9217,3.261e-06)
};

\addplot coordinates{
(7,8.472e-02)    (31,3.044e-02)    (111,1.022e-02)
(351,3.303e-03)  (1023,1.039e-03)  (2815,3.196e-04)
(7423,9.658e-05) (18943,2.873e-05) (47103,8.437e-06)
};

\addplot coordinates{
(9,7.881e-02)     (49,3.243e-02)    (209,1.232e-02)
(769,4.454e-03)   (2561,1.551e-03)  (7937,5.236e-04)
(23297,1.723e-04) (65537,5.545e-05) (178177,1.751e-05)
};

\addplot coordinates{
(11,6.887e-02)    (71,3.177e-02)     (351,1.341e-02)
(1471,5.334e-03)  (5503,2.027e-03)   (18943,7.415e-04)
(61183,2.628e-04) (187903,9.063e-05) (553983,3.053e-05)
};

\addplot coordinates{
(13,5.755e-02)     (97,2.925e-02)     (545,1.351e-02)
(2561,5.842e-03)   (10625,2.397e-03)  (40193,9.414e-04)
(141569,3.564e-04) (471041,1.308e-04) 
(1496065,4.670e-05)
};
}%
\maketitle
\begin{abstract}%
\PGFPlots\ draws high--quality function plots in normal or logarithmic scaling with a user-friendly interface directly in \TeX. The user supplies axis labels, legend entries and the plot coordinates for one or more plots and \PGFPlots\ applies axis scaling, computes any logarithms and axis ticks and draws the plots, supporting line plots, scatter plots, piecewise constant plots, bar plots, area plots, mesh-- and surface plots and some more. It is based on Till Tantau's package \PGF/\Tikz.
\end{abstract}
\tableofcontents
\section{Introduction}
This package provides tools to generate plots and labeled axes easily. It draws normal plots, logplots and semi-logplots, in two and three dimensions. Axis ticks, labels, legends (in case of multiple plots) can be added with key-value options. It can cycle through a set of predefined line/marker/color specifications. In summary, its purpose is to simplify the generation of high-quality function and/or data plots, and solving the problems of
\begin{itemize}
	\item consistency of document and font type and font size,
	\item direct use of \TeX\ math mode in axis descriptions,
	\item consistency of data and figures (no third party tool necessary),
	\item inter document consistency using preamble configurations and styles.
\end{itemize}
Although not necessary, separate |.pdf| or |.eps| graphics can be generated using the |external| library developed as part of \Tikz.

\PGFPlots\ is build completely on \Tikz/\PGF. Knowledge of \Tikz\ will simplify the work with \PGFPlots, although it is not required.

\section[About PGFPlots: Preliminaries]{About {\normalfont\PGFPlots}: Preliminaries}
This section contains information about upgrades, the team, the installation (in case you need to do it manually) and troubleshooting. You may skip it completely except for the upgrade remarks.

However, note that this library requires at least \PGF\ version $2.00$. At the time of this writing, many \TeX-distributions still contain the older \PGF\ version $1.18$, so it may be necessary to install a recent \PGF\ prior to using \PGFPlots.

\subsection{Components}
\PGFPlots\ comes with two components:
\begin{enumerate}
	\item the plotting component (which you are currently reading) and
	\item the \PGFPlotstable\ component which simplifies number formatting and postprocessing of numerical tables. It comes as a separate package and has its own manual \href{file:pgfplotstable.pdf}{pgfplotstable.pdf}.
\end{enumerate}

\subsection{Upgrade remarks}
This release provides a lot of improvements which can be found in all detail in \texttt{ChangeLog} for interested readers. However, some attention is useful with respect to the following changes.

\subsubsection{New Optional Features}
\begin{enumerate}
	\item \PGFPlots\ 1.3 comes with user interface improvements. The technical distinction between ``behavior options'' and ``style options'' of older versions is no longer necessary (although still fully supported).

	\item \PGFPlots\ 1.3 has a new feature which allows to \emph{move axis labels tight to tick labels} automatically.

	Since this affects the spacing, it is not enabled be default.

	Use
\begin{codeexample}[code only]
\usepackage{pgfplots}
\pgfplotsset{compat=1.3}
\end{codeexample}
	in your preamble to benefit from the improved spacing\footnote{The |compat=1.3| setting is necessary for new features which move axis labels around like reversed axis. However, \PGFPlots\ will still work as it does in earlier versions, your documents will remain the same if you don't set it explicitly.}.
	Take a look at the next page for the precise definition of |compat|.

	\item \PGFPlots\ 1.3 now supports reversed axes. It is no longer necessary to use work--arounds with negative units.
\pgfkeys{/pdflinks/search key prefixes in/.add={/pgfplots/,}{}}

	Take a look at the |x dir=reverse| key.

	Existing work--arounds will still function properly. Use |\pgfplotsset{compat=1.3}| together with |x dir=reverse| to switch to the new version.
\end{enumerate}

\subsubsection{Old Features Which May Need Attention}
\begin{enumerate}
	\item The |scatter/classes| feature produces proper legends as of version 1.3. This may change the appearance of existing legends of plots with |scatter/classes|.

	\item Due to a small math bug in \PGF\ $2.00$, you \emph{can't} use the math expression `|-x^2|'. It is necessary to use `|0-x^2|' instead. The same holds for
		`|exp(-x^2)|'; use `|exp(0-x^2)|' instead.

		This will be fixed with \PGF\ $>2.00$.

	\item Starting with \PGFPlots\ $1.1$, |\tikzstyle| should \emph{no longer be used} to set \PGFPlots\ options.
	
	Although |\tikzstyle| is still supported for some older \PGFPlots\ options, you should replace any occurance of |\tikzstyle| with |\pgfplotsset{|\meta{style name}|/.style={|\meta{key-value-list}|}}| or the associated |/.append style| variant. See section~\ref{sec:styles} for more detail.
\end{enumerate}
I apologize for any inconvenience caused by these changes.

\begin{pgfplotskey}{compat=\mchoice{1.3,pre 1.3,default,newest} (initially default)}
	The preamble configuration 
\begin{codeexample}[code only]
\usepackage{pgfplots}
\pgfplotsset{compat=1.3}
\end{codeexample}
	allows to choose between backwards compatibility and most recent features.

	\PGFPlots\ version 1.3 comes with new features which may lead to different spacings compared to earlier versions:
	\begin{enumerate}
		\item Version 1.3 comes with the |xlabel near ticks| feature which places (in this case $x$) axis labels ``near'' the tick labels, taking the size of tick labels into account.
		
		Older versions used |xlabel absolute|, i.e.\ an absolute distance between the axis and axis labels (now matter how large or small tick labels are).

		The initial setting \emph{keeps backwards compatibility}.

		You are encouraged to use
\begin{codeexample}[code only]
\usepackage{pgfplots}
\pgfplotsset{compat=1.3}
\end{codeexample}
		in your preamble.
		
		\item Version 1.3 fixes a bug with |anchor=south west| which occurred mainly for reversed axes: the anchors where upside--down. This is fixed now.

		
		If you have written some work--around and want to keep it going, use

		|\pgfplotsset{compat/anchors=pre 1.3}|\pgfmanualpdflabel{/pgfplots/compat/anchors}{} 

		to disable the bug fix.
	\end{enumerate}

	Use |\pgfplotsset{compat=default}| to restore the factory settings.

	The setting |\pgfplotsset{compat=newest}| will always use features of the most recent version. This might result in small changes in the document's appearance.
\end{pgfplotskey}

\subsection{The Team}
\PGFPlots\ has been written mainly by Christian Feuers�nger with many improvements of Pascal Wolkotte and Nick Papior Andersen as a spare time project. We hope it is useful and provides valuable plots.

If you are interested in writing something but don't know how, consider reading the auxiliary manual \href{file:TeX-programming-notes.pdf}{TeX-programming-notes.pdf} which comes with \PGFPlots. It is far from complete, but maybe it is a good starting point (at least for more literature).

\subsection{Acknowledgements}
I thank God for all hours of enjoyed programming. I thank Pascal Wolkotte and Nick Papior Andersen for their programming efforts and contributions as part of the development team. I thank Stefan Tibus, who contributed the |plot shell| feature. I thank Tom Cashman for the contribution of the |reverse legend| feature. Special thanks go to Stefan Pinnow whose tests of \PGFPlots\ lead to numerous quality improvements. Furthermore, I thank Dr.~Schweitzer for many fruitful discussions and Dr.~Meine for his ideas and suggestions. Thanks as well to the many international contributors who provided feature requests or identified bugs or simply manual improvements!

Last but not least, I thank Till Tantau and Mark Wibrow for their excellent graphics (and more) package \PGF\ and \Tikz\ which is the base of \PGFPlots.

\include{pgfplots.install}%
\include{pgfplots.intro}%
\include{pgfplots.reference}%
\include{pgfplots.libs}%
\include{pgfplots.resources}%
\include{pgfplots.importexport}%
\include{pgfplots.basic.reference}%

\printindex

\bibliographystyle{abbrv} %gerapali} %gerabbrv} %gerunsrt.bst} %gerabbrv}% gerplain}
\nocite{pgfplotstable}
\nocite{programmingnotes}
\bibliography{pgfplots}
\end{document}
%
%%%%%%%%%%%%%%%%%%%%%%%%%%%%%%%%%%%%%%%%%%%%%%%%%%%%%%%%%%%%%%%%%%%%%%%%%%%%%
%
% Package pgfplots.sty documentation. 
%
% Copyright 2007/2008 by Christian Feuersaenger.
%
% This program is free software: you can redistribute it and/or modify
% it under the terms of the GNU General Public License as published by
% the Free Software Foundation, either version 3 of the License, or
% (at your option) any later version.
% 
% This program is distributed in the hope that it will be useful,
% but WITHOUT ANY WARRANTY; without even the implied warranty of
% MERCHANTABILITY or FITNESS FOR A PARTICULAR PURPOSE.  See the
% GNU General Public License for more details.
% 
% You should have received a copy of the GNU General Public License
% along with this program.  If not, see <http://www.gnu.org/licenses/>.
%
%
%%%%%%%%%%%%%%%%%%%%%%%%%%%%%%%%%%%%%%%%%%%%%%%%%%%%%%%%%%%%%%%%%%%%%%%%%%%%%
\input{pgfplots.preamble.tex}
%\RequirePackage[german,english,francais]{babel}

\pgfplotsmanualexternalexpensivetrue

%\usetikzlibrary{external}
%\tikzexternalize[prefix=figures/]{pgfplots}

\title{%
	Manual for Package \PGFPlots\\
	{\small Version \pgfplotsversion}\\
	{\small\href{http://sourceforge.net/projects/pgfplots}{http://sourceforge.net/projects/pgfplots}}}

%\includeonly{pgfplots.libs}


\begin{document}

\def\plotcoords{%
\addplot coordinates {
(5,8.312e-02)    (17,2.547e-02)   (49,7.407e-03)
(129,2.102e-03)  (321,5.874e-04)  (769,1.623e-04)
(1793,4.442e-05) (4097,1.207e-05) (9217,3.261e-06)
};

\addplot coordinates{
(7,8.472e-02)    (31,3.044e-02)    (111,1.022e-02)
(351,3.303e-03)  (1023,1.039e-03)  (2815,3.196e-04)
(7423,9.658e-05) (18943,2.873e-05) (47103,8.437e-06)
};

\addplot coordinates{
(9,7.881e-02)     (49,3.243e-02)    (209,1.232e-02)
(769,4.454e-03)   (2561,1.551e-03)  (7937,5.236e-04)
(23297,1.723e-04) (65537,5.545e-05) (178177,1.751e-05)
};

\addplot coordinates{
(11,6.887e-02)    (71,3.177e-02)     (351,1.341e-02)
(1471,5.334e-03)  (5503,2.027e-03)   (18943,7.415e-04)
(61183,2.628e-04) (187903,9.063e-05) (553983,3.053e-05)
};

\addplot coordinates{
(13,5.755e-02)     (97,2.925e-02)     (545,1.351e-02)
(2561,5.842e-03)   (10625,2.397e-03)  (40193,9.414e-04)
(141569,3.564e-04) (471041,1.308e-04) 
(1496065,4.670e-05)
};
}%
\maketitle
\begin{abstract}%
\PGFPlots\ draws high--quality function plots in normal or logarithmic scaling with a user-friendly interface directly in \TeX. The user supplies axis labels, legend entries and the plot coordinates for one or more plots and \PGFPlots\ applies axis scaling, computes any logarithms and axis ticks and draws the plots, supporting line plots, scatter plots, piecewise constant plots, bar plots, area plots, mesh-- and surface plots and some more. It is based on Till Tantau's package \PGF/\Tikz.
\end{abstract}
\tableofcontents
\section{Introduction}
This package provides tools to generate plots and labeled axes easily. It draws normal plots, logplots and semi-logplots, in two and three dimensions. Axis ticks, labels, legends (in case of multiple plots) can be added with key-value options. It can cycle through a set of predefined line/marker/color specifications. In summary, its purpose is to simplify the generation of high-quality function and/or data plots, and solving the problems of
\begin{itemize}
	\item consistency of document and font type and font size,
	\item direct use of \TeX\ math mode in axis descriptions,
	\item consistency of data and figures (no third party tool necessary),
	\item inter document consistency using preamble configurations and styles.
\end{itemize}
Although not necessary, separate |.pdf| or |.eps| graphics can be generated using the |external| library developed as part of \Tikz.

\PGFPlots\ is build completely on \Tikz/\PGF. Knowledge of \Tikz\ will simplify the work with \PGFPlots, although it is not required.

\section[About PGFPlots: Preliminaries]{About {\normalfont\PGFPlots}: Preliminaries}
This section contains information about upgrades, the team, the installation (in case you need to do it manually) and troubleshooting. You may skip it completely except for the upgrade remarks.

However, note that this library requires at least \PGF\ version $2.00$. At the time of this writing, many \TeX-distributions still contain the older \PGF\ version $1.18$, so it may be necessary to install a recent \PGF\ prior to using \PGFPlots.

\subsection{Components}
\PGFPlots\ comes with two components:
\begin{enumerate}
	\item the plotting component (which you are currently reading) and
	\item the \PGFPlotstable\ component which simplifies number formatting and postprocessing of numerical tables. It comes as a separate package and has its own manual \href{file:pgfplotstable.pdf}{pgfplotstable.pdf}.
\end{enumerate}

\subsection{Upgrade remarks}
This release provides a lot of improvements which can be found in all detail in \texttt{ChangeLog} for interested readers. However, some attention is useful with respect to the following changes.

\subsubsection{New Optional Features}
\begin{enumerate}
	\item \PGFPlots\ 1.3 comes with user interface improvements. The technical distinction between ``behavior options'' and ``style options'' of older versions is no longer necessary (although still fully supported).

	\item \PGFPlots\ 1.3 has a new feature which allows to \emph{move axis labels tight to tick labels} automatically.

	Since this affects the spacing, it is not enabled be default.

	Use
\begin{codeexample}[code only]
\usepackage{pgfplots}
\pgfplotsset{compat=1.3}
\end{codeexample}
	in your preamble to benefit from the improved spacing\footnote{The |compat=1.3| setting is necessary for new features which move axis labels around like reversed axis. However, \PGFPlots\ will still work as it does in earlier versions, your documents will remain the same if you don't set it explicitly.}.
	Take a look at the next page for the precise definition of |compat|.

	\item \PGFPlots\ 1.3 now supports reversed axes. It is no longer necessary to use work--arounds with negative units.
\pgfkeys{/pdflinks/search key prefixes in/.add={/pgfplots/,}{}}

	Take a look at the |x dir=reverse| key.

	Existing work--arounds will still function properly. Use |\pgfplotsset{compat=1.3}| together with |x dir=reverse| to switch to the new version.
\end{enumerate}

\subsubsection{Old Features Which May Need Attention}
\begin{enumerate}
	\item The |scatter/classes| feature produces proper legends as of version 1.3. This may change the appearance of existing legends of plots with |scatter/classes|.

	\item Due to a small math bug in \PGF\ $2.00$, you \emph{can't} use the math expression `|-x^2|'. It is necessary to use `|0-x^2|' instead. The same holds for
		`|exp(-x^2)|'; use `|exp(0-x^2)|' instead.

		This will be fixed with \PGF\ $>2.00$.

	\item Starting with \PGFPlots\ $1.1$, |\tikzstyle| should \emph{no longer be used} to set \PGFPlots\ options.
	
	Although |\tikzstyle| is still supported for some older \PGFPlots\ options, you should replace any occurance of |\tikzstyle| with |\pgfplotsset{|\meta{style name}|/.style={|\meta{key-value-list}|}}| or the associated |/.append style| variant. See section~\ref{sec:styles} for more detail.
\end{enumerate}
I apologize for any inconvenience caused by these changes.

\begin{pgfplotskey}{compat=\mchoice{1.3,pre 1.3,default,newest} (initially default)}
	The preamble configuration 
\begin{codeexample}[code only]
\usepackage{pgfplots}
\pgfplotsset{compat=1.3}
\end{codeexample}
	allows to choose between backwards compatibility and most recent features.

	\PGFPlots\ version 1.3 comes with new features which may lead to different spacings compared to earlier versions:
	\begin{enumerate}
		\item Version 1.3 comes with the |xlabel near ticks| feature which places (in this case $x$) axis labels ``near'' the tick labels, taking the size of tick labels into account.
		
		Older versions used |xlabel absolute|, i.e.\ an absolute distance between the axis and axis labels (now matter how large or small tick labels are).

		The initial setting \emph{keeps backwards compatibility}.

		You are encouraged to use
\begin{codeexample}[code only]
\usepackage{pgfplots}
\pgfplotsset{compat=1.3}
\end{codeexample}
		in your preamble.
		
		\item Version 1.3 fixes a bug with |anchor=south west| which occurred mainly for reversed axes: the anchors where upside--down. This is fixed now.

		
		If you have written some work--around and want to keep it going, use

		|\pgfplotsset{compat/anchors=pre 1.3}|\pgfmanualpdflabel{/pgfplots/compat/anchors}{} 

		to disable the bug fix.
	\end{enumerate}

	Use |\pgfplotsset{compat=default}| to restore the factory settings.

	The setting |\pgfplotsset{compat=newest}| will always use features of the most recent version. This might result in small changes in the document's appearance.
\end{pgfplotskey}

\subsection{The Team}
\PGFPlots\ has been written mainly by Christian Feuers�nger with many improvements of Pascal Wolkotte and Nick Papior Andersen as a spare time project. We hope it is useful and provides valuable plots.

If you are interested in writing something but don't know how, consider reading the auxiliary manual \href{file:TeX-programming-notes.pdf}{TeX-programming-notes.pdf} which comes with \PGFPlots. It is far from complete, but maybe it is a good starting point (at least for more literature).

\subsection{Acknowledgements}
I thank God for all hours of enjoyed programming. I thank Pascal Wolkotte and Nick Papior Andersen for their programming efforts and contributions as part of the development team. I thank Stefan Tibus, who contributed the |plot shell| feature. I thank Tom Cashman for the contribution of the |reverse legend| feature. Special thanks go to Stefan Pinnow whose tests of \PGFPlots\ lead to numerous quality improvements. Furthermore, I thank Dr.~Schweitzer for many fruitful discussions and Dr.~Meine for his ideas and suggestions. Thanks as well to the many international contributors who provided feature requests or identified bugs or simply manual improvements!

Last but not least, I thank Till Tantau and Mark Wibrow for their excellent graphics (and more) package \PGF\ and \Tikz\ which is the base of \PGFPlots.

\include{pgfplots.install}%
\include{pgfplots.intro}%
\include{pgfplots.reference}%
\include{pgfplots.libs}%
\include{pgfplots.resources}%
\include{pgfplots.importexport}%
\include{pgfplots.basic.reference}%

\printindex

\bibliographystyle{abbrv} %gerapali} %gerabbrv} %gerunsrt.bst} %gerabbrv}% gerplain}
\nocite{pgfplotstable}
\nocite{programmingnotes}
\bibliography{pgfplots}
\end{document}
%
%%%%%%%%%%%%%%%%%%%%%%%%%%%%%%%%%%%%%%%%%%%%%%%%%%%%%%%%%%%%%%%%%%%%%%%%%%%%%
%
% Package pgfplots.sty documentation. 
%
% Copyright 2007/2008 by Christian Feuersaenger.
%
% This program is free software: you can redistribute it and/or modify
% it under the terms of the GNU General Public License as published by
% the Free Software Foundation, either version 3 of the License, or
% (at your option) any later version.
% 
% This program is distributed in the hope that it will be useful,
% but WITHOUT ANY WARRANTY; without even the implied warranty of
% MERCHANTABILITY or FITNESS FOR A PARTICULAR PURPOSE.  See the
% GNU General Public License for more details.
% 
% You should have received a copy of the GNU General Public License
% along with this program.  If not, see <http://www.gnu.org/licenses/>.
%
%
%%%%%%%%%%%%%%%%%%%%%%%%%%%%%%%%%%%%%%%%%%%%%%%%%%%%%%%%%%%%%%%%%%%%%%%%%%%%%
\input{pgfplots.preamble.tex}
%\RequirePackage[german,english,francais]{babel}

\pgfplotsmanualexternalexpensivetrue

%\usetikzlibrary{external}
%\tikzexternalize[prefix=figures/]{pgfplots}

\title{%
	Manual for Package \PGFPlots\\
	{\small Version \pgfplotsversion}\\
	{\small\href{http://sourceforge.net/projects/pgfplots}{http://sourceforge.net/projects/pgfplots}}}

%\includeonly{pgfplots.libs}


\begin{document}

\def\plotcoords{%
\addplot coordinates {
(5,8.312e-02)    (17,2.547e-02)   (49,7.407e-03)
(129,2.102e-03)  (321,5.874e-04)  (769,1.623e-04)
(1793,4.442e-05) (4097,1.207e-05) (9217,3.261e-06)
};

\addplot coordinates{
(7,8.472e-02)    (31,3.044e-02)    (111,1.022e-02)
(351,3.303e-03)  (1023,1.039e-03)  (2815,3.196e-04)
(7423,9.658e-05) (18943,2.873e-05) (47103,8.437e-06)
};

\addplot coordinates{
(9,7.881e-02)     (49,3.243e-02)    (209,1.232e-02)
(769,4.454e-03)   (2561,1.551e-03)  (7937,5.236e-04)
(23297,1.723e-04) (65537,5.545e-05) (178177,1.751e-05)
};

\addplot coordinates{
(11,6.887e-02)    (71,3.177e-02)     (351,1.341e-02)
(1471,5.334e-03)  (5503,2.027e-03)   (18943,7.415e-04)
(61183,2.628e-04) (187903,9.063e-05) (553983,3.053e-05)
};

\addplot coordinates{
(13,5.755e-02)     (97,2.925e-02)     (545,1.351e-02)
(2561,5.842e-03)   (10625,2.397e-03)  (40193,9.414e-04)
(141569,3.564e-04) (471041,1.308e-04) 
(1496065,4.670e-05)
};
}%
\maketitle
\begin{abstract}%
\PGFPlots\ draws high--quality function plots in normal or logarithmic scaling with a user-friendly interface directly in \TeX. The user supplies axis labels, legend entries and the plot coordinates for one or more plots and \PGFPlots\ applies axis scaling, computes any logarithms and axis ticks and draws the plots, supporting line plots, scatter plots, piecewise constant plots, bar plots, area plots, mesh-- and surface plots and some more. It is based on Till Tantau's package \PGF/\Tikz.
\end{abstract}
\tableofcontents
\section{Introduction}
This package provides tools to generate plots and labeled axes easily. It draws normal plots, logplots and semi-logplots, in two and three dimensions. Axis ticks, labels, legends (in case of multiple plots) can be added with key-value options. It can cycle through a set of predefined line/marker/color specifications. In summary, its purpose is to simplify the generation of high-quality function and/or data plots, and solving the problems of
\begin{itemize}
	\item consistency of document and font type and font size,
	\item direct use of \TeX\ math mode in axis descriptions,
	\item consistency of data and figures (no third party tool necessary),
	\item inter document consistency using preamble configurations and styles.
\end{itemize}
Although not necessary, separate |.pdf| or |.eps| graphics can be generated using the |external| library developed as part of \Tikz.

\PGFPlots\ is build completely on \Tikz/\PGF. Knowledge of \Tikz\ will simplify the work with \PGFPlots, although it is not required.

\section[About PGFPlots: Preliminaries]{About {\normalfont\PGFPlots}: Preliminaries}
This section contains information about upgrades, the team, the installation (in case you need to do it manually) and troubleshooting. You may skip it completely except for the upgrade remarks.

However, note that this library requires at least \PGF\ version $2.00$. At the time of this writing, many \TeX-distributions still contain the older \PGF\ version $1.18$, so it may be necessary to install a recent \PGF\ prior to using \PGFPlots.

\subsection{Components}
\PGFPlots\ comes with two components:
\begin{enumerate}
	\item the plotting component (which you are currently reading) and
	\item the \PGFPlotstable\ component which simplifies number formatting and postprocessing of numerical tables. It comes as a separate package and has its own manual \href{file:pgfplotstable.pdf}{pgfplotstable.pdf}.
\end{enumerate}

\subsection{Upgrade remarks}
This release provides a lot of improvements which can be found in all detail in \texttt{ChangeLog} for interested readers. However, some attention is useful with respect to the following changes.

\subsubsection{New Optional Features}
\begin{enumerate}
	\item \PGFPlots\ 1.3 comes with user interface improvements. The technical distinction between ``behavior options'' and ``style options'' of older versions is no longer necessary (although still fully supported).

	\item \PGFPlots\ 1.3 has a new feature which allows to \emph{move axis labels tight to tick labels} automatically.

	Since this affects the spacing, it is not enabled be default.

	Use
\begin{codeexample}[code only]
\usepackage{pgfplots}
\pgfplotsset{compat=1.3}
\end{codeexample}
	in your preamble to benefit from the improved spacing\footnote{The |compat=1.3| setting is necessary for new features which move axis labels around like reversed axis. However, \PGFPlots\ will still work as it does in earlier versions, your documents will remain the same if you don't set it explicitly.}.
	Take a look at the next page for the precise definition of |compat|.

	\item \PGFPlots\ 1.3 now supports reversed axes. It is no longer necessary to use work--arounds with negative units.
\pgfkeys{/pdflinks/search key prefixes in/.add={/pgfplots/,}{}}

	Take a look at the |x dir=reverse| key.

	Existing work--arounds will still function properly. Use |\pgfplotsset{compat=1.3}| together with |x dir=reverse| to switch to the new version.
\end{enumerate}

\subsubsection{Old Features Which May Need Attention}
\begin{enumerate}
	\item The |scatter/classes| feature produces proper legends as of version 1.3. This may change the appearance of existing legends of plots with |scatter/classes|.

	\item Due to a small math bug in \PGF\ $2.00$, you \emph{can't} use the math expression `|-x^2|'. It is necessary to use `|0-x^2|' instead. The same holds for
		`|exp(-x^2)|'; use `|exp(0-x^2)|' instead.

		This will be fixed with \PGF\ $>2.00$.

	\item Starting with \PGFPlots\ $1.1$, |\tikzstyle| should \emph{no longer be used} to set \PGFPlots\ options.
	
	Although |\tikzstyle| is still supported for some older \PGFPlots\ options, you should replace any occurance of |\tikzstyle| with |\pgfplotsset{|\meta{style name}|/.style={|\meta{key-value-list}|}}| or the associated |/.append style| variant. See section~\ref{sec:styles} for more detail.
\end{enumerate}
I apologize for any inconvenience caused by these changes.

\begin{pgfplotskey}{compat=\mchoice{1.3,pre 1.3,default,newest} (initially default)}
	The preamble configuration 
\begin{codeexample}[code only]
\usepackage{pgfplots}
\pgfplotsset{compat=1.3}
\end{codeexample}
	allows to choose between backwards compatibility and most recent features.

	\PGFPlots\ version 1.3 comes with new features which may lead to different spacings compared to earlier versions:
	\begin{enumerate}
		\item Version 1.3 comes with the |xlabel near ticks| feature which places (in this case $x$) axis labels ``near'' the tick labels, taking the size of tick labels into account.
		
		Older versions used |xlabel absolute|, i.e.\ an absolute distance between the axis and axis labels (now matter how large or small tick labels are).

		The initial setting \emph{keeps backwards compatibility}.

		You are encouraged to use
\begin{codeexample}[code only]
\usepackage{pgfplots}
\pgfplotsset{compat=1.3}
\end{codeexample}
		in your preamble.
		
		\item Version 1.3 fixes a bug with |anchor=south west| which occurred mainly for reversed axes: the anchors where upside--down. This is fixed now.

		
		If you have written some work--around and want to keep it going, use

		|\pgfplotsset{compat/anchors=pre 1.3}|\pgfmanualpdflabel{/pgfplots/compat/anchors}{} 

		to disable the bug fix.
	\end{enumerate}

	Use |\pgfplotsset{compat=default}| to restore the factory settings.

	The setting |\pgfplotsset{compat=newest}| will always use features of the most recent version. This might result in small changes in the document's appearance.
\end{pgfplotskey}

\subsection{The Team}
\PGFPlots\ has been written mainly by Christian Feuers�nger with many improvements of Pascal Wolkotte and Nick Papior Andersen as a spare time project. We hope it is useful and provides valuable plots.

If you are interested in writing something but don't know how, consider reading the auxiliary manual \href{file:TeX-programming-notes.pdf}{TeX-programming-notes.pdf} which comes with \PGFPlots. It is far from complete, but maybe it is a good starting point (at least for more literature).

\subsection{Acknowledgements}
I thank God for all hours of enjoyed programming. I thank Pascal Wolkotte and Nick Papior Andersen for their programming efforts and contributions as part of the development team. I thank Stefan Tibus, who contributed the |plot shell| feature. I thank Tom Cashman for the contribution of the |reverse legend| feature. Special thanks go to Stefan Pinnow whose tests of \PGFPlots\ lead to numerous quality improvements. Furthermore, I thank Dr.~Schweitzer for many fruitful discussions and Dr.~Meine for his ideas and suggestions. Thanks as well to the many international contributors who provided feature requests or identified bugs or simply manual improvements!

Last but not least, I thank Till Tantau and Mark Wibrow for their excellent graphics (and more) package \PGF\ and \Tikz\ which is the base of \PGFPlots.

\include{pgfplots.install}%
\include{pgfplots.intro}%
\include{pgfplots.reference}%
\include{pgfplots.libs}%
\include{pgfplots.resources}%
\include{pgfplots.importexport}%
\include{pgfplots.basic.reference}%

\printindex

\bibliographystyle{abbrv} %gerapali} %gerabbrv} %gerunsrt.bst} %gerabbrv}% gerplain}
\nocite{pgfplotstable}
\nocite{programmingnotes}
\bibliography{pgfplots}
\end{document}
%
%%%%%%%%%%%%%%%%%%%%%%%%%%%%%%%%%%%%%%%%%%%%%%%%%%%%%%%%%%%%%%%%%%%%%%%%%%%%%
%
% Package pgfplots.sty documentation. 
%
% Copyright 2007/2008 by Christian Feuersaenger.
%
% This program is free software: you can redistribute it and/or modify
% it under the terms of the GNU General Public License as published by
% the Free Software Foundation, either version 3 of the License, or
% (at your option) any later version.
% 
% This program is distributed in the hope that it will be useful,
% but WITHOUT ANY WARRANTY; without even the implied warranty of
% MERCHANTABILITY or FITNESS FOR A PARTICULAR PURPOSE.  See the
% GNU General Public License for more details.
% 
% You should have received a copy of the GNU General Public License
% along with this program.  If not, see <http://www.gnu.org/licenses/>.
%
%
%%%%%%%%%%%%%%%%%%%%%%%%%%%%%%%%%%%%%%%%%%%%%%%%%%%%%%%%%%%%%%%%%%%%%%%%%%%%%
\input{pgfplots.preamble.tex}
%\RequirePackage[german,english,francais]{babel}

\pgfplotsmanualexternalexpensivetrue

%\usetikzlibrary{external}
%\tikzexternalize[prefix=figures/]{pgfplots}

\title{%
	Manual for Package \PGFPlots\\
	{\small Version \pgfplotsversion}\\
	{\small\href{http://sourceforge.net/projects/pgfplots}{http://sourceforge.net/projects/pgfplots}}}

%\includeonly{pgfplots.libs}


\begin{document}

\def\plotcoords{%
\addplot coordinates {
(5,8.312e-02)    (17,2.547e-02)   (49,7.407e-03)
(129,2.102e-03)  (321,5.874e-04)  (769,1.623e-04)
(1793,4.442e-05) (4097,1.207e-05) (9217,3.261e-06)
};

\addplot coordinates{
(7,8.472e-02)    (31,3.044e-02)    (111,1.022e-02)
(351,3.303e-03)  (1023,1.039e-03)  (2815,3.196e-04)
(7423,9.658e-05) (18943,2.873e-05) (47103,8.437e-06)
};

\addplot coordinates{
(9,7.881e-02)     (49,3.243e-02)    (209,1.232e-02)
(769,4.454e-03)   (2561,1.551e-03)  (7937,5.236e-04)
(23297,1.723e-04) (65537,5.545e-05) (178177,1.751e-05)
};

\addplot coordinates{
(11,6.887e-02)    (71,3.177e-02)     (351,1.341e-02)
(1471,5.334e-03)  (5503,2.027e-03)   (18943,7.415e-04)
(61183,2.628e-04) (187903,9.063e-05) (553983,3.053e-05)
};

\addplot coordinates{
(13,5.755e-02)     (97,2.925e-02)     (545,1.351e-02)
(2561,5.842e-03)   (10625,2.397e-03)  (40193,9.414e-04)
(141569,3.564e-04) (471041,1.308e-04) 
(1496065,4.670e-05)
};
}%
\maketitle
\begin{abstract}%
\PGFPlots\ draws high--quality function plots in normal or logarithmic scaling with a user-friendly interface directly in \TeX. The user supplies axis labels, legend entries and the plot coordinates for one or more plots and \PGFPlots\ applies axis scaling, computes any logarithms and axis ticks and draws the plots, supporting line plots, scatter plots, piecewise constant plots, bar plots, area plots, mesh-- and surface plots and some more. It is based on Till Tantau's package \PGF/\Tikz.
\end{abstract}
\tableofcontents
\section{Introduction}
This package provides tools to generate plots and labeled axes easily. It draws normal plots, logplots and semi-logplots, in two and three dimensions. Axis ticks, labels, legends (in case of multiple plots) can be added with key-value options. It can cycle through a set of predefined line/marker/color specifications. In summary, its purpose is to simplify the generation of high-quality function and/or data plots, and solving the problems of
\begin{itemize}
	\item consistency of document and font type and font size,
	\item direct use of \TeX\ math mode in axis descriptions,
	\item consistency of data and figures (no third party tool necessary),
	\item inter document consistency using preamble configurations and styles.
\end{itemize}
Although not necessary, separate |.pdf| or |.eps| graphics can be generated using the |external| library developed as part of \Tikz.

\PGFPlots\ is build completely on \Tikz/\PGF. Knowledge of \Tikz\ will simplify the work with \PGFPlots, although it is not required.

\section[About PGFPlots: Preliminaries]{About {\normalfont\PGFPlots}: Preliminaries}
This section contains information about upgrades, the team, the installation (in case you need to do it manually) and troubleshooting. You may skip it completely except for the upgrade remarks.

However, note that this library requires at least \PGF\ version $2.00$. At the time of this writing, many \TeX-distributions still contain the older \PGF\ version $1.18$, so it may be necessary to install a recent \PGF\ prior to using \PGFPlots.

\subsection{Components}
\PGFPlots\ comes with two components:
\begin{enumerate}
	\item the plotting component (which you are currently reading) and
	\item the \PGFPlotstable\ component which simplifies number formatting and postprocessing of numerical tables. It comes as a separate package and has its own manual \href{file:pgfplotstable.pdf}{pgfplotstable.pdf}.
\end{enumerate}

\subsection{Upgrade remarks}
This release provides a lot of improvements which can be found in all detail in \texttt{ChangeLog} for interested readers. However, some attention is useful with respect to the following changes.

\subsubsection{New Optional Features}
\begin{enumerate}
	\item \PGFPlots\ 1.3 comes with user interface improvements. The technical distinction between ``behavior options'' and ``style options'' of older versions is no longer necessary (although still fully supported).

	\item \PGFPlots\ 1.3 has a new feature which allows to \emph{move axis labels tight to tick labels} automatically.

	Since this affects the spacing, it is not enabled be default.

	Use
\begin{codeexample}[code only]
\usepackage{pgfplots}
\pgfplotsset{compat=1.3}
\end{codeexample}
	in your preamble to benefit from the improved spacing\footnote{The |compat=1.3| setting is necessary for new features which move axis labels around like reversed axis. However, \PGFPlots\ will still work as it does in earlier versions, your documents will remain the same if you don't set it explicitly.}.
	Take a look at the next page for the precise definition of |compat|.

	\item \PGFPlots\ 1.3 now supports reversed axes. It is no longer necessary to use work--arounds with negative units.
\pgfkeys{/pdflinks/search key prefixes in/.add={/pgfplots/,}{}}

	Take a look at the |x dir=reverse| key.

	Existing work--arounds will still function properly. Use |\pgfplotsset{compat=1.3}| together with |x dir=reverse| to switch to the new version.
\end{enumerate}

\subsubsection{Old Features Which May Need Attention}
\begin{enumerate}
	\item The |scatter/classes| feature produces proper legends as of version 1.3. This may change the appearance of existing legends of plots with |scatter/classes|.

	\item Due to a small math bug in \PGF\ $2.00$, you \emph{can't} use the math expression `|-x^2|'. It is necessary to use `|0-x^2|' instead. The same holds for
		`|exp(-x^2)|'; use `|exp(0-x^2)|' instead.

		This will be fixed with \PGF\ $>2.00$.

	\item Starting with \PGFPlots\ $1.1$, |\tikzstyle| should \emph{no longer be used} to set \PGFPlots\ options.
	
	Although |\tikzstyle| is still supported for some older \PGFPlots\ options, you should replace any occurance of |\tikzstyle| with |\pgfplotsset{|\meta{style name}|/.style={|\meta{key-value-list}|}}| or the associated |/.append style| variant. See section~\ref{sec:styles} for more detail.
\end{enumerate}
I apologize for any inconvenience caused by these changes.

\begin{pgfplotskey}{compat=\mchoice{1.3,pre 1.3,default,newest} (initially default)}
	The preamble configuration 
\begin{codeexample}[code only]
\usepackage{pgfplots}
\pgfplotsset{compat=1.3}
\end{codeexample}
	allows to choose between backwards compatibility and most recent features.

	\PGFPlots\ version 1.3 comes with new features which may lead to different spacings compared to earlier versions:
	\begin{enumerate}
		\item Version 1.3 comes with the |xlabel near ticks| feature which places (in this case $x$) axis labels ``near'' the tick labels, taking the size of tick labels into account.
		
		Older versions used |xlabel absolute|, i.e.\ an absolute distance between the axis and axis labels (now matter how large or small tick labels are).

		The initial setting \emph{keeps backwards compatibility}.

		You are encouraged to use
\begin{codeexample}[code only]
\usepackage{pgfplots}
\pgfplotsset{compat=1.3}
\end{codeexample}
		in your preamble.
		
		\item Version 1.3 fixes a bug with |anchor=south west| which occurred mainly for reversed axes: the anchors where upside--down. This is fixed now.

		
		If you have written some work--around and want to keep it going, use

		|\pgfplotsset{compat/anchors=pre 1.3}|\pgfmanualpdflabel{/pgfplots/compat/anchors}{} 

		to disable the bug fix.
	\end{enumerate}

	Use |\pgfplotsset{compat=default}| to restore the factory settings.

	The setting |\pgfplotsset{compat=newest}| will always use features of the most recent version. This might result in small changes in the document's appearance.
\end{pgfplotskey}

\subsection{The Team}
\PGFPlots\ has been written mainly by Christian Feuers�nger with many improvements of Pascal Wolkotte and Nick Papior Andersen as a spare time project. We hope it is useful and provides valuable plots.

If you are interested in writing something but don't know how, consider reading the auxiliary manual \href{file:TeX-programming-notes.pdf}{TeX-programming-notes.pdf} which comes with \PGFPlots. It is far from complete, but maybe it is a good starting point (at least for more literature).

\subsection{Acknowledgements}
I thank God for all hours of enjoyed programming. I thank Pascal Wolkotte and Nick Papior Andersen for their programming efforts and contributions as part of the development team. I thank Stefan Tibus, who contributed the |plot shell| feature. I thank Tom Cashman for the contribution of the |reverse legend| feature. Special thanks go to Stefan Pinnow whose tests of \PGFPlots\ lead to numerous quality improvements. Furthermore, I thank Dr.~Schweitzer for many fruitful discussions and Dr.~Meine for his ideas and suggestions. Thanks as well to the many international contributors who provided feature requests or identified bugs or simply manual improvements!

Last but not least, I thank Till Tantau and Mark Wibrow for their excellent graphics (and more) package \PGF\ and \Tikz\ which is the base of \PGFPlots.

\include{pgfplots.install}%
\include{pgfplots.intro}%
\include{pgfplots.reference}%
\include{pgfplots.libs}%
\include{pgfplots.resources}%
\include{pgfplots.importexport}%
\include{pgfplots.basic.reference}%

\printindex

\bibliographystyle{abbrv} %gerapali} %gerabbrv} %gerunsrt.bst} %gerabbrv}% gerplain}
\nocite{pgfplotstable}
\nocite{programmingnotes}
\bibliography{pgfplots}
\end{document}
%
%%%%%%%%%%%%%%%%%%%%%%%%%%%%%%%%%%%%%%%%%%%%%%%%%%%%%%%%%%%%%%%%%%%%%%%%%%%%%
%
% Package pgfplots.sty documentation. 
%
% Copyright 2007/2008 by Christian Feuersaenger.
%
% This program is free software: you can redistribute it and/or modify
% it under the terms of the GNU General Public License as published by
% the Free Software Foundation, either version 3 of the License, or
% (at your option) any later version.
% 
% This program is distributed in the hope that it will be useful,
% but WITHOUT ANY WARRANTY; without even the implied warranty of
% MERCHANTABILITY or FITNESS FOR A PARTICULAR PURPOSE.  See the
% GNU General Public License for more details.
% 
% You should have received a copy of the GNU General Public License
% along with this program.  If not, see <http://www.gnu.org/licenses/>.
%
%
%%%%%%%%%%%%%%%%%%%%%%%%%%%%%%%%%%%%%%%%%%%%%%%%%%%%%%%%%%%%%%%%%%%%%%%%%%%%%
\input{pgfplots.preamble.tex}
%\RequirePackage[german,english,francais]{babel}

\pgfplotsmanualexternalexpensivetrue

%\usetikzlibrary{external}
%\tikzexternalize[prefix=figures/]{pgfplots}

\title{%
	Manual for Package \PGFPlots\\
	{\small Version \pgfplotsversion}\\
	{\small\href{http://sourceforge.net/projects/pgfplots}{http://sourceforge.net/projects/pgfplots}}}

%\includeonly{pgfplots.libs}


\begin{document}

\def\plotcoords{%
\addplot coordinates {
(5,8.312e-02)    (17,2.547e-02)   (49,7.407e-03)
(129,2.102e-03)  (321,5.874e-04)  (769,1.623e-04)
(1793,4.442e-05) (4097,1.207e-05) (9217,3.261e-06)
};

\addplot coordinates{
(7,8.472e-02)    (31,3.044e-02)    (111,1.022e-02)
(351,3.303e-03)  (1023,1.039e-03)  (2815,3.196e-04)
(7423,9.658e-05) (18943,2.873e-05) (47103,8.437e-06)
};

\addplot coordinates{
(9,7.881e-02)     (49,3.243e-02)    (209,1.232e-02)
(769,4.454e-03)   (2561,1.551e-03)  (7937,5.236e-04)
(23297,1.723e-04) (65537,5.545e-05) (178177,1.751e-05)
};

\addplot coordinates{
(11,6.887e-02)    (71,3.177e-02)     (351,1.341e-02)
(1471,5.334e-03)  (5503,2.027e-03)   (18943,7.415e-04)
(61183,2.628e-04) (187903,9.063e-05) (553983,3.053e-05)
};

\addplot coordinates{
(13,5.755e-02)     (97,2.925e-02)     (545,1.351e-02)
(2561,5.842e-03)   (10625,2.397e-03)  (40193,9.414e-04)
(141569,3.564e-04) (471041,1.308e-04) 
(1496065,4.670e-05)
};
}%
\maketitle
\begin{abstract}%
\PGFPlots\ draws high--quality function plots in normal or logarithmic scaling with a user-friendly interface directly in \TeX. The user supplies axis labels, legend entries and the plot coordinates for one or more plots and \PGFPlots\ applies axis scaling, computes any logarithms and axis ticks and draws the plots, supporting line plots, scatter plots, piecewise constant plots, bar plots, area plots, mesh-- and surface plots and some more. It is based on Till Tantau's package \PGF/\Tikz.
\end{abstract}
\tableofcontents
\section{Introduction}
This package provides tools to generate plots and labeled axes easily. It draws normal plots, logplots and semi-logplots, in two and three dimensions. Axis ticks, labels, legends (in case of multiple plots) can be added with key-value options. It can cycle through a set of predefined line/marker/color specifications. In summary, its purpose is to simplify the generation of high-quality function and/or data plots, and solving the problems of
\begin{itemize}
	\item consistency of document and font type and font size,
	\item direct use of \TeX\ math mode in axis descriptions,
	\item consistency of data and figures (no third party tool necessary),
	\item inter document consistency using preamble configurations and styles.
\end{itemize}
Although not necessary, separate |.pdf| or |.eps| graphics can be generated using the |external| library developed as part of \Tikz.

\PGFPlots\ is build completely on \Tikz/\PGF. Knowledge of \Tikz\ will simplify the work with \PGFPlots, although it is not required.

\section[About PGFPlots: Preliminaries]{About {\normalfont\PGFPlots}: Preliminaries}
This section contains information about upgrades, the team, the installation (in case you need to do it manually) and troubleshooting. You may skip it completely except for the upgrade remarks.

However, note that this library requires at least \PGF\ version $2.00$. At the time of this writing, many \TeX-distributions still contain the older \PGF\ version $1.18$, so it may be necessary to install a recent \PGF\ prior to using \PGFPlots.

\subsection{Components}
\PGFPlots\ comes with two components:
\begin{enumerate}
	\item the plotting component (which you are currently reading) and
	\item the \PGFPlotstable\ component which simplifies number formatting and postprocessing of numerical tables. It comes as a separate package and has its own manual \href{file:pgfplotstable.pdf}{pgfplotstable.pdf}.
\end{enumerate}

\subsection{Upgrade remarks}
This release provides a lot of improvements which can be found in all detail in \texttt{ChangeLog} for interested readers. However, some attention is useful with respect to the following changes.

\subsubsection{New Optional Features}
\begin{enumerate}
	\item \PGFPlots\ 1.3 comes with user interface improvements. The technical distinction between ``behavior options'' and ``style options'' of older versions is no longer necessary (although still fully supported).

	\item \PGFPlots\ 1.3 has a new feature which allows to \emph{move axis labels tight to tick labels} automatically.

	Since this affects the spacing, it is not enabled be default.

	Use
\begin{codeexample}[code only]
\usepackage{pgfplots}
\pgfplotsset{compat=1.3}
\end{codeexample}
	in your preamble to benefit from the improved spacing\footnote{The |compat=1.3| setting is necessary for new features which move axis labels around like reversed axis. However, \PGFPlots\ will still work as it does in earlier versions, your documents will remain the same if you don't set it explicitly.}.
	Take a look at the next page for the precise definition of |compat|.

	\item \PGFPlots\ 1.3 now supports reversed axes. It is no longer necessary to use work--arounds with negative units.
\pgfkeys{/pdflinks/search key prefixes in/.add={/pgfplots/,}{}}

	Take a look at the |x dir=reverse| key.

	Existing work--arounds will still function properly. Use |\pgfplotsset{compat=1.3}| together with |x dir=reverse| to switch to the new version.
\end{enumerate}

\subsubsection{Old Features Which May Need Attention}
\begin{enumerate}
	\item The |scatter/classes| feature produces proper legends as of version 1.3. This may change the appearance of existing legends of plots with |scatter/classes|.

	\item Due to a small math bug in \PGF\ $2.00$, you \emph{can't} use the math expression `|-x^2|'. It is necessary to use `|0-x^2|' instead. The same holds for
		`|exp(-x^2)|'; use `|exp(0-x^2)|' instead.

		This will be fixed with \PGF\ $>2.00$.

	\item Starting with \PGFPlots\ $1.1$, |\tikzstyle| should \emph{no longer be used} to set \PGFPlots\ options.
	
	Although |\tikzstyle| is still supported for some older \PGFPlots\ options, you should replace any occurance of |\tikzstyle| with |\pgfplotsset{|\meta{style name}|/.style={|\meta{key-value-list}|}}| or the associated |/.append style| variant. See section~\ref{sec:styles} for more detail.
\end{enumerate}
I apologize for any inconvenience caused by these changes.

\begin{pgfplotskey}{compat=\mchoice{1.3,pre 1.3,default,newest} (initially default)}
	The preamble configuration 
\begin{codeexample}[code only]
\usepackage{pgfplots}
\pgfplotsset{compat=1.3}
\end{codeexample}
	allows to choose between backwards compatibility and most recent features.

	\PGFPlots\ version 1.3 comes with new features which may lead to different spacings compared to earlier versions:
	\begin{enumerate}
		\item Version 1.3 comes with the |xlabel near ticks| feature which places (in this case $x$) axis labels ``near'' the tick labels, taking the size of tick labels into account.
		
		Older versions used |xlabel absolute|, i.e.\ an absolute distance between the axis and axis labels (now matter how large or small tick labels are).

		The initial setting \emph{keeps backwards compatibility}.

		You are encouraged to use
\begin{codeexample}[code only]
\usepackage{pgfplots}
\pgfplotsset{compat=1.3}
\end{codeexample}
		in your preamble.
		
		\item Version 1.3 fixes a bug with |anchor=south west| which occurred mainly for reversed axes: the anchors where upside--down. This is fixed now.

		
		If you have written some work--around and want to keep it going, use

		|\pgfplotsset{compat/anchors=pre 1.3}|\pgfmanualpdflabel{/pgfplots/compat/anchors}{} 

		to disable the bug fix.
	\end{enumerate}

	Use |\pgfplotsset{compat=default}| to restore the factory settings.

	The setting |\pgfplotsset{compat=newest}| will always use features of the most recent version. This might result in small changes in the document's appearance.
\end{pgfplotskey}

\subsection{The Team}
\PGFPlots\ has been written mainly by Christian Feuers�nger with many improvements of Pascal Wolkotte and Nick Papior Andersen as a spare time project. We hope it is useful and provides valuable plots.

If you are interested in writing something but don't know how, consider reading the auxiliary manual \href{file:TeX-programming-notes.pdf}{TeX-programming-notes.pdf} which comes with \PGFPlots. It is far from complete, but maybe it is a good starting point (at least for more literature).

\subsection{Acknowledgements}
I thank God for all hours of enjoyed programming. I thank Pascal Wolkotte and Nick Papior Andersen for their programming efforts and contributions as part of the development team. I thank Stefan Tibus, who contributed the |plot shell| feature. I thank Tom Cashman for the contribution of the |reverse legend| feature. Special thanks go to Stefan Pinnow whose tests of \PGFPlots\ lead to numerous quality improvements. Furthermore, I thank Dr.~Schweitzer for many fruitful discussions and Dr.~Meine for his ideas and suggestions. Thanks as well to the many international contributors who provided feature requests or identified bugs or simply manual improvements!

Last but not least, I thank Till Tantau and Mark Wibrow for their excellent graphics (and more) package \PGF\ and \Tikz\ which is the base of \PGFPlots.

\include{pgfplots.install}%
\include{pgfplots.intro}%
\include{pgfplots.reference}%
\include{pgfplots.libs}%
\include{pgfplots.resources}%
\include{pgfplots.importexport}%
\include{pgfplots.basic.reference}%

\printindex

\bibliographystyle{abbrv} %gerapali} %gerabbrv} %gerunsrt.bst} %gerabbrv}% gerplain}
\nocite{pgfplotstable}
\nocite{programmingnotes}
\bibliography{pgfplots}
\end{document}
%
%%%%%%%%%%%%%%%%%%%%%%%%%%%%%%%%%%%%%%%%%%%%%%%%%%%%%%%%%%%%%%%%%%%%%%%%%%%%%
%
% Package pgfplots.sty documentation. 
%
% Copyright 2007/2008 by Christian Feuersaenger.
%
% This program is free software: you can redistribute it and/or modify
% it under the terms of the GNU General Public License as published by
% the Free Software Foundation, either version 3 of the License, or
% (at your option) any later version.
% 
% This program is distributed in the hope that it will be useful,
% but WITHOUT ANY WARRANTY; without even the implied warranty of
% MERCHANTABILITY or FITNESS FOR A PARTICULAR PURPOSE.  See the
% GNU General Public License for more details.
% 
% You should have received a copy of the GNU General Public License
% along with this program.  If not, see <http://www.gnu.org/licenses/>.
%
%
%%%%%%%%%%%%%%%%%%%%%%%%%%%%%%%%%%%%%%%%%%%%%%%%%%%%%%%%%%%%%%%%%%%%%%%%%%%%%
\input{pgfplots.preamble.tex}
%\RequirePackage[german,english,francais]{babel}

\pgfplotsmanualexternalexpensivetrue

%\usetikzlibrary{external}
%\tikzexternalize[prefix=figures/]{pgfplots}

\title{%
	Manual for Package \PGFPlots\\
	{\small Version \pgfplotsversion}\\
	{\small\href{http://sourceforge.net/projects/pgfplots}{http://sourceforge.net/projects/pgfplots}}}

%\includeonly{pgfplots.libs}


\begin{document}

\def\plotcoords{%
\addplot coordinates {
(5,8.312e-02)    (17,2.547e-02)   (49,7.407e-03)
(129,2.102e-03)  (321,5.874e-04)  (769,1.623e-04)
(1793,4.442e-05) (4097,1.207e-05) (9217,3.261e-06)
};

\addplot coordinates{
(7,8.472e-02)    (31,3.044e-02)    (111,1.022e-02)
(351,3.303e-03)  (1023,1.039e-03)  (2815,3.196e-04)
(7423,9.658e-05) (18943,2.873e-05) (47103,8.437e-06)
};

\addplot coordinates{
(9,7.881e-02)     (49,3.243e-02)    (209,1.232e-02)
(769,4.454e-03)   (2561,1.551e-03)  (7937,5.236e-04)
(23297,1.723e-04) (65537,5.545e-05) (178177,1.751e-05)
};

\addplot coordinates{
(11,6.887e-02)    (71,3.177e-02)     (351,1.341e-02)
(1471,5.334e-03)  (5503,2.027e-03)   (18943,7.415e-04)
(61183,2.628e-04) (187903,9.063e-05) (553983,3.053e-05)
};

\addplot coordinates{
(13,5.755e-02)     (97,2.925e-02)     (545,1.351e-02)
(2561,5.842e-03)   (10625,2.397e-03)  (40193,9.414e-04)
(141569,3.564e-04) (471041,1.308e-04) 
(1496065,4.670e-05)
};
}%
\maketitle
\begin{abstract}%
\PGFPlots\ draws high--quality function plots in normal or logarithmic scaling with a user-friendly interface directly in \TeX. The user supplies axis labels, legend entries and the plot coordinates for one or more plots and \PGFPlots\ applies axis scaling, computes any logarithms and axis ticks and draws the plots, supporting line plots, scatter plots, piecewise constant plots, bar plots, area plots, mesh-- and surface plots and some more. It is based on Till Tantau's package \PGF/\Tikz.
\end{abstract}
\tableofcontents
\section{Introduction}
This package provides tools to generate plots and labeled axes easily. It draws normal plots, logplots and semi-logplots, in two and three dimensions. Axis ticks, labels, legends (in case of multiple plots) can be added with key-value options. It can cycle through a set of predefined line/marker/color specifications. In summary, its purpose is to simplify the generation of high-quality function and/or data plots, and solving the problems of
\begin{itemize}
	\item consistency of document and font type and font size,
	\item direct use of \TeX\ math mode in axis descriptions,
	\item consistency of data and figures (no third party tool necessary),
	\item inter document consistency using preamble configurations and styles.
\end{itemize}
Although not necessary, separate |.pdf| or |.eps| graphics can be generated using the |external| library developed as part of \Tikz.

\PGFPlots\ is build completely on \Tikz/\PGF. Knowledge of \Tikz\ will simplify the work with \PGFPlots, although it is not required.

\section[About PGFPlots: Preliminaries]{About {\normalfont\PGFPlots}: Preliminaries}
This section contains information about upgrades, the team, the installation (in case you need to do it manually) and troubleshooting. You may skip it completely except for the upgrade remarks.

However, note that this library requires at least \PGF\ version $2.00$. At the time of this writing, many \TeX-distributions still contain the older \PGF\ version $1.18$, so it may be necessary to install a recent \PGF\ prior to using \PGFPlots.

\subsection{Components}
\PGFPlots\ comes with two components:
\begin{enumerate}
	\item the plotting component (which you are currently reading) and
	\item the \PGFPlotstable\ component which simplifies number formatting and postprocessing of numerical tables. It comes as a separate package and has its own manual \href{file:pgfplotstable.pdf}{pgfplotstable.pdf}.
\end{enumerate}

\subsection{Upgrade remarks}
This release provides a lot of improvements which can be found in all detail in \texttt{ChangeLog} for interested readers. However, some attention is useful with respect to the following changes.

\subsubsection{New Optional Features}
\begin{enumerate}
	\item \PGFPlots\ 1.3 comes with user interface improvements. The technical distinction between ``behavior options'' and ``style options'' of older versions is no longer necessary (although still fully supported).

	\item \PGFPlots\ 1.3 has a new feature which allows to \emph{move axis labels tight to tick labels} automatically.

	Since this affects the spacing, it is not enabled be default.

	Use
\begin{codeexample}[code only]
\usepackage{pgfplots}
\pgfplotsset{compat=1.3}
\end{codeexample}
	in your preamble to benefit from the improved spacing\footnote{The |compat=1.3| setting is necessary for new features which move axis labels around like reversed axis. However, \PGFPlots\ will still work as it does in earlier versions, your documents will remain the same if you don't set it explicitly.}.
	Take a look at the next page for the precise definition of |compat|.

	\item \PGFPlots\ 1.3 now supports reversed axes. It is no longer necessary to use work--arounds with negative units.
\pgfkeys{/pdflinks/search key prefixes in/.add={/pgfplots/,}{}}

	Take a look at the |x dir=reverse| key.

	Existing work--arounds will still function properly. Use |\pgfplotsset{compat=1.3}| together with |x dir=reverse| to switch to the new version.
\end{enumerate}

\subsubsection{Old Features Which May Need Attention}
\begin{enumerate}
	\item The |scatter/classes| feature produces proper legends as of version 1.3. This may change the appearance of existing legends of plots with |scatter/classes|.

	\item Due to a small math bug in \PGF\ $2.00$, you \emph{can't} use the math expression `|-x^2|'. It is necessary to use `|0-x^2|' instead. The same holds for
		`|exp(-x^2)|'; use `|exp(0-x^2)|' instead.

		This will be fixed with \PGF\ $>2.00$.

	\item Starting with \PGFPlots\ $1.1$, |\tikzstyle| should \emph{no longer be used} to set \PGFPlots\ options.
	
	Although |\tikzstyle| is still supported for some older \PGFPlots\ options, you should replace any occurance of |\tikzstyle| with |\pgfplotsset{|\meta{style name}|/.style={|\meta{key-value-list}|}}| or the associated |/.append style| variant. See section~\ref{sec:styles} for more detail.
\end{enumerate}
I apologize for any inconvenience caused by these changes.

\begin{pgfplotskey}{compat=\mchoice{1.3,pre 1.3,default,newest} (initially default)}
	The preamble configuration 
\begin{codeexample}[code only]
\usepackage{pgfplots}
\pgfplotsset{compat=1.3}
\end{codeexample}
	allows to choose between backwards compatibility and most recent features.

	\PGFPlots\ version 1.3 comes with new features which may lead to different spacings compared to earlier versions:
	\begin{enumerate}
		\item Version 1.3 comes with the |xlabel near ticks| feature which places (in this case $x$) axis labels ``near'' the tick labels, taking the size of tick labels into account.
		
		Older versions used |xlabel absolute|, i.e.\ an absolute distance between the axis and axis labels (now matter how large or small tick labels are).

		The initial setting \emph{keeps backwards compatibility}.

		You are encouraged to use
\begin{codeexample}[code only]
\usepackage{pgfplots}
\pgfplotsset{compat=1.3}
\end{codeexample}
		in your preamble.
		
		\item Version 1.3 fixes a bug with |anchor=south west| which occurred mainly for reversed axes: the anchors where upside--down. This is fixed now.

		
		If you have written some work--around and want to keep it going, use

		|\pgfplotsset{compat/anchors=pre 1.3}|\pgfmanualpdflabel{/pgfplots/compat/anchors}{} 

		to disable the bug fix.
	\end{enumerate}

	Use |\pgfplotsset{compat=default}| to restore the factory settings.

	The setting |\pgfplotsset{compat=newest}| will always use features of the most recent version. This might result in small changes in the document's appearance.
\end{pgfplotskey}

\subsection{The Team}
\PGFPlots\ has been written mainly by Christian Feuers�nger with many improvements of Pascal Wolkotte and Nick Papior Andersen as a spare time project. We hope it is useful and provides valuable plots.

If you are interested in writing something but don't know how, consider reading the auxiliary manual \href{file:TeX-programming-notes.pdf}{TeX-programming-notes.pdf} which comes with \PGFPlots. It is far from complete, but maybe it is a good starting point (at least for more literature).

\subsection{Acknowledgements}
I thank God for all hours of enjoyed programming. I thank Pascal Wolkotte and Nick Papior Andersen for their programming efforts and contributions as part of the development team. I thank Stefan Tibus, who contributed the |plot shell| feature. I thank Tom Cashman for the contribution of the |reverse legend| feature. Special thanks go to Stefan Pinnow whose tests of \PGFPlots\ lead to numerous quality improvements. Furthermore, I thank Dr.~Schweitzer for many fruitful discussions and Dr.~Meine for his ideas and suggestions. Thanks as well to the many international contributors who provided feature requests or identified bugs or simply manual improvements!

Last but not least, I thank Till Tantau and Mark Wibrow for their excellent graphics (and more) package \PGF\ and \Tikz\ which is the base of \PGFPlots.

\include{pgfplots.install}%
\include{pgfplots.intro}%
\include{pgfplots.reference}%
\include{pgfplots.libs}%
\include{pgfplots.resources}%
\include{pgfplots.importexport}%
\include{pgfplots.basic.reference}%

\printindex

\bibliographystyle{abbrv} %gerapali} %gerabbrv} %gerunsrt.bst} %gerabbrv}% gerplain}
\nocite{pgfplotstable}
\nocite{programmingnotes}
\bibliography{pgfplots}
\end{document}
%
\include{pgfplots.basic.reference}%

\printindex

\bibliographystyle{abbrv} %gerapali} %gerabbrv} %gerunsrt.bst} %gerabbrv}% gerplain}
\nocite{pgfplotstable}
\nocite{programmingnotes}
\bibliography{pgfplots}
\end{document}
%
%%%%%%%%%%%%%%%%%%%%%%%%%%%%%%%%%%%%%%%%%%%%%%%%%%%%%%%%%%%%%%%%%%%%%%%%%%%%%
%
% Package pgfplots.sty documentation. 
%
% Copyright 2007/2008 by Christian Feuersaenger.
%
% This program is free software: you can redistribute it and/or modify
% it under the terms of the GNU General Public License as published by
% the Free Software Foundation, either version 3 of the License, or
% (at your option) any later version.
% 
% This program is distributed in the hope that it will be useful,
% but WITHOUT ANY WARRANTY; without even the implied warranty of
% MERCHANTABILITY or FITNESS FOR A PARTICULAR PURPOSE.  See the
% GNU General Public License for more details.
% 
% You should have received a copy of the GNU General Public License
% along with this program.  If not, see <http://www.gnu.org/licenses/>.
%
%
%%%%%%%%%%%%%%%%%%%%%%%%%%%%%%%%%%%%%%%%%%%%%%%%%%%%%%%%%%%%%%%%%%%%%%%%%%%%%
%%%%%%%%%%%%%%%%%%%%%%%%%%%%%%%%%%%%%%%%%%%%%%%%%%%%%%%%%%%%%%%%%%%%%%%%%%%%%
%
% Package pgfplots.sty documentation. 
%
% Copyright 2007/2008 by Christian Feuersaenger.
%
% This program is free software: you can redistribute it and/or modify
% it under the terms of the GNU General Public License as published by
% the Free Software Foundation, either version 3 of the License, or
% (at your option) any later version.
% 
% This program is distributed in the hope that it will be useful,
% but WITHOUT ANY WARRANTY; without even the implied warranty of
% MERCHANTABILITY or FITNESS FOR A PARTICULAR PURPOSE.  See the
% GNU General Public License for more details.
% 
% You should have received a copy of the GNU General Public License
% along with this program.  If not, see <http://www.gnu.org/licenses/>.
%
%
%%%%%%%%%%%%%%%%%%%%%%%%%%%%%%%%%%%%%%%%%%%%%%%%%%%%%%%%%%%%%%%%%%%%%%%%%%%%%
\documentclass[a4paper]{ltxdoc}

\usepackage{makeidx}

% DON't let hyperref overload the format of index and glossary. 
% I want to do that on my own in the stylefiles for makeindex...
\makeatletter
\let\@old@wrindex=\@wrindex
\makeatother

\usepackage{ifpdf}
\ifpdf
	\usepackage[pdftex]{hyperref}
\else
	\usepackage[dvipdfm]{hyperref}
\fi
\hypersetup{pdfborder={0 0 0}}

\makeatletter
\let\@wrindex=\@old@wrindex
\makeatother

% Formatiere Seitennummern im Index:
\newcommand{\indexpageno}[1]{%
	{\bfseries\hyperpage{#1}}%
}


\newcommand{\R}{\mathbb{R}}
\newcommand{\N}{\mathbb{N}}
\newcommand{\Z}{\mathbb{Z}}

\long\def\COMMENTLOWLEVEL#1\ENDCOMMENT{}
\def\ENDCOMMENT{}

\usepackage{textcomp}
\usepackage{booktabs}

\usepackage{calc}
\usepackage[formats]{listings}
%\usepackage{courier} % don't use it - the '^' character can't be copy-pasted in courier

\usepackage{array}
\lstset{%
	basicstyle=\ttfamily,
	language=[LaTeX]tex, % Seems as if \lstset{language=tex} must be invoked BEFORE loading tikz!?
	tabsize=4,
	breaklines=true,
	breakindent=0pt
}

\ifpdf
	\pdfinfo {
		/Author	(Christian Feuersaenger)
	}

\else
	\def\pgfsysdriver{pgfsys-dvipdfm.def}
\fi
%\def\pgfsysdriver{pgfsys-pdftex.def}
\usepackage{pgfplots}

\ifpdf
	\usetikzlibrary{pgfplotsclickable}
\fi

%\usepackage{fp}
% ATTENTION:
% this requires pgf version NEWER than 2.00 :
%\usetikzlibrary{fixedpointarithmetic}

\usetikzlibrary{dateplot}

\usepackage[a4paper,left=2.25cm,right=2.25cm,top=2.5cm,bottom=2.5cm,nohead]{geometry}
\usepackage{amsmath,amssymb}
\usepackage{xxcolor}
\usepackage{pifont}
\usepackage[latin1]{inputenc}
\usepackage{amsmath}
\usepackage{eurosym}
\input{pgfplots-macros}

\pgfqkeys{/codeexample}{%
	every codeexample/.style={
		width=8cm,
		/pgfplots/every axis/.append style={legend style={fill=graphicbackground}}
	},
	tabsize=4,
}
\pgfplotsset{
	every axis/.append style={width=7cm},
	filter discard warning=false,
}

\usetikzlibrary{backgrounds,patterns}
% Global styles:
\tikzset{
  shape example/.style={
    color=black!30,
    draw,
    fill=yellow!30,
    line width=.5cm,
    inner xsep=2.5cm,
    inner ysep=0.5cm}
}

\newcommand{\FIXME}[1]{\textcolor{red}{(FIXME: #1)}}

% fuer endvironment 'sidewaysfigure' bspw
% \usepackage{rotating}

\newcommand\Tikz{Ti\textit kZ}
\newcommand\PGF{\textsc{pgf}}
\newcommand\PGFPlots{\textsc{pgfplots}}
\def\PGFPlotstable{\textsc{PgfplotsTable}}


\makeindex

% Fix overful hboxes automatically:
\tolerance=2000
\emergencystretch=10pt

\tikzset{prefix=gnuplot/pgfplots_} % prefix for 'plot function'

\author{%
	Christian Feuers\"anger\footnote{\url{http://wissrech.ins.uni-bonn.de/people/feuersaenger}}\\%
	Institut f\"ur Numerische Simulation\\
	Universit\"at Bonn, Germany}


%\RequirePackage[german,english,francais]{babel}

\pgfplotsmanualexternalexpensivetrue

%\usetikzlibrary{external}
%\tikzexternalize[prefix=figures/]{pgfplots}

\title{%
	Manual for Package \PGFPlots\\
	{\small Version \pgfplotsversion}\\
	{\small\href{http://sourceforge.net/projects/pgfplots}{http://sourceforge.net/projects/pgfplots}}}

%\includeonly{pgfplots.libs}


\begin{document}

\def\plotcoords{%
\addplot coordinates {
(5,8.312e-02)    (17,2.547e-02)   (49,7.407e-03)
(129,2.102e-03)  (321,5.874e-04)  (769,1.623e-04)
(1793,4.442e-05) (4097,1.207e-05) (9217,3.261e-06)
};

\addplot coordinates{
(7,8.472e-02)    (31,3.044e-02)    (111,1.022e-02)
(351,3.303e-03)  (1023,1.039e-03)  (2815,3.196e-04)
(7423,9.658e-05) (18943,2.873e-05) (47103,8.437e-06)
};

\addplot coordinates{
(9,7.881e-02)     (49,3.243e-02)    (209,1.232e-02)
(769,4.454e-03)   (2561,1.551e-03)  (7937,5.236e-04)
(23297,1.723e-04) (65537,5.545e-05) (178177,1.751e-05)
};

\addplot coordinates{
(11,6.887e-02)    (71,3.177e-02)     (351,1.341e-02)
(1471,5.334e-03)  (5503,2.027e-03)   (18943,7.415e-04)
(61183,2.628e-04) (187903,9.063e-05) (553983,3.053e-05)
};

\addplot coordinates{
(13,5.755e-02)     (97,2.925e-02)     (545,1.351e-02)
(2561,5.842e-03)   (10625,2.397e-03)  (40193,9.414e-04)
(141569,3.564e-04) (471041,1.308e-04) 
(1496065,4.670e-05)
};
}%
\maketitle
\begin{abstract}%
\PGFPlots\ draws high--quality function plots in normal or logarithmic scaling with a user-friendly interface directly in \TeX. The user supplies axis labels, legend entries and the plot coordinates for one or more plots and \PGFPlots\ applies axis scaling, computes any logarithms and axis ticks and draws the plots, supporting line plots, scatter plots, piecewise constant plots, bar plots, area plots, mesh-- and surface plots and some more. It is based on Till Tantau's package \PGF/\Tikz.
\end{abstract}
\tableofcontents
\section{Introduction}
This package provides tools to generate plots and labeled axes easily. It draws normal plots, logplots and semi-logplots, in two and three dimensions. Axis ticks, labels, legends (in case of multiple plots) can be added with key-value options. It can cycle through a set of predefined line/marker/color specifications. In summary, its purpose is to simplify the generation of high-quality function and/or data plots, and solving the problems of
\begin{itemize}
	\item consistency of document and font type and font size,
	\item direct use of \TeX\ math mode in axis descriptions,
	\item consistency of data and figures (no third party tool necessary),
	\item inter document consistency using preamble configurations and styles.
\end{itemize}
Although not necessary, separate |.pdf| or |.eps| graphics can be generated using the |external| library developed as part of \Tikz.

\PGFPlots\ is build completely on \Tikz/\PGF. Knowledge of \Tikz\ will simplify the work with \PGFPlots, although it is not required.

\section[About PGFPlots: Preliminaries]{About {\normalfont\PGFPlots}: Preliminaries}
This section contains information about upgrades, the team, the installation (in case you need to do it manually) and troubleshooting. You may skip it completely except for the upgrade remarks.

However, note that this library requires at least \PGF\ version $2.00$. At the time of this writing, many \TeX-distributions still contain the older \PGF\ version $1.18$, so it may be necessary to install a recent \PGF\ prior to using \PGFPlots.

\subsection{Components}
\PGFPlots\ comes with two components:
\begin{enumerate}
	\item the plotting component (which you are currently reading) and
	\item the \PGFPlotstable\ component which simplifies number formatting and postprocessing of numerical tables. It comes as a separate package and has its own manual \href{file:pgfplotstable.pdf}{pgfplotstable.pdf}.
\end{enumerate}

\subsection{Upgrade remarks}
This release provides a lot of improvements which can be found in all detail in \texttt{ChangeLog} for interested readers. However, some attention is useful with respect to the following changes.

\subsubsection{New Optional Features}
\begin{enumerate}
	\item \PGFPlots\ 1.3 comes with user interface improvements. The technical distinction between ``behavior options'' and ``style options'' of older versions is no longer necessary (although still fully supported).

	\item \PGFPlots\ 1.3 has a new feature which allows to \emph{move axis labels tight to tick labels} automatically.

	Since this affects the spacing, it is not enabled be default.

	Use
\begin{codeexample}[code only]
\usepackage{pgfplots}
\pgfplotsset{compat=1.3}
\end{codeexample}
	in your preamble to benefit from the improved spacing\footnote{The |compat=1.3| setting is necessary for new features which move axis labels around like reversed axis. However, \PGFPlots\ will still work as it does in earlier versions, your documents will remain the same if you don't set it explicitly.}.
	Take a look at the next page for the precise definition of |compat|.

	\item \PGFPlots\ 1.3 now supports reversed axes. It is no longer necessary to use work--arounds with negative units.
\pgfkeys{/pdflinks/search key prefixes in/.add={/pgfplots/,}{}}

	Take a look at the |x dir=reverse| key.

	Existing work--arounds will still function properly. Use |\pgfplotsset{compat=1.3}| together with |x dir=reverse| to switch to the new version.
\end{enumerate}

\subsubsection{Old Features Which May Need Attention}
\begin{enumerate}
	\item The |scatter/classes| feature produces proper legends as of version 1.3. This may change the appearance of existing legends of plots with |scatter/classes|.

	\item Due to a small math bug in \PGF\ $2.00$, you \emph{can't} use the math expression `|-x^2|'. It is necessary to use `|0-x^2|' instead. The same holds for
		`|exp(-x^2)|'; use `|exp(0-x^2)|' instead.

		This will be fixed with \PGF\ $>2.00$.

	\item Starting with \PGFPlots\ $1.1$, |\tikzstyle| should \emph{no longer be used} to set \PGFPlots\ options.
	
	Although |\tikzstyle| is still supported for some older \PGFPlots\ options, you should replace any occurance of |\tikzstyle| with |\pgfplotsset{|\meta{style name}|/.style={|\meta{key-value-list}|}}| or the associated |/.append style| variant. See section~\ref{sec:styles} for more detail.
\end{enumerate}
I apologize for any inconvenience caused by these changes.

\begin{pgfplotskey}{compat=\mchoice{1.3,pre 1.3,default,newest} (initially default)}
	The preamble configuration 
\begin{codeexample}[code only]
\usepackage{pgfplots}
\pgfplotsset{compat=1.3}
\end{codeexample}
	allows to choose between backwards compatibility and most recent features.

	\PGFPlots\ version 1.3 comes with new features which may lead to different spacings compared to earlier versions:
	\begin{enumerate}
		\item Version 1.3 comes with the |xlabel near ticks| feature which places (in this case $x$) axis labels ``near'' the tick labels, taking the size of tick labels into account.
		
		Older versions used |xlabel absolute|, i.e.\ an absolute distance between the axis and axis labels (now matter how large or small tick labels are).

		The initial setting \emph{keeps backwards compatibility}.

		You are encouraged to use
\begin{codeexample}[code only]
\usepackage{pgfplots}
\pgfplotsset{compat=1.3}
\end{codeexample}
		in your preamble.
		
		\item Version 1.3 fixes a bug with |anchor=south west| which occurred mainly for reversed axes: the anchors where upside--down. This is fixed now.

		
		If you have written some work--around and want to keep it going, use

		|\pgfplotsset{compat/anchors=pre 1.3}|\pgfmanualpdflabel{/pgfplots/compat/anchors}{} 

		to disable the bug fix.
	\end{enumerate}

	Use |\pgfplotsset{compat=default}| to restore the factory settings.

	The setting |\pgfplotsset{compat=newest}| will always use features of the most recent version. This might result in small changes in the document's appearance.
\end{pgfplotskey}

\subsection{The Team}
\PGFPlots\ has been written mainly by Christian Feuers�nger with many improvements of Pascal Wolkotte and Nick Papior Andersen as a spare time project. We hope it is useful and provides valuable plots.

If you are interested in writing something but don't know how, consider reading the auxiliary manual \href{file:TeX-programming-notes.pdf}{TeX-programming-notes.pdf} which comes with \PGFPlots. It is far from complete, but maybe it is a good starting point (at least for more literature).

\subsection{Acknowledgements}
I thank God for all hours of enjoyed programming. I thank Pascal Wolkotte and Nick Papior Andersen for their programming efforts and contributions as part of the development team. I thank Stefan Tibus, who contributed the |plot shell| feature. I thank Tom Cashman for the contribution of the |reverse legend| feature. Special thanks go to Stefan Pinnow whose tests of \PGFPlots\ lead to numerous quality improvements. Furthermore, I thank Dr.~Schweitzer for many fruitful discussions and Dr.~Meine for his ideas and suggestions. Thanks as well to the many international contributors who provided feature requests or identified bugs or simply manual improvements!

Last but not least, I thank Till Tantau and Mark Wibrow for their excellent graphics (and more) package \PGF\ and \Tikz\ which is the base of \PGFPlots.

%%%%%%%%%%%%%%%%%%%%%%%%%%%%%%%%%%%%%%%%%%%%%%%%%%%%%%%%%%%%%%%%%%%%%%%%%%%%%
%
% Package pgfplots.sty documentation. 
%
% Copyright 2007/2008 by Christian Feuersaenger.
%
% This program is free software: you can redistribute it and/or modify
% it under the terms of the GNU General Public License as published by
% the Free Software Foundation, either version 3 of the License, or
% (at your option) any later version.
% 
% This program is distributed in the hope that it will be useful,
% but WITHOUT ANY WARRANTY; without even the implied warranty of
% MERCHANTABILITY or FITNESS FOR A PARTICULAR PURPOSE.  See the
% GNU General Public License for more details.
% 
% You should have received a copy of the GNU General Public License
% along with this program.  If not, see <http://www.gnu.org/licenses/>.
%
%
%%%%%%%%%%%%%%%%%%%%%%%%%%%%%%%%%%%%%%%%%%%%%%%%%%%%%%%%%%%%%%%%%%%%%%%%%%%%%
\input{pgfplots.preamble.tex}
%\RequirePackage[german,english,francais]{babel}

\pgfplotsmanualexternalexpensivetrue

%\usetikzlibrary{external}
%\tikzexternalize[prefix=figures/]{pgfplots}

\title{%
	Manual for Package \PGFPlots\\
	{\small Version \pgfplotsversion}\\
	{\small\href{http://sourceforge.net/projects/pgfplots}{http://sourceforge.net/projects/pgfplots}}}

%\includeonly{pgfplots.libs}


\begin{document}

\def\plotcoords{%
\addplot coordinates {
(5,8.312e-02)    (17,2.547e-02)   (49,7.407e-03)
(129,2.102e-03)  (321,5.874e-04)  (769,1.623e-04)
(1793,4.442e-05) (4097,1.207e-05) (9217,3.261e-06)
};

\addplot coordinates{
(7,8.472e-02)    (31,3.044e-02)    (111,1.022e-02)
(351,3.303e-03)  (1023,1.039e-03)  (2815,3.196e-04)
(7423,9.658e-05) (18943,2.873e-05) (47103,8.437e-06)
};

\addplot coordinates{
(9,7.881e-02)     (49,3.243e-02)    (209,1.232e-02)
(769,4.454e-03)   (2561,1.551e-03)  (7937,5.236e-04)
(23297,1.723e-04) (65537,5.545e-05) (178177,1.751e-05)
};

\addplot coordinates{
(11,6.887e-02)    (71,3.177e-02)     (351,1.341e-02)
(1471,5.334e-03)  (5503,2.027e-03)   (18943,7.415e-04)
(61183,2.628e-04) (187903,9.063e-05) (553983,3.053e-05)
};

\addplot coordinates{
(13,5.755e-02)     (97,2.925e-02)     (545,1.351e-02)
(2561,5.842e-03)   (10625,2.397e-03)  (40193,9.414e-04)
(141569,3.564e-04) (471041,1.308e-04) 
(1496065,4.670e-05)
};
}%
\maketitle
\begin{abstract}%
\PGFPlots\ draws high--quality function plots in normal or logarithmic scaling with a user-friendly interface directly in \TeX. The user supplies axis labels, legend entries and the plot coordinates for one or more plots and \PGFPlots\ applies axis scaling, computes any logarithms and axis ticks and draws the plots, supporting line plots, scatter plots, piecewise constant plots, bar plots, area plots, mesh-- and surface plots and some more. It is based on Till Tantau's package \PGF/\Tikz.
\end{abstract}
\tableofcontents
\section{Introduction}
This package provides tools to generate plots and labeled axes easily. It draws normal plots, logplots and semi-logplots, in two and three dimensions. Axis ticks, labels, legends (in case of multiple plots) can be added with key-value options. It can cycle through a set of predefined line/marker/color specifications. In summary, its purpose is to simplify the generation of high-quality function and/or data plots, and solving the problems of
\begin{itemize}
	\item consistency of document and font type and font size,
	\item direct use of \TeX\ math mode in axis descriptions,
	\item consistency of data and figures (no third party tool necessary),
	\item inter document consistency using preamble configurations and styles.
\end{itemize}
Although not necessary, separate |.pdf| or |.eps| graphics can be generated using the |external| library developed as part of \Tikz.

\PGFPlots\ is build completely on \Tikz/\PGF. Knowledge of \Tikz\ will simplify the work with \PGFPlots, although it is not required.

\section[About PGFPlots: Preliminaries]{About {\normalfont\PGFPlots}: Preliminaries}
This section contains information about upgrades, the team, the installation (in case you need to do it manually) and troubleshooting. You may skip it completely except for the upgrade remarks.

However, note that this library requires at least \PGF\ version $2.00$. At the time of this writing, many \TeX-distributions still contain the older \PGF\ version $1.18$, so it may be necessary to install a recent \PGF\ prior to using \PGFPlots.

\subsection{Components}
\PGFPlots\ comes with two components:
\begin{enumerate}
	\item the plotting component (which you are currently reading) and
	\item the \PGFPlotstable\ component which simplifies number formatting and postprocessing of numerical tables. It comes as a separate package and has its own manual \href{file:pgfplotstable.pdf}{pgfplotstable.pdf}.
\end{enumerate}

\subsection{Upgrade remarks}
This release provides a lot of improvements which can be found in all detail in \texttt{ChangeLog} for interested readers. However, some attention is useful with respect to the following changes.

\subsubsection{New Optional Features}
\begin{enumerate}
	\item \PGFPlots\ 1.3 comes with user interface improvements. The technical distinction between ``behavior options'' and ``style options'' of older versions is no longer necessary (although still fully supported).

	\item \PGFPlots\ 1.3 has a new feature which allows to \emph{move axis labels tight to tick labels} automatically.

	Since this affects the spacing, it is not enabled be default.

	Use
\begin{codeexample}[code only]
\usepackage{pgfplots}
\pgfplotsset{compat=1.3}
\end{codeexample}
	in your preamble to benefit from the improved spacing\footnote{The |compat=1.3| setting is necessary for new features which move axis labels around like reversed axis. However, \PGFPlots\ will still work as it does in earlier versions, your documents will remain the same if you don't set it explicitly.}.
	Take a look at the next page for the precise definition of |compat|.

	\item \PGFPlots\ 1.3 now supports reversed axes. It is no longer necessary to use work--arounds with negative units.
\pgfkeys{/pdflinks/search key prefixes in/.add={/pgfplots/,}{}}

	Take a look at the |x dir=reverse| key.

	Existing work--arounds will still function properly. Use |\pgfplotsset{compat=1.3}| together with |x dir=reverse| to switch to the new version.
\end{enumerate}

\subsubsection{Old Features Which May Need Attention}
\begin{enumerate}
	\item The |scatter/classes| feature produces proper legends as of version 1.3. This may change the appearance of existing legends of plots with |scatter/classes|.

	\item Due to a small math bug in \PGF\ $2.00$, you \emph{can't} use the math expression `|-x^2|'. It is necessary to use `|0-x^2|' instead. The same holds for
		`|exp(-x^2)|'; use `|exp(0-x^2)|' instead.

		This will be fixed with \PGF\ $>2.00$.

	\item Starting with \PGFPlots\ $1.1$, |\tikzstyle| should \emph{no longer be used} to set \PGFPlots\ options.
	
	Although |\tikzstyle| is still supported for some older \PGFPlots\ options, you should replace any occurance of |\tikzstyle| with |\pgfplotsset{|\meta{style name}|/.style={|\meta{key-value-list}|}}| or the associated |/.append style| variant. See section~\ref{sec:styles} for more detail.
\end{enumerate}
I apologize for any inconvenience caused by these changes.

\begin{pgfplotskey}{compat=\mchoice{1.3,pre 1.3,default,newest} (initially default)}
	The preamble configuration 
\begin{codeexample}[code only]
\usepackage{pgfplots}
\pgfplotsset{compat=1.3}
\end{codeexample}
	allows to choose between backwards compatibility and most recent features.

	\PGFPlots\ version 1.3 comes with new features which may lead to different spacings compared to earlier versions:
	\begin{enumerate}
		\item Version 1.3 comes with the |xlabel near ticks| feature which places (in this case $x$) axis labels ``near'' the tick labels, taking the size of tick labels into account.
		
		Older versions used |xlabel absolute|, i.e.\ an absolute distance between the axis and axis labels (now matter how large or small tick labels are).

		The initial setting \emph{keeps backwards compatibility}.

		You are encouraged to use
\begin{codeexample}[code only]
\usepackage{pgfplots}
\pgfplotsset{compat=1.3}
\end{codeexample}
		in your preamble.
		
		\item Version 1.3 fixes a bug with |anchor=south west| which occurred mainly for reversed axes: the anchors where upside--down. This is fixed now.

		
		If you have written some work--around and want to keep it going, use

		|\pgfplotsset{compat/anchors=pre 1.3}|\pgfmanualpdflabel{/pgfplots/compat/anchors}{} 

		to disable the bug fix.
	\end{enumerate}

	Use |\pgfplotsset{compat=default}| to restore the factory settings.

	The setting |\pgfplotsset{compat=newest}| will always use features of the most recent version. This might result in small changes in the document's appearance.
\end{pgfplotskey}

\subsection{The Team}
\PGFPlots\ has been written mainly by Christian Feuers�nger with many improvements of Pascal Wolkotte and Nick Papior Andersen as a spare time project. We hope it is useful and provides valuable plots.

If you are interested in writing something but don't know how, consider reading the auxiliary manual \href{file:TeX-programming-notes.pdf}{TeX-programming-notes.pdf} which comes with \PGFPlots. It is far from complete, but maybe it is a good starting point (at least for more literature).

\subsection{Acknowledgements}
I thank God for all hours of enjoyed programming. I thank Pascal Wolkotte and Nick Papior Andersen for their programming efforts and contributions as part of the development team. I thank Stefan Tibus, who contributed the |plot shell| feature. I thank Tom Cashman for the contribution of the |reverse legend| feature. Special thanks go to Stefan Pinnow whose tests of \PGFPlots\ lead to numerous quality improvements. Furthermore, I thank Dr.~Schweitzer for many fruitful discussions and Dr.~Meine for his ideas and suggestions. Thanks as well to the many international contributors who provided feature requests or identified bugs or simply manual improvements!

Last but not least, I thank Till Tantau and Mark Wibrow for their excellent graphics (and more) package \PGF\ and \Tikz\ which is the base of \PGFPlots.

\include{pgfplots.install}%
\include{pgfplots.intro}%
\include{pgfplots.reference}%
\include{pgfplots.libs}%
\include{pgfplots.resources}%
\include{pgfplots.importexport}%
\include{pgfplots.basic.reference}%

\printindex

\bibliographystyle{abbrv} %gerapali} %gerabbrv} %gerunsrt.bst} %gerabbrv}% gerplain}
\nocite{pgfplotstable}
\nocite{programmingnotes}
\bibliography{pgfplots}
\end{document}
%
%%%%%%%%%%%%%%%%%%%%%%%%%%%%%%%%%%%%%%%%%%%%%%%%%%%%%%%%%%%%%%%%%%%%%%%%%%%%%
%
% Package pgfplots.sty documentation. 
%
% Copyright 2007/2008 by Christian Feuersaenger.
%
% This program is free software: you can redistribute it and/or modify
% it under the terms of the GNU General Public License as published by
% the Free Software Foundation, either version 3 of the License, or
% (at your option) any later version.
% 
% This program is distributed in the hope that it will be useful,
% but WITHOUT ANY WARRANTY; without even the implied warranty of
% MERCHANTABILITY or FITNESS FOR A PARTICULAR PURPOSE.  See the
% GNU General Public License for more details.
% 
% You should have received a copy of the GNU General Public License
% along with this program.  If not, see <http://www.gnu.org/licenses/>.
%
%
%%%%%%%%%%%%%%%%%%%%%%%%%%%%%%%%%%%%%%%%%%%%%%%%%%%%%%%%%%%%%%%%%%%%%%%%%%%%%
\input{pgfplots.preamble.tex}
%\RequirePackage[german,english,francais]{babel}

\pgfplotsmanualexternalexpensivetrue

%\usetikzlibrary{external}
%\tikzexternalize[prefix=figures/]{pgfplots}

\title{%
	Manual for Package \PGFPlots\\
	{\small Version \pgfplotsversion}\\
	{\small\href{http://sourceforge.net/projects/pgfplots}{http://sourceforge.net/projects/pgfplots}}}

%\includeonly{pgfplots.libs}


\begin{document}

\def\plotcoords{%
\addplot coordinates {
(5,8.312e-02)    (17,2.547e-02)   (49,7.407e-03)
(129,2.102e-03)  (321,5.874e-04)  (769,1.623e-04)
(1793,4.442e-05) (4097,1.207e-05) (9217,3.261e-06)
};

\addplot coordinates{
(7,8.472e-02)    (31,3.044e-02)    (111,1.022e-02)
(351,3.303e-03)  (1023,1.039e-03)  (2815,3.196e-04)
(7423,9.658e-05) (18943,2.873e-05) (47103,8.437e-06)
};

\addplot coordinates{
(9,7.881e-02)     (49,3.243e-02)    (209,1.232e-02)
(769,4.454e-03)   (2561,1.551e-03)  (7937,5.236e-04)
(23297,1.723e-04) (65537,5.545e-05) (178177,1.751e-05)
};

\addplot coordinates{
(11,6.887e-02)    (71,3.177e-02)     (351,1.341e-02)
(1471,5.334e-03)  (5503,2.027e-03)   (18943,7.415e-04)
(61183,2.628e-04) (187903,9.063e-05) (553983,3.053e-05)
};

\addplot coordinates{
(13,5.755e-02)     (97,2.925e-02)     (545,1.351e-02)
(2561,5.842e-03)   (10625,2.397e-03)  (40193,9.414e-04)
(141569,3.564e-04) (471041,1.308e-04) 
(1496065,4.670e-05)
};
}%
\maketitle
\begin{abstract}%
\PGFPlots\ draws high--quality function plots in normal or logarithmic scaling with a user-friendly interface directly in \TeX. The user supplies axis labels, legend entries and the plot coordinates for one or more plots and \PGFPlots\ applies axis scaling, computes any logarithms and axis ticks and draws the plots, supporting line plots, scatter plots, piecewise constant plots, bar plots, area plots, mesh-- and surface plots and some more. It is based on Till Tantau's package \PGF/\Tikz.
\end{abstract}
\tableofcontents
\section{Introduction}
This package provides tools to generate plots and labeled axes easily. It draws normal plots, logplots and semi-logplots, in two and three dimensions. Axis ticks, labels, legends (in case of multiple plots) can be added with key-value options. It can cycle through a set of predefined line/marker/color specifications. In summary, its purpose is to simplify the generation of high-quality function and/or data plots, and solving the problems of
\begin{itemize}
	\item consistency of document and font type and font size,
	\item direct use of \TeX\ math mode in axis descriptions,
	\item consistency of data and figures (no third party tool necessary),
	\item inter document consistency using preamble configurations and styles.
\end{itemize}
Although not necessary, separate |.pdf| or |.eps| graphics can be generated using the |external| library developed as part of \Tikz.

\PGFPlots\ is build completely on \Tikz/\PGF. Knowledge of \Tikz\ will simplify the work with \PGFPlots, although it is not required.

\section[About PGFPlots: Preliminaries]{About {\normalfont\PGFPlots}: Preliminaries}
This section contains information about upgrades, the team, the installation (in case you need to do it manually) and troubleshooting. You may skip it completely except for the upgrade remarks.

However, note that this library requires at least \PGF\ version $2.00$. At the time of this writing, many \TeX-distributions still contain the older \PGF\ version $1.18$, so it may be necessary to install a recent \PGF\ prior to using \PGFPlots.

\subsection{Components}
\PGFPlots\ comes with two components:
\begin{enumerate}
	\item the plotting component (which you are currently reading) and
	\item the \PGFPlotstable\ component which simplifies number formatting and postprocessing of numerical tables. It comes as a separate package and has its own manual \href{file:pgfplotstable.pdf}{pgfplotstable.pdf}.
\end{enumerate}

\subsection{Upgrade remarks}
This release provides a lot of improvements which can be found in all detail in \texttt{ChangeLog} for interested readers. However, some attention is useful with respect to the following changes.

\subsubsection{New Optional Features}
\begin{enumerate}
	\item \PGFPlots\ 1.3 comes with user interface improvements. The technical distinction between ``behavior options'' and ``style options'' of older versions is no longer necessary (although still fully supported).

	\item \PGFPlots\ 1.3 has a new feature which allows to \emph{move axis labels tight to tick labels} automatically.

	Since this affects the spacing, it is not enabled be default.

	Use
\begin{codeexample}[code only]
\usepackage{pgfplots}
\pgfplotsset{compat=1.3}
\end{codeexample}
	in your preamble to benefit from the improved spacing\footnote{The |compat=1.3| setting is necessary for new features which move axis labels around like reversed axis. However, \PGFPlots\ will still work as it does in earlier versions, your documents will remain the same if you don't set it explicitly.}.
	Take a look at the next page for the precise definition of |compat|.

	\item \PGFPlots\ 1.3 now supports reversed axes. It is no longer necessary to use work--arounds with negative units.
\pgfkeys{/pdflinks/search key prefixes in/.add={/pgfplots/,}{}}

	Take a look at the |x dir=reverse| key.

	Existing work--arounds will still function properly. Use |\pgfplotsset{compat=1.3}| together with |x dir=reverse| to switch to the new version.
\end{enumerate}

\subsubsection{Old Features Which May Need Attention}
\begin{enumerate}
	\item The |scatter/classes| feature produces proper legends as of version 1.3. This may change the appearance of existing legends of plots with |scatter/classes|.

	\item Due to a small math bug in \PGF\ $2.00$, you \emph{can't} use the math expression `|-x^2|'. It is necessary to use `|0-x^2|' instead. The same holds for
		`|exp(-x^2)|'; use `|exp(0-x^2)|' instead.

		This will be fixed with \PGF\ $>2.00$.

	\item Starting with \PGFPlots\ $1.1$, |\tikzstyle| should \emph{no longer be used} to set \PGFPlots\ options.
	
	Although |\tikzstyle| is still supported for some older \PGFPlots\ options, you should replace any occurance of |\tikzstyle| with |\pgfplotsset{|\meta{style name}|/.style={|\meta{key-value-list}|}}| or the associated |/.append style| variant. See section~\ref{sec:styles} for more detail.
\end{enumerate}
I apologize for any inconvenience caused by these changes.

\begin{pgfplotskey}{compat=\mchoice{1.3,pre 1.3,default,newest} (initially default)}
	The preamble configuration 
\begin{codeexample}[code only]
\usepackage{pgfplots}
\pgfplotsset{compat=1.3}
\end{codeexample}
	allows to choose between backwards compatibility and most recent features.

	\PGFPlots\ version 1.3 comes with new features which may lead to different spacings compared to earlier versions:
	\begin{enumerate}
		\item Version 1.3 comes with the |xlabel near ticks| feature which places (in this case $x$) axis labels ``near'' the tick labels, taking the size of tick labels into account.
		
		Older versions used |xlabel absolute|, i.e.\ an absolute distance between the axis and axis labels (now matter how large or small tick labels are).

		The initial setting \emph{keeps backwards compatibility}.

		You are encouraged to use
\begin{codeexample}[code only]
\usepackage{pgfplots}
\pgfplotsset{compat=1.3}
\end{codeexample}
		in your preamble.
		
		\item Version 1.3 fixes a bug with |anchor=south west| which occurred mainly for reversed axes: the anchors where upside--down. This is fixed now.

		
		If you have written some work--around and want to keep it going, use

		|\pgfplotsset{compat/anchors=pre 1.3}|\pgfmanualpdflabel{/pgfplots/compat/anchors}{} 

		to disable the bug fix.
	\end{enumerate}

	Use |\pgfplotsset{compat=default}| to restore the factory settings.

	The setting |\pgfplotsset{compat=newest}| will always use features of the most recent version. This might result in small changes in the document's appearance.
\end{pgfplotskey}

\subsection{The Team}
\PGFPlots\ has been written mainly by Christian Feuers�nger with many improvements of Pascal Wolkotte and Nick Papior Andersen as a spare time project. We hope it is useful and provides valuable plots.

If you are interested in writing something but don't know how, consider reading the auxiliary manual \href{file:TeX-programming-notes.pdf}{TeX-programming-notes.pdf} which comes with \PGFPlots. It is far from complete, but maybe it is a good starting point (at least for more literature).

\subsection{Acknowledgements}
I thank God for all hours of enjoyed programming. I thank Pascal Wolkotte and Nick Papior Andersen for their programming efforts and contributions as part of the development team. I thank Stefan Tibus, who contributed the |plot shell| feature. I thank Tom Cashman for the contribution of the |reverse legend| feature. Special thanks go to Stefan Pinnow whose tests of \PGFPlots\ lead to numerous quality improvements. Furthermore, I thank Dr.~Schweitzer for many fruitful discussions and Dr.~Meine for his ideas and suggestions. Thanks as well to the many international contributors who provided feature requests or identified bugs or simply manual improvements!

Last but not least, I thank Till Tantau and Mark Wibrow for their excellent graphics (and more) package \PGF\ and \Tikz\ which is the base of \PGFPlots.

\include{pgfplots.install}%
\include{pgfplots.intro}%
\include{pgfplots.reference}%
\include{pgfplots.libs}%
\include{pgfplots.resources}%
\include{pgfplots.importexport}%
\include{pgfplots.basic.reference}%

\printindex

\bibliographystyle{abbrv} %gerapali} %gerabbrv} %gerunsrt.bst} %gerabbrv}% gerplain}
\nocite{pgfplotstable}
\nocite{programmingnotes}
\bibliography{pgfplots}
\end{document}
%
%%%%%%%%%%%%%%%%%%%%%%%%%%%%%%%%%%%%%%%%%%%%%%%%%%%%%%%%%%%%%%%%%%%%%%%%%%%%%
%
% Package pgfplots.sty documentation. 
%
% Copyright 2007/2008 by Christian Feuersaenger.
%
% This program is free software: you can redistribute it and/or modify
% it under the terms of the GNU General Public License as published by
% the Free Software Foundation, either version 3 of the License, or
% (at your option) any later version.
% 
% This program is distributed in the hope that it will be useful,
% but WITHOUT ANY WARRANTY; without even the implied warranty of
% MERCHANTABILITY or FITNESS FOR A PARTICULAR PURPOSE.  See the
% GNU General Public License for more details.
% 
% You should have received a copy of the GNU General Public License
% along with this program.  If not, see <http://www.gnu.org/licenses/>.
%
%
%%%%%%%%%%%%%%%%%%%%%%%%%%%%%%%%%%%%%%%%%%%%%%%%%%%%%%%%%%%%%%%%%%%%%%%%%%%%%
\input{pgfplots.preamble.tex}
%\RequirePackage[german,english,francais]{babel}

\pgfplotsmanualexternalexpensivetrue

%\usetikzlibrary{external}
%\tikzexternalize[prefix=figures/]{pgfplots}

\title{%
	Manual for Package \PGFPlots\\
	{\small Version \pgfplotsversion}\\
	{\small\href{http://sourceforge.net/projects/pgfplots}{http://sourceforge.net/projects/pgfplots}}}

%\includeonly{pgfplots.libs}


\begin{document}

\def\plotcoords{%
\addplot coordinates {
(5,8.312e-02)    (17,2.547e-02)   (49,7.407e-03)
(129,2.102e-03)  (321,5.874e-04)  (769,1.623e-04)
(1793,4.442e-05) (4097,1.207e-05) (9217,3.261e-06)
};

\addplot coordinates{
(7,8.472e-02)    (31,3.044e-02)    (111,1.022e-02)
(351,3.303e-03)  (1023,1.039e-03)  (2815,3.196e-04)
(7423,9.658e-05) (18943,2.873e-05) (47103,8.437e-06)
};

\addplot coordinates{
(9,7.881e-02)     (49,3.243e-02)    (209,1.232e-02)
(769,4.454e-03)   (2561,1.551e-03)  (7937,5.236e-04)
(23297,1.723e-04) (65537,5.545e-05) (178177,1.751e-05)
};

\addplot coordinates{
(11,6.887e-02)    (71,3.177e-02)     (351,1.341e-02)
(1471,5.334e-03)  (5503,2.027e-03)   (18943,7.415e-04)
(61183,2.628e-04) (187903,9.063e-05) (553983,3.053e-05)
};

\addplot coordinates{
(13,5.755e-02)     (97,2.925e-02)     (545,1.351e-02)
(2561,5.842e-03)   (10625,2.397e-03)  (40193,9.414e-04)
(141569,3.564e-04) (471041,1.308e-04) 
(1496065,4.670e-05)
};
}%
\maketitle
\begin{abstract}%
\PGFPlots\ draws high--quality function plots in normal or logarithmic scaling with a user-friendly interface directly in \TeX. The user supplies axis labels, legend entries and the plot coordinates for one or more plots and \PGFPlots\ applies axis scaling, computes any logarithms and axis ticks and draws the plots, supporting line plots, scatter plots, piecewise constant plots, bar plots, area plots, mesh-- and surface plots and some more. It is based on Till Tantau's package \PGF/\Tikz.
\end{abstract}
\tableofcontents
\section{Introduction}
This package provides tools to generate plots and labeled axes easily. It draws normal plots, logplots and semi-logplots, in two and three dimensions. Axis ticks, labels, legends (in case of multiple plots) can be added with key-value options. It can cycle through a set of predefined line/marker/color specifications. In summary, its purpose is to simplify the generation of high-quality function and/or data plots, and solving the problems of
\begin{itemize}
	\item consistency of document and font type and font size,
	\item direct use of \TeX\ math mode in axis descriptions,
	\item consistency of data and figures (no third party tool necessary),
	\item inter document consistency using preamble configurations and styles.
\end{itemize}
Although not necessary, separate |.pdf| or |.eps| graphics can be generated using the |external| library developed as part of \Tikz.

\PGFPlots\ is build completely on \Tikz/\PGF. Knowledge of \Tikz\ will simplify the work with \PGFPlots, although it is not required.

\section[About PGFPlots: Preliminaries]{About {\normalfont\PGFPlots}: Preliminaries}
This section contains information about upgrades, the team, the installation (in case you need to do it manually) and troubleshooting. You may skip it completely except for the upgrade remarks.

However, note that this library requires at least \PGF\ version $2.00$. At the time of this writing, many \TeX-distributions still contain the older \PGF\ version $1.18$, so it may be necessary to install a recent \PGF\ prior to using \PGFPlots.

\subsection{Components}
\PGFPlots\ comes with two components:
\begin{enumerate}
	\item the plotting component (which you are currently reading) and
	\item the \PGFPlotstable\ component which simplifies number formatting and postprocessing of numerical tables. It comes as a separate package and has its own manual \href{file:pgfplotstable.pdf}{pgfplotstable.pdf}.
\end{enumerate}

\subsection{Upgrade remarks}
This release provides a lot of improvements which can be found in all detail in \texttt{ChangeLog} for interested readers. However, some attention is useful with respect to the following changes.

\subsubsection{New Optional Features}
\begin{enumerate}
	\item \PGFPlots\ 1.3 comes with user interface improvements. The technical distinction between ``behavior options'' and ``style options'' of older versions is no longer necessary (although still fully supported).

	\item \PGFPlots\ 1.3 has a new feature which allows to \emph{move axis labels tight to tick labels} automatically.

	Since this affects the spacing, it is not enabled be default.

	Use
\begin{codeexample}[code only]
\usepackage{pgfplots}
\pgfplotsset{compat=1.3}
\end{codeexample}
	in your preamble to benefit from the improved spacing\footnote{The |compat=1.3| setting is necessary for new features which move axis labels around like reversed axis. However, \PGFPlots\ will still work as it does in earlier versions, your documents will remain the same if you don't set it explicitly.}.
	Take a look at the next page for the precise definition of |compat|.

	\item \PGFPlots\ 1.3 now supports reversed axes. It is no longer necessary to use work--arounds with negative units.
\pgfkeys{/pdflinks/search key prefixes in/.add={/pgfplots/,}{}}

	Take a look at the |x dir=reverse| key.

	Existing work--arounds will still function properly. Use |\pgfplotsset{compat=1.3}| together with |x dir=reverse| to switch to the new version.
\end{enumerate}

\subsubsection{Old Features Which May Need Attention}
\begin{enumerate}
	\item The |scatter/classes| feature produces proper legends as of version 1.3. This may change the appearance of existing legends of plots with |scatter/classes|.

	\item Due to a small math bug in \PGF\ $2.00$, you \emph{can't} use the math expression `|-x^2|'. It is necessary to use `|0-x^2|' instead. The same holds for
		`|exp(-x^2)|'; use `|exp(0-x^2)|' instead.

		This will be fixed with \PGF\ $>2.00$.

	\item Starting with \PGFPlots\ $1.1$, |\tikzstyle| should \emph{no longer be used} to set \PGFPlots\ options.
	
	Although |\tikzstyle| is still supported for some older \PGFPlots\ options, you should replace any occurance of |\tikzstyle| with |\pgfplotsset{|\meta{style name}|/.style={|\meta{key-value-list}|}}| or the associated |/.append style| variant. See section~\ref{sec:styles} for more detail.
\end{enumerate}
I apologize for any inconvenience caused by these changes.

\begin{pgfplotskey}{compat=\mchoice{1.3,pre 1.3,default,newest} (initially default)}
	The preamble configuration 
\begin{codeexample}[code only]
\usepackage{pgfplots}
\pgfplotsset{compat=1.3}
\end{codeexample}
	allows to choose between backwards compatibility and most recent features.

	\PGFPlots\ version 1.3 comes with new features which may lead to different spacings compared to earlier versions:
	\begin{enumerate}
		\item Version 1.3 comes with the |xlabel near ticks| feature which places (in this case $x$) axis labels ``near'' the tick labels, taking the size of tick labels into account.
		
		Older versions used |xlabel absolute|, i.e.\ an absolute distance between the axis and axis labels (now matter how large or small tick labels are).

		The initial setting \emph{keeps backwards compatibility}.

		You are encouraged to use
\begin{codeexample}[code only]
\usepackage{pgfplots}
\pgfplotsset{compat=1.3}
\end{codeexample}
		in your preamble.
		
		\item Version 1.3 fixes a bug with |anchor=south west| which occurred mainly for reversed axes: the anchors where upside--down. This is fixed now.

		
		If you have written some work--around and want to keep it going, use

		|\pgfplotsset{compat/anchors=pre 1.3}|\pgfmanualpdflabel{/pgfplots/compat/anchors}{} 

		to disable the bug fix.
	\end{enumerate}

	Use |\pgfplotsset{compat=default}| to restore the factory settings.

	The setting |\pgfplotsset{compat=newest}| will always use features of the most recent version. This might result in small changes in the document's appearance.
\end{pgfplotskey}

\subsection{The Team}
\PGFPlots\ has been written mainly by Christian Feuers�nger with many improvements of Pascal Wolkotte and Nick Papior Andersen as a spare time project. We hope it is useful and provides valuable plots.

If you are interested in writing something but don't know how, consider reading the auxiliary manual \href{file:TeX-programming-notes.pdf}{TeX-programming-notes.pdf} which comes with \PGFPlots. It is far from complete, but maybe it is a good starting point (at least for more literature).

\subsection{Acknowledgements}
I thank God for all hours of enjoyed programming. I thank Pascal Wolkotte and Nick Papior Andersen for their programming efforts and contributions as part of the development team. I thank Stefan Tibus, who contributed the |plot shell| feature. I thank Tom Cashman for the contribution of the |reverse legend| feature. Special thanks go to Stefan Pinnow whose tests of \PGFPlots\ lead to numerous quality improvements. Furthermore, I thank Dr.~Schweitzer for many fruitful discussions and Dr.~Meine for his ideas and suggestions. Thanks as well to the many international contributors who provided feature requests or identified bugs or simply manual improvements!

Last but not least, I thank Till Tantau and Mark Wibrow for their excellent graphics (and more) package \PGF\ and \Tikz\ which is the base of \PGFPlots.

\include{pgfplots.install}%
\include{pgfplots.intro}%
\include{pgfplots.reference}%
\include{pgfplots.libs}%
\include{pgfplots.resources}%
\include{pgfplots.importexport}%
\include{pgfplots.basic.reference}%

\printindex

\bibliographystyle{abbrv} %gerapali} %gerabbrv} %gerunsrt.bst} %gerabbrv}% gerplain}
\nocite{pgfplotstable}
\nocite{programmingnotes}
\bibliography{pgfplots}
\end{document}
%
%%%%%%%%%%%%%%%%%%%%%%%%%%%%%%%%%%%%%%%%%%%%%%%%%%%%%%%%%%%%%%%%%%%%%%%%%%%%%
%
% Package pgfplots.sty documentation. 
%
% Copyright 2007/2008 by Christian Feuersaenger.
%
% This program is free software: you can redistribute it and/or modify
% it under the terms of the GNU General Public License as published by
% the Free Software Foundation, either version 3 of the License, or
% (at your option) any later version.
% 
% This program is distributed in the hope that it will be useful,
% but WITHOUT ANY WARRANTY; without even the implied warranty of
% MERCHANTABILITY or FITNESS FOR A PARTICULAR PURPOSE.  See the
% GNU General Public License for more details.
% 
% You should have received a copy of the GNU General Public License
% along with this program.  If not, see <http://www.gnu.org/licenses/>.
%
%
%%%%%%%%%%%%%%%%%%%%%%%%%%%%%%%%%%%%%%%%%%%%%%%%%%%%%%%%%%%%%%%%%%%%%%%%%%%%%
\input{pgfplots.preamble.tex}
%\RequirePackage[german,english,francais]{babel}

\pgfplotsmanualexternalexpensivetrue

%\usetikzlibrary{external}
%\tikzexternalize[prefix=figures/]{pgfplots}

\title{%
	Manual for Package \PGFPlots\\
	{\small Version \pgfplotsversion}\\
	{\small\href{http://sourceforge.net/projects/pgfplots}{http://sourceforge.net/projects/pgfplots}}}

%\includeonly{pgfplots.libs}


\begin{document}

\def\plotcoords{%
\addplot coordinates {
(5,8.312e-02)    (17,2.547e-02)   (49,7.407e-03)
(129,2.102e-03)  (321,5.874e-04)  (769,1.623e-04)
(1793,4.442e-05) (4097,1.207e-05) (9217,3.261e-06)
};

\addplot coordinates{
(7,8.472e-02)    (31,3.044e-02)    (111,1.022e-02)
(351,3.303e-03)  (1023,1.039e-03)  (2815,3.196e-04)
(7423,9.658e-05) (18943,2.873e-05) (47103,8.437e-06)
};

\addplot coordinates{
(9,7.881e-02)     (49,3.243e-02)    (209,1.232e-02)
(769,4.454e-03)   (2561,1.551e-03)  (7937,5.236e-04)
(23297,1.723e-04) (65537,5.545e-05) (178177,1.751e-05)
};

\addplot coordinates{
(11,6.887e-02)    (71,3.177e-02)     (351,1.341e-02)
(1471,5.334e-03)  (5503,2.027e-03)   (18943,7.415e-04)
(61183,2.628e-04) (187903,9.063e-05) (553983,3.053e-05)
};

\addplot coordinates{
(13,5.755e-02)     (97,2.925e-02)     (545,1.351e-02)
(2561,5.842e-03)   (10625,2.397e-03)  (40193,9.414e-04)
(141569,3.564e-04) (471041,1.308e-04) 
(1496065,4.670e-05)
};
}%
\maketitle
\begin{abstract}%
\PGFPlots\ draws high--quality function plots in normal or logarithmic scaling with a user-friendly interface directly in \TeX. The user supplies axis labels, legend entries and the plot coordinates for one or more plots and \PGFPlots\ applies axis scaling, computes any logarithms and axis ticks and draws the plots, supporting line plots, scatter plots, piecewise constant plots, bar plots, area plots, mesh-- and surface plots and some more. It is based on Till Tantau's package \PGF/\Tikz.
\end{abstract}
\tableofcontents
\section{Introduction}
This package provides tools to generate plots and labeled axes easily. It draws normal plots, logplots and semi-logplots, in two and three dimensions. Axis ticks, labels, legends (in case of multiple plots) can be added with key-value options. It can cycle through a set of predefined line/marker/color specifications. In summary, its purpose is to simplify the generation of high-quality function and/or data plots, and solving the problems of
\begin{itemize}
	\item consistency of document and font type and font size,
	\item direct use of \TeX\ math mode in axis descriptions,
	\item consistency of data and figures (no third party tool necessary),
	\item inter document consistency using preamble configurations and styles.
\end{itemize}
Although not necessary, separate |.pdf| or |.eps| graphics can be generated using the |external| library developed as part of \Tikz.

\PGFPlots\ is build completely on \Tikz/\PGF. Knowledge of \Tikz\ will simplify the work with \PGFPlots, although it is not required.

\section[About PGFPlots: Preliminaries]{About {\normalfont\PGFPlots}: Preliminaries}
This section contains information about upgrades, the team, the installation (in case you need to do it manually) and troubleshooting. You may skip it completely except for the upgrade remarks.

However, note that this library requires at least \PGF\ version $2.00$. At the time of this writing, many \TeX-distributions still contain the older \PGF\ version $1.18$, so it may be necessary to install a recent \PGF\ prior to using \PGFPlots.

\subsection{Components}
\PGFPlots\ comes with two components:
\begin{enumerate}
	\item the plotting component (which you are currently reading) and
	\item the \PGFPlotstable\ component which simplifies number formatting and postprocessing of numerical tables. It comes as a separate package and has its own manual \href{file:pgfplotstable.pdf}{pgfplotstable.pdf}.
\end{enumerate}

\subsection{Upgrade remarks}
This release provides a lot of improvements which can be found in all detail in \texttt{ChangeLog} for interested readers. However, some attention is useful with respect to the following changes.

\subsubsection{New Optional Features}
\begin{enumerate}
	\item \PGFPlots\ 1.3 comes with user interface improvements. The technical distinction between ``behavior options'' and ``style options'' of older versions is no longer necessary (although still fully supported).

	\item \PGFPlots\ 1.3 has a new feature which allows to \emph{move axis labels tight to tick labels} automatically.

	Since this affects the spacing, it is not enabled be default.

	Use
\begin{codeexample}[code only]
\usepackage{pgfplots}
\pgfplotsset{compat=1.3}
\end{codeexample}
	in your preamble to benefit from the improved spacing\footnote{The |compat=1.3| setting is necessary for new features which move axis labels around like reversed axis. However, \PGFPlots\ will still work as it does in earlier versions, your documents will remain the same if you don't set it explicitly.}.
	Take a look at the next page for the precise definition of |compat|.

	\item \PGFPlots\ 1.3 now supports reversed axes. It is no longer necessary to use work--arounds with negative units.
\pgfkeys{/pdflinks/search key prefixes in/.add={/pgfplots/,}{}}

	Take a look at the |x dir=reverse| key.

	Existing work--arounds will still function properly. Use |\pgfplotsset{compat=1.3}| together with |x dir=reverse| to switch to the new version.
\end{enumerate}

\subsubsection{Old Features Which May Need Attention}
\begin{enumerate}
	\item The |scatter/classes| feature produces proper legends as of version 1.3. This may change the appearance of existing legends of plots with |scatter/classes|.

	\item Due to a small math bug in \PGF\ $2.00$, you \emph{can't} use the math expression `|-x^2|'. It is necessary to use `|0-x^2|' instead. The same holds for
		`|exp(-x^2)|'; use `|exp(0-x^2)|' instead.

		This will be fixed with \PGF\ $>2.00$.

	\item Starting with \PGFPlots\ $1.1$, |\tikzstyle| should \emph{no longer be used} to set \PGFPlots\ options.
	
	Although |\tikzstyle| is still supported for some older \PGFPlots\ options, you should replace any occurance of |\tikzstyle| with |\pgfplotsset{|\meta{style name}|/.style={|\meta{key-value-list}|}}| or the associated |/.append style| variant. See section~\ref{sec:styles} for more detail.
\end{enumerate}
I apologize for any inconvenience caused by these changes.

\begin{pgfplotskey}{compat=\mchoice{1.3,pre 1.3,default,newest} (initially default)}
	The preamble configuration 
\begin{codeexample}[code only]
\usepackage{pgfplots}
\pgfplotsset{compat=1.3}
\end{codeexample}
	allows to choose between backwards compatibility and most recent features.

	\PGFPlots\ version 1.3 comes with new features which may lead to different spacings compared to earlier versions:
	\begin{enumerate}
		\item Version 1.3 comes with the |xlabel near ticks| feature which places (in this case $x$) axis labels ``near'' the tick labels, taking the size of tick labels into account.
		
		Older versions used |xlabel absolute|, i.e.\ an absolute distance between the axis and axis labels (now matter how large or small tick labels are).

		The initial setting \emph{keeps backwards compatibility}.

		You are encouraged to use
\begin{codeexample}[code only]
\usepackage{pgfplots}
\pgfplotsset{compat=1.3}
\end{codeexample}
		in your preamble.
		
		\item Version 1.3 fixes a bug with |anchor=south west| which occurred mainly for reversed axes: the anchors where upside--down. This is fixed now.

		
		If you have written some work--around and want to keep it going, use

		|\pgfplotsset{compat/anchors=pre 1.3}|\pgfmanualpdflabel{/pgfplots/compat/anchors}{} 

		to disable the bug fix.
	\end{enumerate}

	Use |\pgfplotsset{compat=default}| to restore the factory settings.

	The setting |\pgfplotsset{compat=newest}| will always use features of the most recent version. This might result in small changes in the document's appearance.
\end{pgfplotskey}

\subsection{The Team}
\PGFPlots\ has been written mainly by Christian Feuers�nger with many improvements of Pascal Wolkotte and Nick Papior Andersen as a spare time project. We hope it is useful and provides valuable plots.

If you are interested in writing something but don't know how, consider reading the auxiliary manual \href{file:TeX-programming-notes.pdf}{TeX-programming-notes.pdf} which comes with \PGFPlots. It is far from complete, but maybe it is a good starting point (at least for more literature).

\subsection{Acknowledgements}
I thank God for all hours of enjoyed programming. I thank Pascal Wolkotte and Nick Papior Andersen for their programming efforts and contributions as part of the development team. I thank Stefan Tibus, who contributed the |plot shell| feature. I thank Tom Cashman for the contribution of the |reverse legend| feature. Special thanks go to Stefan Pinnow whose tests of \PGFPlots\ lead to numerous quality improvements. Furthermore, I thank Dr.~Schweitzer for many fruitful discussions and Dr.~Meine for his ideas and suggestions. Thanks as well to the many international contributors who provided feature requests or identified bugs or simply manual improvements!

Last but not least, I thank Till Tantau and Mark Wibrow for their excellent graphics (and more) package \PGF\ and \Tikz\ which is the base of \PGFPlots.

\include{pgfplots.install}%
\include{pgfplots.intro}%
\include{pgfplots.reference}%
\include{pgfplots.libs}%
\include{pgfplots.resources}%
\include{pgfplots.importexport}%
\include{pgfplots.basic.reference}%

\printindex

\bibliographystyle{abbrv} %gerapali} %gerabbrv} %gerunsrt.bst} %gerabbrv}% gerplain}
\nocite{pgfplotstable}
\nocite{programmingnotes}
\bibliography{pgfplots}
\end{document}
%
%%%%%%%%%%%%%%%%%%%%%%%%%%%%%%%%%%%%%%%%%%%%%%%%%%%%%%%%%%%%%%%%%%%%%%%%%%%%%
%
% Package pgfplots.sty documentation. 
%
% Copyright 2007/2008 by Christian Feuersaenger.
%
% This program is free software: you can redistribute it and/or modify
% it under the terms of the GNU General Public License as published by
% the Free Software Foundation, either version 3 of the License, or
% (at your option) any later version.
% 
% This program is distributed in the hope that it will be useful,
% but WITHOUT ANY WARRANTY; without even the implied warranty of
% MERCHANTABILITY or FITNESS FOR A PARTICULAR PURPOSE.  See the
% GNU General Public License for more details.
% 
% You should have received a copy of the GNU General Public License
% along with this program.  If not, see <http://www.gnu.org/licenses/>.
%
%
%%%%%%%%%%%%%%%%%%%%%%%%%%%%%%%%%%%%%%%%%%%%%%%%%%%%%%%%%%%%%%%%%%%%%%%%%%%%%
\input{pgfplots.preamble.tex}
%\RequirePackage[german,english,francais]{babel}

\pgfplotsmanualexternalexpensivetrue

%\usetikzlibrary{external}
%\tikzexternalize[prefix=figures/]{pgfplots}

\title{%
	Manual for Package \PGFPlots\\
	{\small Version \pgfplotsversion}\\
	{\small\href{http://sourceforge.net/projects/pgfplots}{http://sourceforge.net/projects/pgfplots}}}

%\includeonly{pgfplots.libs}


\begin{document}

\def\plotcoords{%
\addplot coordinates {
(5,8.312e-02)    (17,2.547e-02)   (49,7.407e-03)
(129,2.102e-03)  (321,5.874e-04)  (769,1.623e-04)
(1793,4.442e-05) (4097,1.207e-05) (9217,3.261e-06)
};

\addplot coordinates{
(7,8.472e-02)    (31,3.044e-02)    (111,1.022e-02)
(351,3.303e-03)  (1023,1.039e-03)  (2815,3.196e-04)
(7423,9.658e-05) (18943,2.873e-05) (47103,8.437e-06)
};

\addplot coordinates{
(9,7.881e-02)     (49,3.243e-02)    (209,1.232e-02)
(769,4.454e-03)   (2561,1.551e-03)  (7937,5.236e-04)
(23297,1.723e-04) (65537,5.545e-05) (178177,1.751e-05)
};

\addplot coordinates{
(11,6.887e-02)    (71,3.177e-02)     (351,1.341e-02)
(1471,5.334e-03)  (5503,2.027e-03)   (18943,7.415e-04)
(61183,2.628e-04) (187903,9.063e-05) (553983,3.053e-05)
};

\addplot coordinates{
(13,5.755e-02)     (97,2.925e-02)     (545,1.351e-02)
(2561,5.842e-03)   (10625,2.397e-03)  (40193,9.414e-04)
(141569,3.564e-04) (471041,1.308e-04) 
(1496065,4.670e-05)
};
}%
\maketitle
\begin{abstract}%
\PGFPlots\ draws high--quality function plots in normal or logarithmic scaling with a user-friendly interface directly in \TeX. The user supplies axis labels, legend entries and the plot coordinates for one or more plots and \PGFPlots\ applies axis scaling, computes any logarithms and axis ticks and draws the plots, supporting line plots, scatter plots, piecewise constant plots, bar plots, area plots, mesh-- and surface plots and some more. It is based on Till Tantau's package \PGF/\Tikz.
\end{abstract}
\tableofcontents
\section{Introduction}
This package provides tools to generate plots and labeled axes easily. It draws normal plots, logplots and semi-logplots, in two and three dimensions. Axis ticks, labels, legends (in case of multiple plots) can be added with key-value options. It can cycle through a set of predefined line/marker/color specifications. In summary, its purpose is to simplify the generation of high-quality function and/or data plots, and solving the problems of
\begin{itemize}
	\item consistency of document and font type and font size,
	\item direct use of \TeX\ math mode in axis descriptions,
	\item consistency of data and figures (no third party tool necessary),
	\item inter document consistency using preamble configurations and styles.
\end{itemize}
Although not necessary, separate |.pdf| or |.eps| graphics can be generated using the |external| library developed as part of \Tikz.

\PGFPlots\ is build completely on \Tikz/\PGF. Knowledge of \Tikz\ will simplify the work with \PGFPlots, although it is not required.

\section[About PGFPlots: Preliminaries]{About {\normalfont\PGFPlots}: Preliminaries}
This section contains information about upgrades, the team, the installation (in case you need to do it manually) and troubleshooting. You may skip it completely except for the upgrade remarks.

However, note that this library requires at least \PGF\ version $2.00$. At the time of this writing, many \TeX-distributions still contain the older \PGF\ version $1.18$, so it may be necessary to install a recent \PGF\ prior to using \PGFPlots.

\subsection{Components}
\PGFPlots\ comes with two components:
\begin{enumerate}
	\item the plotting component (which you are currently reading) and
	\item the \PGFPlotstable\ component which simplifies number formatting and postprocessing of numerical tables. It comes as a separate package and has its own manual \href{file:pgfplotstable.pdf}{pgfplotstable.pdf}.
\end{enumerate}

\subsection{Upgrade remarks}
This release provides a lot of improvements which can be found in all detail in \texttt{ChangeLog} for interested readers. However, some attention is useful with respect to the following changes.

\subsubsection{New Optional Features}
\begin{enumerate}
	\item \PGFPlots\ 1.3 comes with user interface improvements. The technical distinction between ``behavior options'' and ``style options'' of older versions is no longer necessary (although still fully supported).

	\item \PGFPlots\ 1.3 has a new feature which allows to \emph{move axis labels tight to tick labels} automatically.

	Since this affects the spacing, it is not enabled be default.

	Use
\begin{codeexample}[code only]
\usepackage{pgfplots}
\pgfplotsset{compat=1.3}
\end{codeexample}
	in your preamble to benefit from the improved spacing\footnote{The |compat=1.3| setting is necessary for new features which move axis labels around like reversed axis. However, \PGFPlots\ will still work as it does in earlier versions, your documents will remain the same if you don't set it explicitly.}.
	Take a look at the next page for the precise definition of |compat|.

	\item \PGFPlots\ 1.3 now supports reversed axes. It is no longer necessary to use work--arounds with negative units.
\pgfkeys{/pdflinks/search key prefixes in/.add={/pgfplots/,}{}}

	Take a look at the |x dir=reverse| key.

	Existing work--arounds will still function properly. Use |\pgfplotsset{compat=1.3}| together with |x dir=reverse| to switch to the new version.
\end{enumerate}

\subsubsection{Old Features Which May Need Attention}
\begin{enumerate}
	\item The |scatter/classes| feature produces proper legends as of version 1.3. This may change the appearance of existing legends of plots with |scatter/classes|.

	\item Due to a small math bug in \PGF\ $2.00$, you \emph{can't} use the math expression `|-x^2|'. It is necessary to use `|0-x^2|' instead. The same holds for
		`|exp(-x^2)|'; use `|exp(0-x^2)|' instead.

		This will be fixed with \PGF\ $>2.00$.

	\item Starting with \PGFPlots\ $1.1$, |\tikzstyle| should \emph{no longer be used} to set \PGFPlots\ options.
	
	Although |\tikzstyle| is still supported for some older \PGFPlots\ options, you should replace any occurance of |\tikzstyle| with |\pgfplotsset{|\meta{style name}|/.style={|\meta{key-value-list}|}}| or the associated |/.append style| variant. See section~\ref{sec:styles} for more detail.
\end{enumerate}
I apologize for any inconvenience caused by these changes.

\begin{pgfplotskey}{compat=\mchoice{1.3,pre 1.3,default,newest} (initially default)}
	The preamble configuration 
\begin{codeexample}[code only]
\usepackage{pgfplots}
\pgfplotsset{compat=1.3}
\end{codeexample}
	allows to choose between backwards compatibility and most recent features.

	\PGFPlots\ version 1.3 comes with new features which may lead to different spacings compared to earlier versions:
	\begin{enumerate}
		\item Version 1.3 comes with the |xlabel near ticks| feature which places (in this case $x$) axis labels ``near'' the tick labels, taking the size of tick labels into account.
		
		Older versions used |xlabel absolute|, i.e.\ an absolute distance between the axis and axis labels (now matter how large or small tick labels are).

		The initial setting \emph{keeps backwards compatibility}.

		You are encouraged to use
\begin{codeexample}[code only]
\usepackage{pgfplots}
\pgfplotsset{compat=1.3}
\end{codeexample}
		in your preamble.
		
		\item Version 1.3 fixes a bug with |anchor=south west| which occurred mainly for reversed axes: the anchors where upside--down. This is fixed now.

		
		If you have written some work--around and want to keep it going, use

		|\pgfplotsset{compat/anchors=pre 1.3}|\pgfmanualpdflabel{/pgfplots/compat/anchors}{} 

		to disable the bug fix.
	\end{enumerate}

	Use |\pgfplotsset{compat=default}| to restore the factory settings.

	The setting |\pgfplotsset{compat=newest}| will always use features of the most recent version. This might result in small changes in the document's appearance.
\end{pgfplotskey}

\subsection{The Team}
\PGFPlots\ has been written mainly by Christian Feuers�nger with many improvements of Pascal Wolkotte and Nick Papior Andersen as a spare time project. We hope it is useful and provides valuable plots.

If you are interested in writing something but don't know how, consider reading the auxiliary manual \href{file:TeX-programming-notes.pdf}{TeX-programming-notes.pdf} which comes with \PGFPlots. It is far from complete, but maybe it is a good starting point (at least for more literature).

\subsection{Acknowledgements}
I thank God for all hours of enjoyed programming. I thank Pascal Wolkotte and Nick Papior Andersen for their programming efforts and contributions as part of the development team. I thank Stefan Tibus, who contributed the |plot shell| feature. I thank Tom Cashman for the contribution of the |reverse legend| feature. Special thanks go to Stefan Pinnow whose tests of \PGFPlots\ lead to numerous quality improvements. Furthermore, I thank Dr.~Schweitzer for many fruitful discussions and Dr.~Meine for his ideas and suggestions. Thanks as well to the many international contributors who provided feature requests or identified bugs or simply manual improvements!

Last but not least, I thank Till Tantau and Mark Wibrow for their excellent graphics (and more) package \PGF\ and \Tikz\ which is the base of \PGFPlots.

\include{pgfplots.install}%
\include{pgfplots.intro}%
\include{pgfplots.reference}%
\include{pgfplots.libs}%
\include{pgfplots.resources}%
\include{pgfplots.importexport}%
\include{pgfplots.basic.reference}%

\printindex

\bibliographystyle{abbrv} %gerapali} %gerabbrv} %gerunsrt.bst} %gerabbrv}% gerplain}
\nocite{pgfplotstable}
\nocite{programmingnotes}
\bibliography{pgfplots}
\end{document}
%
%%%%%%%%%%%%%%%%%%%%%%%%%%%%%%%%%%%%%%%%%%%%%%%%%%%%%%%%%%%%%%%%%%%%%%%%%%%%%
%
% Package pgfplots.sty documentation. 
%
% Copyright 2007/2008 by Christian Feuersaenger.
%
% This program is free software: you can redistribute it and/or modify
% it under the terms of the GNU General Public License as published by
% the Free Software Foundation, either version 3 of the License, or
% (at your option) any later version.
% 
% This program is distributed in the hope that it will be useful,
% but WITHOUT ANY WARRANTY; without even the implied warranty of
% MERCHANTABILITY or FITNESS FOR A PARTICULAR PURPOSE.  See the
% GNU General Public License for more details.
% 
% You should have received a copy of the GNU General Public License
% along with this program.  If not, see <http://www.gnu.org/licenses/>.
%
%
%%%%%%%%%%%%%%%%%%%%%%%%%%%%%%%%%%%%%%%%%%%%%%%%%%%%%%%%%%%%%%%%%%%%%%%%%%%%%
\input{pgfplots.preamble.tex}
%\RequirePackage[german,english,francais]{babel}

\pgfplotsmanualexternalexpensivetrue

%\usetikzlibrary{external}
%\tikzexternalize[prefix=figures/]{pgfplots}

\title{%
	Manual for Package \PGFPlots\\
	{\small Version \pgfplotsversion}\\
	{\small\href{http://sourceforge.net/projects/pgfplots}{http://sourceforge.net/projects/pgfplots}}}

%\includeonly{pgfplots.libs}


\begin{document}

\def\plotcoords{%
\addplot coordinates {
(5,8.312e-02)    (17,2.547e-02)   (49,7.407e-03)
(129,2.102e-03)  (321,5.874e-04)  (769,1.623e-04)
(1793,4.442e-05) (4097,1.207e-05) (9217,3.261e-06)
};

\addplot coordinates{
(7,8.472e-02)    (31,3.044e-02)    (111,1.022e-02)
(351,3.303e-03)  (1023,1.039e-03)  (2815,3.196e-04)
(7423,9.658e-05) (18943,2.873e-05) (47103,8.437e-06)
};

\addplot coordinates{
(9,7.881e-02)     (49,3.243e-02)    (209,1.232e-02)
(769,4.454e-03)   (2561,1.551e-03)  (7937,5.236e-04)
(23297,1.723e-04) (65537,5.545e-05) (178177,1.751e-05)
};

\addplot coordinates{
(11,6.887e-02)    (71,3.177e-02)     (351,1.341e-02)
(1471,5.334e-03)  (5503,2.027e-03)   (18943,7.415e-04)
(61183,2.628e-04) (187903,9.063e-05) (553983,3.053e-05)
};

\addplot coordinates{
(13,5.755e-02)     (97,2.925e-02)     (545,1.351e-02)
(2561,5.842e-03)   (10625,2.397e-03)  (40193,9.414e-04)
(141569,3.564e-04) (471041,1.308e-04) 
(1496065,4.670e-05)
};
}%
\maketitle
\begin{abstract}%
\PGFPlots\ draws high--quality function plots in normal or logarithmic scaling with a user-friendly interface directly in \TeX. The user supplies axis labels, legend entries and the plot coordinates for one or more plots and \PGFPlots\ applies axis scaling, computes any logarithms and axis ticks and draws the plots, supporting line plots, scatter plots, piecewise constant plots, bar plots, area plots, mesh-- and surface plots and some more. It is based on Till Tantau's package \PGF/\Tikz.
\end{abstract}
\tableofcontents
\section{Introduction}
This package provides tools to generate plots and labeled axes easily. It draws normal plots, logplots and semi-logplots, in two and three dimensions. Axis ticks, labels, legends (in case of multiple plots) can be added with key-value options. It can cycle through a set of predefined line/marker/color specifications. In summary, its purpose is to simplify the generation of high-quality function and/or data plots, and solving the problems of
\begin{itemize}
	\item consistency of document and font type and font size,
	\item direct use of \TeX\ math mode in axis descriptions,
	\item consistency of data and figures (no third party tool necessary),
	\item inter document consistency using preamble configurations and styles.
\end{itemize}
Although not necessary, separate |.pdf| or |.eps| graphics can be generated using the |external| library developed as part of \Tikz.

\PGFPlots\ is build completely on \Tikz/\PGF. Knowledge of \Tikz\ will simplify the work with \PGFPlots, although it is not required.

\section[About PGFPlots: Preliminaries]{About {\normalfont\PGFPlots}: Preliminaries}
This section contains information about upgrades, the team, the installation (in case you need to do it manually) and troubleshooting. You may skip it completely except for the upgrade remarks.

However, note that this library requires at least \PGF\ version $2.00$. At the time of this writing, many \TeX-distributions still contain the older \PGF\ version $1.18$, so it may be necessary to install a recent \PGF\ prior to using \PGFPlots.

\subsection{Components}
\PGFPlots\ comes with two components:
\begin{enumerate}
	\item the plotting component (which you are currently reading) and
	\item the \PGFPlotstable\ component which simplifies number formatting and postprocessing of numerical tables. It comes as a separate package and has its own manual \href{file:pgfplotstable.pdf}{pgfplotstable.pdf}.
\end{enumerate}

\subsection{Upgrade remarks}
This release provides a lot of improvements which can be found in all detail in \texttt{ChangeLog} for interested readers. However, some attention is useful with respect to the following changes.

\subsubsection{New Optional Features}
\begin{enumerate}
	\item \PGFPlots\ 1.3 comes with user interface improvements. The technical distinction between ``behavior options'' and ``style options'' of older versions is no longer necessary (although still fully supported).

	\item \PGFPlots\ 1.3 has a new feature which allows to \emph{move axis labels tight to tick labels} automatically.

	Since this affects the spacing, it is not enabled be default.

	Use
\begin{codeexample}[code only]
\usepackage{pgfplots}
\pgfplotsset{compat=1.3}
\end{codeexample}
	in your preamble to benefit from the improved spacing\footnote{The |compat=1.3| setting is necessary for new features which move axis labels around like reversed axis. However, \PGFPlots\ will still work as it does in earlier versions, your documents will remain the same if you don't set it explicitly.}.
	Take a look at the next page for the precise definition of |compat|.

	\item \PGFPlots\ 1.3 now supports reversed axes. It is no longer necessary to use work--arounds with negative units.
\pgfkeys{/pdflinks/search key prefixes in/.add={/pgfplots/,}{}}

	Take a look at the |x dir=reverse| key.

	Existing work--arounds will still function properly. Use |\pgfplotsset{compat=1.3}| together with |x dir=reverse| to switch to the new version.
\end{enumerate}

\subsubsection{Old Features Which May Need Attention}
\begin{enumerate}
	\item The |scatter/classes| feature produces proper legends as of version 1.3. This may change the appearance of existing legends of plots with |scatter/classes|.

	\item Due to a small math bug in \PGF\ $2.00$, you \emph{can't} use the math expression `|-x^2|'. It is necessary to use `|0-x^2|' instead. The same holds for
		`|exp(-x^2)|'; use `|exp(0-x^2)|' instead.

		This will be fixed with \PGF\ $>2.00$.

	\item Starting with \PGFPlots\ $1.1$, |\tikzstyle| should \emph{no longer be used} to set \PGFPlots\ options.
	
	Although |\tikzstyle| is still supported for some older \PGFPlots\ options, you should replace any occurance of |\tikzstyle| with |\pgfplotsset{|\meta{style name}|/.style={|\meta{key-value-list}|}}| or the associated |/.append style| variant. See section~\ref{sec:styles} for more detail.
\end{enumerate}
I apologize for any inconvenience caused by these changes.

\begin{pgfplotskey}{compat=\mchoice{1.3,pre 1.3,default,newest} (initially default)}
	The preamble configuration 
\begin{codeexample}[code only]
\usepackage{pgfplots}
\pgfplotsset{compat=1.3}
\end{codeexample}
	allows to choose between backwards compatibility and most recent features.

	\PGFPlots\ version 1.3 comes with new features which may lead to different spacings compared to earlier versions:
	\begin{enumerate}
		\item Version 1.3 comes with the |xlabel near ticks| feature which places (in this case $x$) axis labels ``near'' the tick labels, taking the size of tick labels into account.
		
		Older versions used |xlabel absolute|, i.e.\ an absolute distance between the axis and axis labels (now matter how large or small tick labels are).

		The initial setting \emph{keeps backwards compatibility}.

		You are encouraged to use
\begin{codeexample}[code only]
\usepackage{pgfplots}
\pgfplotsset{compat=1.3}
\end{codeexample}
		in your preamble.
		
		\item Version 1.3 fixes a bug with |anchor=south west| which occurred mainly for reversed axes: the anchors where upside--down. This is fixed now.

		
		If you have written some work--around and want to keep it going, use

		|\pgfplotsset{compat/anchors=pre 1.3}|\pgfmanualpdflabel{/pgfplots/compat/anchors}{} 

		to disable the bug fix.
	\end{enumerate}

	Use |\pgfplotsset{compat=default}| to restore the factory settings.

	The setting |\pgfplotsset{compat=newest}| will always use features of the most recent version. This might result in small changes in the document's appearance.
\end{pgfplotskey}

\subsection{The Team}
\PGFPlots\ has been written mainly by Christian Feuers�nger with many improvements of Pascal Wolkotte and Nick Papior Andersen as a spare time project. We hope it is useful and provides valuable plots.

If you are interested in writing something but don't know how, consider reading the auxiliary manual \href{file:TeX-programming-notes.pdf}{TeX-programming-notes.pdf} which comes with \PGFPlots. It is far from complete, but maybe it is a good starting point (at least for more literature).

\subsection{Acknowledgements}
I thank God for all hours of enjoyed programming. I thank Pascal Wolkotte and Nick Papior Andersen for their programming efforts and contributions as part of the development team. I thank Stefan Tibus, who contributed the |plot shell| feature. I thank Tom Cashman for the contribution of the |reverse legend| feature. Special thanks go to Stefan Pinnow whose tests of \PGFPlots\ lead to numerous quality improvements. Furthermore, I thank Dr.~Schweitzer for many fruitful discussions and Dr.~Meine for his ideas and suggestions. Thanks as well to the many international contributors who provided feature requests or identified bugs or simply manual improvements!

Last but not least, I thank Till Tantau and Mark Wibrow for their excellent graphics (and more) package \PGF\ and \Tikz\ which is the base of \PGFPlots.

\include{pgfplots.install}%
\include{pgfplots.intro}%
\include{pgfplots.reference}%
\include{pgfplots.libs}%
\include{pgfplots.resources}%
\include{pgfplots.importexport}%
\include{pgfplots.basic.reference}%

\printindex

\bibliographystyle{abbrv} %gerapali} %gerabbrv} %gerunsrt.bst} %gerabbrv}% gerplain}
\nocite{pgfplotstable}
\nocite{programmingnotes}
\bibliography{pgfplots}
\end{document}
%
\include{pgfplots.basic.reference}%

\printindex

\bibliographystyle{abbrv} %gerapali} %gerabbrv} %gerunsrt.bst} %gerabbrv}% gerplain}
\nocite{pgfplotstable}
\nocite{programmingnotes}
\bibliography{pgfplots}
\end{document}
%
\include{pgfplots.basic.reference}%

\printindex

\bibliographystyle{abbrv} %gerapali} %gerabbrv} %gerunsrt.bst} %gerabbrv}% gerplain}
\nocite{pgfplotstable}
\nocite{programmingnotes}
\bibliography{pgfplots}
\end{document}
%
%%%%%%%%%%%%%%%%%%%%%%%%%%%%%%%%%%%%%%%%%%%%%%%%%%%%%%%%%%%%%%%%%%%%%%%%%%%%%
%
% Package pgfplots.sty documentation. 
%
% Copyright 2007/2008 by Christian Feuersaenger.
%
% This program is free software: you can redistribute it and/or modify
% it under the terms of the GNU General Public License as published by
% the Free Software Foundation, either version 3 of the License, or
% (at your option) any later version.
% 
% This program is distributed in the hope that it will be useful,
% but WITHOUT ANY WARRANTY; without even the implied warranty of
% MERCHANTABILITY or FITNESS FOR A PARTICULAR PURPOSE.  See the
% GNU General Public License for more details.
% 
% You should have received a copy of the GNU General Public License
% along with this program.  If not, see <http://www.gnu.org/licenses/>.
%
%
%%%%%%%%%%%%%%%%%%%%%%%%%%%%%%%%%%%%%%%%%%%%%%%%%%%%%%%%%%%%%%%%%%%%%%%%%%%%%
%%%%%%%%%%%%%%%%%%%%%%%%%%%%%%%%%%%%%%%%%%%%%%%%%%%%%%%%%%%%%%%%%%%%%%%%%%%%%
%
% Package pgfplots.sty documentation. 
%
% Copyright 2007/2008 by Christian Feuersaenger.
%
% This program is free software: you can redistribute it and/or modify
% it under the terms of the GNU General Public License as published by
% the Free Software Foundation, either version 3 of the License, or
% (at your option) any later version.
% 
% This program is distributed in the hope that it will be useful,
% but WITHOUT ANY WARRANTY; without even the implied warranty of
% MERCHANTABILITY or FITNESS FOR A PARTICULAR PURPOSE.  See the
% GNU General Public License for more details.
% 
% You should have received a copy of the GNU General Public License
% along with this program.  If not, see <http://www.gnu.org/licenses/>.
%
%
%%%%%%%%%%%%%%%%%%%%%%%%%%%%%%%%%%%%%%%%%%%%%%%%%%%%%%%%%%%%%%%%%%%%%%%%%%%%%
\documentclass[a4paper]{ltxdoc}

\usepackage{makeidx}

% DON't let hyperref overload the format of index and glossary. 
% I want to do that on my own in the stylefiles for makeindex...
\makeatletter
\let\@old@wrindex=\@wrindex
\makeatother

\usepackage{ifpdf}
\ifpdf
	\usepackage[pdftex]{hyperref}
\else
	\usepackage[dvipdfm]{hyperref}
\fi
\hypersetup{pdfborder={0 0 0}}

\makeatletter
\let\@wrindex=\@old@wrindex
\makeatother

% Formatiere Seitennummern im Index:
\newcommand{\indexpageno}[1]{%
	{\bfseries\hyperpage{#1}}%
}


\newcommand{\R}{\mathbb{R}}
\newcommand{\N}{\mathbb{N}}
\newcommand{\Z}{\mathbb{Z}}

\long\def\COMMENTLOWLEVEL#1\ENDCOMMENT{}
\def\ENDCOMMENT{}

\usepackage{textcomp}
\usepackage{booktabs}

\usepackage{calc}
\usepackage[formats]{listings}
%\usepackage{courier} % don't use it - the '^' character can't be copy-pasted in courier

\usepackage{array}
\lstset{%
	basicstyle=\ttfamily,
	language=[LaTeX]tex, % Seems as if \lstset{language=tex} must be invoked BEFORE loading tikz!?
	tabsize=4,
	breaklines=true,
	breakindent=0pt
}

\ifpdf
	\pdfinfo {
		/Author	(Christian Feuersaenger)
	}

\else
	\def\pgfsysdriver{pgfsys-dvipdfm.def}
\fi
%\def\pgfsysdriver{pgfsys-pdftex.def}
\usepackage{pgfplots}

\ifpdf
	\usetikzlibrary{pgfplotsclickable}
\fi

%\usepackage{fp}
% ATTENTION:
% this requires pgf version NEWER than 2.00 :
%\usetikzlibrary{fixedpointarithmetic}

\usetikzlibrary{dateplot}

\usepackage[a4paper,left=2.25cm,right=2.25cm,top=2.5cm,bottom=2.5cm,nohead]{geometry}
\usepackage{amsmath,amssymb}
\usepackage{xxcolor}
\usepackage{pifont}
\usepackage[latin1]{inputenc}
\usepackage{amsmath}
\usepackage{eurosym}

\def\eps{\textsc{eps}}

\input pgfmanual-en-macros.tex

\def\pgfmanualbar{\char`\|}
\makeatletter
\newenvironment{addplotoperation}[3][]{
  \begin{pgfmanualentry}
  	{%
	\let\ltxdoc@marg=\marg
	\let\ltxdoc@oarg=\oarg
	\let\ltxdoc@parg=\parg
	\let\ltxdoc@meta=\meta
	\def\marg##1{{\normalfont\ltxdoc@marg{##1}}}%
	\def\oarg##1{{\normalfont\ltxdoc@oarg{##1}}}%
	\def\parg##1{{\normalfont\ltxdoc@parg{##1}}}%
	\def\meta##1{{\normalfont\ltxdoc@meta{##1}}}%
    \pgfmanualentryheadline{\textcolor{gray}{{\ttfamily\char`\\addplot\ }}%
      \declare{\texttt{#2}} \texttt{#3;}}%
	  \unskip
	 \nobreak
    \pgfmanualentryheadline{\textcolor{gray}{{\texttt{\char`\\addplot}\oarg{style options} \texttt{plot}\oarg{behavior options}}\ }%
      \declare{\texttt{#2}} \texttt{#3} \textcolor{gray}{\meta{trailing path commands}}\texttt{;}}%
    \def\pgfmanualtest{#1}%
    \ifx\pgfmanualtest\@empty%
      \index{#2@\protect\textcolor{gray}{\protect\texttt{plot}}\protect\texttt{ #2}}%
      \index{Plot operations!plot #2@\protect\texttt{plot #2}}%
    \fi%
	}%
    \pgfmanualbody
}
{
  \end{pgfmanualentry}
}

\newenvironment{codekey}[1]{%
  \begin{pgfmanualentry}
 	\pgfmanualentryheadline{{\ttfamily\declare{#1}\textcolor{gray}{/.code}=\marg{...}}\hfill}%
	\def\mykey{#1}%
	\def\mypath{}%
	\def\myname{}%
	\firsttimetrue%
	\decompose#1/\nil%
    \pgfmanualbody
}
{
  \end{pgfmanualentry}
}

\newenvironment{pgfplotscodekey}[1]{%
	\begin{codekey}{/pgfplots/#1}%
}
{
  \end{codekey}
}
\newenvironment{pgfplotscodetwokey}[1]{%
  \begin{pgfmanualentry}
 	\pgfmanualentryheadline{{\ttfamily\declare{/pgfplots/#1}\textcolor{gray}{/.code 2 args}=\marg{...}}\hfill}%
	\def\mykey{/pgfplots/#1}%
	\def\mypath{}%
	\def\myname{}%
	\firsttimetrue%
	\decompose/pgfplots/#1/\nil%
    \pgfmanualbody
}
{
  \end{pgfmanualentry}
}

\newenvironment{pgfplotsxycodekeylist}[1]{%
	\begingroup
	\let\oldpgfmanualentryheadline=\pgfmanualentryheadline
	\def\pgfmanualentryheadline##1{%
		\pgfmanualentryheadline@##1\pgfplots@EOI
	}%
	\def\pgfmanualentryheadline@##1\hfill##2\pgfplots@EOI{%
		\oldpgfmanualentryheadline{{\ttfamily\declare{##1}\textcolor{gray}{/.code}=\marg{...}}\hfill}%
	}
	\begin{pgfplotsxykeylist}{#1}%
}
{
	\end{pgfplotsxykeylist}
	\endgroup
}

\newenvironment{pgfplotskey}[1]{%
  \begin{key}{/pgfplots/#1}%
}
{
  \end{key}
}

\def\choicesep{$\vert$}%
\def\choicearg#1{\texttt{#1}}

\newif\iffirstchoice
\newcommand\mchoice[1]{%
	\begingroup
	\firstchoicetrue
	\foreach \mchoice@ in {#1} {%
		\iffirstchoice
			\global\firstchoicefalse
		\else
			\choicesep
		\fi
		\choicearg{\mchoice@}%
	}%
	\endgroup
}%

% \begin{xykey}{/path/\x label=value}
% \end{xykey}
%
% has same features with 'default', 'initially' etc as key environment
\newenvironment{xykey}[2][]{%
	\begin{pgfmanualentry}
    \def\extrakeytext{}
	\insertpathifneeded{#2}{#1}%
	\expandafter\pgfutil@in@\expandafter=\expandafter{\mykey}%
	\ifpgfutil@in@%
		\expandafter\xykey@eq\mykey\@nil
	\else
		\expandafter\xykey@noeq\mykey\@nil
	\fi
	\pgfmanualbody
}{%
	\end{pgfmanualentry}
}%

% \begin{xystylekey}{/path/\x label=value}
% \end{xystylekey}
%
% has same features with 'default', 'initially' etc as key environment
\newenvironment{xystylekey}[2][]{%
	\begin{pgfmanualentry}
    \def\extrakeytext{style, }
	\insertpathifneeded{#2}{#1}%
	\expandafter\pgfutil@in@\expandafter=\expandafter{\mykey}%
	\ifpgfutil@in@%
		\expandafter\xykey@eq\mykey\@nil
	\else
		\expandafter\xykey@noeq\mykey\@nil
	\fi
	\pgfmanualbody
}{%
	\end{pgfmanualentry}
}%

% \insertpathifneeded{a key}{/pgfplots} -> assign mykey={/pgfplots/a key}
% \insertpathifneeded{/tikz/a key}{/pgfplots} -> assign mykey={/tikz/a key}
%
% #1: the key
% #2: a default path (or empty)
\def\insertpathifneeded#1#2{%
	\def\insertpathifneeded@@{#2}%
	\ifx\insertpathifneeded@@\empty
		\def\mykey{#1}%
	\else
		\insertpathifneeded@#2\@nil
		\ifpgfutil@in@
			\def\mykey{#2/#1}%
		\else
			\def\mykey{#1}%
		\fi
	\fi
}%
\def\insertpathifneeded@#1#2\@nil{%
	\def\insertpathifneeded@@{#1}%
	\def\insertpathifneeded@@@{/}%
	\ifx\insertpathifneeded@@\insertpathifneeded@@@
		\pgfutil@in@true
	\else
		\pgfutil@in@false
	\fi
}%

% \begin{keylist}[default path]
% 	{/path/option 1=value,/path/option 2=value2}
% \end{keylist}
\newenvironment{keylist}[2][]{%
	\begin{pgfmanualentry}
    \def\extrakeytext{}%
	\foreach \xx in {#2} {%
		\expandafter\insertpathifneeded\expandafter{\xx}{#1}%
		\expandafter\extractkey\mykey\@nil%
	}%
	\pgfmanualbody
}{%
  \end{pgfmanualentry}
}%

\newenvironment{pgfplotskeylist}[1]{%
	\begin{keylist}[/pgfplots]{#1}%
}{%
	\end{keylist}%
}

% \begin{xykeylist}[default path]
% 	{/path/option \x1=value,/path/option \x2=value2,/path/option \x3=value}
% \end{xykeylist}
\newenvironment{xykeylist}[2][]{%
	\begin{pgfmanualentry}
    \def\extrakeytext{}
	\foreach \xx in {#2} {%
		\expandafter\insertpathifneeded\expandafter{\xx}{#1}%
		\expandafter\pgfutil@in@\expandafter=\expandafter{\mykey}%
		\ifpgfutil@in@%
			\expandafter\xykey@eq\mykey\@nil
		\else
			\expandafter\xykey@noeq\mykey\@nil
		\fi
	}%
	\pgfmanualbody
}{%
  \end{pgfmanualentry}
}%

\newif\ifxykeyfound

\def\xykey@eq#1=#2\@nil{%
	\def\x{x}%
	\xdef\mykey{#1}%
	\def\xykey@@{#1}%
	\ifx\xykey@@\mykey
		\xykeyfoundfalse
	\else
		\xykeyfoundtrue
	\fi
    \expandafter\extractkey\mykey=#2\@nil%
	\ifxykeyfound
		\def\x{y}%
		\xdef\mykey{#1}%
		\expandafter\extractkey\mykey=#2\@nil%
	\fi
}
\def\xykey@noeq#1\@nil{%
	\def\x{x}%
	\xdef\mykey{#1}%
	\def\xykey@@{#1}%
	\ifx\xykey@@\mykey
		\xykeyfoundfalse
	\else
		\xykeyfoundtrue
	\fi
    \expandafter\extractkey\mykey\@nil%
	\ifxykeyfound
		\def\x{y}%
		\xdef\mykey{#1}%
		\expandafter\extractkey\mykey\@nil%
	\fi
}

% \begin{pgfplotsxykey}{\x label=value}
% \end{pgfplotsxykey}
%
% It introduces the path /pgfplots/ automatically.
%
% has same features with 'default', 'initially' etc as key environment
\newenvironment{pgfplotsxykey}[1]{%
	\begin{xykey}[/pgfplots]{#1}%
}{%
	\end{xykey}%
}


\newenvironment{pgfplotsxykeylist}[1]{%
	\begin{xykeylist}[/pgfplots]{#1}%
}{%
	\end{xykeylist}%
}


\newenvironment{plottype}[1]{%
	\begin{key}{/tikz/#1}%
	\end{key}
  \begin{pgfmanualentry}
    \pgfmanualentryheadline{\textcolor{gray}{{\ttfamily\char`\\addplot+[\declare{#1}]}}}%
    \pgfmanualbody
}
{
  \end{pgfmanualentry}
}

\def\index@prologue{\section*{Index}\addcontentsline{toc}{section}{Index}
}

\newenvironment{pgfplotstablecolumnkey}{%
  \begin{pgfmanualentry}
 	\pgfmanualentryheadline{{\ttfamily\textcolor{gray}{/pgfplots/table/}\declare{columns/\meta{column name}}\textcolor{gray}{/.style}=\marg{key-value-list}}\hfill}%
	\pgfplotsmanualkeyindex{/pgfplots/table/columns}%
    \pgfmanualbody
}
{
  \end{pgfmanualentry}
}
\newenvironment{pgfplotstabledisplaycolumnkey}{%
  \begin{pgfmanualentry}
 	\pgfmanualentryheadline{{\ttfamily\textcolor{gray}{/pgfplots/table/}\declare{display columns/\meta{index}}\textcolor{gray}{/.style}=\marg{key-value-list}}\hfill}%
	\pgfplotsmanualkeyindex{/pgfplots/table/display columns}%
    \pgfmanualbody
}
{
  \end{pgfmanualentry}
}
\newenvironment{pgfplotstablealiaskey}{%
  \begin{pgfmanualentry}
 	\pgfmanualentryheadline{{\ttfamily\textcolor{gray}{/pgfplots/table/}\declare{alias/\meta{col name}}\textcolor{gray}{/.initial}=\marg{real col name}}\hfill}%
	\pgfplotsmanualkeyindex{/pgfplots/table/alias}%
    \pgfmanualbody
}
{
  \end{pgfmanualentry}
}


\def\pgfplotsmanualkeyindex#1{%
	\def\mypath{#1}%
	\def\myname{}%
	\firsttimetrue%
	\decompose#1/\nil%
}
\newenvironment{pgfplotstablecreateonusekey}{%
  \begin{pgfmanualentry}
 	\pgfmanualentryheadline{{\ttfamily\textcolor{gray}{/pgfplots/table/}\declare{create on use/\meta{col name}}\textcolor{gray}{/.style}=\marg{create options}}\hfill}%
	\def\mykey{/pgfplots/table/create on use}%
    \pgfmanualbody
	\pgfplotsmanualkeyindex{/pgfplots/table/create on use}%
}
{
  \end{pgfmanualentry}
}

\def\pgfplotsassertcmdkeyexists#1{%
	\pgfkeysifdefined{/pgfplots/#1/.@cmd}\relax{%
		\pgfplots@error{DOCUMENTATION ERROR: command key /pgfplots/#1 does not exist!}%
	}%
}%

{
\catcode`\ =12%
\gdef\makespaceexpandable{\def\ { }}}%

\def\pgfplotsassertXYcmdkeyexists#1{%
	{\makespaceexpandable\def\x{x}\edef\pgfplotsassertXYcmdkeyexists@tmp{#1}%
	\pgfkeysifdefined{/pgfplots/\pgfplotsassertXYcmdkeyexists@tmp/.@cmd}\relax{%
		\pgfplots@error{DOCUMENTATION ERROR: command key /pgfplots/#1 does not exist!}%
	}}%
	{\makespaceexpandable\def\x{y}\edef\pgfplotsassertXYcmdkeyexists@tmp{#1}%
	\pgfkeysifdefined{/pgfplots/\pgfplotsassertXYcmdkeyexists@tmp/.@cmd}\relax{%
		\pgfplots@error{DOCUMENTATION ERROR: command key /pgfplots/#1 does not exist!}%
	}}%
}%

\def\pgfplotsshortstylekey #1=#2\pgfeov{%
	\pgfplotsassertcmdkeyexists{#1}%
	\pgfplotsassertcmdkeyexists{#2}%
	\begin{pgfplotskey}{#1=\marg{key-value-list}}
		An abbreviation for \texttt{#2/.append style=}\marg{key-value-list}.
	\end{pgfplotskey}
}
\def\pgfplotsshortxystylekey #1=#2\pgfeov{%
	\pgfplotsassertXYcmdkeyexists{#1}%
	\pgfplotsassertXYcmdkeyexists{#2}%
	\begin{pgfplotsxykey}{#1=\marg{key-value-list}}
		An abbreviation for {\def\x{x}\texttt{#2/.append style=}}\marg{key-value-list} (or the respective style for $y$, {\def\x{y}\texttt{#2}}).
	\end{pgfplotsxykey}
}
\def\pgfplotsshortstylekeys #1,#2=#3\pgfeov{%
	\pgfplotsassertcmdkeyexists{#1}%
	\pgfplotsassertcmdkeyexists{#2}%
	\pgfplotsassertcmdkeyexists{#3}%
	\begin{pgfplotskeylist}{%
		#1=\marg{key-value-list},
		#2=\marg{key-value-list}}
		Different abbreviations for \texttt{#3/.append style=}\marg{key-value-list}.
	\end{pgfplotskeylist}
}
\def\pgfplotsshortxystylekeys #1,#2=#3\pgfeov{%
	\pgfplotsassertXYcmdkeyexists{#1}%
	\pgfplotsassertXYcmdkeyexists{#2}%
	\pgfplotsassertXYcmdkeyexists{#3}%
	\begin{pgfplotsxykeylist}{%
		#1=\marg{key-value-list},
		#2=\marg{key-value-list}}
		Different abbreviations for {\def\x{x}\texttt{#3/.append style=}}\marg{key-value-list} (or the respective style for $y$, {\def\x{y}\texttt{#3}}).
	\end{pgfplotsxykeylist}
}

%
% Creates and shows a colormap with specification '#1'.
\def\pgfplotsshowcolormapexample#1{%
	\pgfplotscreatecolormap{tempcolormap}{#1}%
	\pgfplotsshowcolormap{tempcolormap}%
}

% Shows the colormap named '#1'.
\def\pgfplotsshowcolormap#1{%
	\pgfplotscolormaptoshadingspec{#1}{8cm}\result
	\def\tempb{\pgfdeclarehorizontalshading{tempshading}{1cm}}%
	\expandafter\tempb\expandafter{\result}%
	\pgfuseshading{tempshading}%
}

\makeatother


\pgfqkeys{/codeexample}{%
	every codeexample/.style={
		width=8cm,
		/pgfplots/every axis/.append style={legend style={fill=graphicbackground}}
	},
	tabsize=4,
}
\pgfplotsset{
	every axis/.append style={width=7cm},
	filter discard warning=false,
}

\usetikzlibrary{backgrounds,patterns}
% Global styles:
\tikzset{
  shape example/.style={
    color=black!30,
    draw,
    fill=yellow!30,
    line width=.5cm,
    inner xsep=2.5cm,
    inner ysep=0.5cm}
}

\newcommand{\FIXME}[1]{\textcolor{red}{(FIXME: #1)}}

% fuer endvironment 'sidewaysfigure' bspw
% \usepackage{rotating}

\newcommand\Tikz{Ti\textit kZ}
\newcommand\PGF{\textsc{pgf}}
\newcommand\PGFPlots{\textsc{pgfplots}}
\def\PGFPlotstable{\textsc{PgfplotsTable}}


\makeindex

% Fix overful hboxes automatically:
\tolerance=2000
\emergencystretch=10pt

\tikzset{prefix=gnuplot/pgfplots_} % prefix for 'plot function'

\author{%
	Christian Feuers\"anger\footnote{\url{http://wissrech.ins.uni-bonn.de/people/feuersaenger}}\\%
	Institut f\"ur Numerische Simulation\\
	Universit\"at Bonn, Germany}


%\RequirePackage[german,english,francais]{babel}

\pgfplotsmanualexternalexpensivetrue

%\usetikzlibrary{external}
%\tikzexternalize[prefix=figures/]{pgfplots}

\title{%
	Manual for Package \PGFPlots\\
	{\small Version \pgfplotsversion}\\
	{\small\href{http://sourceforge.net/projects/pgfplots}{http://sourceforge.net/projects/pgfplots}}}

%\includeonly{pgfplots.libs}


\begin{document}

\def\plotcoords{%
\addplot coordinates {
(5,8.312e-02)    (17,2.547e-02)   (49,7.407e-03)
(129,2.102e-03)  (321,5.874e-04)  (769,1.623e-04)
(1793,4.442e-05) (4097,1.207e-05) (9217,3.261e-06)
};

\addplot coordinates{
(7,8.472e-02)    (31,3.044e-02)    (111,1.022e-02)
(351,3.303e-03)  (1023,1.039e-03)  (2815,3.196e-04)
(7423,9.658e-05) (18943,2.873e-05) (47103,8.437e-06)
};

\addplot coordinates{
(9,7.881e-02)     (49,3.243e-02)    (209,1.232e-02)
(769,4.454e-03)   (2561,1.551e-03)  (7937,5.236e-04)
(23297,1.723e-04) (65537,5.545e-05) (178177,1.751e-05)
};

\addplot coordinates{
(11,6.887e-02)    (71,3.177e-02)     (351,1.341e-02)
(1471,5.334e-03)  (5503,2.027e-03)   (18943,7.415e-04)
(61183,2.628e-04) (187903,9.063e-05) (553983,3.053e-05)
};

\addplot coordinates{
(13,5.755e-02)     (97,2.925e-02)     (545,1.351e-02)
(2561,5.842e-03)   (10625,2.397e-03)  (40193,9.414e-04)
(141569,3.564e-04) (471041,1.308e-04) 
(1496065,4.670e-05)
};
}%
\maketitle
\begin{abstract}%
\PGFPlots\ draws high--quality function plots in normal or logarithmic scaling with a user-friendly interface directly in \TeX. The user supplies axis labels, legend entries and the plot coordinates for one or more plots and \PGFPlots\ applies axis scaling, computes any logarithms and axis ticks and draws the plots, supporting line plots, scatter plots, piecewise constant plots, bar plots, area plots, mesh-- and surface plots and some more. It is based on Till Tantau's package \PGF/\Tikz.
\end{abstract}
\tableofcontents
\section{Introduction}
This package provides tools to generate plots and labeled axes easily. It draws normal plots, logplots and semi-logplots, in two and three dimensions. Axis ticks, labels, legends (in case of multiple plots) can be added with key-value options. It can cycle through a set of predefined line/marker/color specifications. In summary, its purpose is to simplify the generation of high-quality function and/or data plots, and solving the problems of
\begin{itemize}
	\item consistency of document and font type and font size,
	\item direct use of \TeX\ math mode in axis descriptions,
	\item consistency of data and figures (no third party tool necessary),
	\item inter document consistency using preamble configurations and styles.
\end{itemize}
Although not necessary, separate |.pdf| or |.eps| graphics can be generated using the |external| library developed as part of \Tikz.

\PGFPlots\ is build completely on \Tikz/\PGF. Knowledge of \Tikz\ will simplify the work with \PGFPlots, although it is not required.

\section[About PGFPlots: Preliminaries]{About {\normalfont\PGFPlots}: Preliminaries}
This section contains information about upgrades, the team, the installation (in case you need to do it manually) and troubleshooting. You may skip it completely except for the upgrade remarks.

However, note that this library requires at least \PGF\ version $2.00$. At the time of this writing, many \TeX-distributions still contain the older \PGF\ version $1.18$, so it may be necessary to install a recent \PGF\ prior to using \PGFPlots.

\subsection{Components}
\PGFPlots\ comes with two components:
\begin{enumerate}
	\item the plotting component (which you are currently reading) and
	\item the \PGFPlotstable\ component which simplifies number formatting and postprocessing of numerical tables. It comes as a separate package and has its own manual \href{file:pgfplotstable.pdf}{pgfplotstable.pdf}.
\end{enumerate}

\subsection{Upgrade remarks}
This release provides a lot of improvements which can be found in all detail in \texttt{ChangeLog} for interested readers. However, some attention is useful with respect to the following changes.

\subsubsection{New Optional Features}
\begin{enumerate}
	\item \PGFPlots\ 1.3 comes with user interface improvements. The technical distinction between ``behavior options'' and ``style options'' of older versions is no longer necessary (although still fully supported).

	\item \PGFPlots\ 1.3 has a new feature which allows to \emph{move axis labels tight to tick labels} automatically.

	Since this affects the spacing, it is not enabled be default.

	Use
\begin{codeexample}[code only]
\usepackage{pgfplots}
\pgfplotsset{compat=1.3}
\end{codeexample}
	in your preamble to benefit from the improved spacing\footnote{The |compat=1.3| setting is necessary for new features which move axis labels around like reversed axis. However, \PGFPlots\ will still work as it does in earlier versions, your documents will remain the same if you don't set it explicitly.}.
	Take a look at the next page for the precise definition of |compat|.

	\item \PGFPlots\ 1.3 now supports reversed axes. It is no longer necessary to use work--arounds with negative units.
\pgfkeys{/pdflinks/search key prefixes in/.add={/pgfplots/,}{}}

	Take a look at the |x dir=reverse| key.

	Existing work--arounds will still function properly. Use |\pgfplotsset{compat=1.3}| together with |x dir=reverse| to switch to the new version.
\end{enumerate}

\subsubsection{Old Features Which May Need Attention}
\begin{enumerate}
	\item The |scatter/classes| feature produces proper legends as of version 1.3. This may change the appearance of existing legends of plots with |scatter/classes|.

	\item Due to a small math bug in \PGF\ $2.00$, you \emph{can't} use the math expression `|-x^2|'. It is necessary to use `|0-x^2|' instead. The same holds for
		`|exp(-x^2)|'; use `|exp(0-x^2)|' instead.

		This will be fixed with \PGF\ $>2.00$.

	\item Starting with \PGFPlots\ $1.1$, |\tikzstyle| should \emph{no longer be used} to set \PGFPlots\ options.
	
	Although |\tikzstyle| is still supported for some older \PGFPlots\ options, you should replace any occurance of |\tikzstyle| with |\pgfplotsset{|\meta{style name}|/.style={|\meta{key-value-list}|}}| or the associated |/.append style| variant. See section~\ref{sec:styles} for more detail.
\end{enumerate}
I apologize for any inconvenience caused by these changes.

\begin{pgfplotskey}{compat=\mchoice{1.3,pre 1.3,default,newest} (initially default)}
	The preamble configuration 
\begin{codeexample}[code only]
\usepackage{pgfplots}
\pgfplotsset{compat=1.3}
\end{codeexample}
	allows to choose between backwards compatibility and most recent features.

	\PGFPlots\ version 1.3 comes with new features which may lead to different spacings compared to earlier versions:
	\begin{enumerate}
		\item Version 1.3 comes with the |xlabel near ticks| feature which places (in this case $x$) axis labels ``near'' the tick labels, taking the size of tick labels into account.
		
		Older versions used |xlabel absolute|, i.e.\ an absolute distance between the axis and axis labels (now matter how large or small tick labels are).

		The initial setting \emph{keeps backwards compatibility}.

		You are encouraged to use
\begin{codeexample}[code only]
\usepackage{pgfplots}
\pgfplotsset{compat=1.3}
\end{codeexample}
		in your preamble.
		
		\item Version 1.3 fixes a bug with |anchor=south west| which occurred mainly for reversed axes: the anchors where upside--down. This is fixed now.

		
		If you have written some work--around and want to keep it going, use

		|\pgfplotsset{compat/anchors=pre 1.3}|\pgfmanualpdflabel{/pgfplots/compat/anchors}{} 

		to disable the bug fix.
	\end{enumerate}

	Use |\pgfplotsset{compat=default}| to restore the factory settings.

	The setting |\pgfplotsset{compat=newest}| will always use features of the most recent version. This might result in small changes in the document's appearance.
\end{pgfplotskey}

\subsection{The Team}
\PGFPlots\ has been written mainly by Christian Feuers�nger with many improvements of Pascal Wolkotte and Nick Papior Andersen as a spare time project. We hope it is useful and provides valuable plots.

If you are interested in writing something but don't know how, consider reading the auxiliary manual \href{file:TeX-programming-notes.pdf}{TeX-programming-notes.pdf} which comes with \PGFPlots. It is far from complete, but maybe it is a good starting point (at least for more literature).

\subsection{Acknowledgements}
I thank God for all hours of enjoyed programming. I thank Pascal Wolkotte and Nick Papior Andersen for their programming efforts and contributions as part of the development team. I thank Stefan Tibus, who contributed the |plot shell| feature. I thank Tom Cashman for the contribution of the |reverse legend| feature. Special thanks go to Stefan Pinnow whose tests of \PGFPlots\ lead to numerous quality improvements. Furthermore, I thank Dr.~Schweitzer for many fruitful discussions and Dr.~Meine for his ideas and suggestions. Thanks as well to the many international contributors who provided feature requests or identified bugs or simply manual improvements!

Last but not least, I thank Till Tantau and Mark Wibrow for their excellent graphics (and more) package \PGF\ and \Tikz\ which is the base of \PGFPlots.

%%%%%%%%%%%%%%%%%%%%%%%%%%%%%%%%%%%%%%%%%%%%%%%%%%%%%%%%%%%%%%%%%%%%%%%%%%%%%
%
% Package pgfplots.sty documentation. 
%
% Copyright 2007/2008 by Christian Feuersaenger.
%
% This program is free software: you can redistribute it and/or modify
% it under the terms of the GNU General Public License as published by
% the Free Software Foundation, either version 3 of the License, or
% (at your option) any later version.
% 
% This program is distributed in the hope that it will be useful,
% but WITHOUT ANY WARRANTY; without even the implied warranty of
% MERCHANTABILITY or FITNESS FOR A PARTICULAR PURPOSE.  See the
% GNU General Public License for more details.
% 
% You should have received a copy of the GNU General Public License
% along with this program.  If not, see <http://www.gnu.org/licenses/>.
%
%
%%%%%%%%%%%%%%%%%%%%%%%%%%%%%%%%%%%%%%%%%%%%%%%%%%%%%%%%%%%%%%%%%%%%%%%%%%%%%
%%%%%%%%%%%%%%%%%%%%%%%%%%%%%%%%%%%%%%%%%%%%%%%%%%%%%%%%%%%%%%%%%%%%%%%%%%%%%
%
% Package pgfplots.sty documentation. 
%
% Copyright 2007/2008 by Christian Feuersaenger.
%
% This program is free software: you can redistribute it and/or modify
% it under the terms of the GNU General Public License as published by
% the Free Software Foundation, either version 3 of the License, or
% (at your option) any later version.
% 
% This program is distributed in the hope that it will be useful,
% but WITHOUT ANY WARRANTY; without even the implied warranty of
% MERCHANTABILITY or FITNESS FOR A PARTICULAR PURPOSE.  See the
% GNU General Public License for more details.
% 
% You should have received a copy of the GNU General Public License
% along with this program.  If not, see <http://www.gnu.org/licenses/>.
%
%
%%%%%%%%%%%%%%%%%%%%%%%%%%%%%%%%%%%%%%%%%%%%%%%%%%%%%%%%%%%%%%%%%%%%%%%%%%%%%
\documentclass[a4paper]{ltxdoc}

\usepackage{makeidx}

% DON't let hyperref overload the format of index and glossary. 
% I want to do that on my own in the stylefiles for makeindex...
\makeatletter
\let\@old@wrindex=\@wrindex
\makeatother

\usepackage{ifpdf}
\ifpdf
	\usepackage[pdftex]{hyperref}
\else
	\usepackage[dvipdfm]{hyperref}
\fi
\hypersetup{pdfborder={0 0 0}}

\makeatletter
\let\@wrindex=\@old@wrindex
\makeatother

% Formatiere Seitennummern im Index:
\newcommand{\indexpageno}[1]{%
	{\bfseries\hyperpage{#1}}%
}


\newcommand{\R}{\mathbb{R}}
\newcommand{\N}{\mathbb{N}}
\newcommand{\Z}{\mathbb{Z}}

\long\def\COMMENTLOWLEVEL#1\ENDCOMMENT{}
\def\ENDCOMMENT{}

\usepackage{textcomp}
\usepackage{booktabs}

\usepackage{calc}
\usepackage[formats]{listings}
%\usepackage{courier} % don't use it - the '^' character can't be copy-pasted in courier

\usepackage{array}
\lstset{%
	basicstyle=\ttfamily,
	language=[LaTeX]tex, % Seems as if \lstset{language=tex} must be invoked BEFORE loading tikz!?
	tabsize=4,
	breaklines=true,
	breakindent=0pt
}

\ifpdf
	\pdfinfo {
		/Author	(Christian Feuersaenger)
	}

\else
	\def\pgfsysdriver{pgfsys-dvipdfm.def}
\fi
%\def\pgfsysdriver{pgfsys-pdftex.def}
\usepackage{pgfplots}

\ifpdf
	\usetikzlibrary{pgfplotsclickable}
\fi

%\usepackage{fp}
% ATTENTION:
% this requires pgf version NEWER than 2.00 :
%\usetikzlibrary{fixedpointarithmetic}

\usetikzlibrary{dateplot}

\usepackage[a4paper,left=2.25cm,right=2.25cm,top=2.5cm,bottom=2.5cm,nohead]{geometry}
\usepackage{amsmath,amssymb}
\usepackage{xxcolor}
\usepackage{pifont}
\usepackage[latin1]{inputenc}
\usepackage{amsmath}
\usepackage{eurosym}
\input{pgfplots-macros}

\pgfqkeys{/codeexample}{%
	every codeexample/.style={
		width=8cm,
		/pgfplots/every axis/.append style={legend style={fill=graphicbackground}}
	},
	tabsize=4,
}
\pgfplotsset{
	every axis/.append style={width=7cm},
	filter discard warning=false,
}

\usetikzlibrary{backgrounds,patterns}
% Global styles:
\tikzset{
  shape example/.style={
    color=black!30,
    draw,
    fill=yellow!30,
    line width=.5cm,
    inner xsep=2.5cm,
    inner ysep=0.5cm}
}

\newcommand{\FIXME}[1]{\textcolor{red}{(FIXME: #1)}}

% fuer endvironment 'sidewaysfigure' bspw
% \usepackage{rotating}

\newcommand\Tikz{Ti\textit kZ}
\newcommand\PGF{\textsc{pgf}}
\newcommand\PGFPlots{\textsc{pgfplots}}
\def\PGFPlotstable{\textsc{PgfplotsTable}}


\makeindex

% Fix overful hboxes automatically:
\tolerance=2000
\emergencystretch=10pt

\tikzset{prefix=gnuplot/pgfplots_} % prefix for 'plot function'

\author{%
	Christian Feuers\"anger\footnote{\url{http://wissrech.ins.uni-bonn.de/people/feuersaenger}}\\%
	Institut f\"ur Numerische Simulation\\
	Universit\"at Bonn, Germany}


%\RequirePackage[german,english,francais]{babel}

\pgfplotsmanualexternalexpensivetrue

%\usetikzlibrary{external}
%\tikzexternalize[prefix=figures/]{pgfplots}

\title{%
	Manual for Package \PGFPlots\\
	{\small Version \pgfplotsversion}\\
	{\small\href{http://sourceforge.net/projects/pgfplots}{http://sourceforge.net/projects/pgfplots}}}

%\includeonly{pgfplots.libs}


\begin{document}

\def\plotcoords{%
\addplot coordinates {
(5,8.312e-02)    (17,2.547e-02)   (49,7.407e-03)
(129,2.102e-03)  (321,5.874e-04)  (769,1.623e-04)
(1793,4.442e-05) (4097,1.207e-05) (9217,3.261e-06)
};

\addplot coordinates{
(7,8.472e-02)    (31,3.044e-02)    (111,1.022e-02)
(351,3.303e-03)  (1023,1.039e-03)  (2815,3.196e-04)
(7423,9.658e-05) (18943,2.873e-05) (47103,8.437e-06)
};

\addplot coordinates{
(9,7.881e-02)     (49,3.243e-02)    (209,1.232e-02)
(769,4.454e-03)   (2561,1.551e-03)  (7937,5.236e-04)
(23297,1.723e-04) (65537,5.545e-05) (178177,1.751e-05)
};

\addplot coordinates{
(11,6.887e-02)    (71,3.177e-02)     (351,1.341e-02)
(1471,5.334e-03)  (5503,2.027e-03)   (18943,7.415e-04)
(61183,2.628e-04) (187903,9.063e-05) (553983,3.053e-05)
};

\addplot coordinates{
(13,5.755e-02)     (97,2.925e-02)     (545,1.351e-02)
(2561,5.842e-03)   (10625,2.397e-03)  (40193,9.414e-04)
(141569,3.564e-04) (471041,1.308e-04) 
(1496065,4.670e-05)
};
}%
\maketitle
\begin{abstract}%
\PGFPlots\ draws high--quality function plots in normal or logarithmic scaling with a user-friendly interface directly in \TeX. The user supplies axis labels, legend entries and the plot coordinates for one or more plots and \PGFPlots\ applies axis scaling, computes any logarithms and axis ticks and draws the plots, supporting line plots, scatter plots, piecewise constant plots, bar plots, area plots, mesh-- and surface plots and some more. It is based on Till Tantau's package \PGF/\Tikz.
\end{abstract}
\tableofcontents
\section{Introduction}
This package provides tools to generate plots and labeled axes easily. It draws normal plots, logplots and semi-logplots, in two and three dimensions. Axis ticks, labels, legends (in case of multiple plots) can be added with key-value options. It can cycle through a set of predefined line/marker/color specifications. In summary, its purpose is to simplify the generation of high-quality function and/or data plots, and solving the problems of
\begin{itemize}
	\item consistency of document and font type and font size,
	\item direct use of \TeX\ math mode in axis descriptions,
	\item consistency of data and figures (no third party tool necessary),
	\item inter document consistency using preamble configurations and styles.
\end{itemize}
Although not necessary, separate |.pdf| or |.eps| graphics can be generated using the |external| library developed as part of \Tikz.

\PGFPlots\ is build completely on \Tikz/\PGF. Knowledge of \Tikz\ will simplify the work with \PGFPlots, although it is not required.

\section[About PGFPlots: Preliminaries]{About {\normalfont\PGFPlots}: Preliminaries}
This section contains information about upgrades, the team, the installation (in case you need to do it manually) and troubleshooting. You may skip it completely except for the upgrade remarks.

However, note that this library requires at least \PGF\ version $2.00$. At the time of this writing, many \TeX-distributions still contain the older \PGF\ version $1.18$, so it may be necessary to install a recent \PGF\ prior to using \PGFPlots.

\subsection{Components}
\PGFPlots\ comes with two components:
\begin{enumerate}
	\item the plotting component (which you are currently reading) and
	\item the \PGFPlotstable\ component which simplifies number formatting and postprocessing of numerical tables. It comes as a separate package and has its own manual \href{file:pgfplotstable.pdf}{pgfplotstable.pdf}.
\end{enumerate}

\subsection{Upgrade remarks}
This release provides a lot of improvements which can be found in all detail in \texttt{ChangeLog} for interested readers. However, some attention is useful with respect to the following changes.

\subsubsection{New Optional Features}
\begin{enumerate}
	\item \PGFPlots\ 1.3 comes with user interface improvements. The technical distinction between ``behavior options'' and ``style options'' of older versions is no longer necessary (although still fully supported).

	\item \PGFPlots\ 1.3 has a new feature which allows to \emph{move axis labels tight to tick labels} automatically.

	Since this affects the spacing, it is not enabled be default.

	Use
\begin{codeexample}[code only]
\usepackage{pgfplots}
\pgfplotsset{compat=1.3}
\end{codeexample}
	in your preamble to benefit from the improved spacing\footnote{The |compat=1.3| setting is necessary for new features which move axis labels around like reversed axis. However, \PGFPlots\ will still work as it does in earlier versions, your documents will remain the same if you don't set it explicitly.}.
	Take a look at the next page for the precise definition of |compat|.

	\item \PGFPlots\ 1.3 now supports reversed axes. It is no longer necessary to use work--arounds with negative units.
\pgfkeys{/pdflinks/search key prefixes in/.add={/pgfplots/,}{}}

	Take a look at the |x dir=reverse| key.

	Existing work--arounds will still function properly. Use |\pgfplotsset{compat=1.3}| together with |x dir=reverse| to switch to the new version.
\end{enumerate}

\subsubsection{Old Features Which May Need Attention}
\begin{enumerate}
	\item The |scatter/classes| feature produces proper legends as of version 1.3. This may change the appearance of existing legends of plots with |scatter/classes|.

	\item Due to a small math bug in \PGF\ $2.00$, you \emph{can't} use the math expression `|-x^2|'. It is necessary to use `|0-x^2|' instead. The same holds for
		`|exp(-x^2)|'; use `|exp(0-x^2)|' instead.

		This will be fixed with \PGF\ $>2.00$.

	\item Starting with \PGFPlots\ $1.1$, |\tikzstyle| should \emph{no longer be used} to set \PGFPlots\ options.
	
	Although |\tikzstyle| is still supported for some older \PGFPlots\ options, you should replace any occurance of |\tikzstyle| with |\pgfplotsset{|\meta{style name}|/.style={|\meta{key-value-list}|}}| or the associated |/.append style| variant. See section~\ref{sec:styles} for more detail.
\end{enumerate}
I apologize for any inconvenience caused by these changes.

\begin{pgfplotskey}{compat=\mchoice{1.3,pre 1.3,default,newest} (initially default)}
	The preamble configuration 
\begin{codeexample}[code only]
\usepackage{pgfplots}
\pgfplotsset{compat=1.3}
\end{codeexample}
	allows to choose between backwards compatibility and most recent features.

	\PGFPlots\ version 1.3 comes with new features which may lead to different spacings compared to earlier versions:
	\begin{enumerate}
		\item Version 1.3 comes with the |xlabel near ticks| feature which places (in this case $x$) axis labels ``near'' the tick labels, taking the size of tick labels into account.
		
		Older versions used |xlabel absolute|, i.e.\ an absolute distance between the axis and axis labels (now matter how large or small tick labels are).

		The initial setting \emph{keeps backwards compatibility}.

		You are encouraged to use
\begin{codeexample}[code only]
\usepackage{pgfplots}
\pgfplotsset{compat=1.3}
\end{codeexample}
		in your preamble.
		
		\item Version 1.3 fixes a bug with |anchor=south west| which occurred mainly for reversed axes: the anchors where upside--down. This is fixed now.

		
		If you have written some work--around and want to keep it going, use

		|\pgfplotsset{compat/anchors=pre 1.3}|\pgfmanualpdflabel{/pgfplots/compat/anchors}{} 

		to disable the bug fix.
	\end{enumerate}

	Use |\pgfplotsset{compat=default}| to restore the factory settings.

	The setting |\pgfplotsset{compat=newest}| will always use features of the most recent version. This might result in small changes in the document's appearance.
\end{pgfplotskey}

\subsection{The Team}
\PGFPlots\ has been written mainly by Christian Feuers�nger with many improvements of Pascal Wolkotte and Nick Papior Andersen as a spare time project. We hope it is useful and provides valuable plots.

If you are interested in writing something but don't know how, consider reading the auxiliary manual \href{file:TeX-programming-notes.pdf}{TeX-programming-notes.pdf} which comes with \PGFPlots. It is far from complete, but maybe it is a good starting point (at least for more literature).

\subsection{Acknowledgements}
I thank God for all hours of enjoyed programming. I thank Pascal Wolkotte and Nick Papior Andersen for their programming efforts and contributions as part of the development team. I thank Stefan Tibus, who contributed the |plot shell| feature. I thank Tom Cashman for the contribution of the |reverse legend| feature. Special thanks go to Stefan Pinnow whose tests of \PGFPlots\ lead to numerous quality improvements. Furthermore, I thank Dr.~Schweitzer for many fruitful discussions and Dr.~Meine for his ideas and suggestions. Thanks as well to the many international contributors who provided feature requests or identified bugs or simply manual improvements!

Last but not least, I thank Till Tantau and Mark Wibrow for their excellent graphics (and more) package \PGF\ and \Tikz\ which is the base of \PGFPlots.

%%%%%%%%%%%%%%%%%%%%%%%%%%%%%%%%%%%%%%%%%%%%%%%%%%%%%%%%%%%%%%%%%%%%%%%%%%%%%
%
% Package pgfplots.sty documentation. 
%
% Copyright 2007/2008 by Christian Feuersaenger.
%
% This program is free software: you can redistribute it and/or modify
% it under the terms of the GNU General Public License as published by
% the Free Software Foundation, either version 3 of the License, or
% (at your option) any later version.
% 
% This program is distributed in the hope that it will be useful,
% but WITHOUT ANY WARRANTY; without even the implied warranty of
% MERCHANTABILITY or FITNESS FOR A PARTICULAR PURPOSE.  See the
% GNU General Public License for more details.
% 
% You should have received a copy of the GNU General Public License
% along with this program.  If not, see <http://www.gnu.org/licenses/>.
%
%
%%%%%%%%%%%%%%%%%%%%%%%%%%%%%%%%%%%%%%%%%%%%%%%%%%%%%%%%%%%%%%%%%%%%%%%%%%%%%
\input{pgfplots.preamble.tex}
%\RequirePackage[german,english,francais]{babel}

\pgfplotsmanualexternalexpensivetrue

%\usetikzlibrary{external}
%\tikzexternalize[prefix=figures/]{pgfplots}

\title{%
	Manual for Package \PGFPlots\\
	{\small Version \pgfplotsversion}\\
	{\small\href{http://sourceforge.net/projects/pgfplots}{http://sourceforge.net/projects/pgfplots}}}

%\includeonly{pgfplots.libs}


\begin{document}

\def\plotcoords{%
\addplot coordinates {
(5,8.312e-02)    (17,2.547e-02)   (49,7.407e-03)
(129,2.102e-03)  (321,5.874e-04)  (769,1.623e-04)
(1793,4.442e-05) (4097,1.207e-05) (9217,3.261e-06)
};

\addplot coordinates{
(7,8.472e-02)    (31,3.044e-02)    (111,1.022e-02)
(351,3.303e-03)  (1023,1.039e-03)  (2815,3.196e-04)
(7423,9.658e-05) (18943,2.873e-05) (47103,8.437e-06)
};

\addplot coordinates{
(9,7.881e-02)     (49,3.243e-02)    (209,1.232e-02)
(769,4.454e-03)   (2561,1.551e-03)  (7937,5.236e-04)
(23297,1.723e-04) (65537,5.545e-05) (178177,1.751e-05)
};

\addplot coordinates{
(11,6.887e-02)    (71,3.177e-02)     (351,1.341e-02)
(1471,5.334e-03)  (5503,2.027e-03)   (18943,7.415e-04)
(61183,2.628e-04) (187903,9.063e-05) (553983,3.053e-05)
};

\addplot coordinates{
(13,5.755e-02)     (97,2.925e-02)     (545,1.351e-02)
(2561,5.842e-03)   (10625,2.397e-03)  (40193,9.414e-04)
(141569,3.564e-04) (471041,1.308e-04) 
(1496065,4.670e-05)
};
}%
\maketitle
\begin{abstract}%
\PGFPlots\ draws high--quality function plots in normal or logarithmic scaling with a user-friendly interface directly in \TeX. The user supplies axis labels, legend entries and the plot coordinates for one or more plots and \PGFPlots\ applies axis scaling, computes any logarithms and axis ticks and draws the plots, supporting line plots, scatter plots, piecewise constant plots, bar plots, area plots, mesh-- and surface plots and some more. It is based on Till Tantau's package \PGF/\Tikz.
\end{abstract}
\tableofcontents
\section{Introduction}
This package provides tools to generate plots and labeled axes easily. It draws normal plots, logplots and semi-logplots, in two and three dimensions. Axis ticks, labels, legends (in case of multiple plots) can be added with key-value options. It can cycle through a set of predefined line/marker/color specifications. In summary, its purpose is to simplify the generation of high-quality function and/or data plots, and solving the problems of
\begin{itemize}
	\item consistency of document and font type and font size,
	\item direct use of \TeX\ math mode in axis descriptions,
	\item consistency of data and figures (no third party tool necessary),
	\item inter document consistency using preamble configurations and styles.
\end{itemize}
Although not necessary, separate |.pdf| or |.eps| graphics can be generated using the |external| library developed as part of \Tikz.

\PGFPlots\ is build completely on \Tikz/\PGF. Knowledge of \Tikz\ will simplify the work with \PGFPlots, although it is not required.

\section[About PGFPlots: Preliminaries]{About {\normalfont\PGFPlots}: Preliminaries}
This section contains information about upgrades, the team, the installation (in case you need to do it manually) and troubleshooting. You may skip it completely except for the upgrade remarks.

However, note that this library requires at least \PGF\ version $2.00$. At the time of this writing, many \TeX-distributions still contain the older \PGF\ version $1.18$, so it may be necessary to install a recent \PGF\ prior to using \PGFPlots.

\subsection{Components}
\PGFPlots\ comes with two components:
\begin{enumerate}
	\item the plotting component (which you are currently reading) and
	\item the \PGFPlotstable\ component which simplifies number formatting and postprocessing of numerical tables. It comes as a separate package and has its own manual \href{file:pgfplotstable.pdf}{pgfplotstable.pdf}.
\end{enumerate}

\subsection{Upgrade remarks}
This release provides a lot of improvements which can be found in all detail in \texttt{ChangeLog} for interested readers. However, some attention is useful with respect to the following changes.

\subsubsection{New Optional Features}
\begin{enumerate}
	\item \PGFPlots\ 1.3 comes with user interface improvements. The technical distinction between ``behavior options'' and ``style options'' of older versions is no longer necessary (although still fully supported).

	\item \PGFPlots\ 1.3 has a new feature which allows to \emph{move axis labels tight to tick labels} automatically.

	Since this affects the spacing, it is not enabled be default.

	Use
\begin{codeexample}[code only]
\usepackage{pgfplots}
\pgfplotsset{compat=1.3}
\end{codeexample}
	in your preamble to benefit from the improved spacing\footnote{The |compat=1.3| setting is necessary for new features which move axis labels around like reversed axis. However, \PGFPlots\ will still work as it does in earlier versions, your documents will remain the same if you don't set it explicitly.}.
	Take a look at the next page for the precise definition of |compat|.

	\item \PGFPlots\ 1.3 now supports reversed axes. It is no longer necessary to use work--arounds with negative units.
\pgfkeys{/pdflinks/search key prefixes in/.add={/pgfplots/,}{}}

	Take a look at the |x dir=reverse| key.

	Existing work--arounds will still function properly. Use |\pgfplotsset{compat=1.3}| together with |x dir=reverse| to switch to the new version.
\end{enumerate}

\subsubsection{Old Features Which May Need Attention}
\begin{enumerate}
	\item The |scatter/classes| feature produces proper legends as of version 1.3. This may change the appearance of existing legends of plots with |scatter/classes|.

	\item Due to a small math bug in \PGF\ $2.00$, you \emph{can't} use the math expression `|-x^2|'. It is necessary to use `|0-x^2|' instead. The same holds for
		`|exp(-x^2)|'; use `|exp(0-x^2)|' instead.

		This will be fixed with \PGF\ $>2.00$.

	\item Starting with \PGFPlots\ $1.1$, |\tikzstyle| should \emph{no longer be used} to set \PGFPlots\ options.
	
	Although |\tikzstyle| is still supported for some older \PGFPlots\ options, you should replace any occurance of |\tikzstyle| with |\pgfplotsset{|\meta{style name}|/.style={|\meta{key-value-list}|}}| or the associated |/.append style| variant. See section~\ref{sec:styles} for more detail.
\end{enumerate}
I apologize for any inconvenience caused by these changes.

\begin{pgfplotskey}{compat=\mchoice{1.3,pre 1.3,default,newest} (initially default)}
	The preamble configuration 
\begin{codeexample}[code only]
\usepackage{pgfplots}
\pgfplotsset{compat=1.3}
\end{codeexample}
	allows to choose between backwards compatibility and most recent features.

	\PGFPlots\ version 1.3 comes with new features which may lead to different spacings compared to earlier versions:
	\begin{enumerate}
		\item Version 1.3 comes with the |xlabel near ticks| feature which places (in this case $x$) axis labels ``near'' the tick labels, taking the size of tick labels into account.
		
		Older versions used |xlabel absolute|, i.e.\ an absolute distance between the axis and axis labels (now matter how large or small tick labels are).

		The initial setting \emph{keeps backwards compatibility}.

		You are encouraged to use
\begin{codeexample}[code only]
\usepackage{pgfplots}
\pgfplotsset{compat=1.3}
\end{codeexample}
		in your preamble.
		
		\item Version 1.3 fixes a bug with |anchor=south west| which occurred mainly for reversed axes: the anchors where upside--down. This is fixed now.

		
		If you have written some work--around and want to keep it going, use

		|\pgfplotsset{compat/anchors=pre 1.3}|\pgfmanualpdflabel{/pgfplots/compat/anchors}{} 

		to disable the bug fix.
	\end{enumerate}

	Use |\pgfplotsset{compat=default}| to restore the factory settings.

	The setting |\pgfplotsset{compat=newest}| will always use features of the most recent version. This might result in small changes in the document's appearance.
\end{pgfplotskey}

\subsection{The Team}
\PGFPlots\ has been written mainly by Christian Feuers�nger with many improvements of Pascal Wolkotte and Nick Papior Andersen as a spare time project. We hope it is useful and provides valuable plots.

If you are interested in writing something but don't know how, consider reading the auxiliary manual \href{file:TeX-programming-notes.pdf}{TeX-programming-notes.pdf} which comes with \PGFPlots. It is far from complete, but maybe it is a good starting point (at least for more literature).

\subsection{Acknowledgements}
I thank God for all hours of enjoyed programming. I thank Pascal Wolkotte and Nick Papior Andersen for their programming efforts and contributions as part of the development team. I thank Stefan Tibus, who contributed the |plot shell| feature. I thank Tom Cashman for the contribution of the |reverse legend| feature. Special thanks go to Stefan Pinnow whose tests of \PGFPlots\ lead to numerous quality improvements. Furthermore, I thank Dr.~Schweitzer for many fruitful discussions and Dr.~Meine for his ideas and suggestions. Thanks as well to the many international contributors who provided feature requests or identified bugs or simply manual improvements!

Last but not least, I thank Till Tantau and Mark Wibrow for their excellent graphics (and more) package \PGF\ and \Tikz\ which is the base of \PGFPlots.

\include{pgfplots.install}%
\include{pgfplots.intro}%
\include{pgfplots.reference}%
\include{pgfplots.libs}%
\include{pgfplots.resources}%
\include{pgfplots.importexport}%
\include{pgfplots.basic.reference}%

\printindex

\bibliographystyle{abbrv} %gerapali} %gerabbrv} %gerunsrt.bst} %gerabbrv}% gerplain}
\nocite{pgfplotstable}
\nocite{programmingnotes}
\bibliography{pgfplots}
\end{document}
%
%%%%%%%%%%%%%%%%%%%%%%%%%%%%%%%%%%%%%%%%%%%%%%%%%%%%%%%%%%%%%%%%%%%%%%%%%%%%%
%
% Package pgfplots.sty documentation. 
%
% Copyright 2007/2008 by Christian Feuersaenger.
%
% This program is free software: you can redistribute it and/or modify
% it under the terms of the GNU General Public License as published by
% the Free Software Foundation, either version 3 of the License, or
% (at your option) any later version.
% 
% This program is distributed in the hope that it will be useful,
% but WITHOUT ANY WARRANTY; without even the implied warranty of
% MERCHANTABILITY or FITNESS FOR A PARTICULAR PURPOSE.  See the
% GNU General Public License for more details.
% 
% You should have received a copy of the GNU General Public License
% along with this program.  If not, see <http://www.gnu.org/licenses/>.
%
%
%%%%%%%%%%%%%%%%%%%%%%%%%%%%%%%%%%%%%%%%%%%%%%%%%%%%%%%%%%%%%%%%%%%%%%%%%%%%%
\input{pgfplots.preamble.tex}
%\RequirePackage[german,english,francais]{babel}

\pgfplotsmanualexternalexpensivetrue

%\usetikzlibrary{external}
%\tikzexternalize[prefix=figures/]{pgfplots}

\title{%
	Manual for Package \PGFPlots\\
	{\small Version \pgfplotsversion}\\
	{\small\href{http://sourceforge.net/projects/pgfplots}{http://sourceforge.net/projects/pgfplots}}}

%\includeonly{pgfplots.libs}


\begin{document}

\def\plotcoords{%
\addplot coordinates {
(5,8.312e-02)    (17,2.547e-02)   (49,7.407e-03)
(129,2.102e-03)  (321,5.874e-04)  (769,1.623e-04)
(1793,4.442e-05) (4097,1.207e-05) (9217,3.261e-06)
};

\addplot coordinates{
(7,8.472e-02)    (31,3.044e-02)    (111,1.022e-02)
(351,3.303e-03)  (1023,1.039e-03)  (2815,3.196e-04)
(7423,9.658e-05) (18943,2.873e-05) (47103,8.437e-06)
};

\addplot coordinates{
(9,7.881e-02)     (49,3.243e-02)    (209,1.232e-02)
(769,4.454e-03)   (2561,1.551e-03)  (7937,5.236e-04)
(23297,1.723e-04) (65537,5.545e-05) (178177,1.751e-05)
};

\addplot coordinates{
(11,6.887e-02)    (71,3.177e-02)     (351,1.341e-02)
(1471,5.334e-03)  (5503,2.027e-03)   (18943,7.415e-04)
(61183,2.628e-04) (187903,9.063e-05) (553983,3.053e-05)
};

\addplot coordinates{
(13,5.755e-02)     (97,2.925e-02)     (545,1.351e-02)
(2561,5.842e-03)   (10625,2.397e-03)  (40193,9.414e-04)
(141569,3.564e-04) (471041,1.308e-04) 
(1496065,4.670e-05)
};
}%
\maketitle
\begin{abstract}%
\PGFPlots\ draws high--quality function plots in normal or logarithmic scaling with a user-friendly interface directly in \TeX. The user supplies axis labels, legend entries and the plot coordinates for one or more plots and \PGFPlots\ applies axis scaling, computes any logarithms and axis ticks and draws the plots, supporting line plots, scatter plots, piecewise constant plots, bar plots, area plots, mesh-- and surface plots and some more. It is based on Till Tantau's package \PGF/\Tikz.
\end{abstract}
\tableofcontents
\section{Introduction}
This package provides tools to generate plots and labeled axes easily. It draws normal plots, logplots and semi-logplots, in two and three dimensions. Axis ticks, labels, legends (in case of multiple plots) can be added with key-value options. It can cycle through a set of predefined line/marker/color specifications. In summary, its purpose is to simplify the generation of high-quality function and/or data plots, and solving the problems of
\begin{itemize}
	\item consistency of document and font type and font size,
	\item direct use of \TeX\ math mode in axis descriptions,
	\item consistency of data and figures (no third party tool necessary),
	\item inter document consistency using preamble configurations and styles.
\end{itemize}
Although not necessary, separate |.pdf| or |.eps| graphics can be generated using the |external| library developed as part of \Tikz.

\PGFPlots\ is build completely on \Tikz/\PGF. Knowledge of \Tikz\ will simplify the work with \PGFPlots, although it is not required.

\section[About PGFPlots: Preliminaries]{About {\normalfont\PGFPlots}: Preliminaries}
This section contains information about upgrades, the team, the installation (in case you need to do it manually) and troubleshooting. You may skip it completely except for the upgrade remarks.

However, note that this library requires at least \PGF\ version $2.00$. At the time of this writing, many \TeX-distributions still contain the older \PGF\ version $1.18$, so it may be necessary to install a recent \PGF\ prior to using \PGFPlots.

\subsection{Components}
\PGFPlots\ comes with two components:
\begin{enumerate}
	\item the plotting component (which you are currently reading) and
	\item the \PGFPlotstable\ component which simplifies number formatting and postprocessing of numerical tables. It comes as a separate package and has its own manual \href{file:pgfplotstable.pdf}{pgfplotstable.pdf}.
\end{enumerate}

\subsection{Upgrade remarks}
This release provides a lot of improvements which can be found in all detail in \texttt{ChangeLog} for interested readers. However, some attention is useful with respect to the following changes.

\subsubsection{New Optional Features}
\begin{enumerate}
	\item \PGFPlots\ 1.3 comes with user interface improvements. The technical distinction between ``behavior options'' and ``style options'' of older versions is no longer necessary (although still fully supported).

	\item \PGFPlots\ 1.3 has a new feature which allows to \emph{move axis labels tight to tick labels} automatically.

	Since this affects the spacing, it is not enabled be default.

	Use
\begin{codeexample}[code only]
\usepackage{pgfplots}
\pgfplotsset{compat=1.3}
\end{codeexample}
	in your preamble to benefit from the improved spacing\footnote{The |compat=1.3| setting is necessary for new features which move axis labels around like reversed axis. However, \PGFPlots\ will still work as it does in earlier versions, your documents will remain the same if you don't set it explicitly.}.
	Take a look at the next page for the precise definition of |compat|.

	\item \PGFPlots\ 1.3 now supports reversed axes. It is no longer necessary to use work--arounds with negative units.
\pgfkeys{/pdflinks/search key prefixes in/.add={/pgfplots/,}{}}

	Take a look at the |x dir=reverse| key.

	Existing work--arounds will still function properly. Use |\pgfplotsset{compat=1.3}| together with |x dir=reverse| to switch to the new version.
\end{enumerate}

\subsubsection{Old Features Which May Need Attention}
\begin{enumerate}
	\item The |scatter/classes| feature produces proper legends as of version 1.3. This may change the appearance of existing legends of plots with |scatter/classes|.

	\item Due to a small math bug in \PGF\ $2.00$, you \emph{can't} use the math expression `|-x^2|'. It is necessary to use `|0-x^2|' instead. The same holds for
		`|exp(-x^2)|'; use `|exp(0-x^2)|' instead.

		This will be fixed with \PGF\ $>2.00$.

	\item Starting with \PGFPlots\ $1.1$, |\tikzstyle| should \emph{no longer be used} to set \PGFPlots\ options.
	
	Although |\tikzstyle| is still supported for some older \PGFPlots\ options, you should replace any occurance of |\tikzstyle| with |\pgfplotsset{|\meta{style name}|/.style={|\meta{key-value-list}|}}| or the associated |/.append style| variant. See section~\ref{sec:styles} for more detail.
\end{enumerate}
I apologize for any inconvenience caused by these changes.

\begin{pgfplotskey}{compat=\mchoice{1.3,pre 1.3,default,newest} (initially default)}
	The preamble configuration 
\begin{codeexample}[code only]
\usepackage{pgfplots}
\pgfplotsset{compat=1.3}
\end{codeexample}
	allows to choose between backwards compatibility and most recent features.

	\PGFPlots\ version 1.3 comes with new features which may lead to different spacings compared to earlier versions:
	\begin{enumerate}
		\item Version 1.3 comes with the |xlabel near ticks| feature which places (in this case $x$) axis labels ``near'' the tick labels, taking the size of tick labels into account.
		
		Older versions used |xlabel absolute|, i.e.\ an absolute distance between the axis and axis labels (now matter how large or small tick labels are).

		The initial setting \emph{keeps backwards compatibility}.

		You are encouraged to use
\begin{codeexample}[code only]
\usepackage{pgfplots}
\pgfplotsset{compat=1.3}
\end{codeexample}
		in your preamble.
		
		\item Version 1.3 fixes a bug with |anchor=south west| which occurred mainly for reversed axes: the anchors where upside--down. This is fixed now.

		
		If you have written some work--around and want to keep it going, use

		|\pgfplotsset{compat/anchors=pre 1.3}|\pgfmanualpdflabel{/pgfplots/compat/anchors}{} 

		to disable the bug fix.
	\end{enumerate}

	Use |\pgfplotsset{compat=default}| to restore the factory settings.

	The setting |\pgfplotsset{compat=newest}| will always use features of the most recent version. This might result in small changes in the document's appearance.
\end{pgfplotskey}

\subsection{The Team}
\PGFPlots\ has been written mainly by Christian Feuers�nger with many improvements of Pascal Wolkotte and Nick Papior Andersen as a spare time project. We hope it is useful and provides valuable plots.

If you are interested in writing something but don't know how, consider reading the auxiliary manual \href{file:TeX-programming-notes.pdf}{TeX-programming-notes.pdf} which comes with \PGFPlots. It is far from complete, but maybe it is a good starting point (at least for more literature).

\subsection{Acknowledgements}
I thank God for all hours of enjoyed programming. I thank Pascal Wolkotte and Nick Papior Andersen for their programming efforts and contributions as part of the development team. I thank Stefan Tibus, who contributed the |plot shell| feature. I thank Tom Cashman for the contribution of the |reverse legend| feature. Special thanks go to Stefan Pinnow whose tests of \PGFPlots\ lead to numerous quality improvements. Furthermore, I thank Dr.~Schweitzer for many fruitful discussions and Dr.~Meine for his ideas and suggestions. Thanks as well to the many international contributors who provided feature requests or identified bugs or simply manual improvements!

Last but not least, I thank Till Tantau and Mark Wibrow for their excellent graphics (and more) package \PGF\ and \Tikz\ which is the base of \PGFPlots.

\include{pgfplots.install}%
\include{pgfplots.intro}%
\include{pgfplots.reference}%
\include{pgfplots.libs}%
\include{pgfplots.resources}%
\include{pgfplots.importexport}%
\include{pgfplots.basic.reference}%

\printindex

\bibliographystyle{abbrv} %gerapali} %gerabbrv} %gerunsrt.bst} %gerabbrv}% gerplain}
\nocite{pgfplotstable}
\nocite{programmingnotes}
\bibliography{pgfplots}
\end{document}
%
%%%%%%%%%%%%%%%%%%%%%%%%%%%%%%%%%%%%%%%%%%%%%%%%%%%%%%%%%%%%%%%%%%%%%%%%%%%%%
%
% Package pgfplots.sty documentation. 
%
% Copyright 2007/2008 by Christian Feuersaenger.
%
% This program is free software: you can redistribute it and/or modify
% it under the terms of the GNU General Public License as published by
% the Free Software Foundation, either version 3 of the License, or
% (at your option) any later version.
% 
% This program is distributed in the hope that it will be useful,
% but WITHOUT ANY WARRANTY; without even the implied warranty of
% MERCHANTABILITY or FITNESS FOR A PARTICULAR PURPOSE.  See the
% GNU General Public License for more details.
% 
% You should have received a copy of the GNU General Public License
% along with this program.  If not, see <http://www.gnu.org/licenses/>.
%
%
%%%%%%%%%%%%%%%%%%%%%%%%%%%%%%%%%%%%%%%%%%%%%%%%%%%%%%%%%%%%%%%%%%%%%%%%%%%%%
\input{pgfplots.preamble.tex}
%\RequirePackage[german,english,francais]{babel}

\pgfplotsmanualexternalexpensivetrue

%\usetikzlibrary{external}
%\tikzexternalize[prefix=figures/]{pgfplots}

\title{%
	Manual for Package \PGFPlots\\
	{\small Version \pgfplotsversion}\\
	{\small\href{http://sourceforge.net/projects/pgfplots}{http://sourceforge.net/projects/pgfplots}}}

%\includeonly{pgfplots.libs}


\begin{document}

\def\plotcoords{%
\addplot coordinates {
(5,8.312e-02)    (17,2.547e-02)   (49,7.407e-03)
(129,2.102e-03)  (321,5.874e-04)  (769,1.623e-04)
(1793,4.442e-05) (4097,1.207e-05) (9217,3.261e-06)
};

\addplot coordinates{
(7,8.472e-02)    (31,3.044e-02)    (111,1.022e-02)
(351,3.303e-03)  (1023,1.039e-03)  (2815,3.196e-04)
(7423,9.658e-05) (18943,2.873e-05) (47103,8.437e-06)
};

\addplot coordinates{
(9,7.881e-02)     (49,3.243e-02)    (209,1.232e-02)
(769,4.454e-03)   (2561,1.551e-03)  (7937,5.236e-04)
(23297,1.723e-04) (65537,5.545e-05) (178177,1.751e-05)
};

\addplot coordinates{
(11,6.887e-02)    (71,3.177e-02)     (351,1.341e-02)
(1471,5.334e-03)  (5503,2.027e-03)   (18943,7.415e-04)
(61183,2.628e-04) (187903,9.063e-05) (553983,3.053e-05)
};

\addplot coordinates{
(13,5.755e-02)     (97,2.925e-02)     (545,1.351e-02)
(2561,5.842e-03)   (10625,2.397e-03)  (40193,9.414e-04)
(141569,3.564e-04) (471041,1.308e-04) 
(1496065,4.670e-05)
};
}%
\maketitle
\begin{abstract}%
\PGFPlots\ draws high--quality function plots in normal or logarithmic scaling with a user-friendly interface directly in \TeX. The user supplies axis labels, legend entries and the plot coordinates for one or more plots and \PGFPlots\ applies axis scaling, computes any logarithms and axis ticks and draws the plots, supporting line plots, scatter plots, piecewise constant plots, bar plots, area plots, mesh-- and surface plots and some more. It is based on Till Tantau's package \PGF/\Tikz.
\end{abstract}
\tableofcontents
\section{Introduction}
This package provides tools to generate plots and labeled axes easily. It draws normal plots, logplots and semi-logplots, in two and three dimensions. Axis ticks, labels, legends (in case of multiple plots) can be added with key-value options. It can cycle through a set of predefined line/marker/color specifications. In summary, its purpose is to simplify the generation of high-quality function and/or data plots, and solving the problems of
\begin{itemize}
	\item consistency of document and font type and font size,
	\item direct use of \TeX\ math mode in axis descriptions,
	\item consistency of data and figures (no third party tool necessary),
	\item inter document consistency using preamble configurations and styles.
\end{itemize}
Although not necessary, separate |.pdf| or |.eps| graphics can be generated using the |external| library developed as part of \Tikz.

\PGFPlots\ is build completely on \Tikz/\PGF. Knowledge of \Tikz\ will simplify the work with \PGFPlots, although it is not required.

\section[About PGFPlots: Preliminaries]{About {\normalfont\PGFPlots}: Preliminaries}
This section contains information about upgrades, the team, the installation (in case you need to do it manually) and troubleshooting. You may skip it completely except for the upgrade remarks.

However, note that this library requires at least \PGF\ version $2.00$. At the time of this writing, many \TeX-distributions still contain the older \PGF\ version $1.18$, so it may be necessary to install a recent \PGF\ prior to using \PGFPlots.

\subsection{Components}
\PGFPlots\ comes with two components:
\begin{enumerate}
	\item the plotting component (which you are currently reading) and
	\item the \PGFPlotstable\ component which simplifies number formatting and postprocessing of numerical tables. It comes as a separate package and has its own manual \href{file:pgfplotstable.pdf}{pgfplotstable.pdf}.
\end{enumerate}

\subsection{Upgrade remarks}
This release provides a lot of improvements which can be found in all detail in \texttt{ChangeLog} for interested readers. However, some attention is useful with respect to the following changes.

\subsubsection{New Optional Features}
\begin{enumerate}
	\item \PGFPlots\ 1.3 comes with user interface improvements. The technical distinction between ``behavior options'' and ``style options'' of older versions is no longer necessary (although still fully supported).

	\item \PGFPlots\ 1.3 has a new feature which allows to \emph{move axis labels tight to tick labels} automatically.

	Since this affects the spacing, it is not enabled be default.

	Use
\begin{codeexample}[code only]
\usepackage{pgfplots}
\pgfplotsset{compat=1.3}
\end{codeexample}
	in your preamble to benefit from the improved spacing\footnote{The |compat=1.3| setting is necessary for new features which move axis labels around like reversed axis. However, \PGFPlots\ will still work as it does in earlier versions, your documents will remain the same if you don't set it explicitly.}.
	Take a look at the next page for the precise definition of |compat|.

	\item \PGFPlots\ 1.3 now supports reversed axes. It is no longer necessary to use work--arounds with negative units.
\pgfkeys{/pdflinks/search key prefixes in/.add={/pgfplots/,}{}}

	Take a look at the |x dir=reverse| key.

	Existing work--arounds will still function properly. Use |\pgfplotsset{compat=1.3}| together with |x dir=reverse| to switch to the new version.
\end{enumerate}

\subsubsection{Old Features Which May Need Attention}
\begin{enumerate}
	\item The |scatter/classes| feature produces proper legends as of version 1.3. This may change the appearance of existing legends of plots with |scatter/classes|.

	\item Due to a small math bug in \PGF\ $2.00$, you \emph{can't} use the math expression `|-x^2|'. It is necessary to use `|0-x^2|' instead. The same holds for
		`|exp(-x^2)|'; use `|exp(0-x^2)|' instead.

		This will be fixed with \PGF\ $>2.00$.

	\item Starting with \PGFPlots\ $1.1$, |\tikzstyle| should \emph{no longer be used} to set \PGFPlots\ options.
	
	Although |\tikzstyle| is still supported for some older \PGFPlots\ options, you should replace any occurance of |\tikzstyle| with |\pgfplotsset{|\meta{style name}|/.style={|\meta{key-value-list}|}}| or the associated |/.append style| variant. See section~\ref{sec:styles} for more detail.
\end{enumerate}
I apologize for any inconvenience caused by these changes.

\begin{pgfplotskey}{compat=\mchoice{1.3,pre 1.3,default,newest} (initially default)}
	The preamble configuration 
\begin{codeexample}[code only]
\usepackage{pgfplots}
\pgfplotsset{compat=1.3}
\end{codeexample}
	allows to choose between backwards compatibility and most recent features.

	\PGFPlots\ version 1.3 comes with new features which may lead to different spacings compared to earlier versions:
	\begin{enumerate}
		\item Version 1.3 comes with the |xlabel near ticks| feature which places (in this case $x$) axis labels ``near'' the tick labels, taking the size of tick labels into account.
		
		Older versions used |xlabel absolute|, i.e.\ an absolute distance between the axis and axis labels (now matter how large or small tick labels are).

		The initial setting \emph{keeps backwards compatibility}.

		You are encouraged to use
\begin{codeexample}[code only]
\usepackage{pgfplots}
\pgfplotsset{compat=1.3}
\end{codeexample}
		in your preamble.
		
		\item Version 1.3 fixes a bug with |anchor=south west| which occurred mainly for reversed axes: the anchors where upside--down. This is fixed now.

		
		If you have written some work--around and want to keep it going, use

		|\pgfplotsset{compat/anchors=pre 1.3}|\pgfmanualpdflabel{/pgfplots/compat/anchors}{} 

		to disable the bug fix.
	\end{enumerate}

	Use |\pgfplotsset{compat=default}| to restore the factory settings.

	The setting |\pgfplotsset{compat=newest}| will always use features of the most recent version. This might result in small changes in the document's appearance.
\end{pgfplotskey}

\subsection{The Team}
\PGFPlots\ has been written mainly by Christian Feuers�nger with many improvements of Pascal Wolkotte and Nick Papior Andersen as a spare time project. We hope it is useful and provides valuable plots.

If you are interested in writing something but don't know how, consider reading the auxiliary manual \href{file:TeX-programming-notes.pdf}{TeX-programming-notes.pdf} which comes with \PGFPlots. It is far from complete, but maybe it is a good starting point (at least for more literature).

\subsection{Acknowledgements}
I thank God for all hours of enjoyed programming. I thank Pascal Wolkotte and Nick Papior Andersen for their programming efforts and contributions as part of the development team. I thank Stefan Tibus, who contributed the |plot shell| feature. I thank Tom Cashman for the contribution of the |reverse legend| feature. Special thanks go to Stefan Pinnow whose tests of \PGFPlots\ lead to numerous quality improvements. Furthermore, I thank Dr.~Schweitzer for many fruitful discussions and Dr.~Meine for his ideas and suggestions. Thanks as well to the many international contributors who provided feature requests or identified bugs or simply manual improvements!

Last but not least, I thank Till Tantau and Mark Wibrow for their excellent graphics (and more) package \PGF\ and \Tikz\ which is the base of \PGFPlots.

\include{pgfplots.install}%
\include{pgfplots.intro}%
\include{pgfplots.reference}%
\include{pgfplots.libs}%
\include{pgfplots.resources}%
\include{pgfplots.importexport}%
\include{pgfplots.basic.reference}%

\printindex

\bibliographystyle{abbrv} %gerapali} %gerabbrv} %gerunsrt.bst} %gerabbrv}% gerplain}
\nocite{pgfplotstable}
\nocite{programmingnotes}
\bibliography{pgfplots}
\end{document}
%
%%%%%%%%%%%%%%%%%%%%%%%%%%%%%%%%%%%%%%%%%%%%%%%%%%%%%%%%%%%%%%%%%%%%%%%%%%%%%
%
% Package pgfplots.sty documentation. 
%
% Copyright 2007/2008 by Christian Feuersaenger.
%
% This program is free software: you can redistribute it and/or modify
% it under the terms of the GNU General Public License as published by
% the Free Software Foundation, either version 3 of the License, or
% (at your option) any later version.
% 
% This program is distributed in the hope that it will be useful,
% but WITHOUT ANY WARRANTY; without even the implied warranty of
% MERCHANTABILITY or FITNESS FOR A PARTICULAR PURPOSE.  See the
% GNU General Public License for more details.
% 
% You should have received a copy of the GNU General Public License
% along with this program.  If not, see <http://www.gnu.org/licenses/>.
%
%
%%%%%%%%%%%%%%%%%%%%%%%%%%%%%%%%%%%%%%%%%%%%%%%%%%%%%%%%%%%%%%%%%%%%%%%%%%%%%
\input{pgfplots.preamble.tex}
%\RequirePackage[german,english,francais]{babel}

\pgfplotsmanualexternalexpensivetrue

%\usetikzlibrary{external}
%\tikzexternalize[prefix=figures/]{pgfplots}

\title{%
	Manual for Package \PGFPlots\\
	{\small Version \pgfplotsversion}\\
	{\small\href{http://sourceforge.net/projects/pgfplots}{http://sourceforge.net/projects/pgfplots}}}

%\includeonly{pgfplots.libs}


\begin{document}

\def\plotcoords{%
\addplot coordinates {
(5,8.312e-02)    (17,2.547e-02)   (49,7.407e-03)
(129,2.102e-03)  (321,5.874e-04)  (769,1.623e-04)
(1793,4.442e-05) (4097,1.207e-05) (9217,3.261e-06)
};

\addplot coordinates{
(7,8.472e-02)    (31,3.044e-02)    (111,1.022e-02)
(351,3.303e-03)  (1023,1.039e-03)  (2815,3.196e-04)
(7423,9.658e-05) (18943,2.873e-05) (47103,8.437e-06)
};

\addplot coordinates{
(9,7.881e-02)     (49,3.243e-02)    (209,1.232e-02)
(769,4.454e-03)   (2561,1.551e-03)  (7937,5.236e-04)
(23297,1.723e-04) (65537,5.545e-05) (178177,1.751e-05)
};

\addplot coordinates{
(11,6.887e-02)    (71,3.177e-02)     (351,1.341e-02)
(1471,5.334e-03)  (5503,2.027e-03)   (18943,7.415e-04)
(61183,2.628e-04) (187903,9.063e-05) (553983,3.053e-05)
};

\addplot coordinates{
(13,5.755e-02)     (97,2.925e-02)     (545,1.351e-02)
(2561,5.842e-03)   (10625,2.397e-03)  (40193,9.414e-04)
(141569,3.564e-04) (471041,1.308e-04) 
(1496065,4.670e-05)
};
}%
\maketitle
\begin{abstract}%
\PGFPlots\ draws high--quality function plots in normal or logarithmic scaling with a user-friendly interface directly in \TeX. The user supplies axis labels, legend entries and the plot coordinates for one or more plots and \PGFPlots\ applies axis scaling, computes any logarithms and axis ticks and draws the plots, supporting line plots, scatter plots, piecewise constant plots, bar plots, area plots, mesh-- and surface plots and some more. It is based on Till Tantau's package \PGF/\Tikz.
\end{abstract}
\tableofcontents
\section{Introduction}
This package provides tools to generate plots and labeled axes easily. It draws normal plots, logplots and semi-logplots, in two and three dimensions. Axis ticks, labels, legends (in case of multiple plots) can be added with key-value options. It can cycle through a set of predefined line/marker/color specifications. In summary, its purpose is to simplify the generation of high-quality function and/or data plots, and solving the problems of
\begin{itemize}
	\item consistency of document and font type and font size,
	\item direct use of \TeX\ math mode in axis descriptions,
	\item consistency of data and figures (no third party tool necessary),
	\item inter document consistency using preamble configurations and styles.
\end{itemize}
Although not necessary, separate |.pdf| or |.eps| graphics can be generated using the |external| library developed as part of \Tikz.

\PGFPlots\ is build completely on \Tikz/\PGF. Knowledge of \Tikz\ will simplify the work with \PGFPlots, although it is not required.

\section[About PGFPlots: Preliminaries]{About {\normalfont\PGFPlots}: Preliminaries}
This section contains information about upgrades, the team, the installation (in case you need to do it manually) and troubleshooting. You may skip it completely except for the upgrade remarks.

However, note that this library requires at least \PGF\ version $2.00$. At the time of this writing, many \TeX-distributions still contain the older \PGF\ version $1.18$, so it may be necessary to install a recent \PGF\ prior to using \PGFPlots.

\subsection{Components}
\PGFPlots\ comes with two components:
\begin{enumerate}
	\item the plotting component (which you are currently reading) and
	\item the \PGFPlotstable\ component which simplifies number formatting and postprocessing of numerical tables. It comes as a separate package and has its own manual \href{file:pgfplotstable.pdf}{pgfplotstable.pdf}.
\end{enumerate}

\subsection{Upgrade remarks}
This release provides a lot of improvements which can be found in all detail in \texttt{ChangeLog} for interested readers. However, some attention is useful with respect to the following changes.

\subsubsection{New Optional Features}
\begin{enumerate}
	\item \PGFPlots\ 1.3 comes with user interface improvements. The technical distinction between ``behavior options'' and ``style options'' of older versions is no longer necessary (although still fully supported).

	\item \PGFPlots\ 1.3 has a new feature which allows to \emph{move axis labels tight to tick labels} automatically.

	Since this affects the spacing, it is not enabled be default.

	Use
\begin{codeexample}[code only]
\usepackage{pgfplots}
\pgfplotsset{compat=1.3}
\end{codeexample}
	in your preamble to benefit from the improved spacing\footnote{The |compat=1.3| setting is necessary for new features which move axis labels around like reversed axis. However, \PGFPlots\ will still work as it does in earlier versions, your documents will remain the same if you don't set it explicitly.}.
	Take a look at the next page for the precise definition of |compat|.

	\item \PGFPlots\ 1.3 now supports reversed axes. It is no longer necessary to use work--arounds with negative units.
\pgfkeys{/pdflinks/search key prefixes in/.add={/pgfplots/,}{}}

	Take a look at the |x dir=reverse| key.

	Existing work--arounds will still function properly. Use |\pgfplotsset{compat=1.3}| together with |x dir=reverse| to switch to the new version.
\end{enumerate}

\subsubsection{Old Features Which May Need Attention}
\begin{enumerate}
	\item The |scatter/classes| feature produces proper legends as of version 1.3. This may change the appearance of existing legends of plots with |scatter/classes|.

	\item Due to a small math bug in \PGF\ $2.00$, you \emph{can't} use the math expression `|-x^2|'. It is necessary to use `|0-x^2|' instead. The same holds for
		`|exp(-x^2)|'; use `|exp(0-x^2)|' instead.

		This will be fixed with \PGF\ $>2.00$.

	\item Starting with \PGFPlots\ $1.1$, |\tikzstyle| should \emph{no longer be used} to set \PGFPlots\ options.
	
	Although |\tikzstyle| is still supported for some older \PGFPlots\ options, you should replace any occurance of |\tikzstyle| with |\pgfplotsset{|\meta{style name}|/.style={|\meta{key-value-list}|}}| or the associated |/.append style| variant. See section~\ref{sec:styles} for more detail.
\end{enumerate}
I apologize for any inconvenience caused by these changes.

\begin{pgfplotskey}{compat=\mchoice{1.3,pre 1.3,default,newest} (initially default)}
	The preamble configuration 
\begin{codeexample}[code only]
\usepackage{pgfplots}
\pgfplotsset{compat=1.3}
\end{codeexample}
	allows to choose between backwards compatibility and most recent features.

	\PGFPlots\ version 1.3 comes with new features which may lead to different spacings compared to earlier versions:
	\begin{enumerate}
		\item Version 1.3 comes with the |xlabel near ticks| feature which places (in this case $x$) axis labels ``near'' the tick labels, taking the size of tick labels into account.
		
		Older versions used |xlabel absolute|, i.e.\ an absolute distance between the axis and axis labels (now matter how large or small tick labels are).

		The initial setting \emph{keeps backwards compatibility}.

		You are encouraged to use
\begin{codeexample}[code only]
\usepackage{pgfplots}
\pgfplotsset{compat=1.3}
\end{codeexample}
		in your preamble.
		
		\item Version 1.3 fixes a bug with |anchor=south west| which occurred mainly for reversed axes: the anchors where upside--down. This is fixed now.

		
		If you have written some work--around and want to keep it going, use

		|\pgfplotsset{compat/anchors=pre 1.3}|\pgfmanualpdflabel{/pgfplots/compat/anchors}{} 

		to disable the bug fix.
	\end{enumerate}

	Use |\pgfplotsset{compat=default}| to restore the factory settings.

	The setting |\pgfplotsset{compat=newest}| will always use features of the most recent version. This might result in small changes in the document's appearance.
\end{pgfplotskey}

\subsection{The Team}
\PGFPlots\ has been written mainly by Christian Feuers�nger with many improvements of Pascal Wolkotte and Nick Papior Andersen as a spare time project. We hope it is useful and provides valuable plots.

If you are interested in writing something but don't know how, consider reading the auxiliary manual \href{file:TeX-programming-notes.pdf}{TeX-programming-notes.pdf} which comes with \PGFPlots. It is far from complete, but maybe it is a good starting point (at least for more literature).

\subsection{Acknowledgements}
I thank God for all hours of enjoyed programming. I thank Pascal Wolkotte and Nick Papior Andersen for their programming efforts and contributions as part of the development team. I thank Stefan Tibus, who contributed the |plot shell| feature. I thank Tom Cashman for the contribution of the |reverse legend| feature. Special thanks go to Stefan Pinnow whose tests of \PGFPlots\ lead to numerous quality improvements. Furthermore, I thank Dr.~Schweitzer for many fruitful discussions and Dr.~Meine for his ideas and suggestions. Thanks as well to the many international contributors who provided feature requests or identified bugs or simply manual improvements!

Last but not least, I thank Till Tantau and Mark Wibrow for their excellent graphics (and more) package \PGF\ and \Tikz\ which is the base of \PGFPlots.

\include{pgfplots.install}%
\include{pgfplots.intro}%
\include{pgfplots.reference}%
\include{pgfplots.libs}%
\include{pgfplots.resources}%
\include{pgfplots.importexport}%
\include{pgfplots.basic.reference}%

\printindex

\bibliographystyle{abbrv} %gerapali} %gerabbrv} %gerunsrt.bst} %gerabbrv}% gerplain}
\nocite{pgfplotstable}
\nocite{programmingnotes}
\bibliography{pgfplots}
\end{document}
%
%%%%%%%%%%%%%%%%%%%%%%%%%%%%%%%%%%%%%%%%%%%%%%%%%%%%%%%%%%%%%%%%%%%%%%%%%%%%%
%
% Package pgfplots.sty documentation. 
%
% Copyright 2007/2008 by Christian Feuersaenger.
%
% This program is free software: you can redistribute it and/or modify
% it under the terms of the GNU General Public License as published by
% the Free Software Foundation, either version 3 of the License, or
% (at your option) any later version.
% 
% This program is distributed in the hope that it will be useful,
% but WITHOUT ANY WARRANTY; without even the implied warranty of
% MERCHANTABILITY or FITNESS FOR A PARTICULAR PURPOSE.  See the
% GNU General Public License for more details.
% 
% You should have received a copy of the GNU General Public License
% along with this program.  If not, see <http://www.gnu.org/licenses/>.
%
%
%%%%%%%%%%%%%%%%%%%%%%%%%%%%%%%%%%%%%%%%%%%%%%%%%%%%%%%%%%%%%%%%%%%%%%%%%%%%%
\input{pgfplots.preamble.tex}
%\RequirePackage[german,english,francais]{babel}

\pgfplotsmanualexternalexpensivetrue

%\usetikzlibrary{external}
%\tikzexternalize[prefix=figures/]{pgfplots}

\title{%
	Manual for Package \PGFPlots\\
	{\small Version \pgfplotsversion}\\
	{\small\href{http://sourceforge.net/projects/pgfplots}{http://sourceforge.net/projects/pgfplots}}}

%\includeonly{pgfplots.libs}


\begin{document}

\def\plotcoords{%
\addplot coordinates {
(5,8.312e-02)    (17,2.547e-02)   (49,7.407e-03)
(129,2.102e-03)  (321,5.874e-04)  (769,1.623e-04)
(1793,4.442e-05) (4097,1.207e-05) (9217,3.261e-06)
};

\addplot coordinates{
(7,8.472e-02)    (31,3.044e-02)    (111,1.022e-02)
(351,3.303e-03)  (1023,1.039e-03)  (2815,3.196e-04)
(7423,9.658e-05) (18943,2.873e-05) (47103,8.437e-06)
};

\addplot coordinates{
(9,7.881e-02)     (49,3.243e-02)    (209,1.232e-02)
(769,4.454e-03)   (2561,1.551e-03)  (7937,5.236e-04)
(23297,1.723e-04) (65537,5.545e-05) (178177,1.751e-05)
};

\addplot coordinates{
(11,6.887e-02)    (71,3.177e-02)     (351,1.341e-02)
(1471,5.334e-03)  (5503,2.027e-03)   (18943,7.415e-04)
(61183,2.628e-04) (187903,9.063e-05) (553983,3.053e-05)
};

\addplot coordinates{
(13,5.755e-02)     (97,2.925e-02)     (545,1.351e-02)
(2561,5.842e-03)   (10625,2.397e-03)  (40193,9.414e-04)
(141569,3.564e-04) (471041,1.308e-04) 
(1496065,4.670e-05)
};
}%
\maketitle
\begin{abstract}%
\PGFPlots\ draws high--quality function plots in normal or logarithmic scaling with a user-friendly interface directly in \TeX. The user supplies axis labels, legend entries and the plot coordinates for one or more plots and \PGFPlots\ applies axis scaling, computes any logarithms and axis ticks and draws the plots, supporting line plots, scatter plots, piecewise constant plots, bar plots, area plots, mesh-- and surface plots and some more. It is based on Till Tantau's package \PGF/\Tikz.
\end{abstract}
\tableofcontents
\section{Introduction}
This package provides tools to generate plots and labeled axes easily. It draws normal plots, logplots and semi-logplots, in two and three dimensions. Axis ticks, labels, legends (in case of multiple plots) can be added with key-value options. It can cycle through a set of predefined line/marker/color specifications. In summary, its purpose is to simplify the generation of high-quality function and/or data plots, and solving the problems of
\begin{itemize}
	\item consistency of document and font type and font size,
	\item direct use of \TeX\ math mode in axis descriptions,
	\item consistency of data and figures (no third party tool necessary),
	\item inter document consistency using preamble configurations and styles.
\end{itemize}
Although not necessary, separate |.pdf| or |.eps| graphics can be generated using the |external| library developed as part of \Tikz.

\PGFPlots\ is build completely on \Tikz/\PGF. Knowledge of \Tikz\ will simplify the work with \PGFPlots, although it is not required.

\section[About PGFPlots: Preliminaries]{About {\normalfont\PGFPlots}: Preliminaries}
This section contains information about upgrades, the team, the installation (in case you need to do it manually) and troubleshooting. You may skip it completely except for the upgrade remarks.

However, note that this library requires at least \PGF\ version $2.00$. At the time of this writing, many \TeX-distributions still contain the older \PGF\ version $1.18$, so it may be necessary to install a recent \PGF\ prior to using \PGFPlots.

\subsection{Components}
\PGFPlots\ comes with two components:
\begin{enumerate}
	\item the plotting component (which you are currently reading) and
	\item the \PGFPlotstable\ component which simplifies number formatting and postprocessing of numerical tables. It comes as a separate package and has its own manual \href{file:pgfplotstable.pdf}{pgfplotstable.pdf}.
\end{enumerate}

\subsection{Upgrade remarks}
This release provides a lot of improvements which can be found in all detail in \texttt{ChangeLog} for interested readers. However, some attention is useful with respect to the following changes.

\subsubsection{New Optional Features}
\begin{enumerate}
	\item \PGFPlots\ 1.3 comes with user interface improvements. The technical distinction between ``behavior options'' and ``style options'' of older versions is no longer necessary (although still fully supported).

	\item \PGFPlots\ 1.3 has a new feature which allows to \emph{move axis labels tight to tick labels} automatically.

	Since this affects the spacing, it is not enabled be default.

	Use
\begin{codeexample}[code only]
\usepackage{pgfplots}
\pgfplotsset{compat=1.3}
\end{codeexample}
	in your preamble to benefit from the improved spacing\footnote{The |compat=1.3| setting is necessary for new features which move axis labels around like reversed axis. However, \PGFPlots\ will still work as it does in earlier versions, your documents will remain the same if you don't set it explicitly.}.
	Take a look at the next page for the precise definition of |compat|.

	\item \PGFPlots\ 1.3 now supports reversed axes. It is no longer necessary to use work--arounds with negative units.
\pgfkeys{/pdflinks/search key prefixes in/.add={/pgfplots/,}{}}

	Take a look at the |x dir=reverse| key.

	Existing work--arounds will still function properly. Use |\pgfplotsset{compat=1.3}| together with |x dir=reverse| to switch to the new version.
\end{enumerate}

\subsubsection{Old Features Which May Need Attention}
\begin{enumerate}
	\item The |scatter/classes| feature produces proper legends as of version 1.3. This may change the appearance of existing legends of plots with |scatter/classes|.

	\item Due to a small math bug in \PGF\ $2.00$, you \emph{can't} use the math expression `|-x^2|'. It is necessary to use `|0-x^2|' instead. The same holds for
		`|exp(-x^2)|'; use `|exp(0-x^2)|' instead.

		This will be fixed with \PGF\ $>2.00$.

	\item Starting with \PGFPlots\ $1.1$, |\tikzstyle| should \emph{no longer be used} to set \PGFPlots\ options.
	
	Although |\tikzstyle| is still supported for some older \PGFPlots\ options, you should replace any occurance of |\tikzstyle| with |\pgfplotsset{|\meta{style name}|/.style={|\meta{key-value-list}|}}| or the associated |/.append style| variant. See section~\ref{sec:styles} for more detail.
\end{enumerate}
I apologize for any inconvenience caused by these changes.

\begin{pgfplotskey}{compat=\mchoice{1.3,pre 1.3,default,newest} (initially default)}
	The preamble configuration 
\begin{codeexample}[code only]
\usepackage{pgfplots}
\pgfplotsset{compat=1.3}
\end{codeexample}
	allows to choose between backwards compatibility and most recent features.

	\PGFPlots\ version 1.3 comes with new features which may lead to different spacings compared to earlier versions:
	\begin{enumerate}
		\item Version 1.3 comes with the |xlabel near ticks| feature which places (in this case $x$) axis labels ``near'' the tick labels, taking the size of tick labels into account.
		
		Older versions used |xlabel absolute|, i.e.\ an absolute distance between the axis and axis labels (now matter how large or small tick labels are).

		The initial setting \emph{keeps backwards compatibility}.

		You are encouraged to use
\begin{codeexample}[code only]
\usepackage{pgfplots}
\pgfplotsset{compat=1.3}
\end{codeexample}
		in your preamble.
		
		\item Version 1.3 fixes a bug with |anchor=south west| which occurred mainly for reversed axes: the anchors where upside--down. This is fixed now.

		
		If you have written some work--around and want to keep it going, use

		|\pgfplotsset{compat/anchors=pre 1.3}|\pgfmanualpdflabel{/pgfplots/compat/anchors}{} 

		to disable the bug fix.
	\end{enumerate}

	Use |\pgfplotsset{compat=default}| to restore the factory settings.

	The setting |\pgfplotsset{compat=newest}| will always use features of the most recent version. This might result in small changes in the document's appearance.
\end{pgfplotskey}

\subsection{The Team}
\PGFPlots\ has been written mainly by Christian Feuers�nger with many improvements of Pascal Wolkotte and Nick Papior Andersen as a spare time project. We hope it is useful and provides valuable plots.

If you are interested in writing something but don't know how, consider reading the auxiliary manual \href{file:TeX-programming-notes.pdf}{TeX-programming-notes.pdf} which comes with \PGFPlots. It is far from complete, but maybe it is a good starting point (at least for more literature).

\subsection{Acknowledgements}
I thank God for all hours of enjoyed programming. I thank Pascal Wolkotte and Nick Papior Andersen for their programming efforts and contributions as part of the development team. I thank Stefan Tibus, who contributed the |plot shell| feature. I thank Tom Cashman for the contribution of the |reverse legend| feature. Special thanks go to Stefan Pinnow whose tests of \PGFPlots\ lead to numerous quality improvements. Furthermore, I thank Dr.~Schweitzer for many fruitful discussions and Dr.~Meine for his ideas and suggestions. Thanks as well to the many international contributors who provided feature requests or identified bugs or simply manual improvements!

Last but not least, I thank Till Tantau and Mark Wibrow for their excellent graphics (and more) package \PGF\ and \Tikz\ which is the base of \PGFPlots.

\include{pgfplots.install}%
\include{pgfplots.intro}%
\include{pgfplots.reference}%
\include{pgfplots.libs}%
\include{pgfplots.resources}%
\include{pgfplots.importexport}%
\include{pgfplots.basic.reference}%

\printindex

\bibliographystyle{abbrv} %gerapali} %gerabbrv} %gerunsrt.bst} %gerabbrv}% gerplain}
\nocite{pgfplotstable}
\nocite{programmingnotes}
\bibliography{pgfplots}
\end{document}
%
%%%%%%%%%%%%%%%%%%%%%%%%%%%%%%%%%%%%%%%%%%%%%%%%%%%%%%%%%%%%%%%%%%%%%%%%%%%%%
%
% Package pgfplots.sty documentation. 
%
% Copyright 2007/2008 by Christian Feuersaenger.
%
% This program is free software: you can redistribute it and/or modify
% it under the terms of the GNU General Public License as published by
% the Free Software Foundation, either version 3 of the License, or
% (at your option) any later version.
% 
% This program is distributed in the hope that it will be useful,
% but WITHOUT ANY WARRANTY; without even the implied warranty of
% MERCHANTABILITY or FITNESS FOR A PARTICULAR PURPOSE.  See the
% GNU General Public License for more details.
% 
% You should have received a copy of the GNU General Public License
% along with this program.  If not, see <http://www.gnu.org/licenses/>.
%
%
%%%%%%%%%%%%%%%%%%%%%%%%%%%%%%%%%%%%%%%%%%%%%%%%%%%%%%%%%%%%%%%%%%%%%%%%%%%%%
\input{pgfplots.preamble.tex}
%\RequirePackage[german,english,francais]{babel}

\pgfplotsmanualexternalexpensivetrue

%\usetikzlibrary{external}
%\tikzexternalize[prefix=figures/]{pgfplots}

\title{%
	Manual for Package \PGFPlots\\
	{\small Version \pgfplotsversion}\\
	{\small\href{http://sourceforge.net/projects/pgfplots}{http://sourceforge.net/projects/pgfplots}}}

%\includeonly{pgfplots.libs}


\begin{document}

\def\plotcoords{%
\addplot coordinates {
(5,8.312e-02)    (17,2.547e-02)   (49,7.407e-03)
(129,2.102e-03)  (321,5.874e-04)  (769,1.623e-04)
(1793,4.442e-05) (4097,1.207e-05) (9217,3.261e-06)
};

\addplot coordinates{
(7,8.472e-02)    (31,3.044e-02)    (111,1.022e-02)
(351,3.303e-03)  (1023,1.039e-03)  (2815,3.196e-04)
(7423,9.658e-05) (18943,2.873e-05) (47103,8.437e-06)
};

\addplot coordinates{
(9,7.881e-02)     (49,3.243e-02)    (209,1.232e-02)
(769,4.454e-03)   (2561,1.551e-03)  (7937,5.236e-04)
(23297,1.723e-04) (65537,5.545e-05) (178177,1.751e-05)
};

\addplot coordinates{
(11,6.887e-02)    (71,3.177e-02)     (351,1.341e-02)
(1471,5.334e-03)  (5503,2.027e-03)   (18943,7.415e-04)
(61183,2.628e-04) (187903,9.063e-05) (553983,3.053e-05)
};

\addplot coordinates{
(13,5.755e-02)     (97,2.925e-02)     (545,1.351e-02)
(2561,5.842e-03)   (10625,2.397e-03)  (40193,9.414e-04)
(141569,3.564e-04) (471041,1.308e-04) 
(1496065,4.670e-05)
};
}%
\maketitle
\begin{abstract}%
\PGFPlots\ draws high--quality function plots in normal or logarithmic scaling with a user-friendly interface directly in \TeX. The user supplies axis labels, legend entries and the plot coordinates for one or more plots and \PGFPlots\ applies axis scaling, computes any logarithms and axis ticks and draws the plots, supporting line plots, scatter plots, piecewise constant plots, bar plots, area plots, mesh-- and surface plots and some more. It is based on Till Tantau's package \PGF/\Tikz.
\end{abstract}
\tableofcontents
\section{Introduction}
This package provides tools to generate plots and labeled axes easily. It draws normal plots, logplots and semi-logplots, in two and three dimensions. Axis ticks, labels, legends (in case of multiple plots) can be added with key-value options. It can cycle through a set of predefined line/marker/color specifications. In summary, its purpose is to simplify the generation of high-quality function and/or data plots, and solving the problems of
\begin{itemize}
	\item consistency of document and font type and font size,
	\item direct use of \TeX\ math mode in axis descriptions,
	\item consistency of data and figures (no third party tool necessary),
	\item inter document consistency using preamble configurations and styles.
\end{itemize}
Although not necessary, separate |.pdf| or |.eps| graphics can be generated using the |external| library developed as part of \Tikz.

\PGFPlots\ is build completely on \Tikz/\PGF. Knowledge of \Tikz\ will simplify the work with \PGFPlots, although it is not required.

\section[About PGFPlots: Preliminaries]{About {\normalfont\PGFPlots}: Preliminaries}
This section contains information about upgrades, the team, the installation (in case you need to do it manually) and troubleshooting. You may skip it completely except for the upgrade remarks.

However, note that this library requires at least \PGF\ version $2.00$. At the time of this writing, many \TeX-distributions still contain the older \PGF\ version $1.18$, so it may be necessary to install a recent \PGF\ prior to using \PGFPlots.

\subsection{Components}
\PGFPlots\ comes with two components:
\begin{enumerate}
	\item the plotting component (which you are currently reading) and
	\item the \PGFPlotstable\ component which simplifies number formatting and postprocessing of numerical tables. It comes as a separate package and has its own manual \href{file:pgfplotstable.pdf}{pgfplotstable.pdf}.
\end{enumerate}

\subsection{Upgrade remarks}
This release provides a lot of improvements which can be found in all detail in \texttt{ChangeLog} for interested readers. However, some attention is useful with respect to the following changes.

\subsubsection{New Optional Features}
\begin{enumerate}
	\item \PGFPlots\ 1.3 comes with user interface improvements. The technical distinction between ``behavior options'' and ``style options'' of older versions is no longer necessary (although still fully supported).

	\item \PGFPlots\ 1.3 has a new feature which allows to \emph{move axis labels tight to tick labels} automatically.

	Since this affects the spacing, it is not enabled be default.

	Use
\begin{codeexample}[code only]
\usepackage{pgfplots}
\pgfplotsset{compat=1.3}
\end{codeexample}
	in your preamble to benefit from the improved spacing\footnote{The |compat=1.3| setting is necessary for new features which move axis labels around like reversed axis. However, \PGFPlots\ will still work as it does in earlier versions, your documents will remain the same if you don't set it explicitly.}.
	Take a look at the next page for the precise definition of |compat|.

	\item \PGFPlots\ 1.3 now supports reversed axes. It is no longer necessary to use work--arounds with negative units.
\pgfkeys{/pdflinks/search key prefixes in/.add={/pgfplots/,}{}}

	Take a look at the |x dir=reverse| key.

	Existing work--arounds will still function properly. Use |\pgfplotsset{compat=1.3}| together with |x dir=reverse| to switch to the new version.
\end{enumerate}

\subsubsection{Old Features Which May Need Attention}
\begin{enumerate}
	\item The |scatter/classes| feature produces proper legends as of version 1.3. This may change the appearance of existing legends of plots with |scatter/classes|.

	\item Due to a small math bug in \PGF\ $2.00$, you \emph{can't} use the math expression `|-x^2|'. It is necessary to use `|0-x^2|' instead. The same holds for
		`|exp(-x^2)|'; use `|exp(0-x^2)|' instead.

		This will be fixed with \PGF\ $>2.00$.

	\item Starting with \PGFPlots\ $1.1$, |\tikzstyle| should \emph{no longer be used} to set \PGFPlots\ options.
	
	Although |\tikzstyle| is still supported for some older \PGFPlots\ options, you should replace any occurance of |\tikzstyle| with |\pgfplotsset{|\meta{style name}|/.style={|\meta{key-value-list}|}}| or the associated |/.append style| variant. See section~\ref{sec:styles} for more detail.
\end{enumerate}
I apologize for any inconvenience caused by these changes.

\begin{pgfplotskey}{compat=\mchoice{1.3,pre 1.3,default,newest} (initially default)}
	The preamble configuration 
\begin{codeexample}[code only]
\usepackage{pgfplots}
\pgfplotsset{compat=1.3}
\end{codeexample}
	allows to choose between backwards compatibility and most recent features.

	\PGFPlots\ version 1.3 comes with new features which may lead to different spacings compared to earlier versions:
	\begin{enumerate}
		\item Version 1.3 comes with the |xlabel near ticks| feature which places (in this case $x$) axis labels ``near'' the tick labels, taking the size of tick labels into account.
		
		Older versions used |xlabel absolute|, i.e.\ an absolute distance between the axis and axis labels (now matter how large or small tick labels are).

		The initial setting \emph{keeps backwards compatibility}.

		You are encouraged to use
\begin{codeexample}[code only]
\usepackage{pgfplots}
\pgfplotsset{compat=1.3}
\end{codeexample}
		in your preamble.
		
		\item Version 1.3 fixes a bug with |anchor=south west| which occurred mainly for reversed axes: the anchors where upside--down. This is fixed now.

		
		If you have written some work--around and want to keep it going, use

		|\pgfplotsset{compat/anchors=pre 1.3}|\pgfmanualpdflabel{/pgfplots/compat/anchors}{} 

		to disable the bug fix.
	\end{enumerate}

	Use |\pgfplotsset{compat=default}| to restore the factory settings.

	The setting |\pgfplotsset{compat=newest}| will always use features of the most recent version. This might result in small changes in the document's appearance.
\end{pgfplotskey}

\subsection{The Team}
\PGFPlots\ has been written mainly by Christian Feuers�nger with many improvements of Pascal Wolkotte and Nick Papior Andersen as a spare time project. We hope it is useful and provides valuable plots.

If you are interested in writing something but don't know how, consider reading the auxiliary manual \href{file:TeX-programming-notes.pdf}{TeX-programming-notes.pdf} which comes with \PGFPlots. It is far from complete, but maybe it is a good starting point (at least for more literature).

\subsection{Acknowledgements}
I thank God for all hours of enjoyed programming. I thank Pascal Wolkotte and Nick Papior Andersen for their programming efforts and contributions as part of the development team. I thank Stefan Tibus, who contributed the |plot shell| feature. I thank Tom Cashman for the contribution of the |reverse legend| feature. Special thanks go to Stefan Pinnow whose tests of \PGFPlots\ lead to numerous quality improvements. Furthermore, I thank Dr.~Schweitzer for many fruitful discussions and Dr.~Meine for his ideas and suggestions. Thanks as well to the many international contributors who provided feature requests or identified bugs or simply manual improvements!

Last but not least, I thank Till Tantau and Mark Wibrow for their excellent graphics (and more) package \PGF\ and \Tikz\ which is the base of \PGFPlots.

\include{pgfplots.install}%
\include{pgfplots.intro}%
\include{pgfplots.reference}%
\include{pgfplots.libs}%
\include{pgfplots.resources}%
\include{pgfplots.importexport}%
\include{pgfplots.basic.reference}%

\printindex

\bibliographystyle{abbrv} %gerapali} %gerabbrv} %gerunsrt.bst} %gerabbrv}% gerplain}
\nocite{pgfplotstable}
\nocite{programmingnotes}
\bibliography{pgfplots}
\end{document}
%
\include{pgfplots.basic.reference}%

\printindex

\bibliographystyle{abbrv} %gerapali} %gerabbrv} %gerunsrt.bst} %gerabbrv}% gerplain}
\nocite{pgfplotstable}
\nocite{programmingnotes}
\bibliography{pgfplots}
\end{document}
%
%%%%%%%%%%%%%%%%%%%%%%%%%%%%%%%%%%%%%%%%%%%%%%%%%%%%%%%%%%%%%%%%%%%%%%%%%%%%%
%
% Package pgfplots.sty documentation. 
%
% Copyright 2007/2008 by Christian Feuersaenger.
%
% This program is free software: you can redistribute it and/or modify
% it under the terms of the GNU General Public License as published by
% the Free Software Foundation, either version 3 of the License, or
% (at your option) any later version.
% 
% This program is distributed in the hope that it will be useful,
% but WITHOUT ANY WARRANTY; without even the implied warranty of
% MERCHANTABILITY or FITNESS FOR A PARTICULAR PURPOSE.  See the
% GNU General Public License for more details.
% 
% You should have received a copy of the GNU General Public License
% along with this program.  If not, see <http://www.gnu.org/licenses/>.
%
%
%%%%%%%%%%%%%%%%%%%%%%%%%%%%%%%%%%%%%%%%%%%%%%%%%%%%%%%%%%%%%%%%%%%%%%%%%%%%%
%%%%%%%%%%%%%%%%%%%%%%%%%%%%%%%%%%%%%%%%%%%%%%%%%%%%%%%%%%%%%%%%%%%%%%%%%%%%%
%
% Package pgfplots.sty documentation. 
%
% Copyright 2007/2008 by Christian Feuersaenger.
%
% This program is free software: you can redistribute it and/or modify
% it under the terms of the GNU General Public License as published by
% the Free Software Foundation, either version 3 of the License, or
% (at your option) any later version.
% 
% This program is distributed in the hope that it will be useful,
% but WITHOUT ANY WARRANTY; without even the implied warranty of
% MERCHANTABILITY or FITNESS FOR A PARTICULAR PURPOSE.  See the
% GNU General Public License for more details.
% 
% You should have received a copy of the GNU General Public License
% along with this program.  If not, see <http://www.gnu.org/licenses/>.
%
%
%%%%%%%%%%%%%%%%%%%%%%%%%%%%%%%%%%%%%%%%%%%%%%%%%%%%%%%%%%%%%%%%%%%%%%%%%%%%%
\documentclass[a4paper]{ltxdoc}

\usepackage{makeidx}

% DON't let hyperref overload the format of index and glossary. 
% I want to do that on my own in the stylefiles for makeindex...
\makeatletter
\let\@old@wrindex=\@wrindex
\makeatother

\usepackage{ifpdf}
\ifpdf
	\usepackage[pdftex]{hyperref}
\else
	\usepackage[dvipdfm]{hyperref}
\fi
\hypersetup{pdfborder={0 0 0}}

\makeatletter
\let\@wrindex=\@old@wrindex
\makeatother

% Formatiere Seitennummern im Index:
\newcommand{\indexpageno}[1]{%
	{\bfseries\hyperpage{#1}}%
}


\newcommand{\R}{\mathbb{R}}
\newcommand{\N}{\mathbb{N}}
\newcommand{\Z}{\mathbb{Z}}

\long\def\COMMENTLOWLEVEL#1\ENDCOMMENT{}
\def\ENDCOMMENT{}

\usepackage{textcomp}
\usepackage{booktabs}

\usepackage{calc}
\usepackage[formats]{listings}
%\usepackage{courier} % don't use it - the '^' character can't be copy-pasted in courier

\usepackage{array}
\lstset{%
	basicstyle=\ttfamily,
	language=[LaTeX]tex, % Seems as if \lstset{language=tex} must be invoked BEFORE loading tikz!?
	tabsize=4,
	breaklines=true,
	breakindent=0pt
}

\ifpdf
	\pdfinfo {
		/Author	(Christian Feuersaenger)
	}

\else
	\def\pgfsysdriver{pgfsys-dvipdfm.def}
\fi
%\def\pgfsysdriver{pgfsys-pdftex.def}
\usepackage{pgfplots}

\ifpdf
	\usetikzlibrary{pgfplotsclickable}
\fi

%\usepackage{fp}
% ATTENTION:
% this requires pgf version NEWER than 2.00 :
%\usetikzlibrary{fixedpointarithmetic}

\usetikzlibrary{dateplot}

\usepackage[a4paper,left=2.25cm,right=2.25cm,top=2.5cm,bottom=2.5cm,nohead]{geometry}
\usepackage{amsmath,amssymb}
\usepackage{xxcolor}
\usepackage{pifont}
\usepackage[latin1]{inputenc}
\usepackage{amsmath}
\usepackage{eurosym}
\input{pgfplots-macros}

\pgfqkeys{/codeexample}{%
	every codeexample/.style={
		width=8cm,
		/pgfplots/every axis/.append style={legend style={fill=graphicbackground}}
	},
	tabsize=4,
}
\pgfplotsset{
	every axis/.append style={width=7cm},
	filter discard warning=false,
}

\usetikzlibrary{backgrounds,patterns}
% Global styles:
\tikzset{
  shape example/.style={
    color=black!30,
    draw,
    fill=yellow!30,
    line width=.5cm,
    inner xsep=2.5cm,
    inner ysep=0.5cm}
}

\newcommand{\FIXME}[1]{\textcolor{red}{(FIXME: #1)}}

% fuer endvironment 'sidewaysfigure' bspw
% \usepackage{rotating}

\newcommand\Tikz{Ti\textit kZ}
\newcommand\PGF{\textsc{pgf}}
\newcommand\PGFPlots{\textsc{pgfplots}}
\def\PGFPlotstable{\textsc{PgfplotsTable}}


\makeindex

% Fix overful hboxes automatically:
\tolerance=2000
\emergencystretch=10pt

\tikzset{prefix=gnuplot/pgfplots_} % prefix for 'plot function'

\author{%
	Christian Feuers\"anger\footnote{\url{http://wissrech.ins.uni-bonn.de/people/feuersaenger}}\\%
	Institut f\"ur Numerische Simulation\\
	Universit\"at Bonn, Germany}


%\RequirePackage[german,english,francais]{babel}

\pgfplotsmanualexternalexpensivetrue

%\usetikzlibrary{external}
%\tikzexternalize[prefix=figures/]{pgfplots}

\title{%
	Manual for Package \PGFPlots\\
	{\small Version \pgfplotsversion}\\
	{\small\href{http://sourceforge.net/projects/pgfplots}{http://sourceforge.net/projects/pgfplots}}}

%\includeonly{pgfplots.libs}


\begin{document}

\def\plotcoords{%
\addplot coordinates {
(5,8.312e-02)    (17,2.547e-02)   (49,7.407e-03)
(129,2.102e-03)  (321,5.874e-04)  (769,1.623e-04)
(1793,4.442e-05) (4097,1.207e-05) (9217,3.261e-06)
};

\addplot coordinates{
(7,8.472e-02)    (31,3.044e-02)    (111,1.022e-02)
(351,3.303e-03)  (1023,1.039e-03)  (2815,3.196e-04)
(7423,9.658e-05) (18943,2.873e-05) (47103,8.437e-06)
};

\addplot coordinates{
(9,7.881e-02)     (49,3.243e-02)    (209,1.232e-02)
(769,4.454e-03)   (2561,1.551e-03)  (7937,5.236e-04)
(23297,1.723e-04) (65537,5.545e-05) (178177,1.751e-05)
};

\addplot coordinates{
(11,6.887e-02)    (71,3.177e-02)     (351,1.341e-02)
(1471,5.334e-03)  (5503,2.027e-03)   (18943,7.415e-04)
(61183,2.628e-04) (187903,9.063e-05) (553983,3.053e-05)
};

\addplot coordinates{
(13,5.755e-02)     (97,2.925e-02)     (545,1.351e-02)
(2561,5.842e-03)   (10625,2.397e-03)  (40193,9.414e-04)
(141569,3.564e-04) (471041,1.308e-04) 
(1496065,4.670e-05)
};
}%
\maketitle
\begin{abstract}%
\PGFPlots\ draws high--quality function plots in normal or logarithmic scaling with a user-friendly interface directly in \TeX. The user supplies axis labels, legend entries and the plot coordinates for one or more plots and \PGFPlots\ applies axis scaling, computes any logarithms and axis ticks and draws the plots, supporting line plots, scatter plots, piecewise constant plots, bar plots, area plots, mesh-- and surface plots and some more. It is based on Till Tantau's package \PGF/\Tikz.
\end{abstract}
\tableofcontents
\section{Introduction}
This package provides tools to generate plots and labeled axes easily. It draws normal plots, logplots and semi-logplots, in two and three dimensions. Axis ticks, labels, legends (in case of multiple plots) can be added with key-value options. It can cycle through a set of predefined line/marker/color specifications. In summary, its purpose is to simplify the generation of high-quality function and/or data plots, and solving the problems of
\begin{itemize}
	\item consistency of document and font type and font size,
	\item direct use of \TeX\ math mode in axis descriptions,
	\item consistency of data and figures (no third party tool necessary),
	\item inter document consistency using preamble configurations and styles.
\end{itemize}
Although not necessary, separate |.pdf| or |.eps| graphics can be generated using the |external| library developed as part of \Tikz.

\PGFPlots\ is build completely on \Tikz/\PGF. Knowledge of \Tikz\ will simplify the work with \PGFPlots, although it is not required.

\section[About PGFPlots: Preliminaries]{About {\normalfont\PGFPlots}: Preliminaries}
This section contains information about upgrades, the team, the installation (in case you need to do it manually) and troubleshooting. You may skip it completely except for the upgrade remarks.

However, note that this library requires at least \PGF\ version $2.00$. At the time of this writing, many \TeX-distributions still contain the older \PGF\ version $1.18$, so it may be necessary to install a recent \PGF\ prior to using \PGFPlots.

\subsection{Components}
\PGFPlots\ comes with two components:
\begin{enumerate}
	\item the plotting component (which you are currently reading) and
	\item the \PGFPlotstable\ component which simplifies number formatting and postprocessing of numerical tables. It comes as a separate package and has its own manual \href{file:pgfplotstable.pdf}{pgfplotstable.pdf}.
\end{enumerate}

\subsection{Upgrade remarks}
This release provides a lot of improvements which can be found in all detail in \texttt{ChangeLog} for interested readers. However, some attention is useful with respect to the following changes.

\subsubsection{New Optional Features}
\begin{enumerate}
	\item \PGFPlots\ 1.3 comes with user interface improvements. The technical distinction between ``behavior options'' and ``style options'' of older versions is no longer necessary (although still fully supported).

	\item \PGFPlots\ 1.3 has a new feature which allows to \emph{move axis labels tight to tick labels} automatically.

	Since this affects the spacing, it is not enabled be default.

	Use
\begin{codeexample}[code only]
\usepackage{pgfplots}
\pgfplotsset{compat=1.3}
\end{codeexample}
	in your preamble to benefit from the improved spacing\footnote{The |compat=1.3| setting is necessary for new features which move axis labels around like reversed axis. However, \PGFPlots\ will still work as it does in earlier versions, your documents will remain the same if you don't set it explicitly.}.
	Take a look at the next page for the precise definition of |compat|.

	\item \PGFPlots\ 1.3 now supports reversed axes. It is no longer necessary to use work--arounds with negative units.
\pgfkeys{/pdflinks/search key prefixes in/.add={/pgfplots/,}{}}

	Take a look at the |x dir=reverse| key.

	Existing work--arounds will still function properly. Use |\pgfplotsset{compat=1.3}| together with |x dir=reverse| to switch to the new version.
\end{enumerate}

\subsubsection{Old Features Which May Need Attention}
\begin{enumerate}
	\item The |scatter/classes| feature produces proper legends as of version 1.3. This may change the appearance of existing legends of plots with |scatter/classes|.

	\item Due to a small math bug in \PGF\ $2.00$, you \emph{can't} use the math expression `|-x^2|'. It is necessary to use `|0-x^2|' instead. The same holds for
		`|exp(-x^2)|'; use `|exp(0-x^2)|' instead.

		This will be fixed with \PGF\ $>2.00$.

	\item Starting with \PGFPlots\ $1.1$, |\tikzstyle| should \emph{no longer be used} to set \PGFPlots\ options.
	
	Although |\tikzstyle| is still supported for some older \PGFPlots\ options, you should replace any occurance of |\tikzstyle| with |\pgfplotsset{|\meta{style name}|/.style={|\meta{key-value-list}|}}| or the associated |/.append style| variant. See section~\ref{sec:styles} for more detail.
\end{enumerate}
I apologize for any inconvenience caused by these changes.

\begin{pgfplotskey}{compat=\mchoice{1.3,pre 1.3,default,newest} (initially default)}
	The preamble configuration 
\begin{codeexample}[code only]
\usepackage{pgfplots}
\pgfplotsset{compat=1.3}
\end{codeexample}
	allows to choose between backwards compatibility and most recent features.

	\PGFPlots\ version 1.3 comes with new features which may lead to different spacings compared to earlier versions:
	\begin{enumerate}
		\item Version 1.3 comes with the |xlabel near ticks| feature which places (in this case $x$) axis labels ``near'' the tick labels, taking the size of tick labels into account.
		
		Older versions used |xlabel absolute|, i.e.\ an absolute distance between the axis and axis labels (now matter how large or small tick labels are).

		The initial setting \emph{keeps backwards compatibility}.

		You are encouraged to use
\begin{codeexample}[code only]
\usepackage{pgfplots}
\pgfplotsset{compat=1.3}
\end{codeexample}
		in your preamble.
		
		\item Version 1.3 fixes a bug with |anchor=south west| which occurred mainly for reversed axes: the anchors where upside--down. This is fixed now.

		
		If you have written some work--around and want to keep it going, use

		|\pgfplotsset{compat/anchors=pre 1.3}|\pgfmanualpdflabel{/pgfplots/compat/anchors}{} 

		to disable the bug fix.
	\end{enumerate}

	Use |\pgfplotsset{compat=default}| to restore the factory settings.

	The setting |\pgfplotsset{compat=newest}| will always use features of the most recent version. This might result in small changes in the document's appearance.
\end{pgfplotskey}

\subsection{The Team}
\PGFPlots\ has been written mainly by Christian Feuers�nger with many improvements of Pascal Wolkotte and Nick Papior Andersen as a spare time project. We hope it is useful and provides valuable plots.

If you are interested in writing something but don't know how, consider reading the auxiliary manual \href{file:TeX-programming-notes.pdf}{TeX-programming-notes.pdf} which comes with \PGFPlots. It is far from complete, but maybe it is a good starting point (at least for more literature).

\subsection{Acknowledgements}
I thank God for all hours of enjoyed programming. I thank Pascal Wolkotte and Nick Papior Andersen for their programming efforts and contributions as part of the development team. I thank Stefan Tibus, who contributed the |plot shell| feature. I thank Tom Cashman for the contribution of the |reverse legend| feature. Special thanks go to Stefan Pinnow whose tests of \PGFPlots\ lead to numerous quality improvements. Furthermore, I thank Dr.~Schweitzer for many fruitful discussions and Dr.~Meine for his ideas and suggestions. Thanks as well to the many international contributors who provided feature requests or identified bugs or simply manual improvements!

Last but not least, I thank Till Tantau and Mark Wibrow for their excellent graphics (and more) package \PGF\ and \Tikz\ which is the base of \PGFPlots.

%%%%%%%%%%%%%%%%%%%%%%%%%%%%%%%%%%%%%%%%%%%%%%%%%%%%%%%%%%%%%%%%%%%%%%%%%%%%%
%
% Package pgfplots.sty documentation. 
%
% Copyright 2007/2008 by Christian Feuersaenger.
%
% This program is free software: you can redistribute it and/or modify
% it under the terms of the GNU General Public License as published by
% the Free Software Foundation, either version 3 of the License, or
% (at your option) any later version.
% 
% This program is distributed in the hope that it will be useful,
% but WITHOUT ANY WARRANTY; without even the implied warranty of
% MERCHANTABILITY or FITNESS FOR A PARTICULAR PURPOSE.  See the
% GNU General Public License for more details.
% 
% You should have received a copy of the GNU General Public License
% along with this program.  If not, see <http://www.gnu.org/licenses/>.
%
%
%%%%%%%%%%%%%%%%%%%%%%%%%%%%%%%%%%%%%%%%%%%%%%%%%%%%%%%%%%%%%%%%%%%%%%%%%%%%%
\input{pgfplots.preamble.tex}
%\RequirePackage[german,english,francais]{babel}

\pgfplotsmanualexternalexpensivetrue

%\usetikzlibrary{external}
%\tikzexternalize[prefix=figures/]{pgfplots}

\title{%
	Manual for Package \PGFPlots\\
	{\small Version \pgfplotsversion}\\
	{\small\href{http://sourceforge.net/projects/pgfplots}{http://sourceforge.net/projects/pgfplots}}}

%\includeonly{pgfplots.libs}


\begin{document}

\def\plotcoords{%
\addplot coordinates {
(5,8.312e-02)    (17,2.547e-02)   (49,7.407e-03)
(129,2.102e-03)  (321,5.874e-04)  (769,1.623e-04)
(1793,4.442e-05) (4097,1.207e-05) (9217,3.261e-06)
};

\addplot coordinates{
(7,8.472e-02)    (31,3.044e-02)    (111,1.022e-02)
(351,3.303e-03)  (1023,1.039e-03)  (2815,3.196e-04)
(7423,9.658e-05) (18943,2.873e-05) (47103,8.437e-06)
};

\addplot coordinates{
(9,7.881e-02)     (49,3.243e-02)    (209,1.232e-02)
(769,4.454e-03)   (2561,1.551e-03)  (7937,5.236e-04)
(23297,1.723e-04) (65537,5.545e-05) (178177,1.751e-05)
};

\addplot coordinates{
(11,6.887e-02)    (71,3.177e-02)     (351,1.341e-02)
(1471,5.334e-03)  (5503,2.027e-03)   (18943,7.415e-04)
(61183,2.628e-04) (187903,9.063e-05) (553983,3.053e-05)
};

\addplot coordinates{
(13,5.755e-02)     (97,2.925e-02)     (545,1.351e-02)
(2561,5.842e-03)   (10625,2.397e-03)  (40193,9.414e-04)
(141569,3.564e-04) (471041,1.308e-04) 
(1496065,4.670e-05)
};
}%
\maketitle
\begin{abstract}%
\PGFPlots\ draws high--quality function plots in normal or logarithmic scaling with a user-friendly interface directly in \TeX. The user supplies axis labels, legend entries and the plot coordinates for one or more plots and \PGFPlots\ applies axis scaling, computes any logarithms and axis ticks and draws the plots, supporting line plots, scatter plots, piecewise constant plots, bar plots, area plots, mesh-- and surface plots and some more. It is based on Till Tantau's package \PGF/\Tikz.
\end{abstract}
\tableofcontents
\section{Introduction}
This package provides tools to generate plots and labeled axes easily. It draws normal plots, logplots and semi-logplots, in two and three dimensions. Axis ticks, labels, legends (in case of multiple plots) can be added with key-value options. It can cycle through a set of predefined line/marker/color specifications. In summary, its purpose is to simplify the generation of high-quality function and/or data plots, and solving the problems of
\begin{itemize}
	\item consistency of document and font type and font size,
	\item direct use of \TeX\ math mode in axis descriptions,
	\item consistency of data and figures (no third party tool necessary),
	\item inter document consistency using preamble configurations and styles.
\end{itemize}
Although not necessary, separate |.pdf| or |.eps| graphics can be generated using the |external| library developed as part of \Tikz.

\PGFPlots\ is build completely on \Tikz/\PGF. Knowledge of \Tikz\ will simplify the work with \PGFPlots, although it is not required.

\section[About PGFPlots: Preliminaries]{About {\normalfont\PGFPlots}: Preliminaries}
This section contains information about upgrades, the team, the installation (in case you need to do it manually) and troubleshooting. You may skip it completely except for the upgrade remarks.

However, note that this library requires at least \PGF\ version $2.00$. At the time of this writing, many \TeX-distributions still contain the older \PGF\ version $1.18$, so it may be necessary to install a recent \PGF\ prior to using \PGFPlots.

\subsection{Components}
\PGFPlots\ comes with two components:
\begin{enumerate}
	\item the plotting component (which you are currently reading) and
	\item the \PGFPlotstable\ component which simplifies number formatting and postprocessing of numerical tables. It comes as a separate package and has its own manual \href{file:pgfplotstable.pdf}{pgfplotstable.pdf}.
\end{enumerate}

\subsection{Upgrade remarks}
This release provides a lot of improvements which can be found in all detail in \texttt{ChangeLog} for interested readers. However, some attention is useful with respect to the following changes.

\subsubsection{New Optional Features}
\begin{enumerate}
	\item \PGFPlots\ 1.3 comes with user interface improvements. The technical distinction between ``behavior options'' and ``style options'' of older versions is no longer necessary (although still fully supported).

	\item \PGFPlots\ 1.3 has a new feature which allows to \emph{move axis labels tight to tick labels} automatically.

	Since this affects the spacing, it is not enabled be default.

	Use
\begin{codeexample}[code only]
\usepackage{pgfplots}
\pgfplotsset{compat=1.3}
\end{codeexample}
	in your preamble to benefit from the improved spacing\footnote{The |compat=1.3| setting is necessary for new features which move axis labels around like reversed axis. However, \PGFPlots\ will still work as it does in earlier versions, your documents will remain the same if you don't set it explicitly.}.
	Take a look at the next page for the precise definition of |compat|.

	\item \PGFPlots\ 1.3 now supports reversed axes. It is no longer necessary to use work--arounds with negative units.
\pgfkeys{/pdflinks/search key prefixes in/.add={/pgfplots/,}{}}

	Take a look at the |x dir=reverse| key.

	Existing work--arounds will still function properly. Use |\pgfplotsset{compat=1.3}| together with |x dir=reverse| to switch to the new version.
\end{enumerate}

\subsubsection{Old Features Which May Need Attention}
\begin{enumerate}
	\item The |scatter/classes| feature produces proper legends as of version 1.3. This may change the appearance of existing legends of plots with |scatter/classes|.

	\item Due to a small math bug in \PGF\ $2.00$, you \emph{can't} use the math expression `|-x^2|'. It is necessary to use `|0-x^2|' instead. The same holds for
		`|exp(-x^2)|'; use `|exp(0-x^2)|' instead.

		This will be fixed with \PGF\ $>2.00$.

	\item Starting with \PGFPlots\ $1.1$, |\tikzstyle| should \emph{no longer be used} to set \PGFPlots\ options.
	
	Although |\tikzstyle| is still supported for some older \PGFPlots\ options, you should replace any occurance of |\tikzstyle| with |\pgfplotsset{|\meta{style name}|/.style={|\meta{key-value-list}|}}| or the associated |/.append style| variant. See section~\ref{sec:styles} for more detail.
\end{enumerate}
I apologize for any inconvenience caused by these changes.

\begin{pgfplotskey}{compat=\mchoice{1.3,pre 1.3,default,newest} (initially default)}
	The preamble configuration 
\begin{codeexample}[code only]
\usepackage{pgfplots}
\pgfplotsset{compat=1.3}
\end{codeexample}
	allows to choose between backwards compatibility and most recent features.

	\PGFPlots\ version 1.3 comes with new features which may lead to different spacings compared to earlier versions:
	\begin{enumerate}
		\item Version 1.3 comes with the |xlabel near ticks| feature which places (in this case $x$) axis labels ``near'' the tick labels, taking the size of tick labels into account.
		
		Older versions used |xlabel absolute|, i.e.\ an absolute distance between the axis and axis labels (now matter how large or small tick labels are).

		The initial setting \emph{keeps backwards compatibility}.

		You are encouraged to use
\begin{codeexample}[code only]
\usepackage{pgfplots}
\pgfplotsset{compat=1.3}
\end{codeexample}
		in your preamble.
		
		\item Version 1.3 fixes a bug with |anchor=south west| which occurred mainly for reversed axes: the anchors where upside--down. This is fixed now.

		
		If you have written some work--around and want to keep it going, use

		|\pgfplotsset{compat/anchors=pre 1.3}|\pgfmanualpdflabel{/pgfplots/compat/anchors}{} 

		to disable the bug fix.
	\end{enumerate}

	Use |\pgfplotsset{compat=default}| to restore the factory settings.

	The setting |\pgfplotsset{compat=newest}| will always use features of the most recent version. This might result in small changes in the document's appearance.
\end{pgfplotskey}

\subsection{The Team}
\PGFPlots\ has been written mainly by Christian Feuers�nger with many improvements of Pascal Wolkotte and Nick Papior Andersen as a spare time project. We hope it is useful and provides valuable plots.

If you are interested in writing something but don't know how, consider reading the auxiliary manual \href{file:TeX-programming-notes.pdf}{TeX-programming-notes.pdf} which comes with \PGFPlots. It is far from complete, but maybe it is a good starting point (at least for more literature).

\subsection{Acknowledgements}
I thank God for all hours of enjoyed programming. I thank Pascal Wolkotte and Nick Papior Andersen for their programming efforts and contributions as part of the development team. I thank Stefan Tibus, who contributed the |plot shell| feature. I thank Tom Cashman for the contribution of the |reverse legend| feature. Special thanks go to Stefan Pinnow whose tests of \PGFPlots\ lead to numerous quality improvements. Furthermore, I thank Dr.~Schweitzer for many fruitful discussions and Dr.~Meine for his ideas and suggestions. Thanks as well to the many international contributors who provided feature requests or identified bugs or simply manual improvements!

Last but not least, I thank Till Tantau and Mark Wibrow for their excellent graphics (and more) package \PGF\ and \Tikz\ which is the base of \PGFPlots.

\include{pgfplots.install}%
\include{pgfplots.intro}%
\include{pgfplots.reference}%
\include{pgfplots.libs}%
\include{pgfplots.resources}%
\include{pgfplots.importexport}%
\include{pgfplots.basic.reference}%

\printindex

\bibliographystyle{abbrv} %gerapali} %gerabbrv} %gerunsrt.bst} %gerabbrv}% gerplain}
\nocite{pgfplotstable}
\nocite{programmingnotes}
\bibliography{pgfplots}
\end{document}
%
%%%%%%%%%%%%%%%%%%%%%%%%%%%%%%%%%%%%%%%%%%%%%%%%%%%%%%%%%%%%%%%%%%%%%%%%%%%%%
%
% Package pgfplots.sty documentation. 
%
% Copyright 2007/2008 by Christian Feuersaenger.
%
% This program is free software: you can redistribute it and/or modify
% it under the terms of the GNU General Public License as published by
% the Free Software Foundation, either version 3 of the License, or
% (at your option) any later version.
% 
% This program is distributed in the hope that it will be useful,
% but WITHOUT ANY WARRANTY; without even the implied warranty of
% MERCHANTABILITY or FITNESS FOR A PARTICULAR PURPOSE.  See the
% GNU General Public License for more details.
% 
% You should have received a copy of the GNU General Public License
% along with this program.  If not, see <http://www.gnu.org/licenses/>.
%
%
%%%%%%%%%%%%%%%%%%%%%%%%%%%%%%%%%%%%%%%%%%%%%%%%%%%%%%%%%%%%%%%%%%%%%%%%%%%%%
\input{pgfplots.preamble.tex}
%\RequirePackage[german,english,francais]{babel}

\pgfplotsmanualexternalexpensivetrue

%\usetikzlibrary{external}
%\tikzexternalize[prefix=figures/]{pgfplots}

\title{%
	Manual for Package \PGFPlots\\
	{\small Version \pgfplotsversion}\\
	{\small\href{http://sourceforge.net/projects/pgfplots}{http://sourceforge.net/projects/pgfplots}}}

%\includeonly{pgfplots.libs}


\begin{document}

\def\plotcoords{%
\addplot coordinates {
(5,8.312e-02)    (17,2.547e-02)   (49,7.407e-03)
(129,2.102e-03)  (321,5.874e-04)  (769,1.623e-04)
(1793,4.442e-05) (4097,1.207e-05) (9217,3.261e-06)
};

\addplot coordinates{
(7,8.472e-02)    (31,3.044e-02)    (111,1.022e-02)
(351,3.303e-03)  (1023,1.039e-03)  (2815,3.196e-04)
(7423,9.658e-05) (18943,2.873e-05) (47103,8.437e-06)
};

\addplot coordinates{
(9,7.881e-02)     (49,3.243e-02)    (209,1.232e-02)
(769,4.454e-03)   (2561,1.551e-03)  (7937,5.236e-04)
(23297,1.723e-04) (65537,5.545e-05) (178177,1.751e-05)
};

\addplot coordinates{
(11,6.887e-02)    (71,3.177e-02)     (351,1.341e-02)
(1471,5.334e-03)  (5503,2.027e-03)   (18943,7.415e-04)
(61183,2.628e-04) (187903,9.063e-05) (553983,3.053e-05)
};

\addplot coordinates{
(13,5.755e-02)     (97,2.925e-02)     (545,1.351e-02)
(2561,5.842e-03)   (10625,2.397e-03)  (40193,9.414e-04)
(141569,3.564e-04) (471041,1.308e-04) 
(1496065,4.670e-05)
};
}%
\maketitle
\begin{abstract}%
\PGFPlots\ draws high--quality function plots in normal or logarithmic scaling with a user-friendly interface directly in \TeX. The user supplies axis labels, legend entries and the plot coordinates for one or more plots and \PGFPlots\ applies axis scaling, computes any logarithms and axis ticks and draws the plots, supporting line plots, scatter plots, piecewise constant plots, bar plots, area plots, mesh-- and surface plots and some more. It is based on Till Tantau's package \PGF/\Tikz.
\end{abstract}
\tableofcontents
\section{Introduction}
This package provides tools to generate plots and labeled axes easily. It draws normal plots, logplots and semi-logplots, in two and three dimensions. Axis ticks, labels, legends (in case of multiple plots) can be added with key-value options. It can cycle through a set of predefined line/marker/color specifications. In summary, its purpose is to simplify the generation of high-quality function and/or data plots, and solving the problems of
\begin{itemize}
	\item consistency of document and font type and font size,
	\item direct use of \TeX\ math mode in axis descriptions,
	\item consistency of data and figures (no third party tool necessary),
	\item inter document consistency using preamble configurations and styles.
\end{itemize}
Although not necessary, separate |.pdf| or |.eps| graphics can be generated using the |external| library developed as part of \Tikz.

\PGFPlots\ is build completely on \Tikz/\PGF. Knowledge of \Tikz\ will simplify the work with \PGFPlots, although it is not required.

\section[About PGFPlots: Preliminaries]{About {\normalfont\PGFPlots}: Preliminaries}
This section contains information about upgrades, the team, the installation (in case you need to do it manually) and troubleshooting. You may skip it completely except for the upgrade remarks.

However, note that this library requires at least \PGF\ version $2.00$. At the time of this writing, many \TeX-distributions still contain the older \PGF\ version $1.18$, so it may be necessary to install a recent \PGF\ prior to using \PGFPlots.

\subsection{Components}
\PGFPlots\ comes with two components:
\begin{enumerate}
	\item the plotting component (which you are currently reading) and
	\item the \PGFPlotstable\ component which simplifies number formatting and postprocessing of numerical tables. It comes as a separate package and has its own manual \href{file:pgfplotstable.pdf}{pgfplotstable.pdf}.
\end{enumerate}

\subsection{Upgrade remarks}
This release provides a lot of improvements which can be found in all detail in \texttt{ChangeLog} for interested readers. However, some attention is useful with respect to the following changes.

\subsubsection{New Optional Features}
\begin{enumerate}
	\item \PGFPlots\ 1.3 comes with user interface improvements. The technical distinction between ``behavior options'' and ``style options'' of older versions is no longer necessary (although still fully supported).

	\item \PGFPlots\ 1.3 has a new feature which allows to \emph{move axis labels tight to tick labels} automatically.

	Since this affects the spacing, it is not enabled be default.

	Use
\begin{codeexample}[code only]
\usepackage{pgfplots}
\pgfplotsset{compat=1.3}
\end{codeexample}
	in your preamble to benefit from the improved spacing\footnote{The |compat=1.3| setting is necessary for new features which move axis labels around like reversed axis. However, \PGFPlots\ will still work as it does in earlier versions, your documents will remain the same if you don't set it explicitly.}.
	Take a look at the next page for the precise definition of |compat|.

	\item \PGFPlots\ 1.3 now supports reversed axes. It is no longer necessary to use work--arounds with negative units.
\pgfkeys{/pdflinks/search key prefixes in/.add={/pgfplots/,}{}}

	Take a look at the |x dir=reverse| key.

	Existing work--arounds will still function properly. Use |\pgfplotsset{compat=1.3}| together with |x dir=reverse| to switch to the new version.
\end{enumerate}

\subsubsection{Old Features Which May Need Attention}
\begin{enumerate}
	\item The |scatter/classes| feature produces proper legends as of version 1.3. This may change the appearance of existing legends of plots with |scatter/classes|.

	\item Due to a small math bug in \PGF\ $2.00$, you \emph{can't} use the math expression `|-x^2|'. It is necessary to use `|0-x^2|' instead. The same holds for
		`|exp(-x^2)|'; use `|exp(0-x^2)|' instead.

		This will be fixed with \PGF\ $>2.00$.

	\item Starting with \PGFPlots\ $1.1$, |\tikzstyle| should \emph{no longer be used} to set \PGFPlots\ options.
	
	Although |\tikzstyle| is still supported for some older \PGFPlots\ options, you should replace any occurance of |\tikzstyle| with |\pgfplotsset{|\meta{style name}|/.style={|\meta{key-value-list}|}}| or the associated |/.append style| variant. See section~\ref{sec:styles} for more detail.
\end{enumerate}
I apologize for any inconvenience caused by these changes.

\begin{pgfplotskey}{compat=\mchoice{1.3,pre 1.3,default,newest} (initially default)}
	The preamble configuration 
\begin{codeexample}[code only]
\usepackage{pgfplots}
\pgfplotsset{compat=1.3}
\end{codeexample}
	allows to choose between backwards compatibility and most recent features.

	\PGFPlots\ version 1.3 comes with new features which may lead to different spacings compared to earlier versions:
	\begin{enumerate}
		\item Version 1.3 comes with the |xlabel near ticks| feature which places (in this case $x$) axis labels ``near'' the tick labels, taking the size of tick labels into account.
		
		Older versions used |xlabel absolute|, i.e.\ an absolute distance between the axis and axis labels (now matter how large or small tick labels are).

		The initial setting \emph{keeps backwards compatibility}.

		You are encouraged to use
\begin{codeexample}[code only]
\usepackage{pgfplots}
\pgfplotsset{compat=1.3}
\end{codeexample}
		in your preamble.
		
		\item Version 1.3 fixes a bug with |anchor=south west| which occurred mainly for reversed axes: the anchors where upside--down. This is fixed now.

		
		If you have written some work--around and want to keep it going, use

		|\pgfplotsset{compat/anchors=pre 1.3}|\pgfmanualpdflabel{/pgfplots/compat/anchors}{} 

		to disable the bug fix.
	\end{enumerate}

	Use |\pgfplotsset{compat=default}| to restore the factory settings.

	The setting |\pgfplotsset{compat=newest}| will always use features of the most recent version. This might result in small changes in the document's appearance.
\end{pgfplotskey}

\subsection{The Team}
\PGFPlots\ has been written mainly by Christian Feuers�nger with many improvements of Pascal Wolkotte and Nick Papior Andersen as a spare time project. We hope it is useful and provides valuable plots.

If you are interested in writing something but don't know how, consider reading the auxiliary manual \href{file:TeX-programming-notes.pdf}{TeX-programming-notes.pdf} which comes with \PGFPlots. It is far from complete, but maybe it is a good starting point (at least for more literature).

\subsection{Acknowledgements}
I thank God for all hours of enjoyed programming. I thank Pascal Wolkotte and Nick Papior Andersen for their programming efforts and contributions as part of the development team. I thank Stefan Tibus, who contributed the |plot shell| feature. I thank Tom Cashman for the contribution of the |reverse legend| feature. Special thanks go to Stefan Pinnow whose tests of \PGFPlots\ lead to numerous quality improvements. Furthermore, I thank Dr.~Schweitzer for many fruitful discussions and Dr.~Meine for his ideas and suggestions. Thanks as well to the many international contributors who provided feature requests or identified bugs or simply manual improvements!

Last but not least, I thank Till Tantau and Mark Wibrow for their excellent graphics (and more) package \PGF\ and \Tikz\ which is the base of \PGFPlots.

\include{pgfplots.install}%
\include{pgfplots.intro}%
\include{pgfplots.reference}%
\include{pgfplots.libs}%
\include{pgfplots.resources}%
\include{pgfplots.importexport}%
\include{pgfplots.basic.reference}%

\printindex

\bibliographystyle{abbrv} %gerapali} %gerabbrv} %gerunsrt.bst} %gerabbrv}% gerplain}
\nocite{pgfplotstable}
\nocite{programmingnotes}
\bibliography{pgfplots}
\end{document}
%
%%%%%%%%%%%%%%%%%%%%%%%%%%%%%%%%%%%%%%%%%%%%%%%%%%%%%%%%%%%%%%%%%%%%%%%%%%%%%
%
% Package pgfplots.sty documentation. 
%
% Copyright 2007/2008 by Christian Feuersaenger.
%
% This program is free software: you can redistribute it and/or modify
% it under the terms of the GNU General Public License as published by
% the Free Software Foundation, either version 3 of the License, or
% (at your option) any later version.
% 
% This program is distributed in the hope that it will be useful,
% but WITHOUT ANY WARRANTY; without even the implied warranty of
% MERCHANTABILITY or FITNESS FOR A PARTICULAR PURPOSE.  See the
% GNU General Public License for more details.
% 
% You should have received a copy of the GNU General Public License
% along with this program.  If not, see <http://www.gnu.org/licenses/>.
%
%
%%%%%%%%%%%%%%%%%%%%%%%%%%%%%%%%%%%%%%%%%%%%%%%%%%%%%%%%%%%%%%%%%%%%%%%%%%%%%
\input{pgfplots.preamble.tex}
%\RequirePackage[german,english,francais]{babel}

\pgfplotsmanualexternalexpensivetrue

%\usetikzlibrary{external}
%\tikzexternalize[prefix=figures/]{pgfplots}

\title{%
	Manual for Package \PGFPlots\\
	{\small Version \pgfplotsversion}\\
	{\small\href{http://sourceforge.net/projects/pgfplots}{http://sourceforge.net/projects/pgfplots}}}

%\includeonly{pgfplots.libs}


\begin{document}

\def\plotcoords{%
\addplot coordinates {
(5,8.312e-02)    (17,2.547e-02)   (49,7.407e-03)
(129,2.102e-03)  (321,5.874e-04)  (769,1.623e-04)
(1793,4.442e-05) (4097,1.207e-05) (9217,3.261e-06)
};

\addplot coordinates{
(7,8.472e-02)    (31,3.044e-02)    (111,1.022e-02)
(351,3.303e-03)  (1023,1.039e-03)  (2815,3.196e-04)
(7423,9.658e-05) (18943,2.873e-05) (47103,8.437e-06)
};

\addplot coordinates{
(9,7.881e-02)     (49,3.243e-02)    (209,1.232e-02)
(769,4.454e-03)   (2561,1.551e-03)  (7937,5.236e-04)
(23297,1.723e-04) (65537,5.545e-05) (178177,1.751e-05)
};

\addplot coordinates{
(11,6.887e-02)    (71,3.177e-02)     (351,1.341e-02)
(1471,5.334e-03)  (5503,2.027e-03)   (18943,7.415e-04)
(61183,2.628e-04) (187903,9.063e-05) (553983,3.053e-05)
};

\addplot coordinates{
(13,5.755e-02)     (97,2.925e-02)     (545,1.351e-02)
(2561,5.842e-03)   (10625,2.397e-03)  (40193,9.414e-04)
(141569,3.564e-04) (471041,1.308e-04) 
(1496065,4.670e-05)
};
}%
\maketitle
\begin{abstract}%
\PGFPlots\ draws high--quality function plots in normal or logarithmic scaling with a user-friendly interface directly in \TeX. The user supplies axis labels, legend entries and the plot coordinates for one or more plots and \PGFPlots\ applies axis scaling, computes any logarithms and axis ticks and draws the plots, supporting line plots, scatter plots, piecewise constant plots, bar plots, area plots, mesh-- and surface plots and some more. It is based on Till Tantau's package \PGF/\Tikz.
\end{abstract}
\tableofcontents
\section{Introduction}
This package provides tools to generate plots and labeled axes easily. It draws normal plots, logplots and semi-logplots, in two and three dimensions. Axis ticks, labels, legends (in case of multiple plots) can be added with key-value options. It can cycle through a set of predefined line/marker/color specifications. In summary, its purpose is to simplify the generation of high-quality function and/or data plots, and solving the problems of
\begin{itemize}
	\item consistency of document and font type and font size,
	\item direct use of \TeX\ math mode in axis descriptions,
	\item consistency of data and figures (no third party tool necessary),
	\item inter document consistency using preamble configurations and styles.
\end{itemize}
Although not necessary, separate |.pdf| or |.eps| graphics can be generated using the |external| library developed as part of \Tikz.

\PGFPlots\ is build completely on \Tikz/\PGF. Knowledge of \Tikz\ will simplify the work with \PGFPlots, although it is not required.

\section[About PGFPlots: Preliminaries]{About {\normalfont\PGFPlots}: Preliminaries}
This section contains information about upgrades, the team, the installation (in case you need to do it manually) and troubleshooting. You may skip it completely except for the upgrade remarks.

However, note that this library requires at least \PGF\ version $2.00$. At the time of this writing, many \TeX-distributions still contain the older \PGF\ version $1.18$, so it may be necessary to install a recent \PGF\ prior to using \PGFPlots.

\subsection{Components}
\PGFPlots\ comes with two components:
\begin{enumerate}
	\item the plotting component (which you are currently reading) and
	\item the \PGFPlotstable\ component which simplifies number formatting and postprocessing of numerical tables. It comes as a separate package and has its own manual \href{file:pgfplotstable.pdf}{pgfplotstable.pdf}.
\end{enumerate}

\subsection{Upgrade remarks}
This release provides a lot of improvements which can be found in all detail in \texttt{ChangeLog} for interested readers. However, some attention is useful with respect to the following changes.

\subsubsection{New Optional Features}
\begin{enumerate}
	\item \PGFPlots\ 1.3 comes with user interface improvements. The technical distinction between ``behavior options'' and ``style options'' of older versions is no longer necessary (although still fully supported).

	\item \PGFPlots\ 1.3 has a new feature which allows to \emph{move axis labels tight to tick labels} automatically.

	Since this affects the spacing, it is not enabled be default.

	Use
\begin{codeexample}[code only]
\usepackage{pgfplots}
\pgfplotsset{compat=1.3}
\end{codeexample}
	in your preamble to benefit from the improved spacing\footnote{The |compat=1.3| setting is necessary for new features which move axis labels around like reversed axis. However, \PGFPlots\ will still work as it does in earlier versions, your documents will remain the same if you don't set it explicitly.}.
	Take a look at the next page for the precise definition of |compat|.

	\item \PGFPlots\ 1.3 now supports reversed axes. It is no longer necessary to use work--arounds with negative units.
\pgfkeys{/pdflinks/search key prefixes in/.add={/pgfplots/,}{}}

	Take a look at the |x dir=reverse| key.

	Existing work--arounds will still function properly. Use |\pgfplotsset{compat=1.3}| together with |x dir=reverse| to switch to the new version.
\end{enumerate}

\subsubsection{Old Features Which May Need Attention}
\begin{enumerate}
	\item The |scatter/classes| feature produces proper legends as of version 1.3. This may change the appearance of existing legends of plots with |scatter/classes|.

	\item Due to a small math bug in \PGF\ $2.00$, you \emph{can't} use the math expression `|-x^2|'. It is necessary to use `|0-x^2|' instead. The same holds for
		`|exp(-x^2)|'; use `|exp(0-x^2)|' instead.

		This will be fixed with \PGF\ $>2.00$.

	\item Starting with \PGFPlots\ $1.1$, |\tikzstyle| should \emph{no longer be used} to set \PGFPlots\ options.
	
	Although |\tikzstyle| is still supported for some older \PGFPlots\ options, you should replace any occurance of |\tikzstyle| with |\pgfplotsset{|\meta{style name}|/.style={|\meta{key-value-list}|}}| or the associated |/.append style| variant. See section~\ref{sec:styles} for more detail.
\end{enumerate}
I apologize for any inconvenience caused by these changes.

\begin{pgfplotskey}{compat=\mchoice{1.3,pre 1.3,default,newest} (initially default)}
	The preamble configuration 
\begin{codeexample}[code only]
\usepackage{pgfplots}
\pgfplotsset{compat=1.3}
\end{codeexample}
	allows to choose between backwards compatibility and most recent features.

	\PGFPlots\ version 1.3 comes with new features which may lead to different spacings compared to earlier versions:
	\begin{enumerate}
		\item Version 1.3 comes with the |xlabel near ticks| feature which places (in this case $x$) axis labels ``near'' the tick labels, taking the size of tick labels into account.
		
		Older versions used |xlabel absolute|, i.e.\ an absolute distance between the axis and axis labels (now matter how large or small tick labels are).

		The initial setting \emph{keeps backwards compatibility}.

		You are encouraged to use
\begin{codeexample}[code only]
\usepackage{pgfplots}
\pgfplotsset{compat=1.3}
\end{codeexample}
		in your preamble.
		
		\item Version 1.3 fixes a bug with |anchor=south west| which occurred mainly for reversed axes: the anchors where upside--down. This is fixed now.

		
		If you have written some work--around and want to keep it going, use

		|\pgfplotsset{compat/anchors=pre 1.3}|\pgfmanualpdflabel{/pgfplots/compat/anchors}{} 

		to disable the bug fix.
	\end{enumerate}

	Use |\pgfplotsset{compat=default}| to restore the factory settings.

	The setting |\pgfplotsset{compat=newest}| will always use features of the most recent version. This might result in small changes in the document's appearance.
\end{pgfplotskey}

\subsection{The Team}
\PGFPlots\ has been written mainly by Christian Feuers�nger with many improvements of Pascal Wolkotte and Nick Papior Andersen as a spare time project. We hope it is useful and provides valuable plots.

If you are interested in writing something but don't know how, consider reading the auxiliary manual \href{file:TeX-programming-notes.pdf}{TeX-programming-notes.pdf} which comes with \PGFPlots. It is far from complete, but maybe it is a good starting point (at least for more literature).

\subsection{Acknowledgements}
I thank God for all hours of enjoyed programming. I thank Pascal Wolkotte and Nick Papior Andersen for their programming efforts and contributions as part of the development team. I thank Stefan Tibus, who contributed the |plot shell| feature. I thank Tom Cashman for the contribution of the |reverse legend| feature. Special thanks go to Stefan Pinnow whose tests of \PGFPlots\ lead to numerous quality improvements. Furthermore, I thank Dr.~Schweitzer for many fruitful discussions and Dr.~Meine for his ideas and suggestions. Thanks as well to the many international contributors who provided feature requests or identified bugs or simply manual improvements!

Last but not least, I thank Till Tantau and Mark Wibrow for their excellent graphics (and more) package \PGF\ and \Tikz\ which is the base of \PGFPlots.

\include{pgfplots.install}%
\include{pgfplots.intro}%
\include{pgfplots.reference}%
\include{pgfplots.libs}%
\include{pgfplots.resources}%
\include{pgfplots.importexport}%
\include{pgfplots.basic.reference}%

\printindex

\bibliographystyle{abbrv} %gerapali} %gerabbrv} %gerunsrt.bst} %gerabbrv}% gerplain}
\nocite{pgfplotstable}
\nocite{programmingnotes}
\bibliography{pgfplots}
\end{document}
%
%%%%%%%%%%%%%%%%%%%%%%%%%%%%%%%%%%%%%%%%%%%%%%%%%%%%%%%%%%%%%%%%%%%%%%%%%%%%%
%
% Package pgfplots.sty documentation. 
%
% Copyright 2007/2008 by Christian Feuersaenger.
%
% This program is free software: you can redistribute it and/or modify
% it under the terms of the GNU General Public License as published by
% the Free Software Foundation, either version 3 of the License, or
% (at your option) any later version.
% 
% This program is distributed in the hope that it will be useful,
% but WITHOUT ANY WARRANTY; without even the implied warranty of
% MERCHANTABILITY or FITNESS FOR A PARTICULAR PURPOSE.  See the
% GNU General Public License for more details.
% 
% You should have received a copy of the GNU General Public License
% along with this program.  If not, see <http://www.gnu.org/licenses/>.
%
%
%%%%%%%%%%%%%%%%%%%%%%%%%%%%%%%%%%%%%%%%%%%%%%%%%%%%%%%%%%%%%%%%%%%%%%%%%%%%%
\input{pgfplots.preamble.tex}
%\RequirePackage[german,english,francais]{babel}

\pgfplotsmanualexternalexpensivetrue

%\usetikzlibrary{external}
%\tikzexternalize[prefix=figures/]{pgfplots}

\title{%
	Manual for Package \PGFPlots\\
	{\small Version \pgfplotsversion}\\
	{\small\href{http://sourceforge.net/projects/pgfplots}{http://sourceforge.net/projects/pgfplots}}}

%\includeonly{pgfplots.libs}


\begin{document}

\def\plotcoords{%
\addplot coordinates {
(5,8.312e-02)    (17,2.547e-02)   (49,7.407e-03)
(129,2.102e-03)  (321,5.874e-04)  (769,1.623e-04)
(1793,4.442e-05) (4097,1.207e-05) (9217,3.261e-06)
};

\addplot coordinates{
(7,8.472e-02)    (31,3.044e-02)    (111,1.022e-02)
(351,3.303e-03)  (1023,1.039e-03)  (2815,3.196e-04)
(7423,9.658e-05) (18943,2.873e-05) (47103,8.437e-06)
};

\addplot coordinates{
(9,7.881e-02)     (49,3.243e-02)    (209,1.232e-02)
(769,4.454e-03)   (2561,1.551e-03)  (7937,5.236e-04)
(23297,1.723e-04) (65537,5.545e-05) (178177,1.751e-05)
};

\addplot coordinates{
(11,6.887e-02)    (71,3.177e-02)     (351,1.341e-02)
(1471,5.334e-03)  (5503,2.027e-03)   (18943,7.415e-04)
(61183,2.628e-04) (187903,9.063e-05) (553983,3.053e-05)
};

\addplot coordinates{
(13,5.755e-02)     (97,2.925e-02)     (545,1.351e-02)
(2561,5.842e-03)   (10625,2.397e-03)  (40193,9.414e-04)
(141569,3.564e-04) (471041,1.308e-04) 
(1496065,4.670e-05)
};
}%
\maketitle
\begin{abstract}%
\PGFPlots\ draws high--quality function plots in normal or logarithmic scaling with a user-friendly interface directly in \TeX. The user supplies axis labels, legend entries and the plot coordinates for one or more plots and \PGFPlots\ applies axis scaling, computes any logarithms and axis ticks and draws the plots, supporting line plots, scatter plots, piecewise constant plots, bar plots, area plots, mesh-- and surface plots and some more. It is based on Till Tantau's package \PGF/\Tikz.
\end{abstract}
\tableofcontents
\section{Introduction}
This package provides tools to generate plots and labeled axes easily. It draws normal plots, logplots and semi-logplots, in two and three dimensions. Axis ticks, labels, legends (in case of multiple plots) can be added with key-value options. It can cycle through a set of predefined line/marker/color specifications. In summary, its purpose is to simplify the generation of high-quality function and/or data plots, and solving the problems of
\begin{itemize}
	\item consistency of document and font type and font size,
	\item direct use of \TeX\ math mode in axis descriptions,
	\item consistency of data and figures (no third party tool necessary),
	\item inter document consistency using preamble configurations and styles.
\end{itemize}
Although not necessary, separate |.pdf| or |.eps| graphics can be generated using the |external| library developed as part of \Tikz.

\PGFPlots\ is build completely on \Tikz/\PGF. Knowledge of \Tikz\ will simplify the work with \PGFPlots, although it is not required.

\section[About PGFPlots: Preliminaries]{About {\normalfont\PGFPlots}: Preliminaries}
This section contains information about upgrades, the team, the installation (in case you need to do it manually) and troubleshooting. You may skip it completely except for the upgrade remarks.

However, note that this library requires at least \PGF\ version $2.00$. At the time of this writing, many \TeX-distributions still contain the older \PGF\ version $1.18$, so it may be necessary to install a recent \PGF\ prior to using \PGFPlots.

\subsection{Components}
\PGFPlots\ comes with two components:
\begin{enumerate}
	\item the plotting component (which you are currently reading) and
	\item the \PGFPlotstable\ component which simplifies number formatting and postprocessing of numerical tables. It comes as a separate package and has its own manual \href{file:pgfplotstable.pdf}{pgfplotstable.pdf}.
\end{enumerate}

\subsection{Upgrade remarks}
This release provides a lot of improvements which can be found in all detail in \texttt{ChangeLog} for interested readers. However, some attention is useful with respect to the following changes.

\subsubsection{New Optional Features}
\begin{enumerate}
	\item \PGFPlots\ 1.3 comes with user interface improvements. The technical distinction between ``behavior options'' and ``style options'' of older versions is no longer necessary (although still fully supported).

	\item \PGFPlots\ 1.3 has a new feature which allows to \emph{move axis labels tight to tick labels} automatically.

	Since this affects the spacing, it is not enabled be default.

	Use
\begin{codeexample}[code only]
\usepackage{pgfplots}
\pgfplotsset{compat=1.3}
\end{codeexample}
	in your preamble to benefit from the improved spacing\footnote{The |compat=1.3| setting is necessary for new features which move axis labels around like reversed axis. However, \PGFPlots\ will still work as it does in earlier versions, your documents will remain the same if you don't set it explicitly.}.
	Take a look at the next page for the precise definition of |compat|.

	\item \PGFPlots\ 1.3 now supports reversed axes. It is no longer necessary to use work--arounds with negative units.
\pgfkeys{/pdflinks/search key prefixes in/.add={/pgfplots/,}{}}

	Take a look at the |x dir=reverse| key.

	Existing work--arounds will still function properly. Use |\pgfplotsset{compat=1.3}| together with |x dir=reverse| to switch to the new version.
\end{enumerate}

\subsubsection{Old Features Which May Need Attention}
\begin{enumerate}
	\item The |scatter/classes| feature produces proper legends as of version 1.3. This may change the appearance of existing legends of plots with |scatter/classes|.

	\item Due to a small math bug in \PGF\ $2.00$, you \emph{can't} use the math expression `|-x^2|'. It is necessary to use `|0-x^2|' instead. The same holds for
		`|exp(-x^2)|'; use `|exp(0-x^2)|' instead.

		This will be fixed with \PGF\ $>2.00$.

	\item Starting with \PGFPlots\ $1.1$, |\tikzstyle| should \emph{no longer be used} to set \PGFPlots\ options.
	
	Although |\tikzstyle| is still supported for some older \PGFPlots\ options, you should replace any occurance of |\tikzstyle| with |\pgfplotsset{|\meta{style name}|/.style={|\meta{key-value-list}|}}| or the associated |/.append style| variant. See section~\ref{sec:styles} for more detail.
\end{enumerate}
I apologize for any inconvenience caused by these changes.

\begin{pgfplotskey}{compat=\mchoice{1.3,pre 1.3,default,newest} (initially default)}
	The preamble configuration 
\begin{codeexample}[code only]
\usepackage{pgfplots}
\pgfplotsset{compat=1.3}
\end{codeexample}
	allows to choose between backwards compatibility and most recent features.

	\PGFPlots\ version 1.3 comes with new features which may lead to different spacings compared to earlier versions:
	\begin{enumerate}
		\item Version 1.3 comes with the |xlabel near ticks| feature which places (in this case $x$) axis labels ``near'' the tick labels, taking the size of tick labels into account.
		
		Older versions used |xlabel absolute|, i.e.\ an absolute distance between the axis and axis labels (now matter how large or small tick labels are).

		The initial setting \emph{keeps backwards compatibility}.

		You are encouraged to use
\begin{codeexample}[code only]
\usepackage{pgfplots}
\pgfplotsset{compat=1.3}
\end{codeexample}
		in your preamble.
		
		\item Version 1.3 fixes a bug with |anchor=south west| which occurred mainly for reversed axes: the anchors where upside--down. This is fixed now.

		
		If you have written some work--around and want to keep it going, use

		|\pgfplotsset{compat/anchors=pre 1.3}|\pgfmanualpdflabel{/pgfplots/compat/anchors}{} 

		to disable the bug fix.
	\end{enumerate}

	Use |\pgfplotsset{compat=default}| to restore the factory settings.

	The setting |\pgfplotsset{compat=newest}| will always use features of the most recent version. This might result in small changes in the document's appearance.
\end{pgfplotskey}

\subsection{The Team}
\PGFPlots\ has been written mainly by Christian Feuers�nger with many improvements of Pascal Wolkotte and Nick Papior Andersen as a spare time project. We hope it is useful and provides valuable plots.

If you are interested in writing something but don't know how, consider reading the auxiliary manual \href{file:TeX-programming-notes.pdf}{TeX-programming-notes.pdf} which comes with \PGFPlots. It is far from complete, but maybe it is a good starting point (at least for more literature).

\subsection{Acknowledgements}
I thank God for all hours of enjoyed programming. I thank Pascal Wolkotte and Nick Papior Andersen for their programming efforts and contributions as part of the development team. I thank Stefan Tibus, who contributed the |plot shell| feature. I thank Tom Cashman for the contribution of the |reverse legend| feature. Special thanks go to Stefan Pinnow whose tests of \PGFPlots\ lead to numerous quality improvements. Furthermore, I thank Dr.~Schweitzer for many fruitful discussions and Dr.~Meine for his ideas and suggestions. Thanks as well to the many international contributors who provided feature requests or identified bugs or simply manual improvements!

Last but not least, I thank Till Tantau and Mark Wibrow for their excellent graphics (and more) package \PGF\ and \Tikz\ which is the base of \PGFPlots.

\include{pgfplots.install}%
\include{pgfplots.intro}%
\include{pgfplots.reference}%
\include{pgfplots.libs}%
\include{pgfplots.resources}%
\include{pgfplots.importexport}%
\include{pgfplots.basic.reference}%

\printindex

\bibliographystyle{abbrv} %gerapali} %gerabbrv} %gerunsrt.bst} %gerabbrv}% gerplain}
\nocite{pgfplotstable}
\nocite{programmingnotes}
\bibliography{pgfplots}
\end{document}
%
%%%%%%%%%%%%%%%%%%%%%%%%%%%%%%%%%%%%%%%%%%%%%%%%%%%%%%%%%%%%%%%%%%%%%%%%%%%%%
%
% Package pgfplots.sty documentation. 
%
% Copyright 2007/2008 by Christian Feuersaenger.
%
% This program is free software: you can redistribute it and/or modify
% it under the terms of the GNU General Public License as published by
% the Free Software Foundation, either version 3 of the License, or
% (at your option) any later version.
% 
% This program is distributed in the hope that it will be useful,
% but WITHOUT ANY WARRANTY; without even the implied warranty of
% MERCHANTABILITY or FITNESS FOR A PARTICULAR PURPOSE.  See the
% GNU General Public License for more details.
% 
% You should have received a copy of the GNU General Public License
% along with this program.  If not, see <http://www.gnu.org/licenses/>.
%
%
%%%%%%%%%%%%%%%%%%%%%%%%%%%%%%%%%%%%%%%%%%%%%%%%%%%%%%%%%%%%%%%%%%%%%%%%%%%%%
\input{pgfplots.preamble.tex}
%\RequirePackage[german,english,francais]{babel}

\pgfplotsmanualexternalexpensivetrue

%\usetikzlibrary{external}
%\tikzexternalize[prefix=figures/]{pgfplots}

\title{%
	Manual for Package \PGFPlots\\
	{\small Version \pgfplotsversion}\\
	{\small\href{http://sourceforge.net/projects/pgfplots}{http://sourceforge.net/projects/pgfplots}}}

%\includeonly{pgfplots.libs}


\begin{document}

\def\plotcoords{%
\addplot coordinates {
(5,8.312e-02)    (17,2.547e-02)   (49,7.407e-03)
(129,2.102e-03)  (321,5.874e-04)  (769,1.623e-04)
(1793,4.442e-05) (4097,1.207e-05) (9217,3.261e-06)
};

\addplot coordinates{
(7,8.472e-02)    (31,3.044e-02)    (111,1.022e-02)
(351,3.303e-03)  (1023,1.039e-03)  (2815,3.196e-04)
(7423,9.658e-05) (18943,2.873e-05) (47103,8.437e-06)
};

\addplot coordinates{
(9,7.881e-02)     (49,3.243e-02)    (209,1.232e-02)
(769,4.454e-03)   (2561,1.551e-03)  (7937,5.236e-04)
(23297,1.723e-04) (65537,5.545e-05) (178177,1.751e-05)
};

\addplot coordinates{
(11,6.887e-02)    (71,3.177e-02)     (351,1.341e-02)
(1471,5.334e-03)  (5503,2.027e-03)   (18943,7.415e-04)
(61183,2.628e-04) (187903,9.063e-05) (553983,3.053e-05)
};

\addplot coordinates{
(13,5.755e-02)     (97,2.925e-02)     (545,1.351e-02)
(2561,5.842e-03)   (10625,2.397e-03)  (40193,9.414e-04)
(141569,3.564e-04) (471041,1.308e-04) 
(1496065,4.670e-05)
};
}%
\maketitle
\begin{abstract}%
\PGFPlots\ draws high--quality function plots in normal or logarithmic scaling with a user-friendly interface directly in \TeX. The user supplies axis labels, legend entries and the plot coordinates for one or more plots and \PGFPlots\ applies axis scaling, computes any logarithms and axis ticks and draws the plots, supporting line plots, scatter plots, piecewise constant plots, bar plots, area plots, mesh-- and surface plots and some more. It is based on Till Tantau's package \PGF/\Tikz.
\end{abstract}
\tableofcontents
\section{Introduction}
This package provides tools to generate plots and labeled axes easily. It draws normal plots, logplots and semi-logplots, in two and three dimensions. Axis ticks, labels, legends (in case of multiple plots) can be added with key-value options. It can cycle through a set of predefined line/marker/color specifications. In summary, its purpose is to simplify the generation of high-quality function and/or data plots, and solving the problems of
\begin{itemize}
	\item consistency of document and font type and font size,
	\item direct use of \TeX\ math mode in axis descriptions,
	\item consistency of data and figures (no third party tool necessary),
	\item inter document consistency using preamble configurations and styles.
\end{itemize}
Although not necessary, separate |.pdf| or |.eps| graphics can be generated using the |external| library developed as part of \Tikz.

\PGFPlots\ is build completely on \Tikz/\PGF. Knowledge of \Tikz\ will simplify the work with \PGFPlots, although it is not required.

\section[About PGFPlots: Preliminaries]{About {\normalfont\PGFPlots}: Preliminaries}
This section contains information about upgrades, the team, the installation (in case you need to do it manually) and troubleshooting. You may skip it completely except for the upgrade remarks.

However, note that this library requires at least \PGF\ version $2.00$. At the time of this writing, many \TeX-distributions still contain the older \PGF\ version $1.18$, so it may be necessary to install a recent \PGF\ prior to using \PGFPlots.

\subsection{Components}
\PGFPlots\ comes with two components:
\begin{enumerate}
	\item the plotting component (which you are currently reading) and
	\item the \PGFPlotstable\ component which simplifies number formatting and postprocessing of numerical tables. It comes as a separate package and has its own manual \href{file:pgfplotstable.pdf}{pgfplotstable.pdf}.
\end{enumerate}

\subsection{Upgrade remarks}
This release provides a lot of improvements which can be found in all detail in \texttt{ChangeLog} for interested readers. However, some attention is useful with respect to the following changes.

\subsubsection{New Optional Features}
\begin{enumerate}
	\item \PGFPlots\ 1.3 comes with user interface improvements. The technical distinction between ``behavior options'' and ``style options'' of older versions is no longer necessary (although still fully supported).

	\item \PGFPlots\ 1.3 has a new feature which allows to \emph{move axis labels tight to tick labels} automatically.

	Since this affects the spacing, it is not enabled be default.

	Use
\begin{codeexample}[code only]
\usepackage{pgfplots}
\pgfplotsset{compat=1.3}
\end{codeexample}
	in your preamble to benefit from the improved spacing\footnote{The |compat=1.3| setting is necessary for new features which move axis labels around like reversed axis. However, \PGFPlots\ will still work as it does in earlier versions, your documents will remain the same if you don't set it explicitly.}.
	Take a look at the next page for the precise definition of |compat|.

	\item \PGFPlots\ 1.3 now supports reversed axes. It is no longer necessary to use work--arounds with negative units.
\pgfkeys{/pdflinks/search key prefixes in/.add={/pgfplots/,}{}}

	Take a look at the |x dir=reverse| key.

	Existing work--arounds will still function properly. Use |\pgfplotsset{compat=1.3}| together with |x dir=reverse| to switch to the new version.
\end{enumerate}

\subsubsection{Old Features Which May Need Attention}
\begin{enumerate}
	\item The |scatter/classes| feature produces proper legends as of version 1.3. This may change the appearance of existing legends of plots with |scatter/classes|.

	\item Due to a small math bug in \PGF\ $2.00$, you \emph{can't} use the math expression `|-x^2|'. It is necessary to use `|0-x^2|' instead. The same holds for
		`|exp(-x^2)|'; use `|exp(0-x^2)|' instead.

		This will be fixed with \PGF\ $>2.00$.

	\item Starting with \PGFPlots\ $1.1$, |\tikzstyle| should \emph{no longer be used} to set \PGFPlots\ options.
	
	Although |\tikzstyle| is still supported for some older \PGFPlots\ options, you should replace any occurance of |\tikzstyle| with |\pgfplotsset{|\meta{style name}|/.style={|\meta{key-value-list}|}}| or the associated |/.append style| variant. See section~\ref{sec:styles} for more detail.
\end{enumerate}
I apologize for any inconvenience caused by these changes.

\begin{pgfplotskey}{compat=\mchoice{1.3,pre 1.3,default,newest} (initially default)}
	The preamble configuration 
\begin{codeexample}[code only]
\usepackage{pgfplots}
\pgfplotsset{compat=1.3}
\end{codeexample}
	allows to choose between backwards compatibility and most recent features.

	\PGFPlots\ version 1.3 comes with new features which may lead to different spacings compared to earlier versions:
	\begin{enumerate}
		\item Version 1.3 comes with the |xlabel near ticks| feature which places (in this case $x$) axis labels ``near'' the tick labels, taking the size of tick labels into account.
		
		Older versions used |xlabel absolute|, i.e.\ an absolute distance between the axis and axis labels (now matter how large or small tick labels are).

		The initial setting \emph{keeps backwards compatibility}.

		You are encouraged to use
\begin{codeexample}[code only]
\usepackage{pgfplots}
\pgfplotsset{compat=1.3}
\end{codeexample}
		in your preamble.
		
		\item Version 1.3 fixes a bug with |anchor=south west| which occurred mainly for reversed axes: the anchors where upside--down. This is fixed now.

		
		If you have written some work--around and want to keep it going, use

		|\pgfplotsset{compat/anchors=pre 1.3}|\pgfmanualpdflabel{/pgfplots/compat/anchors}{} 

		to disable the bug fix.
	\end{enumerate}

	Use |\pgfplotsset{compat=default}| to restore the factory settings.

	The setting |\pgfplotsset{compat=newest}| will always use features of the most recent version. This might result in small changes in the document's appearance.
\end{pgfplotskey}

\subsection{The Team}
\PGFPlots\ has been written mainly by Christian Feuers�nger with many improvements of Pascal Wolkotte and Nick Papior Andersen as a spare time project. We hope it is useful and provides valuable plots.

If you are interested in writing something but don't know how, consider reading the auxiliary manual \href{file:TeX-programming-notes.pdf}{TeX-programming-notes.pdf} which comes with \PGFPlots. It is far from complete, but maybe it is a good starting point (at least for more literature).

\subsection{Acknowledgements}
I thank God for all hours of enjoyed programming. I thank Pascal Wolkotte and Nick Papior Andersen for their programming efforts and contributions as part of the development team. I thank Stefan Tibus, who contributed the |plot shell| feature. I thank Tom Cashman for the contribution of the |reverse legend| feature. Special thanks go to Stefan Pinnow whose tests of \PGFPlots\ lead to numerous quality improvements. Furthermore, I thank Dr.~Schweitzer for many fruitful discussions and Dr.~Meine for his ideas and suggestions. Thanks as well to the many international contributors who provided feature requests or identified bugs or simply manual improvements!

Last but not least, I thank Till Tantau and Mark Wibrow for their excellent graphics (and more) package \PGF\ and \Tikz\ which is the base of \PGFPlots.

\include{pgfplots.install}%
\include{pgfplots.intro}%
\include{pgfplots.reference}%
\include{pgfplots.libs}%
\include{pgfplots.resources}%
\include{pgfplots.importexport}%
\include{pgfplots.basic.reference}%

\printindex

\bibliographystyle{abbrv} %gerapali} %gerabbrv} %gerunsrt.bst} %gerabbrv}% gerplain}
\nocite{pgfplotstable}
\nocite{programmingnotes}
\bibliography{pgfplots}
\end{document}
%
%%%%%%%%%%%%%%%%%%%%%%%%%%%%%%%%%%%%%%%%%%%%%%%%%%%%%%%%%%%%%%%%%%%%%%%%%%%%%
%
% Package pgfplots.sty documentation. 
%
% Copyright 2007/2008 by Christian Feuersaenger.
%
% This program is free software: you can redistribute it and/or modify
% it under the terms of the GNU General Public License as published by
% the Free Software Foundation, either version 3 of the License, or
% (at your option) any later version.
% 
% This program is distributed in the hope that it will be useful,
% but WITHOUT ANY WARRANTY; without even the implied warranty of
% MERCHANTABILITY or FITNESS FOR A PARTICULAR PURPOSE.  See the
% GNU General Public License for more details.
% 
% You should have received a copy of the GNU General Public License
% along with this program.  If not, see <http://www.gnu.org/licenses/>.
%
%
%%%%%%%%%%%%%%%%%%%%%%%%%%%%%%%%%%%%%%%%%%%%%%%%%%%%%%%%%%%%%%%%%%%%%%%%%%%%%
\input{pgfplots.preamble.tex}
%\RequirePackage[german,english,francais]{babel}

\pgfplotsmanualexternalexpensivetrue

%\usetikzlibrary{external}
%\tikzexternalize[prefix=figures/]{pgfplots}

\title{%
	Manual for Package \PGFPlots\\
	{\small Version \pgfplotsversion}\\
	{\small\href{http://sourceforge.net/projects/pgfplots}{http://sourceforge.net/projects/pgfplots}}}

%\includeonly{pgfplots.libs}


\begin{document}

\def\plotcoords{%
\addplot coordinates {
(5,8.312e-02)    (17,2.547e-02)   (49,7.407e-03)
(129,2.102e-03)  (321,5.874e-04)  (769,1.623e-04)
(1793,4.442e-05) (4097,1.207e-05) (9217,3.261e-06)
};

\addplot coordinates{
(7,8.472e-02)    (31,3.044e-02)    (111,1.022e-02)
(351,3.303e-03)  (1023,1.039e-03)  (2815,3.196e-04)
(7423,9.658e-05) (18943,2.873e-05) (47103,8.437e-06)
};

\addplot coordinates{
(9,7.881e-02)     (49,3.243e-02)    (209,1.232e-02)
(769,4.454e-03)   (2561,1.551e-03)  (7937,5.236e-04)
(23297,1.723e-04) (65537,5.545e-05) (178177,1.751e-05)
};

\addplot coordinates{
(11,6.887e-02)    (71,3.177e-02)     (351,1.341e-02)
(1471,5.334e-03)  (5503,2.027e-03)   (18943,7.415e-04)
(61183,2.628e-04) (187903,9.063e-05) (553983,3.053e-05)
};

\addplot coordinates{
(13,5.755e-02)     (97,2.925e-02)     (545,1.351e-02)
(2561,5.842e-03)   (10625,2.397e-03)  (40193,9.414e-04)
(141569,3.564e-04) (471041,1.308e-04) 
(1496065,4.670e-05)
};
}%
\maketitle
\begin{abstract}%
\PGFPlots\ draws high--quality function plots in normal or logarithmic scaling with a user-friendly interface directly in \TeX. The user supplies axis labels, legend entries and the plot coordinates for one or more plots and \PGFPlots\ applies axis scaling, computes any logarithms and axis ticks and draws the plots, supporting line plots, scatter plots, piecewise constant plots, bar plots, area plots, mesh-- and surface plots and some more. It is based on Till Tantau's package \PGF/\Tikz.
\end{abstract}
\tableofcontents
\section{Introduction}
This package provides tools to generate plots and labeled axes easily. It draws normal plots, logplots and semi-logplots, in two and three dimensions. Axis ticks, labels, legends (in case of multiple plots) can be added with key-value options. It can cycle through a set of predefined line/marker/color specifications. In summary, its purpose is to simplify the generation of high-quality function and/or data plots, and solving the problems of
\begin{itemize}
	\item consistency of document and font type and font size,
	\item direct use of \TeX\ math mode in axis descriptions,
	\item consistency of data and figures (no third party tool necessary),
	\item inter document consistency using preamble configurations and styles.
\end{itemize}
Although not necessary, separate |.pdf| or |.eps| graphics can be generated using the |external| library developed as part of \Tikz.

\PGFPlots\ is build completely on \Tikz/\PGF. Knowledge of \Tikz\ will simplify the work with \PGFPlots, although it is not required.

\section[About PGFPlots: Preliminaries]{About {\normalfont\PGFPlots}: Preliminaries}
This section contains information about upgrades, the team, the installation (in case you need to do it manually) and troubleshooting. You may skip it completely except for the upgrade remarks.

However, note that this library requires at least \PGF\ version $2.00$. At the time of this writing, many \TeX-distributions still contain the older \PGF\ version $1.18$, so it may be necessary to install a recent \PGF\ prior to using \PGFPlots.

\subsection{Components}
\PGFPlots\ comes with two components:
\begin{enumerate}
	\item the plotting component (which you are currently reading) and
	\item the \PGFPlotstable\ component which simplifies number formatting and postprocessing of numerical tables. It comes as a separate package and has its own manual \href{file:pgfplotstable.pdf}{pgfplotstable.pdf}.
\end{enumerate}

\subsection{Upgrade remarks}
This release provides a lot of improvements which can be found in all detail in \texttt{ChangeLog} for interested readers. However, some attention is useful with respect to the following changes.

\subsubsection{New Optional Features}
\begin{enumerate}
	\item \PGFPlots\ 1.3 comes with user interface improvements. The technical distinction between ``behavior options'' and ``style options'' of older versions is no longer necessary (although still fully supported).

	\item \PGFPlots\ 1.3 has a new feature which allows to \emph{move axis labels tight to tick labels} automatically.

	Since this affects the spacing, it is not enabled be default.

	Use
\begin{codeexample}[code only]
\usepackage{pgfplots}
\pgfplotsset{compat=1.3}
\end{codeexample}
	in your preamble to benefit from the improved spacing\footnote{The |compat=1.3| setting is necessary for new features which move axis labels around like reversed axis. However, \PGFPlots\ will still work as it does in earlier versions, your documents will remain the same if you don't set it explicitly.}.
	Take a look at the next page for the precise definition of |compat|.

	\item \PGFPlots\ 1.3 now supports reversed axes. It is no longer necessary to use work--arounds with negative units.
\pgfkeys{/pdflinks/search key prefixes in/.add={/pgfplots/,}{}}

	Take a look at the |x dir=reverse| key.

	Existing work--arounds will still function properly. Use |\pgfplotsset{compat=1.3}| together with |x dir=reverse| to switch to the new version.
\end{enumerate}

\subsubsection{Old Features Which May Need Attention}
\begin{enumerate}
	\item The |scatter/classes| feature produces proper legends as of version 1.3. This may change the appearance of existing legends of plots with |scatter/classes|.

	\item Due to a small math bug in \PGF\ $2.00$, you \emph{can't} use the math expression `|-x^2|'. It is necessary to use `|0-x^2|' instead. The same holds for
		`|exp(-x^2)|'; use `|exp(0-x^2)|' instead.

		This will be fixed with \PGF\ $>2.00$.

	\item Starting with \PGFPlots\ $1.1$, |\tikzstyle| should \emph{no longer be used} to set \PGFPlots\ options.
	
	Although |\tikzstyle| is still supported for some older \PGFPlots\ options, you should replace any occurance of |\tikzstyle| with |\pgfplotsset{|\meta{style name}|/.style={|\meta{key-value-list}|}}| or the associated |/.append style| variant. See section~\ref{sec:styles} for more detail.
\end{enumerate}
I apologize for any inconvenience caused by these changes.

\begin{pgfplotskey}{compat=\mchoice{1.3,pre 1.3,default,newest} (initially default)}
	The preamble configuration 
\begin{codeexample}[code only]
\usepackage{pgfplots}
\pgfplotsset{compat=1.3}
\end{codeexample}
	allows to choose between backwards compatibility and most recent features.

	\PGFPlots\ version 1.3 comes with new features which may lead to different spacings compared to earlier versions:
	\begin{enumerate}
		\item Version 1.3 comes with the |xlabel near ticks| feature which places (in this case $x$) axis labels ``near'' the tick labels, taking the size of tick labels into account.
		
		Older versions used |xlabel absolute|, i.e.\ an absolute distance between the axis and axis labels (now matter how large or small tick labels are).

		The initial setting \emph{keeps backwards compatibility}.

		You are encouraged to use
\begin{codeexample}[code only]
\usepackage{pgfplots}
\pgfplotsset{compat=1.3}
\end{codeexample}
		in your preamble.
		
		\item Version 1.3 fixes a bug with |anchor=south west| which occurred mainly for reversed axes: the anchors where upside--down. This is fixed now.

		
		If you have written some work--around and want to keep it going, use

		|\pgfplotsset{compat/anchors=pre 1.3}|\pgfmanualpdflabel{/pgfplots/compat/anchors}{} 

		to disable the bug fix.
	\end{enumerate}

	Use |\pgfplotsset{compat=default}| to restore the factory settings.

	The setting |\pgfplotsset{compat=newest}| will always use features of the most recent version. This might result in small changes in the document's appearance.
\end{pgfplotskey}

\subsection{The Team}
\PGFPlots\ has been written mainly by Christian Feuers�nger with many improvements of Pascal Wolkotte and Nick Papior Andersen as a spare time project. We hope it is useful and provides valuable plots.

If you are interested in writing something but don't know how, consider reading the auxiliary manual \href{file:TeX-programming-notes.pdf}{TeX-programming-notes.pdf} which comes with \PGFPlots. It is far from complete, but maybe it is a good starting point (at least for more literature).

\subsection{Acknowledgements}
I thank God for all hours of enjoyed programming. I thank Pascal Wolkotte and Nick Papior Andersen for their programming efforts and contributions as part of the development team. I thank Stefan Tibus, who contributed the |plot shell| feature. I thank Tom Cashman for the contribution of the |reverse legend| feature. Special thanks go to Stefan Pinnow whose tests of \PGFPlots\ lead to numerous quality improvements. Furthermore, I thank Dr.~Schweitzer for many fruitful discussions and Dr.~Meine for his ideas and suggestions. Thanks as well to the many international contributors who provided feature requests or identified bugs or simply manual improvements!

Last but not least, I thank Till Tantau and Mark Wibrow for their excellent graphics (and more) package \PGF\ and \Tikz\ which is the base of \PGFPlots.

\include{pgfplots.install}%
\include{pgfplots.intro}%
\include{pgfplots.reference}%
\include{pgfplots.libs}%
\include{pgfplots.resources}%
\include{pgfplots.importexport}%
\include{pgfplots.basic.reference}%

\printindex

\bibliographystyle{abbrv} %gerapali} %gerabbrv} %gerunsrt.bst} %gerabbrv}% gerplain}
\nocite{pgfplotstable}
\nocite{programmingnotes}
\bibliography{pgfplots}
\end{document}
%
\include{pgfplots.basic.reference}%

\printindex

\bibliographystyle{abbrv} %gerapali} %gerabbrv} %gerunsrt.bst} %gerabbrv}% gerplain}
\nocite{pgfplotstable}
\nocite{programmingnotes}
\bibliography{pgfplots}
\end{document}
%
%%%%%%%%%%%%%%%%%%%%%%%%%%%%%%%%%%%%%%%%%%%%%%%%%%%%%%%%%%%%%%%%%%%%%%%%%%%%%
%
% Package pgfplots.sty documentation. 
%
% Copyright 2007/2008 by Christian Feuersaenger.
%
% This program is free software: you can redistribute it and/or modify
% it under the terms of the GNU General Public License as published by
% the Free Software Foundation, either version 3 of the License, or
% (at your option) any later version.
% 
% This program is distributed in the hope that it will be useful,
% but WITHOUT ANY WARRANTY; without even the implied warranty of
% MERCHANTABILITY or FITNESS FOR A PARTICULAR PURPOSE.  See the
% GNU General Public License for more details.
% 
% You should have received a copy of the GNU General Public License
% along with this program.  If not, see <http://www.gnu.org/licenses/>.
%
%
%%%%%%%%%%%%%%%%%%%%%%%%%%%%%%%%%%%%%%%%%%%%%%%%%%%%%%%%%%%%%%%%%%%%%%%%%%%%%
%%%%%%%%%%%%%%%%%%%%%%%%%%%%%%%%%%%%%%%%%%%%%%%%%%%%%%%%%%%%%%%%%%%%%%%%%%%%%
%
% Package pgfplots.sty documentation. 
%
% Copyright 2007/2008 by Christian Feuersaenger.
%
% This program is free software: you can redistribute it and/or modify
% it under the terms of the GNU General Public License as published by
% the Free Software Foundation, either version 3 of the License, or
% (at your option) any later version.
% 
% This program is distributed in the hope that it will be useful,
% but WITHOUT ANY WARRANTY; without even the implied warranty of
% MERCHANTABILITY or FITNESS FOR A PARTICULAR PURPOSE.  See the
% GNU General Public License for more details.
% 
% You should have received a copy of the GNU General Public License
% along with this program.  If not, see <http://www.gnu.org/licenses/>.
%
%
%%%%%%%%%%%%%%%%%%%%%%%%%%%%%%%%%%%%%%%%%%%%%%%%%%%%%%%%%%%%%%%%%%%%%%%%%%%%%
\documentclass[a4paper]{ltxdoc}

\usepackage{makeidx}

% DON't let hyperref overload the format of index and glossary. 
% I want to do that on my own in the stylefiles for makeindex...
\makeatletter
\let\@old@wrindex=\@wrindex
\makeatother

\usepackage{ifpdf}
\ifpdf
	\usepackage[pdftex]{hyperref}
\else
	\usepackage[dvipdfm]{hyperref}
\fi
\hypersetup{pdfborder={0 0 0}}

\makeatletter
\let\@wrindex=\@old@wrindex
\makeatother

% Formatiere Seitennummern im Index:
\newcommand{\indexpageno}[1]{%
	{\bfseries\hyperpage{#1}}%
}


\newcommand{\R}{\mathbb{R}}
\newcommand{\N}{\mathbb{N}}
\newcommand{\Z}{\mathbb{Z}}

\long\def\COMMENTLOWLEVEL#1\ENDCOMMENT{}
\def\ENDCOMMENT{}

\usepackage{textcomp}
\usepackage{booktabs}

\usepackage{calc}
\usepackage[formats]{listings}
%\usepackage{courier} % don't use it - the '^' character can't be copy-pasted in courier

\usepackage{array}
\lstset{%
	basicstyle=\ttfamily,
	language=[LaTeX]tex, % Seems as if \lstset{language=tex} must be invoked BEFORE loading tikz!?
	tabsize=4,
	breaklines=true,
	breakindent=0pt
}

\ifpdf
	\pdfinfo {
		/Author	(Christian Feuersaenger)
	}

\else
	\def\pgfsysdriver{pgfsys-dvipdfm.def}
\fi
%\def\pgfsysdriver{pgfsys-pdftex.def}
\usepackage{pgfplots}

\ifpdf
	\usetikzlibrary{pgfplotsclickable}
\fi

%\usepackage{fp}
% ATTENTION:
% this requires pgf version NEWER than 2.00 :
%\usetikzlibrary{fixedpointarithmetic}

\usetikzlibrary{dateplot}

\usepackage[a4paper,left=2.25cm,right=2.25cm,top=2.5cm,bottom=2.5cm,nohead]{geometry}
\usepackage{amsmath,amssymb}
\usepackage{xxcolor}
\usepackage{pifont}
\usepackage[latin1]{inputenc}
\usepackage{amsmath}
\usepackage{eurosym}
\input{pgfplots-macros}

\pgfqkeys{/codeexample}{%
	every codeexample/.style={
		width=8cm,
		/pgfplots/every axis/.append style={legend style={fill=graphicbackground}}
	},
	tabsize=4,
}
\pgfplotsset{
	every axis/.append style={width=7cm},
	filter discard warning=false,
}

\usetikzlibrary{backgrounds,patterns}
% Global styles:
\tikzset{
  shape example/.style={
    color=black!30,
    draw,
    fill=yellow!30,
    line width=.5cm,
    inner xsep=2.5cm,
    inner ysep=0.5cm}
}

\newcommand{\FIXME}[1]{\textcolor{red}{(FIXME: #1)}}

% fuer endvironment 'sidewaysfigure' bspw
% \usepackage{rotating}

\newcommand\Tikz{Ti\textit kZ}
\newcommand\PGF{\textsc{pgf}}
\newcommand\PGFPlots{\textsc{pgfplots}}
\def\PGFPlotstable{\textsc{PgfplotsTable}}


\makeindex

% Fix overful hboxes automatically:
\tolerance=2000
\emergencystretch=10pt

\tikzset{prefix=gnuplot/pgfplots_} % prefix for 'plot function'

\author{%
	Christian Feuers\"anger\footnote{\url{http://wissrech.ins.uni-bonn.de/people/feuersaenger}}\\%
	Institut f\"ur Numerische Simulation\\
	Universit\"at Bonn, Germany}


%\RequirePackage[german,english,francais]{babel}

\pgfplotsmanualexternalexpensivetrue

%\usetikzlibrary{external}
%\tikzexternalize[prefix=figures/]{pgfplots}

\title{%
	Manual for Package \PGFPlots\\
	{\small Version \pgfplotsversion}\\
	{\small\href{http://sourceforge.net/projects/pgfplots}{http://sourceforge.net/projects/pgfplots}}}

%\includeonly{pgfplots.libs}


\begin{document}

\def\plotcoords{%
\addplot coordinates {
(5,8.312e-02)    (17,2.547e-02)   (49,7.407e-03)
(129,2.102e-03)  (321,5.874e-04)  (769,1.623e-04)
(1793,4.442e-05) (4097,1.207e-05) (9217,3.261e-06)
};

\addplot coordinates{
(7,8.472e-02)    (31,3.044e-02)    (111,1.022e-02)
(351,3.303e-03)  (1023,1.039e-03)  (2815,3.196e-04)
(7423,9.658e-05) (18943,2.873e-05) (47103,8.437e-06)
};

\addplot coordinates{
(9,7.881e-02)     (49,3.243e-02)    (209,1.232e-02)
(769,4.454e-03)   (2561,1.551e-03)  (7937,5.236e-04)
(23297,1.723e-04) (65537,5.545e-05) (178177,1.751e-05)
};

\addplot coordinates{
(11,6.887e-02)    (71,3.177e-02)     (351,1.341e-02)
(1471,5.334e-03)  (5503,2.027e-03)   (18943,7.415e-04)
(61183,2.628e-04) (187903,9.063e-05) (553983,3.053e-05)
};

\addplot coordinates{
(13,5.755e-02)     (97,2.925e-02)     (545,1.351e-02)
(2561,5.842e-03)   (10625,2.397e-03)  (40193,9.414e-04)
(141569,3.564e-04) (471041,1.308e-04) 
(1496065,4.670e-05)
};
}%
\maketitle
\begin{abstract}%
\PGFPlots\ draws high--quality function plots in normal or logarithmic scaling with a user-friendly interface directly in \TeX. The user supplies axis labels, legend entries and the plot coordinates for one or more plots and \PGFPlots\ applies axis scaling, computes any logarithms and axis ticks and draws the plots, supporting line plots, scatter plots, piecewise constant plots, bar plots, area plots, mesh-- and surface plots and some more. It is based on Till Tantau's package \PGF/\Tikz.
\end{abstract}
\tableofcontents
\section{Introduction}
This package provides tools to generate plots and labeled axes easily. It draws normal plots, logplots and semi-logplots, in two and three dimensions. Axis ticks, labels, legends (in case of multiple plots) can be added with key-value options. It can cycle through a set of predefined line/marker/color specifications. In summary, its purpose is to simplify the generation of high-quality function and/or data plots, and solving the problems of
\begin{itemize}
	\item consistency of document and font type and font size,
	\item direct use of \TeX\ math mode in axis descriptions,
	\item consistency of data and figures (no third party tool necessary),
	\item inter document consistency using preamble configurations and styles.
\end{itemize}
Although not necessary, separate |.pdf| or |.eps| graphics can be generated using the |external| library developed as part of \Tikz.

\PGFPlots\ is build completely on \Tikz/\PGF. Knowledge of \Tikz\ will simplify the work with \PGFPlots, although it is not required.

\section[About PGFPlots: Preliminaries]{About {\normalfont\PGFPlots}: Preliminaries}
This section contains information about upgrades, the team, the installation (in case you need to do it manually) and troubleshooting. You may skip it completely except for the upgrade remarks.

However, note that this library requires at least \PGF\ version $2.00$. At the time of this writing, many \TeX-distributions still contain the older \PGF\ version $1.18$, so it may be necessary to install a recent \PGF\ prior to using \PGFPlots.

\subsection{Components}
\PGFPlots\ comes with two components:
\begin{enumerate}
	\item the plotting component (which you are currently reading) and
	\item the \PGFPlotstable\ component which simplifies number formatting and postprocessing of numerical tables. It comes as a separate package and has its own manual \href{file:pgfplotstable.pdf}{pgfplotstable.pdf}.
\end{enumerate}

\subsection{Upgrade remarks}
This release provides a lot of improvements which can be found in all detail in \texttt{ChangeLog} for interested readers. However, some attention is useful with respect to the following changes.

\subsubsection{New Optional Features}
\begin{enumerate}
	\item \PGFPlots\ 1.3 comes with user interface improvements. The technical distinction between ``behavior options'' and ``style options'' of older versions is no longer necessary (although still fully supported).

	\item \PGFPlots\ 1.3 has a new feature which allows to \emph{move axis labels tight to tick labels} automatically.

	Since this affects the spacing, it is not enabled be default.

	Use
\begin{codeexample}[code only]
\usepackage{pgfplots}
\pgfplotsset{compat=1.3}
\end{codeexample}
	in your preamble to benefit from the improved spacing\footnote{The |compat=1.3| setting is necessary for new features which move axis labels around like reversed axis. However, \PGFPlots\ will still work as it does in earlier versions, your documents will remain the same if you don't set it explicitly.}.
	Take a look at the next page for the precise definition of |compat|.

	\item \PGFPlots\ 1.3 now supports reversed axes. It is no longer necessary to use work--arounds with negative units.
\pgfkeys{/pdflinks/search key prefixes in/.add={/pgfplots/,}{}}

	Take a look at the |x dir=reverse| key.

	Existing work--arounds will still function properly. Use |\pgfplotsset{compat=1.3}| together with |x dir=reverse| to switch to the new version.
\end{enumerate}

\subsubsection{Old Features Which May Need Attention}
\begin{enumerate}
	\item The |scatter/classes| feature produces proper legends as of version 1.3. This may change the appearance of existing legends of plots with |scatter/classes|.

	\item Due to a small math bug in \PGF\ $2.00$, you \emph{can't} use the math expression `|-x^2|'. It is necessary to use `|0-x^2|' instead. The same holds for
		`|exp(-x^2)|'; use `|exp(0-x^2)|' instead.

		This will be fixed with \PGF\ $>2.00$.

	\item Starting with \PGFPlots\ $1.1$, |\tikzstyle| should \emph{no longer be used} to set \PGFPlots\ options.
	
	Although |\tikzstyle| is still supported for some older \PGFPlots\ options, you should replace any occurance of |\tikzstyle| with |\pgfplotsset{|\meta{style name}|/.style={|\meta{key-value-list}|}}| or the associated |/.append style| variant. See section~\ref{sec:styles} for more detail.
\end{enumerate}
I apologize for any inconvenience caused by these changes.

\begin{pgfplotskey}{compat=\mchoice{1.3,pre 1.3,default,newest} (initially default)}
	The preamble configuration 
\begin{codeexample}[code only]
\usepackage{pgfplots}
\pgfplotsset{compat=1.3}
\end{codeexample}
	allows to choose between backwards compatibility and most recent features.

	\PGFPlots\ version 1.3 comes with new features which may lead to different spacings compared to earlier versions:
	\begin{enumerate}
		\item Version 1.3 comes with the |xlabel near ticks| feature which places (in this case $x$) axis labels ``near'' the tick labels, taking the size of tick labels into account.
		
		Older versions used |xlabel absolute|, i.e.\ an absolute distance between the axis and axis labels (now matter how large or small tick labels are).

		The initial setting \emph{keeps backwards compatibility}.

		You are encouraged to use
\begin{codeexample}[code only]
\usepackage{pgfplots}
\pgfplotsset{compat=1.3}
\end{codeexample}
		in your preamble.
		
		\item Version 1.3 fixes a bug with |anchor=south west| which occurred mainly for reversed axes: the anchors where upside--down. This is fixed now.

		
		If you have written some work--around and want to keep it going, use

		|\pgfplotsset{compat/anchors=pre 1.3}|\pgfmanualpdflabel{/pgfplots/compat/anchors}{} 

		to disable the bug fix.
	\end{enumerate}

	Use |\pgfplotsset{compat=default}| to restore the factory settings.

	The setting |\pgfplotsset{compat=newest}| will always use features of the most recent version. This might result in small changes in the document's appearance.
\end{pgfplotskey}

\subsection{The Team}
\PGFPlots\ has been written mainly by Christian Feuers�nger with many improvements of Pascal Wolkotte and Nick Papior Andersen as a spare time project. We hope it is useful and provides valuable plots.

If you are interested in writing something but don't know how, consider reading the auxiliary manual \href{file:TeX-programming-notes.pdf}{TeX-programming-notes.pdf} which comes with \PGFPlots. It is far from complete, but maybe it is a good starting point (at least for more literature).

\subsection{Acknowledgements}
I thank God for all hours of enjoyed programming. I thank Pascal Wolkotte and Nick Papior Andersen for their programming efforts and contributions as part of the development team. I thank Stefan Tibus, who contributed the |plot shell| feature. I thank Tom Cashman for the contribution of the |reverse legend| feature. Special thanks go to Stefan Pinnow whose tests of \PGFPlots\ lead to numerous quality improvements. Furthermore, I thank Dr.~Schweitzer for many fruitful discussions and Dr.~Meine for his ideas and suggestions. Thanks as well to the many international contributors who provided feature requests or identified bugs or simply manual improvements!

Last but not least, I thank Till Tantau and Mark Wibrow for their excellent graphics (and more) package \PGF\ and \Tikz\ which is the base of \PGFPlots.

%%%%%%%%%%%%%%%%%%%%%%%%%%%%%%%%%%%%%%%%%%%%%%%%%%%%%%%%%%%%%%%%%%%%%%%%%%%%%
%
% Package pgfplots.sty documentation. 
%
% Copyright 2007/2008 by Christian Feuersaenger.
%
% This program is free software: you can redistribute it and/or modify
% it under the terms of the GNU General Public License as published by
% the Free Software Foundation, either version 3 of the License, or
% (at your option) any later version.
% 
% This program is distributed in the hope that it will be useful,
% but WITHOUT ANY WARRANTY; without even the implied warranty of
% MERCHANTABILITY or FITNESS FOR A PARTICULAR PURPOSE.  See the
% GNU General Public License for more details.
% 
% You should have received a copy of the GNU General Public License
% along with this program.  If not, see <http://www.gnu.org/licenses/>.
%
%
%%%%%%%%%%%%%%%%%%%%%%%%%%%%%%%%%%%%%%%%%%%%%%%%%%%%%%%%%%%%%%%%%%%%%%%%%%%%%
\input{pgfplots.preamble.tex}
%\RequirePackage[german,english,francais]{babel}

\pgfplotsmanualexternalexpensivetrue

%\usetikzlibrary{external}
%\tikzexternalize[prefix=figures/]{pgfplots}

\title{%
	Manual for Package \PGFPlots\\
	{\small Version \pgfplotsversion}\\
	{\small\href{http://sourceforge.net/projects/pgfplots}{http://sourceforge.net/projects/pgfplots}}}

%\includeonly{pgfplots.libs}


\begin{document}

\def\plotcoords{%
\addplot coordinates {
(5,8.312e-02)    (17,2.547e-02)   (49,7.407e-03)
(129,2.102e-03)  (321,5.874e-04)  (769,1.623e-04)
(1793,4.442e-05) (4097,1.207e-05) (9217,3.261e-06)
};

\addplot coordinates{
(7,8.472e-02)    (31,3.044e-02)    (111,1.022e-02)
(351,3.303e-03)  (1023,1.039e-03)  (2815,3.196e-04)
(7423,9.658e-05) (18943,2.873e-05) (47103,8.437e-06)
};

\addplot coordinates{
(9,7.881e-02)     (49,3.243e-02)    (209,1.232e-02)
(769,4.454e-03)   (2561,1.551e-03)  (7937,5.236e-04)
(23297,1.723e-04) (65537,5.545e-05) (178177,1.751e-05)
};

\addplot coordinates{
(11,6.887e-02)    (71,3.177e-02)     (351,1.341e-02)
(1471,5.334e-03)  (5503,2.027e-03)   (18943,7.415e-04)
(61183,2.628e-04) (187903,9.063e-05) (553983,3.053e-05)
};

\addplot coordinates{
(13,5.755e-02)     (97,2.925e-02)     (545,1.351e-02)
(2561,5.842e-03)   (10625,2.397e-03)  (40193,9.414e-04)
(141569,3.564e-04) (471041,1.308e-04) 
(1496065,4.670e-05)
};
}%
\maketitle
\begin{abstract}%
\PGFPlots\ draws high--quality function plots in normal or logarithmic scaling with a user-friendly interface directly in \TeX. The user supplies axis labels, legend entries and the plot coordinates for one or more plots and \PGFPlots\ applies axis scaling, computes any logarithms and axis ticks and draws the plots, supporting line plots, scatter plots, piecewise constant plots, bar plots, area plots, mesh-- and surface plots and some more. It is based on Till Tantau's package \PGF/\Tikz.
\end{abstract}
\tableofcontents
\section{Introduction}
This package provides tools to generate plots and labeled axes easily. It draws normal plots, logplots and semi-logplots, in two and three dimensions. Axis ticks, labels, legends (in case of multiple plots) can be added with key-value options. It can cycle through a set of predefined line/marker/color specifications. In summary, its purpose is to simplify the generation of high-quality function and/or data plots, and solving the problems of
\begin{itemize}
	\item consistency of document and font type and font size,
	\item direct use of \TeX\ math mode in axis descriptions,
	\item consistency of data and figures (no third party tool necessary),
	\item inter document consistency using preamble configurations and styles.
\end{itemize}
Although not necessary, separate |.pdf| or |.eps| graphics can be generated using the |external| library developed as part of \Tikz.

\PGFPlots\ is build completely on \Tikz/\PGF. Knowledge of \Tikz\ will simplify the work with \PGFPlots, although it is not required.

\section[About PGFPlots: Preliminaries]{About {\normalfont\PGFPlots}: Preliminaries}
This section contains information about upgrades, the team, the installation (in case you need to do it manually) and troubleshooting. You may skip it completely except for the upgrade remarks.

However, note that this library requires at least \PGF\ version $2.00$. At the time of this writing, many \TeX-distributions still contain the older \PGF\ version $1.18$, so it may be necessary to install a recent \PGF\ prior to using \PGFPlots.

\subsection{Components}
\PGFPlots\ comes with two components:
\begin{enumerate}
	\item the plotting component (which you are currently reading) and
	\item the \PGFPlotstable\ component which simplifies number formatting and postprocessing of numerical tables. It comes as a separate package and has its own manual \href{file:pgfplotstable.pdf}{pgfplotstable.pdf}.
\end{enumerate}

\subsection{Upgrade remarks}
This release provides a lot of improvements which can be found in all detail in \texttt{ChangeLog} for interested readers. However, some attention is useful with respect to the following changes.

\subsubsection{New Optional Features}
\begin{enumerate}
	\item \PGFPlots\ 1.3 comes with user interface improvements. The technical distinction between ``behavior options'' and ``style options'' of older versions is no longer necessary (although still fully supported).

	\item \PGFPlots\ 1.3 has a new feature which allows to \emph{move axis labels tight to tick labels} automatically.

	Since this affects the spacing, it is not enabled be default.

	Use
\begin{codeexample}[code only]
\usepackage{pgfplots}
\pgfplotsset{compat=1.3}
\end{codeexample}
	in your preamble to benefit from the improved spacing\footnote{The |compat=1.3| setting is necessary for new features which move axis labels around like reversed axis. However, \PGFPlots\ will still work as it does in earlier versions, your documents will remain the same if you don't set it explicitly.}.
	Take a look at the next page for the precise definition of |compat|.

	\item \PGFPlots\ 1.3 now supports reversed axes. It is no longer necessary to use work--arounds with negative units.
\pgfkeys{/pdflinks/search key prefixes in/.add={/pgfplots/,}{}}

	Take a look at the |x dir=reverse| key.

	Existing work--arounds will still function properly. Use |\pgfplotsset{compat=1.3}| together with |x dir=reverse| to switch to the new version.
\end{enumerate}

\subsubsection{Old Features Which May Need Attention}
\begin{enumerate}
	\item The |scatter/classes| feature produces proper legends as of version 1.3. This may change the appearance of existing legends of plots with |scatter/classes|.

	\item Due to a small math bug in \PGF\ $2.00$, you \emph{can't} use the math expression `|-x^2|'. It is necessary to use `|0-x^2|' instead. The same holds for
		`|exp(-x^2)|'; use `|exp(0-x^2)|' instead.

		This will be fixed with \PGF\ $>2.00$.

	\item Starting with \PGFPlots\ $1.1$, |\tikzstyle| should \emph{no longer be used} to set \PGFPlots\ options.
	
	Although |\tikzstyle| is still supported for some older \PGFPlots\ options, you should replace any occurance of |\tikzstyle| with |\pgfplotsset{|\meta{style name}|/.style={|\meta{key-value-list}|}}| or the associated |/.append style| variant. See section~\ref{sec:styles} for more detail.
\end{enumerate}
I apologize for any inconvenience caused by these changes.

\begin{pgfplotskey}{compat=\mchoice{1.3,pre 1.3,default,newest} (initially default)}
	The preamble configuration 
\begin{codeexample}[code only]
\usepackage{pgfplots}
\pgfplotsset{compat=1.3}
\end{codeexample}
	allows to choose between backwards compatibility and most recent features.

	\PGFPlots\ version 1.3 comes with new features which may lead to different spacings compared to earlier versions:
	\begin{enumerate}
		\item Version 1.3 comes with the |xlabel near ticks| feature which places (in this case $x$) axis labels ``near'' the tick labels, taking the size of tick labels into account.
		
		Older versions used |xlabel absolute|, i.e.\ an absolute distance between the axis and axis labels (now matter how large or small tick labels are).

		The initial setting \emph{keeps backwards compatibility}.

		You are encouraged to use
\begin{codeexample}[code only]
\usepackage{pgfplots}
\pgfplotsset{compat=1.3}
\end{codeexample}
		in your preamble.
		
		\item Version 1.3 fixes a bug with |anchor=south west| which occurred mainly for reversed axes: the anchors where upside--down. This is fixed now.

		
		If you have written some work--around and want to keep it going, use

		|\pgfplotsset{compat/anchors=pre 1.3}|\pgfmanualpdflabel{/pgfplots/compat/anchors}{} 

		to disable the bug fix.
	\end{enumerate}

	Use |\pgfplotsset{compat=default}| to restore the factory settings.

	The setting |\pgfplotsset{compat=newest}| will always use features of the most recent version. This might result in small changes in the document's appearance.
\end{pgfplotskey}

\subsection{The Team}
\PGFPlots\ has been written mainly by Christian Feuers�nger with many improvements of Pascal Wolkotte and Nick Papior Andersen as a spare time project. We hope it is useful and provides valuable plots.

If you are interested in writing something but don't know how, consider reading the auxiliary manual \href{file:TeX-programming-notes.pdf}{TeX-programming-notes.pdf} which comes with \PGFPlots. It is far from complete, but maybe it is a good starting point (at least for more literature).

\subsection{Acknowledgements}
I thank God for all hours of enjoyed programming. I thank Pascal Wolkotte and Nick Papior Andersen for their programming efforts and contributions as part of the development team. I thank Stefan Tibus, who contributed the |plot shell| feature. I thank Tom Cashman for the contribution of the |reverse legend| feature. Special thanks go to Stefan Pinnow whose tests of \PGFPlots\ lead to numerous quality improvements. Furthermore, I thank Dr.~Schweitzer for many fruitful discussions and Dr.~Meine for his ideas and suggestions. Thanks as well to the many international contributors who provided feature requests or identified bugs or simply manual improvements!

Last but not least, I thank Till Tantau and Mark Wibrow for their excellent graphics (and more) package \PGF\ and \Tikz\ which is the base of \PGFPlots.

\include{pgfplots.install}%
\include{pgfplots.intro}%
\include{pgfplots.reference}%
\include{pgfplots.libs}%
\include{pgfplots.resources}%
\include{pgfplots.importexport}%
\include{pgfplots.basic.reference}%

\printindex

\bibliographystyle{abbrv} %gerapali} %gerabbrv} %gerunsrt.bst} %gerabbrv}% gerplain}
\nocite{pgfplotstable}
\nocite{programmingnotes}
\bibliography{pgfplots}
\end{document}
%
%%%%%%%%%%%%%%%%%%%%%%%%%%%%%%%%%%%%%%%%%%%%%%%%%%%%%%%%%%%%%%%%%%%%%%%%%%%%%
%
% Package pgfplots.sty documentation. 
%
% Copyright 2007/2008 by Christian Feuersaenger.
%
% This program is free software: you can redistribute it and/or modify
% it under the terms of the GNU General Public License as published by
% the Free Software Foundation, either version 3 of the License, or
% (at your option) any later version.
% 
% This program is distributed in the hope that it will be useful,
% but WITHOUT ANY WARRANTY; without even the implied warranty of
% MERCHANTABILITY or FITNESS FOR A PARTICULAR PURPOSE.  See the
% GNU General Public License for more details.
% 
% You should have received a copy of the GNU General Public License
% along with this program.  If not, see <http://www.gnu.org/licenses/>.
%
%
%%%%%%%%%%%%%%%%%%%%%%%%%%%%%%%%%%%%%%%%%%%%%%%%%%%%%%%%%%%%%%%%%%%%%%%%%%%%%
\input{pgfplots.preamble.tex}
%\RequirePackage[german,english,francais]{babel}

\pgfplotsmanualexternalexpensivetrue

%\usetikzlibrary{external}
%\tikzexternalize[prefix=figures/]{pgfplots}

\title{%
	Manual for Package \PGFPlots\\
	{\small Version \pgfplotsversion}\\
	{\small\href{http://sourceforge.net/projects/pgfplots}{http://sourceforge.net/projects/pgfplots}}}

%\includeonly{pgfplots.libs}


\begin{document}

\def\plotcoords{%
\addplot coordinates {
(5,8.312e-02)    (17,2.547e-02)   (49,7.407e-03)
(129,2.102e-03)  (321,5.874e-04)  (769,1.623e-04)
(1793,4.442e-05) (4097,1.207e-05) (9217,3.261e-06)
};

\addplot coordinates{
(7,8.472e-02)    (31,3.044e-02)    (111,1.022e-02)
(351,3.303e-03)  (1023,1.039e-03)  (2815,3.196e-04)
(7423,9.658e-05) (18943,2.873e-05) (47103,8.437e-06)
};

\addplot coordinates{
(9,7.881e-02)     (49,3.243e-02)    (209,1.232e-02)
(769,4.454e-03)   (2561,1.551e-03)  (7937,5.236e-04)
(23297,1.723e-04) (65537,5.545e-05) (178177,1.751e-05)
};

\addplot coordinates{
(11,6.887e-02)    (71,3.177e-02)     (351,1.341e-02)
(1471,5.334e-03)  (5503,2.027e-03)   (18943,7.415e-04)
(61183,2.628e-04) (187903,9.063e-05) (553983,3.053e-05)
};

\addplot coordinates{
(13,5.755e-02)     (97,2.925e-02)     (545,1.351e-02)
(2561,5.842e-03)   (10625,2.397e-03)  (40193,9.414e-04)
(141569,3.564e-04) (471041,1.308e-04) 
(1496065,4.670e-05)
};
}%
\maketitle
\begin{abstract}%
\PGFPlots\ draws high--quality function plots in normal or logarithmic scaling with a user-friendly interface directly in \TeX. The user supplies axis labels, legend entries and the plot coordinates for one or more plots and \PGFPlots\ applies axis scaling, computes any logarithms and axis ticks and draws the plots, supporting line plots, scatter plots, piecewise constant plots, bar plots, area plots, mesh-- and surface plots and some more. It is based on Till Tantau's package \PGF/\Tikz.
\end{abstract}
\tableofcontents
\section{Introduction}
This package provides tools to generate plots and labeled axes easily. It draws normal plots, logplots and semi-logplots, in two and three dimensions. Axis ticks, labels, legends (in case of multiple plots) can be added with key-value options. It can cycle through a set of predefined line/marker/color specifications. In summary, its purpose is to simplify the generation of high-quality function and/or data plots, and solving the problems of
\begin{itemize}
	\item consistency of document and font type and font size,
	\item direct use of \TeX\ math mode in axis descriptions,
	\item consistency of data and figures (no third party tool necessary),
	\item inter document consistency using preamble configurations and styles.
\end{itemize}
Although not necessary, separate |.pdf| or |.eps| graphics can be generated using the |external| library developed as part of \Tikz.

\PGFPlots\ is build completely on \Tikz/\PGF. Knowledge of \Tikz\ will simplify the work with \PGFPlots, although it is not required.

\section[About PGFPlots: Preliminaries]{About {\normalfont\PGFPlots}: Preliminaries}
This section contains information about upgrades, the team, the installation (in case you need to do it manually) and troubleshooting. You may skip it completely except for the upgrade remarks.

However, note that this library requires at least \PGF\ version $2.00$. At the time of this writing, many \TeX-distributions still contain the older \PGF\ version $1.18$, so it may be necessary to install a recent \PGF\ prior to using \PGFPlots.

\subsection{Components}
\PGFPlots\ comes with two components:
\begin{enumerate}
	\item the plotting component (which you are currently reading) and
	\item the \PGFPlotstable\ component which simplifies number formatting and postprocessing of numerical tables. It comes as a separate package and has its own manual \href{file:pgfplotstable.pdf}{pgfplotstable.pdf}.
\end{enumerate}

\subsection{Upgrade remarks}
This release provides a lot of improvements which can be found in all detail in \texttt{ChangeLog} for interested readers. However, some attention is useful with respect to the following changes.

\subsubsection{New Optional Features}
\begin{enumerate}
	\item \PGFPlots\ 1.3 comes with user interface improvements. The technical distinction between ``behavior options'' and ``style options'' of older versions is no longer necessary (although still fully supported).

	\item \PGFPlots\ 1.3 has a new feature which allows to \emph{move axis labels tight to tick labels} automatically.

	Since this affects the spacing, it is not enabled be default.

	Use
\begin{codeexample}[code only]
\usepackage{pgfplots}
\pgfplotsset{compat=1.3}
\end{codeexample}
	in your preamble to benefit from the improved spacing\footnote{The |compat=1.3| setting is necessary for new features which move axis labels around like reversed axis. However, \PGFPlots\ will still work as it does in earlier versions, your documents will remain the same if you don't set it explicitly.}.
	Take a look at the next page for the precise definition of |compat|.

	\item \PGFPlots\ 1.3 now supports reversed axes. It is no longer necessary to use work--arounds with negative units.
\pgfkeys{/pdflinks/search key prefixes in/.add={/pgfplots/,}{}}

	Take a look at the |x dir=reverse| key.

	Existing work--arounds will still function properly. Use |\pgfplotsset{compat=1.3}| together with |x dir=reverse| to switch to the new version.
\end{enumerate}

\subsubsection{Old Features Which May Need Attention}
\begin{enumerate}
	\item The |scatter/classes| feature produces proper legends as of version 1.3. This may change the appearance of existing legends of plots with |scatter/classes|.

	\item Due to a small math bug in \PGF\ $2.00$, you \emph{can't} use the math expression `|-x^2|'. It is necessary to use `|0-x^2|' instead. The same holds for
		`|exp(-x^2)|'; use `|exp(0-x^2)|' instead.

		This will be fixed with \PGF\ $>2.00$.

	\item Starting with \PGFPlots\ $1.1$, |\tikzstyle| should \emph{no longer be used} to set \PGFPlots\ options.
	
	Although |\tikzstyle| is still supported for some older \PGFPlots\ options, you should replace any occurance of |\tikzstyle| with |\pgfplotsset{|\meta{style name}|/.style={|\meta{key-value-list}|}}| or the associated |/.append style| variant. See section~\ref{sec:styles} for more detail.
\end{enumerate}
I apologize for any inconvenience caused by these changes.

\begin{pgfplotskey}{compat=\mchoice{1.3,pre 1.3,default,newest} (initially default)}
	The preamble configuration 
\begin{codeexample}[code only]
\usepackage{pgfplots}
\pgfplotsset{compat=1.3}
\end{codeexample}
	allows to choose between backwards compatibility and most recent features.

	\PGFPlots\ version 1.3 comes with new features which may lead to different spacings compared to earlier versions:
	\begin{enumerate}
		\item Version 1.3 comes with the |xlabel near ticks| feature which places (in this case $x$) axis labels ``near'' the tick labels, taking the size of tick labels into account.
		
		Older versions used |xlabel absolute|, i.e.\ an absolute distance between the axis and axis labels (now matter how large or small tick labels are).

		The initial setting \emph{keeps backwards compatibility}.

		You are encouraged to use
\begin{codeexample}[code only]
\usepackage{pgfplots}
\pgfplotsset{compat=1.3}
\end{codeexample}
		in your preamble.
		
		\item Version 1.3 fixes a bug with |anchor=south west| which occurred mainly for reversed axes: the anchors where upside--down. This is fixed now.

		
		If you have written some work--around and want to keep it going, use

		|\pgfplotsset{compat/anchors=pre 1.3}|\pgfmanualpdflabel{/pgfplots/compat/anchors}{} 

		to disable the bug fix.
	\end{enumerate}

	Use |\pgfplotsset{compat=default}| to restore the factory settings.

	The setting |\pgfplotsset{compat=newest}| will always use features of the most recent version. This might result in small changes in the document's appearance.
\end{pgfplotskey}

\subsection{The Team}
\PGFPlots\ has been written mainly by Christian Feuers�nger with many improvements of Pascal Wolkotte and Nick Papior Andersen as a spare time project. We hope it is useful and provides valuable plots.

If you are interested in writing something but don't know how, consider reading the auxiliary manual \href{file:TeX-programming-notes.pdf}{TeX-programming-notes.pdf} which comes with \PGFPlots. It is far from complete, but maybe it is a good starting point (at least for more literature).

\subsection{Acknowledgements}
I thank God for all hours of enjoyed programming. I thank Pascal Wolkotte and Nick Papior Andersen for their programming efforts and contributions as part of the development team. I thank Stefan Tibus, who contributed the |plot shell| feature. I thank Tom Cashman for the contribution of the |reverse legend| feature. Special thanks go to Stefan Pinnow whose tests of \PGFPlots\ lead to numerous quality improvements. Furthermore, I thank Dr.~Schweitzer for many fruitful discussions and Dr.~Meine for his ideas and suggestions. Thanks as well to the many international contributors who provided feature requests or identified bugs or simply manual improvements!

Last but not least, I thank Till Tantau and Mark Wibrow for their excellent graphics (and more) package \PGF\ and \Tikz\ which is the base of \PGFPlots.

\include{pgfplots.install}%
\include{pgfplots.intro}%
\include{pgfplots.reference}%
\include{pgfplots.libs}%
\include{pgfplots.resources}%
\include{pgfplots.importexport}%
\include{pgfplots.basic.reference}%

\printindex

\bibliographystyle{abbrv} %gerapali} %gerabbrv} %gerunsrt.bst} %gerabbrv}% gerplain}
\nocite{pgfplotstable}
\nocite{programmingnotes}
\bibliography{pgfplots}
\end{document}
%
%%%%%%%%%%%%%%%%%%%%%%%%%%%%%%%%%%%%%%%%%%%%%%%%%%%%%%%%%%%%%%%%%%%%%%%%%%%%%
%
% Package pgfplots.sty documentation. 
%
% Copyright 2007/2008 by Christian Feuersaenger.
%
% This program is free software: you can redistribute it and/or modify
% it under the terms of the GNU General Public License as published by
% the Free Software Foundation, either version 3 of the License, or
% (at your option) any later version.
% 
% This program is distributed in the hope that it will be useful,
% but WITHOUT ANY WARRANTY; without even the implied warranty of
% MERCHANTABILITY or FITNESS FOR A PARTICULAR PURPOSE.  See the
% GNU General Public License for more details.
% 
% You should have received a copy of the GNU General Public License
% along with this program.  If not, see <http://www.gnu.org/licenses/>.
%
%
%%%%%%%%%%%%%%%%%%%%%%%%%%%%%%%%%%%%%%%%%%%%%%%%%%%%%%%%%%%%%%%%%%%%%%%%%%%%%
\input{pgfplots.preamble.tex}
%\RequirePackage[german,english,francais]{babel}

\pgfplotsmanualexternalexpensivetrue

%\usetikzlibrary{external}
%\tikzexternalize[prefix=figures/]{pgfplots}

\title{%
	Manual for Package \PGFPlots\\
	{\small Version \pgfplotsversion}\\
	{\small\href{http://sourceforge.net/projects/pgfplots}{http://sourceforge.net/projects/pgfplots}}}

%\includeonly{pgfplots.libs}


\begin{document}

\def\plotcoords{%
\addplot coordinates {
(5,8.312e-02)    (17,2.547e-02)   (49,7.407e-03)
(129,2.102e-03)  (321,5.874e-04)  (769,1.623e-04)
(1793,4.442e-05) (4097,1.207e-05) (9217,3.261e-06)
};

\addplot coordinates{
(7,8.472e-02)    (31,3.044e-02)    (111,1.022e-02)
(351,3.303e-03)  (1023,1.039e-03)  (2815,3.196e-04)
(7423,9.658e-05) (18943,2.873e-05) (47103,8.437e-06)
};

\addplot coordinates{
(9,7.881e-02)     (49,3.243e-02)    (209,1.232e-02)
(769,4.454e-03)   (2561,1.551e-03)  (7937,5.236e-04)
(23297,1.723e-04) (65537,5.545e-05) (178177,1.751e-05)
};

\addplot coordinates{
(11,6.887e-02)    (71,3.177e-02)     (351,1.341e-02)
(1471,5.334e-03)  (5503,2.027e-03)   (18943,7.415e-04)
(61183,2.628e-04) (187903,9.063e-05) (553983,3.053e-05)
};

\addplot coordinates{
(13,5.755e-02)     (97,2.925e-02)     (545,1.351e-02)
(2561,5.842e-03)   (10625,2.397e-03)  (40193,9.414e-04)
(141569,3.564e-04) (471041,1.308e-04) 
(1496065,4.670e-05)
};
}%
\maketitle
\begin{abstract}%
\PGFPlots\ draws high--quality function plots in normal or logarithmic scaling with a user-friendly interface directly in \TeX. The user supplies axis labels, legend entries and the plot coordinates for one or more plots and \PGFPlots\ applies axis scaling, computes any logarithms and axis ticks and draws the plots, supporting line plots, scatter plots, piecewise constant plots, bar plots, area plots, mesh-- and surface plots and some more. It is based on Till Tantau's package \PGF/\Tikz.
\end{abstract}
\tableofcontents
\section{Introduction}
This package provides tools to generate plots and labeled axes easily. It draws normal plots, logplots and semi-logplots, in two and three dimensions. Axis ticks, labels, legends (in case of multiple plots) can be added with key-value options. It can cycle through a set of predefined line/marker/color specifications. In summary, its purpose is to simplify the generation of high-quality function and/or data plots, and solving the problems of
\begin{itemize}
	\item consistency of document and font type and font size,
	\item direct use of \TeX\ math mode in axis descriptions,
	\item consistency of data and figures (no third party tool necessary),
	\item inter document consistency using preamble configurations and styles.
\end{itemize}
Although not necessary, separate |.pdf| or |.eps| graphics can be generated using the |external| library developed as part of \Tikz.

\PGFPlots\ is build completely on \Tikz/\PGF. Knowledge of \Tikz\ will simplify the work with \PGFPlots, although it is not required.

\section[About PGFPlots: Preliminaries]{About {\normalfont\PGFPlots}: Preliminaries}
This section contains information about upgrades, the team, the installation (in case you need to do it manually) and troubleshooting. You may skip it completely except for the upgrade remarks.

However, note that this library requires at least \PGF\ version $2.00$. At the time of this writing, many \TeX-distributions still contain the older \PGF\ version $1.18$, so it may be necessary to install a recent \PGF\ prior to using \PGFPlots.

\subsection{Components}
\PGFPlots\ comes with two components:
\begin{enumerate}
	\item the plotting component (which you are currently reading) and
	\item the \PGFPlotstable\ component which simplifies number formatting and postprocessing of numerical tables. It comes as a separate package and has its own manual \href{file:pgfplotstable.pdf}{pgfplotstable.pdf}.
\end{enumerate}

\subsection{Upgrade remarks}
This release provides a lot of improvements which can be found in all detail in \texttt{ChangeLog} for interested readers. However, some attention is useful with respect to the following changes.

\subsubsection{New Optional Features}
\begin{enumerate}
	\item \PGFPlots\ 1.3 comes with user interface improvements. The technical distinction between ``behavior options'' and ``style options'' of older versions is no longer necessary (although still fully supported).

	\item \PGFPlots\ 1.3 has a new feature which allows to \emph{move axis labels tight to tick labels} automatically.

	Since this affects the spacing, it is not enabled be default.

	Use
\begin{codeexample}[code only]
\usepackage{pgfplots}
\pgfplotsset{compat=1.3}
\end{codeexample}
	in your preamble to benefit from the improved spacing\footnote{The |compat=1.3| setting is necessary for new features which move axis labels around like reversed axis. However, \PGFPlots\ will still work as it does in earlier versions, your documents will remain the same if you don't set it explicitly.}.
	Take a look at the next page for the precise definition of |compat|.

	\item \PGFPlots\ 1.3 now supports reversed axes. It is no longer necessary to use work--arounds with negative units.
\pgfkeys{/pdflinks/search key prefixes in/.add={/pgfplots/,}{}}

	Take a look at the |x dir=reverse| key.

	Existing work--arounds will still function properly. Use |\pgfplotsset{compat=1.3}| together with |x dir=reverse| to switch to the new version.
\end{enumerate}

\subsubsection{Old Features Which May Need Attention}
\begin{enumerate}
	\item The |scatter/classes| feature produces proper legends as of version 1.3. This may change the appearance of existing legends of plots with |scatter/classes|.

	\item Due to a small math bug in \PGF\ $2.00$, you \emph{can't} use the math expression `|-x^2|'. It is necessary to use `|0-x^2|' instead. The same holds for
		`|exp(-x^2)|'; use `|exp(0-x^2)|' instead.

		This will be fixed with \PGF\ $>2.00$.

	\item Starting with \PGFPlots\ $1.1$, |\tikzstyle| should \emph{no longer be used} to set \PGFPlots\ options.
	
	Although |\tikzstyle| is still supported for some older \PGFPlots\ options, you should replace any occurance of |\tikzstyle| with |\pgfplotsset{|\meta{style name}|/.style={|\meta{key-value-list}|}}| or the associated |/.append style| variant. See section~\ref{sec:styles} for more detail.
\end{enumerate}
I apologize for any inconvenience caused by these changes.

\begin{pgfplotskey}{compat=\mchoice{1.3,pre 1.3,default,newest} (initially default)}
	The preamble configuration 
\begin{codeexample}[code only]
\usepackage{pgfplots}
\pgfplotsset{compat=1.3}
\end{codeexample}
	allows to choose between backwards compatibility and most recent features.

	\PGFPlots\ version 1.3 comes with new features which may lead to different spacings compared to earlier versions:
	\begin{enumerate}
		\item Version 1.3 comes with the |xlabel near ticks| feature which places (in this case $x$) axis labels ``near'' the tick labels, taking the size of tick labels into account.
		
		Older versions used |xlabel absolute|, i.e.\ an absolute distance between the axis and axis labels (now matter how large or small tick labels are).

		The initial setting \emph{keeps backwards compatibility}.

		You are encouraged to use
\begin{codeexample}[code only]
\usepackage{pgfplots}
\pgfplotsset{compat=1.3}
\end{codeexample}
		in your preamble.
		
		\item Version 1.3 fixes a bug with |anchor=south west| which occurred mainly for reversed axes: the anchors where upside--down. This is fixed now.

		
		If you have written some work--around and want to keep it going, use

		|\pgfplotsset{compat/anchors=pre 1.3}|\pgfmanualpdflabel{/pgfplots/compat/anchors}{} 

		to disable the bug fix.
	\end{enumerate}

	Use |\pgfplotsset{compat=default}| to restore the factory settings.

	The setting |\pgfplotsset{compat=newest}| will always use features of the most recent version. This might result in small changes in the document's appearance.
\end{pgfplotskey}

\subsection{The Team}
\PGFPlots\ has been written mainly by Christian Feuers�nger with many improvements of Pascal Wolkotte and Nick Papior Andersen as a spare time project. We hope it is useful and provides valuable plots.

If you are interested in writing something but don't know how, consider reading the auxiliary manual \href{file:TeX-programming-notes.pdf}{TeX-programming-notes.pdf} which comes with \PGFPlots. It is far from complete, but maybe it is a good starting point (at least for more literature).

\subsection{Acknowledgements}
I thank God for all hours of enjoyed programming. I thank Pascal Wolkotte and Nick Papior Andersen for their programming efforts and contributions as part of the development team. I thank Stefan Tibus, who contributed the |plot shell| feature. I thank Tom Cashman for the contribution of the |reverse legend| feature. Special thanks go to Stefan Pinnow whose tests of \PGFPlots\ lead to numerous quality improvements. Furthermore, I thank Dr.~Schweitzer for many fruitful discussions and Dr.~Meine for his ideas and suggestions. Thanks as well to the many international contributors who provided feature requests or identified bugs or simply manual improvements!

Last but not least, I thank Till Tantau and Mark Wibrow for their excellent graphics (and more) package \PGF\ and \Tikz\ which is the base of \PGFPlots.

\include{pgfplots.install}%
\include{pgfplots.intro}%
\include{pgfplots.reference}%
\include{pgfplots.libs}%
\include{pgfplots.resources}%
\include{pgfplots.importexport}%
\include{pgfplots.basic.reference}%

\printindex

\bibliographystyle{abbrv} %gerapali} %gerabbrv} %gerunsrt.bst} %gerabbrv}% gerplain}
\nocite{pgfplotstable}
\nocite{programmingnotes}
\bibliography{pgfplots}
\end{document}
%
%%%%%%%%%%%%%%%%%%%%%%%%%%%%%%%%%%%%%%%%%%%%%%%%%%%%%%%%%%%%%%%%%%%%%%%%%%%%%
%
% Package pgfplots.sty documentation. 
%
% Copyright 2007/2008 by Christian Feuersaenger.
%
% This program is free software: you can redistribute it and/or modify
% it under the terms of the GNU General Public License as published by
% the Free Software Foundation, either version 3 of the License, or
% (at your option) any later version.
% 
% This program is distributed in the hope that it will be useful,
% but WITHOUT ANY WARRANTY; without even the implied warranty of
% MERCHANTABILITY or FITNESS FOR A PARTICULAR PURPOSE.  See the
% GNU General Public License for more details.
% 
% You should have received a copy of the GNU General Public License
% along with this program.  If not, see <http://www.gnu.org/licenses/>.
%
%
%%%%%%%%%%%%%%%%%%%%%%%%%%%%%%%%%%%%%%%%%%%%%%%%%%%%%%%%%%%%%%%%%%%%%%%%%%%%%
\input{pgfplots.preamble.tex}
%\RequirePackage[german,english,francais]{babel}

\pgfplotsmanualexternalexpensivetrue

%\usetikzlibrary{external}
%\tikzexternalize[prefix=figures/]{pgfplots}

\title{%
	Manual for Package \PGFPlots\\
	{\small Version \pgfplotsversion}\\
	{\small\href{http://sourceforge.net/projects/pgfplots}{http://sourceforge.net/projects/pgfplots}}}

%\includeonly{pgfplots.libs}


\begin{document}

\def\plotcoords{%
\addplot coordinates {
(5,8.312e-02)    (17,2.547e-02)   (49,7.407e-03)
(129,2.102e-03)  (321,5.874e-04)  (769,1.623e-04)
(1793,4.442e-05) (4097,1.207e-05) (9217,3.261e-06)
};

\addplot coordinates{
(7,8.472e-02)    (31,3.044e-02)    (111,1.022e-02)
(351,3.303e-03)  (1023,1.039e-03)  (2815,3.196e-04)
(7423,9.658e-05) (18943,2.873e-05) (47103,8.437e-06)
};

\addplot coordinates{
(9,7.881e-02)     (49,3.243e-02)    (209,1.232e-02)
(769,4.454e-03)   (2561,1.551e-03)  (7937,5.236e-04)
(23297,1.723e-04) (65537,5.545e-05) (178177,1.751e-05)
};

\addplot coordinates{
(11,6.887e-02)    (71,3.177e-02)     (351,1.341e-02)
(1471,5.334e-03)  (5503,2.027e-03)   (18943,7.415e-04)
(61183,2.628e-04) (187903,9.063e-05) (553983,3.053e-05)
};

\addplot coordinates{
(13,5.755e-02)     (97,2.925e-02)     (545,1.351e-02)
(2561,5.842e-03)   (10625,2.397e-03)  (40193,9.414e-04)
(141569,3.564e-04) (471041,1.308e-04) 
(1496065,4.670e-05)
};
}%
\maketitle
\begin{abstract}%
\PGFPlots\ draws high--quality function plots in normal or logarithmic scaling with a user-friendly interface directly in \TeX. The user supplies axis labels, legend entries and the plot coordinates for one or more plots and \PGFPlots\ applies axis scaling, computes any logarithms and axis ticks and draws the plots, supporting line plots, scatter plots, piecewise constant plots, bar plots, area plots, mesh-- and surface plots and some more. It is based on Till Tantau's package \PGF/\Tikz.
\end{abstract}
\tableofcontents
\section{Introduction}
This package provides tools to generate plots and labeled axes easily. It draws normal plots, logplots and semi-logplots, in two and three dimensions. Axis ticks, labels, legends (in case of multiple plots) can be added with key-value options. It can cycle through a set of predefined line/marker/color specifications. In summary, its purpose is to simplify the generation of high-quality function and/or data plots, and solving the problems of
\begin{itemize}
	\item consistency of document and font type and font size,
	\item direct use of \TeX\ math mode in axis descriptions,
	\item consistency of data and figures (no third party tool necessary),
	\item inter document consistency using preamble configurations and styles.
\end{itemize}
Although not necessary, separate |.pdf| or |.eps| graphics can be generated using the |external| library developed as part of \Tikz.

\PGFPlots\ is build completely on \Tikz/\PGF. Knowledge of \Tikz\ will simplify the work with \PGFPlots, although it is not required.

\section[About PGFPlots: Preliminaries]{About {\normalfont\PGFPlots}: Preliminaries}
This section contains information about upgrades, the team, the installation (in case you need to do it manually) and troubleshooting. You may skip it completely except for the upgrade remarks.

However, note that this library requires at least \PGF\ version $2.00$. At the time of this writing, many \TeX-distributions still contain the older \PGF\ version $1.18$, so it may be necessary to install a recent \PGF\ prior to using \PGFPlots.

\subsection{Components}
\PGFPlots\ comes with two components:
\begin{enumerate}
	\item the plotting component (which you are currently reading) and
	\item the \PGFPlotstable\ component which simplifies number formatting and postprocessing of numerical tables. It comes as a separate package and has its own manual \href{file:pgfplotstable.pdf}{pgfplotstable.pdf}.
\end{enumerate}

\subsection{Upgrade remarks}
This release provides a lot of improvements which can be found in all detail in \texttt{ChangeLog} for interested readers. However, some attention is useful with respect to the following changes.

\subsubsection{New Optional Features}
\begin{enumerate}
	\item \PGFPlots\ 1.3 comes with user interface improvements. The technical distinction between ``behavior options'' and ``style options'' of older versions is no longer necessary (although still fully supported).

	\item \PGFPlots\ 1.3 has a new feature which allows to \emph{move axis labels tight to tick labels} automatically.

	Since this affects the spacing, it is not enabled be default.

	Use
\begin{codeexample}[code only]
\usepackage{pgfplots}
\pgfplotsset{compat=1.3}
\end{codeexample}
	in your preamble to benefit from the improved spacing\footnote{The |compat=1.3| setting is necessary for new features which move axis labels around like reversed axis. However, \PGFPlots\ will still work as it does in earlier versions, your documents will remain the same if you don't set it explicitly.}.
	Take a look at the next page for the precise definition of |compat|.

	\item \PGFPlots\ 1.3 now supports reversed axes. It is no longer necessary to use work--arounds with negative units.
\pgfkeys{/pdflinks/search key prefixes in/.add={/pgfplots/,}{}}

	Take a look at the |x dir=reverse| key.

	Existing work--arounds will still function properly. Use |\pgfplotsset{compat=1.3}| together with |x dir=reverse| to switch to the new version.
\end{enumerate}

\subsubsection{Old Features Which May Need Attention}
\begin{enumerate}
	\item The |scatter/classes| feature produces proper legends as of version 1.3. This may change the appearance of existing legends of plots with |scatter/classes|.

	\item Due to a small math bug in \PGF\ $2.00$, you \emph{can't} use the math expression `|-x^2|'. It is necessary to use `|0-x^2|' instead. The same holds for
		`|exp(-x^2)|'; use `|exp(0-x^2)|' instead.

		This will be fixed with \PGF\ $>2.00$.

	\item Starting with \PGFPlots\ $1.1$, |\tikzstyle| should \emph{no longer be used} to set \PGFPlots\ options.
	
	Although |\tikzstyle| is still supported for some older \PGFPlots\ options, you should replace any occurance of |\tikzstyle| with |\pgfplotsset{|\meta{style name}|/.style={|\meta{key-value-list}|}}| or the associated |/.append style| variant. See section~\ref{sec:styles} for more detail.
\end{enumerate}
I apologize for any inconvenience caused by these changes.

\begin{pgfplotskey}{compat=\mchoice{1.3,pre 1.3,default,newest} (initially default)}
	The preamble configuration 
\begin{codeexample}[code only]
\usepackage{pgfplots}
\pgfplotsset{compat=1.3}
\end{codeexample}
	allows to choose between backwards compatibility and most recent features.

	\PGFPlots\ version 1.3 comes with new features which may lead to different spacings compared to earlier versions:
	\begin{enumerate}
		\item Version 1.3 comes with the |xlabel near ticks| feature which places (in this case $x$) axis labels ``near'' the tick labels, taking the size of tick labels into account.
		
		Older versions used |xlabel absolute|, i.e.\ an absolute distance between the axis and axis labels (now matter how large or small tick labels are).

		The initial setting \emph{keeps backwards compatibility}.

		You are encouraged to use
\begin{codeexample}[code only]
\usepackage{pgfplots}
\pgfplotsset{compat=1.3}
\end{codeexample}
		in your preamble.
		
		\item Version 1.3 fixes a bug with |anchor=south west| which occurred mainly for reversed axes: the anchors where upside--down. This is fixed now.

		
		If you have written some work--around and want to keep it going, use

		|\pgfplotsset{compat/anchors=pre 1.3}|\pgfmanualpdflabel{/pgfplots/compat/anchors}{} 

		to disable the bug fix.
	\end{enumerate}

	Use |\pgfplotsset{compat=default}| to restore the factory settings.

	The setting |\pgfplotsset{compat=newest}| will always use features of the most recent version. This might result in small changes in the document's appearance.
\end{pgfplotskey}

\subsection{The Team}
\PGFPlots\ has been written mainly by Christian Feuers�nger with many improvements of Pascal Wolkotte and Nick Papior Andersen as a spare time project. We hope it is useful and provides valuable plots.

If you are interested in writing something but don't know how, consider reading the auxiliary manual \href{file:TeX-programming-notes.pdf}{TeX-programming-notes.pdf} which comes with \PGFPlots. It is far from complete, but maybe it is a good starting point (at least for more literature).

\subsection{Acknowledgements}
I thank God for all hours of enjoyed programming. I thank Pascal Wolkotte and Nick Papior Andersen for their programming efforts and contributions as part of the development team. I thank Stefan Tibus, who contributed the |plot shell| feature. I thank Tom Cashman for the contribution of the |reverse legend| feature. Special thanks go to Stefan Pinnow whose tests of \PGFPlots\ lead to numerous quality improvements. Furthermore, I thank Dr.~Schweitzer for many fruitful discussions and Dr.~Meine for his ideas and suggestions. Thanks as well to the many international contributors who provided feature requests or identified bugs or simply manual improvements!

Last but not least, I thank Till Tantau and Mark Wibrow for their excellent graphics (and more) package \PGF\ and \Tikz\ which is the base of \PGFPlots.

\include{pgfplots.install}%
\include{pgfplots.intro}%
\include{pgfplots.reference}%
\include{pgfplots.libs}%
\include{pgfplots.resources}%
\include{pgfplots.importexport}%
\include{pgfplots.basic.reference}%

\printindex

\bibliographystyle{abbrv} %gerapali} %gerabbrv} %gerunsrt.bst} %gerabbrv}% gerplain}
\nocite{pgfplotstable}
\nocite{programmingnotes}
\bibliography{pgfplots}
\end{document}
%
%%%%%%%%%%%%%%%%%%%%%%%%%%%%%%%%%%%%%%%%%%%%%%%%%%%%%%%%%%%%%%%%%%%%%%%%%%%%%
%
% Package pgfplots.sty documentation. 
%
% Copyright 2007/2008 by Christian Feuersaenger.
%
% This program is free software: you can redistribute it and/or modify
% it under the terms of the GNU General Public License as published by
% the Free Software Foundation, either version 3 of the License, or
% (at your option) any later version.
% 
% This program is distributed in the hope that it will be useful,
% but WITHOUT ANY WARRANTY; without even the implied warranty of
% MERCHANTABILITY or FITNESS FOR A PARTICULAR PURPOSE.  See the
% GNU General Public License for more details.
% 
% You should have received a copy of the GNU General Public License
% along with this program.  If not, see <http://www.gnu.org/licenses/>.
%
%
%%%%%%%%%%%%%%%%%%%%%%%%%%%%%%%%%%%%%%%%%%%%%%%%%%%%%%%%%%%%%%%%%%%%%%%%%%%%%
\input{pgfplots.preamble.tex}
%\RequirePackage[german,english,francais]{babel}

\pgfplotsmanualexternalexpensivetrue

%\usetikzlibrary{external}
%\tikzexternalize[prefix=figures/]{pgfplots}

\title{%
	Manual for Package \PGFPlots\\
	{\small Version \pgfplotsversion}\\
	{\small\href{http://sourceforge.net/projects/pgfplots}{http://sourceforge.net/projects/pgfplots}}}

%\includeonly{pgfplots.libs}


\begin{document}

\def\plotcoords{%
\addplot coordinates {
(5,8.312e-02)    (17,2.547e-02)   (49,7.407e-03)
(129,2.102e-03)  (321,5.874e-04)  (769,1.623e-04)
(1793,4.442e-05) (4097,1.207e-05) (9217,3.261e-06)
};

\addplot coordinates{
(7,8.472e-02)    (31,3.044e-02)    (111,1.022e-02)
(351,3.303e-03)  (1023,1.039e-03)  (2815,3.196e-04)
(7423,9.658e-05) (18943,2.873e-05) (47103,8.437e-06)
};

\addplot coordinates{
(9,7.881e-02)     (49,3.243e-02)    (209,1.232e-02)
(769,4.454e-03)   (2561,1.551e-03)  (7937,5.236e-04)
(23297,1.723e-04) (65537,5.545e-05) (178177,1.751e-05)
};

\addplot coordinates{
(11,6.887e-02)    (71,3.177e-02)     (351,1.341e-02)
(1471,5.334e-03)  (5503,2.027e-03)   (18943,7.415e-04)
(61183,2.628e-04) (187903,9.063e-05) (553983,3.053e-05)
};

\addplot coordinates{
(13,5.755e-02)     (97,2.925e-02)     (545,1.351e-02)
(2561,5.842e-03)   (10625,2.397e-03)  (40193,9.414e-04)
(141569,3.564e-04) (471041,1.308e-04) 
(1496065,4.670e-05)
};
}%
\maketitle
\begin{abstract}%
\PGFPlots\ draws high--quality function plots in normal or logarithmic scaling with a user-friendly interface directly in \TeX. The user supplies axis labels, legend entries and the plot coordinates for one or more plots and \PGFPlots\ applies axis scaling, computes any logarithms and axis ticks and draws the plots, supporting line plots, scatter plots, piecewise constant plots, bar plots, area plots, mesh-- and surface plots and some more. It is based on Till Tantau's package \PGF/\Tikz.
\end{abstract}
\tableofcontents
\section{Introduction}
This package provides tools to generate plots and labeled axes easily. It draws normal plots, logplots and semi-logplots, in two and three dimensions. Axis ticks, labels, legends (in case of multiple plots) can be added with key-value options. It can cycle through a set of predefined line/marker/color specifications. In summary, its purpose is to simplify the generation of high-quality function and/or data plots, and solving the problems of
\begin{itemize}
	\item consistency of document and font type and font size,
	\item direct use of \TeX\ math mode in axis descriptions,
	\item consistency of data and figures (no third party tool necessary),
	\item inter document consistency using preamble configurations and styles.
\end{itemize}
Although not necessary, separate |.pdf| or |.eps| graphics can be generated using the |external| library developed as part of \Tikz.

\PGFPlots\ is build completely on \Tikz/\PGF. Knowledge of \Tikz\ will simplify the work with \PGFPlots, although it is not required.

\section[About PGFPlots: Preliminaries]{About {\normalfont\PGFPlots}: Preliminaries}
This section contains information about upgrades, the team, the installation (in case you need to do it manually) and troubleshooting. You may skip it completely except for the upgrade remarks.

However, note that this library requires at least \PGF\ version $2.00$. At the time of this writing, many \TeX-distributions still contain the older \PGF\ version $1.18$, so it may be necessary to install a recent \PGF\ prior to using \PGFPlots.

\subsection{Components}
\PGFPlots\ comes with two components:
\begin{enumerate}
	\item the plotting component (which you are currently reading) and
	\item the \PGFPlotstable\ component which simplifies number formatting and postprocessing of numerical tables. It comes as a separate package and has its own manual \href{file:pgfplotstable.pdf}{pgfplotstable.pdf}.
\end{enumerate}

\subsection{Upgrade remarks}
This release provides a lot of improvements which can be found in all detail in \texttt{ChangeLog} for interested readers. However, some attention is useful with respect to the following changes.

\subsubsection{New Optional Features}
\begin{enumerate}
	\item \PGFPlots\ 1.3 comes with user interface improvements. The technical distinction between ``behavior options'' and ``style options'' of older versions is no longer necessary (although still fully supported).

	\item \PGFPlots\ 1.3 has a new feature which allows to \emph{move axis labels tight to tick labels} automatically.

	Since this affects the spacing, it is not enabled be default.

	Use
\begin{codeexample}[code only]
\usepackage{pgfplots}
\pgfplotsset{compat=1.3}
\end{codeexample}
	in your preamble to benefit from the improved spacing\footnote{The |compat=1.3| setting is necessary for new features which move axis labels around like reversed axis. However, \PGFPlots\ will still work as it does in earlier versions, your documents will remain the same if you don't set it explicitly.}.
	Take a look at the next page for the precise definition of |compat|.

	\item \PGFPlots\ 1.3 now supports reversed axes. It is no longer necessary to use work--arounds with negative units.
\pgfkeys{/pdflinks/search key prefixes in/.add={/pgfplots/,}{}}

	Take a look at the |x dir=reverse| key.

	Existing work--arounds will still function properly. Use |\pgfplotsset{compat=1.3}| together with |x dir=reverse| to switch to the new version.
\end{enumerate}

\subsubsection{Old Features Which May Need Attention}
\begin{enumerate}
	\item The |scatter/classes| feature produces proper legends as of version 1.3. This may change the appearance of existing legends of plots with |scatter/classes|.

	\item Due to a small math bug in \PGF\ $2.00$, you \emph{can't} use the math expression `|-x^2|'. It is necessary to use `|0-x^2|' instead. The same holds for
		`|exp(-x^2)|'; use `|exp(0-x^2)|' instead.

		This will be fixed with \PGF\ $>2.00$.

	\item Starting with \PGFPlots\ $1.1$, |\tikzstyle| should \emph{no longer be used} to set \PGFPlots\ options.
	
	Although |\tikzstyle| is still supported for some older \PGFPlots\ options, you should replace any occurance of |\tikzstyle| with |\pgfplotsset{|\meta{style name}|/.style={|\meta{key-value-list}|}}| or the associated |/.append style| variant. See section~\ref{sec:styles} for more detail.
\end{enumerate}
I apologize for any inconvenience caused by these changes.

\begin{pgfplotskey}{compat=\mchoice{1.3,pre 1.3,default,newest} (initially default)}
	The preamble configuration 
\begin{codeexample}[code only]
\usepackage{pgfplots}
\pgfplotsset{compat=1.3}
\end{codeexample}
	allows to choose between backwards compatibility and most recent features.

	\PGFPlots\ version 1.3 comes with new features which may lead to different spacings compared to earlier versions:
	\begin{enumerate}
		\item Version 1.3 comes with the |xlabel near ticks| feature which places (in this case $x$) axis labels ``near'' the tick labels, taking the size of tick labels into account.
		
		Older versions used |xlabel absolute|, i.e.\ an absolute distance between the axis and axis labels (now matter how large or small tick labels are).

		The initial setting \emph{keeps backwards compatibility}.

		You are encouraged to use
\begin{codeexample}[code only]
\usepackage{pgfplots}
\pgfplotsset{compat=1.3}
\end{codeexample}
		in your preamble.
		
		\item Version 1.3 fixes a bug with |anchor=south west| which occurred mainly for reversed axes: the anchors where upside--down. This is fixed now.

		
		If you have written some work--around and want to keep it going, use

		|\pgfplotsset{compat/anchors=pre 1.3}|\pgfmanualpdflabel{/pgfplots/compat/anchors}{} 

		to disable the bug fix.
	\end{enumerate}

	Use |\pgfplotsset{compat=default}| to restore the factory settings.

	The setting |\pgfplotsset{compat=newest}| will always use features of the most recent version. This might result in small changes in the document's appearance.
\end{pgfplotskey}

\subsection{The Team}
\PGFPlots\ has been written mainly by Christian Feuers�nger with many improvements of Pascal Wolkotte and Nick Papior Andersen as a spare time project. We hope it is useful and provides valuable plots.

If you are interested in writing something but don't know how, consider reading the auxiliary manual \href{file:TeX-programming-notes.pdf}{TeX-programming-notes.pdf} which comes with \PGFPlots. It is far from complete, but maybe it is a good starting point (at least for more literature).

\subsection{Acknowledgements}
I thank God for all hours of enjoyed programming. I thank Pascal Wolkotte and Nick Papior Andersen for their programming efforts and contributions as part of the development team. I thank Stefan Tibus, who contributed the |plot shell| feature. I thank Tom Cashman for the contribution of the |reverse legend| feature. Special thanks go to Stefan Pinnow whose tests of \PGFPlots\ lead to numerous quality improvements. Furthermore, I thank Dr.~Schweitzer for many fruitful discussions and Dr.~Meine for his ideas and suggestions. Thanks as well to the many international contributors who provided feature requests or identified bugs or simply manual improvements!

Last but not least, I thank Till Tantau and Mark Wibrow for their excellent graphics (and more) package \PGF\ and \Tikz\ which is the base of \PGFPlots.

\include{pgfplots.install}%
\include{pgfplots.intro}%
\include{pgfplots.reference}%
\include{pgfplots.libs}%
\include{pgfplots.resources}%
\include{pgfplots.importexport}%
\include{pgfplots.basic.reference}%

\printindex

\bibliographystyle{abbrv} %gerapali} %gerabbrv} %gerunsrt.bst} %gerabbrv}% gerplain}
\nocite{pgfplotstable}
\nocite{programmingnotes}
\bibliography{pgfplots}
\end{document}
%
%%%%%%%%%%%%%%%%%%%%%%%%%%%%%%%%%%%%%%%%%%%%%%%%%%%%%%%%%%%%%%%%%%%%%%%%%%%%%
%
% Package pgfplots.sty documentation. 
%
% Copyright 2007/2008 by Christian Feuersaenger.
%
% This program is free software: you can redistribute it and/or modify
% it under the terms of the GNU General Public License as published by
% the Free Software Foundation, either version 3 of the License, or
% (at your option) any later version.
% 
% This program is distributed in the hope that it will be useful,
% but WITHOUT ANY WARRANTY; without even the implied warranty of
% MERCHANTABILITY or FITNESS FOR A PARTICULAR PURPOSE.  See the
% GNU General Public License for more details.
% 
% You should have received a copy of the GNU General Public License
% along with this program.  If not, see <http://www.gnu.org/licenses/>.
%
%
%%%%%%%%%%%%%%%%%%%%%%%%%%%%%%%%%%%%%%%%%%%%%%%%%%%%%%%%%%%%%%%%%%%%%%%%%%%%%
\input{pgfplots.preamble.tex}
%\RequirePackage[german,english,francais]{babel}

\pgfplotsmanualexternalexpensivetrue

%\usetikzlibrary{external}
%\tikzexternalize[prefix=figures/]{pgfplots}

\title{%
	Manual for Package \PGFPlots\\
	{\small Version \pgfplotsversion}\\
	{\small\href{http://sourceforge.net/projects/pgfplots}{http://sourceforge.net/projects/pgfplots}}}

%\includeonly{pgfplots.libs}


\begin{document}

\def\plotcoords{%
\addplot coordinates {
(5,8.312e-02)    (17,2.547e-02)   (49,7.407e-03)
(129,2.102e-03)  (321,5.874e-04)  (769,1.623e-04)
(1793,4.442e-05) (4097,1.207e-05) (9217,3.261e-06)
};

\addplot coordinates{
(7,8.472e-02)    (31,3.044e-02)    (111,1.022e-02)
(351,3.303e-03)  (1023,1.039e-03)  (2815,3.196e-04)
(7423,9.658e-05) (18943,2.873e-05) (47103,8.437e-06)
};

\addplot coordinates{
(9,7.881e-02)     (49,3.243e-02)    (209,1.232e-02)
(769,4.454e-03)   (2561,1.551e-03)  (7937,5.236e-04)
(23297,1.723e-04) (65537,5.545e-05) (178177,1.751e-05)
};

\addplot coordinates{
(11,6.887e-02)    (71,3.177e-02)     (351,1.341e-02)
(1471,5.334e-03)  (5503,2.027e-03)   (18943,7.415e-04)
(61183,2.628e-04) (187903,9.063e-05) (553983,3.053e-05)
};

\addplot coordinates{
(13,5.755e-02)     (97,2.925e-02)     (545,1.351e-02)
(2561,5.842e-03)   (10625,2.397e-03)  (40193,9.414e-04)
(141569,3.564e-04) (471041,1.308e-04) 
(1496065,4.670e-05)
};
}%
\maketitle
\begin{abstract}%
\PGFPlots\ draws high--quality function plots in normal or logarithmic scaling with a user-friendly interface directly in \TeX. The user supplies axis labels, legend entries and the plot coordinates for one or more plots and \PGFPlots\ applies axis scaling, computes any logarithms and axis ticks and draws the plots, supporting line plots, scatter plots, piecewise constant plots, bar plots, area plots, mesh-- and surface plots and some more. It is based on Till Tantau's package \PGF/\Tikz.
\end{abstract}
\tableofcontents
\section{Introduction}
This package provides tools to generate plots and labeled axes easily. It draws normal plots, logplots and semi-logplots, in two and three dimensions. Axis ticks, labels, legends (in case of multiple plots) can be added with key-value options. It can cycle through a set of predefined line/marker/color specifications. In summary, its purpose is to simplify the generation of high-quality function and/or data plots, and solving the problems of
\begin{itemize}
	\item consistency of document and font type and font size,
	\item direct use of \TeX\ math mode in axis descriptions,
	\item consistency of data and figures (no third party tool necessary),
	\item inter document consistency using preamble configurations and styles.
\end{itemize}
Although not necessary, separate |.pdf| or |.eps| graphics can be generated using the |external| library developed as part of \Tikz.

\PGFPlots\ is build completely on \Tikz/\PGF. Knowledge of \Tikz\ will simplify the work with \PGFPlots, although it is not required.

\section[About PGFPlots: Preliminaries]{About {\normalfont\PGFPlots}: Preliminaries}
This section contains information about upgrades, the team, the installation (in case you need to do it manually) and troubleshooting. You may skip it completely except for the upgrade remarks.

However, note that this library requires at least \PGF\ version $2.00$. At the time of this writing, many \TeX-distributions still contain the older \PGF\ version $1.18$, so it may be necessary to install a recent \PGF\ prior to using \PGFPlots.

\subsection{Components}
\PGFPlots\ comes with two components:
\begin{enumerate}
	\item the plotting component (which you are currently reading) and
	\item the \PGFPlotstable\ component which simplifies number formatting and postprocessing of numerical tables. It comes as a separate package and has its own manual \href{file:pgfplotstable.pdf}{pgfplotstable.pdf}.
\end{enumerate}

\subsection{Upgrade remarks}
This release provides a lot of improvements which can be found in all detail in \texttt{ChangeLog} for interested readers. However, some attention is useful with respect to the following changes.

\subsubsection{New Optional Features}
\begin{enumerate}
	\item \PGFPlots\ 1.3 comes with user interface improvements. The technical distinction between ``behavior options'' and ``style options'' of older versions is no longer necessary (although still fully supported).

	\item \PGFPlots\ 1.3 has a new feature which allows to \emph{move axis labels tight to tick labels} automatically.

	Since this affects the spacing, it is not enabled be default.

	Use
\begin{codeexample}[code only]
\usepackage{pgfplots}
\pgfplotsset{compat=1.3}
\end{codeexample}
	in your preamble to benefit from the improved spacing\footnote{The |compat=1.3| setting is necessary for new features which move axis labels around like reversed axis. However, \PGFPlots\ will still work as it does in earlier versions, your documents will remain the same if you don't set it explicitly.}.
	Take a look at the next page for the precise definition of |compat|.

	\item \PGFPlots\ 1.3 now supports reversed axes. It is no longer necessary to use work--arounds with negative units.
\pgfkeys{/pdflinks/search key prefixes in/.add={/pgfplots/,}{}}

	Take a look at the |x dir=reverse| key.

	Existing work--arounds will still function properly. Use |\pgfplotsset{compat=1.3}| together with |x dir=reverse| to switch to the new version.
\end{enumerate}

\subsubsection{Old Features Which May Need Attention}
\begin{enumerate}
	\item The |scatter/classes| feature produces proper legends as of version 1.3. This may change the appearance of existing legends of plots with |scatter/classes|.

	\item Due to a small math bug in \PGF\ $2.00$, you \emph{can't} use the math expression `|-x^2|'. It is necessary to use `|0-x^2|' instead. The same holds for
		`|exp(-x^2)|'; use `|exp(0-x^2)|' instead.

		This will be fixed with \PGF\ $>2.00$.

	\item Starting with \PGFPlots\ $1.1$, |\tikzstyle| should \emph{no longer be used} to set \PGFPlots\ options.
	
	Although |\tikzstyle| is still supported for some older \PGFPlots\ options, you should replace any occurance of |\tikzstyle| with |\pgfplotsset{|\meta{style name}|/.style={|\meta{key-value-list}|}}| or the associated |/.append style| variant. See section~\ref{sec:styles} for more detail.
\end{enumerate}
I apologize for any inconvenience caused by these changes.

\begin{pgfplotskey}{compat=\mchoice{1.3,pre 1.3,default,newest} (initially default)}
	The preamble configuration 
\begin{codeexample}[code only]
\usepackage{pgfplots}
\pgfplotsset{compat=1.3}
\end{codeexample}
	allows to choose between backwards compatibility and most recent features.

	\PGFPlots\ version 1.3 comes with new features which may lead to different spacings compared to earlier versions:
	\begin{enumerate}
		\item Version 1.3 comes with the |xlabel near ticks| feature which places (in this case $x$) axis labels ``near'' the tick labels, taking the size of tick labels into account.
		
		Older versions used |xlabel absolute|, i.e.\ an absolute distance between the axis and axis labels (now matter how large or small tick labels are).

		The initial setting \emph{keeps backwards compatibility}.

		You are encouraged to use
\begin{codeexample}[code only]
\usepackage{pgfplots}
\pgfplotsset{compat=1.3}
\end{codeexample}
		in your preamble.
		
		\item Version 1.3 fixes a bug with |anchor=south west| which occurred mainly for reversed axes: the anchors where upside--down. This is fixed now.

		
		If you have written some work--around and want to keep it going, use

		|\pgfplotsset{compat/anchors=pre 1.3}|\pgfmanualpdflabel{/pgfplots/compat/anchors}{} 

		to disable the bug fix.
	\end{enumerate}

	Use |\pgfplotsset{compat=default}| to restore the factory settings.

	The setting |\pgfplotsset{compat=newest}| will always use features of the most recent version. This might result in small changes in the document's appearance.
\end{pgfplotskey}

\subsection{The Team}
\PGFPlots\ has been written mainly by Christian Feuers�nger with many improvements of Pascal Wolkotte and Nick Papior Andersen as a spare time project. We hope it is useful and provides valuable plots.

If you are interested in writing something but don't know how, consider reading the auxiliary manual \href{file:TeX-programming-notes.pdf}{TeX-programming-notes.pdf} which comes with \PGFPlots. It is far from complete, but maybe it is a good starting point (at least for more literature).

\subsection{Acknowledgements}
I thank God for all hours of enjoyed programming. I thank Pascal Wolkotte and Nick Papior Andersen for their programming efforts and contributions as part of the development team. I thank Stefan Tibus, who contributed the |plot shell| feature. I thank Tom Cashman for the contribution of the |reverse legend| feature. Special thanks go to Stefan Pinnow whose tests of \PGFPlots\ lead to numerous quality improvements. Furthermore, I thank Dr.~Schweitzer for many fruitful discussions and Dr.~Meine for his ideas and suggestions. Thanks as well to the many international contributors who provided feature requests or identified bugs or simply manual improvements!

Last but not least, I thank Till Tantau and Mark Wibrow for their excellent graphics (and more) package \PGF\ and \Tikz\ which is the base of \PGFPlots.

\include{pgfplots.install}%
\include{pgfplots.intro}%
\include{pgfplots.reference}%
\include{pgfplots.libs}%
\include{pgfplots.resources}%
\include{pgfplots.importexport}%
\include{pgfplots.basic.reference}%

\printindex

\bibliographystyle{abbrv} %gerapali} %gerabbrv} %gerunsrt.bst} %gerabbrv}% gerplain}
\nocite{pgfplotstable}
\nocite{programmingnotes}
\bibliography{pgfplots}
\end{document}
%
\include{pgfplots.basic.reference}%

\printindex

\bibliographystyle{abbrv} %gerapali} %gerabbrv} %gerunsrt.bst} %gerabbrv}% gerplain}
\nocite{pgfplotstable}
\nocite{programmingnotes}
\bibliography{pgfplots}
\end{document}
%
%%%%%%%%%%%%%%%%%%%%%%%%%%%%%%%%%%%%%%%%%%%%%%%%%%%%%%%%%%%%%%%%%%%%%%%%%%%%%
%
% Package pgfplots.sty documentation. 
%
% Copyright 2007/2008 by Christian Feuersaenger.
%
% This program is free software: you can redistribute it and/or modify
% it under the terms of the GNU General Public License as published by
% the Free Software Foundation, either version 3 of the License, or
% (at your option) any later version.
% 
% This program is distributed in the hope that it will be useful,
% but WITHOUT ANY WARRANTY; without even the implied warranty of
% MERCHANTABILITY or FITNESS FOR A PARTICULAR PURPOSE.  See the
% GNU General Public License for more details.
% 
% You should have received a copy of the GNU General Public License
% along with this program.  If not, see <http://www.gnu.org/licenses/>.
%
%
%%%%%%%%%%%%%%%%%%%%%%%%%%%%%%%%%%%%%%%%%%%%%%%%%%%%%%%%%%%%%%%%%%%%%%%%%%%%%
%%%%%%%%%%%%%%%%%%%%%%%%%%%%%%%%%%%%%%%%%%%%%%%%%%%%%%%%%%%%%%%%%%%%%%%%%%%%%
%
% Package pgfplots.sty documentation. 
%
% Copyright 2007/2008 by Christian Feuersaenger.
%
% This program is free software: you can redistribute it and/or modify
% it under the terms of the GNU General Public License as published by
% the Free Software Foundation, either version 3 of the License, or
% (at your option) any later version.
% 
% This program is distributed in the hope that it will be useful,
% but WITHOUT ANY WARRANTY; without even the implied warranty of
% MERCHANTABILITY or FITNESS FOR A PARTICULAR PURPOSE.  See the
% GNU General Public License for more details.
% 
% You should have received a copy of the GNU General Public License
% along with this program.  If not, see <http://www.gnu.org/licenses/>.
%
%
%%%%%%%%%%%%%%%%%%%%%%%%%%%%%%%%%%%%%%%%%%%%%%%%%%%%%%%%%%%%%%%%%%%%%%%%%%%%%
\documentclass[a4paper]{ltxdoc}

\usepackage{makeidx}

% DON't let hyperref overload the format of index and glossary. 
% I want to do that on my own in the stylefiles for makeindex...
\makeatletter
\let\@old@wrindex=\@wrindex
\makeatother

\usepackage{ifpdf}
\ifpdf
	\usepackage[pdftex]{hyperref}
\else
	\usepackage[dvipdfm]{hyperref}
\fi
\hypersetup{pdfborder={0 0 0}}

\makeatletter
\let\@wrindex=\@old@wrindex
\makeatother

% Formatiere Seitennummern im Index:
\newcommand{\indexpageno}[1]{%
	{\bfseries\hyperpage{#1}}%
}


\newcommand{\R}{\mathbb{R}}
\newcommand{\N}{\mathbb{N}}
\newcommand{\Z}{\mathbb{Z}}

\long\def\COMMENTLOWLEVEL#1\ENDCOMMENT{}
\def\ENDCOMMENT{}

\usepackage{textcomp}
\usepackage{booktabs}

\usepackage{calc}
\usepackage[formats]{listings}
%\usepackage{courier} % don't use it - the '^' character can't be copy-pasted in courier

\usepackage{array}
\lstset{%
	basicstyle=\ttfamily,
	language=[LaTeX]tex, % Seems as if \lstset{language=tex} must be invoked BEFORE loading tikz!?
	tabsize=4,
	breaklines=true,
	breakindent=0pt
}

\ifpdf
	\pdfinfo {
		/Author	(Christian Feuersaenger)
	}

\else
	\def\pgfsysdriver{pgfsys-dvipdfm.def}
\fi
%\def\pgfsysdriver{pgfsys-pdftex.def}
\usepackage{pgfplots}

\ifpdf
	\usetikzlibrary{pgfplotsclickable}
\fi

%\usepackage{fp}
% ATTENTION:
% this requires pgf version NEWER than 2.00 :
%\usetikzlibrary{fixedpointarithmetic}

\usetikzlibrary{dateplot}

\usepackage[a4paper,left=2.25cm,right=2.25cm,top=2.5cm,bottom=2.5cm,nohead]{geometry}
\usepackage{amsmath,amssymb}
\usepackage{xxcolor}
\usepackage{pifont}
\usepackage[latin1]{inputenc}
\usepackage{amsmath}
\usepackage{eurosym}
\input{pgfplots-macros}

\pgfqkeys{/codeexample}{%
	every codeexample/.style={
		width=8cm,
		/pgfplots/every axis/.append style={legend style={fill=graphicbackground}}
	},
	tabsize=4,
}
\pgfplotsset{
	every axis/.append style={width=7cm},
	filter discard warning=false,
}

\usetikzlibrary{backgrounds,patterns}
% Global styles:
\tikzset{
  shape example/.style={
    color=black!30,
    draw,
    fill=yellow!30,
    line width=.5cm,
    inner xsep=2.5cm,
    inner ysep=0.5cm}
}

\newcommand{\FIXME}[1]{\textcolor{red}{(FIXME: #1)}}

% fuer endvironment 'sidewaysfigure' bspw
% \usepackage{rotating}

\newcommand\Tikz{Ti\textit kZ}
\newcommand\PGF{\textsc{pgf}}
\newcommand\PGFPlots{\textsc{pgfplots}}
\def\PGFPlotstable{\textsc{PgfplotsTable}}


\makeindex

% Fix overful hboxes automatically:
\tolerance=2000
\emergencystretch=10pt

\tikzset{prefix=gnuplot/pgfplots_} % prefix for 'plot function'

\author{%
	Christian Feuers\"anger\footnote{\url{http://wissrech.ins.uni-bonn.de/people/feuersaenger}}\\%
	Institut f\"ur Numerische Simulation\\
	Universit\"at Bonn, Germany}


%\RequirePackage[german,english,francais]{babel}

\pgfplotsmanualexternalexpensivetrue

%\usetikzlibrary{external}
%\tikzexternalize[prefix=figures/]{pgfplots}

\title{%
	Manual for Package \PGFPlots\\
	{\small Version \pgfplotsversion}\\
	{\small\href{http://sourceforge.net/projects/pgfplots}{http://sourceforge.net/projects/pgfplots}}}

%\includeonly{pgfplots.libs}


\begin{document}

\def\plotcoords{%
\addplot coordinates {
(5,8.312e-02)    (17,2.547e-02)   (49,7.407e-03)
(129,2.102e-03)  (321,5.874e-04)  (769,1.623e-04)
(1793,4.442e-05) (4097,1.207e-05) (9217,3.261e-06)
};

\addplot coordinates{
(7,8.472e-02)    (31,3.044e-02)    (111,1.022e-02)
(351,3.303e-03)  (1023,1.039e-03)  (2815,3.196e-04)
(7423,9.658e-05) (18943,2.873e-05) (47103,8.437e-06)
};

\addplot coordinates{
(9,7.881e-02)     (49,3.243e-02)    (209,1.232e-02)
(769,4.454e-03)   (2561,1.551e-03)  (7937,5.236e-04)
(23297,1.723e-04) (65537,5.545e-05) (178177,1.751e-05)
};

\addplot coordinates{
(11,6.887e-02)    (71,3.177e-02)     (351,1.341e-02)
(1471,5.334e-03)  (5503,2.027e-03)   (18943,7.415e-04)
(61183,2.628e-04) (187903,9.063e-05) (553983,3.053e-05)
};

\addplot coordinates{
(13,5.755e-02)     (97,2.925e-02)     (545,1.351e-02)
(2561,5.842e-03)   (10625,2.397e-03)  (40193,9.414e-04)
(141569,3.564e-04) (471041,1.308e-04) 
(1496065,4.670e-05)
};
}%
\maketitle
\begin{abstract}%
\PGFPlots\ draws high--quality function plots in normal or logarithmic scaling with a user-friendly interface directly in \TeX. The user supplies axis labels, legend entries and the plot coordinates for one or more plots and \PGFPlots\ applies axis scaling, computes any logarithms and axis ticks and draws the plots, supporting line plots, scatter plots, piecewise constant plots, bar plots, area plots, mesh-- and surface plots and some more. It is based on Till Tantau's package \PGF/\Tikz.
\end{abstract}
\tableofcontents
\section{Introduction}
This package provides tools to generate plots and labeled axes easily. It draws normal plots, logplots and semi-logplots, in two and three dimensions. Axis ticks, labels, legends (in case of multiple plots) can be added with key-value options. It can cycle through a set of predefined line/marker/color specifications. In summary, its purpose is to simplify the generation of high-quality function and/or data plots, and solving the problems of
\begin{itemize}
	\item consistency of document and font type and font size,
	\item direct use of \TeX\ math mode in axis descriptions,
	\item consistency of data and figures (no third party tool necessary),
	\item inter document consistency using preamble configurations and styles.
\end{itemize}
Although not necessary, separate |.pdf| or |.eps| graphics can be generated using the |external| library developed as part of \Tikz.

\PGFPlots\ is build completely on \Tikz/\PGF. Knowledge of \Tikz\ will simplify the work with \PGFPlots, although it is not required.

\section[About PGFPlots: Preliminaries]{About {\normalfont\PGFPlots}: Preliminaries}
This section contains information about upgrades, the team, the installation (in case you need to do it manually) and troubleshooting. You may skip it completely except for the upgrade remarks.

However, note that this library requires at least \PGF\ version $2.00$. At the time of this writing, many \TeX-distributions still contain the older \PGF\ version $1.18$, so it may be necessary to install a recent \PGF\ prior to using \PGFPlots.

\subsection{Components}
\PGFPlots\ comes with two components:
\begin{enumerate}
	\item the plotting component (which you are currently reading) and
	\item the \PGFPlotstable\ component which simplifies number formatting and postprocessing of numerical tables. It comes as a separate package and has its own manual \href{file:pgfplotstable.pdf}{pgfplotstable.pdf}.
\end{enumerate}

\subsection{Upgrade remarks}
This release provides a lot of improvements which can be found in all detail in \texttt{ChangeLog} for interested readers. However, some attention is useful with respect to the following changes.

\subsubsection{New Optional Features}
\begin{enumerate}
	\item \PGFPlots\ 1.3 comes with user interface improvements. The technical distinction between ``behavior options'' and ``style options'' of older versions is no longer necessary (although still fully supported).

	\item \PGFPlots\ 1.3 has a new feature which allows to \emph{move axis labels tight to tick labels} automatically.

	Since this affects the spacing, it is not enabled be default.

	Use
\begin{codeexample}[code only]
\usepackage{pgfplots}
\pgfplotsset{compat=1.3}
\end{codeexample}
	in your preamble to benefit from the improved spacing\footnote{The |compat=1.3| setting is necessary for new features which move axis labels around like reversed axis. However, \PGFPlots\ will still work as it does in earlier versions, your documents will remain the same if you don't set it explicitly.}.
	Take a look at the next page for the precise definition of |compat|.

	\item \PGFPlots\ 1.3 now supports reversed axes. It is no longer necessary to use work--arounds with negative units.
\pgfkeys{/pdflinks/search key prefixes in/.add={/pgfplots/,}{}}

	Take a look at the |x dir=reverse| key.

	Existing work--arounds will still function properly. Use |\pgfplotsset{compat=1.3}| together with |x dir=reverse| to switch to the new version.
\end{enumerate}

\subsubsection{Old Features Which May Need Attention}
\begin{enumerate}
	\item The |scatter/classes| feature produces proper legends as of version 1.3. This may change the appearance of existing legends of plots with |scatter/classes|.

	\item Due to a small math bug in \PGF\ $2.00$, you \emph{can't} use the math expression `|-x^2|'. It is necessary to use `|0-x^2|' instead. The same holds for
		`|exp(-x^2)|'; use `|exp(0-x^2)|' instead.

		This will be fixed with \PGF\ $>2.00$.

	\item Starting with \PGFPlots\ $1.1$, |\tikzstyle| should \emph{no longer be used} to set \PGFPlots\ options.
	
	Although |\tikzstyle| is still supported for some older \PGFPlots\ options, you should replace any occurance of |\tikzstyle| with |\pgfplotsset{|\meta{style name}|/.style={|\meta{key-value-list}|}}| or the associated |/.append style| variant. See section~\ref{sec:styles} for more detail.
\end{enumerate}
I apologize for any inconvenience caused by these changes.

\begin{pgfplotskey}{compat=\mchoice{1.3,pre 1.3,default,newest} (initially default)}
	The preamble configuration 
\begin{codeexample}[code only]
\usepackage{pgfplots}
\pgfplotsset{compat=1.3}
\end{codeexample}
	allows to choose between backwards compatibility and most recent features.

	\PGFPlots\ version 1.3 comes with new features which may lead to different spacings compared to earlier versions:
	\begin{enumerate}
		\item Version 1.3 comes with the |xlabel near ticks| feature which places (in this case $x$) axis labels ``near'' the tick labels, taking the size of tick labels into account.
		
		Older versions used |xlabel absolute|, i.e.\ an absolute distance between the axis and axis labels (now matter how large or small tick labels are).

		The initial setting \emph{keeps backwards compatibility}.

		You are encouraged to use
\begin{codeexample}[code only]
\usepackage{pgfplots}
\pgfplotsset{compat=1.3}
\end{codeexample}
		in your preamble.
		
		\item Version 1.3 fixes a bug with |anchor=south west| which occurred mainly for reversed axes: the anchors where upside--down. This is fixed now.

		
		If you have written some work--around and want to keep it going, use

		|\pgfplotsset{compat/anchors=pre 1.3}|\pgfmanualpdflabel{/pgfplots/compat/anchors}{} 

		to disable the bug fix.
	\end{enumerate}

	Use |\pgfplotsset{compat=default}| to restore the factory settings.

	The setting |\pgfplotsset{compat=newest}| will always use features of the most recent version. This might result in small changes in the document's appearance.
\end{pgfplotskey}

\subsection{The Team}
\PGFPlots\ has been written mainly by Christian Feuers�nger with many improvements of Pascal Wolkotte and Nick Papior Andersen as a spare time project. We hope it is useful and provides valuable plots.

If you are interested in writing something but don't know how, consider reading the auxiliary manual \href{file:TeX-programming-notes.pdf}{TeX-programming-notes.pdf} which comes with \PGFPlots. It is far from complete, but maybe it is a good starting point (at least for more literature).

\subsection{Acknowledgements}
I thank God for all hours of enjoyed programming. I thank Pascal Wolkotte and Nick Papior Andersen for their programming efforts and contributions as part of the development team. I thank Stefan Tibus, who contributed the |plot shell| feature. I thank Tom Cashman for the contribution of the |reverse legend| feature. Special thanks go to Stefan Pinnow whose tests of \PGFPlots\ lead to numerous quality improvements. Furthermore, I thank Dr.~Schweitzer for many fruitful discussions and Dr.~Meine for his ideas and suggestions. Thanks as well to the many international contributors who provided feature requests or identified bugs or simply manual improvements!

Last but not least, I thank Till Tantau and Mark Wibrow for their excellent graphics (and more) package \PGF\ and \Tikz\ which is the base of \PGFPlots.

%%%%%%%%%%%%%%%%%%%%%%%%%%%%%%%%%%%%%%%%%%%%%%%%%%%%%%%%%%%%%%%%%%%%%%%%%%%%%
%
% Package pgfplots.sty documentation. 
%
% Copyright 2007/2008 by Christian Feuersaenger.
%
% This program is free software: you can redistribute it and/or modify
% it under the terms of the GNU General Public License as published by
% the Free Software Foundation, either version 3 of the License, or
% (at your option) any later version.
% 
% This program is distributed in the hope that it will be useful,
% but WITHOUT ANY WARRANTY; without even the implied warranty of
% MERCHANTABILITY or FITNESS FOR A PARTICULAR PURPOSE.  See the
% GNU General Public License for more details.
% 
% You should have received a copy of the GNU General Public License
% along with this program.  If not, see <http://www.gnu.org/licenses/>.
%
%
%%%%%%%%%%%%%%%%%%%%%%%%%%%%%%%%%%%%%%%%%%%%%%%%%%%%%%%%%%%%%%%%%%%%%%%%%%%%%
\input{pgfplots.preamble.tex}
%\RequirePackage[german,english,francais]{babel}

\pgfplotsmanualexternalexpensivetrue

%\usetikzlibrary{external}
%\tikzexternalize[prefix=figures/]{pgfplots}

\title{%
	Manual for Package \PGFPlots\\
	{\small Version \pgfplotsversion}\\
	{\small\href{http://sourceforge.net/projects/pgfplots}{http://sourceforge.net/projects/pgfplots}}}

%\includeonly{pgfplots.libs}


\begin{document}

\def\plotcoords{%
\addplot coordinates {
(5,8.312e-02)    (17,2.547e-02)   (49,7.407e-03)
(129,2.102e-03)  (321,5.874e-04)  (769,1.623e-04)
(1793,4.442e-05) (4097,1.207e-05) (9217,3.261e-06)
};

\addplot coordinates{
(7,8.472e-02)    (31,3.044e-02)    (111,1.022e-02)
(351,3.303e-03)  (1023,1.039e-03)  (2815,3.196e-04)
(7423,9.658e-05) (18943,2.873e-05) (47103,8.437e-06)
};

\addplot coordinates{
(9,7.881e-02)     (49,3.243e-02)    (209,1.232e-02)
(769,4.454e-03)   (2561,1.551e-03)  (7937,5.236e-04)
(23297,1.723e-04) (65537,5.545e-05) (178177,1.751e-05)
};

\addplot coordinates{
(11,6.887e-02)    (71,3.177e-02)     (351,1.341e-02)
(1471,5.334e-03)  (5503,2.027e-03)   (18943,7.415e-04)
(61183,2.628e-04) (187903,9.063e-05) (553983,3.053e-05)
};

\addplot coordinates{
(13,5.755e-02)     (97,2.925e-02)     (545,1.351e-02)
(2561,5.842e-03)   (10625,2.397e-03)  (40193,9.414e-04)
(141569,3.564e-04) (471041,1.308e-04) 
(1496065,4.670e-05)
};
}%
\maketitle
\begin{abstract}%
\PGFPlots\ draws high--quality function plots in normal or logarithmic scaling with a user-friendly interface directly in \TeX. The user supplies axis labels, legend entries and the plot coordinates for one or more plots and \PGFPlots\ applies axis scaling, computes any logarithms and axis ticks and draws the plots, supporting line plots, scatter plots, piecewise constant plots, bar plots, area plots, mesh-- and surface plots and some more. It is based on Till Tantau's package \PGF/\Tikz.
\end{abstract}
\tableofcontents
\section{Introduction}
This package provides tools to generate plots and labeled axes easily. It draws normal plots, logplots and semi-logplots, in two and three dimensions. Axis ticks, labels, legends (in case of multiple plots) can be added with key-value options. It can cycle through a set of predefined line/marker/color specifications. In summary, its purpose is to simplify the generation of high-quality function and/or data plots, and solving the problems of
\begin{itemize}
	\item consistency of document and font type and font size,
	\item direct use of \TeX\ math mode in axis descriptions,
	\item consistency of data and figures (no third party tool necessary),
	\item inter document consistency using preamble configurations and styles.
\end{itemize}
Although not necessary, separate |.pdf| or |.eps| graphics can be generated using the |external| library developed as part of \Tikz.

\PGFPlots\ is build completely on \Tikz/\PGF. Knowledge of \Tikz\ will simplify the work with \PGFPlots, although it is not required.

\section[About PGFPlots: Preliminaries]{About {\normalfont\PGFPlots}: Preliminaries}
This section contains information about upgrades, the team, the installation (in case you need to do it manually) and troubleshooting. You may skip it completely except for the upgrade remarks.

However, note that this library requires at least \PGF\ version $2.00$. At the time of this writing, many \TeX-distributions still contain the older \PGF\ version $1.18$, so it may be necessary to install a recent \PGF\ prior to using \PGFPlots.

\subsection{Components}
\PGFPlots\ comes with two components:
\begin{enumerate}
	\item the plotting component (which you are currently reading) and
	\item the \PGFPlotstable\ component which simplifies number formatting and postprocessing of numerical tables. It comes as a separate package and has its own manual \href{file:pgfplotstable.pdf}{pgfplotstable.pdf}.
\end{enumerate}

\subsection{Upgrade remarks}
This release provides a lot of improvements which can be found in all detail in \texttt{ChangeLog} for interested readers. However, some attention is useful with respect to the following changes.

\subsubsection{New Optional Features}
\begin{enumerate}
	\item \PGFPlots\ 1.3 comes with user interface improvements. The technical distinction between ``behavior options'' and ``style options'' of older versions is no longer necessary (although still fully supported).

	\item \PGFPlots\ 1.3 has a new feature which allows to \emph{move axis labels tight to tick labels} automatically.

	Since this affects the spacing, it is not enabled be default.

	Use
\begin{codeexample}[code only]
\usepackage{pgfplots}
\pgfplotsset{compat=1.3}
\end{codeexample}
	in your preamble to benefit from the improved spacing\footnote{The |compat=1.3| setting is necessary for new features which move axis labels around like reversed axis. However, \PGFPlots\ will still work as it does in earlier versions, your documents will remain the same if you don't set it explicitly.}.
	Take a look at the next page for the precise definition of |compat|.

	\item \PGFPlots\ 1.3 now supports reversed axes. It is no longer necessary to use work--arounds with negative units.
\pgfkeys{/pdflinks/search key prefixes in/.add={/pgfplots/,}{}}

	Take a look at the |x dir=reverse| key.

	Existing work--arounds will still function properly. Use |\pgfplotsset{compat=1.3}| together with |x dir=reverse| to switch to the new version.
\end{enumerate}

\subsubsection{Old Features Which May Need Attention}
\begin{enumerate}
	\item The |scatter/classes| feature produces proper legends as of version 1.3. This may change the appearance of existing legends of plots with |scatter/classes|.

	\item Due to a small math bug in \PGF\ $2.00$, you \emph{can't} use the math expression `|-x^2|'. It is necessary to use `|0-x^2|' instead. The same holds for
		`|exp(-x^2)|'; use `|exp(0-x^2)|' instead.

		This will be fixed with \PGF\ $>2.00$.

	\item Starting with \PGFPlots\ $1.1$, |\tikzstyle| should \emph{no longer be used} to set \PGFPlots\ options.
	
	Although |\tikzstyle| is still supported for some older \PGFPlots\ options, you should replace any occurance of |\tikzstyle| with |\pgfplotsset{|\meta{style name}|/.style={|\meta{key-value-list}|}}| or the associated |/.append style| variant. See section~\ref{sec:styles} for more detail.
\end{enumerate}
I apologize for any inconvenience caused by these changes.

\begin{pgfplotskey}{compat=\mchoice{1.3,pre 1.3,default,newest} (initially default)}
	The preamble configuration 
\begin{codeexample}[code only]
\usepackage{pgfplots}
\pgfplotsset{compat=1.3}
\end{codeexample}
	allows to choose between backwards compatibility and most recent features.

	\PGFPlots\ version 1.3 comes with new features which may lead to different spacings compared to earlier versions:
	\begin{enumerate}
		\item Version 1.3 comes with the |xlabel near ticks| feature which places (in this case $x$) axis labels ``near'' the tick labels, taking the size of tick labels into account.
		
		Older versions used |xlabel absolute|, i.e.\ an absolute distance between the axis and axis labels (now matter how large or small tick labels are).

		The initial setting \emph{keeps backwards compatibility}.

		You are encouraged to use
\begin{codeexample}[code only]
\usepackage{pgfplots}
\pgfplotsset{compat=1.3}
\end{codeexample}
		in your preamble.
		
		\item Version 1.3 fixes a bug with |anchor=south west| which occurred mainly for reversed axes: the anchors where upside--down. This is fixed now.

		
		If you have written some work--around and want to keep it going, use

		|\pgfplotsset{compat/anchors=pre 1.3}|\pgfmanualpdflabel{/pgfplots/compat/anchors}{} 

		to disable the bug fix.
	\end{enumerate}

	Use |\pgfplotsset{compat=default}| to restore the factory settings.

	The setting |\pgfplotsset{compat=newest}| will always use features of the most recent version. This might result in small changes in the document's appearance.
\end{pgfplotskey}

\subsection{The Team}
\PGFPlots\ has been written mainly by Christian Feuers�nger with many improvements of Pascal Wolkotte and Nick Papior Andersen as a spare time project. We hope it is useful and provides valuable plots.

If you are interested in writing something but don't know how, consider reading the auxiliary manual \href{file:TeX-programming-notes.pdf}{TeX-programming-notes.pdf} which comes with \PGFPlots. It is far from complete, but maybe it is a good starting point (at least for more literature).

\subsection{Acknowledgements}
I thank God for all hours of enjoyed programming. I thank Pascal Wolkotte and Nick Papior Andersen for their programming efforts and contributions as part of the development team. I thank Stefan Tibus, who contributed the |plot shell| feature. I thank Tom Cashman for the contribution of the |reverse legend| feature. Special thanks go to Stefan Pinnow whose tests of \PGFPlots\ lead to numerous quality improvements. Furthermore, I thank Dr.~Schweitzer for many fruitful discussions and Dr.~Meine for his ideas and suggestions. Thanks as well to the many international contributors who provided feature requests or identified bugs or simply manual improvements!

Last but not least, I thank Till Tantau and Mark Wibrow for their excellent graphics (and more) package \PGF\ and \Tikz\ which is the base of \PGFPlots.

\include{pgfplots.install}%
\include{pgfplots.intro}%
\include{pgfplots.reference}%
\include{pgfplots.libs}%
\include{pgfplots.resources}%
\include{pgfplots.importexport}%
\include{pgfplots.basic.reference}%

\printindex

\bibliographystyle{abbrv} %gerapali} %gerabbrv} %gerunsrt.bst} %gerabbrv}% gerplain}
\nocite{pgfplotstable}
\nocite{programmingnotes}
\bibliography{pgfplots}
\end{document}
%
%%%%%%%%%%%%%%%%%%%%%%%%%%%%%%%%%%%%%%%%%%%%%%%%%%%%%%%%%%%%%%%%%%%%%%%%%%%%%
%
% Package pgfplots.sty documentation. 
%
% Copyright 2007/2008 by Christian Feuersaenger.
%
% This program is free software: you can redistribute it and/or modify
% it under the terms of the GNU General Public License as published by
% the Free Software Foundation, either version 3 of the License, or
% (at your option) any later version.
% 
% This program is distributed in the hope that it will be useful,
% but WITHOUT ANY WARRANTY; without even the implied warranty of
% MERCHANTABILITY or FITNESS FOR A PARTICULAR PURPOSE.  See the
% GNU General Public License for more details.
% 
% You should have received a copy of the GNU General Public License
% along with this program.  If not, see <http://www.gnu.org/licenses/>.
%
%
%%%%%%%%%%%%%%%%%%%%%%%%%%%%%%%%%%%%%%%%%%%%%%%%%%%%%%%%%%%%%%%%%%%%%%%%%%%%%
\input{pgfplots.preamble.tex}
%\RequirePackage[german,english,francais]{babel}

\pgfplotsmanualexternalexpensivetrue

%\usetikzlibrary{external}
%\tikzexternalize[prefix=figures/]{pgfplots}

\title{%
	Manual for Package \PGFPlots\\
	{\small Version \pgfplotsversion}\\
	{\small\href{http://sourceforge.net/projects/pgfplots}{http://sourceforge.net/projects/pgfplots}}}

%\includeonly{pgfplots.libs}


\begin{document}

\def\plotcoords{%
\addplot coordinates {
(5,8.312e-02)    (17,2.547e-02)   (49,7.407e-03)
(129,2.102e-03)  (321,5.874e-04)  (769,1.623e-04)
(1793,4.442e-05) (4097,1.207e-05) (9217,3.261e-06)
};

\addplot coordinates{
(7,8.472e-02)    (31,3.044e-02)    (111,1.022e-02)
(351,3.303e-03)  (1023,1.039e-03)  (2815,3.196e-04)
(7423,9.658e-05) (18943,2.873e-05) (47103,8.437e-06)
};

\addplot coordinates{
(9,7.881e-02)     (49,3.243e-02)    (209,1.232e-02)
(769,4.454e-03)   (2561,1.551e-03)  (7937,5.236e-04)
(23297,1.723e-04) (65537,5.545e-05) (178177,1.751e-05)
};

\addplot coordinates{
(11,6.887e-02)    (71,3.177e-02)     (351,1.341e-02)
(1471,5.334e-03)  (5503,2.027e-03)   (18943,7.415e-04)
(61183,2.628e-04) (187903,9.063e-05) (553983,3.053e-05)
};

\addplot coordinates{
(13,5.755e-02)     (97,2.925e-02)     (545,1.351e-02)
(2561,5.842e-03)   (10625,2.397e-03)  (40193,9.414e-04)
(141569,3.564e-04) (471041,1.308e-04) 
(1496065,4.670e-05)
};
}%
\maketitle
\begin{abstract}%
\PGFPlots\ draws high--quality function plots in normal or logarithmic scaling with a user-friendly interface directly in \TeX. The user supplies axis labels, legend entries and the plot coordinates for one or more plots and \PGFPlots\ applies axis scaling, computes any logarithms and axis ticks and draws the plots, supporting line plots, scatter plots, piecewise constant plots, bar plots, area plots, mesh-- and surface plots and some more. It is based on Till Tantau's package \PGF/\Tikz.
\end{abstract}
\tableofcontents
\section{Introduction}
This package provides tools to generate plots and labeled axes easily. It draws normal plots, logplots and semi-logplots, in two and three dimensions. Axis ticks, labels, legends (in case of multiple plots) can be added with key-value options. It can cycle through a set of predefined line/marker/color specifications. In summary, its purpose is to simplify the generation of high-quality function and/or data plots, and solving the problems of
\begin{itemize}
	\item consistency of document and font type and font size,
	\item direct use of \TeX\ math mode in axis descriptions,
	\item consistency of data and figures (no third party tool necessary),
	\item inter document consistency using preamble configurations and styles.
\end{itemize}
Although not necessary, separate |.pdf| or |.eps| graphics can be generated using the |external| library developed as part of \Tikz.

\PGFPlots\ is build completely on \Tikz/\PGF. Knowledge of \Tikz\ will simplify the work with \PGFPlots, although it is not required.

\section[About PGFPlots: Preliminaries]{About {\normalfont\PGFPlots}: Preliminaries}
This section contains information about upgrades, the team, the installation (in case you need to do it manually) and troubleshooting. You may skip it completely except for the upgrade remarks.

However, note that this library requires at least \PGF\ version $2.00$. At the time of this writing, many \TeX-distributions still contain the older \PGF\ version $1.18$, so it may be necessary to install a recent \PGF\ prior to using \PGFPlots.

\subsection{Components}
\PGFPlots\ comes with two components:
\begin{enumerate}
	\item the plotting component (which you are currently reading) and
	\item the \PGFPlotstable\ component which simplifies number formatting and postprocessing of numerical tables. It comes as a separate package and has its own manual \href{file:pgfplotstable.pdf}{pgfplotstable.pdf}.
\end{enumerate}

\subsection{Upgrade remarks}
This release provides a lot of improvements which can be found in all detail in \texttt{ChangeLog} for interested readers. However, some attention is useful with respect to the following changes.

\subsubsection{New Optional Features}
\begin{enumerate}
	\item \PGFPlots\ 1.3 comes with user interface improvements. The technical distinction between ``behavior options'' and ``style options'' of older versions is no longer necessary (although still fully supported).

	\item \PGFPlots\ 1.3 has a new feature which allows to \emph{move axis labels tight to tick labels} automatically.

	Since this affects the spacing, it is not enabled be default.

	Use
\begin{codeexample}[code only]
\usepackage{pgfplots}
\pgfplotsset{compat=1.3}
\end{codeexample}
	in your preamble to benefit from the improved spacing\footnote{The |compat=1.3| setting is necessary for new features which move axis labels around like reversed axis. However, \PGFPlots\ will still work as it does in earlier versions, your documents will remain the same if you don't set it explicitly.}.
	Take a look at the next page for the precise definition of |compat|.

	\item \PGFPlots\ 1.3 now supports reversed axes. It is no longer necessary to use work--arounds with negative units.
\pgfkeys{/pdflinks/search key prefixes in/.add={/pgfplots/,}{}}

	Take a look at the |x dir=reverse| key.

	Existing work--arounds will still function properly. Use |\pgfplotsset{compat=1.3}| together with |x dir=reverse| to switch to the new version.
\end{enumerate}

\subsubsection{Old Features Which May Need Attention}
\begin{enumerate}
	\item The |scatter/classes| feature produces proper legends as of version 1.3. This may change the appearance of existing legends of plots with |scatter/classes|.

	\item Due to a small math bug in \PGF\ $2.00$, you \emph{can't} use the math expression `|-x^2|'. It is necessary to use `|0-x^2|' instead. The same holds for
		`|exp(-x^2)|'; use `|exp(0-x^2)|' instead.

		This will be fixed with \PGF\ $>2.00$.

	\item Starting with \PGFPlots\ $1.1$, |\tikzstyle| should \emph{no longer be used} to set \PGFPlots\ options.
	
	Although |\tikzstyle| is still supported for some older \PGFPlots\ options, you should replace any occurance of |\tikzstyle| with |\pgfplotsset{|\meta{style name}|/.style={|\meta{key-value-list}|}}| or the associated |/.append style| variant. See section~\ref{sec:styles} for more detail.
\end{enumerate}
I apologize for any inconvenience caused by these changes.

\begin{pgfplotskey}{compat=\mchoice{1.3,pre 1.3,default,newest} (initially default)}
	The preamble configuration 
\begin{codeexample}[code only]
\usepackage{pgfplots}
\pgfplotsset{compat=1.3}
\end{codeexample}
	allows to choose between backwards compatibility and most recent features.

	\PGFPlots\ version 1.3 comes with new features which may lead to different spacings compared to earlier versions:
	\begin{enumerate}
		\item Version 1.3 comes with the |xlabel near ticks| feature which places (in this case $x$) axis labels ``near'' the tick labels, taking the size of tick labels into account.
		
		Older versions used |xlabel absolute|, i.e.\ an absolute distance between the axis and axis labels (now matter how large or small tick labels are).

		The initial setting \emph{keeps backwards compatibility}.

		You are encouraged to use
\begin{codeexample}[code only]
\usepackage{pgfplots}
\pgfplotsset{compat=1.3}
\end{codeexample}
		in your preamble.
		
		\item Version 1.3 fixes a bug with |anchor=south west| which occurred mainly for reversed axes: the anchors where upside--down. This is fixed now.

		
		If you have written some work--around and want to keep it going, use

		|\pgfplotsset{compat/anchors=pre 1.3}|\pgfmanualpdflabel{/pgfplots/compat/anchors}{} 

		to disable the bug fix.
	\end{enumerate}

	Use |\pgfplotsset{compat=default}| to restore the factory settings.

	The setting |\pgfplotsset{compat=newest}| will always use features of the most recent version. This might result in small changes in the document's appearance.
\end{pgfplotskey}

\subsection{The Team}
\PGFPlots\ has been written mainly by Christian Feuers�nger with many improvements of Pascal Wolkotte and Nick Papior Andersen as a spare time project. We hope it is useful and provides valuable plots.

If you are interested in writing something but don't know how, consider reading the auxiliary manual \href{file:TeX-programming-notes.pdf}{TeX-programming-notes.pdf} which comes with \PGFPlots. It is far from complete, but maybe it is a good starting point (at least for more literature).

\subsection{Acknowledgements}
I thank God for all hours of enjoyed programming. I thank Pascal Wolkotte and Nick Papior Andersen for their programming efforts and contributions as part of the development team. I thank Stefan Tibus, who contributed the |plot shell| feature. I thank Tom Cashman for the contribution of the |reverse legend| feature. Special thanks go to Stefan Pinnow whose tests of \PGFPlots\ lead to numerous quality improvements. Furthermore, I thank Dr.~Schweitzer for many fruitful discussions and Dr.~Meine for his ideas and suggestions. Thanks as well to the many international contributors who provided feature requests or identified bugs or simply manual improvements!

Last but not least, I thank Till Tantau and Mark Wibrow for their excellent graphics (and more) package \PGF\ and \Tikz\ which is the base of \PGFPlots.

\include{pgfplots.install}%
\include{pgfplots.intro}%
\include{pgfplots.reference}%
\include{pgfplots.libs}%
\include{pgfplots.resources}%
\include{pgfplots.importexport}%
\include{pgfplots.basic.reference}%

\printindex

\bibliographystyle{abbrv} %gerapali} %gerabbrv} %gerunsrt.bst} %gerabbrv}% gerplain}
\nocite{pgfplotstable}
\nocite{programmingnotes}
\bibliography{pgfplots}
\end{document}
%
%%%%%%%%%%%%%%%%%%%%%%%%%%%%%%%%%%%%%%%%%%%%%%%%%%%%%%%%%%%%%%%%%%%%%%%%%%%%%
%
% Package pgfplots.sty documentation. 
%
% Copyright 2007/2008 by Christian Feuersaenger.
%
% This program is free software: you can redistribute it and/or modify
% it under the terms of the GNU General Public License as published by
% the Free Software Foundation, either version 3 of the License, or
% (at your option) any later version.
% 
% This program is distributed in the hope that it will be useful,
% but WITHOUT ANY WARRANTY; without even the implied warranty of
% MERCHANTABILITY or FITNESS FOR A PARTICULAR PURPOSE.  See the
% GNU General Public License for more details.
% 
% You should have received a copy of the GNU General Public License
% along with this program.  If not, see <http://www.gnu.org/licenses/>.
%
%
%%%%%%%%%%%%%%%%%%%%%%%%%%%%%%%%%%%%%%%%%%%%%%%%%%%%%%%%%%%%%%%%%%%%%%%%%%%%%
\input{pgfplots.preamble.tex}
%\RequirePackage[german,english,francais]{babel}

\pgfplotsmanualexternalexpensivetrue

%\usetikzlibrary{external}
%\tikzexternalize[prefix=figures/]{pgfplots}

\title{%
	Manual for Package \PGFPlots\\
	{\small Version \pgfplotsversion}\\
	{\small\href{http://sourceforge.net/projects/pgfplots}{http://sourceforge.net/projects/pgfplots}}}

%\includeonly{pgfplots.libs}


\begin{document}

\def\plotcoords{%
\addplot coordinates {
(5,8.312e-02)    (17,2.547e-02)   (49,7.407e-03)
(129,2.102e-03)  (321,5.874e-04)  (769,1.623e-04)
(1793,4.442e-05) (4097,1.207e-05) (9217,3.261e-06)
};

\addplot coordinates{
(7,8.472e-02)    (31,3.044e-02)    (111,1.022e-02)
(351,3.303e-03)  (1023,1.039e-03)  (2815,3.196e-04)
(7423,9.658e-05) (18943,2.873e-05) (47103,8.437e-06)
};

\addplot coordinates{
(9,7.881e-02)     (49,3.243e-02)    (209,1.232e-02)
(769,4.454e-03)   (2561,1.551e-03)  (7937,5.236e-04)
(23297,1.723e-04) (65537,5.545e-05) (178177,1.751e-05)
};

\addplot coordinates{
(11,6.887e-02)    (71,3.177e-02)     (351,1.341e-02)
(1471,5.334e-03)  (5503,2.027e-03)   (18943,7.415e-04)
(61183,2.628e-04) (187903,9.063e-05) (553983,3.053e-05)
};

\addplot coordinates{
(13,5.755e-02)     (97,2.925e-02)     (545,1.351e-02)
(2561,5.842e-03)   (10625,2.397e-03)  (40193,9.414e-04)
(141569,3.564e-04) (471041,1.308e-04) 
(1496065,4.670e-05)
};
}%
\maketitle
\begin{abstract}%
\PGFPlots\ draws high--quality function plots in normal or logarithmic scaling with a user-friendly interface directly in \TeX. The user supplies axis labels, legend entries and the plot coordinates for one or more plots and \PGFPlots\ applies axis scaling, computes any logarithms and axis ticks and draws the plots, supporting line plots, scatter plots, piecewise constant plots, bar plots, area plots, mesh-- and surface plots and some more. It is based on Till Tantau's package \PGF/\Tikz.
\end{abstract}
\tableofcontents
\section{Introduction}
This package provides tools to generate plots and labeled axes easily. It draws normal plots, logplots and semi-logplots, in two and three dimensions. Axis ticks, labels, legends (in case of multiple plots) can be added with key-value options. It can cycle through a set of predefined line/marker/color specifications. In summary, its purpose is to simplify the generation of high-quality function and/or data plots, and solving the problems of
\begin{itemize}
	\item consistency of document and font type and font size,
	\item direct use of \TeX\ math mode in axis descriptions,
	\item consistency of data and figures (no third party tool necessary),
	\item inter document consistency using preamble configurations and styles.
\end{itemize}
Although not necessary, separate |.pdf| or |.eps| graphics can be generated using the |external| library developed as part of \Tikz.

\PGFPlots\ is build completely on \Tikz/\PGF. Knowledge of \Tikz\ will simplify the work with \PGFPlots, although it is not required.

\section[About PGFPlots: Preliminaries]{About {\normalfont\PGFPlots}: Preliminaries}
This section contains information about upgrades, the team, the installation (in case you need to do it manually) and troubleshooting. You may skip it completely except for the upgrade remarks.

However, note that this library requires at least \PGF\ version $2.00$. At the time of this writing, many \TeX-distributions still contain the older \PGF\ version $1.18$, so it may be necessary to install a recent \PGF\ prior to using \PGFPlots.

\subsection{Components}
\PGFPlots\ comes with two components:
\begin{enumerate}
	\item the plotting component (which you are currently reading) and
	\item the \PGFPlotstable\ component which simplifies number formatting and postprocessing of numerical tables. It comes as a separate package and has its own manual \href{file:pgfplotstable.pdf}{pgfplotstable.pdf}.
\end{enumerate}

\subsection{Upgrade remarks}
This release provides a lot of improvements which can be found in all detail in \texttt{ChangeLog} for interested readers. However, some attention is useful with respect to the following changes.

\subsubsection{New Optional Features}
\begin{enumerate}
	\item \PGFPlots\ 1.3 comes with user interface improvements. The technical distinction between ``behavior options'' and ``style options'' of older versions is no longer necessary (although still fully supported).

	\item \PGFPlots\ 1.3 has a new feature which allows to \emph{move axis labels tight to tick labels} automatically.

	Since this affects the spacing, it is not enabled be default.

	Use
\begin{codeexample}[code only]
\usepackage{pgfplots}
\pgfplotsset{compat=1.3}
\end{codeexample}
	in your preamble to benefit from the improved spacing\footnote{The |compat=1.3| setting is necessary for new features which move axis labels around like reversed axis. However, \PGFPlots\ will still work as it does in earlier versions, your documents will remain the same if you don't set it explicitly.}.
	Take a look at the next page for the precise definition of |compat|.

	\item \PGFPlots\ 1.3 now supports reversed axes. It is no longer necessary to use work--arounds with negative units.
\pgfkeys{/pdflinks/search key prefixes in/.add={/pgfplots/,}{}}

	Take a look at the |x dir=reverse| key.

	Existing work--arounds will still function properly. Use |\pgfplotsset{compat=1.3}| together with |x dir=reverse| to switch to the new version.
\end{enumerate}

\subsubsection{Old Features Which May Need Attention}
\begin{enumerate}
	\item The |scatter/classes| feature produces proper legends as of version 1.3. This may change the appearance of existing legends of plots with |scatter/classes|.

	\item Due to a small math bug in \PGF\ $2.00$, you \emph{can't} use the math expression `|-x^2|'. It is necessary to use `|0-x^2|' instead. The same holds for
		`|exp(-x^2)|'; use `|exp(0-x^2)|' instead.

		This will be fixed with \PGF\ $>2.00$.

	\item Starting with \PGFPlots\ $1.1$, |\tikzstyle| should \emph{no longer be used} to set \PGFPlots\ options.
	
	Although |\tikzstyle| is still supported for some older \PGFPlots\ options, you should replace any occurance of |\tikzstyle| with |\pgfplotsset{|\meta{style name}|/.style={|\meta{key-value-list}|}}| or the associated |/.append style| variant. See section~\ref{sec:styles} for more detail.
\end{enumerate}
I apologize for any inconvenience caused by these changes.

\begin{pgfplotskey}{compat=\mchoice{1.3,pre 1.3,default,newest} (initially default)}
	The preamble configuration 
\begin{codeexample}[code only]
\usepackage{pgfplots}
\pgfplotsset{compat=1.3}
\end{codeexample}
	allows to choose between backwards compatibility and most recent features.

	\PGFPlots\ version 1.3 comes with new features which may lead to different spacings compared to earlier versions:
	\begin{enumerate}
		\item Version 1.3 comes with the |xlabel near ticks| feature which places (in this case $x$) axis labels ``near'' the tick labels, taking the size of tick labels into account.
		
		Older versions used |xlabel absolute|, i.e.\ an absolute distance between the axis and axis labels (now matter how large or small tick labels are).

		The initial setting \emph{keeps backwards compatibility}.

		You are encouraged to use
\begin{codeexample}[code only]
\usepackage{pgfplots}
\pgfplotsset{compat=1.3}
\end{codeexample}
		in your preamble.
		
		\item Version 1.3 fixes a bug with |anchor=south west| which occurred mainly for reversed axes: the anchors where upside--down. This is fixed now.

		
		If you have written some work--around and want to keep it going, use

		|\pgfplotsset{compat/anchors=pre 1.3}|\pgfmanualpdflabel{/pgfplots/compat/anchors}{} 

		to disable the bug fix.
	\end{enumerate}

	Use |\pgfplotsset{compat=default}| to restore the factory settings.

	The setting |\pgfplotsset{compat=newest}| will always use features of the most recent version. This might result in small changes in the document's appearance.
\end{pgfplotskey}

\subsection{The Team}
\PGFPlots\ has been written mainly by Christian Feuers�nger with many improvements of Pascal Wolkotte and Nick Papior Andersen as a spare time project. We hope it is useful and provides valuable plots.

If you are interested in writing something but don't know how, consider reading the auxiliary manual \href{file:TeX-programming-notes.pdf}{TeX-programming-notes.pdf} which comes with \PGFPlots. It is far from complete, but maybe it is a good starting point (at least for more literature).

\subsection{Acknowledgements}
I thank God for all hours of enjoyed programming. I thank Pascal Wolkotte and Nick Papior Andersen for their programming efforts and contributions as part of the development team. I thank Stefan Tibus, who contributed the |plot shell| feature. I thank Tom Cashman for the contribution of the |reverse legend| feature. Special thanks go to Stefan Pinnow whose tests of \PGFPlots\ lead to numerous quality improvements. Furthermore, I thank Dr.~Schweitzer for many fruitful discussions and Dr.~Meine for his ideas and suggestions. Thanks as well to the many international contributors who provided feature requests or identified bugs or simply manual improvements!

Last but not least, I thank Till Tantau and Mark Wibrow for their excellent graphics (and more) package \PGF\ and \Tikz\ which is the base of \PGFPlots.

\include{pgfplots.install}%
\include{pgfplots.intro}%
\include{pgfplots.reference}%
\include{pgfplots.libs}%
\include{pgfplots.resources}%
\include{pgfplots.importexport}%
\include{pgfplots.basic.reference}%

\printindex

\bibliographystyle{abbrv} %gerapali} %gerabbrv} %gerunsrt.bst} %gerabbrv}% gerplain}
\nocite{pgfplotstable}
\nocite{programmingnotes}
\bibliography{pgfplots}
\end{document}
%
%%%%%%%%%%%%%%%%%%%%%%%%%%%%%%%%%%%%%%%%%%%%%%%%%%%%%%%%%%%%%%%%%%%%%%%%%%%%%
%
% Package pgfplots.sty documentation. 
%
% Copyright 2007/2008 by Christian Feuersaenger.
%
% This program is free software: you can redistribute it and/or modify
% it under the terms of the GNU General Public License as published by
% the Free Software Foundation, either version 3 of the License, or
% (at your option) any later version.
% 
% This program is distributed in the hope that it will be useful,
% but WITHOUT ANY WARRANTY; without even the implied warranty of
% MERCHANTABILITY or FITNESS FOR A PARTICULAR PURPOSE.  See the
% GNU General Public License for more details.
% 
% You should have received a copy of the GNU General Public License
% along with this program.  If not, see <http://www.gnu.org/licenses/>.
%
%
%%%%%%%%%%%%%%%%%%%%%%%%%%%%%%%%%%%%%%%%%%%%%%%%%%%%%%%%%%%%%%%%%%%%%%%%%%%%%
\input{pgfplots.preamble.tex}
%\RequirePackage[german,english,francais]{babel}

\pgfplotsmanualexternalexpensivetrue

%\usetikzlibrary{external}
%\tikzexternalize[prefix=figures/]{pgfplots}

\title{%
	Manual for Package \PGFPlots\\
	{\small Version \pgfplotsversion}\\
	{\small\href{http://sourceforge.net/projects/pgfplots}{http://sourceforge.net/projects/pgfplots}}}

%\includeonly{pgfplots.libs}


\begin{document}

\def\plotcoords{%
\addplot coordinates {
(5,8.312e-02)    (17,2.547e-02)   (49,7.407e-03)
(129,2.102e-03)  (321,5.874e-04)  (769,1.623e-04)
(1793,4.442e-05) (4097,1.207e-05) (9217,3.261e-06)
};

\addplot coordinates{
(7,8.472e-02)    (31,3.044e-02)    (111,1.022e-02)
(351,3.303e-03)  (1023,1.039e-03)  (2815,3.196e-04)
(7423,9.658e-05) (18943,2.873e-05) (47103,8.437e-06)
};

\addplot coordinates{
(9,7.881e-02)     (49,3.243e-02)    (209,1.232e-02)
(769,4.454e-03)   (2561,1.551e-03)  (7937,5.236e-04)
(23297,1.723e-04) (65537,5.545e-05) (178177,1.751e-05)
};

\addplot coordinates{
(11,6.887e-02)    (71,3.177e-02)     (351,1.341e-02)
(1471,5.334e-03)  (5503,2.027e-03)   (18943,7.415e-04)
(61183,2.628e-04) (187903,9.063e-05) (553983,3.053e-05)
};

\addplot coordinates{
(13,5.755e-02)     (97,2.925e-02)     (545,1.351e-02)
(2561,5.842e-03)   (10625,2.397e-03)  (40193,9.414e-04)
(141569,3.564e-04) (471041,1.308e-04) 
(1496065,4.670e-05)
};
}%
\maketitle
\begin{abstract}%
\PGFPlots\ draws high--quality function plots in normal or logarithmic scaling with a user-friendly interface directly in \TeX. The user supplies axis labels, legend entries and the plot coordinates for one or more plots and \PGFPlots\ applies axis scaling, computes any logarithms and axis ticks and draws the plots, supporting line plots, scatter plots, piecewise constant plots, bar plots, area plots, mesh-- and surface plots and some more. It is based on Till Tantau's package \PGF/\Tikz.
\end{abstract}
\tableofcontents
\section{Introduction}
This package provides tools to generate plots and labeled axes easily. It draws normal plots, logplots and semi-logplots, in two and three dimensions. Axis ticks, labels, legends (in case of multiple plots) can be added with key-value options. It can cycle through a set of predefined line/marker/color specifications. In summary, its purpose is to simplify the generation of high-quality function and/or data plots, and solving the problems of
\begin{itemize}
	\item consistency of document and font type and font size,
	\item direct use of \TeX\ math mode in axis descriptions,
	\item consistency of data and figures (no third party tool necessary),
	\item inter document consistency using preamble configurations and styles.
\end{itemize}
Although not necessary, separate |.pdf| or |.eps| graphics can be generated using the |external| library developed as part of \Tikz.

\PGFPlots\ is build completely on \Tikz/\PGF. Knowledge of \Tikz\ will simplify the work with \PGFPlots, although it is not required.

\section[About PGFPlots: Preliminaries]{About {\normalfont\PGFPlots}: Preliminaries}
This section contains information about upgrades, the team, the installation (in case you need to do it manually) and troubleshooting. You may skip it completely except for the upgrade remarks.

However, note that this library requires at least \PGF\ version $2.00$. At the time of this writing, many \TeX-distributions still contain the older \PGF\ version $1.18$, so it may be necessary to install a recent \PGF\ prior to using \PGFPlots.

\subsection{Components}
\PGFPlots\ comes with two components:
\begin{enumerate}
	\item the plotting component (which you are currently reading) and
	\item the \PGFPlotstable\ component which simplifies number formatting and postprocessing of numerical tables. It comes as a separate package and has its own manual \href{file:pgfplotstable.pdf}{pgfplotstable.pdf}.
\end{enumerate}

\subsection{Upgrade remarks}
This release provides a lot of improvements which can be found in all detail in \texttt{ChangeLog} for interested readers. However, some attention is useful with respect to the following changes.

\subsubsection{New Optional Features}
\begin{enumerate}
	\item \PGFPlots\ 1.3 comes with user interface improvements. The technical distinction between ``behavior options'' and ``style options'' of older versions is no longer necessary (although still fully supported).

	\item \PGFPlots\ 1.3 has a new feature which allows to \emph{move axis labels tight to tick labels} automatically.

	Since this affects the spacing, it is not enabled be default.

	Use
\begin{codeexample}[code only]
\usepackage{pgfplots}
\pgfplotsset{compat=1.3}
\end{codeexample}
	in your preamble to benefit from the improved spacing\footnote{The |compat=1.3| setting is necessary for new features which move axis labels around like reversed axis. However, \PGFPlots\ will still work as it does in earlier versions, your documents will remain the same if you don't set it explicitly.}.
	Take a look at the next page for the precise definition of |compat|.

	\item \PGFPlots\ 1.3 now supports reversed axes. It is no longer necessary to use work--arounds with negative units.
\pgfkeys{/pdflinks/search key prefixes in/.add={/pgfplots/,}{}}

	Take a look at the |x dir=reverse| key.

	Existing work--arounds will still function properly. Use |\pgfplotsset{compat=1.3}| together with |x dir=reverse| to switch to the new version.
\end{enumerate}

\subsubsection{Old Features Which May Need Attention}
\begin{enumerate}
	\item The |scatter/classes| feature produces proper legends as of version 1.3. This may change the appearance of existing legends of plots with |scatter/classes|.

	\item Due to a small math bug in \PGF\ $2.00$, you \emph{can't} use the math expression `|-x^2|'. It is necessary to use `|0-x^2|' instead. The same holds for
		`|exp(-x^2)|'; use `|exp(0-x^2)|' instead.

		This will be fixed with \PGF\ $>2.00$.

	\item Starting with \PGFPlots\ $1.1$, |\tikzstyle| should \emph{no longer be used} to set \PGFPlots\ options.
	
	Although |\tikzstyle| is still supported for some older \PGFPlots\ options, you should replace any occurance of |\tikzstyle| with |\pgfplotsset{|\meta{style name}|/.style={|\meta{key-value-list}|}}| or the associated |/.append style| variant. See section~\ref{sec:styles} for more detail.
\end{enumerate}
I apologize for any inconvenience caused by these changes.

\begin{pgfplotskey}{compat=\mchoice{1.3,pre 1.3,default,newest} (initially default)}
	The preamble configuration 
\begin{codeexample}[code only]
\usepackage{pgfplots}
\pgfplotsset{compat=1.3}
\end{codeexample}
	allows to choose between backwards compatibility and most recent features.

	\PGFPlots\ version 1.3 comes with new features which may lead to different spacings compared to earlier versions:
	\begin{enumerate}
		\item Version 1.3 comes with the |xlabel near ticks| feature which places (in this case $x$) axis labels ``near'' the tick labels, taking the size of tick labels into account.
		
		Older versions used |xlabel absolute|, i.e.\ an absolute distance between the axis and axis labels (now matter how large or small tick labels are).

		The initial setting \emph{keeps backwards compatibility}.

		You are encouraged to use
\begin{codeexample}[code only]
\usepackage{pgfplots}
\pgfplotsset{compat=1.3}
\end{codeexample}
		in your preamble.
		
		\item Version 1.3 fixes a bug with |anchor=south west| which occurred mainly for reversed axes: the anchors where upside--down. This is fixed now.

		
		If you have written some work--around and want to keep it going, use

		|\pgfplotsset{compat/anchors=pre 1.3}|\pgfmanualpdflabel{/pgfplots/compat/anchors}{} 

		to disable the bug fix.
	\end{enumerate}

	Use |\pgfplotsset{compat=default}| to restore the factory settings.

	The setting |\pgfplotsset{compat=newest}| will always use features of the most recent version. This might result in small changes in the document's appearance.
\end{pgfplotskey}

\subsection{The Team}
\PGFPlots\ has been written mainly by Christian Feuers�nger with many improvements of Pascal Wolkotte and Nick Papior Andersen as a spare time project. We hope it is useful and provides valuable plots.

If you are interested in writing something but don't know how, consider reading the auxiliary manual \href{file:TeX-programming-notes.pdf}{TeX-programming-notes.pdf} which comes with \PGFPlots. It is far from complete, but maybe it is a good starting point (at least for more literature).

\subsection{Acknowledgements}
I thank God for all hours of enjoyed programming. I thank Pascal Wolkotte and Nick Papior Andersen for their programming efforts and contributions as part of the development team. I thank Stefan Tibus, who contributed the |plot shell| feature. I thank Tom Cashman for the contribution of the |reverse legend| feature. Special thanks go to Stefan Pinnow whose tests of \PGFPlots\ lead to numerous quality improvements. Furthermore, I thank Dr.~Schweitzer for many fruitful discussions and Dr.~Meine for his ideas and suggestions. Thanks as well to the many international contributors who provided feature requests or identified bugs or simply manual improvements!

Last but not least, I thank Till Tantau and Mark Wibrow for their excellent graphics (and more) package \PGF\ and \Tikz\ which is the base of \PGFPlots.

\include{pgfplots.install}%
\include{pgfplots.intro}%
\include{pgfplots.reference}%
\include{pgfplots.libs}%
\include{pgfplots.resources}%
\include{pgfplots.importexport}%
\include{pgfplots.basic.reference}%

\printindex

\bibliographystyle{abbrv} %gerapali} %gerabbrv} %gerunsrt.bst} %gerabbrv}% gerplain}
\nocite{pgfplotstable}
\nocite{programmingnotes}
\bibliography{pgfplots}
\end{document}
%
%%%%%%%%%%%%%%%%%%%%%%%%%%%%%%%%%%%%%%%%%%%%%%%%%%%%%%%%%%%%%%%%%%%%%%%%%%%%%
%
% Package pgfplots.sty documentation. 
%
% Copyright 2007/2008 by Christian Feuersaenger.
%
% This program is free software: you can redistribute it and/or modify
% it under the terms of the GNU General Public License as published by
% the Free Software Foundation, either version 3 of the License, or
% (at your option) any later version.
% 
% This program is distributed in the hope that it will be useful,
% but WITHOUT ANY WARRANTY; without even the implied warranty of
% MERCHANTABILITY or FITNESS FOR A PARTICULAR PURPOSE.  See the
% GNU General Public License for more details.
% 
% You should have received a copy of the GNU General Public License
% along with this program.  If not, see <http://www.gnu.org/licenses/>.
%
%
%%%%%%%%%%%%%%%%%%%%%%%%%%%%%%%%%%%%%%%%%%%%%%%%%%%%%%%%%%%%%%%%%%%%%%%%%%%%%
\input{pgfplots.preamble.tex}
%\RequirePackage[german,english,francais]{babel}

\pgfplotsmanualexternalexpensivetrue

%\usetikzlibrary{external}
%\tikzexternalize[prefix=figures/]{pgfplots}

\title{%
	Manual for Package \PGFPlots\\
	{\small Version \pgfplotsversion}\\
	{\small\href{http://sourceforge.net/projects/pgfplots}{http://sourceforge.net/projects/pgfplots}}}

%\includeonly{pgfplots.libs}


\begin{document}

\def\plotcoords{%
\addplot coordinates {
(5,8.312e-02)    (17,2.547e-02)   (49,7.407e-03)
(129,2.102e-03)  (321,5.874e-04)  (769,1.623e-04)
(1793,4.442e-05) (4097,1.207e-05) (9217,3.261e-06)
};

\addplot coordinates{
(7,8.472e-02)    (31,3.044e-02)    (111,1.022e-02)
(351,3.303e-03)  (1023,1.039e-03)  (2815,3.196e-04)
(7423,9.658e-05) (18943,2.873e-05) (47103,8.437e-06)
};

\addplot coordinates{
(9,7.881e-02)     (49,3.243e-02)    (209,1.232e-02)
(769,4.454e-03)   (2561,1.551e-03)  (7937,5.236e-04)
(23297,1.723e-04) (65537,5.545e-05) (178177,1.751e-05)
};

\addplot coordinates{
(11,6.887e-02)    (71,3.177e-02)     (351,1.341e-02)
(1471,5.334e-03)  (5503,2.027e-03)   (18943,7.415e-04)
(61183,2.628e-04) (187903,9.063e-05) (553983,3.053e-05)
};

\addplot coordinates{
(13,5.755e-02)     (97,2.925e-02)     (545,1.351e-02)
(2561,5.842e-03)   (10625,2.397e-03)  (40193,9.414e-04)
(141569,3.564e-04) (471041,1.308e-04) 
(1496065,4.670e-05)
};
}%
\maketitle
\begin{abstract}%
\PGFPlots\ draws high--quality function plots in normal or logarithmic scaling with a user-friendly interface directly in \TeX. The user supplies axis labels, legend entries and the plot coordinates for one or more plots and \PGFPlots\ applies axis scaling, computes any logarithms and axis ticks and draws the plots, supporting line plots, scatter plots, piecewise constant plots, bar plots, area plots, mesh-- and surface plots and some more. It is based on Till Tantau's package \PGF/\Tikz.
\end{abstract}
\tableofcontents
\section{Introduction}
This package provides tools to generate plots and labeled axes easily. It draws normal plots, logplots and semi-logplots, in two and three dimensions. Axis ticks, labels, legends (in case of multiple plots) can be added with key-value options. It can cycle through a set of predefined line/marker/color specifications. In summary, its purpose is to simplify the generation of high-quality function and/or data plots, and solving the problems of
\begin{itemize}
	\item consistency of document and font type and font size,
	\item direct use of \TeX\ math mode in axis descriptions,
	\item consistency of data and figures (no third party tool necessary),
	\item inter document consistency using preamble configurations and styles.
\end{itemize}
Although not necessary, separate |.pdf| or |.eps| graphics can be generated using the |external| library developed as part of \Tikz.

\PGFPlots\ is build completely on \Tikz/\PGF. Knowledge of \Tikz\ will simplify the work with \PGFPlots, although it is not required.

\section[About PGFPlots: Preliminaries]{About {\normalfont\PGFPlots}: Preliminaries}
This section contains information about upgrades, the team, the installation (in case you need to do it manually) and troubleshooting. You may skip it completely except for the upgrade remarks.

However, note that this library requires at least \PGF\ version $2.00$. At the time of this writing, many \TeX-distributions still contain the older \PGF\ version $1.18$, so it may be necessary to install a recent \PGF\ prior to using \PGFPlots.

\subsection{Components}
\PGFPlots\ comes with two components:
\begin{enumerate}
	\item the plotting component (which you are currently reading) and
	\item the \PGFPlotstable\ component which simplifies number formatting and postprocessing of numerical tables. It comes as a separate package and has its own manual \href{file:pgfplotstable.pdf}{pgfplotstable.pdf}.
\end{enumerate}

\subsection{Upgrade remarks}
This release provides a lot of improvements which can be found in all detail in \texttt{ChangeLog} for interested readers. However, some attention is useful with respect to the following changes.

\subsubsection{New Optional Features}
\begin{enumerate}
	\item \PGFPlots\ 1.3 comes with user interface improvements. The technical distinction between ``behavior options'' and ``style options'' of older versions is no longer necessary (although still fully supported).

	\item \PGFPlots\ 1.3 has a new feature which allows to \emph{move axis labels tight to tick labels} automatically.

	Since this affects the spacing, it is not enabled be default.

	Use
\begin{codeexample}[code only]
\usepackage{pgfplots}
\pgfplotsset{compat=1.3}
\end{codeexample}
	in your preamble to benefit from the improved spacing\footnote{The |compat=1.3| setting is necessary for new features which move axis labels around like reversed axis. However, \PGFPlots\ will still work as it does in earlier versions, your documents will remain the same if you don't set it explicitly.}.
	Take a look at the next page for the precise definition of |compat|.

	\item \PGFPlots\ 1.3 now supports reversed axes. It is no longer necessary to use work--arounds with negative units.
\pgfkeys{/pdflinks/search key prefixes in/.add={/pgfplots/,}{}}

	Take a look at the |x dir=reverse| key.

	Existing work--arounds will still function properly. Use |\pgfplotsset{compat=1.3}| together with |x dir=reverse| to switch to the new version.
\end{enumerate}

\subsubsection{Old Features Which May Need Attention}
\begin{enumerate}
	\item The |scatter/classes| feature produces proper legends as of version 1.3. This may change the appearance of existing legends of plots with |scatter/classes|.

	\item Due to a small math bug in \PGF\ $2.00$, you \emph{can't} use the math expression `|-x^2|'. It is necessary to use `|0-x^2|' instead. The same holds for
		`|exp(-x^2)|'; use `|exp(0-x^2)|' instead.

		This will be fixed with \PGF\ $>2.00$.

	\item Starting with \PGFPlots\ $1.1$, |\tikzstyle| should \emph{no longer be used} to set \PGFPlots\ options.
	
	Although |\tikzstyle| is still supported for some older \PGFPlots\ options, you should replace any occurance of |\tikzstyle| with |\pgfplotsset{|\meta{style name}|/.style={|\meta{key-value-list}|}}| or the associated |/.append style| variant. See section~\ref{sec:styles} for more detail.
\end{enumerate}
I apologize for any inconvenience caused by these changes.

\begin{pgfplotskey}{compat=\mchoice{1.3,pre 1.3,default,newest} (initially default)}
	The preamble configuration 
\begin{codeexample}[code only]
\usepackage{pgfplots}
\pgfplotsset{compat=1.3}
\end{codeexample}
	allows to choose between backwards compatibility and most recent features.

	\PGFPlots\ version 1.3 comes with new features which may lead to different spacings compared to earlier versions:
	\begin{enumerate}
		\item Version 1.3 comes with the |xlabel near ticks| feature which places (in this case $x$) axis labels ``near'' the tick labels, taking the size of tick labels into account.
		
		Older versions used |xlabel absolute|, i.e.\ an absolute distance between the axis and axis labels (now matter how large or small tick labels are).

		The initial setting \emph{keeps backwards compatibility}.

		You are encouraged to use
\begin{codeexample}[code only]
\usepackage{pgfplots}
\pgfplotsset{compat=1.3}
\end{codeexample}
		in your preamble.
		
		\item Version 1.3 fixes a bug with |anchor=south west| which occurred mainly for reversed axes: the anchors where upside--down. This is fixed now.

		
		If you have written some work--around and want to keep it going, use

		|\pgfplotsset{compat/anchors=pre 1.3}|\pgfmanualpdflabel{/pgfplots/compat/anchors}{} 

		to disable the bug fix.
	\end{enumerate}

	Use |\pgfplotsset{compat=default}| to restore the factory settings.

	The setting |\pgfplotsset{compat=newest}| will always use features of the most recent version. This might result in small changes in the document's appearance.
\end{pgfplotskey}

\subsection{The Team}
\PGFPlots\ has been written mainly by Christian Feuers�nger with many improvements of Pascal Wolkotte and Nick Papior Andersen as a spare time project. We hope it is useful and provides valuable plots.

If you are interested in writing something but don't know how, consider reading the auxiliary manual \href{file:TeX-programming-notes.pdf}{TeX-programming-notes.pdf} which comes with \PGFPlots. It is far from complete, but maybe it is a good starting point (at least for more literature).

\subsection{Acknowledgements}
I thank God for all hours of enjoyed programming. I thank Pascal Wolkotte and Nick Papior Andersen for their programming efforts and contributions as part of the development team. I thank Stefan Tibus, who contributed the |plot shell| feature. I thank Tom Cashman for the contribution of the |reverse legend| feature. Special thanks go to Stefan Pinnow whose tests of \PGFPlots\ lead to numerous quality improvements. Furthermore, I thank Dr.~Schweitzer for many fruitful discussions and Dr.~Meine for his ideas and suggestions. Thanks as well to the many international contributors who provided feature requests or identified bugs or simply manual improvements!

Last but not least, I thank Till Tantau and Mark Wibrow for their excellent graphics (and more) package \PGF\ and \Tikz\ which is the base of \PGFPlots.

\include{pgfplots.install}%
\include{pgfplots.intro}%
\include{pgfplots.reference}%
\include{pgfplots.libs}%
\include{pgfplots.resources}%
\include{pgfplots.importexport}%
\include{pgfplots.basic.reference}%

\printindex

\bibliographystyle{abbrv} %gerapali} %gerabbrv} %gerunsrt.bst} %gerabbrv}% gerplain}
\nocite{pgfplotstable}
\nocite{programmingnotes}
\bibliography{pgfplots}
\end{document}
%
%%%%%%%%%%%%%%%%%%%%%%%%%%%%%%%%%%%%%%%%%%%%%%%%%%%%%%%%%%%%%%%%%%%%%%%%%%%%%
%
% Package pgfplots.sty documentation. 
%
% Copyright 2007/2008 by Christian Feuersaenger.
%
% This program is free software: you can redistribute it and/or modify
% it under the terms of the GNU General Public License as published by
% the Free Software Foundation, either version 3 of the License, or
% (at your option) any later version.
% 
% This program is distributed in the hope that it will be useful,
% but WITHOUT ANY WARRANTY; without even the implied warranty of
% MERCHANTABILITY or FITNESS FOR A PARTICULAR PURPOSE.  See the
% GNU General Public License for more details.
% 
% You should have received a copy of the GNU General Public License
% along with this program.  If not, see <http://www.gnu.org/licenses/>.
%
%
%%%%%%%%%%%%%%%%%%%%%%%%%%%%%%%%%%%%%%%%%%%%%%%%%%%%%%%%%%%%%%%%%%%%%%%%%%%%%
\input{pgfplots.preamble.tex}
%\RequirePackage[german,english,francais]{babel}

\pgfplotsmanualexternalexpensivetrue

%\usetikzlibrary{external}
%\tikzexternalize[prefix=figures/]{pgfplots}

\title{%
	Manual for Package \PGFPlots\\
	{\small Version \pgfplotsversion}\\
	{\small\href{http://sourceforge.net/projects/pgfplots}{http://sourceforge.net/projects/pgfplots}}}

%\includeonly{pgfplots.libs}


\begin{document}

\def\plotcoords{%
\addplot coordinates {
(5,8.312e-02)    (17,2.547e-02)   (49,7.407e-03)
(129,2.102e-03)  (321,5.874e-04)  (769,1.623e-04)
(1793,4.442e-05) (4097,1.207e-05) (9217,3.261e-06)
};

\addplot coordinates{
(7,8.472e-02)    (31,3.044e-02)    (111,1.022e-02)
(351,3.303e-03)  (1023,1.039e-03)  (2815,3.196e-04)
(7423,9.658e-05) (18943,2.873e-05) (47103,8.437e-06)
};

\addplot coordinates{
(9,7.881e-02)     (49,3.243e-02)    (209,1.232e-02)
(769,4.454e-03)   (2561,1.551e-03)  (7937,5.236e-04)
(23297,1.723e-04) (65537,5.545e-05) (178177,1.751e-05)
};

\addplot coordinates{
(11,6.887e-02)    (71,3.177e-02)     (351,1.341e-02)
(1471,5.334e-03)  (5503,2.027e-03)   (18943,7.415e-04)
(61183,2.628e-04) (187903,9.063e-05) (553983,3.053e-05)
};

\addplot coordinates{
(13,5.755e-02)     (97,2.925e-02)     (545,1.351e-02)
(2561,5.842e-03)   (10625,2.397e-03)  (40193,9.414e-04)
(141569,3.564e-04) (471041,1.308e-04) 
(1496065,4.670e-05)
};
}%
\maketitle
\begin{abstract}%
\PGFPlots\ draws high--quality function plots in normal or logarithmic scaling with a user-friendly interface directly in \TeX. The user supplies axis labels, legend entries and the plot coordinates for one or more plots and \PGFPlots\ applies axis scaling, computes any logarithms and axis ticks and draws the plots, supporting line plots, scatter plots, piecewise constant plots, bar plots, area plots, mesh-- and surface plots and some more. It is based on Till Tantau's package \PGF/\Tikz.
\end{abstract}
\tableofcontents
\section{Introduction}
This package provides tools to generate plots and labeled axes easily. It draws normal plots, logplots and semi-logplots, in two and three dimensions. Axis ticks, labels, legends (in case of multiple plots) can be added with key-value options. It can cycle through a set of predefined line/marker/color specifications. In summary, its purpose is to simplify the generation of high-quality function and/or data plots, and solving the problems of
\begin{itemize}
	\item consistency of document and font type and font size,
	\item direct use of \TeX\ math mode in axis descriptions,
	\item consistency of data and figures (no third party tool necessary),
	\item inter document consistency using preamble configurations and styles.
\end{itemize}
Although not necessary, separate |.pdf| or |.eps| graphics can be generated using the |external| library developed as part of \Tikz.

\PGFPlots\ is build completely on \Tikz/\PGF. Knowledge of \Tikz\ will simplify the work with \PGFPlots, although it is not required.

\section[About PGFPlots: Preliminaries]{About {\normalfont\PGFPlots}: Preliminaries}
This section contains information about upgrades, the team, the installation (in case you need to do it manually) and troubleshooting. You may skip it completely except for the upgrade remarks.

However, note that this library requires at least \PGF\ version $2.00$. At the time of this writing, many \TeX-distributions still contain the older \PGF\ version $1.18$, so it may be necessary to install a recent \PGF\ prior to using \PGFPlots.

\subsection{Components}
\PGFPlots\ comes with two components:
\begin{enumerate}
	\item the plotting component (which you are currently reading) and
	\item the \PGFPlotstable\ component which simplifies number formatting and postprocessing of numerical tables. It comes as a separate package and has its own manual \href{file:pgfplotstable.pdf}{pgfplotstable.pdf}.
\end{enumerate}

\subsection{Upgrade remarks}
This release provides a lot of improvements which can be found in all detail in \texttt{ChangeLog} for interested readers. However, some attention is useful with respect to the following changes.

\subsubsection{New Optional Features}
\begin{enumerate}
	\item \PGFPlots\ 1.3 comes with user interface improvements. The technical distinction between ``behavior options'' and ``style options'' of older versions is no longer necessary (although still fully supported).

	\item \PGFPlots\ 1.3 has a new feature which allows to \emph{move axis labels tight to tick labels} automatically.

	Since this affects the spacing, it is not enabled be default.

	Use
\begin{codeexample}[code only]
\usepackage{pgfplots}
\pgfplotsset{compat=1.3}
\end{codeexample}
	in your preamble to benefit from the improved spacing\footnote{The |compat=1.3| setting is necessary for new features which move axis labels around like reversed axis. However, \PGFPlots\ will still work as it does in earlier versions, your documents will remain the same if you don't set it explicitly.}.
	Take a look at the next page for the precise definition of |compat|.

	\item \PGFPlots\ 1.3 now supports reversed axes. It is no longer necessary to use work--arounds with negative units.
\pgfkeys{/pdflinks/search key prefixes in/.add={/pgfplots/,}{}}

	Take a look at the |x dir=reverse| key.

	Existing work--arounds will still function properly. Use |\pgfplotsset{compat=1.3}| together with |x dir=reverse| to switch to the new version.
\end{enumerate}

\subsubsection{Old Features Which May Need Attention}
\begin{enumerate}
	\item The |scatter/classes| feature produces proper legends as of version 1.3. This may change the appearance of existing legends of plots with |scatter/classes|.

	\item Due to a small math bug in \PGF\ $2.00$, you \emph{can't} use the math expression `|-x^2|'. It is necessary to use `|0-x^2|' instead. The same holds for
		`|exp(-x^2)|'; use `|exp(0-x^2)|' instead.

		This will be fixed with \PGF\ $>2.00$.

	\item Starting with \PGFPlots\ $1.1$, |\tikzstyle| should \emph{no longer be used} to set \PGFPlots\ options.
	
	Although |\tikzstyle| is still supported for some older \PGFPlots\ options, you should replace any occurance of |\tikzstyle| with |\pgfplotsset{|\meta{style name}|/.style={|\meta{key-value-list}|}}| or the associated |/.append style| variant. See section~\ref{sec:styles} for more detail.
\end{enumerate}
I apologize for any inconvenience caused by these changes.

\begin{pgfplotskey}{compat=\mchoice{1.3,pre 1.3,default,newest} (initially default)}
	The preamble configuration 
\begin{codeexample}[code only]
\usepackage{pgfplots}
\pgfplotsset{compat=1.3}
\end{codeexample}
	allows to choose between backwards compatibility and most recent features.

	\PGFPlots\ version 1.3 comes with new features which may lead to different spacings compared to earlier versions:
	\begin{enumerate}
		\item Version 1.3 comes with the |xlabel near ticks| feature which places (in this case $x$) axis labels ``near'' the tick labels, taking the size of tick labels into account.
		
		Older versions used |xlabel absolute|, i.e.\ an absolute distance between the axis and axis labels (now matter how large or small tick labels are).

		The initial setting \emph{keeps backwards compatibility}.

		You are encouraged to use
\begin{codeexample}[code only]
\usepackage{pgfplots}
\pgfplotsset{compat=1.3}
\end{codeexample}
		in your preamble.
		
		\item Version 1.3 fixes a bug with |anchor=south west| which occurred mainly for reversed axes: the anchors where upside--down. This is fixed now.

		
		If you have written some work--around and want to keep it going, use

		|\pgfplotsset{compat/anchors=pre 1.3}|\pgfmanualpdflabel{/pgfplots/compat/anchors}{} 

		to disable the bug fix.
	\end{enumerate}

	Use |\pgfplotsset{compat=default}| to restore the factory settings.

	The setting |\pgfplotsset{compat=newest}| will always use features of the most recent version. This might result in small changes in the document's appearance.
\end{pgfplotskey}

\subsection{The Team}
\PGFPlots\ has been written mainly by Christian Feuers�nger with many improvements of Pascal Wolkotte and Nick Papior Andersen as a spare time project. We hope it is useful and provides valuable plots.

If you are interested in writing something but don't know how, consider reading the auxiliary manual \href{file:TeX-programming-notes.pdf}{TeX-programming-notes.pdf} which comes with \PGFPlots. It is far from complete, but maybe it is a good starting point (at least for more literature).

\subsection{Acknowledgements}
I thank God for all hours of enjoyed programming. I thank Pascal Wolkotte and Nick Papior Andersen for their programming efforts and contributions as part of the development team. I thank Stefan Tibus, who contributed the |plot shell| feature. I thank Tom Cashman for the contribution of the |reverse legend| feature. Special thanks go to Stefan Pinnow whose tests of \PGFPlots\ lead to numerous quality improvements. Furthermore, I thank Dr.~Schweitzer for many fruitful discussions and Dr.~Meine for his ideas and suggestions. Thanks as well to the many international contributors who provided feature requests or identified bugs or simply manual improvements!

Last but not least, I thank Till Tantau and Mark Wibrow for their excellent graphics (and more) package \PGF\ and \Tikz\ which is the base of \PGFPlots.

\include{pgfplots.install}%
\include{pgfplots.intro}%
\include{pgfplots.reference}%
\include{pgfplots.libs}%
\include{pgfplots.resources}%
\include{pgfplots.importexport}%
\include{pgfplots.basic.reference}%

\printindex

\bibliographystyle{abbrv} %gerapali} %gerabbrv} %gerunsrt.bst} %gerabbrv}% gerplain}
\nocite{pgfplotstable}
\nocite{programmingnotes}
\bibliography{pgfplots}
\end{document}
%
\include{pgfplots.basic.reference}%

\printindex

\bibliographystyle{abbrv} %gerapali} %gerabbrv} %gerunsrt.bst} %gerabbrv}% gerplain}
\nocite{pgfplotstable}
\nocite{programmingnotes}
\bibliography{pgfplots}
\end{document}
%
%%%%%%%%%%%%%%%%%%%%%%%%%%%%%%%%%%%%%%%%%%%%%%%%%%%%%%%%%%%%%%%%%%%%%%%%%%%%%
%
% Package pgfplots.sty documentation. 
%
% Copyright 2007/2008 by Christian Feuersaenger.
%
% This program is free software: you can redistribute it and/or modify
% it under the terms of the GNU General Public License as published by
% the Free Software Foundation, either version 3 of the License, or
% (at your option) any later version.
% 
% This program is distributed in the hope that it will be useful,
% but WITHOUT ANY WARRANTY; without even the implied warranty of
% MERCHANTABILITY or FITNESS FOR A PARTICULAR PURPOSE.  See the
% GNU General Public License for more details.
% 
% You should have received a copy of the GNU General Public License
% along with this program.  If not, see <http://www.gnu.org/licenses/>.
%
%
%%%%%%%%%%%%%%%%%%%%%%%%%%%%%%%%%%%%%%%%%%%%%%%%%%%%%%%%%%%%%%%%%%%%%%%%%%%%%
%%%%%%%%%%%%%%%%%%%%%%%%%%%%%%%%%%%%%%%%%%%%%%%%%%%%%%%%%%%%%%%%%%%%%%%%%%%%%
%
% Package pgfplots.sty documentation. 
%
% Copyright 2007/2008 by Christian Feuersaenger.
%
% This program is free software: you can redistribute it and/or modify
% it under the terms of the GNU General Public License as published by
% the Free Software Foundation, either version 3 of the License, or
% (at your option) any later version.
% 
% This program is distributed in the hope that it will be useful,
% but WITHOUT ANY WARRANTY; without even the implied warranty of
% MERCHANTABILITY or FITNESS FOR A PARTICULAR PURPOSE.  See the
% GNU General Public License for more details.
% 
% You should have received a copy of the GNU General Public License
% along with this program.  If not, see <http://www.gnu.org/licenses/>.
%
%
%%%%%%%%%%%%%%%%%%%%%%%%%%%%%%%%%%%%%%%%%%%%%%%%%%%%%%%%%%%%%%%%%%%%%%%%%%%%%
\documentclass[a4paper]{ltxdoc}

\usepackage{makeidx}

% DON't let hyperref overload the format of index and glossary. 
% I want to do that on my own in the stylefiles for makeindex...
\makeatletter
\let\@old@wrindex=\@wrindex
\makeatother

\usepackage{ifpdf}
\ifpdf
	\usepackage[pdftex]{hyperref}
\else
	\usepackage[dvipdfm]{hyperref}
\fi
\hypersetup{pdfborder={0 0 0}}

\makeatletter
\let\@wrindex=\@old@wrindex
\makeatother

% Formatiere Seitennummern im Index:
\newcommand{\indexpageno}[1]{%
	{\bfseries\hyperpage{#1}}%
}


\newcommand{\R}{\mathbb{R}}
\newcommand{\N}{\mathbb{N}}
\newcommand{\Z}{\mathbb{Z}}

\long\def\COMMENTLOWLEVEL#1\ENDCOMMENT{}
\def\ENDCOMMENT{}

\usepackage{textcomp}
\usepackage{booktabs}

\usepackage{calc}
\usepackage[formats]{listings}
%\usepackage{courier} % don't use it - the '^' character can't be copy-pasted in courier

\usepackage{array}
\lstset{%
	basicstyle=\ttfamily,
	language=[LaTeX]tex, % Seems as if \lstset{language=tex} must be invoked BEFORE loading tikz!?
	tabsize=4,
	breaklines=true,
	breakindent=0pt
}

\ifpdf
	\pdfinfo {
		/Author	(Christian Feuersaenger)
	}

\else
	\def\pgfsysdriver{pgfsys-dvipdfm.def}
\fi
%\def\pgfsysdriver{pgfsys-pdftex.def}
\usepackage{pgfplots}

\ifpdf
	\usetikzlibrary{pgfplotsclickable}
\fi

%\usepackage{fp}
% ATTENTION:
% this requires pgf version NEWER than 2.00 :
%\usetikzlibrary{fixedpointarithmetic}

\usetikzlibrary{dateplot}

\usepackage[a4paper,left=2.25cm,right=2.25cm,top=2.5cm,bottom=2.5cm,nohead]{geometry}
\usepackage{amsmath,amssymb}
\usepackage{xxcolor}
\usepackage{pifont}
\usepackage[latin1]{inputenc}
\usepackage{amsmath}
\usepackage{eurosym}
\input{pgfplots-macros}

\pgfqkeys{/codeexample}{%
	every codeexample/.style={
		width=8cm,
		/pgfplots/every axis/.append style={legend style={fill=graphicbackground}}
	},
	tabsize=4,
}
\pgfplotsset{
	every axis/.append style={width=7cm},
	filter discard warning=false,
}

\usetikzlibrary{backgrounds,patterns}
% Global styles:
\tikzset{
  shape example/.style={
    color=black!30,
    draw,
    fill=yellow!30,
    line width=.5cm,
    inner xsep=2.5cm,
    inner ysep=0.5cm}
}

\newcommand{\FIXME}[1]{\textcolor{red}{(FIXME: #1)}}

% fuer endvironment 'sidewaysfigure' bspw
% \usepackage{rotating}

\newcommand\Tikz{Ti\textit kZ}
\newcommand\PGF{\textsc{pgf}}
\newcommand\PGFPlots{\textsc{pgfplots}}
\def\PGFPlotstable{\textsc{PgfplotsTable}}


\makeindex

% Fix overful hboxes automatically:
\tolerance=2000
\emergencystretch=10pt

\tikzset{prefix=gnuplot/pgfplots_} % prefix for 'plot function'

\author{%
	Christian Feuers\"anger\footnote{\url{http://wissrech.ins.uni-bonn.de/people/feuersaenger}}\\%
	Institut f\"ur Numerische Simulation\\
	Universit\"at Bonn, Germany}


%\RequirePackage[german,english,francais]{babel}

\pgfplotsmanualexternalexpensivetrue

%\usetikzlibrary{external}
%\tikzexternalize[prefix=figures/]{pgfplots}

\title{%
	Manual for Package \PGFPlots\\
	{\small Version \pgfplotsversion}\\
	{\small\href{http://sourceforge.net/projects/pgfplots}{http://sourceforge.net/projects/pgfplots}}}

%\includeonly{pgfplots.libs}


\begin{document}

\def\plotcoords{%
\addplot coordinates {
(5,8.312e-02)    (17,2.547e-02)   (49,7.407e-03)
(129,2.102e-03)  (321,5.874e-04)  (769,1.623e-04)
(1793,4.442e-05) (4097,1.207e-05) (9217,3.261e-06)
};

\addplot coordinates{
(7,8.472e-02)    (31,3.044e-02)    (111,1.022e-02)
(351,3.303e-03)  (1023,1.039e-03)  (2815,3.196e-04)
(7423,9.658e-05) (18943,2.873e-05) (47103,8.437e-06)
};

\addplot coordinates{
(9,7.881e-02)     (49,3.243e-02)    (209,1.232e-02)
(769,4.454e-03)   (2561,1.551e-03)  (7937,5.236e-04)
(23297,1.723e-04) (65537,5.545e-05) (178177,1.751e-05)
};

\addplot coordinates{
(11,6.887e-02)    (71,3.177e-02)     (351,1.341e-02)
(1471,5.334e-03)  (5503,2.027e-03)   (18943,7.415e-04)
(61183,2.628e-04) (187903,9.063e-05) (553983,3.053e-05)
};

\addplot coordinates{
(13,5.755e-02)     (97,2.925e-02)     (545,1.351e-02)
(2561,5.842e-03)   (10625,2.397e-03)  (40193,9.414e-04)
(141569,3.564e-04) (471041,1.308e-04) 
(1496065,4.670e-05)
};
}%
\maketitle
\begin{abstract}%
\PGFPlots\ draws high--quality function plots in normal or logarithmic scaling with a user-friendly interface directly in \TeX. The user supplies axis labels, legend entries and the plot coordinates for one or more plots and \PGFPlots\ applies axis scaling, computes any logarithms and axis ticks and draws the plots, supporting line plots, scatter plots, piecewise constant plots, bar plots, area plots, mesh-- and surface plots and some more. It is based on Till Tantau's package \PGF/\Tikz.
\end{abstract}
\tableofcontents
\section{Introduction}
This package provides tools to generate plots and labeled axes easily. It draws normal plots, logplots and semi-logplots, in two and three dimensions. Axis ticks, labels, legends (in case of multiple plots) can be added with key-value options. It can cycle through a set of predefined line/marker/color specifications. In summary, its purpose is to simplify the generation of high-quality function and/or data plots, and solving the problems of
\begin{itemize}
	\item consistency of document and font type and font size,
	\item direct use of \TeX\ math mode in axis descriptions,
	\item consistency of data and figures (no third party tool necessary),
	\item inter document consistency using preamble configurations and styles.
\end{itemize}
Although not necessary, separate |.pdf| or |.eps| graphics can be generated using the |external| library developed as part of \Tikz.

\PGFPlots\ is build completely on \Tikz/\PGF. Knowledge of \Tikz\ will simplify the work with \PGFPlots, although it is not required.

\section[About PGFPlots: Preliminaries]{About {\normalfont\PGFPlots}: Preliminaries}
This section contains information about upgrades, the team, the installation (in case you need to do it manually) and troubleshooting. You may skip it completely except for the upgrade remarks.

However, note that this library requires at least \PGF\ version $2.00$. At the time of this writing, many \TeX-distributions still contain the older \PGF\ version $1.18$, so it may be necessary to install a recent \PGF\ prior to using \PGFPlots.

\subsection{Components}
\PGFPlots\ comes with two components:
\begin{enumerate}
	\item the plotting component (which you are currently reading) and
	\item the \PGFPlotstable\ component which simplifies number formatting and postprocessing of numerical tables. It comes as a separate package and has its own manual \href{file:pgfplotstable.pdf}{pgfplotstable.pdf}.
\end{enumerate}

\subsection{Upgrade remarks}
This release provides a lot of improvements which can be found in all detail in \texttt{ChangeLog} for interested readers. However, some attention is useful with respect to the following changes.

\subsubsection{New Optional Features}
\begin{enumerate}
	\item \PGFPlots\ 1.3 comes with user interface improvements. The technical distinction between ``behavior options'' and ``style options'' of older versions is no longer necessary (although still fully supported).

	\item \PGFPlots\ 1.3 has a new feature which allows to \emph{move axis labels tight to tick labels} automatically.

	Since this affects the spacing, it is not enabled be default.

	Use
\begin{codeexample}[code only]
\usepackage{pgfplots}
\pgfplotsset{compat=1.3}
\end{codeexample}
	in your preamble to benefit from the improved spacing\footnote{The |compat=1.3| setting is necessary for new features which move axis labels around like reversed axis. However, \PGFPlots\ will still work as it does in earlier versions, your documents will remain the same if you don't set it explicitly.}.
	Take a look at the next page for the precise definition of |compat|.

	\item \PGFPlots\ 1.3 now supports reversed axes. It is no longer necessary to use work--arounds with negative units.
\pgfkeys{/pdflinks/search key prefixes in/.add={/pgfplots/,}{}}

	Take a look at the |x dir=reverse| key.

	Existing work--arounds will still function properly. Use |\pgfplotsset{compat=1.3}| together with |x dir=reverse| to switch to the new version.
\end{enumerate}

\subsubsection{Old Features Which May Need Attention}
\begin{enumerate}
	\item The |scatter/classes| feature produces proper legends as of version 1.3. This may change the appearance of existing legends of plots with |scatter/classes|.

	\item Due to a small math bug in \PGF\ $2.00$, you \emph{can't} use the math expression `|-x^2|'. It is necessary to use `|0-x^2|' instead. The same holds for
		`|exp(-x^2)|'; use `|exp(0-x^2)|' instead.

		This will be fixed with \PGF\ $>2.00$.

	\item Starting with \PGFPlots\ $1.1$, |\tikzstyle| should \emph{no longer be used} to set \PGFPlots\ options.
	
	Although |\tikzstyle| is still supported for some older \PGFPlots\ options, you should replace any occurance of |\tikzstyle| with |\pgfplotsset{|\meta{style name}|/.style={|\meta{key-value-list}|}}| or the associated |/.append style| variant. See section~\ref{sec:styles} for more detail.
\end{enumerate}
I apologize for any inconvenience caused by these changes.

\begin{pgfplotskey}{compat=\mchoice{1.3,pre 1.3,default,newest} (initially default)}
	The preamble configuration 
\begin{codeexample}[code only]
\usepackage{pgfplots}
\pgfplotsset{compat=1.3}
\end{codeexample}
	allows to choose between backwards compatibility and most recent features.

	\PGFPlots\ version 1.3 comes with new features which may lead to different spacings compared to earlier versions:
	\begin{enumerate}
		\item Version 1.3 comes with the |xlabel near ticks| feature which places (in this case $x$) axis labels ``near'' the tick labels, taking the size of tick labels into account.
		
		Older versions used |xlabel absolute|, i.e.\ an absolute distance between the axis and axis labels (now matter how large or small tick labels are).

		The initial setting \emph{keeps backwards compatibility}.

		You are encouraged to use
\begin{codeexample}[code only]
\usepackage{pgfplots}
\pgfplotsset{compat=1.3}
\end{codeexample}
		in your preamble.
		
		\item Version 1.3 fixes a bug with |anchor=south west| which occurred mainly for reversed axes: the anchors where upside--down. This is fixed now.

		
		If you have written some work--around and want to keep it going, use

		|\pgfplotsset{compat/anchors=pre 1.3}|\pgfmanualpdflabel{/pgfplots/compat/anchors}{} 

		to disable the bug fix.
	\end{enumerate}

	Use |\pgfplotsset{compat=default}| to restore the factory settings.

	The setting |\pgfplotsset{compat=newest}| will always use features of the most recent version. This might result in small changes in the document's appearance.
\end{pgfplotskey}

\subsection{The Team}
\PGFPlots\ has been written mainly by Christian Feuers�nger with many improvements of Pascal Wolkotte and Nick Papior Andersen as a spare time project. We hope it is useful and provides valuable plots.

If you are interested in writing something but don't know how, consider reading the auxiliary manual \href{file:TeX-programming-notes.pdf}{TeX-programming-notes.pdf} which comes with \PGFPlots. It is far from complete, but maybe it is a good starting point (at least for more literature).

\subsection{Acknowledgements}
I thank God for all hours of enjoyed programming. I thank Pascal Wolkotte and Nick Papior Andersen for their programming efforts and contributions as part of the development team. I thank Stefan Tibus, who contributed the |plot shell| feature. I thank Tom Cashman for the contribution of the |reverse legend| feature. Special thanks go to Stefan Pinnow whose tests of \PGFPlots\ lead to numerous quality improvements. Furthermore, I thank Dr.~Schweitzer for many fruitful discussions and Dr.~Meine for his ideas and suggestions. Thanks as well to the many international contributors who provided feature requests or identified bugs or simply manual improvements!

Last but not least, I thank Till Tantau and Mark Wibrow for their excellent graphics (and more) package \PGF\ and \Tikz\ which is the base of \PGFPlots.

%%%%%%%%%%%%%%%%%%%%%%%%%%%%%%%%%%%%%%%%%%%%%%%%%%%%%%%%%%%%%%%%%%%%%%%%%%%%%
%
% Package pgfplots.sty documentation. 
%
% Copyright 2007/2008 by Christian Feuersaenger.
%
% This program is free software: you can redistribute it and/or modify
% it under the terms of the GNU General Public License as published by
% the Free Software Foundation, either version 3 of the License, or
% (at your option) any later version.
% 
% This program is distributed in the hope that it will be useful,
% but WITHOUT ANY WARRANTY; without even the implied warranty of
% MERCHANTABILITY or FITNESS FOR A PARTICULAR PURPOSE.  See the
% GNU General Public License for more details.
% 
% You should have received a copy of the GNU General Public License
% along with this program.  If not, see <http://www.gnu.org/licenses/>.
%
%
%%%%%%%%%%%%%%%%%%%%%%%%%%%%%%%%%%%%%%%%%%%%%%%%%%%%%%%%%%%%%%%%%%%%%%%%%%%%%
\input{pgfplots.preamble.tex}
%\RequirePackage[german,english,francais]{babel}

\pgfplotsmanualexternalexpensivetrue

%\usetikzlibrary{external}
%\tikzexternalize[prefix=figures/]{pgfplots}

\title{%
	Manual for Package \PGFPlots\\
	{\small Version \pgfplotsversion}\\
	{\small\href{http://sourceforge.net/projects/pgfplots}{http://sourceforge.net/projects/pgfplots}}}

%\includeonly{pgfplots.libs}


\begin{document}

\def\plotcoords{%
\addplot coordinates {
(5,8.312e-02)    (17,2.547e-02)   (49,7.407e-03)
(129,2.102e-03)  (321,5.874e-04)  (769,1.623e-04)
(1793,4.442e-05) (4097,1.207e-05) (9217,3.261e-06)
};

\addplot coordinates{
(7,8.472e-02)    (31,3.044e-02)    (111,1.022e-02)
(351,3.303e-03)  (1023,1.039e-03)  (2815,3.196e-04)
(7423,9.658e-05) (18943,2.873e-05) (47103,8.437e-06)
};

\addplot coordinates{
(9,7.881e-02)     (49,3.243e-02)    (209,1.232e-02)
(769,4.454e-03)   (2561,1.551e-03)  (7937,5.236e-04)
(23297,1.723e-04) (65537,5.545e-05) (178177,1.751e-05)
};

\addplot coordinates{
(11,6.887e-02)    (71,3.177e-02)     (351,1.341e-02)
(1471,5.334e-03)  (5503,2.027e-03)   (18943,7.415e-04)
(61183,2.628e-04) (187903,9.063e-05) (553983,3.053e-05)
};

\addplot coordinates{
(13,5.755e-02)     (97,2.925e-02)     (545,1.351e-02)
(2561,5.842e-03)   (10625,2.397e-03)  (40193,9.414e-04)
(141569,3.564e-04) (471041,1.308e-04) 
(1496065,4.670e-05)
};
}%
\maketitle
\begin{abstract}%
\PGFPlots\ draws high--quality function plots in normal or logarithmic scaling with a user-friendly interface directly in \TeX. The user supplies axis labels, legend entries and the plot coordinates for one or more plots and \PGFPlots\ applies axis scaling, computes any logarithms and axis ticks and draws the plots, supporting line plots, scatter plots, piecewise constant plots, bar plots, area plots, mesh-- and surface plots and some more. It is based on Till Tantau's package \PGF/\Tikz.
\end{abstract}
\tableofcontents
\section{Introduction}
This package provides tools to generate plots and labeled axes easily. It draws normal plots, logplots and semi-logplots, in two and three dimensions. Axis ticks, labels, legends (in case of multiple plots) can be added with key-value options. It can cycle through a set of predefined line/marker/color specifications. In summary, its purpose is to simplify the generation of high-quality function and/or data plots, and solving the problems of
\begin{itemize}
	\item consistency of document and font type and font size,
	\item direct use of \TeX\ math mode in axis descriptions,
	\item consistency of data and figures (no third party tool necessary),
	\item inter document consistency using preamble configurations and styles.
\end{itemize}
Although not necessary, separate |.pdf| or |.eps| graphics can be generated using the |external| library developed as part of \Tikz.

\PGFPlots\ is build completely on \Tikz/\PGF. Knowledge of \Tikz\ will simplify the work with \PGFPlots, although it is not required.

\section[About PGFPlots: Preliminaries]{About {\normalfont\PGFPlots}: Preliminaries}
This section contains information about upgrades, the team, the installation (in case you need to do it manually) and troubleshooting. You may skip it completely except for the upgrade remarks.

However, note that this library requires at least \PGF\ version $2.00$. At the time of this writing, many \TeX-distributions still contain the older \PGF\ version $1.18$, so it may be necessary to install a recent \PGF\ prior to using \PGFPlots.

\subsection{Components}
\PGFPlots\ comes with two components:
\begin{enumerate}
	\item the plotting component (which you are currently reading) and
	\item the \PGFPlotstable\ component which simplifies number formatting and postprocessing of numerical tables. It comes as a separate package and has its own manual \href{file:pgfplotstable.pdf}{pgfplotstable.pdf}.
\end{enumerate}

\subsection{Upgrade remarks}
This release provides a lot of improvements which can be found in all detail in \texttt{ChangeLog} for interested readers. However, some attention is useful with respect to the following changes.

\subsubsection{New Optional Features}
\begin{enumerate}
	\item \PGFPlots\ 1.3 comes with user interface improvements. The technical distinction between ``behavior options'' and ``style options'' of older versions is no longer necessary (although still fully supported).

	\item \PGFPlots\ 1.3 has a new feature which allows to \emph{move axis labels tight to tick labels} automatically.

	Since this affects the spacing, it is not enabled be default.

	Use
\begin{codeexample}[code only]
\usepackage{pgfplots}
\pgfplotsset{compat=1.3}
\end{codeexample}
	in your preamble to benefit from the improved spacing\footnote{The |compat=1.3| setting is necessary for new features which move axis labels around like reversed axis. However, \PGFPlots\ will still work as it does in earlier versions, your documents will remain the same if you don't set it explicitly.}.
	Take a look at the next page for the precise definition of |compat|.

	\item \PGFPlots\ 1.3 now supports reversed axes. It is no longer necessary to use work--arounds with negative units.
\pgfkeys{/pdflinks/search key prefixes in/.add={/pgfplots/,}{}}

	Take a look at the |x dir=reverse| key.

	Existing work--arounds will still function properly. Use |\pgfplotsset{compat=1.3}| together with |x dir=reverse| to switch to the new version.
\end{enumerate}

\subsubsection{Old Features Which May Need Attention}
\begin{enumerate}
	\item The |scatter/classes| feature produces proper legends as of version 1.3. This may change the appearance of existing legends of plots with |scatter/classes|.

	\item Due to a small math bug in \PGF\ $2.00$, you \emph{can't} use the math expression `|-x^2|'. It is necessary to use `|0-x^2|' instead. The same holds for
		`|exp(-x^2)|'; use `|exp(0-x^2)|' instead.

		This will be fixed with \PGF\ $>2.00$.

	\item Starting with \PGFPlots\ $1.1$, |\tikzstyle| should \emph{no longer be used} to set \PGFPlots\ options.
	
	Although |\tikzstyle| is still supported for some older \PGFPlots\ options, you should replace any occurance of |\tikzstyle| with |\pgfplotsset{|\meta{style name}|/.style={|\meta{key-value-list}|}}| or the associated |/.append style| variant. See section~\ref{sec:styles} for more detail.
\end{enumerate}
I apologize for any inconvenience caused by these changes.

\begin{pgfplotskey}{compat=\mchoice{1.3,pre 1.3,default,newest} (initially default)}
	The preamble configuration 
\begin{codeexample}[code only]
\usepackage{pgfplots}
\pgfplotsset{compat=1.3}
\end{codeexample}
	allows to choose between backwards compatibility and most recent features.

	\PGFPlots\ version 1.3 comes with new features which may lead to different spacings compared to earlier versions:
	\begin{enumerate}
		\item Version 1.3 comes with the |xlabel near ticks| feature which places (in this case $x$) axis labels ``near'' the tick labels, taking the size of tick labels into account.
		
		Older versions used |xlabel absolute|, i.e.\ an absolute distance between the axis and axis labels (now matter how large or small tick labels are).

		The initial setting \emph{keeps backwards compatibility}.

		You are encouraged to use
\begin{codeexample}[code only]
\usepackage{pgfplots}
\pgfplotsset{compat=1.3}
\end{codeexample}
		in your preamble.
		
		\item Version 1.3 fixes a bug with |anchor=south west| which occurred mainly for reversed axes: the anchors where upside--down. This is fixed now.

		
		If you have written some work--around and want to keep it going, use

		|\pgfplotsset{compat/anchors=pre 1.3}|\pgfmanualpdflabel{/pgfplots/compat/anchors}{} 

		to disable the bug fix.
	\end{enumerate}

	Use |\pgfplotsset{compat=default}| to restore the factory settings.

	The setting |\pgfplotsset{compat=newest}| will always use features of the most recent version. This might result in small changes in the document's appearance.
\end{pgfplotskey}

\subsection{The Team}
\PGFPlots\ has been written mainly by Christian Feuers�nger with many improvements of Pascal Wolkotte and Nick Papior Andersen as a spare time project. We hope it is useful and provides valuable plots.

If you are interested in writing something but don't know how, consider reading the auxiliary manual \href{file:TeX-programming-notes.pdf}{TeX-programming-notes.pdf} which comes with \PGFPlots. It is far from complete, but maybe it is a good starting point (at least for more literature).

\subsection{Acknowledgements}
I thank God for all hours of enjoyed programming. I thank Pascal Wolkotte and Nick Papior Andersen for their programming efforts and contributions as part of the development team. I thank Stefan Tibus, who contributed the |plot shell| feature. I thank Tom Cashman for the contribution of the |reverse legend| feature. Special thanks go to Stefan Pinnow whose tests of \PGFPlots\ lead to numerous quality improvements. Furthermore, I thank Dr.~Schweitzer for many fruitful discussions and Dr.~Meine for his ideas and suggestions. Thanks as well to the many international contributors who provided feature requests or identified bugs or simply manual improvements!

Last but not least, I thank Till Tantau and Mark Wibrow for their excellent graphics (and more) package \PGF\ and \Tikz\ which is the base of \PGFPlots.

\include{pgfplots.install}%
\include{pgfplots.intro}%
\include{pgfplots.reference}%
\include{pgfplots.libs}%
\include{pgfplots.resources}%
\include{pgfplots.importexport}%
\include{pgfplots.basic.reference}%

\printindex

\bibliographystyle{abbrv} %gerapali} %gerabbrv} %gerunsrt.bst} %gerabbrv}% gerplain}
\nocite{pgfplotstable}
\nocite{programmingnotes}
\bibliography{pgfplots}
\end{document}
%
%%%%%%%%%%%%%%%%%%%%%%%%%%%%%%%%%%%%%%%%%%%%%%%%%%%%%%%%%%%%%%%%%%%%%%%%%%%%%
%
% Package pgfplots.sty documentation. 
%
% Copyright 2007/2008 by Christian Feuersaenger.
%
% This program is free software: you can redistribute it and/or modify
% it under the terms of the GNU General Public License as published by
% the Free Software Foundation, either version 3 of the License, or
% (at your option) any later version.
% 
% This program is distributed in the hope that it will be useful,
% but WITHOUT ANY WARRANTY; without even the implied warranty of
% MERCHANTABILITY or FITNESS FOR A PARTICULAR PURPOSE.  See the
% GNU General Public License for more details.
% 
% You should have received a copy of the GNU General Public License
% along with this program.  If not, see <http://www.gnu.org/licenses/>.
%
%
%%%%%%%%%%%%%%%%%%%%%%%%%%%%%%%%%%%%%%%%%%%%%%%%%%%%%%%%%%%%%%%%%%%%%%%%%%%%%
\input{pgfplots.preamble.tex}
%\RequirePackage[german,english,francais]{babel}

\pgfplotsmanualexternalexpensivetrue

%\usetikzlibrary{external}
%\tikzexternalize[prefix=figures/]{pgfplots}

\title{%
	Manual for Package \PGFPlots\\
	{\small Version \pgfplotsversion}\\
	{\small\href{http://sourceforge.net/projects/pgfplots}{http://sourceforge.net/projects/pgfplots}}}

%\includeonly{pgfplots.libs}


\begin{document}

\def\plotcoords{%
\addplot coordinates {
(5,8.312e-02)    (17,2.547e-02)   (49,7.407e-03)
(129,2.102e-03)  (321,5.874e-04)  (769,1.623e-04)
(1793,4.442e-05) (4097,1.207e-05) (9217,3.261e-06)
};

\addplot coordinates{
(7,8.472e-02)    (31,3.044e-02)    (111,1.022e-02)
(351,3.303e-03)  (1023,1.039e-03)  (2815,3.196e-04)
(7423,9.658e-05) (18943,2.873e-05) (47103,8.437e-06)
};

\addplot coordinates{
(9,7.881e-02)     (49,3.243e-02)    (209,1.232e-02)
(769,4.454e-03)   (2561,1.551e-03)  (7937,5.236e-04)
(23297,1.723e-04) (65537,5.545e-05) (178177,1.751e-05)
};

\addplot coordinates{
(11,6.887e-02)    (71,3.177e-02)     (351,1.341e-02)
(1471,5.334e-03)  (5503,2.027e-03)   (18943,7.415e-04)
(61183,2.628e-04) (187903,9.063e-05) (553983,3.053e-05)
};

\addplot coordinates{
(13,5.755e-02)     (97,2.925e-02)     (545,1.351e-02)
(2561,5.842e-03)   (10625,2.397e-03)  (40193,9.414e-04)
(141569,3.564e-04) (471041,1.308e-04) 
(1496065,4.670e-05)
};
}%
\maketitle
\begin{abstract}%
\PGFPlots\ draws high--quality function plots in normal or logarithmic scaling with a user-friendly interface directly in \TeX. The user supplies axis labels, legend entries and the plot coordinates for one or more plots and \PGFPlots\ applies axis scaling, computes any logarithms and axis ticks and draws the plots, supporting line plots, scatter plots, piecewise constant plots, bar plots, area plots, mesh-- and surface plots and some more. It is based on Till Tantau's package \PGF/\Tikz.
\end{abstract}
\tableofcontents
\section{Introduction}
This package provides tools to generate plots and labeled axes easily. It draws normal plots, logplots and semi-logplots, in two and three dimensions. Axis ticks, labels, legends (in case of multiple plots) can be added with key-value options. It can cycle through a set of predefined line/marker/color specifications. In summary, its purpose is to simplify the generation of high-quality function and/or data plots, and solving the problems of
\begin{itemize}
	\item consistency of document and font type and font size,
	\item direct use of \TeX\ math mode in axis descriptions,
	\item consistency of data and figures (no third party tool necessary),
	\item inter document consistency using preamble configurations and styles.
\end{itemize}
Although not necessary, separate |.pdf| or |.eps| graphics can be generated using the |external| library developed as part of \Tikz.

\PGFPlots\ is build completely on \Tikz/\PGF. Knowledge of \Tikz\ will simplify the work with \PGFPlots, although it is not required.

\section[About PGFPlots: Preliminaries]{About {\normalfont\PGFPlots}: Preliminaries}
This section contains information about upgrades, the team, the installation (in case you need to do it manually) and troubleshooting. You may skip it completely except for the upgrade remarks.

However, note that this library requires at least \PGF\ version $2.00$. At the time of this writing, many \TeX-distributions still contain the older \PGF\ version $1.18$, so it may be necessary to install a recent \PGF\ prior to using \PGFPlots.

\subsection{Components}
\PGFPlots\ comes with two components:
\begin{enumerate}
	\item the plotting component (which you are currently reading) and
	\item the \PGFPlotstable\ component which simplifies number formatting and postprocessing of numerical tables. It comes as a separate package and has its own manual \href{file:pgfplotstable.pdf}{pgfplotstable.pdf}.
\end{enumerate}

\subsection{Upgrade remarks}
This release provides a lot of improvements which can be found in all detail in \texttt{ChangeLog} for interested readers. However, some attention is useful with respect to the following changes.

\subsubsection{New Optional Features}
\begin{enumerate}
	\item \PGFPlots\ 1.3 comes with user interface improvements. The technical distinction between ``behavior options'' and ``style options'' of older versions is no longer necessary (although still fully supported).

	\item \PGFPlots\ 1.3 has a new feature which allows to \emph{move axis labels tight to tick labels} automatically.

	Since this affects the spacing, it is not enabled be default.

	Use
\begin{codeexample}[code only]
\usepackage{pgfplots}
\pgfplotsset{compat=1.3}
\end{codeexample}
	in your preamble to benefit from the improved spacing\footnote{The |compat=1.3| setting is necessary for new features which move axis labels around like reversed axis. However, \PGFPlots\ will still work as it does in earlier versions, your documents will remain the same if you don't set it explicitly.}.
	Take a look at the next page for the precise definition of |compat|.

	\item \PGFPlots\ 1.3 now supports reversed axes. It is no longer necessary to use work--arounds with negative units.
\pgfkeys{/pdflinks/search key prefixes in/.add={/pgfplots/,}{}}

	Take a look at the |x dir=reverse| key.

	Existing work--arounds will still function properly. Use |\pgfplotsset{compat=1.3}| together with |x dir=reverse| to switch to the new version.
\end{enumerate}

\subsubsection{Old Features Which May Need Attention}
\begin{enumerate}
	\item The |scatter/classes| feature produces proper legends as of version 1.3. This may change the appearance of existing legends of plots with |scatter/classes|.

	\item Due to a small math bug in \PGF\ $2.00$, you \emph{can't} use the math expression `|-x^2|'. It is necessary to use `|0-x^2|' instead. The same holds for
		`|exp(-x^2)|'; use `|exp(0-x^2)|' instead.

		This will be fixed with \PGF\ $>2.00$.

	\item Starting with \PGFPlots\ $1.1$, |\tikzstyle| should \emph{no longer be used} to set \PGFPlots\ options.
	
	Although |\tikzstyle| is still supported for some older \PGFPlots\ options, you should replace any occurance of |\tikzstyle| with |\pgfplotsset{|\meta{style name}|/.style={|\meta{key-value-list}|}}| or the associated |/.append style| variant. See section~\ref{sec:styles} for more detail.
\end{enumerate}
I apologize for any inconvenience caused by these changes.

\begin{pgfplotskey}{compat=\mchoice{1.3,pre 1.3,default,newest} (initially default)}
	The preamble configuration 
\begin{codeexample}[code only]
\usepackage{pgfplots}
\pgfplotsset{compat=1.3}
\end{codeexample}
	allows to choose between backwards compatibility and most recent features.

	\PGFPlots\ version 1.3 comes with new features which may lead to different spacings compared to earlier versions:
	\begin{enumerate}
		\item Version 1.3 comes with the |xlabel near ticks| feature which places (in this case $x$) axis labels ``near'' the tick labels, taking the size of tick labels into account.
		
		Older versions used |xlabel absolute|, i.e.\ an absolute distance between the axis and axis labels (now matter how large or small tick labels are).

		The initial setting \emph{keeps backwards compatibility}.

		You are encouraged to use
\begin{codeexample}[code only]
\usepackage{pgfplots}
\pgfplotsset{compat=1.3}
\end{codeexample}
		in your preamble.
		
		\item Version 1.3 fixes a bug with |anchor=south west| which occurred mainly for reversed axes: the anchors where upside--down. This is fixed now.

		
		If you have written some work--around and want to keep it going, use

		|\pgfplotsset{compat/anchors=pre 1.3}|\pgfmanualpdflabel{/pgfplots/compat/anchors}{} 

		to disable the bug fix.
	\end{enumerate}

	Use |\pgfplotsset{compat=default}| to restore the factory settings.

	The setting |\pgfplotsset{compat=newest}| will always use features of the most recent version. This might result in small changes in the document's appearance.
\end{pgfplotskey}

\subsection{The Team}
\PGFPlots\ has been written mainly by Christian Feuers�nger with many improvements of Pascal Wolkotte and Nick Papior Andersen as a spare time project. We hope it is useful and provides valuable plots.

If you are interested in writing something but don't know how, consider reading the auxiliary manual \href{file:TeX-programming-notes.pdf}{TeX-programming-notes.pdf} which comes with \PGFPlots. It is far from complete, but maybe it is a good starting point (at least for more literature).

\subsection{Acknowledgements}
I thank God for all hours of enjoyed programming. I thank Pascal Wolkotte and Nick Papior Andersen for their programming efforts and contributions as part of the development team. I thank Stefan Tibus, who contributed the |plot shell| feature. I thank Tom Cashman for the contribution of the |reverse legend| feature. Special thanks go to Stefan Pinnow whose tests of \PGFPlots\ lead to numerous quality improvements. Furthermore, I thank Dr.~Schweitzer for many fruitful discussions and Dr.~Meine for his ideas and suggestions. Thanks as well to the many international contributors who provided feature requests or identified bugs or simply manual improvements!

Last but not least, I thank Till Tantau and Mark Wibrow for their excellent graphics (and more) package \PGF\ and \Tikz\ which is the base of \PGFPlots.

\include{pgfplots.install}%
\include{pgfplots.intro}%
\include{pgfplots.reference}%
\include{pgfplots.libs}%
\include{pgfplots.resources}%
\include{pgfplots.importexport}%
\include{pgfplots.basic.reference}%

\printindex

\bibliographystyle{abbrv} %gerapali} %gerabbrv} %gerunsrt.bst} %gerabbrv}% gerplain}
\nocite{pgfplotstable}
\nocite{programmingnotes}
\bibliography{pgfplots}
\end{document}
%
%%%%%%%%%%%%%%%%%%%%%%%%%%%%%%%%%%%%%%%%%%%%%%%%%%%%%%%%%%%%%%%%%%%%%%%%%%%%%
%
% Package pgfplots.sty documentation. 
%
% Copyright 2007/2008 by Christian Feuersaenger.
%
% This program is free software: you can redistribute it and/or modify
% it under the terms of the GNU General Public License as published by
% the Free Software Foundation, either version 3 of the License, or
% (at your option) any later version.
% 
% This program is distributed in the hope that it will be useful,
% but WITHOUT ANY WARRANTY; without even the implied warranty of
% MERCHANTABILITY or FITNESS FOR A PARTICULAR PURPOSE.  See the
% GNU General Public License for more details.
% 
% You should have received a copy of the GNU General Public License
% along with this program.  If not, see <http://www.gnu.org/licenses/>.
%
%
%%%%%%%%%%%%%%%%%%%%%%%%%%%%%%%%%%%%%%%%%%%%%%%%%%%%%%%%%%%%%%%%%%%%%%%%%%%%%
\input{pgfplots.preamble.tex}
%\RequirePackage[german,english,francais]{babel}

\pgfplotsmanualexternalexpensivetrue

%\usetikzlibrary{external}
%\tikzexternalize[prefix=figures/]{pgfplots}

\title{%
	Manual for Package \PGFPlots\\
	{\small Version \pgfplotsversion}\\
	{\small\href{http://sourceforge.net/projects/pgfplots}{http://sourceforge.net/projects/pgfplots}}}

%\includeonly{pgfplots.libs}


\begin{document}

\def\plotcoords{%
\addplot coordinates {
(5,8.312e-02)    (17,2.547e-02)   (49,7.407e-03)
(129,2.102e-03)  (321,5.874e-04)  (769,1.623e-04)
(1793,4.442e-05) (4097,1.207e-05) (9217,3.261e-06)
};

\addplot coordinates{
(7,8.472e-02)    (31,3.044e-02)    (111,1.022e-02)
(351,3.303e-03)  (1023,1.039e-03)  (2815,3.196e-04)
(7423,9.658e-05) (18943,2.873e-05) (47103,8.437e-06)
};

\addplot coordinates{
(9,7.881e-02)     (49,3.243e-02)    (209,1.232e-02)
(769,4.454e-03)   (2561,1.551e-03)  (7937,5.236e-04)
(23297,1.723e-04) (65537,5.545e-05) (178177,1.751e-05)
};

\addplot coordinates{
(11,6.887e-02)    (71,3.177e-02)     (351,1.341e-02)
(1471,5.334e-03)  (5503,2.027e-03)   (18943,7.415e-04)
(61183,2.628e-04) (187903,9.063e-05) (553983,3.053e-05)
};

\addplot coordinates{
(13,5.755e-02)     (97,2.925e-02)     (545,1.351e-02)
(2561,5.842e-03)   (10625,2.397e-03)  (40193,9.414e-04)
(141569,3.564e-04) (471041,1.308e-04) 
(1496065,4.670e-05)
};
}%
\maketitle
\begin{abstract}%
\PGFPlots\ draws high--quality function plots in normal or logarithmic scaling with a user-friendly interface directly in \TeX. The user supplies axis labels, legend entries and the plot coordinates for one or more plots and \PGFPlots\ applies axis scaling, computes any logarithms and axis ticks and draws the plots, supporting line plots, scatter plots, piecewise constant plots, bar plots, area plots, mesh-- and surface plots and some more. It is based on Till Tantau's package \PGF/\Tikz.
\end{abstract}
\tableofcontents
\section{Introduction}
This package provides tools to generate plots and labeled axes easily. It draws normal plots, logplots and semi-logplots, in two and three dimensions. Axis ticks, labels, legends (in case of multiple plots) can be added with key-value options. It can cycle through a set of predefined line/marker/color specifications. In summary, its purpose is to simplify the generation of high-quality function and/or data plots, and solving the problems of
\begin{itemize}
	\item consistency of document and font type and font size,
	\item direct use of \TeX\ math mode in axis descriptions,
	\item consistency of data and figures (no third party tool necessary),
	\item inter document consistency using preamble configurations and styles.
\end{itemize}
Although not necessary, separate |.pdf| or |.eps| graphics can be generated using the |external| library developed as part of \Tikz.

\PGFPlots\ is build completely on \Tikz/\PGF. Knowledge of \Tikz\ will simplify the work with \PGFPlots, although it is not required.

\section[About PGFPlots: Preliminaries]{About {\normalfont\PGFPlots}: Preliminaries}
This section contains information about upgrades, the team, the installation (in case you need to do it manually) and troubleshooting. You may skip it completely except for the upgrade remarks.

However, note that this library requires at least \PGF\ version $2.00$. At the time of this writing, many \TeX-distributions still contain the older \PGF\ version $1.18$, so it may be necessary to install a recent \PGF\ prior to using \PGFPlots.

\subsection{Components}
\PGFPlots\ comes with two components:
\begin{enumerate}
	\item the plotting component (which you are currently reading) and
	\item the \PGFPlotstable\ component which simplifies number formatting and postprocessing of numerical tables. It comes as a separate package and has its own manual \href{file:pgfplotstable.pdf}{pgfplotstable.pdf}.
\end{enumerate}

\subsection{Upgrade remarks}
This release provides a lot of improvements which can be found in all detail in \texttt{ChangeLog} for interested readers. However, some attention is useful with respect to the following changes.

\subsubsection{New Optional Features}
\begin{enumerate}
	\item \PGFPlots\ 1.3 comes with user interface improvements. The technical distinction between ``behavior options'' and ``style options'' of older versions is no longer necessary (although still fully supported).

	\item \PGFPlots\ 1.3 has a new feature which allows to \emph{move axis labels tight to tick labels} automatically.

	Since this affects the spacing, it is not enabled be default.

	Use
\begin{codeexample}[code only]
\usepackage{pgfplots}
\pgfplotsset{compat=1.3}
\end{codeexample}
	in your preamble to benefit from the improved spacing\footnote{The |compat=1.3| setting is necessary for new features which move axis labels around like reversed axis. However, \PGFPlots\ will still work as it does in earlier versions, your documents will remain the same if you don't set it explicitly.}.
	Take a look at the next page for the precise definition of |compat|.

	\item \PGFPlots\ 1.3 now supports reversed axes. It is no longer necessary to use work--arounds with negative units.
\pgfkeys{/pdflinks/search key prefixes in/.add={/pgfplots/,}{}}

	Take a look at the |x dir=reverse| key.

	Existing work--arounds will still function properly. Use |\pgfplotsset{compat=1.3}| together with |x dir=reverse| to switch to the new version.
\end{enumerate}

\subsubsection{Old Features Which May Need Attention}
\begin{enumerate}
	\item The |scatter/classes| feature produces proper legends as of version 1.3. This may change the appearance of existing legends of plots with |scatter/classes|.

	\item Due to a small math bug in \PGF\ $2.00$, you \emph{can't} use the math expression `|-x^2|'. It is necessary to use `|0-x^2|' instead. The same holds for
		`|exp(-x^2)|'; use `|exp(0-x^2)|' instead.

		This will be fixed with \PGF\ $>2.00$.

	\item Starting with \PGFPlots\ $1.1$, |\tikzstyle| should \emph{no longer be used} to set \PGFPlots\ options.
	
	Although |\tikzstyle| is still supported for some older \PGFPlots\ options, you should replace any occurance of |\tikzstyle| with |\pgfplotsset{|\meta{style name}|/.style={|\meta{key-value-list}|}}| or the associated |/.append style| variant. See section~\ref{sec:styles} for more detail.
\end{enumerate}
I apologize for any inconvenience caused by these changes.

\begin{pgfplotskey}{compat=\mchoice{1.3,pre 1.3,default,newest} (initially default)}
	The preamble configuration 
\begin{codeexample}[code only]
\usepackage{pgfplots}
\pgfplotsset{compat=1.3}
\end{codeexample}
	allows to choose between backwards compatibility and most recent features.

	\PGFPlots\ version 1.3 comes with new features which may lead to different spacings compared to earlier versions:
	\begin{enumerate}
		\item Version 1.3 comes with the |xlabel near ticks| feature which places (in this case $x$) axis labels ``near'' the tick labels, taking the size of tick labels into account.
		
		Older versions used |xlabel absolute|, i.e.\ an absolute distance between the axis and axis labels (now matter how large or small tick labels are).

		The initial setting \emph{keeps backwards compatibility}.

		You are encouraged to use
\begin{codeexample}[code only]
\usepackage{pgfplots}
\pgfplotsset{compat=1.3}
\end{codeexample}
		in your preamble.
		
		\item Version 1.3 fixes a bug with |anchor=south west| which occurred mainly for reversed axes: the anchors where upside--down. This is fixed now.

		
		If you have written some work--around and want to keep it going, use

		|\pgfplotsset{compat/anchors=pre 1.3}|\pgfmanualpdflabel{/pgfplots/compat/anchors}{} 

		to disable the bug fix.
	\end{enumerate}

	Use |\pgfplotsset{compat=default}| to restore the factory settings.

	The setting |\pgfplotsset{compat=newest}| will always use features of the most recent version. This might result in small changes in the document's appearance.
\end{pgfplotskey}

\subsection{The Team}
\PGFPlots\ has been written mainly by Christian Feuers�nger with many improvements of Pascal Wolkotte and Nick Papior Andersen as a spare time project. We hope it is useful and provides valuable plots.

If you are interested in writing something but don't know how, consider reading the auxiliary manual \href{file:TeX-programming-notes.pdf}{TeX-programming-notes.pdf} which comes with \PGFPlots. It is far from complete, but maybe it is a good starting point (at least for more literature).

\subsection{Acknowledgements}
I thank God for all hours of enjoyed programming. I thank Pascal Wolkotte and Nick Papior Andersen for their programming efforts and contributions as part of the development team. I thank Stefan Tibus, who contributed the |plot shell| feature. I thank Tom Cashman for the contribution of the |reverse legend| feature. Special thanks go to Stefan Pinnow whose tests of \PGFPlots\ lead to numerous quality improvements. Furthermore, I thank Dr.~Schweitzer for many fruitful discussions and Dr.~Meine for his ideas and suggestions. Thanks as well to the many international contributors who provided feature requests or identified bugs or simply manual improvements!

Last but not least, I thank Till Tantau and Mark Wibrow for their excellent graphics (and more) package \PGF\ and \Tikz\ which is the base of \PGFPlots.

\include{pgfplots.install}%
\include{pgfplots.intro}%
\include{pgfplots.reference}%
\include{pgfplots.libs}%
\include{pgfplots.resources}%
\include{pgfplots.importexport}%
\include{pgfplots.basic.reference}%

\printindex

\bibliographystyle{abbrv} %gerapali} %gerabbrv} %gerunsrt.bst} %gerabbrv}% gerplain}
\nocite{pgfplotstable}
\nocite{programmingnotes}
\bibliography{pgfplots}
\end{document}
%
%%%%%%%%%%%%%%%%%%%%%%%%%%%%%%%%%%%%%%%%%%%%%%%%%%%%%%%%%%%%%%%%%%%%%%%%%%%%%
%
% Package pgfplots.sty documentation. 
%
% Copyright 2007/2008 by Christian Feuersaenger.
%
% This program is free software: you can redistribute it and/or modify
% it under the terms of the GNU General Public License as published by
% the Free Software Foundation, either version 3 of the License, or
% (at your option) any later version.
% 
% This program is distributed in the hope that it will be useful,
% but WITHOUT ANY WARRANTY; without even the implied warranty of
% MERCHANTABILITY or FITNESS FOR A PARTICULAR PURPOSE.  See the
% GNU General Public License for more details.
% 
% You should have received a copy of the GNU General Public License
% along with this program.  If not, see <http://www.gnu.org/licenses/>.
%
%
%%%%%%%%%%%%%%%%%%%%%%%%%%%%%%%%%%%%%%%%%%%%%%%%%%%%%%%%%%%%%%%%%%%%%%%%%%%%%
\input{pgfplots.preamble.tex}
%\RequirePackage[german,english,francais]{babel}

\pgfplotsmanualexternalexpensivetrue

%\usetikzlibrary{external}
%\tikzexternalize[prefix=figures/]{pgfplots}

\title{%
	Manual for Package \PGFPlots\\
	{\small Version \pgfplotsversion}\\
	{\small\href{http://sourceforge.net/projects/pgfplots}{http://sourceforge.net/projects/pgfplots}}}

%\includeonly{pgfplots.libs}


\begin{document}

\def\plotcoords{%
\addplot coordinates {
(5,8.312e-02)    (17,2.547e-02)   (49,7.407e-03)
(129,2.102e-03)  (321,5.874e-04)  (769,1.623e-04)
(1793,4.442e-05) (4097,1.207e-05) (9217,3.261e-06)
};

\addplot coordinates{
(7,8.472e-02)    (31,3.044e-02)    (111,1.022e-02)
(351,3.303e-03)  (1023,1.039e-03)  (2815,3.196e-04)
(7423,9.658e-05) (18943,2.873e-05) (47103,8.437e-06)
};

\addplot coordinates{
(9,7.881e-02)     (49,3.243e-02)    (209,1.232e-02)
(769,4.454e-03)   (2561,1.551e-03)  (7937,5.236e-04)
(23297,1.723e-04) (65537,5.545e-05) (178177,1.751e-05)
};

\addplot coordinates{
(11,6.887e-02)    (71,3.177e-02)     (351,1.341e-02)
(1471,5.334e-03)  (5503,2.027e-03)   (18943,7.415e-04)
(61183,2.628e-04) (187903,9.063e-05) (553983,3.053e-05)
};

\addplot coordinates{
(13,5.755e-02)     (97,2.925e-02)     (545,1.351e-02)
(2561,5.842e-03)   (10625,2.397e-03)  (40193,9.414e-04)
(141569,3.564e-04) (471041,1.308e-04) 
(1496065,4.670e-05)
};
}%
\maketitle
\begin{abstract}%
\PGFPlots\ draws high--quality function plots in normal or logarithmic scaling with a user-friendly interface directly in \TeX. The user supplies axis labels, legend entries and the plot coordinates for one or more plots and \PGFPlots\ applies axis scaling, computes any logarithms and axis ticks and draws the plots, supporting line plots, scatter plots, piecewise constant plots, bar plots, area plots, mesh-- and surface plots and some more. It is based on Till Tantau's package \PGF/\Tikz.
\end{abstract}
\tableofcontents
\section{Introduction}
This package provides tools to generate plots and labeled axes easily. It draws normal plots, logplots and semi-logplots, in two and three dimensions. Axis ticks, labels, legends (in case of multiple plots) can be added with key-value options. It can cycle through a set of predefined line/marker/color specifications. In summary, its purpose is to simplify the generation of high-quality function and/or data plots, and solving the problems of
\begin{itemize}
	\item consistency of document and font type and font size,
	\item direct use of \TeX\ math mode in axis descriptions,
	\item consistency of data and figures (no third party tool necessary),
	\item inter document consistency using preamble configurations and styles.
\end{itemize}
Although not necessary, separate |.pdf| or |.eps| graphics can be generated using the |external| library developed as part of \Tikz.

\PGFPlots\ is build completely on \Tikz/\PGF. Knowledge of \Tikz\ will simplify the work with \PGFPlots, although it is not required.

\section[About PGFPlots: Preliminaries]{About {\normalfont\PGFPlots}: Preliminaries}
This section contains information about upgrades, the team, the installation (in case you need to do it manually) and troubleshooting. You may skip it completely except for the upgrade remarks.

However, note that this library requires at least \PGF\ version $2.00$. At the time of this writing, many \TeX-distributions still contain the older \PGF\ version $1.18$, so it may be necessary to install a recent \PGF\ prior to using \PGFPlots.

\subsection{Components}
\PGFPlots\ comes with two components:
\begin{enumerate}
	\item the plotting component (which you are currently reading) and
	\item the \PGFPlotstable\ component which simplifies number formatting and postprocessing of numerical tables. It comes as a separate package and has its own manual \href{file:pgfplotstable.pdf}{pgfplotstable.pdf}.
\end{enumerate}

\subsection{Upgrade remarks}
This release provides a lot of improvements which can be found in all detail in \texttt{ChangeLog} for interested readers. However, some attention is useful with respect to the following changes.

\subsubsection{New Optional Features}
\begin{enumerate}
	\item \PGFPlots\ 1.3 comes with user interface improvements. The technical distinction between ``behavior options'' and ``style options'' of older versions is no longer necessary (although still fully supported).

	\item \PGFPlots\ 1.3 has a new feature which allows to \emph{move axis labels tight to tick labels} automatically.

	Since this affects the spacing, it is not enabled be default.

	Use
\begin{codeexample}[code only]
\usepackage{pgfplots}
\pgfplotsset{compat=1.3}
\end{codeexample}
	in your preamble to benefit from the improved spacing\footnote{The |compat=1.3| setting is necessary for new features which move axis labels around like reversed axis. However, \PGFPlots\ will still work as it does in earlier versions, your documents will remain the same if you don't set it explicitly.}.
	Take a look at the next page for the precise definition of |compat|.

	\item \PGFPlots\ 1.3 now supports reversed axes. It is no longer necessary to use work--arounds with negative units.
\pgfkeys{/pdflinks/search key prefixes in/.add={/pgfplots/,}{}}

	Take a look at the |x dir=reverse| key.

	Existing work--arounds will still function properly. Use |\pgfplotsset{compat=1.3}| together with |x dir=reverse| to switch to the new version.
\end{enumerate}

\subsubsection{Old Features Which May Need Attention}
\begin{enumerate}
	\item The |scatter/classes| feature produces proper legends as of version 1.3. This may change the appearance of existing legends of plots with |scatter/classes|.

	\item Due to a small math bug in \PGF\ $2.00$, you \emph{can't} use the math expression `|-x^2|'. It is necessary to use `|0-x^2|' instead. The same holds for
		`|exp(-x^2)|'; use `|exp(0-x^2)|' instead.

		This will be fixed with \PGF\ $>2.00$.

	\item Starting with \PGFPlots\ $1.1$, |\tikzstyle| should \emph{no longer be used} to set \PGFPlots\ options.
	
	Although |\tikzstyle| is still supported for some older \PGFPlots\ options, you should replace any occurance of |\tikzstyle| with |\pgfplotsset{|\meta{style name}|/.style={|\meta{key-value-list}|}}| or the associated |/.append style| variant. See section~\ref{sec:styles} for more detail.
\end{enumerate}
I apologize for any inconvenience caused by these changes.

\begin{pgfplotskey}{compat=\mchoice{1.3,pre 1.3,default,newest} (initially default)}
	The preamble configuration 
\begin{codeexample}[code only]
\usepackage{pgfplots}
\pgfplotsset{compat=1.3}
\end{codeexample}
	allows to choose between backwards compatibility and most recent features.

	\PGFPlots\ version 1.3 comes with new features which may lead to different spacings compared to earlier versions:
	\begin{enumerate}
		\item Version 1.3 comes with the |xlabel near ticks| feature which places (in this case $x$) axis labels ``near'' the tick labels, taking the size of tick labels into account.
		
		Older versions used |xlabel absolute|, i.e.\ an absolute distance between the axis and axis labels (now matter how large or small tick labels are).

		The initial setting \emph{keeps backwards compatibility}.

		You are encouraged to use
\begin{codeexample}[code only]
\usepackage{pgfplots}
\pgfplotsset{compat=1.3}
\end{codeexample}
		in your preamble.
		
		\item Version 1.3 fixes a bug with |anchor=south west| which occurred mainly for reversed axes: the anchors where upside--down. This is fixed now.

		
		If you have written some work--around and want to keep it going, use

		|\pgfplotsset{compat/anchors=pre 1.3}|\pgfmanualpdflabel{/pgfplots/compat/anchors}{} 

		to disable the bug fix.
	\end{enumerate}

	Use |\pgfplotsset{compat=default}| to restore the factory settings.

	The setting |\pgfplotsset{compat=newest}| will always use features of the most recent version. This might result in small changes in the document's appearance.
\end{pgfplotskey}

\subsection{The Team}
\PGFPlots\ has been written mainly by Christian Feuers�nger with many improvements of Pascal Wolkotte and Nick Papior Andersen as a spare time project. We hope it is useful and provides valuable plots.

If you are interested in writing something but don't know how, consider reading the auxiliary manual \href{file:TeX-programming-notes.pdf}{TeX-programming-notes.pdf} which comes with \PGFPlots. It is far from complete, but maybe it is a good starting point (at least for more literature).

\subsection{Acknowledgements}
I thank God for all hours of enjoyed programming. I thank Pascal Wolkotte and Nick Papior Andersen for their programming efforts and contributions as part of the development team. I thank Stefan Tibus, who contributed the |plot shell| feature. I thank Tom Cashman for the contribution of the |reverse legend| feature. Special thanks go to Stefan Pinnow whose tests of \PGFPlots\ lead to numerous quality improvements. Furthermore, I thank Dr.~Schweitzer for many fruitful discussions and Dr.~Meine for his ideas and suggestions. Thanks as well to the many international contributors who provided feature requests or identified bugs or simply manual improvements!

Last but not least, I thank Till Tantau and Mark Wibrow for their excellent graphics (and more) package \PGF\ and \Tikz\ which is the base of \PGFPlots.

\include{pgfplots.install}%
\include{pgfplots.intro}%
\include{pgfplots.reference}%
\include{pgfplots.libs}%
\include{pgfplots.resources}%
\include{pgfplots.importexport}%
\include{pgfplots.basic.reference}%

\printindex

\bibliographystyle{abbrv} %gerapali} %gerabbrv} %gerunsrt.bst} %gerabbrv}% gerplain}
\nocite{pgfplotstable}
\nocite{programmingnotes}
\bibliography{pgfplots}
\end{document}
%
%%%%%%%%%%%%%%%%%%%%%%%%%%%%%%%%%%%%%%%%%%%%%%%%%%%%%%%%%%%%%%%%%%%%%%%%%%%%%
%
% Package pgfplots.sty documentation. 
%
% Copyright 2007/2008 by Christian Feuersaenger.
%
% This program is free software: you can redistribute it and/or modify
% it under the terms of the GNU General Public License as published by
% the Free Software Foundation, either version 3 of the License, or
% (at your option) any later version.
% 
% This program is distributed in the hope that it will be useful,
% but WITHOUT ANY WARRANTY; without even the implied warranty of
% MERCHANTABILITY or FITNESS FOR A PARTICULAR PURPOSE.  See the
% GNU General Public License for more details.
% 
% You should have received a copy of the GNU General Public License
% along with this program.  If not, see <http://www.gnu.org/licenses/>.
%
%
%%%%%%%%%%%%%%%%%%%%%%%%%%%%%%%%%%%%%%%%%%%%%%%%%%%%%%%%%%%%%%%%%%%%%%%%%%%%%
\input{pgfplots.preamble.tex}
%\RequirePackage[german,english,francais]{babel}

\pgfplotsmanualexternalexpensivetrue

%\usetikzlibrary{external}
%\tikzexternalize[prefix=figures/]{pgfplots}

\title{%
	Manual for Package \PGFPlots\\
	{\small Version \pgfplotsversion}\\
	{\small\href{http://sourceforge.net/projects/pgfplots}{http://sourceforge.net/projects/pgfplots}}}

%\includeonly{pgfplots.libs}


\begin{document}

\def\plotcoords{%
\addplot coordinates {
(5,8.312e-02)    (17,2.547e-02)   (49,7.407e-03)
(129,2.102e-03)  (321,5.874e-04)  (769,1.623e-04)
(1793,4.442e-05) (4097,1.207e-05) (9217,3.261e-06)
};

\addplot coordinates{
(7,8.472e-02)    (31,3.044e-02)    (111,1.022e-02)
(351,3.303e-03)  (1023,1.039e-03)  (2815,3.196e-04)
(7423,9.658e-05) (18943,2.873e-05) (47103,8.437e-06)
};

\addplot coordinates{
(9,7.881e-02)     (49,3.243e-02)    (209,1.232e-02)
(769,4.454e-03)   (2561,1.551e-03)  (7937,5.236e-04)
(23297,1.723e-04) (65537,5.545e-05) (178177,1.751e-05)
};

\addplot coordinates{
(11,6.887e-02)    (71,3.177e-02)     (351,1.341e-02)
(1471,5.334e-03)  (5503,2.027e-03)   (18943,7.415e-04)
(61183,2.628e-04) (187903,9.063e-05) (553983,3.053e-05)
};

\addplot coordinates{
(13,5.755e-02)     (97,2.925e-02)     (545,1.351e-02)
(2561,5.842e-03)   (10625,2.397e-03)  (40193,9.414e-04)
(141569,3.564e-04) (471041,1.308e-04) 
(1496065,4.670e-05)
};
}%
\maketitle
\begin{abstract}%
\PGFPlots\ draws high--quality function plots in normal or logarithmic scaling with a user-friendly interface directly in \TeX. The user supplies axis labels, legend entries and the plot coordinates for one or more plots and \PGFPlots\ applies axis scaling, computes any logarithms and axis ticks and draws the plots, supporting line plots, scatter plots, piecewise constant plots, bar plots, area plots, mesh-- and surface plots and some more. It is based on Till Tantau's package \PGF/\Tikz.
\end{abstract}
\tableofcontents
\section{Introduction}
This package provides tools to generate plots and labeled axes easily. It draws normal plots, logplots and semi-logplots, in two and three dimensions. Axis ticks, labels, legends (in case of multiple plots) can be added with key-value options. It can cycle through a set of predefined line/marker/color specifications. In summary, its purpose is to simplify the generation of high-quality function and/or data plots, and solving the problems of
\begin{itemize}
	\item consistency of document and font type and font size,
	\item direct use of \TeX\ math mode in axis descriptions,
	\item consistency of data and figures (no third party tool necessary),
	\item inter document consistency using preamble configurations and styles.
\end{itemize}
Although not necessary, separate |.pdf| or |.eps| graphics can be generated using the |external| library developed as part of \Tikz.

\PGFPlots\ is build completely on \Tikz/\PGF. Knowledge of \Tikz\ will simplify the work with \PGFPlots, although it is not required.

\section[About PGFPlots: Preliminaries]{About {\normalfont\PGFPlots}: Preliminaries}
This section contains information about upgrades, the team, the installation (in case you need to do it manually) and troubleshooting. You may skip it completely except for the upgrade remarks.

However, note that this library requires at least \PGF\ version $2.00$. At the time of this writing, many \TeX-distributions still contain the older \PGF\ version $1.18$, so it may be necessary to install a recent \PGF\ prior to using \PGFPlots.

\subsection{Components}
\PGFPlots\ comes with two components:
\begin{enumerate}
	\item the plotting component (which you are currently reading) and
	\item the \PGFPlotstable\ component which simplifies number formatting and postprocessing of numerical tables. It comes as a separate package and has its own manual \href{file:pgfplotstable.pdf}{pgfplotstable.pdf}.
\end{enumerate}

\subsection{Upgrade remarks}
This release provides a lot of improvements which can be found in all detail in \texttt{ChangeLog} for interested readers. However, some attention is useful with respect to the following changes.

\subsubsection{New Optional Features}
\begin{enumerate}
	\item \PGFPlots\ 1.3 comes with user interface improvements. The technical distinction between ``behavior options'' and ``style options'' of older versions is no longer necessary (although still fully supported).

	\item \PGFPlots\ 1.3 has a new feature which allows to \emph{move axis labels tight to tick labels} automatically.

	Since this affects the spacing, it is not enabled be default.

	Use
\begin{codeexample}[code only]
\usepackage{pgfplots}
\pgfplotsset{compat=1.3}
\end{codeexample}
	in your preamble to benefit from the improved spacing\footnote{The |compat=1.3| setting is necessary for new features which move axis labels around like reversed axis. However, \PGFPlots\ will still work as it does in earlier versions, your documents will remain the same if you don't set it explicitly.}.
	Take a look at the next page for the precise definition of |compat|.

	\item \PGFPlots\ 1.3 now supports reversed axes. It is no longer necessary to use work--arounds with negative units.
\pgfkeys{/pdflinks/search key prefixes in/.add={/pgfplots/,}{}}

	Take a look at the |x dir=reverse| key.

	Existing work--arounds will still function properly. Use |\pgfplotsset{compat=1.3}| together with |x dir=reverse| to switch to the new version.
\end{enumerate}

\subsubsection{Old Features Which May Need Attention}
\begin{enumerate}
	\item The |scatter/classes| feature produces proper legends as of version 1.3. This may change the appearance of existing legends of plots with |scatter/classes|.

	\item Due to a small math bug in \PGF\ $2.00$, you \emph{can't} use the math expression `|-x^2|'. It is necessary to use `|0-x^2|' instead. The same holds for
		`|exp(-x^2)|'; use `|exp(0-x^2)|' instead.

		This will be fixed with \PGF\ $>2.00$.

	\item Starting with \PGFPlots\ $1.1$, |\tikzstyle| should \emph{no longer be used} to set \PGFPlots\ options.
	
	Although |\tikzstyle| is still supported for some older \PGFPlots\ options, you should replace any occurance of |\tikzstyle| with |\pgfplotsset{|\meta{style name}|/.style={|\meta{key-value-list}|}}| or the associated |/.append style| variant. See section~\ref{sec:styles} for more detail.
\end{enumerate}
I apologize for any inconvenience caused by these changes.

\begin{pgfplotskey}{compat=\mchoice{1.3,pre 1.3,default,newest} (initially default)}
	The preamble configuration 
\begin{codeexample}[code only]
\usepackage{pgfplots}
\pgfplotsset{compat=1.3}
\end{codeexample}
	allows to choose between backwards compatibility and most recent features.

	\PGFPlots\ version 1.3 comes with new features which may lead to different spacings compared to earlier versions:
	\begin{enumerate}
		\item Version 1.3 comes with the |xlabel near ticks| feature which places (in this case $x$) axis labels ``near'' the tick labels, taking the size of tick labels into account.
		
		Older versions used |xlabel absolute|, i.e.\ an absolute distance between the axis and axis labels (now matter how large or small tick labels are).

		The initial setting \emph{keeps backwards compatibility}.

		You are encouraged to use
\begin{codeexample}[code only]
\usepackage{pgfplots}
\pgfplotsset{compat=1.3}
\end{codeexample}
		in your preamble.
		
		\item Version 1.3 fixes a bug with |anchor=south west| which occurred mainly for reversed axes: the anchors where upside--down. This is fixed now.

		
		If you have written some work--around and want to keep it going, use

		|\pgfplotsset{compat/anchors=pre 1.3}|\pgfmanualpdflabel{/pgfplots/compat/anchors}{} 

		to disable the bug fix.
	\end{enumerate}

	Use |\pgfplotsset{compat=default}| to restore the factory settings.

	The setting |\pgfplotsset{compat=newest}| will always use features of the most recent version. This might result in small changes in the document's appearance.
\end{pgfplotskey}

\subsection{The Team}
\PGFPlots\ has been written mainly by Christian Feuers�nger with many improvements of Pascal Wolkotte and Nick Papior Andersen as a spare time project. We hope it is useful and provides valuable plots.

If you are interested in writing something but don't know how, consider reading the auxiliary manual \href{file:TeX-programming-notes.pdf}{TeX-programming-notes.pdf} which comes with \PGFPlots. It is far from complete, but maybe it is a good starting point (at least for more literature).

\subsection{Acknowledgements}
I thank God for all hours of enjoyed programming. I thank Pascal Wolkotte and Nick Papior Andersen for their programming efforts and contributions as part of the development team. I thank Stefan Tibus, who contributed the |plot shell| feature. I thank Tom Cashman for the contribution of the |reverse legend| feature. Special thanks go to Stefan Pinnow whose tests of \PGFPlots\ lead to numerous quality improvements. Furthermore, I thank Dr.~Schweitzer for many fruitful discussions and Dr.~Meine for his ideas and suggestions. Thanks as well to the many international contributors who provided feature requests or identified bugs or simply manual improvements!

Last but not least, I thank Till Tantau and Mark Wibrow for their excellent graphics (and more) package \PGF\ and \Tikz\ which is the base of \PGFPlots.

\include{pgfplots.install}%
\include{pgfplots.intro}%
\include{pgfplots.reference}%
\include{pgfplots.libs}%
\include{pgfplots.resources}%
\include{pgfplots.importexport}%
\include{pgfplots.basic.reference}%

\printindex

\bibliographystyle{abbrv} %gerapali} %gerabbrv} %gerunsrt.bst} %gerabbrv}% gerplain}
\nocite{pgfplotstable}
\nocite{programmingnotes}
\bibliography{pgfplots}
\end{document}
%
%%%%%%%%%%%%%%%%%%%%%%%%%%%%%%%%%%%%%%%%%%%%%%%%%%%%%%%%%%%%%%%%%%%%%%%%%%%%%
%
% Package pgfplots.sty documentation. 
%
% Copyright 2007/2008 by Christian Feuersaenger.
%
% This program is free software: you can redistribute it and/or modify
% it under the terms of the GNU General Public License as published by
% the Free Software Foundation, either version 3 of the License, or
% (at your option) any later version.
% 
% This program is distributed in the hope that it will be useful,
% but WITHOUT ANY WARRANTY; without even the implied warranty of
% MERCHANTABILITY or FITNESS FOR A PARTICULAR PURPOSE.  See the
% GNU General Public License for more details.
% 
% You should have received a copy of the GNU General Public License
% along with this program.  If not, see <http://www.gnu.org/licenses/>.
%
%
%%%%%%%%%%%%%%%%%%%%%%%%%%%%%%%%%%%%%%%%%%%%%%%%%%%%%%%%%%%%%%%%%%%%%%%%%%%%%
\input{pgfplots.preamble.tex}
%\RequirePackage[german,english,francais]{babel}

\pgfplotsmanualexternalexpensivetrue

%\usetikzlibrary{external}
%\tikzexternalize[prefix=figures/]{pgfplots}

\title{%
	Manual for Package \PGFPlots\\
	{\small Version \pgfplotsversion}\\
	{\small\href{http://sourceforge.net/projects/pgfplots}{http://sourceforge.net/projects/pgfplots}}}

%\includeonly{pgfplots.libs}


\begin{document}

\def\plotcoords{%
\addplot coordinates {
(5,8.312e-02)    (17,2.547e-02)   (49,7.407e-03)
(129,2.102e-03)  (321,5.874e-04)  (769,1.623e-04)
(1793,4.442e-05) (4097,1.207e-05) (9217,3.261e-06)
};

\addplot coordinates{
(7,8.472e-02)    (31,3.044e-02)    (111,1.022e-02)
(351,3.303e-03)  (1023,1.039e-03)  (2815,3.196e-04)
(7423,9.658e-05) (18943,2.873e-05) (47103,8.437e-06)
};

\addplot coordinates{
(9,7.881e-02)     (49,3.243e-02)    (209,1.232e-02)
(769,4.454e-03)   (2561,1.551e-03)  (7937,5.236e-04)
(23297,1.723e-04) (65537,5.545e-05) (178177,1.751e-05)
};

\addplot coordinates{
(11,6.887e-02)    (71,3.177e-02)     (351,1.341e-02)
(1471,5.334e-03)  (5503,2.027e-03)   (18943,7.415e-04)
(61183,2.628e-04) (187903,9.063e-05) (553983,3.053e-05)
};

\addplot coordinates{
(13,5.755e-02)     (97,2.925e-02)     (545,1.351e-02)
(2561,5.842e-03)   (10625,2.397e-03)  (40193,9.414e-04)
(141569,3.564e-04) (471041,1.308e-04) 
(1496065,4.670e-05)
};
}%
\maketitle
\begin{abstract}%
\PGFPlots\ draws high--quality function plots in normal or logarithmic scaling with a user-friendly interface directly in \TeX. The user supplies axis labels, legend entries and the plot coordinates for one or more plots and \PGFPlots\ applies axis scaling, computes any logarithms and axis ticks and draws the plots, supporting line plots, scatter plots, piecewise constant plots, bar plots, area plots, mesh-- and surface plots and some more. It is based on Till Tantau's package \PGF/\Tikz.
\end{abstract}
\tableofcontents
\section{Introduction}
This package provides tools to generate plots and labeled axes easily. It draws normal plots, logplots and semi-logplots, in two and three dimensions. Axis ticks, labels, legends (in case of multiple plots) can be added with key-value options. It can cycle through a set of predefined line/marker/color specifications. In summary, its purpose is to simplify the generation of high-quality function and/or data plots, and solving the problems of
\begin{itemize}
	\item consistency of document and font type and font size,
	\item direct use of \TeX\ math mode in axis descriptions,
	\item consistency of data and figures (no third party tool necessary),
	\item inter document consistency using preamble configurations and styles.
\end{itemize}
Although not necessary, separate |.pdf| or |.eps| graphics can be generated using the |external| library developed as part of \Tikz.

\PGFPlots\ is build completely on \Tikz/\PGF. Knowledge of \Tikz\ will simplify the work with \PGFPlots, although it is not required.

\section[About PGFPlots: Preliminaries]{About {\normalfont\PGFPlots}: Preliminaries}
This section contains information about upgrades, the team, the installation (in case you need to do it manually) and troubleshooting. You may skip it completely except for the upgrade remarks.

However, note that this library requires at least \PGF\ version $2.00$. At the time of this writing, many \TeX-distributions still contain the older \PGF\ version $1.18$, so it may be necessary to install a recent \PGF\ prior to using \PGFPlots.

\subsection{Components}
\PGFPlots\ comes with two components:
\begin{enumerate}
	\item the plotting component (which you are currently reading) and
	\item the \PGFPlotstable\ component which simplifies number formatting and postprocessing of numerical tables. It comes as a separate package and has its own manual \href{file:pgfplotstable.pdf}{pgfplotstable.pdf}.
\end{enumerate}

\subsection{Upgrade remarks}
This release provides a lot of improvements which can be found in all detail in \texttt{ChangeLog} for interested readers. However, some attention is useful with respect to the following changes.

\subsubsection{New Optional Features}
\begin{enumerate}
	\item \PGFPlots\ 1.3 comes with user interface improvements. The technical distinction between ``behavior options'' and ``style options'' of older versions is no longer necessary (although still fully supported).

	\item \PGFPlots\ 1.3 has a new feature which allows to \emph{move axis labels tight to tick labels} automatically.

	Since this affects the spacing, it is not enabled be default.

	Use
\begin{codeexample}[code only]
\usepackage{pgfplots}
\pgfplotsset{compat=1.3}
\end{codeexample}
	in your preamble to benefit from the improved spacing\footnote{The |compat=1.3| setting is necessary for new features which move axis labels around like reversed axis. However, \PGFPlots\ will still work as it does in earlier versions, your documents will remain the same if you don't set it explicitly.}.
	Take a look at the next page for the precise definition of |compat|.

	\item \PGFPlots\ 1.3 now supports reversed axes. It is no longer necessary to use work--arounds with negative units.
\pgfkeys{/pdflinks/search key prefixes in/.add={/pgfplots/,}{}}

	Take a look at the |x dir=reverse| key.

	Existing work--arounds will still function properly. Use |\pgfplotsset{compat=1.3}| together with |x dir=reverse| to switch to the new version.
\end{enumerate}

\subsubsection{Old Features Which May Need Attention}
\begin{enumerate}
	\item The |scatter/classes| feature produces proper legends as of version 1.3. This may change the appearance of existing legends of plots with |scatter/classes|.

	\item Due to a small math bug in \PGF\ $2.00$, you \emph{can't} use the math expression `|-x^2|'. It is necessary to use `|0-x^2|' instead. The same holds for
		`|exp(-x^2)|'; use `|exp(0-x^2)|' instead.

		This will be fixed with \PGF\ $>2.00$.

	\item Starting with \PGFPlots\ $1.1$, |\tikzstyle| should \emph{no longer be used} to set \PGFPlots\ options.
	
	Although |\tikzstyle| is still supported for some older \PGFPlots\ options, you should replace any occurance of |\tikzstyle| with |\pgfplotsset{|\meta{style name}|/.style={|\meta{key-value-list}|}}| or the associated |/.append style| variant. See section~\ref{sec:styles} for more detail.
\end{enumerate}
I apologize for any inconvenience caused by these changes.

\begin{pgfplotskey}{compat=\mchoice{1.3,pre 1.3,default,newest} (initially default)}
	The preamble configuration 
\begin{codeexample}[code only]
\usepackage{pgfplots}
\pgfplotsset{compat=1.3}
\end{codeexample}
	allows to choose between backwards compatibility and most recent features.

	\PGFPlots\ version 1.3 comes with new features which may lead to different spacings compared to earlier versions:
	\begin{enumerate}
		\item Version 1.3 comes with the |xlabel near ticks| feature which places (in this case $x$) axis labels ``near'' the tick labels, taking the size of tick labels into account.
		
		Older versions used |xlabel absolute|, i.e.\ an absolute distance between the axis and axis labels (now matter how large or small tick labels are).

		The initial setting \emph{keeps backwards compatibility}.

		You are encouraged to use
\begin{codeexample}[code only]
\usepackage{pgfplots}
\pgfplotsset{compat=1.3}
\end{codeexample}
		in your preamble.
		
		\item Version 1.3 fixes a bug with |anchor=south west| which occurred mainly for reversed axes: the anchors where upside--down. This is fixed now.

		
		If you have written some work--around and want to keep it going, use

		|\pgfplotsset{compat/anchors=pre 1.3}|\pgfmanualpdflabel{/pgfplots/compat/anchors}{} 

		to disable the bug fix.
	\end{enumerate}

	Use |\pgfplotsset{compat=default}| to restore the factory settings.

	The setting |\pgfplotsset{compat=newest}| will always use features of the most recent version. This might result in small changes in the document's appearance.
\end{pgfplotskey}

\subsection{The Team}
\PGFPlots\ has been written mainly by Christian Feuers�nger with many improvements of Pascal Wolkotte and Nick Papior Andersen as a spare time project. We hope it is useful and provides valuable plots.

If you are interested in writing something but don't know how, consider reading the auxiliary manual \href{file:TeX-programming-notes.pdf}{TeX-programming-notes.pdf} which comes with \PGFPlots. It is far from complete, but maybe it is a good starting point (at least for more literature).

\subsection{Acknowledgements}
I thank God for all hours of enjoyed programming. I thank Pascal Wolkotte and Nick Papior Andersen for their programming efforts and contributions as part of the development team. I thank Stefan Tibus, who contributed the |plot shell| feature. I thank Tom Cashman for the contribution of the |reverse legend| feature. Special thanks go to Stefan Pinnow whose tests of \PGFPlots\ lead to numerous quality improvements. Furthermore, I thank Dr.~Schweitzer for many fruitful discussions and Dr.~Meine for his ideas and suggestions. Thanks as well to the many international contributors who provided feature requests or identified bugs or simply manual improvements!

Last but not least, I thank Till Tantau and Mark Wibrow for their excellent graphics (and more) package \PGF\ and \Tikz\ which is the base of \PGFPlots.

\include{pgfplots.install}%
\include{pgfplots.intro}%
\include{pgfplots.reference}%
\include{pgfplots.libs}%
\include{pgfplots.resources}%
\include{pgfplots.importexport}%
\include{pgfplots.basic.reference}%

\printindex

\bibliographystyle{abbrv} %gerapali} %gerabbrv} %gerunsrt.bst} %gerabbrv}% gerplain}
\nocite{pgfplotstable}
\nocite{programmingnotes}
\bibliography{pgfplots}
\end{document}
%
\include{pgfplots.basic.reference}%

\printindex

\bibliographystyle{abbrv} %gerapali} %gerabbrv} %gerunsrt.bst} %gerabbrv}% gerplain}
\nocite{pgfplotstable}
\nocite{programmingnotes}
\bibliography{pgfplots}
\end{document}
%
%%%%%%%%%%%%%%%%%%%%%%%%%%%%%%%%%%%%%%%%%%%%%%%%%%%%%%%%%%%%%%%%%%%%%%%%%%%%%
%
% Package pgfplots.sty documentation. 
%
% Copyright 2007/2008 by Christian Feuersaenger.
%
% This program is free software: you can redistribute it and/or modify
% it under the terms of the GNU General Public License as published by
% the Free Software Foundation, either version 3 of the License, or
% (at your option) any later version.
% 
% This program is distributed in the hope that it will be useful,
% but WITHOUT ANY WARRANTY; without even the implied warranty of
% MERCHANTABILITY or FITNESS FOR A PARTICULAR PURPOSE.  See the
% GNU General Public License for more details.
% 
% You should have received a copy of the GNU General Public License
% along with this program.  If not, see <http://www.gnu.org/licenses/>.
%
%
%%%%%%%%%%%%%%%%%%%%%%%%%%%%%%%%%%%%%%%%%%%%%%%%%%%%%%%%%%%%%%%%%%%%%%%%%%%%%
%%%%%%%%%%%%%%%%%%%%%%%%%%%%%%%%%%%%%%%%%%%%%%%%%%%%%%%%%%%%%%%%%%%%%%%%%%%%%
%
% Package pgfplots.sty documentation. 
%
% Copyright 2007/2008 by Christian Feuersaenger.
%
% This program is free software: you can redistribute it and/or modify
% it under the terms of the GNU General Public License as published by
% the Free Software Foundation, either version 3 of the License, or
% (at your option) any later version.
% 
% This program is distributed in the hope that it will be useful,
% but WITHOUT ANY WARRANTY; without even the implied warranty of
% MERCHANTABILITY or FITNESS FOR A PARTICULAR PURPOSE.  See the
% GNU General Public License for more details.
% 
% You should have received a copy of the GNU General Public License
% along with this program.  If not, see <http://www.gnu.org/licenses/>.
%
%
%%%%%%%%%%%%%%%%%%%%%%%%%%%%%%%%%%%%%%%%%%%%%%%%%%%%%%%%%%%%%%%%%%%%%%%%%%%%%
\documentclass[a4paper]{ltxdoc}

\usepackage{makeidx}

% DON't let hyperref overload the format of index and glossary. 
% I want to do that on my own in the stylefiles for makeindex...
\makeatletter
\let\@old@wrindex=\@wrindex
\makeatother

\usepackage{ifpdf}
\ifpdf
	\usepackage[pdftex]{hyperref}
\else
	\usepackage[dvipdfm]{hyperref}
\fi
\hypersetup{pdfborder={0 0 0}}

\makeatletter
\let\@wrindex=\@old@wrindex
\makeatother

% Formatiere Seitennummern im Index:
\newcommand{\indexpageno}[1]{%
	{\bfseries\hyperpage{#1}}%
}


\newcommand{\R}{\mathbb{R}}
\newcommand{\N}{\mathbb{N}}
\newcommand{\Z}{\mathbb{Z}}

\long\def\COMMENTLOWLEVEL#1\ENDCOMMENT{}
\def\ENDCOMMENT{}

\usepackage{textcomp}
\usepackage{booktabs}

\usepackage{calc}
\usepackage[formats]{listings}
%\usepackage{courier} % don't use it - the '^' character can't be copy-pasted in courier

\usepackage{array}
\lstset{%
	basicstyle=\ttfamily,
	language=[LaTeX]tex, % Seems as if \lstset{language=tex} must be invoked BEFORE loading tikz!?
	tabsize=4,
	breaklines=true,
	breakindent=0pt
}

\ifpdf
	\pdfinfo {
		/Author	(Christian Feuersaenger)
	}

\else
	\def\pgfsysdriver{pgfsys-dvipdfm.def}
\fi
%\def\pgfsysdriver{pgfsys-pdftex.def}
\usepackage{pgfplots}

\ifpdf
	\usetikzlibrary{pgfplotsclickable}
\fi

%\usepackage{fp}
% ATTENTION:
% this requires pgf version NEWER than 2.00 :
%\usetikzlibrary{fixedpointarithmetic}

\usetikzlibrary{dateplot}

\usepackage[a4paper,left=2.25cm,right=2.25cm,top=2.5cm,bottom=2.5cm,nohead]{geometry}
\usepackage{amsmath,amssymb}
\usepackage{xxcolor}
\usepackage{pifont}
\usepackage[latin1]{inputenc}
\usepackage{amsmath}
\usepackage{eurosym}
\input{pgfplots-macros}

\pgfqkeys{/codeexample}{%
	every codeexample/.style={
		width=8cm,
		/pgfplots/every axis/.append style={legend style={fill=graphicbackground}}
	},
	tabsize=4,
}
\pgfplotsset{
	every axis/.append style={width=7cm},
	filter discard warning=false,
}

\usetikzlibrary{backgrounds,patterns}
% Global styles:
\tikzset{
  shape example/.style={
    color=black!30,
    draw,
    fill=yellow!30,
    line width=.5cm,
    inner xsep=2.5cm,
    inner ysep=0.5cm}
}

\newcommand{\FIXME}[1]{\textcolor{red}{(FIXME: #1)}}

% fuer endvironment 'sidewaysfigure' bspw
% \usepackage{rotating}

\newcommand\Tikz{Ti\textit kZ}
\newcommand\PGF{\textsc{pgf}}
\newcommand\PGFPlots{\textsc{pgfplots}}
\def\PGFPlotstable{\textsc{PgfplotsTable}}


\makeindex

% Fix overful hboxes automatically:
\tolerance=2000
\emergencystretch=10pt

\tikzset{prefix=gnuplot/pgfplots_} % prefix for 'plot function'

\author{%
	Christian Feuers\"anger\footnote{\url{http://wissrech.ins.uni-bonn.de/people/feuersaenger}}\\%
	Institut f\"ur Numerische Simulation\\
	Universit\"at Bonn, Germany}


%\RequirePackage[german,english,francais]{babel}

\pgfplotsmanualexternalexpensivetrue

%\usetikzlibrary{external}
%\tikzexternalize[prefix=figures/]{pgfplots}

\title{%
	Manual for Package \PGFPlots\\
	{\small Version \pgfplotsversion}\\
	{\small\href{http://sourceforge.net/projects/pgfplots}{http://sourceforge.net/projects/pgfplots}}}

%\includeonly{pgfplots.libs}


\begin{document}

\def\plotcoords{%
\addplot coordinates {
(5,8.312e-02)    (17,2.547e-02)   (49,7.407e-03)
(129,2.102e-03)  (321,5.874e-04)  (769,1.623e-04)
(1793,4.442e-05) (4097,1.207e-05) (9217,3.261e-06)
};

\addplot coordinates{
(7,8.472e-02)    (31,3.044e-02)    (111,1.022e-02)
(351,3.303e-03)  (1023,1.039e-03)  (2815,3.196e-04)
(7423,9.658e-05) (18943,2.873e-05) (47103,8.437e-06)
};

\addplot coordinates{
(9,7.881e-02)     (49,3.243e-02)    (209,1.232e-02)
(769,4.454e-03)   (2561,1.551e-03)  (7937,5.236e-04)
(23297,1.723e-04) (65537,5.545e-05) (178177,1.751e-05)
};

\addplot coordinates{
(11,6.887e-02)    (71,3.177e-02)     (351,1.341e-02)
(1471,5.334e-03)  (5503,2.027e-03)   (18943,7.415e-04)
(61183,2.628e-04) (187903,9.063e-05) (553983,3.053e-05)
};

\addplot coordinates{
(13,5.755e-02)     (97,2.925e-02)     (545,1.351e-02)
(2561,5.842e-03)   (10625,2.397e-03)  (40193,9.414e-04)
(141569,3.564e-04) (471041,1.308e-04) 
(1496065,4.670e-05)
};
}%
\maketitle
\begin{abstract}%
\PGFPlots\ draws high--quality function plots in normal or logarithmic scaling with a user-friendly interface directly in \TeX. The user supplies axis labels, legend entries and the plot coordinates for one or more plots and \PGFPlots\ applies axis scaling, computes any logarithms and axis ticks and draws the plots, supporting line plots, scatter plots, piecewise constant plots, bar plots, area plots, mesh-- and surface plots and some more. It is based on Till Tantau's package \PGF/\Tikz.
\end{abstract}
\tableofcontents
\section{Introduction}
This package provides tools to generate plots and labeled axes easily. It draws normal plots, logplots and semi-logplots, in two and three dimensions. Axis ticks, labels, legends (in case of multiple plots) can be added with key-value options. It can cycle through a set of predefined line/marker/color specifications. In summary, its purpose is to simplify the generation of high-quality function and/or data plots, and solving the problems of
\begin{itemize}
	\item consistency of document and font type and font size,
	\item direct use of \TeX\ math mode in axis descriptions,
	\item consistency of data and figures (no third party tool necessary),
	\item inter document consistency using preamble configurations and styles.
\end{itemize}
Although not necessary, separate |.pdf| or |.eps| graphics can be generated using the |external| library developed as part of \Tikz.

\PGFPlots\ is build completely on \Tikz/\PGF. Knowledge of \Tikz\ will simplify the work with \PGFPlots, although it is not required.

\section[About PGFPlots: Preliminaries]{About {\normalfont\PGFPlots}: Preliminaries}
This section contains information about upgrades, the team, the installation (in case you need to do it manually) and troubleshooting. You may skip it completely except for the upgrade remarks.

However, note that this library requires at least \PGF\ version $2.00$. At the time of this writing, many \TeX-distributions still contain the older \PGF\ version $1.18$, so it may be necessary to install a recent \PGF\ prior to using \PGFPlots.

\subsection{Components}
\PGFPlots\ comes with two components:
\begin{enumerate}
	\item the plotting component (which you are currently reading) and
	\item the \PGFPlotstable\ component which simplifies number formatting and postprocessing of numerical tables. It comes as a separate package and has its own manual \href{file:pgfplotstable.pdf}{pgfplotstable.pdf}.
\end{enumerate}

\subsection{Upgrade remarks}
This release provides a lot of improvements which can be found in all detail in \texttt{ChangeLog} for interested readers. However, some attention is useful with respect to the following changes.

\subsubsection{New Optional Features}
\begin{enumerate}
	\item \PGFPlots\ 1.3 comes with user interface improvements. The technical distinction between ``behavior options'' and ``style options'' of older versions is no longer necessary (although still fully supported).

	\item \PGFPlots\ 1.3 has a new feature which allows to \emph{move axis labels tight to tick labels} automatically.

	Since this affects the spacing, it is not enabled be default.

	Use
\begin{codeexample}[code only]
\usepackage{pgfplots}
\pgfplotsset{compat=1.3}
\end{codeexample}
	in your preamble to benefit from the improved spacing\footnote{The |compat=1.3| setting is necessary for new features which move axis labels around like reversed axis. However, \PGFPlots\ will still work as it does in earlier versions, your documents will remain the same if you don't set it explicitly.}.
	Take a look at the next page for the precise definition of |compat|.

	\item \PGFPlots\ 1.3 now supports reversed axes. It is no longer necessary to use work--arounds with negative units.
\pgfkeys{/pdflinks/search key prefixes in/.add={/pgfplots/,}{}}

	Take a look at the |x dir=reverse| key.

	Existing work--arounds will still function properly. Use |\pgfplotsset{compat=1.3}| together with |x dir=reverse| to switch to the new version.
\end{enumerate}

\subsubsection{Old Features Which May Need Attention}
\begin{enumerate}
	\item The |scatter/classes| feature produces proper legends as of version 1.3. This may change the appearance of existing legends of plots with |scatter/classes|.

	\item Due to a small math bug in \PGF\ $2.00$, you \emph{can't} use the math expression `|-x^2|'. It is necessary to use `|0-x^2|' instead. The same holds for
		`|exp(-x^2)|'; use `|exp(0-x^2)|' instead.

		This will be fixed with \PGF\ $>2.00$.

	\item Starting with \PGFPlots\ $1.1$, |\tikzstyle| should \emph{no longer be used} to set \PGFPlots\ options.
	
	Although |\tikzstyle| is still supported for some older \PGFPlots\ options, you should replace any occurance of |\tikzstyle| with |\pgfplotsset{|\meta{style name}|/.style={|\meta{key-value-list}|}}| or the associated |/.append style| variant. See section~\ref{sec:styles} for more detail.
\end{enumerate}
I apologize for any inconvenience caused by these changes.

\begin{pgfplotskey}{compat=\mchoice{1.3,pre 1.3,default,newest} (initially default)}
	The preamble configuration 
\begin{codeexample}[code only]
\usepackage{pgfplots}
\pgfplotsset{compat=1.3}
\end{codeexample}
	allows to choose between backwards compatibility and most recent features.

	\PGFPlots\ version 1.3 comes with new features which may lead to different spacings compared to earlier versions:
	\begin{enumerate}
		\item Version 1.3 comes with the |xlabel near ticks| feature which places (in this case $x$) axis labels ``near'' the tick labels, taking the size of tick labels into account.
		
		Older versions used |xlabel absolute|, i.e.\ an absolute distance between the axis and axis labels (now matter how large or small tick labels are).

		The initial setting \emph{keeps backwards compatibility}.

		You are encouraged to use
\begin{codeexample}[code only]
\usepackage{pgfplots}
\pgfplotsset{compat=1.3}
\end{codeexample}
		in your preamble.
		
		\item Version 1.3 fixes a bug with |anchor=south west| which occurred mainly for reversed axes: the anchors where upside--down. This is fixed now.

		
		If you have written some work--around and want to keep it going, use

		|\pgfplotsset{compat/anchors=pre 1.3}|\pgfmanualpdflabel{/pgfplots/compat/anchors}{} 

		to disable the bug fix.
	\end{enumerate}

	Use |\pgfplotsset{compat=default}| to restore the factory settings.

	The setting |\pgfplotsset{compat=newest}| will always use features of the most recent version. This might result in small changes in the document's appearance.
\end{pgfplotskey}

\subsection{The Team}
\PGFPlots\ has been written mainly by Christian Feuers�nger with many improvements of Pascal Wolkotte and Nick Papior Andersen as a spare time project. We hope it is useful and provides valuable plots.

If you are interested in writing something but don't know how, consider reading the auxiliary manual \href{file:TeX-programming-notes.pdf}{TeX-programming-notes.pdf} which comes with \PGFPlots. It is far from complete, but maybe it is a good starting point (at least for more literature).

\subsection{Acknowledgements}
I thank God for all hours of enjoyed programming. I thank Pascal Wolkotte and Nick Papior Andersen for their programming efforts and contributions as part of the development team. I thank Stefan Tibus, who contributed the |plot shell| feature. I thank Tom Cashman for the contribution of the |reverse legend| feature. Special thanks go to Stefan Pinnow whose tests of \PGFPlots\ lead to numerous quality improvements. Furthermore, I thank Dr.~Schweitzer for many fruitful discussions and Dr.~Meine for his ideas and suggestions. Thanks as well to the many international contributors who provided feature requests or identified bugs or simply manual improvements!

Last but not least, I thank Till Tantau and Mark Wibrow for their excellent graphics (and more) package \PGF\ and \Tikz\ which is the base of \PGFPlots.

%%%%%%%%%%%%%%%%%%%%%%%%%%%%%%%%%%%%%%%%%%%%%%%%%%%%%%%%%%%%%%%%%%%%%%%%%%%%%
%
% Package pgfplots.sty documentation. 
%
% Copyright 2007/2008 by Christian Feuersaenger.
%
% This program is free software: you can redistribute it and/or modify
% it under the terms of the GNU General Public License as published by
% the Free Software Foundation, either version 3 of the License, or
% (at your option) any later version.
% 
% This program is distributed in the hope that it will be useful,
% but WITHOUT ANY WARRANTY; without even the implied warranty of
% MERCHANTABILITY or FITNESS FOR A PARTICULAR PURPOSE.  See the
% GNU General Public License for more details.
% 
% You should have received a copy of the GNU General Public License
% along with this program.  If not, see <http://www.gnu.org/licenses/>.
%
%
%%%%%%%%%%%%%%%%%%%%%%%%%%%%%%%%%%%%%%%%%%%%%%%%%%%%%%%%%%%%%%%%%%%%%%%%%%%%%
\input{pgfplots.preamble.tex}
%\RequirePackage[german,english,francais]{babel}

\pgfplotsmanualexternalexpensivetrue

%\usetikzlibrary{external}
%\tikzexternalize[prefix=figures/]{pgfplots}

\title{%
	Manual for Package \PGFPlots\\
	{\small Version \pgfplotsversion}\\
	{\small\href{http://sourceforge.net/projects/pgfplots}{http://sourceforge.net/projects/pgfplots}}}

%\includeonly{pgfplots.libs}


\begin{document}

\def\plotcoords{%
\addplot coordinates {
(5,8.312e-02)    (17,2.547e-02)   (49,7.407e-03)
(129,2.102e-03)  (321,5.874e-04)  (769,1.623e-04)
(1793,4.442e-05) (4097,1.207e-05) (9217,3.261e-06)
};

\addplot coordinates{
(7,8.472e-02)    (31,3.044e-02)    (111,1.022e-02)
(351,3.303e-03)  (1023,1.039e-03)  (2815,3.196e-04)
(7423,9.658e-05) (18943,2.873e-05) (47103,8.437e-06)
};

\addplot coordinates{
(9,7.881e-02)     (49,3.243e-02)    (209,1.232e-02)
(769,4.454e-03)   (2561,1.551e-03)  (7937,5.236e-04)
(23297,1.723e-04) (65537,5.545e-05) (178177,1.751e-05)
};

\addplot coordinates{
(11,6.887e-02)    (71,3.177e-02)     (351,1.341e-02)
(1471,5.334e-03)  (5503,2.027e-03)   (18943,7.415e-04)
(61183,2.628e-04) (187903,9.063e-05) (553983,3.053e-05)
};

\addplot coordinates{
(13,5.755e-02)     (97,2.925e-02)     (545,1.351e-02)
(2561,5.842e-03)   (10625,2.397e-03)  (40193,9.414e-04)
(141569,3.564e-04) (471041,1.308e-04) 
(1496065,4.670e-05)
};
}%
\maketitle
\begin{abstract}%
\PGFPlots\ draws high--quality function plots in normal or logarithmic scaling with a user-friendly interface directly in \TeX. The user supplies axis labels, legend entries and the plot coordinates for one or more plots and \PGFPlots\ applies axis scaling, computes any logarithms and axis ticks and draws the plots, supporting line plots, scatter plots, piecewise constant plots, bar plots, area plots, mesh-- and surface plots and some more. It is based on Till Tantau's package \PGF/\Tikz.
\end{abstract}
\tableofcontents
\section{Introduction}
This package provides tools to generate plots and labeled axes easily. It draws normal plots, logplots and semi-logplots, in two and three dimensions. Axis ticks, labels, legends (in case of multiple plots) can be added with key-value options. It can cycle through a set of predefined line/marker/color specifications. In summary, its purpose is to simplify the generation of high-quality function and/or data plots, and solving the problems of
\begin{itemize}
	\item consistency of document and font type and font size,
	\item direct use of \TeX\ math mode in axis descriptions,
	\item consistency of data and figures (no third party tool necessary),
	\item inter document consistency using preamble configurations and styles.
\end{itemize}
Although not necessary, separate |.pdf| or |.eps| graphics can be generated using the |external| library developed as part of \Tikz.

\PGFPlots\ is build completely on \Tikz/\PGF. Knowledge of \Tikz\ will simplify the work with \PGFPlots, although it is not required.

\section[About PGFPlots: Preliminaries]{About {\normalfont\PGFPlots}: Preliminaries}
This section contains information about upgrades, the team, the installation (in case you need to do it manually) and troubleshooting. You may skip it completely except for the upgrade remarks.

However, note that this library requires at least \PGF\ version $2.00$. At the time of this writing, many \TeX-distributions still contain the older \PGF\ version $1.18$, so it may be necessary to install a recent \PGF\ prior to using \PGFPlots.

\subsection{Components}
\PGFPlots\ comes with two components:
\begin{enumerate}
	\item the plotting component (which you are currently reading) and
	\item the \PGFPlotstable\ component which simplifies number formatting and postprocessing of numerical tables. It comes as a separate package and has its own manual \href{file:pgfplotstable.pdf}{pgfplotstable.pdf}.
\end{enumerate}

\subsection{Upgrade remarks}
This release provides a lot of improvements which can be found in all detail in \texttt{ChangeLog} for interested readers. However, some attention is useful with respect to the following changes.

\subsubsection{New Optional Features}
\begin{enumerate}
	\item \PGFPlots\ 1.3 comes with user interface improvements. The technical distinction between ``behavior options'' and ``style options'' of older versions is no longer necessary (although still fully supported).

	\item \PGFPlots\ 1.3 has a new feature which allows to \emph{move axis labels tight to tick labels} automatically.

	Since this affects the spacing, it is not enabled be default.

	Use
\begin{codeexample}[code only]
\usepackage{pgfplots}
\pgfplotsset{compat=1.3}
\end{codeexample}
	in your preamble to benefit from the improved spacing\footnote{The |compat=1.3| setting is necessary for new features which move axis labels around like reversed axis. However, \PGFPlots\ will still work as it does in earlier versions, your documents will remain the same if you don't set it explicitly.}.
	Take a look at the next page for the precise definition of |compat|.

	\item \PGFPlots\ 1.3 now supports reversed axes. It is no longer necessary to use work--arounds with negative units.
\pgfkeys{/pdflinks/search key prefixes in/.add={/pgfplots/,}{}}

	Take a look at the |x dir=reverse| key.

	Existing work--arounds will still function properly. Use |\pgfplotsset{compat=1.3}| together with |x dir=reverse| to switch to the new version.
\end{enumerate}

\subsubsection{Old Features Which May Need Attention}
\begin{enumerate}
	\item The |scatter/classes| feature produces proper legends as of version 1.3. This may change the appearance of existing legends of plots with |scatter/classes|.

	\item Due to a small math bug in \PGF\ $2.00$, you \emph{can't} use the math expression `|-x^2|'. It is necessary to use `|0-x^2|' instead. The same holds for
		`|exp(-x^2)|'; use `|exp(0-x^2)|' instead.

		This will be fixed with \PGF\ $>2.00$.

	\item Starting with \PGFPlots\ $1.1$, |\tikzstyle| should \emph{no longer be used} to set \PGFPlots\ options.
	
	Although |\tikzstyle| is still supported for some older \PGFPlots\ options, you should replace any occurance of |\tikzstyle| with |\pgfplotsset{|\meta{style name}|/.style={|\meta{key-value-list}|}}| or the associated |/.append style| variant. See section~\ref{sec:styles} for more detail.
\end{enumerate}
I apologize for any inconvenience caused by these changes.

\begin{pgfplotskey}{compat=\mchoice{1.3,pre 1.3,default,newest} (initially default)}
	The preamble configuration 
\begin{codeexample}[code only]
\usepackage{pgfplots}
\pgfplotsset{compat=1.3}
\end{codeexample}
	allows to choose between backwards compatibility and most recent features.

	\PGFPlots\ version 1.3 comes with new features which may lead to different spacings compared to earlier versions:
	\begin{enumerate}
		\item Version 1.3 comes with the |xlabel near ticks| feature which places (in this case $x$) axis labels ``near'' the tick labels, taking the size of tick labels into account.
		
		Older versions used |xlabel absolute|, i.e.\ an absolute distance between the axis and axis labels (now matter how large or small tick labels are).

		The initial setting \emph{keeps backwards compatibility}.

		You are encouraged to use
\begin{codeexample}[code only]
\usepackage{pgfplots}
\pgfplotsset{compat=1.3}
\end{codeexample}
		in your preamble.
		
		\item Version 1.3 fixes a bug with |anchor=south west| which occurred mainly for reversed axes: the anchors where upside--down. This is fixed now.

		
		If you have written some work--around and want to keep it going, use

		|\pgfplotsset{compat/anchors=pre 1.3}|\pgfmanualpdflabel{/pgfplots/compat/anchors}{} 

		to disable the bug fix.
	\end{enumerate}

	Use |\pgfplotsset{compat=default}| to restore the factory settings.

	The setting |\pgfplotsset{compat=newest}| will always use features of the most recent version. This might result in small changes in the document's appearance.
\end{pgfplotskey}

\subsection{The Team}
\PGFPlots\ has been written mainly by Christian Feuers�nger with many improvements of Pascal Wolkotte and Nick Papior Andersen as a spare time project. We hope it is useful and provides valuable plots.

If you are interested in writing something but don't know how, consider reading the auxiliary manual \href{file:TeX-programming-notes.pdf}{TeX-programming-notes.pdf} which comes with \PGFPlots. It is far from complete, but maybe it is a good starting point (at least for more literature).

\subsection{Acknowledgements}
I thank God for all hours of enjoyed programming. I thank Pascal Wolkotte and Nick Papior Andersen for their programming efforts and contributions as part of the development team. I thank Stefan Tibus, who contributed the |plot shell| feature. I thank Tom Cashman for the contribution of the |reverse legend| feature. Special thanks go to Stefan Pinnow whose tests of \PGFPlots\ lead to numerous quality improvements. Furthermore, I thank Dr.~Schweitzer for many fruitful discussions and Dr.~Meine for his ideas and suggestions. Thanks as well to the many international contributors who provided feature requests or identified bugs or simply manual improvements!

Last but not least, I thank Till Tantau and Mark Wibrow for their excellent graphics (and more) package \PGF\ and \Tikz\ which is the base of \PGFPlots.

\include{pgfplots.install}%
\include{pgfplots.intro}%
\include{pgfplots.reference}%
\include{pgfplots.libs}%
\include{pgfplots.resources}%
\include{pgfplots.importexport}%
\include{pgfplots.basic.reference}%

\printindex

\bibliographystyle{abbrv} %gerapali} %gerabbrv} %gerunsrt.bst} %gerabbrv}% gerplain}
\nocite{pgfplotstable}
\nocite{programmingnotes}
\bibliography{pgfplots}
\end{document}
%
%%%%%%%%%%%%%%%%%%%%%%%%%%%%%%%%%%%%%%%%%%%%%%%%%%%%%%%%%%%%%%%%%%%%%%%%%%%%%
%
% Package pgfplots.sty documentation. 
%
% Copyright 2007/2008 by Christian Feuersaenger.
%
% This program is free software: you can redistribute it and/or modify
% it under the terms of the GNU General Public License as published by
% the Free Software Foundation, either version 3 of the License, or
% (at your option) any later version.
% 
% This program is distributed in the hope that it will be useful,
% but WITHOUT ANY WARRANTY; without even the implied warranty of
% MERCHANTABILITY or FITNESS FOR A PARTICULAR PURPOSE.  See the
% GNU General Public License for more details.
% 
% You should have received a copy of the GNU General Public License
% along with this program.  If not, see <http://www.gnu.org/licenses/>.
%
%
%%%%%%%%%%%%%%%%%%%%%%%%%%%%%%%%%%%%%%%%%%%%%%%%%%%%%%%%%%%%%%%%%%%%%%%%%%%%%
\input{pgfplots.preamble.tex}
%\RequirePackage[german,english,francais]{babel}

\pgfplotsmanualexternalexpensivetrue

%\usetikzlibrary{external}
%\tikzexternalize[prefix=figures/]{pgfplots}

\title{%
	Manual for Package \PGFPlots\\
	{\small Version \pgfplotsversion}\\
	{\small\href{http://sourceforge.net/projects/pgfplots}{http://sourceforge.net/projects/pgfplots}}}

%\includeonly{pgfplots.libs}


\begin{document}

\def\plotcoords{%
\addplot coordinates {
(5,8.312e-02)    (17,2.547e-02)   (49,7.407e-03)
(129,2.102e-03)  (321,5.874e-04)  (769,1.623e-04)
(1793,4.442e-05) (4097,1.207e-05) (9217,3.261e-06)
};

\addplot coordinates{
(7,8.472e-02)    (31,3.044e-02)    (111,1.022e-02)
(351,3.303e-03)  (1023,1.039e-03)  (2815,3.196e-04)
(7423,9.658e-05) (18943,2.873e-05) (47103,8.437e-06)
};

\addplot coordinates{
(9,7.881e-02)     (49,3.243e-02)    (209,1.232e-02)
(769,4.454e-03)   (2561,1.551e-03)  (7937,5.236e-04)
(23297,1.723e-04) (65537,5.545e-05) (178177,1.751e-05)
};

\addplot coordinates{
(11,6.887e-02)    (71,3.177e-02)     (351,1.341e-02)
(1471,5.334e-03)  (5503,2.027e-03)   (18943,7.415e-04)
(61183,2.628e-04) (187903,9.063e-05) (553983,3.053e-05)
};

\addplot coordinates{
(13,5.755e-02)     (97,2.925e-02)     (545,1.351e-02)
(2561,5.842e-03)   (10625,2.397e-03)  (40193,9.414e-04)
(141569,3.564e-04) (471041,1.308e-04) 
(1496065,4.670e-05)
};
}%
\maketitle
\begin{abstract}%
\PGFPlots\ draws high--quality function plots in normal or logarithmic scaling with a user-friendly interface directly in \TeX. The user supplies axis labels, legend entries and the plot coordinates for one or more plots and \PGFPlots\ applies axis scaling, computes any logarithms and axis ticks and draws the plots, supporting line plots, scatter plots, piecewise constant plots, bar plots, area plots, mesh-- and surface plots and some more. It is based on Till Tantau's package \PGF/\Tikz.
\end{abstract}
\tableofcontents
\section{Introduction}
This package provides tools to generate plots and labeled axes easily. It draws normal plots, logplots and semi-logplots, in two and three dimensions. Axis ticks, labels, legends (in case of multiple plots) can be added with key-value options. It can cycle through a set of predefined line/marker/color specifications. In summary, its purpose is to simplify the generation of high-quality function and/or data plots, and solving the problems of
\begin{itemize}
	\item consistency of document and font type and font size,
	\item direct use of \TeX\ math mode in axis descriptions,
	\item consistency of data and figures (no third party tool necessary),
	\item inter document consistency using preamble configurations and styles.
\end{itemize}
Although not necessary, separate |.pdf| or |.eps| graphics can be generated using the |external| library developed as part of \Tikz.

\PGFPlots\ is build completely on \Tikz/\PGF. Knowledge of \Tikz\ will simplify the work with \PGFPlots, although it is not required.

\section[About PGFPlots: Preliminaries]{About {\normalfont\PGFPlots}: Preliminaries}
This section contains information about upgrades, the team, the installation (in case you need to do it manually) and troubleshooting. You may skip it completely except for the upgrade remarks.

However, note that this library requires at least \PGF\ version $2.00$. At the time of this writing, many \TeX-distributions still contain the older \PGF\ version $1.18$, so it may be necessary to install a recent \PGF\ prior to using \PGFPlots.

\subsection{Components}
\PGFPlots\ comes with two components:
\begin{enumerate}
	\item the plotting component (which you are currently reading) and
	\item the \PGFPlotstable\ component which simplifies number formatting and postprocessing of numerical tables. It comes as a separate package and has its own manual \href{file:pgfplotstable.pdf}{pgfplotstable.pdf}.
\end{enumerate}

\subsection{Upgrade remarks}
This release provides a lot of improvements which can be found in all detail in \texttt{ChangeLog} for interested readers. However, some attention is useful with respect to the following changes.

\subsubsection{New Optional Features}
\begin{enumerate}
	\item \PGFPlots\ 1.3 comes with user interface improvements. The technical distinction between ``behavior options'' and ``style options'' of older versions is no longer necessary (although still fully supported).

	\item \PGFPlots\ 1.3 has a new feature which allows to \emph{move axis labels tight to tick labels} automatically.

	Since this affects the spacing, it is not enabled be default.

	Use
\begin{codeexample}[code only]
\usepackage{pgfplots}
\pgfplotsset{compat=1.3}
\end{codeexample}
	in your preamble to benefit from the improved spacing\footnote{The |compat=1.3| setting is necessary for new features which move axis labels around like reversed axis. However, \PGFPlots\ will still work as it does in earlier versions, your documents will remain the same if you don't set it explicitly.}.
	Take a look at the next page for the precise definition of |compat|.

	\item \PGFPlots\ 1.3 now supports reversed axes. It is no longer necessary to use work--arounds with negative units.
\pgfkeys{/pdflinks/search key prefixes in/.add={/pgfplots/,}{}}

	Take a look at the |x dir=reverse| key.

	Existing work--arounds will still function properly. Use |\pgfplotsset{compat=1.3}| together with |x dir=reverse| to switch to the new version.
\end{enumerate}

\subsubsection{Old Features Which May Need Attention}
\begin{enumerate}
	\item The |scatter/classes| feature produces proper legends as of version 1.3. This may change the appearance of existing legends of plots with |scatter/classes|.

	\item Due to a small math bug in \PGF\ $2.00$, you \emph{can't} use the math expression `|-x^2|'. It is necessary to use `|0-x^2|' instead. The same holds for
		`|exp(-x^2)|'; use `|exp(0-x^2)|' instead.

		This will be fixed with \PGF\ $>2.00$.

	\item Starting with \PGFPlots\ $1.1$, |\tikzstyle| should \emph{no longer be used} to set \PGFPlots\ options.
	
	Although |\tikzstyle| is still supported for some older \PGFPlots\ options, you should replace any occurance of |\tikzstyle| with |\pgfplotsset{|\meta{style name}|/.style={|\meta{key-value-list}|}}| or the associated |/.append style| variant. See section~\ref{sec:styles} for more detail.
\end{enumerate}
I apologize for any inconvenience caused by these changes.

\begin{pgfplotskey}{compat=\mchoice{1.3,pre 1.3,default,newest} (initially default)}
	The preamble configuration 
\begin{codeexample}[code only]
\usepackage{pgfplots}
\pgfplotsset{compat=1.3}
\end{codeexample}
	allows to choose between backwards compatibility and most recent features.

	\PGFPlots\ version 1.3 comes with new features which may lead to different spacings compared to earlier versions:
	\begin{enumerate}
		\item Version 1.3 comes with the |xlabel near ticks| feature which places (in this case $x$) axis labels ``near'' the tick labels, taking the size of tick labels into account.
		
		Older versions used |xlabel absolute|, i.e.\ an absolute distance between the axis and axis labels (now matter how large or small tick labels are).

		The initial setting \emph{keeps backwards compatibility}.

		You are encouraged to use
\begin{codeexample}[code only]
\usepackage{pgfplots}
\pgfplotsset{compat=1.3}
\end{codeexample}
		in your preamble.
		
		\item Version 1.3 fixes a bug with |anchor=south west| which occurred mainly for reversed axes: the anchors where upside--down. This is fixed now.

		
		If you have written some work--around and want to keep it going, use

		|\pgfplotsset{compat/anchors=pre 1.3}|\pgfmanualpdflabel{/pgfplots/compat/anchors}{} 

		to disable the bug fix.
	\end{enumerate}

	Use |\pgfplotsset{compat=default}| to restore the factory settings.

	The setting |\pgfplotsset{compat=newest}| will always use features of the most recent version. This might result in small changes in the document's appearance.
\end{pgfplotskey}

\subsection{The Team}
\PGFPlots\ has been written mainly by Christian Feuers�nger with many improvements of Pascal Wolkotte and Nick Papior Andersen as a spare time project. We hope it is useful and provides valuable plots.

If you are interested in writing something but don't know how, consider reading the auxiliary manual \href{file:TeX-programming-notes.pdf}{TeX-programming-notes.pdf} which comes with \PGFPlots. It is far from complete, but maybe it is a good starting point (at least for more literature).

\subsection{Acknowledgements}
I thank God for all hours of enjoyed programming. I thank Pascal Wolkotte and Nick Papior Andersen for their programming efforts and contributions as part of the development team. I thank Stefan Tibus, who contributed the |plot shell| feature. I thank Tom Cashman for the contribution of the |reverse legend| feature. Special thanks go to Stefan Pinnow whose tests of \PGFPlots\ lead to numerous quality improvements. Furthermore, I thank Dr.~Schweitzer for many fruitful discussions and Dr.~Meine for his ideas and suggestions. Thanks as well to the many international contributors who provided feature requests or identified bugs or simply manual improvements!

Last but not least, I thank Till Tantau and Mark Wibrow for their excellent graphics (and more) package \PGF\ and \Tikz\ which is the base of \PGFPlots.

\include{pgfplots.install}%
\include{pgfplots.intro}%
\include{pgfplots.reference}%
\include{pgfplots.libs}%
\include{pgfplots.resources}%
\include{pgfplots.importexport}%
\include{pgfplots.basic.reference}%

\printindex

\bibliographystyle{abbrv} %gerapali} %gerabbrv} %gerunsrt.bst} %gerabbrv}% gerplain}
\nocite{pgfplotstable}
\nocite{programmingnotes}
\bibliography{pgfplots}
\end{document}
%
%%%%%%%%%%%%%%%%%%%%%%%%%%%%%%%%%%%%%%%%%%%%%%%%%%%%%%%%%%%%%%%%%%%%%%%%%%%%%
%
% Package pgfplots.sty documentation. 
%
% Copyright 2007/2008 by Christian Feuersaenger.
%
% This program is free software: you can redistribute it and/or modify
% it under the terms of the GNU General Public License as published by
% the Free Software Foundation, either version 3 of the License, or
% (at your option) any later version.
% 
% This program is distributed in the hope that it will be useful,
% but WITHOUT ANY WARRANTY; without even the implied warranty of
% MERCHANTABILITY or FITNESS FOR A PARTICULAR PURPOSE.  See the
% GNU General Public License for more details.
% 
% You should have received a copy of the GNU General Public License
% along with this program.  If not, see <http://www.gnu.org/licenses/>.
%
%
%%%%%%%%%%%%%%%%%%%%%%%%%%%%%%%%%%%%%%%%%%%%%%%%%%%%%%%%%%%%%%%%%%%%%%%%%%%%%
\input{pgfplots.preamble.tex}
%\RequirePackage[german,english,francais]{babel}

\pgfplotsmanualexternalexpensivetrue

%\usetikzlibrary{external}
%\tikzexternalize[prefix=figures/]{pgfplots}

\title{%
	Manual for Package \PGFPlots\\
	{\small Version \pgfplotsversion}\\
	{\small\href{http://sourceforge.net/projects/pgfplots}{http://sourceforge.net/projects/pgfplots}}}

%\includeonly{pgfplots.libs}


\begin{document}

\def\plotcoords{%
\addplot coordinates {
(5,8.312e-02)    (17,2.547e-02)   (49,7.407e-03)
(129,2.102e-03)  (321,5.874e-04)  (769,1.623e-04)
(1793,4.442e-05) (4097,1.207e-05) (9217,3.261e-06)
};

\addplot coordinates{
(7,8.472e-02)    (31,3.044e-02)    (111,1.022e-02)
(351,3.303e-03)  (1023,1.039e-03)  (2815,3.196e-04)
(7423,9.658e-05) (18943,2.873e-05) (47103,8.437e-06)
};

\addplot coordinates{
(9,7.881e-02)     (49,3.243e-02)    (209,1.232e-02)
(769,4.454e-03)   (2561,1.551e-03)  (7937,5.236e-04)
(23297,1.723e-04) (65537,5.545e-05) (178177,1.751e-05)
};

\addplot coordinates{
(11,6.887e-02)    (71,3.177e-02)     (351,1.341e-02)
(1471,5.334e-03)  (5503,2.027e-03)   (18943,7.415e-04)
(61183,2.628e-04) (187903,9.063e-05) (553983,3.053e-05)
};

\addplot coordinates{
(13,5.755e-02)     (97,2.925e-02)     (545,1.351e-02)
(2561,5.842e-03)   (10625,2.397e-03)  (40193,9.414e-04)
(141569,3.564e-04) (471041,1.308e-04) 
(1496065,4.670e-05)
};
}%
\maketitle
\begin{abstract}%
\PGFPlots\ draws high--quality function plots in normal or logarithmic scaling with a user-friendly interface directly in \TeX. The user supplies axis labels, legend entries and the plot coordinates for one or more plots and \PGFPlots\ applies axis scaling, computes any logarithms and axis ticks and draws the plots, supporting line plots, scatter plots, piecewise constant plots, bar plots, area plots, mesh-- and surface plots and some more. It is based on Till Tantau's package \PGF/\Tikz.
\end{abstract}
\tableofcontents
\section{Introduction}
This package provides tools to generate plots and labeled axes easily. It draws normal plots, logplots and semi-logplots, in two and three dimensions. Axis ticks, labels, legends (in case of multiple plots) can be added with key-value options. It can cycle through a set of predefined line/marker/color specifications. In summary, its purpose is to simplify the generation of high-quality function and/or data plots, and solving the problems of
\begin{itemize}
	\item consistency of document and font type and font size,
	\item direct use of \TeX\ math mode in axis descriptions,
	\item consistency of data and figures (no third party tool necessary),
	\item inter document consistency using preamble configurations and styles.
\end{itemize}
Although not necessary, separate |.pdf| or |.eps| graphics can be generated using the |external| library developed as part of \Tikz.

\PGFPlots\ is build completely on \Tikz/\PGF. Knowledge of \Tikz\ will simplify the work with \PGFPlots, although it is not required.

\section[About PGFPlots: Preliminaries]{About {\normalfont\PGFPlots}: Preliminaries}
This section contains information about upgrades, the team, the installation (in case you need to do it manually) and troubleshooting. You may skip it completely except for the upgrade remarks.

However, note that this library requires at least \PGF\ version $2.00$. At the time of this writing, many \TeX-distributions still contain the older \PGF\ version $1.18$, so it may be necessary to install a recent \PGF\ prior to using \PGFPlots.

\subsection{Components}
\PGFPlots\ comes with two components:
\begin{enumerate}
	\item the plotting component (which you are currently reading) and
	\item the \PGFPlotstable\ component which simplifies number formatting and postprocessing of numerical tables. It comes as a separate package and has its own manual \href{file:pgfplotstable.pdf}{pgfplotstable.pdf}.
\end{enumerate}

\subsection{Upgrade remarks}
This release provides a lot of improvements which can be found in all detail in \texttt{ChangeLog} for interested readers. However, some attention is useful with respect to the following changes.

\subsubsection{New Optional Features}
\begin{enumerate}
	\item \PGFPlots\ 1.3 comes with user interface improvements. The technical distinction between ``behavior options'' and ``style options'' of older versions is no longer necessary (although still fully supported).

	\item \PGFPlots\ 1.3 has a new feature which allows to \emph{move axis labels tight to tick labels} automatically.

	Since this affects the spacing, it is not enabled be default.

	Use
\begin{codeexample}[code only]
\usepackage{pgfplots}
\pgfplotsset{compat=1.3}
\end{codeexample}
	in your preamble to benefit from the improved spacing\footnote{The |compat=1.3| setting is necessary for new features which move axis labels around like reversed axis. However, \PGFPlots\ will still work as it does in earlier versions, your documents will remain the same if you don't set it explicitly.}.
	Take a look at the next page for the precise definition of |compat|.

	\item \PGFPlots\ 1.3 now supports reversed axes. It is no longer necessary to use work--arounds with negative units.
\pgfkeys{/pdflinks/search key prefixes in/.add={/pgfplots/,}{}}

	Take a look at the |x dir=reverse| key.

	Existing work--arounds will still function properly. Use |\pgfplotsset{compat=1.3}| together with |x dir=reverse| to switch to the new version.
\end{enumerate}

\subsubsection{Old Features Which May Need Attention}
\begin{enumerate}
	\item The |scatter/classes| feature produces proper legends as of version 1.3. This may change the appearance of existing legends of plots with |scatter/classes|.

	\item Due to a small math bug in \PGF\ $2.00$, you \emph{can't} use the math expression `|-x^2|'. It is necessary to use `|0-x^2|' instead. The same holds for
		`|exp(-x^2)|'; use `|exp(0-x^2)|' instead.

		This will be fixed with \PGF\ $>2.00$.

	\item Starting with \PGFPlots\ $1.1$, |\tikzstyle| should \emph{no longer be used} to set \PGFPlots\ options.
	
	Although |\tikzstyle| is still supported for some older \PGFPlots\ options, you should replace any occurance of |\tikzstyle| with |\pgfplotsset{|\meta{style name}|/.style={|\meta{key-value-list}|}}| or the associated |/.append style| variant. See section~\ref{sec:styles} for more detail.
\end{enumerate}
I apologize for any inconvenience caused by these changes.

\begin{pgfplotskey}{compat=\mchoice{1.3,pre 1.3,default,newest} (initially default)}
	The preamble configuration 
\begin{codeexample}[code only]
\usepackage{pgfplots}
\pgfplotsset{compat=1.3}
\end{codeexample}
	allows to choose between backwards compatibility and most recent features.

	\PGFPlots\ version 1.3 comes with new features which may lead to different spacings compared to earlier versions:
	\begin{enumerate}
		\item Version 1.3 comes with the |xlabel near ticks| feature which places (in this case $x$) axis labels ``near'' the tick labels, taking the size of tick labels into account.
		
		Older versions used |xlabel absolute|, i.e.\ an absolute distance between the axis and axis labels (now matter how large or small tick labels are).

		The initial setting \emph{keeps backwards compatibility}.

		You are encouraged to use
\begin{codeexample}[code only]
\usepackage{pgfplots}
\pgfplotsset{compat=1.3}
\end{codeexample}
		in your preamble.
		
		\item Version 1.3 fixes a bug with |anchor=south west| which occurred mainly for reversed axes: the anchors where upside--down. This is fixed now.

		
		If you have written some work--around and want to keep it going, use

		|\pgfplotsset{compat/anchors=pre 1.3}|\pgfmanualpdflabel{/pgfplots/compat/anchors}{} 

		to disable the bug fix.
	\end{enumerate}

	Use |\pgfplotsset{compat=default}| to restore the factory settings.

	The setting |\pgfplotsset{compat=newest}| will always use features of the most recent version. This might result in small changes in the document's appearance.
\end{pgfplotskey}

\subsection{The Team}
\PGFPlots\ has been written mainly by Christian Feuers�nger with many improvements of Pascal Wolkotte and Nick Papior Andersen as a spare time project. We hope it is useful and provides valuable plots.

If you are interested in writing something but don't know how, consider reading the auxiliary manual \href{file:TeX-programming-notes.pdf}{TeX-programming-notes.pdf} which comes with \PGFPlots. It is far from complete, but maybe it is a good starting point (at least for more literature).

\subsection{Acknowledgements}
I thank God for all hours of enjoyed programming. I thank Pascal Wolkotte and Nick Papior Andersen for their programming efforts and contributions as part of the development team. I thank Stefan Tibus, who contributed the |plot shell| feature. I thank Tom Cashman for the contribution of the |reverse legend| feature. Special thanks go to Stefan Pinnow whose tests of \PGFPlots\ lead to numerous quality improvements. Furthermore, I thank Dr.~Schweitzer for many fruitful discussions and Dr.~Meine for his ideas and suggestions. Thanks as well to the many international contributors who provided feature requests or identified bugs or simply manual improvements!

Last but not least, I thank Till Tantau and Mark Wibrow for their excellent graphics (and more) package \PGF\ and \Tikz\ which is the base of \PGFPlots.

\include{pgfplots.install}%
\include{pgfplots.intro}%
\include{pgfplots.reference}%
\include{pgfplots.libs}%
\include{pgfplots.resources}%
\include{pgfplots.importexport}%
\include{pgfplots.basic.reference}%

\printindex

\bibliographystyle{abbrv} %gerapali} %gerabbrv} %gerunsrt.bst} %gerabbrv}% gerplain}
\nocite{pgfplotstable}
\nocite{programmingnotes}
\bibliography{pgfplots}
\end{document}
%
%%%%%%%%%%%%%%%%%%%%%%%%%%%%%%%%%%%%%%%%%%%%%%%%%%%%%%%%%%%%%%%%%%%%%%%%%%%%%
%
% Package pgfplots.sty documentation. 
%
% Copyright 2007/2008 by Christian Feuersaenger.
%
% This program is free software: you can redistribute it and/or modify
% it under the terms of the GNU General Public License as published by
% the Free Software Foundation, either version 3 of the License, or
% (at your option) any later version.
% 
% This program is distributed in the hope that it will be useful,
% but WITHOUT ANY WARRANTY; without even the implied warranty of
% MERCHANTABILITY or FITNESS FOR A PARTICULAR PURPOSE.  See the
% GNU General Public License for more details.
% 
% You should have received a copy of the GNU General Public License
% along with this program.  If not, see <http://www.gnu.org/licenses/>.
%
%
%%%%%%%%%%%%%%%%%%%%%%%%%%%%%%%%%%%%%%%%%%%%%%%%%%%%%%%%%%%%%%%%%%%%%%%%%%%%%
\input{pgfplots.preamble.tex}
%\RequirePackage[german,english,francais]{babel}

\pgfplotsmanualexternalexpensivetrue

%\usetikzlibrary{external}
%\tikzexternalize[prefix=figures/]{pgfplots}

\title{%
	Manual for Package \PGFPlots\\
	{\small Version \pgfplotsversion}\\
	{\small\href{http://sourceforge.net/projects/pgfplots}{http://sourceforge.net/projects/pgfplots}}}

%\includeonly{pgfplots.libs}


\begin{document}

\def\plotcoords{%
\addplot coordinates {
(5,8.312e-02)    (17,2.547e-02)   (49,7.407e-03)
(129,2.102e-03)  (321,5.874e-04)  (769,1.623e-04)
(1793,4.442e-05) (4097,1.207e-05) (9217,3.261e-06)
};

\addplot coordinates{
(7,8.472e-02)    (31,3.044e-02)    (111,1.022e-02)
(351,3.303e-03)  (1023,1.039e-03)  (2815,3.196e-04)
(7423,9.658e-05) (18943,2.873e-05) (47103,8.437e-06)
};

\addplot coordinates{
(9,7.881e-02)     (49,3.243e-02)    (209,1.232e-02)
(769,4.454e-03)   (2561,1.551e-03)  (7937,5.236e-04)
(23297,1.723e-04) (65537,5.545e-05) (178177,1.751e-05)
};

\addplot coordinates{
(11,6.887e-02)    (71,3.177e-02)     (351,1.341e-02)
(1471,5.334e-03)  (5503,2.027e-03)   (18943,7.415e-04)
(61183,2.628e-04) (187903,9.063e-05) (553983,3.053e-05)
};

\addplot coordinates{
(13,5.755e-02)     (97,2.925e-02)     (545,1.351e-02)
(2561,5.842e-03)   (10625,2.397e-03)  (40193,9.414e-04)
(141569,3.564e-04) (471041,1.308e-04) 
(1496065,4.670e-05)
};
}%
\maketitle
\begin{abstract}%
\PGFPlots\ draws high--quality function plots in normal or logarithmic scaling with a user-friendly interface directly in \TeX. The user supplies axis labels, legend entries and the plot coordinates for one or more plots and \PGFPlots\ applies axis scaling, computes any logarithms and axis ticks and draws the plots, supporting line plots, scatter plots, piecewise constant plots, bar plots, area plots, mesh-- and surface plots and some more. It is based on Till Tantau's package \PGF/\Tikz.
\end{abstract}
\tableofcontents
\section{Introduction}
This package provides tools to generate plots and labeled axes easily. It draws normal plots, logplots and semi-logplots, in two and three dimensions. Axis ticks, labels, legends (in case of multiple plots) can be added with key-value options. It can cycle through a set of predefined line/marker/color specifications. In summary, its purpose is to simplify the generation of high-quality function and/or data plots, and solving the problems of
\begin{itemize}
	\item consistency of document and font type and font size,
	\item direct use of \TeX\ math mode in axis descriptions,
	\item consistency of data and figures (no third party tool necessary),
	\item inter document consistency using preamble configurations and styles.
\end{itemize}
Although not necessary, separate |.pdf| or |.eps| graphics can be generated using the |external| library developed as part of \Tikz.

\PGFPlots\ is build completely on \Tikz/\PGF. Knowledge of \Tikz\ will simplify the work with \PGFPlots, although it is not required.

\section[About PGFPlots: Preliminaries]{About {\normalfont\PGFPlots}: Preliminaries}
This section contains information about upgrades, the team, the installation (in case you need to do it manually) and troubleshooting. You may skip it completely except for the upgrade remarks.

However, note that this library requires at least \PGF\ version $2.00$. At the time of this writing, many \TeX-distributions still contain the older \PGF\ version $1.18$, so it may be necessary to install a recent \PGF\ prior to using \PGFPlots.

\subsection{Components}
\PGFPlots\ comes with two components:
\begin{enumerate}
	\item the plotting component (which you are currently reading) and
	\item the \PGFPlotstable\ component which simplifies number formatting and postprocessing of numerical tables. It comes as a separate package and has its own manual \href{file:pgfplotstable.pdf}{pgfplotstable.pdf}.
\end{enumerate}

\subsection{Upgrade remarks}
This release provides a lot of improvements which can be found in all detail in \texttt{ChangeLog} for interested readers. However, some attention is useful with respect to the following changes.

\subsubsection{New Optional Features}
\begin{enumerate}
	\item \PGFPlots\ 1.3 comes with user interface improvements. The technical distinction between ``behavior options'' and ``style options'' of older versions is no longer necessary (although still fully supported).

	\item \PGFPlots\ 1.3 has a new feature which allows to \emph{move axis labels tight to tick labels} automatically.

	Since this affects the spacing, it is not enabled be default.

	Use
\begin{codeexample}[code only]
\usepackage{pgfplots}
\pgfplotsset{compat=1.3}
\end{codeexample}
	in your preamble to benefit from the improved spacing\footnote{The |compat=1.3| setting is necessary for new features which move axis labels around like reversed axis. However, \PGFPlots\ will still work as it does in earlier versions, your documents will remain the same if you don't set it explicitly.}.
	Take a look at the next page for the precise definition of |compat|.

	\item \PGFPlots\ 1.3 now supports reversed axes. It is no longer necessary to use work--arounds with negative units.
\pgfkeys{/pdflinks/search key prefixes in/.add={/pgfplots/,}{}}

	Take a look at the |x dir=reverse| key.

	Existing work--arounds will still function properly. Use |\pgfplotsset{compat=1.3}| together with |x dir=reverse| to switch to the new version.
\end{enumerate}

\subsubsection{Old Features Which May Need Attention}
\begin{enumerate}
	\item The |scatter/classes| feature produces proper legends as of version 1.3. This may change the appearance of existing legends of plots with |scatter/classes|.

	\item Due to a small math bug in \PGF\ $2.00$, you \emph{can't} use the math expression `|-x^2|'. It is necessary to use `|0-x^2|' instead. The same holds for
		`|exp(-x^2)|'; use `|exp(0-x^2)|' instead.

		This will be fixed with \PGF\ $>2.00$.

	\item Starting with \PGFPlots\ $1.1$, |\tikzstyle| should \emph{no longer be used} to set \PGFPlots\ options.
	
	Although |\tikzstyle| is still supported for some older \PGFPlots\ options, you should replace any occurance of |\tikzstyle| with |\pgfplotsset{|\meta{style name}|/.style={|\meta{key-value-list}|}}| or the associated |/.append style| variant. See section~\ref{sec:styles} for more detail.
\end{enumerate}
I apologize for any inconvenience caused by these changes.

\begin{pgfplotskey}{compat=\mchoice{1.3,pre 1.3,default,newest} (initially default)}
	The preamble configuration 
\begin{codeexample}[code only]
\usepackage{pgfplots}
\pgfplotsset{compat=1.3}
\end{codeexample}
	allows to choose between backwards compatibility and most recent features.

	\PGFPlots\ version 1.3 comes with new features which may lead to different spacings compared to earlier versions:
	\begin{enumerate}
		\item Version 1.3 comes with the |xlabel near ticks| feature which places (in this case $x$) axis labels ``near'' the tick labels, taking the size of tick labels into account.
		
		Older versions used |xlabel absolute|, i.e.\ an absolute distance between the axis and axis labels (now matter how large or small tick labels are).

		The initial setting \emph{keeps backwards compatibility}.

		You are encouraged to use
\begin{codeexample}[code only]
\usepackage{pgfplots}
\pgfplotsset{compat=1.3}
\end{codeexample}
		in your preamble.
		
		\item Version 1.3 fixes a bug with |anchor=south west| which occurred mainly for reversed axes: the anchors where upside--down. This is fixed now.

		
		If you have written some work--around and want to keep it going, use

		|\pgfplotsset{compat/anchors=pre 1.3}|\pgfmanualpdflabel{/pgfplots/compat/anchors}{} 

		to disable the bug fix.
	\end{enumerate}

	Use |\pgfplotsset{compat=default}| to restore the factory settings.

	The setting |\pgfplotsset{compat=newest}| will always use features of the most recent version. This might result in small changes in the document's appearance.
\end{pgfplotskey}

\subsection{The Team}
\PGFPlots\ has been written mainly by Christian Feuers�nger with many improvements of Pascal Wolkotte and Nick Papior Andersen as a spare time project. We hope it is useful and provides valuable plots.

If you are interested in writing something but don't know how, consider reading the auxiliary manual \href{file:TeX-programming-notes.pdf}{TeX-programming-notes.pdf} which comes with \PGFPlots. It is far from complete, but maybe it is a good starting point (at least for more literature).

\subsection{Acknowledgements}
I thank God for all hours of enjoyed programming. I thank Pascal Wolkotte and Nick Papior Andersen for their programming efforts and contributions as part of the development team. I thank Stefan Tibus, who contributed the |plot shell| feature. I thank Tom Cashman for the contribution of the |reverse legend| feature. Special thanks go to Stefan Pinnow whose tests of \PGFPlots\ lead to numerous quality improvements. Furthermore, I thank Dr.~Schweitzer for many fruitful discussions and Dr.~Meine for his ideas and suggestions. Thanks as well to the many international contributors who provided feature requests or identified bugs or simply manual improvements!

Last but not least, I thank Till Tantau and Mark Wibrow for their excellent graphics (and more) package \PGF\ and \Tikz\ which is the base of \PGFPlots.

\include{pgfplots.install}%
\include{pgfplots.intro}%
\include{pgfplots.reference}%
\include{pgfplots.libs}%
\include{pgfplots.resources}%
\include{pgfplots.importexport}%
\include{pgfplots.basic.reference}%

\printindex

\bibliographystyle{abbrv} %gerapali} %gerabbrv} %gerunsrt.bst} %gerabbrv}% gerplain}
\nocite{pgfplotstable}
\nocite{programmingnotes}
\bibliography{pgfplots}
\end{document}
%
%%%%%%%%%%%%%%%%%%%%%%%%%%%%%%%%%%%%%%%%%%%%%%%%%%%%%%%%%%%%%%%%%%%%%%%%%%%%%
%
% Package pgfplots.sty documentation. 
%
% Copyright 2007/2008 by Christian Feuersaenger.
%
% This program is free software: you can redistribute it and/or modify
% it under the terms of the GNU General Public License as published by
% the Free Software Foundation, either version 3 of the License, or
% (at your option) any later version.
% 
% This program is distributed in the hope that it will be useful,
% but WITHOUT ANY WARRANTY; without even the implied warranty of
% MERCHANTABILITY or FITNESS FOR A PARTICULAR PURPOSE.  See the
% GNU General Public License for more details.
% 
% You should have received a copy of the GNU General Public License
% along with this program.  If not, see <http://www.gnu.org/licenses/>.
%
%
%%%%%%%%%%%%%%%%%%%%%%%%%%%%%%%%%%%%%%%%%%%%%%%%%%%%%%%%%%%%%%%%%%%%%%%%%%%%%
\input{pgfplots.preamble.tex}
%\RequirePackage[german,english,francais]{babel}

\pgfplotsmanualexternalexpensivetrue

%\usetikzlibrary{external}
%\tikzexternalize[prefix=figures/]{pgfplots}

\title{%
	Manual for Package \PGFPlots\\
	{\small Version \pgfplotsversion}\\
	{\small\href{http://sourceforge.net/projects/pgfplots}{http://sourceforge.net/projects/pgfplots}}}

%\includeonly{pgfplots.libs}


\begin{document}

\def\plotcoords{%
\addplot coordinates {
(5,8.312e-02)    (17,2.547e-02)   (49,7.407e-03)
(129,2.102e-03)  (321,5.874e-04)  (769,1.623e-04)
(1793,4.442e-05) (4097,1.207e-05) (9217,3.261e-06)
};

\addplot coordinates{
(7,8.472e-02)    (31,3.044e-02)    (111,1.022e-02)
(351,3.303e-03)  (1023,1.039e-03)  (2815,3.196e-04)
(7423,9.658e-05) (18943,2.873e-05) (47103,8.437e-06)
};

\addplot coordinates{
(9,7.881e-02)     (49,3.243e-02)    (209,1.232e-02)
(769,4.454e-03)   (2561,1.551e-03)  (7937,5.236e-04)
(23297,1.723e-04) (65537,5.545e-05) (178177,1.751e-05)
};

\addplot coordinates{
(11,6.887e-02)    (71,3.177e-02)     (351,1.341e-02)
(1471,5.334e-03)  (5503,2.027e-03)   (18943,7.415e-04)
(61183,2.628e-04) (187903,9.063e-05) (553983,3.053e-05)
};

\addplot coordinates{
(13,5.755e-02)     (97,2.925e-02)     (545,1.351e-02)
(2561,5.842e-03)   (10625,2.397e-03)  (40193,9.414e-04)
(141569,3.564e-04) (471041,1.308e-04) 
(1496065,4.670e-05)
};
}%
\maketitle
\begin{abstract}%
\PGFPlots\ draws high--quality function plots in normal or logarithmic scaling with a user-friendly interface directly in \TeX. The user supplies axis labels, legend entries and the plot coordinates for one or more plots and \PGFPlots\ applies axis scaling, computes any logarithms and axis ticks and draws the plots, supporting line plots, scatter plots, piecewise constant plots, bar plots, area plots, mesh-- and surface plots and some more. It is based on Till Tantau's package \PGF/\Tikz.
\end{abstract}
\tableofcontents
\section{Introduction}
This package provides tools to generate plots and labeled axes easily. It draws normal plots, logplots and semi-logplots, in two and three dimensions. Axis ticks, labels, legends (in case of multiple plots) can be added with key-value options. It can cycle through a set of predefined line/marker/color specifications. In summary, its purpose is to simplify the generation of high-quality function and/or data plots, and solving the problems of
\begin{itemize}
	\item consistency of document and font type and font size,
	\item direct use of \TeX\ math mode in axis descriptions,
	\item consistency of data and figures (no third party tool necessary),
	\item inter document consistency using preamble configurations and styles.
\end{itemize}
Although not necessary, separate |.pdf| or |.eps| graphics can be generated using the |external| library developed as part of \Tikz.

\PGFPlots\ is build completely on \Tikz/\PGF. Knowledge of \Tikz\ will simplify the work with \PGFPlots, although it is not required.

\section[About PGFPlots: Preliminaries]{About {\normalfont\PGFPlots}: Preliminaries}
This section contains information about upgrades, the team, the installation (in case you need to do it manually) and troubleshooting. You may skip it completely except for the upgrade remarks.

However, note that this library requires at least \PGF\ version $2.00$. At the time of this writing, many \TeX-distributions still contain the older \PGF\ version $1.18$, so it may be necessary to install a recent \PGF\ prior to using \PGFPlots.

\subsection{Components}
\PGFPlots\ comes with two components:
\begin{enumerate}
	\item the plotting component (which you are currently reading) and
	\item the \PGFPlotstable\ component which simplifies number formatting and postprocessing of numerical tables. It comes as a separate package and has its own manual \href{file:pgfplotstable.pdf}{pgfplotstable.pdf}.
\end{enumerate}

\subsection{Upgrade remarks}
This release provides a lot of improvements which can be found in all detail in \texttt{ChangeLog} for interested readers. However, some attention is useful with respect to the following changes.

\subsubsection{New Optional Features}
\begin{enumerate}
	\item \PGFPlots\ 1.3 comes with user interface improvements. The technical distinction between ``behavior options'' and ``style options'' of older versions is no longer necessary (although still fully supported).

	\item \PGFPlots\ 1.3 has a new feature which allows to \emph{move axis labels tight to tick labels} automatically.

	Since this affects the spacing, it is not enabled be default.

	Use
\begin{codeexample}[code only]
\usepackage{pgfplots}
\pgfplotsset{compat=1.3}
\end{codeexample}
	in your preamble to benefit from the improved spacing\footnote{The |compat=1.3| setting is necessary for new features which move axis labels around like reversed axis. However, \PGFPlots\ will still work as it does in earlier versions, your documents will remain the same if you don't set it explicitly.}.
	Take a look at the next page for the precise definition of |compat|.

	\item \PGFPlots\ 1.3 now supports reversed axes. It is no longer necessary to use work--arounds with negative units.
\pgfkeys{/pdflinks/search key prefixes in/.add={/pgfplots/,}{}}

	Take a look at the |x dir=reverse| key.

	Existing work--arounds will still function properly. Use |\pgfplotsset{compat=1.3}| together with |x dir=reverse| to switch to the new version.
\end{enumerate}

\subsubsection{Old Features Which May Need Attention}
\begin{enumerate}
	\item The |scatter/classes| feature produces proper legends as of version 1.3. This may change the appearance of existing legends of plots with |scatter/classes|.

	\item Due to a small math bug in \PGF\ $2.00$, you \emph{can't} use the math expression `|-x^2|'. It is necessary to use `|0-x^2|' instead. The same holds for
		`|exp(-x^2)|'; use `|exp(0-x^2)|' instead.

		This will be fixed with \PGF\ $>2.00$.

	\item Starting with \PGFPlots\ $1.1$, |\tikzstyle| should \emph{no longer be used} to set \PGFPlots\ options.
	
	Although |\tikzstyle| is still supported for some older \PGFPlots\ options, you should replace any occurance of |\tikzstyle| with |\pgfplotsset{|\meta{style name}|/.style={|\meta{key-value-list}|}}| or the associated |/.append style| variant. See section~\ref{sec:styles} for more detail.
\end{enumerate}
I apologize for any inconvenience caused by these changes.

\begin{pgfplotskey}{compat=\mchoice{1.3,pre 1.3,default,newest} (initially default)}
	The preamble configuration 
\begin{codeexample}[code only]
\usepackage{pgfplots}
\pgfplotsset{compat=1.3}
\end{codeexample}
	allows to choose between backwards compatibility and most recent features.

	\PGFPlots\ version 1.3 comes with new features which may lead to different spacings compared to earlier versions:
	\begin{enumerate}
		\item Version 1.3 comes with the |xlabel near ticks| feature which places (in this case $x$) axis labels ``near'' the tick labels, taking the size of tick labels into account.
		
		Older versions used |xlabel absolute|, i.e.\ an absolute distance between the axis and axis labels (now matter how large or small tick labels are).

		The initial setting \emph{keeps backwards compatibility}.

		You are encouraged to use
\begin{codeexample}[code only]
\usepackage{pgfplots}
\pgfplotsset{compat=1.3}
\end{codeexample}
		in your preamble.
		
		\item Version 1.3 fixes a bug with |anchor=south west| which occurred mainly for reversed axes: the anchors where upside--down. This is fixed now.

		
		If you have written some work--around and want to keep it going, use

		|\pgfplotsset{compat/anchors=pre 1.3}|\pgfmanualpdflabel{/pgfplots/compat/anchors}{} 

		to disable the bug fix.
	\end{enumerate}

	Use |\pgfplotsset{compat=default}| to restore the factory settings.

	The setting |\pgfplotsset{compat=newest}| will always use features of the most recent version. This might result in small changes in the document's appearance.
\end{pgfplotskey}

\subsection{The Team}
\PGFPlots\ has been written mainly by Christian Feuers�nger with many improvements of Pascal Wolkotte and Nick Papior Andersen as a spare time project. We hope it is useful and provides valuable plots.

If you are interested in writing something but don't know how, consider reading the auxiliary manual \href{file:TeX-programming-notes.pdf}{TeX-programming-notes.pdf} which comes with \PGFPlots. It is far from complete, but maybe it is a good starting point (at least for more literature).

\subsection{Acknowledgements}
I thank God for all hours of enjoyed programming. I thank Pascal Wolkotte and Nick Papior Andersen for their programming efforts and contributions as part of the development team. I thank Stefan Tibus, who contributed the |plot shell| feature. I thank Tom Cashman for the contribution of the |reverse legend| feature. Special thanks go to Stefan Pinnow whose tests of \PGFPlots\ lead to numerous quality improvements. Furthermore, I thank Dr.~Schweitzer for many fruitful discussions and Dr.~Meine for his ideas and suggestions. Thanks as well to the many international contributors who provided feature requests or identified bugs or simply manual improvements!

Last but not least, I thank Till Tantau and Mark Wibrow for their excellent graphics (and more) package \PGF\ and \Tikz\ which is the base of \PGFPlots.

\include{pgfplots.install}%
\include{pgfplots.intro}%
\include{pgfplots.reference}%
\include{pgfplots.libs}%
\include{pgfplots.resources}%
\include{pgfplots.importexport}%
\include{pgfplots.basic.reference}%

\printindex

\bibliographystyle{abbrv} %gerapali} %gerabbrv} %gerunsrt.bst} %gerabbrv}% gerplain}
\nocite{pgfplotstable}
\nocite{programmingnotes}
\bibliography{pgfplots}
\end{document}
%
%%%%%%%%%%%%%%%%%%%%%%%%%%%%%%%%%%%%%%%%%%%%%%%%%%%%%%%%%%%%%%%%%%%%%%%%%%%%%
%
% Package pgfplots.sty documentation. 
%
% Copyright 2007/2008 by Christian Feuersaenger.
%
% This program is free software: you can redistribute it and/or modify
% it under the terms of the GNU General Public License as published by
% the Free Software Foundation, either version 3 of the License, or
% (at your option) any later version.
% 
% This program is distributed in the hope that it will be useful,
% but WITHOUT ANY WARRANTY; without even the implied warranty of
% MERCHANTABILITY or FITNESS FOR A PARTICULAR PURPOSE.  See the
% GNU General Public License for more details.
% 
% You should have received a copy of the GNU General Public License
% along with this program.  If not, see <http://www.gnu.org/licenses/>.
%
%
%%%%%%%%%%%%%%%%%%%%%%%%%%%%%%%%%%%%%%%%%%%%%%%%%%%%%%%%%%%%%%%%%%%%%%%%%%%%%
\input{pgfplots.preamble.tex}
%\RequirePackage[german,english,francais]{babel}

\pgfplotsmanualexternalexpensivetrue

%\usetikzlibrary{external}
%\tikzexternalize[prefix=figures/]{pgfplots}

\title{%
	Manual for Package \PGFPlots\\
	{\small Version \pgfplotsversion}\\
	{\small\href{http://sourceforge.net/projects/pgfplots}{http://sourceforge.net/projects/pgfplots}}}

%\includeonly{pgfplots.libs}


\begin{document}

\def\plotcoords{%
\addplot coordinates {
(5,8.312e-02)    (17,2.547e-02)   (49,7.407e-03)
(129,2.102e-03)  (321,5.874e-04)  (769,1.623e-04)
(1793,4.442e-05) (4097,1.207e-05) (9217,3.261e-06)
};

\addplot coordinates{
(7,8.472e-02)    (31,3.044e-02)    (111,1.022e-02)
(351,3.303e-03)  (1023,1.039e-03)  (2815,3.196e-04)
(7423,9.658e-05) (18943,2.873e-05) (47103,8.437e-06)
};

\addplot coordinates{
(9,7.881e-02)     (49,3.243e-02)    (209,1.232e-02)
(769,4.454e-03)   (2561,1.551e-03)  (7937,5.236e-04)
(23297,1.723e-04) (65537,5.545e-05) (178177,1.751e-05)
};

\addplot coordinates{
(11,6.887e-02)    (71,3.177e-02)     (351,1.341e-02)
(1471,5.334e-03)  (5503,2.027e-03)   (18943,7.415e-04)
(61183,2.628e-04) (187903,9.063e-05) (553983,3.053e-05)
};

\addplot coordinates{
(13,5.755e-02)     (97,2.925e-02)     (545,1.351e-02)
(2561,5.842e-03)   (10625,2.397e-03)  (40193,9.414e-04)
(141569,3.564e-04) (471041,1.308e-04) 
(1496065,4.670e-05)
};
}%
\maketitle
\begin{abstract}%
\PGFPlots\ draws high--quality function plots in normal or logarithmic scaling with a user-friendly interface directly in \TeX. The user supplies axis labels, legend entries and the plot coordinates for one or more plots and \PGFPlots\ applies axis scaling, computes any logarithms and axis ticks and draws the plots, supporting line plots, scatter plots, piecewise constant plots, bar plots, area plots, mesh-- and surface plots and some more. It is based on Till Tantau's package \PGF/\Tikz.
\end{abstract}
\tableofcontents
\section{Introduction}
This package provides tools to generate plots and labeled axes easily. It draws normal plots, logplots and semi-logplots, in two and three dimensions. Axis ticks, labels, legends (in case of multiple plots) can be added with key-value options. It can cycle through a set of predefined line/marker/color specifications. In summary, its purpose is to simplify the generation of high-quality function and/or data plots, and solving the problems of
\begin{itemize}
	\item consistency of document and font type and font size,
	\item direct use of \TeX\ math mode in axis descriptions,
	\item consistency of data and figures (no third party tool necessary),
	\item inter document consistency using preamble configurations and styles.
\end{itemize}
Although not necessary, separate |.pdf| or |.eps| graphics can be generated using the |external| library developed as part of \Tikz.

\PGFPlots\ is build completely on \Tikz/\PGF. Knowledge of \Tikz\ will simplify the work with \PGFPlots, although it is not required.

\section[About PGFPlots: Preliminaries]{About {\normalfont\PGFPlots}: Preliminaries}
This section contains information about upgrades, the team, the installation (in case you need to do it manually) and troubleshooting. You may skip it completely except for the upgrade remarks.

However, note that this library requires at least \PGF\ version $2.00$. At the time of this writing, many \TeX-distributions still contain the older \PGF\ version $1.18$, so it may be necessary to install a recent \PGF\ prior to using \PGFPlots.

\subsection{Components}
\PGFPlots\ comes with two components:
\begin{enumerate}
	\item the plotting component (which you are currently reading) and
	\item the \PGFPlotstable\ component which simplifies number formatting and postprocessing of numerical tables. It comes as a separate package and has its own manual \href{file:pgfplotstable.pdf}{pgfplotstable.pdf}.
\end{enumerate}

\subsection{Upgrade remarks}
This release provides a lot of improvements which can be found in all detail in \texttt{ChangeLog} for interested readers. However, some attention is useful with respect to the following changes.

\subsubsection{New Optional Features}
\begin{enumerate}
	\item \PGFPlots\ 1.3 comes with user interface improvements. The technical distinction between ``behavior options'' and ``style options'' of older versions is no longer necessary (although still fully supported).

	\item \PGFPlots\ 1.3 has a new feature which allows to \emph{move axis labels tight to tick labels} automatically.

	Since this affects the spacing, it is not enabled be default.

	Use
\begin{codeexample}[code only]
\usepackage{pgfplots}
\pgfplotsset{compat=1.3}
\end{codeexample}
	in your preamble to benefit from the improved spacing\footnote{The |compat=1.3| setting is necessary for new features which move axis labels around like reversed axis. However, \PGFPlots\ will still work as it does in earlier versions, your documents will remain the same if you don't set it explicitly.}.
	Take a look at the next page for the precise definition of |compat|.

	\item \PGFPlots\ 1.3 now supports reversed axes. It is no longer necessary to use work--arounds with negative units.
\pgfkeys{/pdflinks/search key prefixes in/.add={/pgfplots/,}{}}

	Take a look at the |x dir=reverse| key.

	Existing work--arounds will still function properly. Use |\pgfplotsset{compat=1.3}| together with |x dir=reverse| to switch to the new version.
\end{enumerate}

\subsubsection{Old Features Which May Need Attention}
\begin{enumerate}
	\item The |scatter/classes| feature produces proper legends as of version 1.3. This may change the appearance of existing legends of plots with |scatter/classes|.

	\item Due to a small math bug in \PGF\ $2.00$, you \emph{can't} use the math expression `|-x^2|'. It is necessary to use `|0-x^2|' instead. The same holds for
		`|exp(-x^2)|'; use `|exp(0-x^2)|' instead.

		This will be fixed with \PGF\ $>2.00$.

	\item Starting with \PGFPlots\ $1.1$, |\tikzstyle| should \emph{no longer be used} to set \PGFPlots\ options.
	
	Although |\tikzstyle| is still supported for some older \PGFPlots\ options, you should replace any occurance of |\tikzstyle| with |\pgfplotsset{|\meta{style name}|/.style={|\meta{key-value-list}|}}| or the associated |/.append style| variant. See section~\ref{sec:styles} for more detail.
\end{enumerate}
I apologize for any inconvenience caused by these changes.

\begin{pgfplotskey}{compat=\mchoice{1.3,pre 1.3,default,newest} (initially default)}
	The preamble configuration 
\begin{codeexample}[code only]
\usepackage{pgfplots}
\pgfplotsset{compat=1.3}
\end{codeexample}
	allows to choose between backwards compatibility and most recent features.

	\PGFPlots\ version 1.3 comes with new features which may lead to different spacings compared to earlier versions:
	\begin{enumerate}
		\item Version 1.3 comes with the |xlabel near ticks| feature which places (in this case $x$) axis labels ``near'' the tick labels, taking the size of tick labels into account.
		
		Older versions used |xlabel absolute|, i.e.\ an absolute distance between the axis and axis labels (now matter how large or small tick labels are).

		The initial setting \emph{keeps backwards compatibility}.

		You are encouraged to use
\begin{codeexample}[code only]
\usepackage{pgfplots}
\pgfplotsset{compat=1.3}
\end{codeexample}
		in your preamble.
		
		\item Version 1.3 fixes a bug with |anchor=south west| which occurred mainly for reversed axes: the anchors where upside--down. This is fixed now.

		
		If you have written some work--around and want to keep it going, use

		|\pgfplotsset{compat/anchors=pre 1.3}|\pgfmanualpdflabel{/pgfplots/compat/anchors}{} 

		to disable the bug fix.
	\end{enumerate}

	Use |\pgfplotsset{compat=default}| to restore the factory settings.

	The setting |\pgfplotsset{compat=newest}| will always use features of the most recent version. This might result in small changes in the document's appearance.
\end{pgfplotskey}

\subsection{The Team}
\PGFPlots\ has been written mainly by Christian Feuers�nger with many improvements of Pascal Wolkotte and Nick Papior Andersen as a spare time project. We hope it is useful and provides valuable plots.

If you are interested in writing something but don't know how, consider reading the auxiliary manual \href{file:TeX-programming-notes.pdf}{TeX-programming-notes.pdf} which comes with \PGFPlots. It is far from complete, but maybe it is a good starting point (at least for more literature).

\subsection{Acknowledgements}
I thank God for all hours of enjoyed programming. I thank Pascal Wolkotte and Nick Papior Andersen for their programming efforts and contributions as part of the development team. I thank Stefan Tibus, who contributed the |plot shell| feature. I thank Tom Cashman for the contribution of the |reverse legend| feature. Special thanks go to Stefan Pinnow whose tests of \PGFPlots\ lead to numerous quality improvements. Furthermore, I thank Dr.~Schweitzer for many fruitful discussions and Dr.~Meine for his ideas and suggestions. Thanks as well to the many international contributors who provided feature requests or identified bugs or simply manual improvements!

Last but not least, I thank Till Tantau and Mark Wibrow for their excellent graphics (and more) package \PGF\ and \Tikz\ which is the base of \PGFPlots.

\include{pgfplots.install}%
\include{pgfplots.intro}%
\include{pgfplots.reference}%
\include{pgfplots.libs}%
\include{pgfplots.resources}%
\include{pgfplots.importexport}%
\include{pgfplots.basic.reference}%

\printindex

\bibliographystyle{abbrv} %gerapali} %gerabbrv} %gerunsrt.bst} %gerabbrv}% gerplain}
\nocite{pgfplotstable}
\nocite{programmingnotes}
\bibliography{pgfplots}
\end{document}
%
\include{pgfplots.basic.reference}%

\printindex

\bibliographystyle{abbrv} %gerapali} %gerabbrv} %gerunsrt.bst} %gerabbrv}% gerplain}
\nocite{pgfplotstable}
\nocite{programmingnotes}
\bibliography{pgfplots}
\end{document}
%
\include{pgfplots.basic.reference}%

\printindex

\bibliographystyle{abbrv} %gerapali} %gerabbrv} %gerunsrt.bst} %gerabbrv}% gerplain}
\nocite{pgfplotstable}
\nocite{programmingnotes}
\bibliography{pgfplots}
\end{document}
%
\include{pgfplots.basic.reference}%

\printindex

\bibliographystyle{abbrv} %gerapali} %gerabbrv} %gerunsrt.bst} %gerabbrv}% gerplain}
\nocite{pgfplotstable}
\nocite{programmingnotes}
\bibliography{pgfplots}
\end{document}
