%%%%%%%%%%%%%%%%%%%%%%%%%%%%%%%%%%%%%%%%%%%%%%%%%%%%%%%%%%%%%%%%%%%%%%%%%%%%%
%
% Package pgfplots.sty documentation. 
%
% Copyright 2007/2008 by Christian Feuersaenger.
%
% This program is free software: you can redistribute it and/or modify
% it under the terms of the GNU General Public License as published by
% the Free Software Foundation, either version 3 of the License, or
% (at your option) any later version.
% 
% This program is distributed in the hope that it will be useful,
% but WITHOUT ANY WARRANTY; without even the implied warranty of
% MERCHANTABILITY or FITNESS FOR A PARTICULAR PURPOSE.  See the
% GNU General Public License for more details.
% 
% You should have received a copy of the GNU General Public License
% along with this program.  If not, see <http://www.gnu.org/licenses/>.
%
%
%%%%%%%%%%%%%%%%%%%%%%%%%%%%%%%%%%%%%%%%%%%%%%%%%%%%%%%%%%%%%%%%%%%%%%%%%%%%%

%%%%%%%%%%%%%%%%%%%%%%%%%%%%%%%%%%%%%%%%%%%%%%%%%%%%%%%%%%%%%%%%%%%%%%%%%%%%%
%% Packages loaded in this file
% \usepackage{makeidx}
% \usepackage{ifpdf}
% \usepackage[pdfborder=0 0 0]{hyperref}
% \usepackage{textcomp}
% \usepackage{booktabs}
% \usepackage{calc}
% \usepackage[formats]{listings}
% \usepackage{array}
% \usepackage{pgfplots}
% \usepackage{pgfplotstable}
% \usepackage[a4paper,left=2.25cm,right=2.25cm,top=2.5cm,bottom=2.5cm,nohead]{geometry}
% \usepackage{amsmath,amssymb}
% \usepackage{xxcolor}
% \usepackage{pifont}
% \usepackage[latin1]{inputenc}
% \usepackage{amsmath}
% \usepackage{eurosym}
% \usepackage{nicefrac}
% \input{pgfplots-macros}
% \usepackage{nicefrac}
%%%%%%%%%%%%%%%%%%%%%%%%%%%%%%%%%%%%%%%%%%%%%%%%%%%%%%%%%%%%%%%%%%%%%%%%%%%%%

%%%%%%%%%%%%%%%%%%%%%%%%%%%%%%%%%%%%%%%%%%%%%%%%%%%%%%%%%%%%%%%%%%%%%%%%%%%%%
%% TikZ libraries
% \usepgfplotslibrary{dateplot,units,groupplots}
% \usetikzlibrary{backgrounds,patterns}
%%%%%%%%%%%%%%%%%%%%%%%%%%%%%%%%%%%%%%%%%%%%%%%%%%%%%%%%%%%%%%%%%%%%%%%%%%%%%

%%%%%%%%%%%%%%%%%%%%%%%%%%%%%%%%%%%%%%%%%%%%%%%%%%%%%%%%%%%%%%%%%%%%%%%%%%%%%
%%
%% 2009-02-15 Create the file base on pgfplots.preamble.tex.
%%
%% 2009-02-15 Comment the line '\documentclass[a4paper]{ltxdoc}'.
%%
%% 2010-09-19 Comment the lines before '\documentclass[a4paper]{ltxdoc}'.
%%
%%%%%%%%%%%%%%%%%%%%%%%%%%%%%%%%%%%%%%%%%%%%%%%%%%%%%%%%%%%%%%%%%%%%%%%%%%%%%

%% !!!
%% Changed by Dazhi Jiang - 2010-09-19
%% The following lines can be added in your main tex file
%%
% \pdfminorversion=5 % to allow compression
% \pdfobjcompresslevel=2
% \documentclass[a4paper]{ltxdoc}

\usepackage{makeidx}

% DON't let hyperref overload the format of index and glossary. 
% I want to do that on my own in the stylefiles for makeindex...
\makeatletter
\let\@old@wrindex=\@wrindex
\makeatother

\usepackage{ifpdf}
%% Load hyperref without any options. One of benefit of this is to
%% co-operate with some other document classes, e.g., memoir.
%% Option 'pdfborder' is special. It can be loaded using either one of the
%% following two syntaxes.
%% \usepackage[pdfborder={0 0 0}]{hyperref}
%% \hypersetup{pdfborder=0 0 0}
%% Note that the former one is protected by curly braces.
\usepackage{hyperref}
	\hypersetup{%
		colorlinks=true,	% use true to enable colors below:
		linkcolor=blue,%red,
		filecolor=blue,%magenta,
		%% !!!
		%% Changed by Dazhi Jiang - 2010-09-19
		%% Option 'pagecolor' has been removed in v6.76a, so the below line
		%% is commented.
		% pagecolor=blue,%red,
		urlcolor=blue,%cyan,
		citecolor=blue,
		%frenchlinks=false,	% small caps instead of colors
		%% The below line is not necessary, because the colorlinks=true will
		%% set pdfborder as 0pt automatically.
		pdfborder=0 0 0,	% PDF link-darstellung, falls colorlinks=false. 0 0 0: nix. 0 0 1: default.
		%plainpages=false,	% Das ist notwendig, wenn die Seitenzahlen z.T. in Arabischen und z.T. in römischen Ziffern gemacht werden.
		%% !!!
		%% Changed by Dazhi Jiang - 2010-09-19
		%% You can set up the below options by  \hypersetup{} command
		% pdftitle=Package PGFPLOTS manual,
		% pdfauthor=Christian Feuersänger,
		% %pdfsubject=,
		% pdfkeywords={pgfplots pgf tikz tex latex},
	}
%% !!!
%% The package 'hyperref' is quite special. The following lines are
%% commented because some options have been changed or removed in a recent
%% version of 'hyperref'.
% \usepackage[pdfborder=0 0 0]{hyperref}
% 	\hypersetup{%
% 		colorlinks=true,	% use true to enable colors below:
% 		linkcolor=blue,%red,
% 		filecolor=blue,%magenta,
% 		pagecolor=blue,%red,
% 		urlcolor=blue,%cyan,
% 		citecolor=blue,
% 		%frenchlinks=false,	% small caps instead of colors
% 		pdfborder=0 0 0,	% PDF link-darstellung, falls colorlinks=false. 0 0 0: nix. 0 0 1: default.
% 		%plainpages=false,	% Das ist notwendig, wenn die Seitenzahlen z.T. in Arabischen und z.T. in römischen Ziffern gemacht werden.
% 		pdftitle=Package PGFPLOTS manual,
% 		pdfauthor=Christian Feuersänger,
% 		%pdfsubject=,
% 		pdfkeywords={pgfplots pgf tikz tex latex},
% 	}

\makeatletter
\let\@wrindex=\@old@wrindex
\makeatother


\makeatletter
% disables colorlinks for all following \ref commands
\def\pgfplotsmanualdisablecolorforref{%
	\pgfutil@ifundefined{pgfplotsmanual@oldref}{%
		\let\pgfplotsmanual@oldref=\ref
	}{}%
	\def\ref##1{%
		\begingroup
		\let\Hy@colorlink=\pgfplots@disabled@Hy@colorlink
		\let\Hy@endcolorlink=\pgfplots@disabled@Hy@endcolorlink
		\pgfplotsmanual@oldref{##1}%
		\endgroup
	}%
}%
\def\pgfplots@disabled@Hy@colorlink#1{\begingroup}%
\def\pgfplots@disabled@Hy@endcolorlink{\endgroup}%
\makeatother

% Formatiere Seitennummern im Index:
\newcommand{\indexpageno}[1]{%
	{\bfseries\hyperpage{#1}}%
}


\newcommand{\R}{\mathbb{R}}
\newcommand{\N}{\mathbb{N}}
\newcommand{\Z}{\mathbb{Z}}

\long\def\COMMENTLOWLEVEL#1\ENDCOMMENT{}
\def\ENDCOMMENT{}

\usepackage{textcomp}
\usepackage{booktabs}

\usepackage{calc}
\usepackage[formats]{listings}
%\usepackage{courier} % don't use it - the '^' character can't be copy-pasted in courier

\usepackage{array}
\lstset{%
	basicstyle=\ttfamily,
	language=[LaTeX]tex, % Seems as if \lstset{language=tex} must be invoked BEFORE loading tikz!?
	tabsize=4,
	breaklines=true,
	breakindent=0pt
}

\ifpdf
	\pdfinfo {
		/Author	(Christian Feuersaenger)
	}

\else
%	\def\pgfsysdriver{pgfsys-dvipdfm.def}
\fi
%\def\pgfsysdriver{pgfsys-pdftex.def}
\usepackage{pgfplots}
\usepackage{pgfplotstable}

%% !!!
%% Changed by Dazhi Jiang - 2010-09-20
%% The below code snippet is commented because using 'clickable' library
%% causes some problems. 
%% TODO: need more investigation
% \ifpdf
% 	% this allows to disable the clickable lib from command line using
% 	% \pdflatex '\def\pgfplotsclickabledisabled{1}\input{pgfplots.tex}'
% 	\expandafter\ifx\csname pgfplotsclickabledisabled\endcsname\relax
% 		\usepgfplotslibrary{clickable}
% 	\fi
% \fi

%\usepackage{fp}
% ATTENTION:
% this requires pgf version NEWER than 2.00 :
%\usetikzlibrary{fixedpointarithmetic}

\usepgfplotslibrary{dateplot,units,groupplots}

\usepackage[a4paper,left=2.25cm,right=2.25cm,top=2.5cm,bottom=2.5cm,nohead]{geometry}
\usepackage{amsmath,amssymb}
\usepackage{xxcolor}
\usepackage{pifont}
\usepackage[latin1]{inputenc}
\usepackage{amsmath}
\usepackage{eurosym}
\usepackage{nicefrac}
\input{DJpgfplots-macros}

\usepackage{nicefrac}

\graphicspath{{figures/}}

\def\preambleconfig{width=7cm,compat=1.3}


\expandafter\pgfplotsset\expandafter{\preambleconfig}


\makeatletter
% And now, invoke
% 	/codeexample/typeset listing/.add={% Preamble:\pgfplotsset{\preambleconfig}}{}}
% since listings are VERBATIM, I need to do some low-level things
% here to get the correct \catcodes:
\pgfkeys{/codeexample/typeset listing/.add code={%
		\ifcode@execute
			\pgfutil@in@{axis}{#1}%
			\ifpgfutil@in@
				{\tiny
					\% Preamble: \pgfmanualpdfref{\textbackslash pgfplotsset}{\pgfmanual@pretty@backslash pgfplotsset}%
						\pgfmanual@pretty@lbrace \expandafter\pgfmanualprettyprintpgfkeys\expandafter{\preambleconfig}\pgfmanual@pretty@rbrace
				}%
			\fi
		\fi
	}{},%
	%/codeexample/typeset listing/.show code,
}%
\makeatother

\pgfplotsset{
	%every axis/.append style={width=7cm},
	filter discard warning=false,
}

\pgfqkeys{/codeexample}{%
	every codeexample/.append style={
		width=8cm,
		/pgfplots/legend style={fill=graphicbackground}
	},
	tabsize=4,
}

\usetikzlibrary{backgrounds,patterns}
% Global styles:
\tikzset{
  shape example/.style={
    color=black!30,
    draw,
    fill=yellow!30,
    line width=.5cm,
    inner xsep=2.5cm,
    inner ysep=0.5cm}
}

\newcommand{\FIXME}[1]{\textcolor{red}{(FIXME: #1)}}

% fuer endvironment 'sidewaysfigure' bspw
% \usepackage{rotating}

\newcommand\Tikz{Ti\textit kZ}
\newcommand\PGF{\textsc{pgf}}
\newcommand\PGFPlots{\pgfplotsmakefilelinkifuseful{pgfplots}{\textsc{pgfplots}}}
\newcommand\PGFPlotstable{\pgfplotsmakefilelinkifuseful{pgfplotstable}{\textsc{PgfplotsTable}}}

%% !!!
%% Change by Dazhi Jiang - 2010-09-20
%% The index can be made later in your preamble. So this line was commented.
% \makeindex

% Fix overful hboxes automatically:
\tolerance=2000
\emergencystretch=10pt

\tikzset{prefix=gnuplot/pgfplots_} % prefix for 'plot function'

%% !!!
%% Changed by Dazhi Jiang - 2010-09-19
% \author{%
% 	Christian Feuers\"anger\footnote{\url{http://wissrech.ins.uni-bonn.de/people/feuersaenger}}\\%
% 	Institut f\"ur Numerische Simulation\\
% 	Universit\"at Bonn, Germany}

