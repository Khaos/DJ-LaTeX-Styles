%
% 2010-09-06
% (c) Copyright 2010 Dazhi Jiang. All Rights Reserved. 

%% TextMate Settings
%!TEX TS-program = pdflatex
%!TEX TS-options = "-interaction=nonstopmode -file-line-error-style"

%% This part of comment has intent to calculate the number of stream used in it.
%% Input:
%%			.tex
%%			.aux
%% Output:
%%			\openout1 = `DJPgfplotsPreambleTest01.aux'.
%%			\DJPgfplotsPreambleTest01@idxfile=\write5
%%			\openout5 = `DJPgfplotsPreambleTest01.idx'.
%%			Writing index file DJPgfplotsPreambleTest01.idx 
%%			\DJPgfplotsPreambleTest01memglofile=\write6
%%			\openout6 = `DJPgfplotsPreambleTest01.glo'.
%%			Writing glossary file DJPgfplotsPreambleTest01.glo 
%%			\@notefile=\write7
%%			\openout7 = `DJPgfplotsPreambleTest01.ent'.
%%			\egoutoutstre@m=\write8
%%			\openout8 = `DJPgfplotsPreambleTest01.ego'.
%%			\lines@idxfile=\write9
%%			\openout9 = `lines.idx'.
%%			\openout12 = `DJPgfplotsPreambleTest01.out'.
%%			\tf@toc=\write13
%%			\openout13 = `DJPgfplotsPreambleTest01.toc'.
%%			\tf@lof=\write14
%%			\openout14 = `DJPgfplotsPreambleTest01.lof'.
%%			\tf@lot=\write15
%%			\openout15 = `DJPgfplotsPreambleTest01.lot'.
%% 					\makeindex[lines]
%%					\tableofcontents
%%					\listoffigures
%%					\listoftables
%%					!! Dropped for saving \write output streams
%%					\listofegresults
%%					\pgfplotsmanualexternalexpensivetrue
%%					!! Three additional streams in memsty.sty
%%					\makeglossary
%%					\makepagenote
%%					\newoutputstream{egout}
%%					  \openoutputfile{\jobname.ego}{egout}


\pdfminorversion=5 % to allow compression
\pdfobjcompresslevel=2

%% !!!
% \RequirePackage{etex} % This needs an additional output stream
\documentclass[10pt,a4paper,extrafontsizes]{memoir}
\listfiles
\usepackage{etex}
\usepackage{comment}


% For (non-printing) notes  \PWnote{date}{text}
\newcommand{\PWnote}[2]{} 
\PWnote{2009/04/29}{Added fonttable to the used packages}
\PWnote{2009/08/19}{Made Part I a separate doc (memdesign.tex).}

% same
\newcommand{\LMnote}[2]{} 


\usepackage{memsty}
%% Check out whether stream 'egout' is open
% \IfStreamOpen{egout}{\closeoutputstream{egout}}{}

%%%%%%%%%%%%%%%%%%%%%%%%%%%%
\usepackage{titlepages}  % code of the example titlepages
\usepackage{memlays}     % extra layout diagrams
\usepackage{dpfloat}     % floats on facing pages
\usepackage{fonttable}[2009/04/01]   % font tables
%%%%\usepackage{xr-hyper} \externaldocument{memdesign} Doesn't work, 
%%%%                      Idea won't work in general for memman/memdesin
%%%%                      as at display time, who knows where everything
%%%%                      will be located on the individual's computer.
%%%%%%%%%%%%%%%%%%%%%%%%%%%%

%%%% Change section heading styles
%%%\memmansecheads

%%%% Use the built-in division styling
\headstyles{memman}

%%% ToC down to subsections
\settocdepth{subsection}
%%% Numbering down to subsections as well
\setsecnumdepth{subsection}

%%%% extra index for first lines
%% !!!
%% Comment this line to save room for \write
%% This the second index in memoir manual, named index of the first lines
% \makeindex[lines]


% this 'if' is used to determine whether we are compiling the memoir
% master in the subversion repository, or the public memman.tex
\newif\ifMASTER
% \MASTERfalse
%% Note that an additional stream is require to enable this feature.
\MASTERtrue

\ifMASTER
% private package, not in circulation
% enables us to gather svn information on a single file basis
% \usepackage[filehooks]{svn-multi-private}
%% Use 'svn-multi' stead
\usepackage[filehooks]{svn-multi}


% \svnidlong
% {}
% {$LastChangedDate: 2010-07-01 00:21:04 +0200 (Thu, 01 Jul 2010) $}
% {$LastChangedRevision: 244 $}
% {$LastChangedBy: daleif $}



\makeatletter
\newcommand\addRevisionData{%
  \begin{picture}(0,0)%
    \put(0,-20){%
      \tiny%
      \expandafter\@ifmtarg\expandafter{\svnfiledate}{}{%
        \textit{\textcolor{darkgray}{Chapter last updated \svnfileyear/\svnfilemonth/\svnfileday
         \enspace (revision \svnfilerev)}}
     }%
    }%
  \end{picture}%
}
\makeatother

% we add this to the first page of each chapter

\makepagestyle{chapter}
\makeoddfoot{chapter}{\addRevisionData}{\thepage}{}
\makeevenfoot{chapter}{\addRevisionData}{\thepage}{}

\else
% disable svn info collecting
\newcommand\svnidlong[4]{}
\fi

%% Load DJ's pgfplots preamble
%%%%%%%%%%%%%%%%%%%%%%%%%%%%%%%%%%%%%%%%%%%%%%%%%%%%%%%%%%%%%%%%%%%%%%%%%%%%%
%
% Package pgfplots.sty documentation. 
%
% Copyright 2007/2008 by Christian Feuersaenger.
%
% This program is free software: you can redistribute it and/or modify
% it under the terms of the GNU General Public License as published by
% the Free Software Foundation, either version 3 of the License, or
% (at your option) any later version.
% 
% This program is distributed in the hope that it will be useful,
% but WITHOUT ANY WARRANTY; without even the implied warranty of
% MERCHANTABILITY or FITNESS FOR A PARTICULAR PURPOSE.  See the
% GNU General Public License for more details.
% 
% You should have received a copy of the GNU General Public License
% along with this program.  If not, see <http://www.gnu.org/licenses/>.
%
%
%%%%%%%%%%%%%%%%%%%%%%%%%%%%%%%%%%%%%%%%%%%%%%%%%%%%%%%%%%%%%%%%%%%%%%%%%%%%%

%%%%%%%%%%%%%%%%%%%%%%%%%%%%%%%%%%%%%%%%%%%%%%%%%%%%%%%%%%%%%%%%%%%%%%%%%%%%%
%% Packages loaded in this file
% \usepackage{makeidx}
% \usepackage{ifpdf}
% \usepackage[pdfborder=0 0 0]{hyperref}
% \usepackage{textcomp}
% \usepackage{booktabs}
% \usepackage{calc}
% \usepackage[formats]{listings}
% \usepackage{array}
% \usepackage{pgfplots}
% \usepackage{pgfplotstable}
% \usepackage[a4paper,left=2.25cm,right=2.25cm,top=2.5cm,bottom=2.5cm,nohead]{geometry}
% \usepackage{amsmath,amssymb}
% \usepackage{xxcolor}
% \usepackage{pifont}
% \usepackage[latin1]{inputenc}
% \usepackage{amsmath}
% \usepackage{eurosym}
% \usepackage{nicefrac}
% \input{pgfplots-macros}
% \usepackage{nicefrac}
%%%%%%%%%%%%%%%%%%%%%%%%%%%%%%%%%%%%%%%%%%%%%%%%%%%%%%%%%%%%%%%%%%%%%%%%%%%%%

%%%%%%%%%%%%%%%%%%%%%%%%%%%%%%%%%%%%%%%%%%%%%%%%%%%%%%%%%%%%%%%%%%%%%%%%%%%%%
%% TikZ libraries
% \usepgfplotslibrary{dateplot,units,groupplots}
% \usetikzlibrary{backgrounds,patterns}
%%%%%%%%%%%%%%%%%%%%%%%%%%%%%%%%%%%%%%%%%%%%%%%%%%%%%%%%%%%%%%%%%%%%%%%%%%%%%

%%%%%%%%%%%%%%%%%%%%%%%%%%%%%%%%%%%%%%%%%%%%%%%%%%%%%%%%%%%%%%%%%%%%%%%%%%%%%
%%
%% 2009-02-15 Create the file base on pgfplots.preamble.tex.
%%
%% 2009-02-15 Comment the line '\documentclass[a4paper]{ltxdoc}'.
%%
%% 2010-09-19 Comment the lines before '\documentclass[a4paper]{ltxdoc}'.
%%
%%%%%%%%%%%%%%%%%%%%%%%%%%%%%%%%%%%%%%%%%%%%%%%%%%%%%%%%%%%%%%%%%%%%%%%%%%%%%

%% !!!
%% Changed by Dazhi Jiang - 2010-09-19
%% The following lines can be added in your main tex file
%%
% \pdfminorversion=5 % to allow compression
% \pdfobjcompresslevel=2
% \documentclass[a4paper]{ltxdoc}

\usepackage{makeidx}

% DON't let hyperref overload the format of index and glossary. 
% I want to do that on my own in the stylefiles for makeindex...
\makeatletter
\let\@old@wrindex=\@wrindex
\makeatother

\usepackage{ifpdf}
%% Load hyperref without any options. One of benefit of this is to
%% co-operate with some other document classes, e.g., memoir.
%% Option 'pdfborder' is special. It can be loaded using either one of the
%% following two syntaxes.
%% \usepackage[pdfborder={0 0 0}]{hyperref}
%% \hypersetup{pdfborder=0 0 0}
%% Note that the former one is protected by curly braces.
\usepackage{hyperref}
	\hypersetup{%
		colorlinks=true,	% use true to enable colors below:
		linkcolor=blue,%red,
		filecolor=blue,%magenta,
		%% !!!
		%% Changed by Dazhi Jiang - 2010-09-19
		%% Option 'pagecolor' has been removed in v6.76a, so the below line
		%% is commented.
		% pagecolor=blue,%red,
		urlcolor=blue,%cyan,
		citecolor=blue,
		%frenchlinks=false,	% small caps instead of colors
		%% The below line is not necessary, because the colorlinks=true will
		%% set pdfborder as 0pt automatically.
		pdfborder=0 0 0,	% PDF link-darstellung, falls colorlinks=false. 0 0 0: nix. 0 0 1: default.
		%plainpages=false,	% Das ist notwendig, wenn die Seitenzahlen z.T. in Arabischen und z.T. in römischen Ziffern gemacht werden.
		%% !!!
		%% Changed by Dazhi Jiang - 2010-09-19
		%% You can set up the below options by  \hypersetup{} command
		% pdftitle=Package PGFPLOTS manual,
		% pdfauthor=Christian Feuersänger,
		% %pdfsubject=,
		% pdfkeywords={pgfplots pgf tikz tex latex},
	}
%% !!!
%% The package 'hyperref' is quite special. The following lines are
%% commented because some options have been changed or removed in a recent
%% version of 'hyperref'.
% \usepackage[pdfborder=0 0 0]{hyperref}
% 	\hypersetup{%
% 		colorlinks=true,	% use true to enable colors below:
% 		linkcolor=blue,%red,
% 		filecolor=blue,%magenta,
% 		pagecolor=blue,%red,
% 		urlcolor=blue,%cyan,
% 		citecolor=blue,
% 		%frenchlinks=false,	% small caps instead of colors
% 		pdfborder=0 0 0,	% PDF link-darstellung, falls colorlinks=false. 0 0 0: nix. 0 0 1: default.
% 		%plainpages=false,	% Das ist notwendig, wenn die Seitenzahlen z.T. in Arabischen und z.T. in römischen Ziffern gemacht werden.
% 		pdftitle=Package PGFPLOTS manual,
% 		pdfauthor=Christian Feuersänger,
% 		%pdfsubject=,
% 		pdfkeywords={pgfplots pgf tikz tex latex},
% 	}

\makeatletter
\let\@wrindex=\@old@wrindex
\makeatother


\makeatletter
% disables colorlinks for all following \ref commands
\def\pgfplotsmanualdisablecolorforref{%
	\pgfutil@ifundefined{pgfplotsmanual@oldref}{%
		\let\pgfplotsmanual@oldref=\ref
	}{}%
	\def\ref##1{%
		\begingroup
		\let\Hy@colorlink=\pgfplots@disabled@Hy@colorlink
		\let\Hy@endcolorlink=\pgfplots@disabled@Hy@endcolorlink
		\pgfplotsmanual@oldref{##1}%
		\endgroup
	}%
}%
\def\pgfplots@disabled@Hy@colorlink#1{\begingroup}%
\def\pgfplots@disabled@Hy@endcolorlink{\endgroup}%
\makeatother

% Formatiere Seitennummern im Index:
\newcommand{\indexpageno}[1]{%
	{\bfseries\hyperpage{#1}}%
}


\newcommand{\R}{\mathbb{R}}
\newcommand{\N}{\mathbb{N}}
\newcommand{\Z}{\mathbb{Z}}

\long\def\COMMENTLOWLEVEL#1\ENDCOMMENT{}
\def\ENDCOMMENT{}

\usepackage{textcomp}
\usepackage{booktabs}

\usepackage{calc}
\usepackage[formats]{listings}
%\usepackage{courier} % don't use it - the '^' character can't be copy-pasted in courier

\usepackage{array}
\lstset{%
	basicstyle=\ttfamily,
	language=[LaTeX]tex, % Seems as if \lstset{language=tex} must be invoked BEFORE loading tikz!?
	tabsize=4,
	breaklines=true,
	breakindent=0pt
}

\ifpdf
	\pdfinfo {
		/Author	(Christian Feuersaenger)
	}

\else
%	\def\pgfsysdriver{pgfsys-dvipdfm.def}
\fi
%\def\pgfsysdriver{pgfsys-pdftex.def}
\usepackage{pgfplots}
\usepackage{pgfplotstable}

%% !!!
%% Changed by Dazhi Jiang - 2010-09-20
%% The below code snippet is commented because using 'clickable' library
%% causes some problems. 
%% TODO: need more investigation
% \ifpdf
% 	% this allows to disable the clickable lib from command line using
% 	% \pdflatex '\def\pgfplotsclickabledisabled{1}\input{pgfplots.tex}'
% 	\expandafter\ifx\csname pgfplotsclickabledisabled\endcsname\relax
% 		\usepgfplotslibrary{clickable}
% 	\fi
% \fi

%\usepackage{fp}
% ATTENTION:
% this requires pgf version NEWER than 2.00 :
%\usetikzlibrary{fixedpointarithmetic}

\usepgfplotslibrary{dateplot,units,groupplots}

\usepackage[a4paper,left=2.25cm,right=2.25cm,top=2.5cm,bottom=2.5cm,nohead]{geometry}
\usepackage{amsmath,amssymb}
\usepackage{xxcolor}
\usepackage{pifont}
\usepackage[latin1]{inputenc}
\usepackage{amsmath}
\usepackage{eurosym}
\usepackage{nicefrac}
\input{DJpgfplots-macros}

\usepackage{nicefrac}

\graphicspath{{figures/}}

\def\preambleconfig{width=7cm,compat=1.3}


\expandafter\pgfplotsset\expandafter{\preambleconfig}


\makeatletter
% And now, invoke
% 	/codeexample/typeset listing/.add={% Preamble:\pgfplotsset{\preambleconfig}}{}}
% since listings are VERBATIM, I need to do some low-level things
% here to get the correct \catcodes:
\pgfkeys{/codeexample/typeset listing/.add code={%
		\ifcode@execute
			\pgfutil@in@{axis}{#1}%
			\ifpgfutil@in@
				{\tiny
					\% Preamble: \pgfmanualpdfref{\textbackslash pgfplotsset}{\pgfmanual@pretty@backslash pgfplotsset}%
						\pgfmanual@pretty@lbrace \expandafter\pgfmanualprettyprintpgfkeys\expandafter{\preambleconfig}\pgfmanual@pretty@rbrace
				}%
			\fi
		\fi
	}{},%
	%/codeexample/typeset listing/.show code,
}%
\makeatother

\pgfplotsset{
	%every axis/.append style={width=7cm},
	filter discard warning=false,
}

\pgfqkeys{/codeexample}{%
	every codeexample/.append style={
		width=8cm,
		/pgfplots/legend style={fill=graphicbackground}
	},
	tabsize=4,
}

\usetikzlibrary{backgrounds,patterns}
% Global styles:
\tikzset{
  shape example/.style={
    color=black!30,
    draw,
    fill=yellow!30,
    line width=.5cm,
    inner xsep=2.5cm,
    inner ysep=0.5cm}
}

\newcommand{\FIXME}[1]{\textcolor{red}{(FIXME: #1)}}

% fuer endvironment 'sidewaysfigure' bspw
% \usepackage{rotating}

\newcommand\Tikz{Ti\textit kZ}
\newcommand\PGF{\textsc{pgf}}
\newcommand\PGFPlots{\pgfplotsmakefilelinkifuseful{pgfplots}{\textsc{pgfplots}}}
\newcommand\PGFPlotstable{\pgfplotsmakefilelinkifuseful{pgfplotstable}{\textsc{PgfplotsTable}}}

%% !!!
%% Change by Dazhi Jiang - 2010-09-20
%% The index can be made later in your preamble. So this line was commented.
% \makeindex

% Fix overful hboxes automatically:
\tolerance=2000
\emergencystretch=10pt

\tikzset{prefix=gnuplot/pgfplots_} % prefix for 'plot function'

%% !!!
%% Changed by Dazhi Jiang - 2010-09-19
% \author{%
% 	Christian Feuers\"anger\footnote{\url{http://wissrech.ins.uni-bonn.de/people/feuersaenger}}\\%
% 	Institut f\"ur Numerische Simulation\\
% 	Universit\"at Bonn, Germany}


	
% \pgfplotsmanualexternalexpensivetrue


% \usepackage{hyperref}
% \hypersetup{pdfborder=0 0 0}
% \hypersetup{pdfborder=0 0 1}
% \usepackage{memhfixc}
% \usepackage{mempatch}

\usepackage{cleveref}

%% end preamble
%%%%%%%%%%%%%%%%%%%%%%%%%%%%%%%%%%%%%%%%%%%%%%%%%%%%%%%
%#% extend

\begin{document}

%#% extstart input intro.tex

\tightlists
%%%%\firmlists
\midsloppy
\raggedbottom
\chapterstyle{demo3}

%%%%%%%%%%%%%%%%%%%%%%%%%%%%%%%%%%%%%%%%%%%%%%%%%%%%%%%

% \input{memnoidxnum}

\frontmatter

%% TextMate Settings
%!TEX root = DJPgfplotsPreambleTest01.tex

\pagestyle{empty}

%% ++++++++++++++++++++++++++++++++++++++++++++++++++++++++++++
%% half-title page
\vspace*{\fill}
\begin{adjustwidth}{1in}{1in}
\begin{flushleft}
\HUGE\sffamily Notes
\end{flushleft}
\begin{center}
\HUGE\sffamily  on
\end{center}
\begin{flushright}
\HUGE\sffamily  PGF
\end{flushright}
%%\begin{center}
%%\sffamily (Draft Edition 7)
%%\end{center}
\end{adjustwidth}
\vspace*{\fill}
\cleardoublepage

%% ++++++++++++++++++++++++++++++++++++++++++++++++++++++++++++
%% title page
\vspace*{\fill}
\begin{center}
\HUGE\textsf{Notes on PGF}\par
\end{center}
\begin{center}
\LARGE\textsf{for}\par
\end{center}
\begin{center}
\HUGE\textsf{Configurable Typesetting}\par
\end{center}

\begin{center}
\Huge\textsf{User Guide}\par
\end{center}
\begin{center}
\LARGE\textsf{Dazhi Jiang}\par
\end{center}
\vspace*{\fill}
\def\THP{T\kern-0.2em H\kern-0.4em P}%   OK for CMR
\def\THP{T\kern-0.15em H\kern-0.3em P}%   OK for Palatino
\newcommand*{\THPress}{The Herries Press}%
\begin{center}
\settowidth{\droptitle}{\textsf{\THPress}}%
\textrm{\normalsize \THP} \\
\textsf{\THPress} \\[0.2\baselineskip]
% \includegraphics[width=\droptitle]{anvil2.mps}
\setlength{\droptitle}{0pt}%
\end{center}
\clearpage

%% ++++++++++++++++++++++++++++++++++++++++++++++++++++++++++++
\PWnote{2009/06/26}{Updated the copyright page for 9th impression}
%% copyright page
\begingroup
\footnotesize
\setlength{\parindent}{0pt}
\setlength{\parskip}{\baselineskip}
%%\ttfamily
\textcopyright{} 2010 Dazhi Jiang \\
All rights reserved


\ifMASTER
Manual last changed \svnyear/\svnmonth/\svnday
\fi

\endgroup

\clearpage
\vspace*{\fill}
\begin{quote}
\textbf{PGF,} \textit{n.} a LaTeX graph-drawing backend \\[0.5\baselineskip]
  \hspace*{\fill} 
      \textit{Dazhi Jiang}, 2010.
\end{quote}

\vspace{2\baselineskip}

\begin{quote}
\textbf{TikZ,} a extension package of PGF \\[0.5\baselineskip]
  \hspace*{\fill} \textit{Dazhi Jiang}.
\end{quote}

\vspace{2\baselineskip}

\vspace*{\fill}

\cleardoublepage

%% ++++++++++++++++++++++++++++++++++++++++++++++++++++++++++++
% ToC, etc
%%%\pagenumbering{roman}
\pagestyle{headings}
%%%%\pagestyle{Ruled}

%% !!!
%% I don't want the short table of content, so the following three lines were
%% commented. It saves the output stream.
% \setupshorttoc
% \tableofcontents
% \clearpage
\setupparasubsecs
\setupmaintoc
\tableofcontents
\setlength{\unitlength}{1pt}
\clearpage
\listoffigures
\clearpage
\listoftables
% \clearpage
% \listofegresults

%#% extend


%% ++++++++++++++++++++++++++++++++++++++++++++++++++++++++++++
%#% extstart include preface.tex
%\chapter{Foreword}

\svnidlong
{$Ignore: $}
{$LastChangedDate: 2010-05-13 17:10:00 +0200 (Thu, 13 May 2010) $}
{$LastChangedRevision: 210 $}
{$LastChangedBy: daleif $}

\chapter{Preface}

    From personal experience and also from lurking on the \url{comp.text.tex}

%#% extend


%% ++++++++++++++++++++++++++++++++++++++++++++++++++++++++++++
%#% extstart include preface.tex
%\chapter{Foreword}

\svnidlong
{$Ignore: $}
{$LastChangedDate: 2010-05-13 17:10:00 +0200 (Thu, 13 May 2010) $}
{$LastChangedRevision: 210 $}
{$LastChangedBy: daleif $}

\chapter{Introduction to the eighth edition}


\PWnote{2009/07/26}{Added `Remarks to the user' chapter}
%%%%%%%%%%%%%%%%%%%%%%%%%%%%%%%%%%
\chapter{Remarks to the user}
%%%%%%%%%%%%%%%%%%%%%%%%%%%%%%%%%%


%#% extend

%% ++++++++++++++++++++++++++++++++++++++++++++++++++++++++++++
%#% extstart include terminology.tex

\svnidlong
{$Ignore: $}
{$LastChangedDate: 2010-05-13 17:10:00 +0200 (Thu, 13 May 2010) $}
{$LastChangedRevision: 210 $}
{$LastChangedBy: daleif $}

%%%%%%%%%%%%%%%%%%%%%%%%%%%%%%%%%
\chapter{Terminology}
%%%%%%%%%%%%%%%%%%%%%%%%%%%%%%%%%%



%#% extend

\cleardoublepage
\pagenumbering{arabic}

% body
\mainmatter


%% ++++++++++++++++++++++++++++++++++++++++++++++++++++++++++++

%%%%%%%%%%%%%%%%%%%%%%%%%%%%
%%%%%\part{Practice} \label{part:practice}
%%%%%%%%%%%%%%%%%%%%%%%%%%%%%

%#% extstart include *.tex


\svnidlong
{$Ignore: $}
{$LastChangedDate: 2010-07-15 22:37:02 +0200 (Thu, 15 Jul 2010) $}
{$LastChangedRevision: 256 $}
{$LastChangedBy: daleif $}


\chapter{Examples from Books} % (fold)
\label{chap:examples_from_books}

% chapter examples_from_books (end)


%% ++++++++++++++++++++++++++++++++++++++++++++++++++++++++++++
\chapter{Test} % (fold)
\label{chap:test}

% chapter test (end)

An example of codeexample environment, which is defined in pgfmanual-en-macro.tex
\begin{codeexample}[code only]
\usepackage{pgfplots}
\pgfplotsset{compat=1.3}
\end{codeexample}


\Cref{sub:subsec1}

\begin{codeexample}[]
\begin{tikzpicture}
	\begin{semilogyaxis}[
		xlabel=Index,ylabel=Value]
	\addplot[color=blue,mark=*] coordinates {
		(1,8)
		(2,16)
		(3,32)
		(4,64)
		(5,128)
		(6,256)
		(7,512)
	};
	\end{semilogyaxis}%
\end{tikzpicture}%
\end{codeexample}

% chapter test (end)

%% ++++++++++++++++++++++++++++++++++++++++++++++++++++++++++++

\chapter{Dump} % (fold)
\label{chap:dump}

\section{section name} % (fold)
\label{sec:section_name}


% section section_name (end)

\subsection{subsection name} % (fold)
\label{sub:subsection_name}

% subsection subsection_name (end)

\section{Section Test}

{
\catcode`\^^I=12
\begin{verbatimoutput}{memoir-verbatim.txt}
This is a 'verbatimoutput' environment which is used to test the tab character in the output file.
	At the head of this line, there is a tab.
    At the head of this line, there are four spaces.
Tab is eaten while spaces can be kept.
Also, it seems that the 'verbatimoutput' environment uses a temporary stream, this implies that no additional stream needed.
\end{verbatimoutput}
}

\subsection{SubSection Test1}
\label{sub:subsec1}

% chapter dump (end)

\end{document}

%\endinput


%%% Local Variables: 
%%% mode: latex
%%% TeX-master: t
%%%% (TeX-source-specials-mode: t)
%%% TeX-PDF-mode: t
%%% End: 
