%#% extstart input preamble.tex
%
% memman.tex  Memoir class user manual (Part II only)  last updated 2009/09/07
%             Author: Peter Wilson
%             Copyright 2001, 2002, 2003, 2004, 2008, 2009 Peter R. wilson
%\listfiles
\documentclass[10pt,letterpaper,extrafontsizes]{memoir}
\listfiles
\usepackage{comment}
\usepackage[nofancy]{svninfo}
\svnInfo $Id: memman.tex 178 2010-02-16 09:33:34Z daleif $ 


% For (non-printing) notes  \PWnote{date}{text}
\newcommand{\PWnote}[2]{} 
\PWnote{2009/04/29}{Added fonttable to the used packages}
\PWnote{2009/08/19}{Made Part I a separate doc (memdesign.tex).}

% same
\newcommand{\LMnote}[2]{} 


\usepackage{memsty}
%%%%%%%%%%%%%%%%%%%%%%%%%%%%
\usepackage{titlepages}  % code of the example titlepages
\usepackage{memlays}     % extra layout diagrams
\usepackage{dpfloat}     % floats on facing pages
\usepackage{fonttable}[2009/04/01]   % font tables
%%%%\usepackage{xr-hyper} \externaldocument{memdesign} Doesn't work, 
%%%%                      Idea won't work in general for memman/memdesin
%%%%                      as at display time, who knows where everything
%%%%                      will be located on the individual's computer.
%%%%%%%%%%%%%%%%%%%%%%%%%%%%

%%%% Change section heading styles
%%%\memmansecheads

%%%% Use the built-in division styling
\headstyles{memman}

%%% ToC down to subsections
\settocdepth{subsection}
%%% Numbering down to subsections as well
\setsecnumdepth{subsection}

%%%% extra index for first lines
\makeindex[lines]

%% end preamble
%%%%%%%%%%%%%%%%%%%%%%%%%%%%%%%%%%%%%%%%%%%%%%%%%%%%%%%
%#% extend

\begin{document}

%#% extstart input intro.tex
\tightlists
%%%%\firmlists
\midsloppy
\raggedbottom
\chapterstyle{demo3}

%%%%%%%%%%%%%%%%%%%%%%%%%%%%%%%%%%%%%%%%%%%%%%%%%%%%%%%


\ProvidesFile{memnoidxnum}[2009/04/30  some index entries for memman]
\newcommand*{\idxat}{\index{@?\texttt{@}|noidxnum}} \idxat
%%\index{@?\texttt{@}|noidxnum}
\index{argument|noidxnum}
%%\index{array|noidxnum}
\index{cardinal|noidxnum}
\index{centering|noidxnum}
%%\index{chapterstyle|noidxnum}
%%\index{counter|noidxnum}
\index{default|noidxnum}
\index{division|noidxnum}
\index{division!sectional|seealso{subhead}}
\index{double column|noidxnum}
\index{endnote!mark|seealso{reference mark}}
\index{environment|noidxnum}
\index{error message|noidxnum}
\index{figures|noidxnum}
%%\index{file|noidxnum}
\index{font characteristic|noidxnum}
\index{footnote!mark|seealso{reference mark}}
\index{footnotes|noidxnum}
\index{frame|noidxnum}
\index{framed|noidxnum}
\index{full stop|seealso{period}}
\index{hanging|noidxnum}
\index{headstyles|noidxnum}
%%\index{horizontal|noidxnum}
\index{Hurenkinder|see{widow}}
\index{interlinear space|see{leading}}
\index{keyword|noidxnum}
%%\index{label|noidxnum}
\index{LaTeX?\ltx|noidxnum}
%%\index{length|noidxnum}
\index{line|noidxnum}
\index{line too long|see{overfull lines}}
\index{lining|noidxnum}
%%\index{list|noidxnum}
\index{lowercase|noidxnum}
\index{MakeIndex?\Pmakeindex|noidxnum}
\index{margin!spine|seealso{inner}}
\index{margin!inner|seealso{spine}}
\index{margin!foredge?\foredge|seealso{outer}}
\index{margin!outer|seealso{\foredge}}
\index{margin!upper|seealso{top}}
\index{margin!top|seealso{upper}}
\index{math|noidxnum}
%%\index{memoir class|noidxnum}
\index{minipage|noidxnum}
\index{name|noidxnum}
\index{named|noidxnum}
\index{new|noidxnum}
%%\index{number|noidxnum}
\index{numeric|noidxnum}
\index{old-style|noidxnum}
\index{option|noidxnum}
\index{ordinal|noidxnum}
\index{outline|noidxnum}
\index{package|noidxnum}
\index{page break|noidxnum}
%%\index{pagestyle|noidxnum}
\index{paragraph break|noidxnum}
\index{period|seealso{full stop}}
\index{poem|noidxnum}
\index{program|noidxnum}
\index{ranging|noidxnum}
\index{reference|noidxnum}
\index{reference mark|seealso{endnote mark, footnote mark}}
\index{representation|noidxnum}
\index{rule|noidxnum}
\index{ruled|noidxnum}
%%\index{section|noidxnum}
\index{Schusterjungen|see{orphan}}
\index{section|seealso{subhead}}
\index{sectional division|seealso{subhead}}
\index{single column|noidxnum}
\index{size|noidxnum}
\index{space|noidxnum}
\index{space!double|see(double spacing)}
\index{space!between lines|see{leading}}
\index{stanza|noidxnum}
%%\index{subfloat|noidxnum}
\index{TeX?\tx|noidxnum}
\index{text|noidxnum}
\index{titling|noidxnum}
\index{trim|noidxnum}
%%\index{type size|noidxnum}
\index{vertical|noidxnum}
\index{warning|noidxnum}
\index{write|noidxnum}
%%\index{XeTeX?\xetx|noidxnum}

%%%%%%%% Deleted the font indexing (now done as typefaces) 2009/04/30

\begin{comment}
\index{table of contents|see{ToC}}
\index{list!of figures|see{LoF}}
\index{figure!list of|see{LoF}}
\index{list!of tables|see{LoT}}
\index{table!list of|see{LoT}}
\index{marginal note|see{marginalia}}
\index{footnote!in title|see{thanks}}
\index{illustration|seealso{float, figure}}
\index{figure|seealso{float}}
\index{table|seealso{float}}
\index{chapter!style|see{chapterstyle}}
\index{chapter!heading|see{heading}}
\index{page!style|see{pagestyle}}
\index{part!heading|see{heading}}
\end{comment}

\begin{comment}

%%%% deleted the \nocites
%
\index{anonymous division|see{division}}
\index{array|seealso{tabular}}
%
\index{Berne Convention|see{copyright}}
\index{blank page|see{page}}
\index{Buenes Aires Convention|see{copyright}}
\index{box!rule|seealso{rule}}
%
\index{chapter|seealso{division}}
\index{chapter!style|see{chapterstyle}}
\index{command|seealso{declaration, macro}}
\index{comptexttex?\texttt{comp.text.tex} newsgroup|see{\ctt}}
\index{Comprehensive TeX Archive Network?\cTeXan|see{\ctan}}
\index{contents list|see{ToC}}
\index{counter representation!Alph tt?\texttt{Alph}|see{\texttt{Alph}}}
\index{counter representation!alph tt?\texttt{alph}|see{\texttt{alph}}}
\index{counter representation!arabic tt?\texttt{arabic}|see{\texttt{arabic}}}
\index{counter representation!Roman tt?\texttt{Roman}|see{\texttt{Roman}}}
\index{counter representation!roman tt?\texttt{roman}|see{\texttt{roman}}}
\index{counter representation!fnsymbol tt?\texttt{fnsymbol}|see{\texttt{fnsymbol}}}
\index{cross reference|seealso{reference}}
%
\index{descriptive list|see{list}}
\index{display math|see{math}}
\index{display mode|see{display}}
\index{division|seealso{heading}}
%
\index{electronic book|see{ebook}}
\index{enumerated list|see{list}}
%
\index{figure!list of|see{LoF}}
\index{figure|seealso{float}}
\index{float!numbered captioning|see{caption}}
\index{float!unnumbered captioning|see{legend}}
\index{font characteristic!weight|see{series}}
%
\index{file|seealso{stream}}
\index{footnote!in title|see{thanks}}
\index{fragile command|seealso{protect}}
\index{free tabular|seealso{tabular}}
%
\index{header|seealso{running header}}
\index{heading|seealso{division}}
%
\index{illustration|seealso{float, figure}}
\index{inline math|see{math}}
\index{International Standard Book Number|see{ISBN}}
\index{itemized list|see{list}}
%
\index{label|seealso{reference}}
\index{left-to-right|see{LR}}
\index{list!new list of|see{list of, new}}
\index{list!of contents|see{ToC}}
\index{list!of figures|see{LoF}}
\index{list!of tables|see{LoT}}
\index{list of!contents|see{ToC}}
\index{list of!figures|see{LoF}}
\index{list of!tables|see{LoT}}
\index{LoF|seealso{ToC}}
\index{LoT|seealso{ToC}}
\index{log-like function|see{function}}
%
\index{macro|seealso{command}}
\index{margin note|seealso{marginalia}}
\index{marginalia|seealso{marginal note, side note, sidebar}}
%
\index{named division|see{division}}
%
\index{page!of floats|see{float, page}}
\index{page!start new|see{start new page}}
\index{page!style|see{pagestyle}}
\index{paragraph|seealso{division}}
\index{part|seealso{division}}
\index{picture object!Bezier curve|see{Bezier curve}}
\index{picture object!circle|see{circle}}
\index{picture object!line|see{line}}
\index{picture object!oval|see{box, rounded}}
\index{picture object!vector|see{vector}}
\index{poem|see{verse}}
\index{poetry|see{verse}}
\index{print run|see{impression}}
\index{protect|seealso{fragile command}}
%
\index{recto|seealso{odd page}}
\index{reference|seealso{label}}
\index{river|see{white space}}
\index{rivulet|see{white space}}
\index{running footer|see{footer}}
\index{running header|seealso{header}}
%
\index{section|seealso{division}}
\index{side note|seealso{marginalia}}
\index{sidebar|seealso{marginalia}}
\index{stanza|seealso{verse}}
\index{stanza!line number|see{line number}}
\index{subparagraph|seealso{division}}
\index{subsection|seealso{division}}
\index{subsubsection|seealso{division}}
%
\index{table of contents|see{ToC}}
\index{table!list of|see{LoT}}
\index{table|seealso{float}}
\index{tabular|seealso{array}}
\index{tabular!free|see{free tabular}}
\index{tabulation|see{tabular}}\
\index{TeX Users Group?\TeXUG|see{\tug}}
\index{textblock|see{typeblock}}
%
\index{Universal Copyright Convention|see{copyright}}
%
\index{verbatim!line number|see{line number}}
\index{verse|seealso{stanza}}
\index{verse!title|see{poem title}}
\index{verse!line number|see{line number}}
\index{verso|seealso{even page}}
\index{visual markup|see{visual design}}
%
\index{x coordinate|see{coordinate}}
%
\index{y coordinate|see{coordinate}}
%
%


\end{comment}

\endinput



\frontmatter
\pagestyle{empty}


% half-title page
\vspace*{\fill}
\begin{adjustwidth}{1in}{1in}
\begin{flushleft}
\HUGE\sffamily The
\end{flushleft}
\begin{center}
\HUGE\sffamily  Memoir
\end{center}
\begin{flushright}
\HUGE\sffamily  Class
\end{flushright}
%%\begin{center}
%%\sffamily (Draft Edition 7)
%%\end{center}
\end{adjustwidth}
\vspace*{\fill}
\cleardoublepage

% title page
\vspace*{\fill}
\begin{center}
\HUGE\textsf{The Memoir Class}\par
\end{center}
\begin{center}
\LARGE\textsf{for}\par
\end{center}
\begin{center}
\HUGE\textsf{Configurable Typesetting}\par
\end{center}

\begin{center}
\Huge\textsf{User Guide}\par
\end{center}
\begin{center}
\LARGE\textsf{Peter Wilson}\par
\end{center}
\vspace*{\fill}
\def\THP{T\kern-0.2em H\kern-0.4em P}%   OK for CMR
\def\THP{T\kern-0.15em H\kern-0.3em P}%   OK for Palatino
\newcommand*{\THPress}{The Herries Press}%
\begin{center}
\settowidth{\droptitle}{\textsf{\THPress}}%
\textrm{\normalsize \THP} \\
\textsf{\THPress} \\[0.2\baselineskip]
\includegraphics[width=\droptitle]{anvil2.mps}
\setlength{\droptitle}{0pt}%
\end{center}
\clearpage

\PWnote{2009/06/26}{Updated the copyright page for 9th impression}
% copyright page
\begingroup
\footnotesize
\setlength{\parindent}{0pt}
\setlength{\parskip}{\baselineskip}
%%\ttfamily
\textcopyright{} 2001 --- 2010 Peter R. Wilson \\
All rights reserved

The Herries Press, Normandy Park, WA.

Printed in the World 

The paper used in this publication may meet the minimum requirements
of the American National Standard for Information 
Sciences --- Permanence of Paper for Printed Library Materials, 
ANSI Z39.48--1984.

\PWnote{2009/07/08}{Changed manual date to 8 July 2009}
\begin{center}
10 09 08 07 06 05 04 03 02 01\hspace{2em}19 18 17 16 15 14 13
\end{center}
\begin{center}
\begin{tabular}{ll}
First edition:                        & 3 June 2001 \\
Second impression, with corrections:    & 2 July 2001 \\
Second edition:                       & 14 July 2001 \\
Second impression, with corrections:    & 3 August 2001 \\
Third impression, with minor additions: & 31 August 2001 \\
Third edition:                        & 17 November 2001 \\
Fourth edition:                       & 16 March 2002 \\
Fifth edition:                        & 10 August 2002 \\
Sixth edition:                        & 31 January 2004 \\
%%Draft Seventh edition:                & 31 January 2008 \\
Seventh edition:                       & 10 May 2008 \\
Eighth impression, with very minor corrections: & 12 July 2008 \\
Ninth impression, with additions and corrections: & 8 July 2009 \\
Eighth edition:                        & August 2009 \\
\end{tabular}
\end{center}
Last changed \svnInfoDate

\endgroup

\clearpage
\vspace*{\fill}
\begin{quote}
\textbf{memoir,} \textit{n.} a written record set down as material
  for a history or biography: 
  a biographical sketch:
  a record of some study investigated by the writer:
  (in \textit{pl.}) the transactions of a society.
  [Fr. \textit{m\'{e}moire} --- L. \textit{memoria,} memory ---
   \textit{memor}, mindful.] \\[0.5\baselineskip]
  \hspace*{\fill} 
      \textit{Chambers Twentieth Century Dictionary, New Edition}, 1972.
\end{quote}

\vspace{2\baselineskip}

\begin{quote}
\textbf{memoir,} \textit{n.} [Fr. \textit{m\'{e}moire,} masc., a memorandum,
    memoir, fem., memory $<$ L. \textit{memoria,} \textsc{memory}]
  \hspace{1ex} \textbf{1.} a biography or biographical notice, 
      usually written by a relative or personal friend of the subject 
  \hspace{1ex} \textbf{2.} [\textit{pl.}] an autobiography, 
      usually a full or highly personal account
  \hspace{1ex} \textbf{3.} [\textit{pl.}] a report or record of 
      important events based on the writer's personal observation, 
      special knowledge, etc.
  \hspace{1ex} \textbf{4.} a report or record of a scholarly 
      investigation, scientific study, etc.
  \hspace{1ex} \textbf{5.} [\textit{pl.}] the record of the proceedings
      of a learned society \\[0.5\baselineskip]
  \hspace*{\fill} \textit{Webster's New World Dictionary, Second College Edition}.
\end{quote}

\vspace{2\baselineskip}


\begin{quote}
\textbf{memoir,} \textit{n.} a fiction designed to flatter the subject 
  and to impress the reader. \\[0.5\baselineskip]
\hspace*{\fill} With apologies to Ambrose Bierce % and Reuben Thomas
\end{quote}

\vspace*{\fill}

\cleardoublepage

% ToC, etc
%%%\pagenumbering{roman}
\pagestyle{headings}
%%%%\pagestyle{Ruled}

\setupshorttoc
\tableofcontents
\clearpage
\setupparasubsecs
\setupmaintoc
\tableofcontents
\setlength{\unitlength}{1pt}
\clearpage
\listoffigures
\clearpage
\listoftables
\clearpage
\listofegresults

%#% extend


%#% extstart include preface.tex
%\chapter{Foreword}
\chapter{Preface}

    From personal experience and also from lurking on the \url{comp.text.tex}
newsgroup the major problems with using LaTeX are related to document
design. Some years ago most questions on \texttt{ctt} were answered by
someone providing a piece of code that solved a particular problem, and
again and again. More recently these questions are answered along the
lines of `Use the ---------{} package', and again and again.

    I have used many of the more common of these packages but my filing system
is not always well ordered and I tend to mislay the various user manuals,
even for the packages I have written. The \Pclass{memoir} class is an attempt
to integrate some of the more design-related packages with the LaTeX
\Pclass{book} class. I chose the \Pclass{book} class as the \Pclass{report} class
is virtually identical to \Pclass{book}, except that \Pclass{book} does
not have an \Ie{abstract} environment while \Pclass{report} does; however it is 
easy to fake an \Ie{abstract} if it is needed. With a little bit of tweaking,
\Pclass{book} class documents can be made to look just like \Pclass{article}
class documents, and the \Pclass{memoir} class is designed with tweaking very
much in mind.



\end{document}

%\endinput


%%% Local Variables: 
%%% mode: latex
%%% TeX-master: t
%%% TeX-source-specials-mode: t
%%% TeX-PDF-mode: nil
%%% End: 
