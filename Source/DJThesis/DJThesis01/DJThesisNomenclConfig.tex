% ---------------------------------------------------------------------|
% Commenly-used nomenclature relavent preamble
% Put this file in the preamble of your main document rather than the class file
% ---------------------------------------------------------------------|
%
% (c) Copyright 2009 by Dazhi Jiang
% ---------------------------------------------------------------------|

%
% 20090608	Initial version, package <>
%
% 20090815  Change command \makeglossary to \makenomenclature
% 
% 20080826	Add intoc option to package loading
%
%



% *********************************************************************
% *********************************************************************
% Nomenclature

% Default, numenclature is not in your table of content
% \usepackage{nomencl}

% Comment this line, if you don't want to list nomenclature in your table of content
\usepackage[intoc]{nomencl}

%
% \makeglossary
\makenomenclature
\renewcommand\nomgroup[1]{%
  \ifthenelse{\equal{#1}{A}}{%
   \item[\textbf{Roman Symbols}] }{%             A - Roman
    \ifthenelse{\equal{#1}{G}}{%
     \item[\textbf{Greek Symbols}]}{%             G - Greek
      \ifthenelse{\equal{#1}{R}}{%
        \item[\textbf{Superscripts}]}{%              R - Superscripts
          \ifthenelse{\equal{#1}{S}}{%
           \item[\textbf{Subscripts}]}{{%             S - Subscripts
        \ifthenelse{\equal{#1}{X}}{%
         \item[\textbf{Other Symbols}]}{{%    X - Other Symbols
        \ifthenelse{\equal{#1}{Z}}{%
         \item[\textbf{Acronyms}]}%              Z - Acronyms
                        {{}}}}}}}}}} 
						
% To use glossary in a book style, just put the following lines in your frontmatter of your main document
% \cleardoublepage% or \clearpage
% \markboth{\nomname}{\nomname}% maybe with \MakeUppercase
% \printnomenclature

% If use glossary in an article style, just simply put the command
% \printnomenclature

% You can also specify the lable width of nomenclature by giving the option to \printnomenclature, e.g.,
% \printnomenclature[3cm]
% This option actually redefine the 