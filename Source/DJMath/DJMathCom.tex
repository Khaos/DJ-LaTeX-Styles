% ---------------------------------------------------------------------|
% Commenly-used mathmatical preamble
% Put this file in the preamble of your main document rather than the class file
% ---------------------------------------------------------------------|
%
% (c) Copyright 2009 by Dazhi Jiang
% ---------------------------------------------------------------------|

%
% 20090322	Initial version
%
%
% 20090327	Redefine commands with package xargs.
%			Examples:
%				\E x or \E{x}, but \E[f]{x} for the argument which has default value
%
%
% 20090526	Define the new command \sign
%
%
% 20090605	Rewirte the file, now you have to load the required packages yourself.
%
%


%%
% The followings come from the LaTeX template ---- tpavlic_masters_thesis
%
% Upper-case    A B C D E F G H I J K L M N O P Q R S T U V W X Y Z
% Lower-case    a b c d e f g h i j k l m n o p q r s t u v w x y z
% Digits        0 1 2 3 4 5 6 7 8 9
% Exclamation   !           Double quote "          Hash (number) #
% Dollar        $           Percent      %          Ampersand     &
% Acute accent  '           Left paren   (          Right paren   )
% Asterisk      *           Plus         +          Comma         ,
% Minus         -           Point        .          Solidus       /
% Colon         :           Semicolon    ;          Less than     <
% Equals        =           Greater than >          Question mark ?
% At            @           Left bracket [          Backslash     \
% Right bracket ]           Circumflex   ^          Underscore    _
% Grave accent  `           Left brace   {          Vertical bar  |
% Right brace   }           Tilde        ~

% ---------------------------------------------------------------------|
% --------------------------- 72 characters ---------------------------|
% ---------------------------------------------------------------------|

% ---------------------------------------------------------------------|
% Special Symbols/Macros
% ---------------------------------------------------------------------|
%
% (c) Copyright 2007 by Theodore P. Pavlic
%

% ---------------------------------------------------------------------|
% Special Symbols/Macros
% ---------------------------------------------------------------------|
%
% (c) Copyright 2009 by Dazhi Jiang
%


% Additional symbols, phrases, etc.
% DJ's Definition

% *********************************************************************
% *********************************************************************
% Part I:		Abbreviation

\newcommand{\eg}{e.g.,\xspace}
\newcommand{\ie}{i.e.,\xspace}
\newcommand{\etc}{etc.\@\xspace}

% Statistical Abbreviation
\newcommand{\iid}{i.i.d.\xspace} 		% Should be able to be a label

\newcommand{\djpdf}{p.d.f.\xspace}
\newcommand{\djcdf}{c.d.f.\xspace}

% Original Ones
% \newcommand{\ie}{i.e.{}}
% \newcommand{\eg}{e.g.{}}
% \newcommand{\exante}{ex ante}
% \newcommand{\expost}{ex post}
% \newcommand{\apriori}{a priori}
% \newcommand{\adinfinitum}{ad infinitum}
% \newcommand{\aka}{a.k.a.{}}


% *********************************************************************
% *********************************************************************
% Part II:		Ordinal number

\global\long\def\upst{\ensuremath{^{\text{st}}}\xspace}
\global\long\def\upnd{\ensuremath{^{\text{nd}}}\xspace}
\global\long\def\uprd{\ensuremath{^{\text{rd}}}\xspace}
% Do not use \th command, it is not compatible with OT1 font
\global\long\def\upth{\ensuremath{^{\text{th}}}\xspace}
% \th has already been defined
%\global\long\def\th{\ensuremath{^{\text{th}}}}
%\newcommand{\th}{\upth}



% *********************************************************************
% *********************************************************************
% Part III:		Math Conventions

% *********************************************************************
% Math function and operator

% Symbols for the AMA6020 Report
\DeclareMathOperator \AMISE {AMISE}
\DeclareMathOperator \MISE {MISE}
\DeclareMathOperator \Bias {Bias}

% Statistics & statistical
\DeclareMathOperator \kurt {kurt}			% Kurtosis

\DeclareMathOperator \EX {E}				% Expectation

% Functions
\DeclareMathOperator* \argmin {arg\,min}
\DeclareMathOperator* \argmax {arg\,max}
\DeclareMathOperator \sign {sign}

\DeclareMathOperator \SPAN {span}
\DeclareMathOperator \NULL {null}

% Differential
\DeclareMathOperator \diff {d}

% Objective function
\DeclareMathOperator \obJ {J}


% Phantom for math relationship operator
\global\long\def\relphantom#1{\mathrel{\phantom{#1}}}

% Definition Equal
\global\long\def\defeq{\stackrel{\mathrm{def}}{=}}




% *********************************************************************
% Math set

% Common-defined set
\global\long\def\R{\ensuremath{\mathbb{R}}}		% Real number set
\global\long\def\N{\ensuremath{\mathbb{N}}}		% Natural number set
\global\long\def\Z{\ensuremath{\mathbb{Z}}}		% Integer number set
\global\long\def\Q{\ensuremath{\mathbb{Q}}}		% Rational number set
\global\long\def\C{\ensuremath{\mathbb{C}}}		% Complex number set

% DJ style set
% The set of set
\global\long\def\setset#1{\ensuremath{\mathscr{#1}}}
% set?
\global\long\def\setdef#1{\ensuremath{\mathcal{#1}}}

% Matrix or vector
% Identity matrix
\global\long\def\I{\ensuremath{\mathbf{I}}}
% zero matrix or vector
\global\long\def\matzero{\ensuremath{\mathbf{0}}}



% *********************************************************************
% Information theory

% Mutual information
\global\long\def\muinform#1{\mathrm{I}{\left[#1\right]}}
\global\long\def\minform#1{\mathrm{I}{\left[#1\right]}}
% Entropy
\global\long\def\entropy#1{\mathrm{H}{\left[#1\right]}}



% *********************************************************************
% Statistical function
% The following commands require package xargs

% Expectation
\newcommandx\E[2][usedefault, addprefix=\global, 1=]{\mathbb{E}_{#1}{\left[#2\right]}}

% Covariance
\newcommandx\cov[2][usedefault, addprefix=\global, 1=]{\mathrm{cov}_{#1}{\left[#2\right]}}

% Variance
\newcommandx\var[2][usedefault, addprefix=\global, 1=]{\mathrm{var}_{#1}{\left[#2\right]}}

% Normal Distribution
\newcommandx\Normal[4][usedefault, addprefix=\global, 1=D]{\mathcal{N}{(#2|#3,#4)} = \frac{1}{(2\pi)^{#1/2}}\frac{1}{|#4|^{1/2}}\exp\left\{  -\frac{1}{2}(#2-#3)^{T}#4^{-1}(#2-#3)\right\}  }

% Normal Distribution right part
\newcommandx\Normalr[4][usedefault, addprefix=\global, 1=D]{\frac{1}{(2\pi)^{#1/2}}\frac{1}{|#4|^{1/2}}\exp\left\{  -\frac{1}{2}(#2-#3)^{T}#4^{-1}(#2-#3)\right\}  }

% Normal Distribution left part
\newcommandx\Normall[3][usedefault, addprefix=\global, 1=x]{\mathcal{N}{(#1|#2,#3)}  }




% *********************************************************************
% *********************************************************************
% Part IV:


% *********************************************************************
% *********************************************************************
% Part V:		Others

\newcommand{\ttbs}{\char'134} % double quotation marks


%\DeclareMathOperator \var {var}

% \newcommand{\E}[1]{\mathbb{E}{\left[#1\right]}}				% Expected value
% \newcommand{\cov}[1]{\mathrm{cov}{\left[#1\right]}}			% Covariance
% \newcommand{\var}[1]{\mathrm{var}{\left[#1\right]}}			% Variance
% \newcommand{\Ef}[2]{\mathbb{E}_{#1}{\left[#2\right]}}		% Expected value of f
% \newcommand{\covf}[2]{\mathrm{cov}_{#1}{\left[#2\right]}}	% Cov of f
% \newcommand{\varf}[2]{\mathrm{var}_{#1}{\left[#2\right]}}	% Var of f
% \newcommand{\Normal}[3]{\mathcal{N}{(#1|#2,#3)} = \frac{1}{(2\pi)^{D/2}}
%                         \frac{1}{|#3|^{1/2}}
%                         \exp{\biggl\{-\frac{1}{2}(#1-#2)^T#3^{-1}(#1-#2)\biggl\}}}
% \newcommand{\Normall}[3]{\frac{1}{(2\pi)^{D/2}}
%                         \frac{1}{|#3|^{1/2}}
%                         \exp{\biggl\{-\frac{1}{2}(#1-#2)^T#3^{-1}(#1-#2)\biggl\}}}

% \newcommand{\relphantom}[1]{\mathrel{\phantom{#1}}}

% \newcommand{\minform}[1]{\mathrm{I}{\left[#1\right]}}	% Mutual Information
% \newcommand{\entropy}[1]{\mathrm{H}{\left[#1\right]}}	% Entropy

% \newcommand{\defeq}{\stackrel{\mathrm{def}}{=}}


%\newcommand{\AmS}{{\protect\the\textfont2
%  A\kern-.1667em\lower.5ex\hbox{M}\kern-.125emS}}


% \global\long\def\sign{\mathrm{sign}}


% add words to TeX's hyphenation exception list
% \hyphenation{MATLAB author another created financial paper re-commend-ed
% Post-Script}




% \newcommand{\seq}[1]{\ensuremath{\mathcal{#1}}}

% \newcommand{\setset}[1]{\ensuremath{\mathscr{#1}}}
% \newcommand{\setdef}[1]{\ensuremath{\mathcal{#1}}}

% \newcommand{\rel}[1]{\ensuremath{\mathrel{#1}}}
% \newcommand{\bin}[1]{\ensuremath{\mathbin{#1}}}
%\renewcommand{\v}[1]{\ensuremath{#1}}
%\renewcommand{\v}[1]{{\ensuremath{\underline{#1}}}}
% \renewcommand{\v}[1]{{\ensuremath{#1}}}
% \newcommand{\symdiff}{\ensuremath{\mathbin{\Delta}}}
% \newcommand{\setdiff}{\ensuremath{-}}
%\newcommand{\setdiff}{\ensuremath{\setminus}}
% \newcommand{\comp}{\ensuremath{\circ}}
% \newcommand{\mat}[1]{\ensuremath{#1}}

% \newcommand{\I}{\ensuremath{\mathbb{I}}}	% Identity matrix

%\newcommand{\mat}[1]{\ensuremath{\mathbf{#1}}}
% \newcommand{\T}{\ensuremath{\top}}
% \newcommand{\interior}{\ensuremath{\operatorname{int}}}
% \newcommand{\biject}{\ensuremath{\mathrel{\leftrightarrow}}}
% \newcommand{\Pow}{\ensuremath{\mathscr{P}}}
% \newcommand{\Borel}{\ensuremath{\mathfrak{B}}}

% \renewcommand{\th}{\ensuremath{^\text{th}}}	% i^th

% \newcommand{\nhd}{\ensuremath{\setset{N}}}

%\newcommand{\R}{\ensuremath{\mathbb{R}}}	% Real number set

% \newcommand{\extR}{\ensuremath{\mathbb{\overline{R}}}}

% \newcommand{\N}{\ensuremath{\mathbb{N}}}	% Natural number set

% \newcommand{\W}{\ensuremath{\mathbb{W}}}

% \newcommand{\Z}{\ensuremath{\mathbb{Z}}}	% Integer number set
% \newcommand{\Q}{\ensuremath{\mathbb{Q}}}	% The set of all rational numbers
% \newcommand{\C}{\ensuremath{\mathbb{C}}}	% The set of all complex numbers

% \newcommand{\bang}{!}
% \newcommand{\pipe}{|}
% \newcommand{\ppipe}{\|}
% \newcommand{\total}{\ensuremath{\operatorname{d}}}
% \newcommand{\E}{\ensuremath{\operatorname{E}}}
% \newcommand{\iid}{i.i.d.} % Should be able to be a label
% \newcommand{\var}{\ensuremath{\operatorname{var}}}
% \newcommand{\cov}{\ensuremath{\operatorname{cov}}}
% \newcommand{\std}{\ensuremath{\operatorname{std}}}
% \newcommand{\LPV}{\ensuremath{\operatorname{LPV}}}
% \newcommand{\LPM}{\ensuremath{\operatorname{LPM}}}
% \newcommand{\range}{\ensuremath{\operatorname{range}}}
% \newcommand{\sgn}{\ensuremath{\operatorname{sgn}}}
% \newcommand{\aslim}{\ensuremath{\operatorname*{aslim}}}
% \newcommand{\limean}{\ensuremath{\operatorname*{l.i.m.}}}
% \newcommand{\mslim}{\ensuremath{\operatorname*{mslim}}}
% \newcommand{\plim}{\ensuremath{\operatorname*{plim}}}
% \newcommand{\dlim}{\ensuremath{\operatorname*{dlim}}}
% \newcommand{\xto}[1]{\ensuremath{\mathrel{\xrightarrow{#1}}}}
% The classical OFT model
%\newcommand{\oft}[1]{\ensuremath{\tilde{#1}}}
% \newcommand{\oft}[1]{\ensuremath{\widetilde{#1}}}

% A special box to catch my eye to highlight spots where work still
% needs to be done
% \newcommand{\todo}[1]{\vspace{0.25in}%
                      % \noindent%
                      % \fbox{\parbox{\columnwidth}%
                                   % {\noindent\textbf{\fbox{TODO:} #1}}}%
                      % \vspace{0.25in}}

