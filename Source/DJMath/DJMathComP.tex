% ---------------------------------------------------------------------|
% Commenly-used mathmatical preamble
% Put this file in the preamble of your main document rather than the class file
% ---------------------------------------------------------------------|
%
% (c) Copyright 2009 by Dazhi Jiang
% ---------------------------------------------------------------------|

%
% 20090322	Initial version
%
%
% 20090327	Redefine commands with package xargs.
%			Examples:
%				\E x or \E{x}, but \E[f]{x} for the argument which has default value
%
%
% 20090526	Define the new command \sign
%
%
% 20090605	Rewirte the file, now you have to load the required packages yourself.
%
%
% 2010-05-07	Refine this file using my Macbook Pro, add more comments and packages


%%
% The followings come from the LaTeX template ---- tpavlic_masters_thesis
%
% Upper-case    A B C D E F G H I J K L M N O P Q R S T U V W X Y Z
% Lower-case    a b c d e f g h i j k l m n o p q r s t u v w x y z
% Digits        0 1 2 3 4 5 6 7 8 9
% Exclamation   !           Double quote "          Hash (number) #
% Dollar        $           Percent      %          Ampersand     &
% Acute accent  '           Left paren   (          Right paren   )
% Asterisk      *           Plus         +          Comma         ,
% Minus         -           Point        .          Solidus       /
% Colon         :           Semicolon    ;          Less than     <
% Equals        =           Greater than >          Question mark ?
% At            @           Left bracket [          Backslash     \
% Right bracket ]           Circumflex   ^          Underscore    _
% Grave accent  `           Left brace   {          Vertical bar  |
% Right brace   }           Tilde        ~

% ---------------------------------------------------------------------|
% --------------------------- 72 characters ---------------------------|
% ---------------------------------------------------------------------|

% ---------------------------------------------------------------------|
% Special Symbols/Macros
% ---------------------------------------------------------------------|
%
% (c) Copyright 2007 by Theodore P. Pavlic
%



% ---------------------------------------------------------------------|
% Declare your package here
\usepackage{xspace}			% For command \xspace

\usepackage{savesym}		% Use this package to save commands,
							% such as iint, iiint in amsmath, because LyX use esint
							% package which confilct with amsmath.

\usepackage{amsmath}		% For command \ensuremath
							% 			  \DeclareMathOperator
							%			  \mathrel

% Top-or-bottom tags
% For a split equation, place equation numbers level with the last (resp. first) line, if numbers are on the right (resp. left).
% \usepackage[tbtags]{amsmath}



% !!!!!!!!!!
% Note that if you use LyX, please turn off the LyX options <use esint package automatically>
% in [Document]->[Settings]->[Math Options]

\savesymbol{iint}
\savesymbol{iiint}
\savesymbol{iiiint}
\usepackage{esint}			% Create nice multiple integral signs
\restoresymbol{ES}{iint}	% Now use ESiint to generate nice double integral sigh
\restoresymbol{ES}{iiint}	% Now use ESiiint to generate nice triple integral sigh
\restoresymbol{ES}{iiiint}	% Now use ESiiiint to generate nice multiple integral sigh



\usepackage{amssymb}		% For command \mathbb

\usepackage{amsfonts}		% For command \mathbb

\usepackage{xargs}			% For command with optional arguments

\usepackage{mathrsfs}		% For command \mathscr


\usepackage{eucal}			% Euler math font

\usepackage{commath}		% \begin{bmatrix} ... \end{bmatrix}, etc.
\usepackage{amsthm}			% \theoremstyle
\usepackage{bm}				% \bm
\usepackage{esvect}
\usepackage{vector}
\usepackage{tensor}
%\usepackage{delarray}		% Delimiters surrounding an array
\usepackage{pmat}
% \usepackage{array}