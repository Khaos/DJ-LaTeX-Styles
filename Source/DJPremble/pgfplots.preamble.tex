%%%%%%%%%%%%%%%%%%%%%%%%%%%%%%%%%%%%%%%%%%%%%%%%%%%%%%%%%%%%%%%%%%%%%%%%%%%%%
%
% Package pgfplots.sty documentation. 
%
% Copyright 2007/2008 by Christian Feuersaenger.
%
% This program is free software: you can redistribute it and/or modify
% it under the terms of the GNU General Public License as published by
% the Free Software Foundation, either version 3 of the License, or
% (at your option) any later version.
% 
% This program is distributed in the hope that it will be useful,
% but WITHOUT ANY WARRANTY; without even the implied warranty of
% MERCHANTABILITY or FITNESS FOR A PARTICULAR PURPOSE.  See the
% GNU General Public License for more details.
% 
% You should have received a copy of the GNU General Public License
% along with this program.  If not, see <http://www.gnu.org/licenses/>.
%
%
%%%%%%%%%%%%%%%%%%%%%%%%%%%%%%%%%%%%%%%%%%%%%%%%%%%%%%%%%%%%%%%%%%%%%%%%%%%%%
\documentclass[a4paper]{ltxdoc}

\usepackage{makeidx}

% DON't let hyperref overload the format of index and glossary. 
% I want to do that on my own in the stylefiles for makeindex...
\makeatletter
\let\@old@wrindex=\@wrindex
\makeatother

\usepackage{ifpdf}
\ifpdf
	\usepackage[pdftex]{hyperref}
\else
	\usepackage[dvipdfm]{hyperref}
\fi
\hypersetup{pdfborder={0 0 0}}

\makeatletter
\let\@wrindex=\@old@wrindex
\makeatother

% Formatiere Seitennummern im Index:
\newcommand{\indexpageno}[1]{%
	{\bfseries\hyperpage{#1}}%
}


\newcommand{\R}{\mathbb{R}}
\newcommand{\N}{\mathbb{N}}
\newcommand{\Z}{\mathbb{Z}}

\long\def\COMMENTLOWLEVEL#1\ENDCOMMENT{}
\def\ENDCOMMENT{}

\usepackage{textcomp}
\usepackage{booktabs}

\usepackage{calc}
\usepackage[formats]{listings}
%\usepackage{courier} % don't use it - the '^' character can't be copy-pasted in courier

\usepackage{array}
\lstset{%
	basicstyle=\ttfamily,
	language=[LaTeX]tex, % Seems as if \lstset{language=tex} must be invoked BEFORE loading tikz!?
	tabsize=4,
	breaklines=true,
	breakindent=0pt
}

\ifpdf
	\pdfinfo {
		/Author	(Christian Feuersaenger)
	}

\else
	\def\pgfsysdriver{pgfsys-dvipdfm.def}
\fi
%\def\pgfsysdriver{pgfsys-pdftex.def}
\usepackage{pgfplots}

\ifpdf
	\usetikzlibrary{pgfplotsclickable}
\fi

%\usepackage{fp}
% ATTENTION:
% this requires pgf version NEWER than 2.00 :
%\usetikzlibrary{fixedpointarithmetic}

\usetikzlibrary{dateplot}

\usepackage[a4paper,left=2.25cm,right=2.25cm,top=2.5cm,bottom=2.5cm,nohead]{geometry}
\usepackage{amsmath,amssymb}
\usepackage{xxcolor}
\usepackage{pifont}
\usepackage[latin1]{inputenc}
\usepackage{amsmath}
\usepackage{eurosym}

\def\eps{\textsc{eps}}

\input pgfmanual-en-macros.tex

\def\pgfmanualbar{\char`\|}
\makeatletter
\newenvironment{addplotoperation}[3][]{
  \begin{pgfmanualentry}
  	{%
	\let\ltxdoc@marg=\marg
	\let\ltxdoc@oarg=\oarg
	\let\ltxdoc@parg=\parg
	\let\ltxdoc@meta=\meta
	\def\marg##1{{\normalfont\ltxdoc@marg{##1}}}%
	\def\oarg##1{{\normalfont\ltxdoc@oarg{##1}}}%
	\def\parg##1{{\normalfont\ltxdoc@parg{##1}}}%
	\def\meta##1{{\normalfont\ltxdoc@meta{##1}}}%
    \pgfmanualentryheadline{\textcolor{gray}{{\ttfamily\char`\\addplot\ }}%
      \declare{\texttt{#2}} \texttt{#3;}}%
	  \unskip
	 \nobreak
    \pgfmanualentryheadline{\textcolor{gray}{{\texttt{\char`\\addplot}\oarg{style options} \texttt{plot}\oarg{behavior options}}\ }%
      \declare{\texttt{#2}} \texttt{#3} \textcolor{gray}{\meta{trailing path commands}}\texttt{;}}%
    \def\pgfmanualtest{#1}%
    \ifx\pgfmanualtest\@empty%
      \index{#2@\protect\textcolor{gray}{\protect\texttt{plot}}\protect\texttt{ #2}}%
      \index{Plot operations!plot #2@\protect\texttt{plot #2}}%
    \fi%
	}%
    \pgfmanualbody
}
{
  \end{pgfmanualentry}
}

\newenvironment{codekey}[1]{%
  \begin{pgfmanualentry}
 	\pgfmanualentryheadline{{\ttfamily\declare{#1}\textcolor{gray}{/.code}=\marg{...}}\hfill}%
	\def\mykey{#1}%
	\def\mypath{}%
	\def\myname{}%
	\firsttimetrue%
	\decompose#1/\nil%
    \pgfmanualbody
}
{
  \end{pgfmanualentry}
}

\newenvironment{pgfplotscodekey}[1]{%
	\begin{codekey}{/pgfplots/#1}%
}
{
  \end{codekey}
}
\newenvironment{pgfplotscodetwokey}[1]{%
  \begin{pgfmanualentry}
 	\pgfmanualentryheadline{{\ttfamily\declare{/pgfplots/#1}\textcolor{gray}{/.code 2 args}=\marg{...}}\hfill}%
	\def\mykey{/pgfplots/#1}%
	\def\mypath{}%
	\def\myname{}%
	\firsttimetrue%
	\decompose/pgfplots/#1/\nil%
    \pgfmanualbody
}
{
  \end{pgfmanualentry}
}

\newenvironment{pgfplotsxycodekeylist}[1]{%
	\begingroup
	\let\oldpgfmanualentryheadline=\pgfmanualentryheadline
	\def\pgfmanualentryheadline##1{%
		\pgfmanualentryheadline@##1\pgfplots@EOI
	}%
	\def\pgfmanualentryheadline@##1\hfill##2\pgfplots@EOI{%
		\oldpgfmanualentryheadline{{\ttfamily\declare{##1}\textcolor{gray}{/.code}=\marg{...}}\hfill}%
	}
	\begin{pgfplotsxykeylist}{#1}%
}
{
	\end{pgfplotsxykeylist}
	\endgroup
}

\newenvironment{pgfplotskey}[1]{%
  \begin{key}{/pgfplots/#1}%
}
{
  \end{key}
}

\def\choicesep{$\vert$}%
\def\choicearg#1{\texttt{#1}}

\newif\iffirstchoice
\newcommand\mchoice[1]{%
	\begingroup
	\firstchoicetrue
	\foreach \mchoice@ in {#1} {%
		\iffirstchoice
			\global\firstchoicefalse
		\else
			\choicesep
		\fi
		\choicearg{\mchoice@}%
	}%
	\endgroup
}%

% \begin{xykey}{/path/\x label=value}
% \end{xykey}
%
% has same features with 'default', 'initially' etc as key environment
\newenvironment{xykey}[2][]{%
	\begin{pgfmanualentry}
    \def\extrakeytext{}
	\insertpathifneeded{#2}{#1}%
	\expandafter\pgfutil@in@\expandafter=\expandafter{\mykey}%
	\ifpgfutil@in@%
		\expandafter\xykey@eq\mykey\@nil
	\else
		\expandafter\xykey@noeq\mykey\@nil
	\fi
	\pgfmanualbody
}{%
	\end{pgfmanualentry}
}%

% \begin{xystylekey}{/path/\x label=value}
% \end{xystylekey}
%
% has same features with 'default', 'initially' etc as key environment
\newenvironment{xystylekey}[2][]{%
	\begin{pgfmanualentry}
    \def\extrakeytext{style, }
	\insertpathifneeded{#2}{#1}%
	\expandafter\pgfutil@in@\expandafter=\expandafter{\mykey}%
	\ifpgfutil@in@%
		\expandafter\xykey@eq\mykey\@nil
	\else
		\expandafter\xykey@noeq\mykey\@nil
	\fi
	\pgfmanualbody
}{%
	\end{pgfmanualentry}
}%

% \insertpathifneeded{a key}{/pgfplots} -> assign mykey={/pgfplots/a key}
% \insertpathifneeded{/tikz/a key}{/pgfplots} -> assign mykey={/tikz/a key}
%
% #1: the key
% #2: a default path (or empty)
\def\insertpathifneeded#1#2{%
	\def\insertpathifneeded@@{#2}%
	\ifx\insertpathifneeded@@\empty
		\def\mykey{#1}%
	\else
		\insertpathifneeded@#2\@nil
		\ifpgfutil@in@
			\def\mykey{#2/#1}%
		\else
			\def\mykey{#1}%
		\fi
	\fi
}%
\def\insertpathifneeded@#1#2\@nil{%
	\def\insertpathifneeded@@{#1}%
	\def\insertpathifneeded@@@{/}%
	\ifx\insertpathifneeded@@\insertpathifneeded@@@
		\pgfutil@in@true
	\else
		\pgfutil@in@false
	\fi
}%

% \begin{keylist}[default path]
% 	{/path/option 1=value,/path/option 2=value2}
% \end{keylist}
\newenvironment{keylist}[2][]{%
	\begin{pgfmanualentry}
    \def\extrakeytext{}%
	\foreach \xx in {#2} {%
		\expandafter\insertpathifneeded\expandafter{\xx}{#1}%
		\expandafter\extractkey\mykey\@nil%
	}%
	\pgfmanualbody
}{%
  \end{pgfmanualentry}
}%

\newenvironment{pgfplotskeylist}[1]{%
	\begin{keylist}[/pgfplots]{#1}%
}{%
	\end{keylist}%
}

% \begin{xykeylist}[default path]
% 	{/path/option \x1=value,/path/option \x2=value2,/path/option \x3=value}
% \end{xykeylist}
\newenvironment{xykeylist}[2][]{%
	\begin{pgfmanualentry}
    \def\extrakeytext{}
	\foreach \xx in {#2} {%
		\expandafter\insertpathifneeded\expandafter{\xx}{#1}%
		\expandafter\pgfutil@in@\expandafter=\expandafter{\mykey}%
		\ifpgfutil@in@%
			\expandafter\xykey@eq\mykey\@nil
		\else
			\expandafter\xykey@noeq\mykey\@nil
		\fi
	}%
	\pgfmanualbody
}{%
  \end{pgfmanualentry}
}%

\newif\ifxykeyfound

\def\xykey@eq#1=#2\@nil{%
	\def\x{x}%
	\xdef\mykey{#1}%
	\def\xykey@@{#1}%
	\ifx\xykey@@\mykey
		\xykeyfoundfalse
	\else
		\xykeyfoundtrue
	\fi
    \expandafter\extractkey\mykey=#2\@nil%
	\ifxykeyfound
		\def\x{y}%
		\xdef\mykey{#1}%
		\expandafter\extractkey\mykey=#2\@nil%
	\fi
}
\def\xykey@noeq#1\@nil{%
	\def\x{x}%
	\xdef\mykey{#1}%
	\def\xykey@@{#1}%
	\ifx\xykey@@\mykey
		\xykeyfoundfalse
	\else
		\xykeyfoundtrue
	\fi
    \expandafter\extractkey\mykey\@nil%
	\ifxykeyfound
		\def\x{y}%
		\xdef\mykey{#1}%
		\expandafter\extractkey\mykey\@nil%
	\fi
}

% \begin{pgfplotsxykey}{\x label=value}
% \end{pgfplotsxykey}
%
% It introduces the path /pgfplots/ automatically.
%
% has same features with 'default', 'initially' etc as key environment
\newenvironment{pgfplotsxykey}[1]{%
	\begin{xykey}[/pgfplots]{#1}%
}{%
	\end{xykey}%
}


\newenvironment{pgfplotsxykeylist}[1]{%
	\begin{xykeylist}[/pgfplots]{#1}%
}{%
	\end{xykeylist}%
}


\newenvironment{plottype}[1]{%
	\begin{key}{/tikz/#1}%
	\end{key}
  \begin{pgfmanualentry}
    \pgfmanualentryheadline{\textcolor{gray}{{\ttfamily\char`\\addplot+[\declare{#1}]}}}%
    \pgfmanualbody
}
{
  \end{pgfmanualentry}
}

\def\index@prologue{\section*{Index}\addcontentsline{toc}{section}{Index}
}

\newenvironment{pgfplotstablecolumnkey}{%
  \begin{pgfmanualentry}
 	\pgfmanualentryheadline{{\ttfamily\textcolor{gray}{/pgfplots/table/}\declare{columns/\meta{column name}}\textcolor{gray}{/.style}=\marg{key-value-list}}\hfill}%
	\pgfplotsmanualkeyindex{/pgfplots/table/columns}%
    \pgfmanualbody
}
{
  \end{pgfmanualentry}
}
\newenvironment{pgfplotstabledisplaycolumnkey}{%
  \begin{pgfmanualentry}
 	\pgfmanualentryheadline{{\ttfamily\textcolor{gray}{/pgfplots/table/}\declare{display columns/\meta{index}}\textcolor{gray}{/.style}=\marg{key-value-list}}\hfill}%
	\pgfplotsmanualkeyindex{/pgfplots/table/display columns}%
    \pgfmanualbody
}
{
  \end{pgfmanualentry}
}
\newenvironment{pgfplotstablealiaskey}{%
  \begin{pgfmanualentry}
 	\pgfmanualentryheadline{{\ttfamily\textcolor{gray}{/pgfplots/table/}\declare{alias/\meta{col name}}\textcolor{gray}{/.initial}=\marg{real col name}}\hfill}%
	\pgfplotsmanualkeyindex{/pgfplots/table/alias}%
    \pgfmanualbody
}
{
  \end{pgfmanualentry}
}


\def\pgfplotsmanualkeyindex#1{%
	\def\mypath{#1}%
	\def\myname{}%
	\firsttimetrue%
	\decompose#1/\nil%
}
\newenvironment{pgfplotstablecreateonusekey}{%
  \begin{pgfmanualentry}
 	\pgfmanualentryheadline{{\ttfamily\textcolor{gray}{/pgfplots/table/}\declare{create on use/\meta{col name}}\textcolor{gray}{/.style}=\marg{create options}}\hfill}%
	\def\mykey{/pgfplots/table/create on use}%
    \pgfmanualbody
	\pgfplotsmanualkeyindex{/pgfplots/table/create on use}%
}
{
  \end{pgfmanualentry}
}

\def\pgfplotsassertcmdkeyexists#1{%
	\pgfkeysifdefined{/pgfplots/#1/.@cmd}\relax{%
		\pgfplots@error{DOCUMENTATION ERROR: command key /pgfplots/#1 does not exist!}%
	}%
}%

{
\catcode`\ =12%
\gdef\makespaceexpandable{\def\ { }}}%

\def\pgfplotsassertXYcmdkeyexists#1{%
	{\makespaceexpandable\def\x{x}\edef\pgfplotsassertXYcmdkeyexists@tmp{#1}%
	\pgfkeysifdefined{/pgfplots/\pgfplotsassertXYcmdkeyexists@tmp/.@cmd}\relax{%
		\pgfplots@error{DOCUMENTATION ERROR: command key /pgfplots/#1 does not exist!}%
	}}%
	{\makespaceexpandable\def\x{y}\edef\pgfplotsassertXYcmdkeyexists@tmp{#1}%
	\pgfkeysifdefined{/pgfplots/\pgfplotsassertXYcmdkeyexists@tmp/.@cmd}\relax{%
		\pgfplots@error{DOCUMENTATION ERROR: command key /pgfplots/#1 does not exist!}%
	}}%
}%

\def\pgfplotsshortstylekey #1=#2\pgfeov{%
	\pgfplotsassertcmdkeyexists{#1}%
	\pgfplotsassertcmdkeyexists{#2}%
	\begin{pgfplotskey}{#1=\marg{key-value-list}}
		An abbreviation for \texttt{#2/.append style=}\marg{key-value-list}.
	\end{pgfplotskey}
}
\def\pgfplotsshortxystylekey #1=#2\pgfeov{%
	\pgfplotsassertXYcmdkeyexists{#1}%
	\pgfplotsassertXYcmdkeyexists{#2}%
	\begin{pgfplotsxykey}{#1=\marg{key-value-list}}
		An abbreviation for {\def\x{x}\texttt{#2/.append style=}}\marg{key-value-list} (or the respective style for $y$, {\def\x{y}\texttt{#2}}).
	\end{pgfplotsxykey}
}
\def\pgfplotsshortstylekeys #1,#2=#3\pgfeov{%
	\pgfplotsassertcmdkeyexists{#1}%
	\pgfplotsassertcmdkeyexists{#2}%
	\pgfplotsassertcmdkeyexists{#3}%
	\begin{pgfplotskeylist}{%
		#1=\marg{key-value-list},
		#2=\marg{key-value-list}}
		Different abbreviations for \texttt{#3/.append style=}\marg{key-value-list}.
	\end{pgfplotskeylist}
}
\def\pgfplotsshortxystylekeys #1,#2=#3\pgfeov{%
	\pgfplotsassertXYcmdkeyexists{#1}%
	\pgfplotsassertXYcmdkeyexists{#2}%
	\pgfplotsassertXYcmdkeyexists{#3}%
	\begin{pgfplotsxykeylist}{%
		#1=\marg{key-value-list},
		#2=\marg{key-value-list}}
		Different abbreviations for {\def\x{x}\texttt{#3/.append style=}}\marg{key-value-list} (or the respective style for $y$, {\def\x{y}\texttt{#3}}).
	\end{pgfplotsxykeylist}
}

%
% Creates and shows a colormap with specification '#1'.
\def\pgfplotsshowcolormapexample#1{%
	\pgfplotscreatecolormap{tempcolormap}{#1}%
	\pgfplotsshowcolormap{tempcolormap}%
}

% Shows the colormap named '#1'.
\def\pgfplotsshowcolormap#1{%
	\pgfplotscolormaptoshadingspec{#1}{8cm}\result
	\def\tempb{\pgfdeclarehorizontalshading{tempshading}{1cm}}%
	\expandafter\tempb\expandafter{\result}%
	\pgfuseshading{tempshading}%
}

\makeatother


\pgfqkeys{/codeexample}{%
	every codeexample/.style={
		width=8cm,
		/pgfplots/every axis/.append style={legend style={fill=graphicbackground}}
	},
	tabsize=4,
}
\pgfplotsset{
	every axis/.append style={width=7cm},
	filter discard warning=false,
}

\usetikzlibrary{backgrounds,patterns}
% Global styles:
\tikzset{
  shape example/.style={
    color=black!30,
    draw,
    fill=yellow!30,
    line width=.5cm,
    inner xsep=2.5cm,
    inner ysep=0.5cm}
}

\newcommand{\FIXME}[1]{\textcolor{red}{(FIXME: #1)}}

% fuer endvironment 'sidewaysfigure' bspw
% \usepackage{rotating}

\newcommand\Tikz{Ti\textit kZ}
\newcommand\PGF{\textsc{pgf}}
\newcommand\PGFPlots{\textsc{pgfplots}}
\def\PGFPlotstable{\textsc{PgfplotsTable}}


\makeindex

% Fix overful hboxes automatically:
\tolerance=2000
\emergencystretch=10pt

\tikzset{prefix=gnuplot/pgfplots_} % prefix for 'plot function'

\author{%
	Christian Feuers\"anger\footnote{\url{http://wissrech.ins.uni-bonn.de/people/feuersaenger}}\\%
	Institut f\"ur Numerische Simulation\\
	Universit\"at Bonn, Germany}

